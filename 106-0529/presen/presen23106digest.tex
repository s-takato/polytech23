%%% Title presen23106
\documentclass[landscape,10pt]{ujarticle}
\special{papersize=\the\paperwidth,\the\paperheight}
\usepackage{ketpic,ketlayer}
\usepackage{ketslide}
\usepackage{amsmath,amssymb}
\usepackage{bm,enumerate}
\usepackage[dvipdfmx]{graphicx}
\usepackage{color}
\definecolor{slidecolora}{cmyk}{0.98,0.13,0,0.43}
\definecolor{slidecolorb}{cmyk}{0.2,0,0,0}
\definecolor{slidecolorc}{cmyk}{0.2,0,0,0}
\definecolor{slidecolord}{cmyk}{0.2,0,0,0}
\definecolor{slidecolore}{cmyk}{0,0,0,0.5}
\definecolor{slidecolorf}{cmyk}{0,0,0,0.5}
\definecolor{slidecolori}{cmyk}{0.98,0.13,0,0.43}
\def\setthin#1{\def\thin{#1}}
\setthin{0}
\newcommand{\slidepage}[1][s]{%
\setcounter{ketpicctra}{18}%
\if#1m \setcounter{ketpicctra}{1}\fi
\hypersetup{linkcolor=black}%

\begin{layer}{118}{0}
\putnotee{122}{-\theketpicctra.05}{\small\thepage/\pageref{pageend}}
\end{layer}\hypersetup{linkcolor=blue}

}
\usepackage{emath}
\usepackage{emathEy}
\usepackage{emathMw}
\usepackage{pict2e}
\usepackage{ketlayermorewith2e}
\usepackage[dvipdfmx,colorlinks=true,linkcolor=blue,filecolor=blue]{hyperref}
\newcommand{\hiduke}{0529}
\newcommand{\hako}[2][1]{\fbox{\raisebox{#1mm}{\mbox{}}\raisebox{-#1mm}{\mbox{}}\,\phantom{#2}\,}}
\newcommand{\hakoa}[2][1]{\fbox{\raisebox{#1mm}{\mbox{}}\raisebox{-#1mm}{\mbox{}}\,#2\,}}
\newcommand{\hakom}[2][1]{\hako[#1]{$#2$}}
\newcommand{\hakoma}[2][1]{\hakoa[#1]{$#2$}}
\def\rad{\;\mathrm{rad}}
\def\deg#1{#1^{\circ}}
\newcommand{\sbunsuu}[2]{\scalebox{0.6}{$\bunsuu{#1}{#2}$}}
\def\pow{$\hspace{-1.5mm}^\hspace{-1mm}$}
\def\dlim{\displaystyle\lim}
\newcommand{\brd}[2][1]{\scalebox{#1}{\color{red}\fbox{\color{black}$#2$}}}
\newcommand\down[1][0.5zw]{\vspace{#1}\\}
\newcommand{\sfrac}[3][0.65]{\scalebox{#1}{$\frac{#2}{#3}$}}
\newcommand{\phn}[1]{\phantom{#1}}
\newcommand{\scb}[2][0.6]{\scalebox{#1}{#2}}
\newcommand{\dsum}{\displaystyle\sum}

\setmargin{25}{145}{15}{100}

\ketslideinit

\pagestyle{empty}

\begin{document}

\begin{layer}{120}{0}
\putnotese{0}{0}{{\Large\bf
\color[cmyk]{1,1,0,0}

\begin{layer}{120}{0}
{\Huge \putnotes{60}{20}{三角比と三角関数}}
\putnotes{60}{70}{2022.04.25}
\end{layer}

}
}
\end{layer}

\def\mainslidetitley{22}
\def\ketcletter{slidecolora}
\def\ketcbox{slidecolorb}
\def\ketdbox{slidecolorc}
\def\ketcframe{slidecolord}
\def\ketcshadow{slidecolore}
\def\ketdshadow{slidecolorf}
\def\slidetitlex{6}
\def\slidetitlesize{1.3}
\def\mketcletter{slidecolori}
\def\mketcbox{yellow}
\def\mketdbox{yellow}
\def\mketcframe{yellow}
\def\mslidetitlex{62}
\def\mslidetitlesize{2}

\color{black}
\Large\bf\boldmath
\addtocounter{page}{-1}

\def\MARU{}
\renewcommand{\MARU}[1]{{\ooalign{\hfil$#1$\/\hfil\crcr\raise.167ex\hbox{\mathhexbox20D}}}}
\renewcommand{\slidepage}[1][s]{%
\setcounter{ketpicctra}{18}%
\if#1m \setcounter{ketpicctra}{1}\fi
\hypersetup{linkcolor=black}%
\begin{layer}{118}{0}
\putnotee{115}{-\theketpicctra.05}{\small\hiduke-\thepage/\pageref{pageend}}
\end{layer}\hypersetup{linkcolor=blue}
}
\newcounter{ban}
\setcounter{ban}{1}
\newcommand{\monban}[1][\hiduke]{%
#1-\theban\ %
\addtocounter{ban}{1}%
}
\newcommand{\monbannoadd}[1][\hiduke]{%
#1-\theban\ %
}
\newcommand{\addban}{%
\addtocounter{ban}{1}%%210614
}
\newcounter{edawidth}
\newcounter{edactr}
\newcommand{\seteda}[1]{
\setcounter{edawidth}{#1}
\setcounter{edactr}{1}
}
\newcommand{\eda}[2][\theedawidth ]{%
\noindent\Ltab{#1 mm}{[\theedactr]\ #2}%
\addtocounter{edactr}{1}%
}
%%%%%%%%%%%%

%%%%%%%%%%%%%%%%%%%%

\mainslide{指数対数(復習+)}


\slidepage[m]
%%%%%%%%%%%%

%%%%%%%%%%%%%%%%%%%%

\newslide{指数関数$y=a^x$}

\vspace*{18mm}

\slidepage

\begin{layer}{120}{0}
\putnotese{85}{30}{\scalebox{0.7}{%%% /polytech22.git/106-0530/presen/fig/fig220530_1.tex 
%%% Generator=fig220530.cdy 
{\unitlength=1cm%
\begin{picture}%
(6,6.5)(-3,-0.5)%
\linethickness{0.008in}%%
\polyline(-3.00000,1.00000)(-2.88000,1.00000)(-2.76000,1.00000)(-2.64000,1.00000)%
(-2.52000,1.00000)(-2.40000,1.00000)(-2.28000,1.00000)(-2.16000,1.00000)(-2.04000,1.00000)%
(-1.92000,1.00000)(-1.80000,1.00000)(-1.68000,1.00000)(-1.56000,1.00000)(-1.44000,1.00000)%
(-1.32000,1.00000)(-1.20000,1.00000)(-1.08000,1.00000)(-0.96000,1.00000)(-0.84000,1.00000)%
(-0.72000,1.00000)(-0.60000,1.00000)(-0.48000,1.00000)(-0.36000,1.00000)(-0.24000,1.00000)%
(-0.12000,1.00000)(0.00000,1.00000)(0.12000,1.00000)(0.24000,1.00000)(0.36000,1.00000)%
(0.48000,1.00000)(0.60000,1.00000)(0.72000,1.00000)(0.84000,1.00000)(0.96000,1.00000)%
(1.08000,1.00000)(1.20000,1.00000)(1.32000,1.00000)(1.44000,1.00000)(1.56000,1.00000)%
(1.68000,1.00000)(1.80000,1.00000)(1.92000,1.00000)(2.04000,1.00000)(2.16000,1.00000)%
(2.28000,1.00000)(2.40000,1.00000)(2.52000,1.00000)(2.64000,1.00000)(2.76000,1.00000)%
(2.88000,1.00000)(3.00000,1.00000)%
%
\polyline(-3.00000,0.12500)(-2.88000,0.13584)(-2.76000,0.14762)(-2.64000,0.16043)%
(-2.52000,0.17434)(-2.40000,0.18946)(-2.28000,0.20590)(-2.16000,0.22376)(-2.04000,0.24316)%
(-1.92000,0.26425)(-1.80000,0.28717)(-1.68000,0.31208)(-1.56000,0.33915)(-1.44000,0.36857)%
(-1.32000,0.40053)(-1.20000,0.43528)(-1.08000,0.47303)(-0.96000,0.51406)(-0.84000,0.55864)%
(-0.72000,0.60710)(-0.60000,0.65975)(-0.48000,0.71698)(-0.36000,0.77916)(-0.24000,0.84675)%
(-0.12000,0.92019)(0.00000,1.00000)(0.12000,1.08673)(0.24000,1.18099)(0.36000,1.28343)%
(0.48000,1.39474)(0.60000,1.51572)(0.72000,1.64718)(0.84000,1.79005)(0.96000,1.94531)%
(1.08000,2.11404)(1.20000,2.29740)(1.32000,2.49666)(1.44000,2.71321)(1.56000,2.94854)%
(1.68000,3.20428)(1.80000,3.48220)(1.92000,3.78423)(2.04000,4.11246)(2.16000,4.46915)%
(2.28000,4.85678)(2.40000,5.27803)(2.52000,5.73582)(2.58372,6.00000)%
%
\polyline(-3.00000,0.03704)(-2.88000,0.04226)(-2.76000,0.04821)(-2.64000,0.05500)%
(-2.52000,0.06276)(-2.40000,0.07160)(-2.28000,0.08169)(-2.16000,0.09320)(-2.04000,0.10633)%
(-1.92000,0.12132)(-1.80000,0.13841)(-1.68000,0.15792)(-1.56000,0.18017)(-1.44000,0.20556)%
(-1.32000,0.23453)(-1.20000,0.26758)(-1.08000,0.30529)(-0.96000,0.34831)(-0.84000,0.39739)%
(-0.72000,0.45339)(-0.60000,0.51728)(-0.48000,0.59018)(-0.36000,0.67334)(-0.24000,0.76823)%
(-0.12000,0.87649)(0.00000,1.00000)(0.12000,1.14092)(0.24000,1.30169)(0.36000,1.48513)%
(0.48000,1.69441)(0.60000,1.93318)(0.72000,2.20560)(0.84000,2.51641)(0.96000,2.87102)%
(1.08000,3.27560)(1.20000,3.73719)(1.32000,4.26383)(1.44000,4.86468)(1.56000,5.55021)%
(1.62901,6.00000)%
%
{%
\color[cmyk]{0,1,1,0}%
\polyline(-2.58372,6.00000)(-2.52000,5.73582)(-2.40000,5.27803)(-2.28000,4.85678)%
(-2.16000,4.46915)(-2.04000,4.11246)(-1.92000,3.78423)(-1.80000,3.48220)(-1.68000,3.20428)%
(-1.56000,2.94854)(-1.44000,2.71321)(-1.32000,2.49666)(-1.20000,2.29740)(-1.08000,2.11404)%
(-0.96000,1.94531)(-0.84000,1.79005)(-0.72000,1.64718)(-0.60000,1.51572)(-0.48000,1.39474)%
(-0.36000,1.28343)(-0.24000,1.18099)(-0.12000,1.08673)(0.00000,1.00000)(0.12000,0.92019)%
(0.24000,0.84675)(0.36000,0.77916)(0.48000,0.71698)(0.60000,0.65975)(0.72000,0.60710)%
(0.84000,0.55864)(0.96000,0.51406)(1.08000,0.47303)(1.20000,0.43528)(1.32000,0.40053)%
(1.44000,0.36857)(1.56000,0.33915)(1.68000,0.31208)(1.80000,0.28717)(1.92000,0.26425)%
(2.04000,0.24316)(2.16000,0.22376)(2.28000,0.20590)(2.40000,0.18946)(2.52000,0.17434)%
(2.64000,0.16043)(2.76000,0.14762)(2.88000,0.13584)(3.00000,0.12500)%
%
}%
{%
\color[cmyk]{0,1,1,0}%
\polyline(-1.29069,6.00000)(-1.20000,5.27803)(-1.08000,4.46915)(-0.96000,3.78423)%
(-0.84000,3.20428)(-0.72000,2.71321)(-0.60000,2.29740)(-0.48000,1.94531)(-0.36000,1.64718)%
(-0.24000,1.39474)(-0.12000,1.18099)(0.00000,1.00000)(0.12000,0.84675)(0.24000,0.71698)%
(0.36000,0.60710)(0.48000,0.51406)(0.60000,0.43528)(0.72000,0.36857)(0.84000,0.31208)%
(0.96000,0.26425)(1.08000,0.22376)(1.20000,0.18946)(1.32000,0.16043)(1.44000,0.13584)%
(1.56000,0.11502)(1.68000,0.09740)(1.80000,0.08247)(1.92000,0.06983)(2.04000,0.05913)%
(2.16000,0.05007)(2.28000,0.04239)(2.40000,0.03590)(2.52000,0.03040)(2.64000,0.02574)%
(2.76000,0.02179)(2.88000,0.01845)(3.00000,0.01562)%
%
}%
\settowidth{\Width}{$a=1$}\setlength{\Width}{0\Width}%
\settoheight{\Height}{$a=1$}\settodepth{\Depth}{$a=1$}\setlength{\Height}{-0.5\Height}\setlength{\Depth}{0.5\Depth}\addtolength{\Height}{\Depth}%
\put(2.1000000,1.2400000){\hspace*{\Width}\raisebox{\Height}{$a=1$}}%
%
\settowidth{\Width}{$a=2$}\setlength{\Width}{0\Width}%
\settoheight{\Height}{$a=2$}\settodepth{\Depth}{$a=2$}\setlength{\Height}{-0.5\Height}\setlength{\Depth}{0.5\Depth}\addtolength{\Height}{\Depth}%
\put(1.7700000,3.1500000){\hspace*{\Width}\raisebox{\Height}{$a=2$}}%
%
\settowidth{\Width}{$a=4$}\setlength{\Width}{0\Width}%
\settoheight{\Height}{$a=4$}\settodepth{\Depth}{$a=4$}\setlength{\Height}{-0.5\Height}\setlength{\Depth}{0.5\Depth}\addtolength{\Height}{\Depth}%
\put(1.6500000,5.7600000){\hspace*{\Width}\raisebox{\Height}{$a=4$}}%
%
\settowidth{\Width}{$a=\frac{1}{2}$}\setlength{\Width}{0\Width}%
\settoheight{\Height}{$a=\frac{1}{2}$}\settodepth{\Depth}{$a=\frac{1}{2}$}\setlength{\Height}{-0.5\Height}\setlength{\Depth}{0.5\Depth}\addtolength{\Height}{\Depth}%
\put(-1.0300000,5.0500000){\hspace*{\Width}\raisebox{\Height}{$a=\frac{1}{2}$}}%
%
\settowidth{\Width}{$a=\frac{1}{4}$}\setlength{\Width}{0\Width}%
\settoheight{\Height}{$a=\frac{1}{4}$}\settodepth{\Depth}{$a=\frac{1}{4}$}\setlength{\Height}{-0.5\Height}\setlength{\Depth}{0.5\Depth}\addtolength{\Height}{\Depth}%
\put(-2.0600000,4.3000000){\hspace*{\Width}\raisebox{\Height}{$a=\frac{1}{4}$}}%
%
\polyline(0.05000,1.00000)(-0.05000,1.00000)%
%
\settowidth{\Width}{$1$}\setlength{\Width}{-1\Width}%
\settoheight{\Height}{$1$}\settodepth{\Depth}{$1$}\setlength{\Height}{\Depth}%
\put(-0.0500000,1.0500000){\hspace*{\Width}\raisebox{\Height}{$1$}}%
%
\polyline(-3.00000,0.00000)(3.00000,0.00000)%
%
\polyline(0.00000,-0.50000)(0.00000,6.00000)%
%
\settowidth{\Width}{$x$}\setlength{\Width}{0\Width}%
\settoheight{\Height}{$x$}\settodepth{\Depth}{$x$}\setlength{\Height}{-0.5\Height}\setlength{\Depth}{0.5\Depth}\addtolength{\Height}{\Depth}%
\put(3.0500000,0.0000000){\hspace*{\Width}\raisebox{\Height}{$x$}}%
%
\settowidth{\Width}{$y$}\setlength{\Width}{-0.5\Width}%
\settoheight{\Height}{$y$}\settodepth{\Depth}{$y$}\setlength{\Height}{\Depth}%
\put(0.0000000,6.0500000){\hspace*{\Width}\raisebox{\Height}{$y$}}%
%
\settowidth{\Width}{O}\setlength{\Width}{-1\Width}%
\settoheight{\Height}{O}\settodepth{\Depth}{O}\setlength{\Height}{-\Height}%
\put(-0.0500000,-0.0500000){\hspace*{\Width}\raisebox{\Height}{O}}%
%
\end{picture}}%}}
\end{layer}

\begin{itemize}
\item
$a$は正の定数\vspace{-1mm}
\item
任意の実数$x$について$a^x$が定まる.\\
\hspace*{1zw}$a^0=1,\ a^{-n}=\bunsuu{1}{a^n},\ a^{\frac{n}{m}}=\sqrt[m]{a^n}$\vspace{-1mm}
\item
指数法則\\
(1)\ $a^p a^{q}=a^{p+q}$\\
(2)\ $(a^p)^{q}=a^{pq}$\\
(3)\ $(ab)^p=a^p b^p$\vspace{-1mm}
\item
$y=a^x$のグラフ
\end{itemize}

\mainslide{対数関数}


\slidepage[m]
%%%%%%%%%%%%

%%%%%%%%%%%%%%%%%%%%

\newslide{対数の定義}

\vspace*{18mm}


\begin{layer}{120}{0}
\putnotew{96}{73}{\hyperlink{para0pg6}{\fbox{\Ctab{2.5mm}{\scalebox{1}{\scriptsize $\mathstrut||\!\lhd$}}}}}
\putnotew{101}{73}{\hyperlink{para1pg1}{\fbox{\Ctab{2.5mm}{\scalebox{1}{\scriptsize $\mathstrut|\!\lhd$}}}}}
\putnotew{108}{73}{\hyperlink{para1pg9}{\fbox{\Ctab{4.5mm}{\scalebox{1}{\scriptsize $\mathstrut\!\!\lhd\!\!$}}}}}
\putnotew{115}{73}{\hyperlink{para1pg10}{\fbox{\Ctab{4.5mm}{\scalebox{1}{\scriptsize $\mathstrut\!\rhd\!$}}}}}
\putnotew{120}{73}{\hyperlink{para1pg10}{\fbox{\Ctab{2.5mm}{\scalebox{1}{\scriptsize $\mathstrut \!\rhd\!\!|$}}}}}
\putnotew{125}{73}{\hyperlink{para2pg1}{\fbox{\Ctab{2.5mm}{\scalebox{1}{\scriptsize $\mathstrut \!\rhd\!\!||$}}}}}
\putnotee{126}{73}{\scriptsize\color{blue} 10/10}
\end{layer}

\slidepage
\begin{itemize}
\item
$y=\log_a x$\\
\hspace*{4zw}$a$を{\color{red}底},$x$を{\color{red}真数}\\
\hspace*{4zw}$y$を$a$を底とする$x$の{\color{red}対数}という.
\item
対数$y$は,$a$を何乗したら$x$になるかという数\\
\hspace*{4zw}$a^{\fbox{$y$}}=x$となる\;\fbox{$\mathstrut y$}\;のこと
\item
[例)]$y=\log_3 9$\\
\hspace*{1zw}$3^{\fbox{$y$}}=9$となる$y$のこと\\
\hspace*{1zw}$3^2=9$だから
\hspace*{1zw}$y=\log_3 9=2$
\end{itemize}

\newslide{対数の値と対数法則}

\vspace*{18mm}


\begin{layer}{120}{0}
\putnotew{96}{73}{\hyperlink{para1pg10}{\fbox{\Ctab{2.5mm}{\scalebox{1}{\scriptsize $\mathstrut||\!\lhd$}}}}}
\putnotew{101}{73}{\hyperlink{para2pg1}{\fbox{\Ctab{2.5mm}{\scalebox{1}{\scriptsize $\mathstrut|\!\lhd$}}}}}
\putnotew{108}{73}{\hyperlink{para2pg5}{\fbox{\Ctab{4.5mm}{\scalebox{1}{\scriptsize $\mathstrut\!\!\lhd\!\!$}}}}}
\putnotew{115}{73}{\hyperlink{para2pg6}{\fbox{\Ctab{4.5mm}{\scalebox{1}{\scriptsize $\mathstrut\!\rhd\!$}}}}}
\putnotew{120}{73}{\hyperlink{para2pg6}{\fbox{\Ctab{2.5mm}{\scalebox{1}{\scriptsize $\mathstrut \!\rhd\!\!|$}}}}}
\putnotew{125}{73}{\hyperlink{para3pg1}{\fbox{\Ctab{2.5mm}{\scalebox{1}{\scriptsize $\mathstrut \!\rhd\!\!||$}}}}}
\putnotee{126}{73}{\scriptsize\color{blue} 6/6}
\end{layer}

\slidepage
\begin{itemize}
\item
$y=\log_a x \Longleftrightarrow a^y=x$
\item
[(例)]$y=\log_2 16$
$\Longleftrightarrow 2^y=16$
$=2^4$\\
 $y=4$となるから $\log_2 16=4$
\item
対数法則\\
(1)\ $\log_a b+\log_a c=\log_a(bc)$\\
(2)\ $\log_a b-\log_a c=\log_a\bunsuu{\ b\ }{c}$\\
(3)\ $\log_a b^{\,p}=p\log_a b$
\end{itemize}

\newslide{対数の計算}

\vspace*{18mm}


\begin{layer}{120}{0}
\putnotew{96}{73}{\hyperlink{para2pg6}{\fbox{\Ctab{2.5mm}{\scalebox{1}{\scriptsize $\mathstrut||\!\lhd$}}}}}
\putnotew{101}{73}{\hyperlink{para3pg1}{\fbox{\Ctab{2.5mm}{\scalebox{1}{\scriptsize $\mathstrut|\!\lhd$}}}}}
\putnotew{108}{73}{\hyperlink{para3pg11}{\fbox{\Ctab{4.5mm}{\scalebox{1}{\scriptsize $\mathstrut\!\!\lhd\!\!$}}}}}
\putnotew{115}{73}{\hyperlink{para3pg12}{\fbox{\Ctab{4.5mm}{\scalebox{1}{\scriptsize $\mathstrut\!\rhd\!$}}}}}
\putnotew{120}{73}{\hyperlink{para3pg12}{\fbox{\Ctab{2.5mm}{\scalebox{1}{\scriptsize $\mathstrut \!\rhd\!\!|$}}}}}
\putnotew{125}{73}{\hyperlink{para4pg1}{\fbox{\Ctab{2.5mm}{\scalebox{1}{\scriptsize $\mathstrut \!\rhd\!\!||$}}}}}
\putnotee{126}{73}{\scriptsize\color{blue} 12/12}
\end{layer}

\slidepage
\begin{itemize}
\item
[(1)]$\log_{10} 5+\log_{10} 2$
\item
[]与式$=\log_{10}(5\times 2)$
$=\log_{10} 10$
$=1$
\item
[(2)]$\log_2 12-\log_2 3$
\item
[]与式$=\log_2(\bunsuu{12}{3})$
$=\log_2 4$
$=2$
\item
[(3)]$2\log_3 4+\log_3 4-\log_3 8$
\item
[]与式$=\log_3 4^2+\log_3 4-\log_3 8$\\
\phantom{与式}$=\log_3 \frac{16\times 4}{8}$
$=\log_3 8$
\end{itemize}

\newslide{底・真数・対数の条件}

\vspace*{18mm}


\begin{layer}{120}{0}
\putnotew{96}{73}{\hyperlink{para3pg12}{\fbox{\Ctab{2.5mm}{\scalebox{1}{\scriptsize $\mathstrut||\!\lhd$}}}}}
\putnotew{101}{73}{\hyperlink{para4pg1}{\fbox{\Ctab{2.5mm}{\scalebox{1}{\scriptsize $\mathstrut|\!\lhd$}}}}}
\putnotew{108}{73}{\hyperlink{para4pg9}{\fbox{\Ctab{4.5mm}{\scalebox{1}{\scriptsize $\mathstrut\!\!\lhd\!\!$}}}}}
\putnotew{115}{73}{\hyperlink{para4pg10}{\fbox{\Ctab{4.5mm}{\scalebox{1}{\scriptsize $\mathstrut\!\rhd\!$}}}}}
\putnotew{120}{73}{\hyperlink{para4pg10}{\fbox{\Ctab{2.5mm}{\scalebox{1}{\scriptsize $\mathstrut \!\rhd\!\!|$}}}}}
\putnotew{125}{73}{\hyperlink{para5pg1}{\fbox{\Ctab{2.5mm}{\scalebox{1}{\scriptsize $\mathstrut \!\rhd\!\!||$}}}}}
\putnotee{126}{73}{\scriptsize\color{blue} 10/10}
\end{layer}

\slidepage
\begin{itemize}
\item
$y=\log_a x \Longleftrightarrow a^y=x$
\item
$\log_a 1=0,\ \log_a a=1$
\item
底$a$の条件は $a>0,a\neqq 1$
\item
真数$x$の条件は $x>0$
\item
対数$y=\log_a x$の値の範囲は 実数全部
\end{itemize}

\newslide{対数関数のグラフ}

\vspace*{18mm}


\begin{layer}{120}{0}
\putnotew{96}{73}{\hyperlink{para4pg10}{\fbox{\Ctab{2.5mm}{\scalebox{1}{\scriptsize $\mathstrut||\!\lhd$}}}}}
\putnotew{101}{73}{\hyperlink{para5pg1}{\fbox{\Ctab{2.5mm}{\scalebox{1}{\scriptsize $\mathstrut|\!\lhd$}}}}}
\putnotew{108}{73}{\hyperlink{para5pg4}{\fbox{\Ctab{4.5mm}{\scalebox{1}{\scriptsize $\mathstrut\!\!\lhd\!\!$}}}}}
\putnotew{115}{73}{\hyperlink{para5pg5}{\fbox{\Ctab{4.5mm}{\scalebox{1}{\scriptsize $\mathstrut\!\rhd\!$}}}}}
\putnotew{120}{73}{\hyperlink{para5pg5}{\fbox{\Ctab{2.5mm}{\scalebox{1}{\scriptsize $\mathstrut \!\rhd\!\!|$}}}}}
\putnotew{125}{73}{\hyperlink{para6pg1}{\fbox{\Ctab{2.5mm}{\scalebox{1}{\scriptsize $\mathstrut \!\rhd\!\!||$}}}}}
\putnotee{126}{73}{\scriptsize\color{blue} 5/5}
\end{layer}

\slidepage
\begin{itemize}
\item
$y=\log_a x\Longleftrightarrow a^y=x\ (x=a^y)$
\item
$x$と$y$を入れ替えれば$y=a^x$のグラフになる.
\item
これを{\color{red}逆関数}という.
\item
アプリ「%
\href{https://s-takato.github.io/polytech22/}{指数と対数}」を動かしてみよう
\item
[課題]\monban $y=\log_a x$と$y=a^x$について( )を埋めよ.\seteda{100}\\
\eda{グラフは直線$y=x$に関して(  )}\\
\eda{$y=\log_a x$は$y=a^x$の(  )}
\end{itemize}

\newslide{底の変換公式}

\vspace*{18mm}


\begin{layer}{120}{0}
\putnotew{96}{73}{\hyperlink{para5pg5}{\fbox{\Ctab{2.5mm}{\scalebox{1}{\scriptsize $\mathstrut||\!\lhd$}}}}}
\putnotew{101}{73}{\hyperlink{para6pg1}{\fbox{\Ctab{2.5mm}{\scalebox{1}{\scriptsize $\mathstrut|\!\lhd$}}}}}
\putnotew{108}{73}{\hyperlink{para6pg5}{\fbox{\Ctab{4.5mm}{\scalebox{1}{\scriptsize $\mathstrut\!\!\lhd\!\!$}}}}}
\putnotew{115}{73}{\hyperlink{para6pg6}{\fbox{\Ctab{4.5mm}{\scalebox{1}{\scriptsize $\mathstrut\!\rhd\!$}}}}}
\putnotew{120}{73}{\hyperlink{para6pg6}{\fbox{\Ctab{2.5mm}{\scalebox{1}{\scriptsize $\mathstrut \!\rhd\!\!|$}}}}}
\putnotew{125}{73}{\hyperlink{para7pg1}{\fbox{\Ctab{2.5mm}{\scalebox{1}{\scriptsize $\mathstrut \!\rhd\!\!||$}}}}}
\putnotee{126}{73}{\scriptsize\color{blue} 6/6}
\end{layer}

\slidepage
{\color{blue}

\begin{layer}{120}{0}
\putnotee{65}{18}{\normalsize$\log_2 3=\bunsuu{\log_{10} 3}{\log_{10}2}$}
\end{layer}

}
\noindent
底を$a$から別の$c$に変える公式\vspace{2mm}\\
\hspace*{20mm}\fbox{$\log_a b=\bunsuu{\log_c b}{\log_c a}$}
\begin{itemize}
\item
[例)]$\log_3 8$を底が$2$の対数に変える\vspace{-2mm}
\item
[]$\log_3 8=\bunsuu{\log_2\hakoma{8}}{\log_2\hakoma{3}}$
$=\bunsuu{\log_2 2^3}{\log_2 3}=\bunsuu{3\log_2 2}{\log_2 3}=\bunsuu{3}{\log_2 3}$\vspace{-2mm}
\item
[課題]\monban 底を変換して計算せよ\hfill Text P193問1\seteda{35}\\
\eda{$\log_4 32$}\eda{$\log_9 3$}\eda{$\log_3 2 \log_2 27$}\\
\eda{$\log_a b\times \log_b a$}
\end{itemize}

\newslide{常用対数}

\vspace*{18mm}


\begin{layer}{120}{0}
\putnotew{96}{73}{\hyperlink{para6pg6}{\fbox{\Ctab{2.5mm}{\scalebox{1}{\scriptsize $\mathstrut||\!\lhd$}}}}}
\putnotew{101}{73}{\hyperlink{para7pg1}{\fbox{\Ctab{2.5mm}{\scalebox{1}{\scriptsize $\mathstrut|\!\lhd$}}}}}
\putnotew{108}{73}{\hyperlink{para7pg4}{\fbox{\Ctab{4.5mm}{\scalebox{1}{\scriptsize $\mathstrut\!\!\lhd\!\!$}}}}}
\putnotew{115}{73}{\hyperlink{para7pg5}{\fbox{\Ctab{4.5mm}{\scalebox{1}{\scriptsize $\mathstrut\!\rhd\!$}}}}}
\putnotew{120}{73}{\hyperlink{para7pg5}{\fbox{\Ctab{2.5mm}{\scalebox{1}{\scriptsize $\mathstrut \!\rhd\!\!|$}}}}}
\putnotew{125}{73}{\hyperlink{para8pg1}{\fbox{\Ctab{2.5mm}{\scalebox{1}{\scriptsize $\mathstrut \!\rhd\!\!||$}}}}}
\putnotee{126}{73}{\scriptsize\color{blue} 5/5}
\end{layer}

\slidepage
\begin{itemize}
\item
底が$10$の対数 $\log_{10}x$
\item
数値計算ではよく用いられる(対数表,関数電卓)
\item
$\log_{10}2=0.3010,\ \log_{10}3=0.4771$(近似値)
\item
[例] $\log_{10}6=\log_{10}2+\log_{10}3=0.3010+0.4771$
\item
[] $\phantom{\log_{10}6}=0.7781$
\item
[課題]\monban 次の値を求めよ\seteda{32}\\
\eda{$\log_{10}4$}\eda{$\log_{10}8$}\eda{$\log_{10}\bunsuu{1}{2}$}\eda{$\log_{10}5$}
\end{itemize}

\newslide{常用対数と桁数}

\vspace*{18mm}

\slidepage
\begin{itemize}
\item
$100000$の桁数は \hakoa{6桁}
\item
$100000=10\cdot 10\cdot 10\cdot 10\cdot 10\cdot 10=10^5$より\\
\hspace*{1zw}$\log_{10}100000=\log_{10}10^5$
$=5\log_{10}10=5$
\item
常用対数と桁数 \fbox{\color{red}桁数$=$常用対数の整数部分$+1$}
\item
[例)]$2^{100}$の桁数\\
$\log_{10}2^{100}=100\log_{10}2$
$=100\times 0.3010=30.10$
\\ よって 31桁
\item
[課題]\monban $3^{30}$の桁数を求めよ\hfill Text P196問2
\end{itemize}

\newslide{自然対数}

\vspace*{18mm}

\slidepage
\begin{itemize}
\item
もう1つ,数学では大切な{\color{red}自然対数}がある.
\item
ネイピアの定数$e$を底とする対数\\
\hspace*{2zw}$e=2.718281828$
\item
自然対数と常用対数の変換\\
\hspace*{2zw}$\log_e 10=\dfrac{\log_{10}{10}}{\log_{10}e}=\dfrac{1}{\log_{10}e}$
\item
詳しくは,微分のときに説明する
\end{itemize}

\mainslide{数列}


\slidepage[m]
%%%%%%%%%%%%
%%%%%%%%%%%%

%%%%%%%%%%%%%%%%%%%%

\newslide{数列とは}

\vspace*{18mm}


\begin{layer}{120}{0}
\putnotew{96}{73}{\hyperlink{para7pg4}{\fbox{\Ctab{2.5mm}{\scalebox{1}{\scriptsize $\mathstrut||\!\lhd$}}}}}
\putnotew{101}{73}{\hyperlink{para8pg1}{\fbox{\Ctab{2.5mm}{\scalebox{1}{\scriptsize $\mathstrut|\!\lhd$}}}}}
\putnotew{108}{73}{\hyperlink{para8pg6}{\fbox{\Ctab{4.5mm}{\scalebox{1}{\scriptsize $\mathstrut\!\!\lhd\!\!$}}}}}
\putnotew{115}{73}{\hyperlink{para8pg7}{\fbox{\Ctab{4.5mm}{\scalebox{1}{\scriptsize $\mathstrut\!\rhd\!$}}}}}
\putnotew{120}{73}{\hyperlink{para8pg7}{\fbox{\Ctab{2.5mm}{\scalebox{1}{\scriptsize $\mathstrut \!\rhd\!\!|$}}}}}
\putnotew{125}{73}{\hyperlink{para9pg1}{\fbox{\Ctab{2.5mm}{\scalebox{1}{\scriptsize $\mathstrut \!\rhd\!\!||$}}}}}
\putnotee{126}{73}{\scriptsize\color{blue} 7/7}
\end{layer}

\slidepage
{\color{blue}

\begin{layer}{120}{0}
\putnotee{60}{15}{KeTMath a\_n}
\end{layer}

}
\begin{itemize}
\item
数の列\\
 $a_1,\ a_2,\ \cdots$
\item
ここでは規則的に並んだ数列とする\\
 $1,\ 3,\ 5,\ 7,\ \hakoma{9},\ \hakoma{11},\ \cdots$
\item
最初の項を{\color{red}初項}という
\item
最後の項({\color{red}末項})があるとき,項の数を{\color{red}項数}という
\end{itemize}

\newslide{数列の一般項}

\vspace*{18mm}


\begin{layer}{120}{0}
\putnotew{96}{73}{\hyperlink{para8pg7}{\fbox{\Ctab{2.5mm}{\scalebox{1}{\scriptsize $\mathstrut||\!\lhd$}}}}}
\putnotew{101}{73}{\hyperlink{para9pg1}{\fbox{\Ctab{2.5mm}{\scalebox{1}{\scriptsize $\mathstrut|\!\lhd$}}}}}
\putnotew{108}{73}{\hyperlink{para9pg5}{\fbox{\Ctab{4.5mm}{\scalebox{1}{\scriptsize $\mathstrut\!\!\lhd\!\!$}}}}}
\putnotew{115}{73}{\hyperlink{para9pg6}{\fbox{\Ctab{4.5mm}{\scalebox{1}{\scriptsize $\mathstrut\!\rhd\!$}}}}}
\putnotew{120}{73}{\hyperlink{para9pg6}{\fbox{\Ctab{2.5mm}{\scalebox{1}{\scriptsize $\mathstrut \!\rhd\!\!|$}}}}}
\putnotew{125}{73}{\hyperlink{para10pg1}{\fbox{\Ctab{2.5mm}{\scalebox{1}{\scriptsize $\mathstrut \!\rhd\!\!||$}}}}}
\putnotee{126}{73}{\scriptsize\color{blue} 6/6}
\end{layer}

\slidepage
\begin{itemize}
\item
$n$を正の整数とするとき,第$n$項を表す式を{\color{red}一般項}
\item
[例1)]$2,\ 3,\ 4,\ \cdots$\\
 一般項(第$n$項)は\ \hakoma{n+1}\vspace{-2mm}
\item
[例2)]$2,\ 4,\ 8,\ \cdots$\\
 一般項(第$n$項)は\ \hakoma{2^n}\vspace{-2mm}
\item
[課題]\monban 次を求めよ\hfill TextP200問1,問2\seteda{100}\\
\eda{一般項が$a_n=2^n$のとき,$a_1$から$a_5$までの値}\\
\eda{数列$\bunsuu{1}{2},\bunsuu{2}{3},\bunsuu{3}{4},\bunsuu{4}{5},\cdots$の一般項$a_n$}
\end{itemize}

\newslide{交互に符号が変わる数列の一般項}

\vspace*{18mm}


\begin{layer}{120}{0}
\putnotew{96}{73}{\hyperlink{para9pg6}{\fbox{\Ctab{2.5mm}{\scalebox{1}{\scriptsize $\mathstrut||\!\lhd$}}}}}
\putnotew{101}{73}{\hyperlink{para10pg1}{\fbox{\Ctab{2.5mm}{\scalebox{1}{\scriptsize $\mathstrut|\!\lhd$}}}}}
\putnotew{108}{73}{\hyperlink{para10pg8}{\fbox{\Ctab{4.5mm}{\scalebox{1}{\scriptsize $\mathstrut\!\!\lhd\!\!$}}}}}
\putnotew{115}{73}{\hyperlink{para10pg9}{\fbox{\Ctab{4.5mm}{\scalebox{1}{\scriptsize $\mathstrut\!\rhd\!$}}}}}
\putnotew{120}{73}{\hyperlink{para10pg9}{\fbox{\Ctab{2.5mm}{\scalebox{1}{\scriptsize $\mathstrut \!\rhd\!\!|$}}}}}
\putnotew{125}{73}{\hyperlink{para11pg1}{\fbox{\Ctab{2.5mm}{\scalebox{1}{\scriptsize $\mathstrut \!\rhd\!\!||$}}}}}
\putnotee{126}{73}{\scriptsize\color{blue} 9/9}
\end{layer}

\slidepage
\begin{itemize}
\item
$-1,\ 1,\ -1,\ 1,\ \cdots$\ の一般項は\ \hakoma{(-1)^n}
\item
$1,\ -1,\ 1,\ -1,\ \cdots$\ の一般項は\ \hakoma{(-1)^{n-1}}
\item
$1,\ -2,\ 3,\ -4,\ \cdots$\ の一般項は\ \hakoma{(-1)^{n-1}n}
\item
$2,\ 0,\ 2,\ 0,\ \cdots$\ の一般項は\ \hakoma{(-1)^{n-1}+1}
\item
[課題]\monban 次の数列の一般項はどうなるか.\seteda{50}\\
\eda{$1,\ 0,\ 1,\ 0,\ \cdots$}\eda{$0,\ 1,\ 0,\ 1,\ \cdots$}
\end{itemize}

\newslide{等差数列}

\vspace*{18mm}


\begin{layer}{120}{0}
\putnotew{96}{73}{\hyperlink{para10pg9}{\fbox{\Ctab{2.5mm}{\scalebox{1}{\scriptsize $\mathstrut||\!\lhd$}}}}}
\putnotew{101}{73}{\hyperlink{para11pg1}{\fbox{\Ctab{2.5mm}{\scalebox{1}{\scriptsize $\mathstrut|\!\lhd$}}}}}
\putnotew{108}{73}{\hyperlink{para11pg6}{\fbox{\Ctab{4.5mm}{\scalebox{1}{\scriptsize $\mathstrut\!\!\lhd\!\!$}}}}}
\putnotew{115}{73}{\hyperlink{para11pg7}{\fbox{\Ctab{4.5mm}{\scalebox{1}{\scriptsize $\mathstrut\!\rhd\!$}}}}}
\putnotew{120}{73}{\hyperlink{para11pg7}{\fbox{\Ctab{2.5mm}{\scalebox{1}{\scriptsize $\mathstrut \!\rhd\!\!|$}}}}}
\putnotew{125}{73}{\hyperlink{para12pg1}{\fbox{\Ctab{2.5mm}{\scalebox{1}{\scriptsize $\mathstrut \!\rhd\!\!||$}}}}}
\putnotee{126}{73}{\scriptsize\color{blue} 7/7}
\end{layer}

\slidepage

\begin{layer}{120}{0}
\putnotes{28}{19}{\scb[1.2]{$\smile$}}
\putnotes{28}{21}{\small $+2$}
\putnotes{35.5}{19}{\scb[1.2]{$\smile$}}
\putnotes{35.5}{21}{\small $+2$}
\putnotes{44}{19}{\scb[1.2]{$\smile$}}
\putnotes{44}{21}{\small $+2$}
\end{layer}

\begin{itemize}
\item
差({\color{red}公差})が等しい数列\\
 例)$1,\ 3,\ 5,\ 7,\ \cdots$\vspace{6mm}
\item
初項を$a$,公差を$d$とおくと\\
 $a,\ a+d,\ a+2d,\ \cdots,\ \mbox{第$n$項}$は?
\item
第$n$項(一般項)は\ $a_n=a+(n-1)d$
\item
[(例)]等差数列$1,\ 3,\ 5,\ 7,\ \cdots$の一般項は\\
\hspace*{2zw}$a_n=1+2\hakoma{(n-1)}=2n-1$
\end{itemize}

\newslide{等比数列}

\vspace*{18mm}


\begin{layer}{120}{0}
\putnotew{96}{73}{\hyperlink{para11pg7}{\fbox{\Ctab{2.5mm}{\scalebox{1}{\scriptsize $\mathstrut||\!\lhd$}}}}}
\putnotew{101}{73}{\hyperlink{para12pg1}{\fbox{\Ctab{2.5mm}{\scalebox{1}{\scriptsize $\mathstrut|\!\lhd$}}}}}
\putnotew{108}{73}{\hyperlink{para12pg5}{\fbox{\Ctab{4.5mm}{\scalebox{1}{\scriptsize $\mathstrut\!\!\lhd\!\!$}}}}}
\putnotew{115}{73}{\hyperlink{para12pg6}{\fbox{\Ctab{4.5mm}{\scalebox{1}{\scriptsize $\mathstrut\!\rhd\!$}}}}}
\putnotew{120}{73}{\hyperlink{para12pg6}{\fbox{\Ctab{2.5mm}{\scalebox{1}{\scriptsize $\mathstrut \!\rhd\!\!|$}}}}}
\putnotew{125}{73}{\hyperlink{para13pg1}{\fbox{\Ctab{2.5mm}{\scalebox{1}{\scriptsize $\mathstrut \!\rhd\!\!||$}}}}}
\putnotee{126}{73}{\scriptsize\color{blue} 6/6}
\end{layer}

\slidepage

\begin{layer}{120}{0}
\putnotes{28}{20}{\scb[1.2]{$\smile$}}
\putnotes{28}{22}{\small $\times 3$}
\putnotes{36}{20}{\scb[1.2]{$\smile$}}
\putnotes{36}{22}{\small $\times 3$}
\putnotes{46}{20}{\scb[1.2]{$\smile$}}
\putnotes{46}{22}{\small $\times 3$}
\putnotee{51}{57}{{\color{red}$=6^{n-1}$としない}}
\end{layer}

\begin{itemize}
\item
比({\color{red}公比})が等しい数列\\
 例)$2,\ 6,\ 18,\ 54,\ \cdots$\vspace{6mm}
\item
初項を$a$,公比を$r$とおくと\\
 $a,\ ar,\ ar^2,\ \cdots,\ \mbox{第$n$項}a_n=\hakoma{a\,r^{n-1}}$
\item
[例)]等比数列$2,\ 6,\ 18,\ 54,\ \cdots$\;の一般項は\\
\hspace*{2zw}$a_n=2\cdot 3^{n-1}$
\end{itemize}

\newslide{課題(等差数列と等比数列)}

\vspace*{18mm}


\begin{layer}{120}{0}
\putnotew{96}{73}{\hyperlink{para12pg6}{\fbox{\Ctab{2.5mm}{\scalebox{1}{\scriptsize $\mathstrut||\!\lhd$}}}}}
\putnotew{125}{73}{\hyperlink{para14pg1}{\fbox{\Ctab{2.5mm}{\scalebox{1}{\scriptsize $\mathstrut \!\rhd\!\!||$}}}}}
\putnotee{126}{73}{\scriptsize\color{blue} 1/1}
\end{layer}

\slidepage
\seteda{70}
\begin{itemize}
\item
[課題]\monban 次を求めよ\hfill Text P201,203\seteda{100}\\
\eda{初項2,\ 公差3の等差数列$\{a_n\}$の一般項$a_n$}\\
\eda{$a_{10}$}\\
\eda{初項2,\ 公比$-3$の等比数列$\{b_n\}$の一般項$b_n$}\\
\eda{$b_{5}$}
\end{itemize}

%%%%%%%%%%%%%%%%%%%%


\newslide{等差数列の和}

\vspace*{18mm}


\begin{layer}{120}{0}
\putnotew{96}{73}{\hyperlink{para13pg1}{\fbox{\Ctab{2.5mm}{\scalebox{1}{\scriptsize $\mathstrut||\!\lhd$}}}}}
\putnotew{101}{73}{\hyperlink{para14pg1}{\fbox{\Ctab{2.5mm}{\scalebox{1}{\scriptsize $\mathstrut|\!\lhd$}}}}}
\putnotew{108}{73}{\hyperlink{para14pg4}{\fbox{\Ctab{4.5mm}{\scalebox{1}{\scriptsize $\mathstrut\!\!\lhd\!\!$}}}}}
\putnotew{115}{73}{\hyperlink{para14pg5}{\fbox{\Ctab{4.5mm}{\scalebox{1}{\scriptsize $\mathstrut\!\rhd\!$}}}}}
\putnotew{120}{73}{\hyperlink{para14pg5}{\fbox{\Ctab{2.5mm}{\scalebox{1}{\scriptsize $\mathstrut \!\rhd\!\!|$}}}}}
\putnotew{125}{73}{\hyperlink{para15pg1}{\fbox{\Ctab{2.5mm}{\scalebox{1}{\scriptsize $\mathstrut \!\rhd\!\!||$}}}}}
\putnotee{126}{73}{\scriptsize\color{blue} 5/5}
\end{layer}

\slidepage
{\color{red}

\begin{layer}{120}{0}
\boxframec{32}{24}{20}{18}{}
\boxframec{58}{24}{20}{18}{}
\boxframec{85}{24}{20}{18}{}
\boxframec{112}{24}{20}{18}{}
\end{layer}

}
\vspace{2mm}

初項$a$,公差$d$,項数が4の場合で説明する
\begin{itemize}
\item
[]$S=\Ctab{20mm}{a}+\Ctab{17mm}{(a+d)}+\Ctab{17mm}{(a+2d)}+\Ctab{17mm}{(a+3d)}$
\item
[]$S=\Ctab{20mm}{(a+3d)}+\Ctab{17mm}{(a+2d)}+\Ctab{17mm}{(a+d)}+\Ctab{17mm}{a}$
\item
[]2つの式を加えると\\
 $2S=(2a+3d)\times 4$\\
  したがって $S=\bunsuu{4(2a+3d)}{2}$
\end{itemize}

\newslide{等差数列の和の公式}

\vspace*{18mm}


\begin{layer}{120}{0}
\putnotew{96}{73}{\hyperlink{para14pg5}{\fbox{\Ctab{2.5mm}{\scalebox{1}{\scriptsize $\mathstrut||\!\lhd$}}}}}
\putnotew{101}{73}{\hyperlink{para15pg1}{\fbox{\Ctab{2.5mm}{\scalebox{1}{\scriptsize $\mathstrut|\!\lhd$}}}}}
\putnotew{108}{73}{\hyperlink{para15pg2}{\fbox{\Ctab{4.5mm}{\scalebox{1}{\scriptsize $\mathstrut\!\!\lhd\!\!$}}}}}
\putnotew{115}{73}{\hyperlink{para15pg3}{\fbox{\Ctab{4.5mm}{\scalebox{1}{\scriptsize $\mathstrut\!\rhd\!$}}}}}
\putnotew{120}{73}{\hyperlink{para15pg3}{\fbox{\Ctab{2.5mm}{\scalebox{1}{\scriptsize $\mathstrut \!\rhd\!\!|$}}}}}
\putnotew{125}{73}{\hyperlink{para16pg1}{\fbox{\Ctab{2.5mm}{\scalebox{1}{\scriptsize $\mathstrut \!\rhd\!\!||$}}}}}
\putnotee{126}{73}{\scriptsize\color{blue} 3/3}
\end{layer}

\slidepage
\begin{itemize}
\item
[]初項$a$,公差$d$,項数$n$の等差数列の和$S$は\vspace{2mm}\\
  \fbox{$S=\bunsuu{n(2a+(n-1)d)}{2}$}
\item
[]$2a+(n-1)d=a+(a+(n-1)d)=\mbox{初項}+\mbox{末項}$
\item
[]  \raisebox{-5mm}{\fbox{$S=\bunsuu{\mbox{項数}\times (\mbox{初項}+\mbox{末項})}{2}$}}
\end{itemize}

\newslide{等差数列の和の例題}

\vspace*{18mm}


\begin{layer}{120}{0}
\putnotew{96}{73}{\hyperlink{para15pg3}{\fbox{\Ctab{2.5mm}{\scalebox{1}{\scriptsize $\mathstrut||\!\lhd$}}}}}
\putnotew{101}{73}{\hyperlink{para16pg1}{\fbox{\Ctab{2.5mm}{\scalebox{1}{\scriptsize $\mathstrut|\!\lhd$}}}}}
\putnotew{108}{73}{\hyperlink{para16pg8}{\fbox{\Ctab{4.5mm}{\scalebox{1}{\scriptsize $\mathstrut\!\!\lhd\!\!$}}}}}
\putnotew{115}{73}{\hyperlink{para16pg9}{\fbox{\Ctab{4.5mm}{\scalebox{1}{\scriptsize $\mathstrut\!\rhd\!$}}}}}
\putnotew{120}{73}{\hyperlink{para16pg9}{\fbox{\Ctab{2.5mm}{\scalebox{1}{\scriptsize $\mathstrut \!\rhd\!\!|$}}}}}
\putnotew{125}{73}{\hyperlink{para17pg1}{\fbox{\Ctab{2.5mm}{\scalebox{1}{\scriptsize $\mathstrut \!\rhd\!\!||$}}}}}
\putnotee{126}{73}{\scriptsize\color{blue} 9/9}
\end{layer}

\slidepage
\begin{itemize}
\item
[例題)]$S=1+3+5+7+\cdots+99$を求めよ\vspace{-3mm}
\item
[解)]項数$n$を求める.\\
初項$1$,公差$2$より,第$n$項$a_n$は\\
   $a_n=1+2(n-1)=2n-1$\\
$a_n=2n-1=99$より
 $n=\bunsuu{99+1}{2}=50$\vspace{-2mm}\\
したがって $S=\bunsuu{50(1+99)}{2}$
$=2500$\vspace{-3mm}
\item
[課題]\monban $S=1+2+3+\cdots+100$について\seteda{50}\\
\eda{項数を求めよ}\eda{和$S$を求めよ}
\end{itemize}
\label{pageend}\mbox{}

\end{document}
