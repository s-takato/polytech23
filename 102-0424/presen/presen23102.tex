%%% Title presen23102
\documentclass[landscape,10pt]{ujarticle}
\special{papersize=\the\paperwidth,\the\paperheight}
\usepackage{ketpic,ketlayer}
\usepackage{ketslide}
\usepackage{amsmath,amssymb}
\usepackage{bm,enumerate}
\usepackage[dvipdfmx]{graphicx}
\usepackage{color}
\definecolor{slidecolora}{cmyk}{0.98,0.13,0,0.43}
\definecolor{slidecolorb}{cmyk}{0.2,0,0,0}
\definecolor{slidecolorc}{cmyk}{0.2,0,0,0}
\definecolor{slidecolord}{cmyk}{0.2,0,0,0}
\definecolor{slidecolore}{cmyk}{0,0,0,0.5}
\definecolor{slidecolorf}{cmyk}{0,0,0,0.5}
\definecolor{slidecolori}{cmyk}{0.98,0.13,0,0.43}
\def\setthin#1{\def\thin{#1}}
\setthin{0}
\newcommand{\slidepage}[1][s]{%
\setcounter{ketpicctra}{18}%
\if#1m \setcounter{ketpicctra}{1}\fi
\hypersetup{linkcolor=black}%

\begin{layer}{118}{0}
\putnotee{122}{-\theketpicctra.05}{\small\thepage/\pageref{pageend}}
\end{layer}\hypersetup{linkcolor=blue}

}
\usepackage{emath}
\usepackage{emathEy}
\usepackage{emathMw}
\usepackage{pict2e}
\usepackage{ketlayermorewith2e}
\usepackage[dvipdfmx,colorlinks=true,linkcolor=blue,filecolor=blue]{hyperref}
\newcommand{\hiduke}{0424}
\newcommand{\hako}[2][1]{\fbox{\raisebox{#1mm}{\mbox{}}\raisebox{-#1mm}{\mbox{}}\,\phantom{#2}\,}}
\newcommand{\hakoa}[2][1]{\fbox{\raisebox{#1mm}{\mbox{}}\raisebox{-#1mm}{\mbox{}}\,#2\,}}
\newcommand{\hakom}[2][1]{\hako[#1]{$#2$}}
\newcommand{\hakoma}[2][1]{\hakoa[#1]{$#2$}}
\def\rad{\;\mathrm{rad}}
\def\deg#1{#1^{\circ}}
\newcommand{\sbunsuu}[2]{\scalebox{0.6}{$\bunsuu{#1}{#2}$}}
\def\pow{$\hspace{-1.5mm}^\hspace{-1mm}$}
\def\dlim{\displaystyle\lim}
\newcommand{\brd}[2][1]{\scalebox{#1}{\color{red}\fbox{\color{black}$#2$}}}
\newcommand\down[1][0.5zw]{\vspace{#1}\\}
\newcommand{\sfrac}[3][0.65]{\scalebox{#1}{$\frac{#2}{#3}$}}
\newcommand{\phn}[1]{\phantom{#1}}
\newcommand{\scb}[2][0.6]{\scalebox{#1}{#2}}
\newcommand{\dsum}{\displaystyle\sum}
\def\pow{$\hspace{-1.5mm}^\hspace{-1mm}$}
\def\dlim{\displaystyle\lim}
\def\dint{\displaystyle\int}

\setmargin{25}{145}{15}{100}

\ketslideinit

\pagestyle{empty}

\begin{document}

\begin{layer}{120}{0}
\putnotese{0}{0}{{\Large\bf
\color[cmyk]{1,1,0,0}

\begin{layer}{120}{0}
{\Huge \putnotes{60}{20}{三角比と三角関数}}
\putnotes{60}{70}{2022.04.25}
\end{layer}

}
}
\end{layer}

\def\mainslidetitley{22}
\def\ketcletter{slidecolora}
\def\ketcbox{slidecolorb}
\def\ketdbox{slidecolorc}
\def\ketcframe{slidecolord}
\def\ketcshadow{slidecolore}
\def\ketdshadow{slidecolorf}
\def\slidetitlex{6}
\def\slidetitlesize{1.3}
\def\mketcletter{slidecolori}
\def\mketcbox{yellow}
\def\mketdbox{yellow}
\def\mketcframe{yellow}
\def\mslidetitlex{62}
\def\mslidetitlesize{2}

\color{black}
\Large\bf\boldmath
\addtocounter{page}{-1}

\def\MARU{}
\renewcommand{\MARU}[1]{{\ooalign{\hfil$#1$\/\hfil\crcr\raise.167ex\hbox{\mathhexbox20D}}}}
\renewcommand{\slidepage}[1][s]{%
\setcounter{ketpicctra}{18}%
\if#1m \setcounter{ketpicctra}{1}\fi
\hypersetup{linkcolor=black}%
\begin{layer}{118}{0}
\putnotee{115}{-\theketpicctra.05}{\small\hiduke-\thepage/\pageref{pageend}}
\end{layer}\hypersetup{linkcolor=blue}
}
\newcounter{ban}
\setcounter{ban}{1}
\newcommand{\monban}[1][\hiduke]{%
#1-\theban\ %
\addtocounter{ban}{1}%
}
\newcommand{\monbannoadd}[1][\hiduke]{%
#1-\theban\ %
}
\newcommand{\addban}{%
\addtocounter{ban}{1}%
}
\newcounter{edawidth}
\newcounter{edactr}
\newcommand{\seteda}[1]{% 20220708 modified
\setcounter{edawidth}{#1}
\setcounter{edactr}{1}
}
\newcommand{\eda}[2][\theedawidth]{%
\Ltab{#1 mm}{[\theedactr]\ #2}%
\addtocounter{edactr}{1}%
}
%%%%%%%%%%%%

%%%%%%%%%%%%%%%%%%%%

\mainslide{復習(関数)}


\slidepage[m]
%%%%%%%%%%%%

%%%%%%%%%%%%%%%%%%%%

\newslide{関数}

\vspace*{18mm}

\slidepage
\begin{itemize}
\item
変数$x$の値を与えると変数$y$の値が求まる\\
\hspace*{1zw}例)$y=2x+1,\ y=x^2+2x+1$
\end{itemize}
%%%%%%%%%%%%

%%%%%%%%%%%%%%%%%%%%


\sameslide

\vspace*{18mm}

\slidepage
\begin{itemize}
\item
変数$x$の値を与えると変数$y$の値が求まる\\
\hspace*{1zw}例)$y=2x+1,\ y=x^2+2x+1$
\item
これを変数$x$の関数という
\end{itemize}

\sameslide

\vspace*{18mm}

\slidepage
\begin{itemize}
\item
変数$x$の値を与えると変数$y$の値が求まる\\
\hspace*{1zw}例)$y=2x+1,\ y=x^2+2x+1$
\item
これを変数$x$の関数という
\item
変数$x$の関数であることを$f(x)$などで表す\\
 例1)$f(x)=2x+1$(1次関数)\\
 例2)$g(x)=x^2+2x+1$(2次関数)
\end{itemize}

\newslide{関数記号}

\vspace*{18mm}

\slidepage
\seteda{55}
\begin{itemize}
\item
関数$f(x)$の$x$に定数$a$を代入した値を$f(a)$で表す
\item
例)$f(x)=x^2+x-1$のとき $f(2)=2^2+2-1=5$
\end{itemize}
%%%%%%%%%%%%

%%%%%%%%%%%%%%%%%%%%


\sameslide

\vspace*{18mm}

\slidepage
\seteda{55}
\begin{itemize}
\item
関数$f(x)$の$x$に定数$a$を代入した値を$f(a)$で表す
\item
例)$f(x)=x^2+x-1$のとき $f(2)=2^2+2-1=5$
\item
課題\monban $f(x)=x^2-1$のとき,次を求めよ.\\
\eda{$f(0)$}\eda{$f(1)$}\\
\eda{$f(-2)$}\eda{$f(a+1)$\ \ \ ($a$は定数)}\\
\end{itemize}

\newslide{関数のグラフ}

\vspace*{18mm}

\slidepage
\down
関数$y=f(x)$
\begin{itemize}
\item
$x$を変えるとき,点$\bigl(x,\ f(x)\bigr)$も変わる.
\item
[]例) 1次関数$y=2x+1$\vspace{1mm}\\
%%% /polytech22.git/102(0418)/presen/fig/table1aans.tex 
%%% Generator=presen22102.cdy 
{\unitlength=1cm%
\begin{picture}%
(9.6,1.2)(0,0)%
\linethickness{0.008in}%%
\Large\bf\boldmath%
\small%
\polyline(0.00000,1.20000)(0.00000,0.00000)%
%
\polyline(0.80000,1.20000)(0.80000,0.00000)%
%
\polyline(1.60000,1.20000)(1.60000,0.00000)%
%
\polyline(2.40000,1.20000)(2.40000,0.00000)%
%
\polyline(3.20000,1.20000)(3.20000,0.00000)%
%
\polyline(4.00000,1.20000)(4.00000,0.00000)%
%
\polyline(4.80000,1.20000)(4.80000,0.00000)%
%
\polyline(5.60000,1.20000)(5.60000,0.00000)%
%
\polyline(6.40000,1.20000)(6.40000,0.00000)%
%
\polyline(7.20000,1.20000)(7.20000,0.00000)%
%
\polyline(8.00000,1.20000)(8.00000,0.00000)%
%
\polyline(8.80000,1.20000)(8.80000,0.00000)%
%
\polyline(9.60000,1.20000)(9.60000,0.00000)%
%
\polyline(0.00000,1.20000)(9.60000,1.20000)%
%
\polyline(0.00000,0.60000)(9.60000,0.60000)%
%
\polyline(0.00000,0.00000)(9.60000,0.00000)%
%
\settowidth{\Width}{$x$}\setlength{\Width}{-0.5\Width}%
\settoheight{\Height}{$x$}\settodepth{\Depth}{$x$}\setlength{\Height}{-0.5\Height}\setlength{\Depth}{0.5\Depth}\addtolength{\Height}{\Depth}%
\put(0.4000000,0.9000000){\hspace*{\Width}\raisebox{\Height}{$x$}}%
%
\settowidth{\Width}{$-5$}\setlength{\Width}{-0.5\Width}%
\settoheight{\Height}{$-5$}\settodepth{\Depth}{$-5$}\setlength{\Height}{-0.5\Height}\setlength{\Depth}{0.5\Depth}\addtolength{\Height}{\Depth}%
\put(1.2000000,0.9000000){\hspace*{\Width}\raisebox{\Height}{$-5$}}%
%
\settowidth{\Width}{$-4$}\setlength{\Width}{-0.5\Width}%
\settoheight{\Height}{$-4$}\settodepth{\Depth}{$-4$}\setlength{\Height}{-0.5\Height}\setlength{\Depth}{0.5\Depth}\addtolength{\Height}{\Depth}%
\put(2.0000000,0.9000000){\hspace*{\Width}\raisebox{\Height}{$-4$}}%
%
\settowidth{\Width}{$-3$}\setlength{\Width}{-0.5\Width}%
\settoheight{\Height}{$-3$}\settodepth{\Depth}{$-3$}\setlength{\Height}{-0.5\Height}\setlength{\Depth}{0.5\Depth}\addtolength{\Height}{\Depth}%
\put(2.8000000,0.9000000){\hspace*{\Width}\raisebox{\Height}{$-3$}}%
%
\settowidth{\Width}{$-2$}\setlength{\Width}{-0.5\Width}%
\settoheight{\Height}{$-2$}\settodepth{\Depth}{$-2$}\setlength{\Height}{-0.5\Height}\setlength{\Depth}{0.5\Depth}\addtolength{\Height}{\Depth}%
\put(3.6000000,0.9000000){\hspace*{\Width}\raisebox{\Height}{$-2$}}%
%
\settowidth{\Width}{$-1$}\setlength{\Width}{-0.5\Width}%
\settoheight{\Height}{$-1$}\settodepth{\Depth}{$-1$}\setlength{\Height}{-0.5\Height}\setlength{\Depth}{0.5\Depth}\addtolength{\Height}{\Depth}%
\put(4.4000000,0.9000000){\hspace*{\Width}\raisebox{\Height}{$-1$}}%
%
\settowidth{\Width}{$0$}\setlength{\Width}{-0.5\Width}%
\settoheight{\Height}{$0$}\settodepth{\Depth}{$0$}\setlength{\Height}{-0.5\Height}\setlength{\Depth}{0.5\Depth}\addtolength{\Height}{\Depth}%
\put(5.2000000,0.9000000){\hspace*{\Width}\raisebox{\Height}{$0$}}%
%
\settowidth{\Width}{$1$}\setlength{\Width}{-0.5\Width}%
\settoheight{\Height}{$1$}\settodepth{\Depth}{$1$}\setlength{\Height}{-0.5\Height}\setlength{\Depth}{0.5\Depth}\addtolength{\Height}{\Depth}%
\put(6.0000000,0.9000000){\hspace*{\Width}\raisebox{\Height}{$1$}}%
%
\settowidth{\Width}{$2$}\setlength{\Width}{-0.5\Width}%
\settoheight{\Height}{$2$}\settodepth{\Depth}{$2$}\setlength{\Height}{-0.5\Height}\setlength{\Depth}{0.5\Depth}\addtolength{\Height}{\Depth}%
\put(6.8000000,0.9000000){\hspace*{\Width}\raisebox{\Height}{$2$}}%
%
\settowidth{\Width}{$3$}\setlength{\Width}{-0.5\Width}%
\settoheight{\Height}{$3$}\settodepth{\Depth}{$3$}\setlength{\Height}{-0.5\Height}\setlength{\Depth}{0.5\Depth}\addtolength{\Height}{\Depth}%
\put(7.6000000,0.9000000){\hspace*{\Width}\raisebox{\Height}{$3$}}%
%
\settowidth{\Width}{$4$}\setlength{\Width}{-0.5\Width}%
\settoheight{\Height}{$4$}\settodepth{\Depth}{$4$}\setlength{\Height}{-0.5\Height}\setlength{\Depth}{0.5\Depth}\addtolength{\Height}{\Depth}%
\put(8.4000000,0.9000000){\hspace*{\Width}\raisebox{\Height}{$4$}}%
%
\settowidth{\Width}{$5$}\setlength{\Width}{-0.5\Width}%
\settoheight{\Height}{$5$}\settodepth{\Depth}{$5$}\setlength{\Height}{-0.5\Height}\setlength{\Depth}{0.5\Depth}\addtolength{\Height}{\Depth}%
\put(9.2000000,0.9000000){\hspace*{\Width}\raisebox{\Height}{$5$}}%
%
\settowidth{\Width}{$y$}\setlength{\Width}{-0.5\Width}%
\settoheight{\Height}{$y$}\settodepth{\Depth}{$y$}\setlength{\Height}{-0.5\Height}\setlength{\Depth}{0.5\Depth}\addtolength{\Height}{\Depth}%
\put(0.4000000,0.3000000){\hspace*{\Width}\raisebox{\Height}{$y$}}%
%
\settowidth{\Width}{$-9$}\setlength{\Width}{-0.5\Width}%
\settoheight{\Height}{$-9$}\settodepth{\Depth}{$-9$}\setlength{\Height}{-0.5\Height}\setlength{\Depth}{0.5\Depth}\addtolength{\Height}{\Depth}%
\put(1.2000000,0.3000000){\hspace*{\Width}\raisebox{\Height}{$-9$}}%
%
\settowidth{\Width}{$-7$}\setlength{\Width}{-0.5\Width}%
\settoheight{\Height}{$-7$}\settodepth{\Depth}{$-7$}\setlength{\Height}{-0.5\Height}\setlength{\Depth}{0.5\Depth}\addtolength{\Height}{\Depth}%
\put(2.0000000,0.3000000){\hspace*{\Width}\raisebox{\Height}{$-7$}}%
%
\settowidth{\Width}{$-5$}\setlength{\Width}{-0.5\Width}%
\settoheight{\Height}{$-5$}\settodepth{\Depth}{$-5$}\setlength{\Height}{-0.5\Height}\setlength{\Depth}{0.5\Depth}\addtolength{\Height}{\Depth}%
\put(2.8000000,0.3000000){\hspace*{\Width}\raisebox{\Height}{$-5$}}%
%
\settowidth{\Width}{$-3$}\setlength{\Width}{-0.5\Width}%
\settoheight{\Height}{$-3$}\settodepth{\Depth}{$-3$}\setlength{\Height}{-0.5\Height}\setlength{\Depth}{0.5\Depth}\addtolength{\Height}{\Depth}%
\put(3.6000000,0.3000000){\hspace*{\Width}\raisebox{\Height}{$-3$}}%
%
\settowidth{\Width}{$-1$}\setlength{\Width}{-0.5\Width}%
\settoheight{\Height}{$-1$}\settodepth{\Depth}{$-1$}\setlength{\Height}{-0.5\Height}\setlength{\Depth}{0.5\Depth}\addtolength{\Height}{\Depth}%
\put(4.4000000,0.3000000){\hspace*{\Width}\raisebox{\Height}{$-1$}}%
%
\settowidth{\Width}{$1$}\setlength{\Width}{-0.5\Width}%
\settoheight{\Height}{$1$}\settodepth{\Depth}{$1$}\setlength{\Height}{-0.5\Height}\setlength{\Depth}{0.5\Depth}\addtolength{\Height}{\Depth}%
\put(5.2000000,0.3000000){\hspace*{\Width}\raisebox{\Height}{$1$}}%
%
\settowidth{\Width}{$3$}\setlength{\Width}{-0.5\Width}%
\settoheight{\Height}{$3$}\settodepth{\Depth}{$3$}\setlength{\Height}{-0.5\Height}\setlength{\Depth}{0.5\Depth}\addtolength{\Height}{\Depth}%
\put(6.0000000,0.3000000){\hspace*{\Width}\raisebox{\Height}{$3$}}%
%
\settowidth{\Width}{$5$}\setlength{\Width}{-0.5\Width}%
\settoheight{\Height}{$5$}\settodepth{\Depth}{$5$}\setlength{\Height}{-0.5\Height}\setlength{\Depth}{0.5\Depth}\addtolength{\Height}{\Depth}%
\put(6.8000000,0.3000000){\hspace*{\Width}\raisebox{\Height}{$5$}}%
%
\settowidth{\Width}{$7$}\setlength{\Width}{-0.5\Width}%
\settoheight{\Height}{$7$}\settodepth{\Depth}{$7$}\setlength{\Height}{-0.5\Height}\setlength{\Depth}{0.5\Depth}\addtolength{\Height}{\Depth}%
\put(7.6000000,0.3000000){\hspace*{\Width}\raisebox{\Height}{$7$}}%
%
\settowidth{\Width}{$9$}\setlength{\Width}{-0.5\Width}%
\settoheight{\Height}{$9$}\settodepth{\Depth}{$9$}\setlength{\Height}{-0.5\Height}\setlength{\Depth}{0.5\Depth}\addtolength{\Height}{\Depth}%
\put(8.4000000,0.3000000){\hspace*{\Width}\raisebox{\Height}{$9$}}%
%
\settowidth{\Width}{$11$}\setlength{\Width}{-0.5\Width}%
\settoheight{\Height}{$11$}\settodepth{\Depth}{$11$}\setlength{\Height}{-0.5\Height}\setlength{\Depth}{0.5\Depth}\addtolength{\Height}{\Depth}%
\put(9.2000000,0.3000000){\hspace*{\Width}\raisebox{\Height}{$11$}}%
%
\end{picture}}%
\end{itemize}
%%%%%%%%%%%%

%%%%%%%%%%%%%%%%%%%%


\sameslide

\vspace*{18mm}

\slidepage
\down
関数$y=f(x)$
\begin{itemize}
\item
$x$を変えるとき,点$\bigl(x,\ f(x)\bigr)$も変わる.
\item
[]例) 1次関数$y=2x+1$\vspace{1mm}\\
\item
この点の集まりを,その関数の{\color{red}グラフ}という.
\end{itemize}

\newslide{1次関数のグラフ}

\vspace*{18mm}

\slidepage
\down
例)$y=2x+1$

\begin{layer}{120}{0}
\putnotes{60}{6}{\scalebox{0.5}{%%% /Users/takatoosetsuo/polytech23.git/102-0424/presen/fig/graphpaper1.tex 
%%% Generator=200601.cdy 
{\unitlength=1cm%
\begin{picture}%
(12.4,12.4)(-6.2,-6.2)%
\linethickness{0.008in}%%
\Large\bf\boldmath%
\small%
\linethickness{0.006in}%%
\polyline(-6,6)(-6,-6)%
%
\linethickness{0.008in}%%
\linethickness{0.006in}%%
\polyline(-5,6)(-5,-6)%
%
\linethickness{0.008in}%%
\linethickness{0.006in}%%
\polyline(-4,6)(-4,-6)%
%
\linethickness{0.008in}%%
\linethickness{0.006in}%%
\polyline(-3,6)(-3,-6)%
%
\linethickness{0.008in}%%
\linethickness{0.006in}%%
\polyline(-2,6)(-2,-6)%
%
\linethickness{0.008in}%%
\linethickness{0.006in}%%
\polyline(-1,6)(-1,-6)%
%
\linethickness{0.008in}%%
\linethickness{0.006in}%%
\polyline(0,6)(0,-6)%
%
\linethickness{0.008in}%%
\linethickness{0.006in}%%
\polyline(1,6)(1,-6)%
%
\linethickness{0.008in}%%
\linethickness{0.006in}%%
\polyline(2,6)(2,-6)%
%
\linethickness{0.008in}%%
\linethickness{0.006in}%%
\polyline(3,6)(3,-6)%
%
\linethickness{0.008in}%%
\linethickness{0.006in}%%
\polyline(4,6)(4,-6)%
%
\linethickness{0.008in}%%
\linethickness{0.006in}%%
\polyline(5,6)(5,-6)%
%
\linethickness{0.008in}%%
\linethickness{0.006in}%%
\polyline(6,6)(6,-6)%
%
\linethickness{0.008in}%%
\linethickness{0.006in}%%
\polyline(-6,6)(6,6)%
%
\linethickness{0.008in}%%
\linethickness{0.006in}%%
\polyline(-6,5)(6,5)%
%
\linethickness{0.008in}%%
\linethickness{0.006in}%%
\polyline(-6,4)(6,4)%
%
\linethickness{0.008in}%%
\linethickness{0.006in}%%
\polyline(-6,3)(6,3)%
%
\linethickness{0.008in}%%
\linethickness{0.006in}%%
\polyline(-6,2)(6,2)%
%
\linethickness{0.008in}%%
\linethickness{0.006in}%%
\polyline(-6,1)(6,1)%
%
\linethickness{0.008in}%%
\linethickness{0.006in}%%
\polyline(-6,0)(6,0)%
%
\linethickness{0.008in}%%
\linethickness{0.006in}%%
\polyline(-6,-1)(6,-1)%
%
\linethickness{0.008in}%%
\linethickness{0.006in}%%
\polyline(-6,-2)(6,-2)%
%
\linethickness{0.008in}%%
\linethickness{0.006in}%%
\polyline(-6,-3)(6,-3)%
%
\linethickness{0.008in}%%
\linethickness{0.006in}%%
\polyline(-6,-4)(6,-4)%
%
\linethickness{0.008in}%%
\linethickness{0.006in}%%
\polyline(-6,-5)(6,-5)%
%
\linethickness{0.008in}%%
\linethickness{0.006in}%%
\polyline(-6,-6)(6,-6)%
%
\linethickness{0.008in}%%
\linethickness{0.004in}%%
\polyline(-5.5,-6)(-5.5,-5.901)\polyline(-5.5,-5.802)(-5.5,-5.702)\polyline(-5.5,-5.603)(-5.5,-5.504)%
\polyline(-5.5,-5.405)(-5.5,-5.306)\polyline(-5.5,-5.207)(-5.5,-5.107)\polyline(-5.5,-5.008)(-5.5,-4.909)%
\polyline(-5.5,-4.81)(-5.5,-4.711)\polyline(-5.5,-4.612)(-5.5,-4.512)\polyline(-5.5,-4.413)(-5.5,-4.314)%
\polyline(-5.5,-4.215)(-5.5,-4.116)\polyline(-5.5,-4.017)(-5.5,-3.917)\polyline(-5.5,-3.818)(-5.5,-3.719)%
\polyline(-5.5,-3.62)(-5.5,-3.521)\polyline(-5.5,-3.421)(-5.5,-3.322)\polyline(-5.5,-3.223)(-5.5,-3.124)%
\polyline(-5.5,-3.025)(-5.5,-2.926)\polyline(-5.5,-2.826)(-5.5,-2.727)\polyline(-5.5,-2.628)(-5.5,-2.529)%
\polyline(-5.5,-2.43)(-5.5,-2.331)\polyline(-5.5,-2.231)(-5.5,-2.132)\polyline(-5.5,-2.033)(-5.5,-1.934)%
\polyline(-5.5,-1.835)(-5.5,-1.736)\polyline(-5.5,-1.636)(-5.5,-1.537)\polyline(-5.5,-1.438)(-5.5,-1.339)%
\polyline(-5.5,-1.24)(-5.5,-1.14)\polyline(-5.5,-1.041)(-5.5,-0.942)\polyline(-5.5,-0.843)(-5.5,-0.744)%
\polyline(-5.5,-0.645)(-5.5,-0.545)\polyline(-5.5,-0.446)(-5.5,-0.347)\polyline(-5.5,-0.248)(-5.5,-0.149)%
\polyline(-5.5,-0.05)(-5.5,0.05)\polyline(-5.5,0.149)(-5.5,0.248)\polyline(-5.5,0.347)(-5.5,0.446)%
\polyline(-5.5,0.545)(-5.5,0.645)\polyline(-5.5,0.744)(-5.5,0.843)\polyline(-5.5,0.942)(-5.5,1.041)%
\polyline(-5.5,1.14)(-5.5,1.24)\polyline(-5.5,1.339)(-5.5,1.438)\polyline(-5.5,1.537)(-5.5,1.636)%
\polyline(-5.5,1.736)(-5.5,1.835)\polyline(-5.5,1.934)(-5.5,2.033)\polyline(-5.5,2.132)(-5.5,2.231)%
\polyline(-5.5,2.331)(-5.5,2.43)\polyline(-5.5,2.529)(-5.5,2.628)\polyline(-5.5,2.727)(-5.5,2.826)%
\polyline(-5.5,2.926)(-5.5,3.025)\polyline(-5.5,3.124)(-5.5,3.223)\polyline(-5.5,3.322)(-5.5,3.421)%
\polyline(-5.5,3.521)(-5.5,3.62)\polyline(-5.5,3.719)(-5.5,3.818)\polyline(-5.5,3.917)(-5.5,4.017)%
\polyline(-5.5,4.116)(-5.5,4.215)\polyline(-5.5,4.314)(-5.5,4.413)\polyline(-5.5,4.512)(-5.5,4.612)%
\polyline(-5.5,4.711)(-5.5,4.81)\polyline(-5.5,4.909)(-5.5,5.008)\polyline(-5.5,5.107)(-5.5,5.207)%
\polyline(-5.5,5.306)(-5.5,5.405)\polyline(-5.5,5.504)(-5.5,5.603)\polyline(-5.5,5.702)(-5.5,5.802)%
\polyline(-5.5,5.901)(-5.5,6)%
%
\polyline(-6,-5.5)(-5.901,-5.5)\polyline(-5.802,-5.5)(-5.702,-5.5)\polyline(-5.603,-5.5)(-5.504,-5.5)%
\polyline(-5.405,-5.5)(-5.306,-5.5)\polyline(-5.207,-5.5)(-5.107,-5.5)\polyline(-5.008,-5.5)(-4.909,-5.5)%
\polyline(-4.81,-5.5)(-4.711,-5.5)\polyline(-4.612,-5.5)(-4.512,-5.5)\polyline(-4.413,-5.5)(-4.314,-5.5)%
\polyline(-4.215,-5.5)(-4.116,-5.5)\polyline(-4.017,-5.5)(-3.917,-5.5)\polyline(-3.818,-5.5)(-3.719,-5.5)%
\polyline(-3.62,-5.5)(-3.521,-5.5)\polyline(-3.421,-5.5)(-3.322,-5.5)\polyline(-3.223,-5.5)(-3.124,-5.5)%
\polyline(-3.025,-5.5)(-2.926,-5.5)\polyline(-2.826,-5.5)(-2.727,-5.5)\polyline(-2.628,-5.5)(-2.529,-5.5)%
\polyline(-2.43,-5.5)(-2.331,-5.5)\polyline(-2.231,-5.5)(-2.132,-5.5)\polyline(-2.033,-5.5)(-1.934,-5.5)%
\polyline(-1.835,-5.5)(-1.736,-5.5)\polyline(-1.636,-5.5)(-1.537,-5.5)\polyline(-1.438,-5.5)(-1.339,-5.5)%
\polyline(-1.24,-5.5)(-1.14,-5.5)\polyline(-1.041,-5.5)(-0.942,-5.5)\polyline(-0.843,-5.5)(-0.744,-5.5)%
\polyline(-0.645,-5.5)(-0.545,-5.5)\polyline(-0.446,-5.5)(-0.347,-5.5)\polyline(-0.248,-5.5)(-0.149,-5.5)%
\polyline(-0.05,-5.5)(0.05,-5.5)\polyline(0.149,-5.5)(0.248,-5.5)\polyline(0.347,-5.5)(0.446,-5.5)%
\polyline(0.545,-5.5)(0.645,-5.5)\polyline(0.744,-5.5)(0.843,-5.5)\polyline(0.942,-5.5)(1.041,-5.5)%
\polyline(1.14,-5.5)(1.24,-5.5)\polyline(1.339,-5.5)(1.438,-5.5)\polyline(1.537,-5.5)(1.636,-5.5)%
\polyline(1.736,-5.5)(1.835,-5.5)\polyline(1.934,-5.5)(2.033,-5.5)\polyline(2.132,-5.5)(2.231,-5.5)%
\polyline(2.331,-5.5)(2.43,-5.5)\polyline(2.529,-5.5)(2.628,-5.5)\polyline(2.727,-5.5)(2.826,-5.5)%
\polyline(2.926,-5.5)(3.025,-5.5)\polyline(3.124,-5.5)(3.223,-5.5)\polyline(3.322,-5.5)(3.421,-5.5)%
\polyline(3.521,-5.5)(3.62,-5.5)\polyline(3.719,-5.5)(3.818,-5.5)\polyline(3.917,-5.5)(4.017,-5.5)%
\polyline(4.116,-5.5)(4.215,-5.5)\polyline(4.314,-5.5)(4.413,-5.5)\polyline(4.512,-5.5)(4.612,-5.5)%
\polyline(4.711,-5.5)(4.81,-5.5)\polyline(4.909,-5.5)(5.008,-5.5)\polyline(5.107,-5.5)(5.207,-5.5)%
\polyline(5.306,-5.5)(5.405,-5.5)\polyline(5.504,-5.5)(5.603,-5.5)\polyline(5.702,-5.5)(5.802,-5.5)%
\polyline(5.901,-5.5)(6,-5.5)%
%
\polyline(-4.5,-6)(-4.5,-5.901)\polyline(-4.5,-5.802)(-4.5,-5.702)\polyline(-4.5,-5.603)(-4.5,-5.504)%
\polyline(-4.5,-5.405)(-4.5,-5.306)\polyline(-4.5,-5.207)(-4.5,-5.107)\polyline(-4.5,-5.008)(-4.5,-4.909)%
\polyline(-4.5,-4.81)(-4.5,-4.711)\polyline(-4.5,-4.612)(-4.5,-4.512)\polyline(-4.5,-4.413)(-4.5,-4.314)%
\polyline(-4.5,-4.215)(-4.5,-4.116)\polyline(-4.5,-4.017)(-4.5,-3.917)\polyline(-4.5,-3.818)(-4.5,-3.719)%
\polyline(-4.5,-3.62)(-4.5,-3.521)\polyline(-4.5,-3.421)(-4.5,-3.322)\polyline(-4.5,-3.223)(-4.5,-3.124)%
\polyline(-4.5,-3.025)(-4.5,-2.926)\polyline(-4.5,-2.826)(-4.5,-2.727)\polyline(-4.5,-2.628)(-4.5,-2.529)%
\polyline(-4.5,-2.43)(-4.5,-2.331)\polyline(-4.5,-2.231)(-4.5,-2.132)\polyline(-4.5,-2.033)(-4.5,-1.934)%
\polyline(-4.5,-1.835)(-4.5,-1.736)\polyline(-4.5,-1.636)(-4.5,-1.537)\polyline(-4.5,-1.438)(-4.5,-1.339)%
\polyline(-4.5,-1.24)(-4.5,-1.14)\polyline(-4.5,-1.041)(-4.5,-0.942)\polyline(-4.5,-0.843)(-4.5,-0.744)%
\polyline(-4.5,-0.645)(-4.5,-0.545)\polyline(-4.5,-0.446)(-4.5,-0.347)\polyline(-4.5,-0.248)(-4.5,-0.149)%
\polyline(-4.5,-0.05)(-4.5,0.05)\polyline(-4.5,0.149)(-4.5,0.248)\polyline(-4.5,0.347)(-4.5,0.446)%
\polyline(-4.5,0.545)(-4.5,0.645)\polyline(-4.5,0.744)(-4.5,0.843)\polyline(-4.5,0.942)(-4.5,1.041)%
\polyline(-4.5,1.14)(-4.5,1.24)\polyline(-4.5,1.339)(-4.5,1.438)\polyline(-4.5,1.537)(-4.5,1.636)%
\polyline(-4.5,1.736)(-4.5,1.835)\polyline(-4.5,1.934)(-4.5,2.033)\polyline(-4.5,2.132)(-4.5,2.231)%
\polyline(-4.5,2.331)(-4.5,2.43)\polyline(-4.5,2.529)(-4.5,2.628)\polyline(-4.5,2.727)(-4.5,2.826)%
\polyline(-4.5,2.926)(-4.5,3.025)\polyline(-4.5,3.124)(-4.5,3.223)\polyline(-4.5,3.322)(-4.5,3.421)%
\polyline(-4.5,3.521)(-4.5,3.62)\polyline(-4.5,3.719)(-4.5,3.818)\polyline(-4.5,3.917)(-4.5,4.017)%
\polyline(-4.5,4.116)(-4.5,4.215)\polyline(-4.5,4.314)(-4.5,4.413)\polyline(-4.5,4.512)(-4.5,4.612)%
\polyline(-4.5,4.711)(-4.5,4.81)\polyline(-4.5,4.909)(-4.5,5.008)\polyline(-4.5,5.107)(-4.5,5.207)%
\polyline(-4.5,5.306)(-4.5,5.405)\polyline(-4.5,5.504)(-4.5,5.603)\polyline(-4.5,5.702)(-4.5,5.802)%
\polyline(-4.5,5.901)(-4.5,6)%
%
\polyline(-6,-4.5)(-5.901,-4.5)\polyline(-5.802,-4.5)(-5.702,-4.5)\polyline(-5.603,-4.5)(-5.504,-4.5)%
\polyline(-5.405,-4.5)(-5.306,-4.5)\polyline(-5.207,-4.5)(-5.107,-4.5)\polyline(-5.008,-4.5)(-4.909,-4.5)%
\polyline(-4.81,-4.5)(-4.711,-4.5)\polyline(-4.612,-4.5)(-4.512,-4.5)\polyline(-4.413,-4.5)(-4.314,-4.5)%
\polyline(-4.215,-4.5)(-4.116,-4.5)\polyline(-4.017,-4.5)(-3.917,-4.5)\polyline(-3.818,-4.5)(-3.719,-4.5)%
\polyline(-3.62,-4.5)(-3.521,-4.5)\polyline(-3.421,-4.5)(-3.322,-4.5)\polyline(-3.223,-4.5)(-3.124,-4.5)%
\polyline(-3.025,-4.5)(-2.926,-4.5)\polyline(-2.826,-4.5)(-2.727,-4.5)\polyline(-2.628,-4.5)(-2.529,-4.5)%
\polyline(-2.43,-4.5)(-2.331,-4.5)\polyline(-2.231,-4.5)(-2.132,-4.5)\polyline(-2.033,-4.5)(-1.934,-4.5)%
\polyline(-1.835,-4.5)(-1.736,-4.5)\polyline(-1.636,-4.5)(-1.537,-4.5)\polyline(-1.438,-4.5)(-1.339,-4.5)%
\polyline(-1.24,-4.5)(-1.14,-4.5)\polyline(-1.041,-4.5)(-0.942,-4.5)\polyline(-0.843,-4.5)(-0.744,-4.5)%
\polyline(-0.645,-4.5)(-0.545,-4.5)\polyline(-0.446,-4.5)(-0.347,-4.5)\polyline(-0.248,-4.5)(-0.149,-4.5)%
\polyline(-0.05,-4.5)(0.05,-4.5)\polyline(0.149,-4.5)(0.248,-4.5)\polyline(0.347,-4.5)(0.446,-4.5)%
\polyline(0.545,-4.5)(0.645,-4.5)\polyline(0.744,-4.5)(0.843,-4.5)\polyline(0.942,-4.5)(1.041,-4.5)%
\polyline(1.14,-4.5)(1.24,-4.5)\polyline(1.339,-4.5)(1.438,-4.5)\polyline(1.537,-4.5)(1.636,-4.5)%
\polyline(1.736,-4.5)(1.835,-4.5)\polyline(1.934,-4.5)(2.033,-4.5)\polyline(2.132,-4.5)(2.231,-4.5)%
\polyline(2.331,-4.5)(2.43,-4.5)\polyline(2.529,-4.5)(2.628,-4.5)\polyline(2.727,-4.5)(2.826,-4.5)%
\polyline(2.926,-4.5)(3.025,-4.5)\polyline(3.124,-4.5)(3.223,-4.5)\polyline(3.322,-4.5)(3.421,-4.5)%
\polyline(3.521,-4.5)(3.62,-4.5)\polyline(3.719,-4.5)(3.818,-4.5)\polyline(3.917,-4.5)(4.017,-4.5)%
\polyline(4.116,-4.5)(4.215,-4.5)\polyline(4.314,-4.5)(4.413,-4.5)\polyline(4.512,-4.5)(4.612,-4.5)%
\polyline(4.711,-4.5)(4.81,-4.5)\polyline(4.909,-4.5)(5.008,-4.5)\polyline(5.107,-4.5)(5.207,-4.5)%
\polyline(5.306,-4.5)(5.405,-4.5)\polyline(5.504,-4.5)(5.603,-4.5)\polyline(5.702,-4.5)(5.802,-4.5)%
\polyline(5.901,-4.5)(6,-4.5)%
%
\polyline(-3.5,-6)(-3.5,-5.901)\polyline(-3.5,-5.802)(-3.5,-5.702)\polyline(-3.5,-5.603)(-3.5,-5.504)%
\polyline(-3.5,-5.405)(-3.5,-5.306)\polyline(-3.5,-5.207)(-3.5,-5.107)\polyline(-3.5,-5.008)(-3.5,-4.909)%
\polyline(-3.5,-4.81)(-3.5,-4.711)\polyline(-3.5,-4.612)(-3.5,-4.512)\polyline(-3.5,-4.413)(-3.5,-4.314)%
\polyline(-3.5,-4.215)(-3.5,-4.116)\polyline(-3.5,-4.017)(-3.5,-3.917)\polyline(-3.5,-3.818)(-3.5,-3.719)%
\polyline(-3.5,-3.62)(-3.5,-3.521)\polyline(-3.5,-3.421)(-3.5,-3.322)\polyline(-3.5,-3.223)(-3.5,-3.124)%
\polyline(-3.5,-3.025)(-3.5,-2.926)\polyline(-3.5,-2.826)(-3.5,-2.727)\polyline(-3.5,-2.628)(-3.5,-2.529)%
\polyline(-3.5,-2.43)(-3.5,-2.331)\polyline(-3.5,-2.231)(-3.5,-2.132)\polyline(-3.5,-2.033)(-3.5,-1.934)%
\polyline(-3.5,-1.835)(-3.5,-1.736)\polyline(-3.5,-1.636)(-3.5,-1.537)\polyline(-3.5,-1.438)(-3.5,-1.339)%
\polyline(-3.5,-1.24)(-3.5,-1.14)\polyline(-3.5,-1.041)(-3.5,-0.942)\polyline(-3.5,-0.843)(-3.5,-0.744)%
\polyline(-3.5,-0.645)(-3.5,-0.545)\polyline(-3.5,-0.446)(-3.5,-0.347)\polyline(-3.5,-0.248)(-3.5,-0.149)%
\polyline(-3.5,-0.05)(-3.5,0.05)\polyline(-3.5,0.149)(-3.5,0.248)\polyline(-3.5,0.347)(-3.5,0.446)%
\polyline(-3.5,0.545)(-3.5,0.645)\polyline(-3.5,0.744)(-3.5,0.843)\polyline(-3.5,0.942)(-3.5,1.041)%
\polyline(-3.5,1.14)(-3.5,1.24)\polyline(-3.5,1.339)(-3.5,1.438)\polyline(-3.5,1.537)(-3.5,1.636)%
\polyline(-3.5,1.736)(-3.5,1.835)\polyline(-3.5,1.934)(-3.5,2.033)\polyline(-3.5,2.132)(-3.5,2.231)%
\polyline(-3.5,2.331)(-3.5,2.43)\polyline(-3.5,2.529)(-3.5,2.628)\polyline(-3.5,2.727)(-3.5,2.826)%
\polyline(-3.5,2.926)(-3.5,3.025)\polyline(-3.5,3.124)(-3.5,3.223)\polyline(-3.5,3.322)(-3.5,3.421)%
\polyline(-3.5,3.521)(-3.5,3.62)\polyline(-3.5,3.719)(-3.5,3.818)\polyline(-3.5,3.917)(-3.5,4.017)%
\polyline(-3.5,4.116)(-3.5,4.215)\polyline(-3.5,4.314)(-3.5,4.413)\polyline(-3.5,4.512)(-3.5,4.612)%
\polyline(-3.5,4.711)(-3.5,4.81)\polyline(-3.5,4.909)(-3.5,5.008)\polyline(-3.5,5.107)(-3.5,5.207)%
\polyline(-3.5,5.306)(-3.5,5.405)\polyline(-3.5,5.504)(-3.5,5.603)\polyline(-3.5,5.702)(-3.5,5.802)%
\polyline(-3.5,5.901)(-3.5,6)%
%
\polyline(-6,-3.5)(-5.901,-3.5)\polyline(-5.802,-3.5)(-5.702,-3.5)\polyline(-5.603,-3.5)(-5.504,-3.5)%
\polyline(-5.405,-3.5)(-5.306,-3.5)\polyline(-5.207,-3.5)(-5.107,-3.5)\polyline(-5.008,-3.5)(-4.909,-3.5)%
\polyline(-4.81,-3.5)(-4.711,-3.5)\polyline(-4.612,-3.5)(-4.512,-3.5)\polyline(-4.413,-3.5)(-4.314,-3.5)%
\polyline(-4.215,-3.5)(-4.116,-3.5)\polyline(-4.017,-3.5)(-3.917,-3.5)\polyline(-3.818,-3.5)(-3.719,-3.5)%
\polyline(-3.62,-3.5)(-3.521,-3.5)\polyline(-3.421,-3.5)(-3.322,-3.5)\polyline(-3.223,-3.5)(-3.124,-3.5)%
\polyline(-3.025,-3.5)(-2.926,-3.5)\polyline(-2.826,-3.5)(-2.727,-3.5)\polyline(-2.628,-3.5)(-2.529,-3.5)%
\polyline(-2.43,-3.5)(-2.331,-3.5)\polyline(-2.231,-3.5)(-2.132,-3.5)\polyline(-2.033,-3.5)(-1.934,-3.5)%
\polyline(-1.835,-3.5)(-1.736,-3.5)\polyline(-1.636,-3.5)(-1.537,-3.5)\polyline(-1.438,-3.5)(-1.339,-3.5)%
\polyline(-1.24,-3.5)(-1.14,-3.5)\polyline(-1.041,-3.5)(-0.942,-3.5)\polyline(-0.843,-3.5)(-0.744,-3.5)%
\polyline(-0.645,-3.5)(-0.545,-3.5)\polyline(-0.446,-3.5)(-0.347,-3.5)\polyline(-0.248,-3.5)(-0.149,-3.5)%
\polyline(-0.05,-3.5)(0.05,-3.5)\polyline(0.149,-3.5)(0.248,-3.5)\polyline(0.347,-3.5)(0.446,-3.5)%
\polyline(0.545,-3.5)(0.645,-3.5)\polyline(0.744,-3.5)(0.843,-3.5)\polyline(0.942,-3.5)(1.041,-3.5)%
\polyline(1.14,-3.5)(1.24,-3.5)\polyline(1.339,-3.5)(1.438,-3.5)\polyline(1.537,-3.5)(1.636,-3.5)%
\polyline(1.736,-3.5)(1.835,-3.5)\polyline(1.934,-3.5)(2.033,-3.5)\polyline(2.132,-3.5)(2.231,-3.5)%
\polyline(2.331,-3.5)(2.43,-3.5)\polyline(2.529,-3.5)(2.628,-3.5)\polyline(2.727,-3.5)(2.826,-3.5)%
\polyline(2.926,-3.5)(3.025,-3.5)\polyline(3.124,-3.5)(3.223,-3.5)\polyline(3.322,-3.5)(3.421,-3.5)%
\polyline(3.521,-3.5)(3.62,-3.5)\polyline(3.719,-3.5)(3.818,-3.5)\polyline(3.917,-3.5)(4.017,-3.5)%
\polyline(4.116,-3.5)(4.215,-3.5)\polyline(4.314,-3.5)(4.413,-3.5)\polyline(4.512,-3.5)(4.612,-3.5)%
\polyline(4.711,-3.5)(4.81,-3.5)\polyline(4.909,-3.5)(5.008,-3.5)\polyline(5.107,-3.5)(5.207,-3.5)%
\polyline(5.306,-3.5)(5.405,-3.5)\polyline(5.504,-3.5)(5.603,-3.5)\polyline(5.702,-3.5)(5.802,-3.5)%
\polyline(5.901,-3.5)(6,-3.5)%
%
\polyline(-2.5,-6)(-2.5,-5.901)\polyline(-2.5,-5.802)(-2.5,-5.702)\polyline(-2.5,-5.603)(-2.5,-5.504)%
\polyline(-2.5,-5.405)(-2.5,-5.306)\polyline(-2.5,-5.207)(-2.5,-5.107)\polyline(-2.5,-5.008)(-2.5,-4.909)%
\polyline(-2.5,-4.81)(-2.5,-4.711)\polyline(-2.5,-4.612)(-2.5,-4.512)\polyline(-2.5,-4.413)(-2.5,-4.314)%
\polyline(-2.5,-4.215)(-2.5,-4.116)\polyline(-2.5,-4.017)(-2.5,-3.917)\polyline(-2.5,-3.818)(-2.5,-3.719)%
\polyline(-2.5,-3.62)(-2.5,-3.521)\polyline(-2.5,-3.421)(-2.5,-3.322)\polyline(-2.5,-3.223)(-2.5,-3.124)%
\polyline(-2.5,-3.025)(-2.5,-2.926)\polyline(-2.5,-2.826)(-2.5,-2.727)\polyline(-2.5,-2.628)(-2.5,-2.529)%
\polyline(-2.5,-2.43)(-2.5,-2.331)\polyline(-2.5,-2.231)(-2.5,-2.132)\polyline(-2.5,-2.033)(-2.5,-1.934)%
\polyline(-2.5,-1.835)(-2.5,-1.736)\polyline(-2.5,-1.636)(-2.5,-1.537)\polyline(-2.5,-1.438)(-2.5,-1.339)%
\polyline(-2.5,-1.24)(-2.5,-1.14)\polyline(-2.5,-1.041)(-2.5,-0.942)\polyline(-2.5,-0.843)(-2.5,-0.744)%
\polyline(-2.5,-0.645)(-2.5,-0.545)\polyline(-2.5,-0.446)(-2.5,-0.347)\polyline(-2.5,-0.248)(-2.5,-0.149)%
\polyline(-2.5,-0.05)(-2.5,0.05)\polyline(-2.5,0.149)(-2.5,0.248)\polyline(-2.5,0.347)(-2.5,0.446)%
\polyline(-2.5,0.545)(-2.5,0.645)\polyline(-2.5,0.744)(-2.5,0.843)\polyline(-2.5,0.942)(-2.5,1.041)%
\polyline(-2.5,1.14)(-2.5,1.24)\polyline(-2.5,1.339)(-2.5,1.438)\polyline(-2.5,1.537)(-2.5,1.636)%
\polyline(-2.5,1.736)(-2.5,1.835)\polyline(-2.5,1.934)(-2.5,2.033)\polyline(-2.5,2.132)(-2.5,2.231)%
\polyline(-2.5,2.331)(-2.5,2.43)\polyline(-2.5,2.529)(-2.5,2.628)\polyline(-2.5,2.727)(-2.5,2.826)%
\polyline(-2.5,2.926)(-2.5,3.025)\polyline(-2.5,3.124)(-2.5,3.223)\polyline(-2.5,3.322)(-2.5,3.421)%
\polyline(-2.5,3.521)(-2.5,3.62)\polyline(-2.5,3.719)(-2.5,3.818)\polyline(-2.5,3.917)(-2.5,4.017)%
\polyline(-2.5,4.116)(-2.5,4.215)\polyline(-2.5,4.314)(-2.5,4.413)\polyline(-2.5,4.512)(-2.5,4.612)%
\polyline(-2.5,4.711)(-2.5,4.81)\polyline(-2.5,4.909)(-2.5,5.008)\polyline(-2.5,5.107)(-2.5,5.207)%
\polyline(-2.5,5.306)(-2.5,5.405)\polyline(-2.5,5.504)(-2.5,5.603)\polyline(-2.5,5.702)(-2.5,5.802)%
\polyline(-2.5,5.901)(-2.5,6)%
%
\polyline(-6,-2.5)(-5.901,-2.5)\polyline(-5.802,-2.5)(-5.702,-2.5)\polyline(-5.603,-2.5)(-5.504,-2.5)%
\polyline(-5.405,-2.5)(-5.306,-2.5)\polyline(-5.207,-2.5)(-5.107,-2.5)\polyline(-5.008,-2.5)(-4.909,-2.5)%
\polyline(-4.81,-2.5)(-4.711,-2.5)\polyline(-4.612,-2.5)(-4.512,-2.5)\polyline(-4.413,-2.5)(-4.314,-2.5)%
\polyline(-4.215,-2.5)(-4.116,-2.5)\polyline(-4.017,-2.5)(-3.917,-2.5)\polyline(-3.818,-2.5)(-3.719,-2.5)%
\polyline(-3.62,-2.5)(-3.521,-2.5)\polyline(-3.421,-2.5)(-3.322,-2.5)\polyline(-3.223,-2.5)(-3.124,-2.5)%
\polyline(-3.025,-2.5)(-2.926,-2.5)\polyline(-2.826,-2.5)(-2.727,-2.5)\polyline(-2.628,-2.5)(-2.529,-2.5)%
\polyline(-2.43,-2.5)(-2.331,-2.5)\polyline(-2.231,-2.5)(-2.132,-2.5)\polyline(-2.033,-2.5)(-1.934,-2.5)%
\polyline(-1.835,-2.5)(-1.736,-2.5)\polyline(-1.636,-2.5)(-1.537,-2.5)\polyline(-1.438,-2.5)(-1.339,-2.5)%
\polyline(-1.24,-2.5)(-1.14,-2.5)\polyline(-1.041,-2.5)(-0.942,-2.5)\polyline(-0.843,-2.5)(-0.744,-2.5)%
\polyline(-0.645,-2.5)(-0.545,-2.5)\polyline(-0.446,-2.5)(-0.347,-2.5)\polyline(-0.248,-2.5)(-0.149,-2.5)%
\polyline(-0.05,-2.5)(0.05,-2.5)\polyline(0.149,-2.5)(0.248,-2.5)\polyline(0.347,-2.5)(0.446,-2.5)%
\polyline(0.545,-2.5)(0.645,-2.5)\polyline(0.744,-2.5)(0.843,-2.5)\polyline(0.942,-2.5)(1.041,-2.5)%
\polyline(1.14,-2.5)(1.24,-2.5)\polyline(1.339,-2.5)(1.438,-2.5)\polyline(1.537,-2.5)(1.636,-2.5)%
\polyline(1.736,-2.5)(1.835,-2.5)\polyline(1.934,-2.5)(2.033,-2.5)\polyline(2.132,-2.5)(2.231,-2.5)%
\polyline(2.331,-2.5)(2.43,-2.5)\polyline(2.529,-2.5)(2.628,-2.5)\polyline(2.727,-2.5)(2.826,-2.5)%
\polyline(2.926,-2.5)(3.025,-2.5)\polyline(3.124,-2.5)(3.223,-2.5)\polyline(3.322,-2.5)(3.421,-2.5)%
\polyline(3.521,-2.5)(3.62,-2.5)\polyline(3.719,-2.5)(3.818,-2.5)\polyline(3.917,-2.5)(4.017,-2.5)%
\polyline(4.116,-2.5)(4.215,-2.5)\polyline(4.314,-2.5)(4.413,-2.5)\polyline(4.512,-2.5)(4.612,-2.5)%
\polyline(4.711,-2.5)(4.81,-2.5)\polyline(4.909,-2.5)(5.008,-2.5)\polyline(5.107,-2.5)(5.207,-2.5)%
\polyline(5.306,-2.5)(5.405,-2.5)\polyline(5.504,-2.5)(5.603,-2.5)\polyline(5.702,-2.5)(5.802,-2.5)%
\polyline(5.901,-2.5)(6,-2.5)%
%
\polyline(-1.5,-6)(-1.5,-5.901)\polyline(-1.5,-5.802)(-1.5,-5.702)\polyline(-1.5,-5.603)(-1.5,-5.504)%
\polyline(-1.5,-5.405)(-1.5,-5.306)\polyline(-1.5,-5.207)(-1.5,-5.107)\polyline(-1.5,-5.008)(-1.5,-4.909)%
\polyline(-1.5,-4.81)(-1.5,-4.711)\polyline(-1.5,-4.612)(-1.5,-4.512)\polyline(-1.5,-4.413)(-1.5,-4.314)%
\polyline(-1.5,-4.215)(-1.5,-4.116)\polyline(-1.5,-4.017)(-1.5,-3.917)\polyline(-1.5,-3.818)(-1.5,-3.719)%
\polyline(-1.5,-3.62)(-1.5,-3.521)\polyline(-1.5,-3.421)(-1.5,-3.322)\polyline(-1.5,-3.223)(-1.5,-3.124)%
\polyline(-1.5,-3.025)(-1.5,-2.926)\polyline(-1.5,-2.826)(-1.5,-2.727)\polyline(-1.5,-2.628)(-1.5,-2.529)%
\polyline(-1.5,-2.43)(-1.5,-2.331)\polyline(-1.5,-2.231)(-1.5,-2.132)\polyline(-1.5,-2.033)(-1.5,-1.934)%
\polyline(-1.5,-1.835)(-1.5,-1.736)\polyline(-1.5,-1.636)(-1.5,-1.537)\polyline(-1.5,-1.438)(-1.5,-1.339)%
\polyline(-1.5,-1.24)(-1.5,-1.14)\polyline(-1.5,-1.041)(-1.5,-0.942)\polyline(-1.5,-0.843)(-1.5,-0.744)%
\polyline(-1.5,-0.645)(-1.5,-0.545)\polyline(-1.5,-0.446)(-1.5,-0.347)\polyline(-1.5,-0.248)(-1.5,-0.149)%
\polyline(-1.5,-0.05)(-1.5,0.05)\polyline(-1.5,0.149)(-1.5,0.248)\polyline(-1.5,0.347)(-1.5,0.446)%
\polyline(-1.5,0.545)(-1.5,0.645)\polyline(-1.5,0.744)(-1.5,0.843)\polyline(-1.5,0.942)(-1.5,1.041)%
\polyline(-1.5,1.14)(-1.5,1.24)\polyline(-1.5,1.339)(-1.5,1.438)\polyline(-1.5,1.537)(-1.5,1.636)%
\polyline(-1.5,1.736)(-1.5,1.835)\polyline(-1.5,1.934)(-1.5,2.033)\polyline(-1.5,2.132)(-1.5,2.231)%
\polyline(-1.5,2.331)(-1.5,2.43)\polyline(-1.5,2.529)(-1.5,2.628)\polyline(-1.5,2.727)(-1.5,2.826)%
\polyline(-1.5,2.926)(-1.5,3.025)\polyline(-1.5,3.124)(-1.5,3.223)\polyline(-1.5,3.322)(-1.5,3.421)%
\polyline(-1.5,3.521)(-1.5,3.62)\polyline(-1.5,3.719)(-1.5,3.818)\polyline(-1.5,3.917)(-1.5,4.017)%
\polyline(-1.5,4.116)(-1.5,4.215)\polyline(-1.5,4.314)(-1.5,4.413)\polyline(-1.5,4.512)(-1.5,4.612)%
\polyline(-1.5,4.711)(-1.5,4.81)\polyline(-1.5,4.909)(-1.5,5.008)\polyline(-1.5,5.107)(-1.5,5.207)%
\polyline(-1.5,5.306)(-1.5,5.405)\polyline(-1.5,5.504)(-1.5,5.603)\polyline(-1.5,5.702)(-1.5,5.802)%
\polyline(-1.5,5.901)(-1.5,6)%
%
\polyline(-6,-1.5)(-5.901,-1.5)\polyline(-5.802,-1.5)(-5.702,-1.5)\polyline(-5.603,-1.5)(-5.504,-1.5)%
\polyline(-5.405,-1.5)(-5.306,-1.5)\polyline(-5.207,-1.5)(-5.107,-1.5)\polyline(-5.008,-1.5)(-4.909,-1.5)%
\polyline(-4.81,-1.5)(-4.711,-1.5)\polyline(-4.612,-1.5)(-4.512,-1.5)\polyline(-4.413,-1.5)(-4.314,-1.5)%
\polyline(-4.215,-1.5)(-4.116,-1.5)\polyline(-4.017,-1.5)(-3.917,-1.5)\polyline(-3.818,-1.5)(-3.719,-1.5)%
\polyline(-3.62,-1.5)(-3.521,-1.5)\polyline(-3.421,-1.5)(-3.322,-1.5)\polyline(-3.223,-1.5)(-3.124,-1.5)%
\polyline(-3.025,-1.5)(-2.926,-1.5)\polyline(-2.826,-1.5)(-2.727,-1.5)\polyline(-2.628,-1.5)(-2.529,-1.5)%
\polyline(-2.43,-1.5)(-2.331,-1.5)\polyline(-2.231,-1.5)(-2.132,-1.5)\polyline(-2.033,-1.5)(-1.934,-1.5)%
\polyline(-1.835,-1.5)(-1.736,-1.5)\polyline(-1.636,-1.5)(-1.537,-1.5)\polyline(-1.438,-1.5)(-1.339,-1.5)%
\polyline(-1.24,-1.5)(-1.14,-1.5)\polyline(-1.041,-1.5)(-0.942,-1.5)\polyline(-0.843,-1.5)(-0.744,-1.5)%
\polyline(-0.645,-1.5)(-0.545,-1.5)\polyline(-0.446,-1.5)(-0.347,-1.5)\polyline(-0.248,-1.5)(-0.149,-1.5)%
\polyline(-0.05,-1.5)(0.05,-1.5)\polyline(0.149,-1.5)(0.248,-1.5)\polyline(0.347,-1.5)(0.446,-1.5)%
\polyline(0.545,-1.5)(0.645,-1.5)\polyline(0.744,-1.5)(0.843,-1.5)\polyline(0.942,-1.5)(1.041,-1.5)%
\polyline(1.14,-1.5)(1.24,-1.5)\polyline(1.339,-1.5)(1.438,-1.5)\polyline(1.537,-1.5)(1.636,-1.5)%
\polyline(1.736,-1.5)(1.835,-1.5)\polyline(1.934,-1.5)(2.033,-1.5)\polyline(2.132,-1.5)(2.231,-1.5)%
\polyline(2.331,-1.5)(2.43,-1.5)\polyline(2.529,-1.5)(2.628,-1.5)\polyline(2.727,-1.5)(2.826,-1.5)%
\polyline(2.926,-1.5)(3.025,-1.5)\polyline(3.124,-1.5)(3.223,-1.5)\polyline(3.322,-1.5)(3.421,-1.5)%
\polyline(3.521,-1.5)(3.62,-1.5)\polyline(3.719,-1.5)(3.818,-1.5)\polyline(3.917,-1.5)(4.017,-1.5)%
\polyline(4.116,-1.5)(4.215,-1.5)\polyline(4.314,-1.5)(4.413,-1.5)\polyline(4.512,-1.5)(4.612,-1.5)%
\polyline(4.711,-1.5)(4.81,-1.5)\polyline(4.909,-1.5)(5.008,-1.5)\polyline(5.107,-1.5)(5.207,-1.5)%
\polyline(5.306,-1.5)(5.405,-1.5)\polyline(5.504,-1.5)(5.603,-1.5)\polyline(5.702,-1.5)(5.802,-1.5)%
\polyline(5.901,-1.5)(6,-1.5)%
%
\polyline(-0.5,-6)(-0.5,-5.901)\polyline(-0.5,-5.802)(-0.5,-5.702)\polyline(-0.5,-5.603)(-0.5,-5.504)%
\polyline(-0.5,-5.405)(-0.5,-5.306)\polyline(-0.5,-5.207)(-0.5,-5.107)\polyline(-0.5,-5.008)(-0.5,-4.909)%
\polyline(-0.5,-4.81)(-0.5,-4.711)\polyline(-0.5,-4.612)(-0.5,-4.512)\polyline(-0.5,-4.413)(-0.5,-4.314)%
\polyline(-0.5,-4.215)(-0.5,-4.116)\polyline(-0.5,-4.017)(-0.5,-3.917)\polyline(-0.5,-3.818)(-0.5,-3.719)%
\polyline(-0.5,-3.62)(-0.5,-3.521)\polyline(-0.5,-3.421)(-0.5,-3.322)\polyline(-0.5,-3.223)(-0.5,-3.124)%
\polyline(-0.5,-3.025)(-0.5,-2.926)\polyline(-0.5,-2.826)(-0.5,-2.727)\polyline(-0.5,-2.628)(-0.5,-2.529)%
\polyline(-0.5,-2.43)(-0.5,-2.331)\polyline(-0.5,-2.231)(-0.5,-2.132)\polyline(-0.5,-2.033)(-0.5,-1.934)%
\polyline(-0.5,-1.835)(-0.5,-1.736)\polyline(-0.5,-1.636)(-0.5,-1.537)\polyline(-0.5,-1.438)(-0.5,-1.339)%
\polyline(-0.5,-1.24)(-0.5,-1.14)\polyline(-0.5,-1.041)(-0.5,-0.942)\polyline(-0.5,-0.843)(-0.5,-0.744)%
\polyline(-0.5,-0.645)(-0.5,-0.545)\polyline(-0.5,-0.446)(-0.5,-0.347)\polyline(-0.5,-0.248)(-0.5,-0.149)%
\polyline(-0.5,-0.05)(-0.5,0.05)\polyline(-0.5,0.149)(-0.5,0.248)\polyline(-0.5,0.347)(-0.5,0.446)%
\polyline(-0.5,0.545)(-0.5,0.645)\polyline(-0.5,0.744)(-0.5,0.843)\polyline(-0.5,0.942)(-0.5,1.041)%
\polyline(-0.5,1.14)(-0.5,1.24)\polyline(-0.5,1.339)(-0.5,1.438)\polyline(-0.5,1.537)(-0.5,1.636)%
\polyline(-0.5,1.736)(-0.5,1.835)\polyline(-0.5,1.934)(-0.5,2.033)\polyline(-0.5,2.132)(-0.5,2.231)%
\polyline(-0.5,2.331)(-0.5,2.43)\polyline(-0.5,2.529)(-0.5,2.628)\polyline(-0.5,2.727)(-0.5,2.826)%
\polyline(-0.5,2.926)(-0.5,3.025)\polyline(-0.5,3.124)(-0.5,3.223)\polyline(-0.5,3.322)(-0.5,3.421)%
\polyline(-0.5,3.521)(-0.5,3.62)\polyline(-0.5,3.719)(-0.5,3.818)\polyline(-0.5,3.917)(-0.5,4.017)%
\polyline(-0.5,4.116)(-0.5,4.215)\polyline(-0.5,4.314)(-0.5,4.413)\polyline(-0.5,4.512)(-0.5,4.612)%
\polyline(-0.5,4.711)(-0.5,4.81)\polyline(-0.5,4.909)(-0.5,5.008)\polyline(-0.5,5.107)(-0.5,5.207)%
\polyline(-0.5,5.306)(-0.5,5.405)\polyline(-0.5,5.504)(-0.5,5.603)\polyline(-0.5,5.702)(-0.5,5.802)%
\polyline(-0.5,5.901)(-0.5,6)%
%
\polyline(-6,-0.5)(-5.901,-0.5)\polyline(-5.802,-0.5)(-5.702,-0.5)\polyline(-5.603,-0.5)(-5.504,-0.5)%
\polyline(-5.405,-0.5)(-5.306,-0.5)\polyline(-5.207,-0.5)(-5.107,-0.5)\polyline(-5.008,-0.5)(-4.909,-0.5)%
\polyline(-4.81,-0.5)(-4.711,-0.5)\polyline(-4.612,-0.5)(-4.512,-0.5)\polyline(-4.413,-0.5)(-4.314,-0.5)%
\polyline(-4.215,-0.5)(-4.116,-0.5)\polyline(-4.017,-0.5)(-3.917,-0.5)\polyline(-3.818,-0.5)(-3.719,-0.5)%
\polyline(-3.62,-0.5)(-3.521,-0.5)\polyline(-3.421,-0.5)(-3.322,-0.5)\polyline(-3.223,-0.5)(-3.124,-0.5)%
\polyline(-3.025,-0.5)(-2.926,-0.5)\polyline(-2.826,-0.5)(-2.727,-0.5)\polyline(-2.628,-0.5)(-2.529,-0.5)%
\polyline(-2.43,-0.5)(-2.331,-0.5)\polyline(-2.231,-0.5)(-2.132,-0.5)\polyline(-2.033,-0.5)(-1.934,-0.5)%
\polyline(-1.835,-0.5)(-1.736,-0.5)\polyline(-1.636,-0.5)(-1.537,-0.5)\polyline(-1.438,-0.5)(-1.339,-0.5)%
\polyline(-1.24,-0.5)(-1.14,-0.5)\polyline(-1.041,-0.5)(-0.942,-0.5)\polyline(-0.843,-0.5)(-0.744,-0.5)%
\polyline(-0.645,-0.5)(-0.545,-0.5)\polyline(-0.446,-0.5)(-0.347,-0.5)\polyline(-0.248,-0.5)(-0.149,-0.5)%
\polyline(-0.05,-0.5)(0.05,-0.5)\polyline(0.149,-0.5)(0.248,-0.5)\polyline(0.347,-0.5)(0.446,-0.5)%
\polyline(0.545,-0.5)(0.645,-0.5)\polyline(0.744,-0.5)(0.843,-0.5)\polyline(0.942,-0.5)(1.041,-0.5)%
\polyline(1.14,-0.5)(1.24,-0.5)\polyline(1.339,-0.5)(1.438,-0.5)\polyline(1.537,-0.5)(1.636,-0.5)%
\polyline(1.736,-0.5)(1.835,-0.5)\polyline(1.934,-0.5)(2.033,-0.5)\polyline(2.132,-0.5)(2.231,-0.5)%
\polyline(2.331,-0.5)(2.43,-0.5)\polyline(2.529,-0.5)(2.628,-0.5)\polyline(2.727,-0.5)(2.826,-0.5)%
\polyline(2.926,-0.5)(3.025,-0.5)\polyline(3.124,-0.5)(3.223,-0.5)\polyline(3.322,-0.5)(3.421,-0.5)%
\polyline(3.521,-0.5)(3.62,-0.5)\polyline(3.719,-0.5)(3.818,-0.5)\polyline(3.917,-0.5)(4.017,-0.5)%
\polyline(4.116,-0.5)(4.215,-0.5)\polyline(4.314,-0.5)(4.413,-0.5)\polyline(4.512,-0.5)(4.612,-0.5)%
\polyline(4.711,-0.5)(4.81,-0.5)\polyline(4.909,-0.5)(5.008,-0.5)\polyline(5.107,-0.5)(5.207,-0.5)%
\polyline(5.306,-0.5)(5.405,-0.5)\polyline(5.504,-0.5)(5.603,-0.5)\polyline(5.702,-0.5)(5.802,-0.5)%
\polyline(5.901,-0.5)(6,-0.5)%
%
\polyline(0.5,-6)(0.5,-5.901)\polyline(0.5,-5.802)(0.5,-5.702)\polyline(0.5,-5.603)(0.5,-5.504)%
\polyline(0.5,-5.405)(0.5,-5.306)\polyline(0.5,-5.207)(0.5,-5.107)\polyline(0.5,-5.008)(0.5,-4.909)%
\polyline(0.5,-4.81)(0.5,-4.711)\polyline(0.5,-4.612)(0.5,-4.512)\polyline(0.5,-4.413)(0.5,-4.314)%
\polyline(0.5,-4.215)(0.5,-4.116)\polyline(0.5,-4.017)(0.5,-3.917)\polyline(0.5,-3.818)(0.5,-3.719)%
\polyline(0.5,-3.62)(0.5,-3.521)\polyline(0.5,-3.421)(0.5,-3.322)\polyline(0.5,-3.223)(0.5,-3.124)%
\polyline(0.5,-3.025)(0.5,-2.926)\polyline(0.5,-2.826)(0.5,-2.727)\polyline(0.5,-2.628)(0.5,-2.529)%
\polyline(0.5,-2.43)(0.5,-2.331)\polyline(0.5,-2.231)(0.5,-2.132)\polyline(0.5,-2.033)(0.5,-1.934)%
\polyline(0.5,-1.835)(0.5,-1.736)\polyline(0.5,-1.636)(0.5,-1.537)\polyline(0.5,-1.438)(0.5,-1.339)%
\polyline(0.5,-1.24)(0.5,-1.14)\polyline(0.5,-1.041)(0.5,-0.942)\polyline(0.5,-0.843)(0.5,-0.744)%
\polyline(0.5,-0.645)(0.5,-0.545)\polyline(0.5,-0.446)(0.5,-0.347)\polyline(0.5,-0.248)(0.5,-0.149)%
\polyline(0.5,-0.05)(0.5,0.05)\polyline(0.5,0.149)(0.5,0.248)\polyline(0.5,0.347)(0.5,0.446)%
\polyline(0.5,0.545)(0.5,0.645)\polyline(0.5,0.744)(0.5,0.843)\polyline(0.5,0.942)(0.5,1.041)%
\polyline(0.5,1.14)(0.5,1.24)\polyline(0.5,1.339)(0.5,1.438)\polyline(0.5,1.537)(0.5,1.636)%
\polyline(0.5,1.736)(0.5,1.835)\polyline(0.5,1.934)(0.5,2.033)\polyline(0.5,2.132)(0.5,2.231)%
\polyline(0.5,2.331)(0.5,2.43)\polyline(0.5,2.529)(0.5,2.628)\polyline(0.5,2.727)(0.5,2.826)%
\polyline(0.5,2.926)(0.5,3.025)\polyline(0.5,3.124)(0.5,3.223)\polyline(0.5,3.322)(0.5,3.421)%
\polyline(0.5,3.521)(0.5,3.62)\polyline(0.5,3.719)(0.5,3.818)\polyline(0.5,3.917)(0.5,4.017)%
\polyline(0.5,4.116)(0.5,4.215)\polyline(0.5,4.314)(0.5,4.413)\polyline(0.5,4.512)(0.5,4.612)%
\polyline(0.5,4.711)(0.5,4.81)\polyline(0.5,4.909)(0.5,5.008)\polyline(0.5,5.107)(0.5,5.207)%
\polyline(0.5,5.306)(0.5,5.405)\polyline(0.5,5.504)(0.5,5.603)\polyline(0.5,5.702)(0.5,5.802)%
\polyline(0.5,5.901)(0.5,6)%
%
\polyline(-6,0.5)(-5.901,0.5)\polyline(-5.802,0.5)(-5.702,0.5)\polyline(-5.603,0.5)(-5.504,0.5)%
\polyline(-5.405,0.5)(-5.306,0.5)\polyline(-5.207,0.5)(-5.107,0.5)\polyline(-5.008,0.5)(-4.909,0.5)%
\polyline(-4.81,0.5)(-4.711,0.5)\polyline(-4.612,0.5)(-4.512,0.5)\polyline(-4.413,0.5)(-4.314,0.5)%
\polyline(-4.215,0.5)(-4.116,0.5)\polyline(-4.017,0.5)(-3.917,0.5)\polyline(-3.818,0.5)(-3.719,0.5)%
\polyline(-3.62,0.5)(-3.521,0.5)\polyline(-3.421,0.5)(-3.322,0.5)\polyline(-3.223,0.5)(-3.124,0.5)%
\polyline(-3.025,0.5)(-2.926,0.5)\polyline(-2.826,0.5)(-2.727,0.5)\polyline(-2.628,0.5)(-2.529,0.5)%
\polyline(-2.43,0.5)(-2.331,0.5)\polyline(-2.231,0.5)(-2.132,0.5)\polyline(-2.033,0.5)(-1.934,0.5)%
\polyline(-1.835,0.5)(-1.736,0.5)\polyline(-1.636,0.5)(-1.537,0.5)\polyline(-1.438,0.5)(-1.339,0.5)%
\polyline(-1.24,0.5)(-1.14,0.5)\polyline(-1.041,0.5)(-0.942,0.5)\polyline(-0.843,0.5)(-0.744,0.5)%
\polyline(-0.645,0.5)(-0.545,0.5)\polyline(-0.446,0.5)(-0.347,0.5)\polyline(-0.248,0.5)(-0.149,0.5)%
\polyline(-0.05,0.5)(0.05,0.5)\polyline(0.149,0.5)(0.248,0.5)\polyline(0.347,0.5)(0.446,0.5)%
\polyline(0.545,0.5)(0.645,0.5)\polyline(0.744,0.5)(0.843,0.5)\polyline(0.942,0.5)(1.041,0.5)%
\polyline(1.14,0.5)(1.24,0.5)\polyline(1.339,0.5)(1.438,0.5)\polyline(1.537,0.5)(1.636,0.5)%
\polyline(1.736,0.5)(1.835,0.5)\polyline(1.934,0.5)(2.033,0.5)\polyline(2.132,0.5)(2.231,0.5)%
\polyline(2.331,0.5)(2.43,0.5)\polyline(2.529,0.5)(2.628,0.5)\polyline(2.727,0.5)(2.826,0.5)%
\polyline(2.926,0.5)(3.025,0.5)\polyline(3.124,0.5)(3.223,0.5)\polyline(3.322,0.5)(3.421,0.5)%
\polyline(3.521,0.5)(3.62,0.5)\polyline(3.719,0.5)(3.818,0.5)\polyline(3.917,0.5)(4.017,0.5)%
\polyline(4.116,0.5)(4.215,0.5)\polyline(4.314,0.5)(4.413,0.5)\polyline(4.512,0.5)(4.612,0.5)%
\polyline(4.711,0.5)(4.81,0.5)\polyline(4.909,0.5)(5.008,0.5)\polyline(5.107,0.5)(5.207,0.5)%
\polyline(5.306,0.5)(5.405,0.5)\polyline(5.504,0.5)(5.603,0.5)\polyline(5.702,0.5)(5.802,0.5)%
\polyline(5.901,0.5)(6,0.5)%
%
\polyline(1.5,-6)(1.5,-5.901)\polyline(1.5,-5.802)(1.5,-5.702)\polyline(1.5,-5.603)(1.5,-5.504)%
\polyline(1.5,-5.405)(1.5,-5.306)\polyline(1.5,-5.207)(1.5,-5.107)\polyline(1.5,-5.008)(1.5,-4.909)%
\polyline(1.5,-4.81)(1.5,-4.711)\polyline(1.5,-4.612)(1.5,-4.512)\polyline(1.5,-4.413)(1.5,-4.314)%
\polyline(1.5,-4.215)(1.5,-4.116)\polyline(1.5,-4.017)(1.5,-3.917)\polyline(1.5,-3.818)(1.5,-3.719)%
\polyline(1.5,-3.62)(1.5,-3.521)\polyline(1.5,-3.421)(1.5,-3.322)\polyline(1.5,-3.223)(1.5,-3.124)%
\polyline(1.5,-3.025)(1.5,-2.926)\polyline(1.5,-2.826)(1.5,-2.727)\polyline(1.5,-2.628)(1.5,-2.529)%
\polyline(1.5,-2.43)(1.5,-2.331)\polyline(1.5,-2.231)(1.5,-2.132)\polyline(1.5,-2.033)(1.5,-1.934)%
\polyline(1.5,-1.835)(1.5,-1.736)\polyline(1.5,-1.636)(1.5,-1.537)\polyline(1.5,-1.438)(1.5,-1.339)%
\polyline(1.5,-1.24)(1.5,-1.14)\polyline(1.5,-1.041)(1.5,-0.942)\polyline(1.5,-0.843)(1.5,-0.744)%
\polyline(1.5,-0.645)(1.5,-0.545)\polyline(1.5,-0.446)(1.5,-0.347)\polyline(1.5,-0.248)(1.5,-0.149)%
\polyline(1.5,-0.05)(1.5,0.05)\polyline(1.5,0.149)(1.5,0.248)\polyline(1.5,0.347)(1.5,0.446)%
\polyline(1.5,0.545)(1.5,0.645)\polyline(1.5,0.744)(1.5,0.843)\polyline(1.5,0.942)(1.5,1.041)%
\polyline(1.5,1.14)(1.5,1.24)\polyline(1.5,1.339)(1.5,1.438)\polyline(1.5,1.537)(1.5,1.636)%
\polyline(1.5,1.736)(1.5,1.835)\polyline(1.5,1.934)(1.5,2.033)\polyline(1.5,2.132)(1.5,2.231)%
\polyline(1.5,2.331)(1.5,2.43)\polyline(1.5,2.529)(1.5,2.628)\polyline(1.5,2.727)(1.5,2.826)%
\polyline(1.5,2.926)(1.5,3.025)\polyline(1.5,3.124)(1.5,3.223)\polyline(1.5,3.322)(1.5,3.421)%
\polyline(1.5,3.521)(1.5,3.62)\polyline(1.5,3.719)(1.5,3.818)\polyline(1.5,3.917)(1.5,4.017)%
\polyline(1.5,4.116)(1.5,4.215)\polyline(1.5,4.314)(1.5,4.413)\polyline(1.5,4.512)(1.5,4.612)%
\polyline(1.5,4.711)(1.5,4.81)\polyline(1.5,4.909)(1.5,5.008)\polyline(1.5,5.107)(1.5,5.207)%
\polyline(1.5,5.306)(1.5,5.405)\polyline(1.5,5.504)(1.5,5.603)\polyline(1.5,5.702)(1.5,5.802)%
\polyline(1.5,5.901)(1.5,6)%
%
\polyline(-6,1.5)(-5.901,1.5)\polyline(-5.802,1.5)(-5.702,1.5)\polyline(-5.603,1.5)(-5.504,1.5)%
\polyline(-5.405,1.5)(-5.306,1.5)\polyline(-5.207,1.5)(-5.107,1.5)\polyline(-5.008,1.5)(-4.909,1.5)%
\polyline(-4.81,1.5)(-4.711,1.5)\polyline(-4.612,1.5)(-4.512,1.5)\polyline(-4.413,1.5)(-4.314,1.5)%
\polyline(-4.215,1.5)(-4.116,1.5)\polyline(-4.017,1.5)(-3.917,1.5)\polyline(-3.818,1.5)(-3.719,1.5)%
\polyline(-3.62,1.5)(-3.521,1.5)\polyline(-3.421,1.5)(-3.322,1.5)\polyline(-3.223,1.5)(-3.124,1.5)%
\polyline(-3.025,1.5)(-2.926,1.5)\polyline(-2.826,1.5)(-2.727,1.5)\polyline(-2.628,1.5)(-2.529,1.5)%
\polyline(-2.43,1.5)(-2.331,1.5)\polyline(-2.231,1.5)(-2.132,1.5)\polyline(-2.033,1.5)(-1.934,1.5)%
\polyline(-1.835,1.5)(-1.736,1.5)\polyline(-1.636,1.5)(-1.537,1.5)\polyline(-1.438,1.5)(-1.339,1.5)%
\polyline(-1.24,1.5)(-1.14,1.5)\polyline(-1.041,1.5)(-0.942,1.5)\polyline(-0.843,1.5)(-0.744,1.5)%
\polyline(-0.645,1.5)(-0.545,1.5)\polyline(-0.446,1.5)(-0.347,1.5)\polyline(-0.248,1.5)(-0.149,1.5)%
\polyline(-0.05,1.5)(0.05,1.5)\polyline(0.149,1.5)(0.248,1.5)\polyline(0.347,1.5)(0.446,1.5)%
\polyline(0.545,1.5)(0.645,1.5)\polyline(0.744,1.5)(0.843,1.5)\polyline(0.942,1.5)(1.041,1.5)%
\polyline(1.14,1.5)(1.24,1.5)\polyline(1.339,1.5)(1.438,1.5)\polyline(1.537,1.5)(1.636,1.5)%
\polyline(1.736,1.5)(1.835,1.5)\polyline(1.934,1.5)(2.033,1.5)\polyline(2.132,1.5)(2.231,1.5)%
\polyline(2.331,1.5)(2.43,1.5)\polyline(2.529,1.5)(2.628,1.5)\polyline(2.727,1.5)(2.826,1.5)%
\polyline(2.926,1.5)(3.025,1.5)\polyline(3.124,1.5)(3.223,1.5)\polyline(3.322,1.5)(3.421,1.5)%
\polyline(3.521,1.5)(3.62,1.5)\polyline(3.719,1.5)(3.818,1.5)\polyline(3.917,1.5)(4.017,1.5)%
\polyline(4.116,1.5)(4.215,1.5)\polyline(4.314,1.5)(4.413,1.5)\polyline(4.512,1.5)(4.612,1.5)%
\polyline(4.711,1.5)(4.81,1.5)\polyline(4.909,1.5)(5.008,1.5)\polyline(5.107,1.5)(5.207,1.5)%
\polyline(5.306,1.5)(5.405,1.5)\polyline(5.504,1.5)(5.603,1.5)\polyline(5.702,1.5)(5.802,1.5)%
\polyline(5.901,1.5)(6,1.5)%
%
\polyline(2.5,-6)(2.5,-5.901)\polyline(2.5,-5.802)(2.5,-5.702)\polyline(2.5,-5.603)(2.5,-5.504)%
\polyline(2.5,-5.405)(2.5,-5.306)\polyline(2.5,-5.207)(2.5,-5.107)\polyline(2.5,-5.008)(2.5,-4.909)%
\polyline(2.5,-4.81)(2.5,-4.711)\polyline(2.5,-4.612)(2.5,-4.512)\polyline(2.5,-4.413)(2.5,-4.314)%
\polyline(2.5,-4.215)(2.5,-4.116)\polyline(2.5,-4.017)(2.5,-3.917)\polyline(2.5,-3.818)(2.5,-3.719)%
\polyline(2.5,-3.62)(2.5,-3.521)\polyline(2.5,-3.421)(2.5,-3.322)\polyline(2.5,-3.223)(2.5,-3.124)%
\polyline(2.5,-3.025)(2.5,-2.926)\polyline(2.5,-2.826)(2.5,-2.727)\polyline(2.5,-2.628)(2.5,-2.529)%
\polyline(2.5,-2.43)(2.5,-2.331)\polyline(2.5,-2.231)(2.5,-2.132)\polyline(2.5,-2.033)(2.5,-1.934)%
\polyline(2.5,-1.835)(2.5,-1.736)\polyline(2.5,-1.636)(2.5,-1.537)\polyline(2.5,-1.438)(2.5,-1.339)%
\polyline(2.5,-1.24)(2.5,-1.14)\polyline(2.5,-1.041)(2.5,-0.942)\polyline(2.5,-0.843)(2.5,-0.744)%
\polyline(2.5,-0.645)(2.5,-0.545)\polyline(2.5,-0.446)(2.5,-0.347)\polyline(2.5,-0.248)(2.5,-0.149)%
\polyline(2.5,-0.05)(2.5,0.05)\polyline(2.5,0.149)(2.5,0.248)\polyline(2.5,0.347)(2.5,0.446)%
\polyline(2.5,0.545)(2.5,0.645)\polyline(2.5,0.744)(2.5,0.843)\polyline(2.5,0.942)(2.5,1.041)%
\polyline(2.5,1.14)(2.5,1.24)\polyline(2.5,1.339)(2.5,1.438)\polyline(2.5,1.537)(2.5,1.636)%
\polyline(2.5,1.736)(2.5,1.835)\polyline(2.5,1.934)(2.5,2.033)\polyline(2.5,2.132)(2.5,2.231)%
\polyline(2.5,2.331)(2.5,2.43)\polyline(2.5,2.529)(2.5,2.628)\polyline(2.5,2.727)(2.5,2.826)%
\polyline(2.5,2.926)(2.5,3.025)\polyline(2.5,3.124)(2.5,3.223)\polyline(2.5,3.322)(2.5,3.421)%
\polyline(2.5,3.521)(2.5,3.62)\polyline(2.5,3.719)(2.5,3.818)\polyline(2.5,3.917)(2.5,4.017)%
\polyline(2.5,4.116)(2.5,4.215)\polyline(2.5,4.314)(2.5,4.413)\polyline(2.5,4.512)(2.5,4.612)%
\polyline(2.5,4.711)(2.5,4.81)\polyline(2.5,4.909)(2.5,5.008)\polyline(2.5,5.107)(2.5,5.207)%
\polyline(2.5,5.306)(2.5,5.405)\polyline(2.5,5.504)(2.5,5.603)\polyline(2.5,5.702)(2.5,5.802)%
\polyline(2.5,5.901)(2.5,6)%
%
\polyline(-6,2.5)(-5.901,2.5)\polyline(-5.802,2.5)(-5.702,2.5)\polyline(-5.603,2.5)(-5.504,2.5)%
\polyline(-5.405,2.5)(-5.306,2.5)\polyline(-5.207,2.5)(-5.107,2.5)\polyline(-5.008,2.5)(-4.909,2.5)%
\polyline(-4.81,2.5)(-4.711,2.5)\polyline(-4.612,2.5)(-4.512,2.5)\polyline(-4.413,2.5)(-4.314,2.5)%
\polyline(-4.215,2.5)(-4.116,2.5)\polyline(-4.017,2.5)(-3.917,2.5)\polyline(-3.818,2.5)(-3.719,2.5)%
\polyline(-3.62,2.5)(-3.521,2.5)\polyline(-3.421,2.5)(-3.322,2.5)\polyline(-3.223,2.5)(-3.124,2.5)%
\polyline(-3.025,2.5)(-2.926,2.5)\polyline(-2.826,2.5)(-2.727,2.5)\polyline(-2.628,2.5)(-2.529,2.5)%
\polyline(-2.43,2.5)(-2.331,2.5)\polyline(-2.231,2.5)(-2.132,2.5)\polyline(-2.033,2.5)(-1.934,2.5)%
\polyline(-1.835,2.5)(-1.736,2.5)\polyline(-1.636,2.5)(-1.537,2.5)\polyline(-1.438,2.5)(-1.339,2.5)%
\polyline(-1.24,2.5)(-1.14,2.5)\polyline(-1.041,2.5)(-0.942,2.5)\polyline(-0.843,2.5)(-0.744,2.5)%
\polyline(-0.645,2.5)(-0.545,2.5)\polyline(-0.446,2.5)(-0.347,2.5)\polyline(-0.248,2.5)(-0.149,2.5)%
\polyline(-0.05,2.5)(0.05,2.5)\polyline(0.149,2.5)(0.248,2.5)\polyline(0.347,2.5)(0.446,2.5)%
\polyline(0.545,2.5)(0.645,2.5)\polyline(0.744,2.5)(0.843,2.5)\polyline(0.942,2.5)(1.041,2.5)%
\polyline(1.14,2.5)(1.24,2.5)\polyline(1.339,2.5)(1.438,2.5)\polyline(1.537,2.5)(1.636,2.5)%
\polyline(1.736,2.5)(1.835,2.5)\polyline(1.934,2.5)(2.033,2.5)\polyline(2.132,2.5)(2.231,2.5)%
\polyline(2.331,2.5)(2.43,2.5)\polyline(2.529,2.5)(2.628,2.5)\polyline(2.727,2.5)(2.826,2.5)%
\polyline(2.926,2.5)(3.025,2.5)\polyline(3.124,2.5)(3.223,2.5)\polyline(3.322,2.5)(3.421,2.5)%
\polyline(3.521,2.5)(3.62,2.5)\polyline(3.719,2.5)(3.818,2.5)\polyline(3.917,2.5)(4.017,2.5)%
\polyline(4.116,2.5)(4.215,2.5)\polyline(4.314,2.5)(4.413,2.5)\polyline(4.512,2.5)(4.612,2.5)%
\polyline(4.711,2.5)(4.81,2.5)\polyline(4.909,2.5)(5.008,2.5)\polyline(5.107,2.5)(5.207,2.5)%
\polyline(5.306,2.5)(5.405,2.5)\polyline(5.504,2.5)(5.603,2.5)\polyline(5.702,2.5)(5.802,2.5)%
\polyline(5.901,2.5)(6,2.5)%
%
\polyline(3.5,-6)(3.5,-5.901)\polyline(3.5,-5.802)(3.5,-5.702)\polyline(3.5,-5.603)(3.5,-5.504)%
\polyline(3.5,-5.405)(3.5,-5.306)\polyline(3.5,-5.207)(3.5,-5.107)\polyline(3.5,-5.008)(3.5,-4.909)%
\polyline(3.5,-4.81)(3.5,-4.711)\polyline(3.5,-4.612)(3.5,-4.512)\polyline(3.5,-4.413)(3.5,-4.314)%
\polyline(3.5,-4.215)(3.5,-4.116)\polyline(3.5,-4.017)(3.5,-3.917)\polyline(3.5,-3.818)(3.5,-3.719)%
\polyline(3.5,-3.62)(3.5,-3.521)\polyline(3.5,-3.421)(3.5,-3.322)\polyline(3.5,-3.223)(3.5,-3.124)%
\polyline(3.5,-3.025)(3.5,-2.926)\polyline(3.5,-2.826)(3.5,-2.727)\polyline(3.5,-2.628)(3.5,-2.529)%
\polyline(3.5,-2.43)(3.5,-2.331)\polyline(3.5,-2.231)(3.5,-2.132)\polyline(3.5,-2.033)(3.5,-1.934)%
\polyline(3.5,-1.835)(3.5,-1.736)\polyline(3.5,-1.636)(3.5,-1.537)\polyline(3.5,-1.438)(3.5,-1.339)%
\polyline(3.5,-1.24)(3.5,-1.14)\polyline(3.5,-1.041)(3.5,-0.942)\polyline(3.5,-0.843)(3.5,-0.744)%
\polyline(3.5,-0.645)(3.5,-0.545)\polyline(3.5,-0.446)(3.5,-0.347)\polyline(3.5,-0.248)(3.5,-0.149)%
\polyline(3.5,-0.05)(3.5,0.05)\polyline(3.5,0.149)(3.5,0.248)\polyline(3.5,0.347)(3.5,0.446)%
\polyline(3.5,0.545)(3.5,0.645)\polyline(3.5,0.744)(3.5,0.843)\polyline(3.5,0.942)(3.5,1.041)%
\polyline(3.5,1.14)(3.5,1.24)\polyline(3.5,1.339)(3.5,1.438)\polyline(3.5,1.537)(3.5,1.636)%
\polyline(3.5,1.736)(3.5,1.835)\polyline(3.5,1.934)(3.5,2.033)\polyline(3.5,2.132)(3.5,2.231)%
\polyline(3.5,2.331)(3.5,2.43)\polyline(3.5,2.529)(3.5,2.628)\polyline(3.5,2.727)(3.5,2.826)%
\polyline(3.5,2.926)(3.5,3.025)\polyline(3.5,3.124)(3.5,3.223)\polyline(3.5,3.322)(3.5,3.421)%
\polyline(3.5,3.521)(3.5,3.62)\polyline(3.5,3.719)(3.5,3.818)\polyline(3.5,3.917)(3.5,4.017)%
\polyline(3.5,4.116)(3.5,4.215)\polyline(3.5,4.314)(3.5,4.413)\polyline(3.5,4.512)(3.5,4.612)%
\polyline(3.5,4.711)(3.5,4.81)\polyline(3.5,4.909)(3.5,5.008)\polyline(3.5,5.107)(3.5,5.207)%
\polyline(3.5,5.306)(3.5,5.405)\polyline(3.5,5.504)(3.5,5.603)\polyline(3.5,5.702)(3.5,5.802)%
\polyline(3.5,5.901)(3.5,6)%
%
\polyline(-6,3.5)(-5.901,3.5)\polyline(-5.802,3.5)(-5.702,3.5)\polyline(-5.603,3.5)(-5.504,3.5)%
\polyline(-5.405,3.5)(-5.306,3.5)\polyline(-5.207,3.5)(-5.107,3.5)\polyline(-5.008,3.5)(-4.909,3.5)%
\polyline(-4.81,3.5)(-4.711,3.5)\polyline(-4.612,3.5)(-4.512,3.5)\polyline(-4.413,3.5)(-4.314,3.5)%
\polyline(-4.215,3.5)(-4.116,3.5)\polyline(-4.017,3.5)(-3.917,3.5)\polyline(-3.818,3.5)(-3.719,3.5)%
\polyline(-3.62,3.5)(-3.521,3.5)\polyline(-3.421,3.5)(-3.322,3.5)\polyline(-3.223,3.5)(-3.124,3.5)%
\polyline(-3.025,3.5)(-2.926,3.5)\polyline(-2.826,3.5)(-2.727,3.5)\polyline(-2.628,3.5)(-2.529,3.5)%
\polyline(-2.43,3.5)(-2.331,3.5)\polyline(-2.231,3.5)(-2.132,3.5)\polyline(-2.033,3.5)(-1.934,3.5)%
\polyline(-1.835,3.5)(-1.736,3.5)\polyline(-1.636,3.5)(-1.537,3.5)\polyline(-1.438,3.5)(-1.339,3.5)%
\polyline(-1.24,3.5)(-1.14,3.5)\polyline(-1.041,3.5)(-0.942,3.5)\polyline(-0.843,3.5)(-0.744,3.5)%
\polyline(-0.645,3.5)(-0.545,3.5)\polyline(-0.446,3.5)(-0.347,3.5)\polyline(-0.248,3.5)(-0.149,3.5)%
\polyline(-0.05,3.5)(0.05,3.5)\polyline(0.149,3.5)(0.248,3.5)\polyline(0.347,3.5)(0.446,3.5)%
\polyline(0.545,3.5)(0.645,3.5)\polyline(0.744,3.5)(0.843,3.5)\polyline(0.942,3.5)(1.041,3.5)%
\polyline(1.14,3.5)(1.24,3.5)\polyline(1.339,3.5)(1.438,3.5)\polyline(1.537,3.5)(1.636,3.5)%
\polyline(1.736,3.5)(1.835,3.5)\polyline(1.934,3.5)(2.033,3.5)\polyline(2.132,3.5)(2.231,3.5)%
\polyline(2.331,3.5)(2.43,3.5)\polyline(2.529,3.5)(2.628,3.5)\polyline(2.727,3.5)(2.826,3.5)%
\polyline(2.926,3.5)(3.025,3.5)\polyline(3.124,3.5)(3.223,3.5)\polyline(3.322,3.5)(3.421,3.5)%
\polyline(3.521,3.5)(3.62,3.5)\polyline(3.719,3.5)(3.818,3.5)\polyline(3.917,3.5)(4.017,3.5)%
\polyline(4.116,3.5)(4.215,3.5)\polyline(4.314,3.5)(4.413,3.5)\polyline(4.512,3.5)(4.612,3.5)%
\polyline(4.711,3.5)(4.81,3.5)\polyline(4.909,3.5)(5.008,3.5)\polyline(5.107,3.5)(5.207,3.5)%
\polyline(5.306,3.5)(5.405,3.5)\polyline(5.504,3.5)(5.603,3.5)\polyline(5.702,3.5)(5.802,3.5)%
\polyline(5.901,3.5)(6,3.5)%
%
\polyline(4.5,-6)(4.5,-5.901)\polyline(4.5,-5.802)(4.5,-5.702)\polyline(4.5,-5.603)(4.5,-5.504)%
\polyline(4.5,-5.405)(4.5,-5.306)\polyline(4.5,-5.207)(4.5,-5.107)\polyline(4.5,-5.008)(4.5,-4.909)%
\polyline(4.5,-4.81)(4.5,-4.711)\polyline(4.5,-4.612)(4.5,-4.512)\polyline(4.5,-4.413)(4.5,-4.314)%
\polyline(4.5,-4.215)(4.5,-4.116)\polyline(4.5,-4.017)(4.5,-3.917)\polyline(4.5,-3.818)(4.5,-3.719)%
\polyline(4.5,-3.62)(4.5,-3.521)\polyline(4.5,-3.421)(4.5,-3.322)\polyline(4.5,-3.223)(4.5,-3.124)%
\polyline(4.5,-3.025)(4.5,-2.926)\polyline(4.5,-2.826)(4.5,-2.727)\polyline(4.5,-2.628)(4.5,-2.529)%
\polyline(4.5,-2.43)(4.5,-2.331)\polyline(4.5,-2.231)(4.5,-2.132)\polyline(4.5,-2.033)(4.5,-1.934)%
\polyline(4.5,-1.835)(4.5,-1.736)\polyline(4.5,-1.636)(4.5,-1.537)\polyline(4.5,-1.438)(4.5,-1.339)%
\polyline(4.5,-1.24)(4.5,-1.14)\polyline(4.5,-1.041)(4.5,-0.942)\polyline(4.5,-0.843)(4.5,-0.744)%
\polyline(4.5,-0.645)(4.5,-0.545)\polyline(4.5,-0.446)(4.5,-0.347)\polyline(4.5,-0.248)(4.5,-0.149)%
\polyline(4.5,-0.05)(4.5,0.05)\polyline(4.5,0.149)(4.5,0.248)\polyline(4.5,0.347)(4.5,0.446)%
\polyline(4.5,0.545)(4.5,0.645)\polyline(4.5,0.744)(4.5,0.843)\polyline(4.5,0.942)(4.5,1.041)%
\polyline(4.5,1.14)(4.5,1.24)\polyline(4.5,1.339)(4.5,1.438)\polyline(4.5,1.537)(4.5,1.636)%
\polyline(4.5,1.736)(4.5,1.835)\polyline(4.5,1.934)(4.5,2.033)\polyline(4.5,2.132)(4.5,2.231)%
\polyline(4.5,2.331)(4.5,2.43)\polyline(4.5,2.529)(4.5,2.628)\polyline(4.5,2.727)(4.5,2.826)%
\polyline(4.5,2.926)(4.5,3.025)\polyline(4.5,3.124)(4.5,3.223)\polyline(4.5,3.322)(4.5,3.421)%
\polyline(4.5,3.521)(4.5,3.62)\polyline(4.5,3.719)(4.5,3.818)\polyline(4.5,3.917)(4.5,4.017)%
\polyline(4.5,4.116)(4.5,4.215)\polyline(4.5,4.314)(4.5,4.413)\polyline(4.5,4.512)(4.5,4.612)%
\polyline(4.5,4.711)(4.5,4.81)\polyline(4.5,4.909)(4.5,5.008)\polyline(4.5,5.107)(4.5,5.207)%
\polyline(4.5,5.306)(4.5,5.405)\polyline(4.5,5.504)(4.5,5.603)\polyline(4.5,5.702)(4.5,5.802)%
\polyline(4.5,5.901)(4.5,6)%
%
\polyline(-6,4.5)(-5.901,4.5)\polyline(-5.802,4.5)(-5.702,4.5)\polyline(-5.603,4.5)(-5.504,4.5)%
\polyline(-5.405,4.5)(-5.306,4.5)\polyline(-5.207,4.5)(-5.107,4.5)\polyline(-5.008,4.5)(-4.909,4.5)%
\polyline(-4.81,4.5)(-4.711,4.5)\polyline(-4.612,4.5)(-4.512,4.5)\polyline(-4.413,4.5)(-4.314,4.5)%
\polyline(-4.215,4.5)(-4.116,4.5)\polyline(-4.017,4.5)(-3.917,4.5)\polyline(-3.818,4.5)(-3.719,4.5)%
\polyline(-3.62,4.5)(-3.521,4.5)\polyline(-3.421,4.5)(-3.322,4.5)\polyline(-3.223,4.5)(-3.124,4.5)%
\polyline(-3.025,4.5)(-2.926,4.5)\polyline(-2.826,4.5)(-2.727,4.5)\polyline(-2.628,4.5)(-2.529,4.5)%
\polyline(-2.43,4.5)(-2.331,4.5)\polyline(-2.231,4.5)(-2.132,4.5)\polyline(-2.033,4.5)(-1.934,4.5)%
\polyline(-1.835,4.5)(-1.736,4.5)\polyline(-1.636,4.5)(-1.537,4.5)\polyline(-1.438,4.5)(-1.339,4.5)%
\polyline(-1.24,4.5)(-1.14,4.5)\polyline(-1.041,4.5)(-0.942,4.5)\polyline(-0.843,4.5)(-0.744,4.5)%
\polyline(-0.645,4.5)(-0.545,4.5)\polyline(-0.446,4.5)(-0.347,4.5)\polyline(-0.248,4.5)(-0.149,4.5)%
\polyline(-0.05,4.5)(0.05,4.5)\polyline(0.149,4.5)(0.248,4.5)\polyline(0.347,4.5)(0.446,4.5)%
\polyline(0.545,4.5)(0.645,4.5)\polyline(0.744,4.5)(0.843,4.5)\polyline(0.942,4.5)(1.041,4.5)%
\polyline(1.14,4.5)(1.24,4.5)\polyline(1.339,4.5)(1.438,4.5)\polyline(1.537,4.5)(1.636,4.5)%
\polyline(1.736,4.5)(1.835,4.5)\polyline(1.934,4.5)(2.033,4.5)\polyline(2.132,4.5)(2.231,4.5)%
\polyline(2.331,4.5)(2.43,4.5)\polyline(2.529,4.5)(2.628,4.5)\polyline(2.727,4.5)(2.826,4.5)%
\polyline(2.926,4.5)(3.025,4.5)\polyline(3.124,4.5)(3.223,4.5)\polyline(3.322,4.5)(3.421,4.5)%
\polyline(3.521,4.5)(3.62,4.5)\polyline(3.719,4.5)(3.818,4.5)\polyline(3.917,4.5)(4.017,4.5)%
\polyline(4.116,4.5)(4.215,4.5)\polyline(4.314,4.5)(4.413,4.5)\polyline(4.512,4.5)(4.612,4.5)%
\polyline(4.711,4.5)(4.81,4.5)\polyline(4.909,4.5)(5.008,4.5)\polyline(5.107,4.5)(5.207,4.5)%
\polyline(5.306,4.5)(5.405,4.5)\polyline(5.504,4.5)(5.603,4.5)\polyline(5.702,4.5)(5.802,4.5)%
\polyline(5.901,4.5)(6,4.5)%
%
\polyline(5.5,-6)(5.5,-5.901)\polyline(5.5,-5.802)(5.5,-5.702)\polyline(5.5,-5.603)(5.5,-5.504)%
\polyline(5.5,-5.405)(5.5,-5.306)\polyline(5.5,-5.207)(5.5,-5.107)\polyline(5.5,-5.008)(5.5,-4.909)%
\polyline(5.5,-4.81)(5.5,-4.711)\polyline(5.5,-4.612)(5.5,-4.512)\polyline(5.5,-4.413)(5.5,-4.314)%
\polyline(5.5,-4.215)(5.5,-4.116)\polyline(5.5,-4.017)(5.5,-3.917)\polyline(5.5,-3.818)(5.5,-3.719)%
\polyline(5.5,-3.62)(5.5,-3.521)\polyline(5.5,-3.421)(5.5,-3.322)\polyline(5.5,-3.223)(5.5,-3.124)%
\polyline(5.5,-3.025)(5.5,-2.926)\polyline(5.5,-2.826)(5.5,-2.727)\polyline(5.5,-2.628)(5.5,-2.529)%
\polyline(5.5,-2.43)(5.5,-2.331)\polyline(5.5,-2.231)(5.5,-2.132)\polyline(5.5,-2.033)(5.5,-1.934)%
\polyline(5.5,-1.835)(5.5,-1.736)\polyline(5.5,-1.636)(5.5,-1.537)\polyline(5.5,-1.438)(5.5,-1.339)%
\polyline(5.5,-1.24)(5.5,-1.14)\polyline(5.5,-1.041)(5.5,-0.942)\polyline(5.5,-0.843)(5.5,-0.744)%
\polyline(5.5,-0.645)(5.5,-0.545)\polyline(5.5,-0.446)(5.5,-0.347)\polyline(5.5,-0.248)(5.5,-0.149)%
\polyline(5.5,-0.05)(5.5,0.05)\polyline(5.5,0.149)(5.5,0.248)\polyline(5.5,0.347)(5.5,0.446)%
\polyline(5.5,0.545)(5.5,0.645)\polyline(5.5,0.744)(5.5,0.843)\polyline(5.5,0.942)(5.5,1.041)%
\polyline(5.5,1.14)(5.5,1.24)\polyline(5.5,1.339)(5.5,1.438)\polyline(5.5,1.537)(5.5,1.636)%
\polyline(5.5,1.736)(5.5,1.835)\polyline(5.5,1.934)(5.5,2.033)\polyline(5.5,2.132)(5.5,2.231)%
\polyline(5.5,2.331)(5.5,2.43)\polyline(5.5,2.529)(5.5,2.628)\polyline(5.5,2.727)(5.5,2.826)%
\polyline(5.5,2.926)(5.5,3.025)\polyline(5.5,3.124)(5.5,3.223)\polyline(5.5,3.322)(5.5,3.421)%
\polyline(5.5,3.521)(5.5,3.62)\polyline(5.5,3.719)(5.5,3.818)\polyline(5.5,3.917)(5.5,4.017)%
\polyline(5.5,4.116)(5.5,4.215)\polyline(5.5,4.314)(5.5,4.413)\polyline(5.5,4.512)(5.5,4.612)%
\polyline(5.5,4.711)(5.5,4.81)\polyline(5.5,4.909)(5.5,5.008)\polyline(5.5,5.107)(5.5,5.207)%
\polyline(5.5,5.306)(5.5,5.405)\polyline(5.5,5.504)(5.5,5.603)\polyline(5.5,5.702)(5.5,5.802)%
\polyline(5.5,5.901)(5.5,6)%
%
\polyline(-6,5.5)(-5.901,5.5)\polyline(-5.802,5.5)(-5.702,5.5)\polyline(-5.603,5.5)(-5.504,5.5)%
\polyline(-5.405,5.5)(-5.306,5.5)\polyline(-5.207,5.5)(-5.107,5.5)\polyline(-5.008,5.5)(-4.909,5.5)%
\polyline(-4.81,5.5)(-4.711,5.5)\polyline(-4.612,5.5)(-4.512,5.5)\polyline(-4.413,5.5)(-4.314,5.5)%
\polyline(-4.215,5.5)(-4.116,5.5)\polyline(-4.017,5.5)(-3.917,5.5)\polyline(-3.818,5.5)(-3.719,5.5)%
\polyline(-3.62,5.5)(-3.521,5.5)\polyline(-3.421,5.5)(-3.322,5.5)\polyline(-3.223,5.5)(-3.124,5.5)%
\polyline(-3.025,5.5)(-2.926,5.5)\polyline(-2.826,5.5)(-2.727,5.5)\polyline(-2.628,5.5)(-2.529,5.5)%
\polyline(-2.43,5.5)(-2.331,5.5)\polyline(-2.231,5.5)(-2.132,5.5)\polyline(-2.033,5.5)(-1.934,5.5)%
\polyline(-1.835,5.5)(-1.736,5.5)\polyline(-1.636,5.5)(-1.537,5.5)\polyline(-1.438,5.5)(-1.339,5.5)%
\polyline(-1.24,5.5)(-1.14,5.5)\polyline(-1.041,5.5)(-0.942,5.5)\polyline(-0.843,5.5)(-0.744,5.5)%
\polyline(-0.645,5.5)(-0.545,5.5)\polyline(-0.446,5.5)(-0.347,5.5)\polyline(-0.248,5.5)(-0.149,5.5)%
\polyline(-0.05,5.5)(0.05,5.5)\polyline(0.149,5.5)(0.248,5.5)\polyline(0.347,5.5)(0.446,5.5)%
\polyline(0.545,5.5)(0.645,5.5)\polyline(0.744,5.5)(0.843,5.5)\polyline(0.942,5.5)(1.041,5.5)%
\polyline(1.14,5.5)(1.24,5.5)\polyline(1.339,5.5)(1.438,5.5)\polyline(1.537,5.5)(1.636,5.5)%
\polyline(1.736,5.5)(1.835,5.5)\polyline(1.934,5.5)(2.033,5.5)\polyline(2.132,5.5)(2.231,5.5)%
\polyline(2.331,5.5)(2.43,5.5)\polyline(2.529,5.5)(2.628,5.5)\polyline(2.727,5.5)(2.826,5.5)%
\polyline(2.926,5.5)(3.025,5.5)\polyline(3.124,5.5)(3.223,5.5)\polyline(3.322,5.5)(3.421,5.5)%
\polyline(3.521,5.5)(3.62,5.5)\polyline(3.719,5.5)(3.818,5.5)\polyline(3.917,5.5)(4.017,5.5)%
\polyline(4.116,5.5)(4.215,5.5)\polyline(4.314,5.5)(4.413,5.5)\polyline(4.512,5.5)(4.612,5.5)%
\polyline(4.711,5.5)(4.81,5.5)\polyline(4.909,5.5)(5.008,5.5)\polyline(5.107,5.5)(5.207,5.5)%
\polyline(5.306,5.5)(5.405,5.5)\polyline(5.504,5.5)(5.603,5.5)\polyline(5.702,5.5)(5.802,5.5)%
\polyline(5.901,5.5)(6,5.5)%
%
\linethickness{0.008in}%%
\polyline(6,-0.05)(6,0.05)%
%
\settowidth{\Width}{$-6$}\setlength{\Width}{-0.5\Width}%
\settoheight{\Height}{$-6$}\settodepth{\Depth}{$-6$}\setlength{\Height}{-\Height}%
\put( -6.000, -0.100){\hspace*{\Width}\raisebox{\Height}{$-6$}}%
%
\polyline(-0.05,6)(0.05,6)%
%
\settowidth{\Width}{$-6$}\setlength{\Width}{-1\Width}%
\settoheight{\Height}{$-6$}\settodepth{\Depth}{$-6$}\setlength{\Height}{-0.5\Height}\setlength{\Depth}{0.5\Depth}\addtolength{\Height}{\Depth}%
\put( -0.100, -6.000){\hspace*{\Width}\raisebox{\Height}{$-6$}}%
%
\polyline(6,-0.05)(6,0.05)%
%
\settowidth{\Width}{$-5$}\setlength{\Width}{-0.5\Width}%
\settoheight{\Height}{$-5$}\settodepth{\Depth}{$-5$}\setlength{\Height}{-\Height}%
\put( -5.000, -0.100){\hspace*{\Width}\raisebox{\Height}{$-5$}}%
%
\polyline(-0.05,6)(0.05,6)%
%
\settowidth{\Width}{$-5$}\setlength{\Width}{-1\Width}%
\settoheight{\Height}{$-5$}\settodepth{\Depth}{$-5$}\setlength{\Height}{-0.5\Height}\setlength{\Depth}{0.5\Depth}\addtolength{\Height}{\Depth}%
\put( -0.100, -5.000){\hspace*{\Width}\raisebox{\Height}{$-5$}}%
%
\polyline(6,-0.05)(6,0.05)%
%
\settowidth{\Width}{$-4$}\setlength{\Width}{-0.5\Width}%
\settoheight{\Height}{$-4$}\settodepth{\Depth}{$-4$}\setlength{\Height}{-\Height}%
\put( -4.000, -0.100){\hspace*{\Width}\raisebox{\Height}{$-4$}}%
%
\polyline(-0.05,6)(0.05,6)%
%
\settowidth{\Width}{$-4$}\setlength{\Width}{-1\Width}%
\settoheight{\Height}{$-4$}\settodepth{\Depth}{$-4$}\setlength{\Height}{-0.5\Height}\setlength{\Depth}{0.5\Depth}\addtolength{\Height}{\Depth}%
\put( -0.100, -4.000){\hspace*{\Width}\raisebox{\Height}{$-4$}}%
%
\polyline(6,-0.05)(6,0.05)%
%
\settowidth{\Width}{$-3$}\setlength{\Width}{-0.5\Width}%
\settoheight{\Height}{$-3$}\settodepth{\Depth}{$-3$}\setlength{\Height}{-\Height}%
\put( -3.000, -0.100){\hspace*{\Width}\raisebox{\Height}{$-3$}}%
%
\polyline(-0.05,6)(0.05,6)%
%
\settowidth{\Width}{$-3$}\setlength{\Width}{-1\Width}%
\settoheight{\Height}{$-3$}\settodepth{\Depth}{$-3$}\setlength{\Height}{-0.5\Height}\setlength{\Depth}{0.5\Depth}\addtolength{\Height}{\Depth}%
\put( -0.100, -3.000){\hspace*{\Width}\raisebox{\Height}{$-3$}}%
%
\polyline(6,-0.05)(6,0.05)%
%
\settowidth{\Width}{$-2$}\setlength{\Width}{-0.5\Width}%
\settoheight{\Height}{$-2$}\settodepth{\Depth}{$-2$}\setlength{\Height}{-\Height}%
\put( -2.000, -0.100){\hspace*{\Width}\raisebox{\Height}{$-2$}}%
%
\polyline(-0.05,6)(0.05,6)%
%
\settowidth{\Width}{$-2$}\setlength{\Width}{-1\Width}%
\settoheight{\Height}{$-2$}\settodepth{\Depth}{$-2$}\setlength{\Height}{-0.5\Height}\setlength{\Depth}{0.5\Depth}\addtolength{\Height}{\Depth}%
\put( -0.100, -2.000){\hspace*{\Width}\raisebox{\Height}{$-2$}}%
%
\polyline(6,-0.05)(6,0.05)%
%
\settowidth{\Width}{$-1$}\setlength{\Width}{-0.5\Width}%
\settoheight{\Height}{$-1$}\settodepth{\Depth}{$-1$}\setlength{\Height}{-\Height}%
\put( -1.000, -0.100){\hspace*{\Width}\raisebox{\Height}{$-1$}}%
%
\polyline(-0.05,6)(0.05,6)%
%
\settowidth{\Width}{$-1$}\setlength{\Width}{-1\Width}%
\settoheight{\Height}{$-1$}\settodepth{\Depth}{$-1$}\setlength{\Height}{-0.5\Height}\setlength{\Depth}{0.5\Depth}\addtolength{\Height}{\Depth}%
\put( -0.100, -1.000){\hspace*{\Width}\raisebox{\Height}{$-1$}}%
%
\polyline(6,-0.05)(6,0.05)%
%
\settowidth{\Width}{$1$}\setlength{\Width}{-0.5\Width}%
\settoheight{\Height}{$1$}\settodepth{\Depth}{$1$}\setlength{\Height}{-\Height}%
\put(  1.000, -0.100){\hspace*{\Width}\raisebox{\Height}{$1$}}%
%
\polyline(-0.05,6)(0.05,6)%
%
\settowidth{\Width}{$1$}\setlength{\Width}{-1\Width}%
\settoheight{\Height}{$1$}\settodepth{\Depth}{$1$}\setlength{\Height}{-0.5\Height}\setlength{\Depth}{0.5\Depth}\addtolength{\Height}{\Depth}%
\put( -0.100,  1.000){\hspace*{\Width}\raisebox{\Height}{$1$}}%
%
\polyline(6,-0.05)(6,0.05)%
%
\settowidth{\Width}{$2$}\setlength{\Width}{-0.5\Width}%
\settoheight{\Height}{$2$}\settodepth{\Depth}{$2$}\setlength{\Height}{-\Height}%
\put(  2.000, -0.100){\hspace*{\Width}\raisebox{\Height}{$2$}}%
%
\polyline(-0.05,6)(0.05,6)%
%
\settowidth{\Width}{$2$}\setlength{\Width}{-1\Width}%
\settoheight{\Height}{$2$}\settodepth{\Depth}{$2$}\setlength{\Height}{-0.5\Height}\setlength{\Depth}{0.5\Depth}\addtolength{\Height}{\Depth}%
\put( -0.100,  2.000){\hspace*{\Width}\raisebox{\Height}{$2$}}%
%
\polyline(6,-0.05)(6,0.05)%
%
\settowidth{\Width}{$3$}\setlength{\Width}{-0.5\Width}%
\settoheight{\Height}{$3$}\settodepth{\Depth}{$3$}\setlength{\Height}{-\Height}%
\put(  3.000, -0.100){\hspace*{\Width}\raisebox{\Height}{$3$}}%
%
\polyline(-0.05,6)(0.05,6)%
%
\settowidth{\Width}{$3$}\setlength{\Width}{-1\Width}%
\settoheight{\Height}{$3$}\settodepth{\Depth}{$3$}\setlength{\Height}{-0.5\Height}\setlength{\Depth}{0.5\Depth}\addtolength{\Height}{\Depth}%
\put( -0.100,  3.000){\hspace*{\Width}\raisebox{\Height}{$3$}}%
%
\polyline(6,-0.05)(6,0.05)%
%
\settowidth{\Width}{$4$}\setlength{\Width}{-0.5\Width}%
\settoheight{\Height}{$4$}\settodepth{\Depth}{$4$}\setlength{\Height}{-\Height}%
\put(  4.000, -0.100){\hspace*{\Width}\raisebox{\Height}{$4$}}%
%
\polyline(-0.05,6)(0.05,6)%
%
\settowidth{\Width}{$4$}\setlength{\Width}{-1\Width}%
\settoheight{\Height}{$4$}\settodepth{\Depth}{$4$}\setlength{\Height}{-0.5\Height}\setlength{\Depth}{0.5\Depth}\addtolength{\Height}{\Depth}%
\put( -0.100,  4.000){\hspace*{\Width}\raisebox{\Height}{$4$}}%
%
\polyline(6,-0.05)(6,0.05)%
%
\settowidth{\Width}{$5$}\setlength{\Width}{-0.5\Width}%
\settoheight{\Height}{$5$}\settodepth{\Depth}{$5$}\setlength{\Height}{-\Height}%
\put(  5.000, -0.100){\hspace*{\Width}\raisebox{\Height}{$5$}}%
%
\polyline(-0.05,6)(0.05,6)%
%
\settowidth{\Width}{$5$}\setlength{\Width}{-1\Width}%
\settoheight{\Height}{$5$}\settodepth{\Depth}{$5$}\setlength{\Height}{-0.5\Height}\setlength{\Depth}{0.5\Depth}\addtolength{\Height}{\Depth}%
\put( -0.100,  5.000){\hspace*{\Width}\raisebox{\Height}{$5$}}%
%
\polyline(6,-0.05)(6,0.05)%
%
\settowidth{\Width}{$6$}\setlength{\Width}{-0.5\Width}%
\settoheight{\Height}{$6$}\settodepth{\Depth}{$6$}\setlength{\Height}{-\Height}%
\put(  6.000, -0.100){\hspace*{\Width}\raisebox{\Height}{$6$}}%
%
\polyline(-0.05,6)(0.05,6)%
%
\settowidth{\Width}{$6$}\setlength{\Width}{-1\Width}%
\settoheight{\Height}{$6$}\settodepth{\Depth}{$6$}\setlength{\Height}{-0.5\Height}\setlength{\Depth}{0.5\Depth}\addtolength{\Height}{\Depth}%
\put( -0.100,  6.000){\hspace*{\Width}\raisebox{\Height}{$6$}}%
%
\polyline(-6.2,0)(6.2,0)%
%
\polyline(0,-6.2)(0,6.2)%
%
\settowidth{\Width}{$x$}\setlength{\Width}{0\Width}%
\settoheight{\Height}{$x$}\settodepth{\Depth}{$x$}\setlength{\Height}{-0.5\Height}\setlength{\Depth}{0.5\Depth}\addtolength{\Height}{\Depth}%
\put(  6.250,  0.000){\hspace*{\Width}\raisebox{\Height}{$x$}}%
%
\settowidth{\Width}{$y$}\setlength{\Width}{-0.5\Width}%
\settoheight{\Height}{$y$}\settodepth{\Depth}{$y$}\setlength{\Height}{\Depth}%
\put(  0.000,  6.250){\hspace*{\Width}\raisebox{\Height}{$y$}}%
%
\settowidth{\Width}{O}\setlength{\Width}{-1\Width}%
\settoheight{\Height}{O}\settodepth{\Depth}{O}\setlength{\Height}{-\Height}%
\put( -0.050, -0.050){\hspace*{\Width}\raisebox{\Height}{O}}%
%
\end{picture}}%}}
\end{layer}

%%%%%%%%%%%%

%%%%%%%%%%%%%%%%%%%%


\sameslide

\vspace*{18mm}

\slidepage
\down
例)$y=2x+1$

\begin{layer}{120}{0}
\putnotese{70}{-3}{\scalebox{0.6}{%%% /Users/takatoosetsuo/polytech23.git/101-0417/presen/fig/table1b.tex 
%%% Generator=presen23101.cdy 
{\unitlength=1cm%
\begin{picture}%
(9.6,1.2)(0,0)%
\linethickness{0.008in}%%
\Large\bf\boldmath%
\small%
\polyline(0,1.2)(0,0)%
%
\polyline(0.8,1.2)(0.8,0)%
%
\polyline(1.6,1.2)(1.6,0)%
%
\polyline(2.4,1.2)(2.4,0)%
%
\polyline(3.2,1.2)(3.2,0)%
%
\polyline(4,1.2)(4,0)%
%
\polyline(4.8,1.2)(4.8,0)%
%
\polyline(5.6,1.2)(5.6,0)%
%
\polyline(6.4,1.2)(6.4,0)%
%
\polyline(7.2,1.2)(7.2,0)%
%
\polyline(8,1.2)(8,0)%
%
\polyline(8.8,1.2)(8.8,0)%
%
\polyline(9.6,1.2)(9.6,0)%
%
\polyline(0,1.2)(9.6,1.2)%
%
\polyline(0,0.6)(9.6,0.6)%
%
\polyline(0,0)(9.6,0)%
%
\settowidth{\Width}{$x$}\setlength{\Width}{-0.5\Width}%
\settoheight{\Height}{$x$}\settodepth{\Depth}{$x$}\setlength{\Height}{-0.5\Height}\setlength{\Depth}{0.5\Depth}\addtolength{\Height}{\Depth}%
\put(  0.400,  0.900){\hspace*{\Width}\raisebox{\Height}{$x$}}%
%
\settowidth{\Width}{$-5$}\setlength{\Width}{-0.5\Width}%
\settoheight{\Height}{$-5$}\settodepth{\Depth}{$-5$}\setlength{\Height}{-0.5\Height}\setlength{\Depth}{0.5\Depth}\addtolength{\Height}{\Depth}%
\put(  1.200,  0.900){\hspace*{\Width}\raisebox{\Height}{$-5$}}%
%
\settowidth{\Width}{$-4$}\setlength{\Width}{-0.5\Width}%
\settoheight{\Height}{$-4$}\settodepth{\Depth}{$-4$}\setlength{\Height}{-0.5\Height}\setlength{\Depth}{0.5\Depth}\addtolength{\Height}{\Depth}%
\put(  2.000,  0.900){\hspace*{\Width}\raisebox{\Height}{$-4$}}%
%
\settowidth{\Width}{$-3$}\setlength{\Width}{-0.5\Width}%
\settoheight{\Height}{$-3$}\settodepth{\Depth}{$-3$}\setlength{\Height}{-0.5\Height}\setlength{\Depth}{0.5\Depth}\addtolength{\Height}{\Depth}%
\put(  2.800,  0.900){\hspace*{\Width}\raisebox{\Height}{$-3$}}%
%
\settowidth{\Width}{$-2$}\setlength{\Width}{-0.5\Width}%
\settoheight{\Height}{$-2$}\settodepth{\Depth}{$-2$}\setlength{\Height}{-0.5\Height}\setlength{\Depth}{0.5\Depth}\addtolength{\Height}{\Depth}%
\put(  3.600,  0.900){\hspace*{\Width}\raisebox{\Height}{$-2$}}%
%
\settowidth{\Width}{$-1$}\setlength{\Width}{-0.5\Width}%
\settoheight{\Height}{$-1$}\settodepth{\Depth}{$-1$}\setlength{\Height}{-0.5\Height}\setlength{\Depth}{0.5\Depth}\addtolength{\Height}{\Depth}%
\put(  4.400,  0.900){\hspace*{\Width}\raisebox{\Height}{$-1$}}%
%
\settowidth{\Width}{$0$}\setlength{\Width}{-0.5\Width}%
\settoheight{\Height}{$0$}\settodepth{\Depth}{$0$}\setlength{\Height}{-0.5\Height}\setlength{\Depth}{0.5\Depth}\addtolength{\Height}{\Depth}%
\put(  5.200,  0.900){\hspace*{\Width}\raisebox{\Height}{$0$}}%
%
\settowidth{\Width}{$1$}\setlength{\Width}{-0.5\Width}%
\settoheight{\Height}{$1$}\settodepth{\Depth}{$1$}\setlength{\Height}{-0.5\Height}\setlength{\Depth}{0.5\Depth}\addtolength{\Height}{\Depth}%
\put(  6.000,  0.900){\hspace*{\Width}\raisebox{\Height}{$1$}}%
%
\settowidth{\Width}{$2$}\setlength{\Width}{-0.5\Width}%
\settoheight{\Height}{$2$}\settodepth{\Depth}{$2$}\setlength{\Height}{-0.5\Height}\setlength{\Depth}{0.5\Depth}\addtolength{\Height}{\Depth}%
\put(  6.800,  0.900){\hspace*{\Width}\raisebox{\Height}{$2$}}%
%
\settowidth{\Width}{$3$}\setlength{\Width}{-0.5\Width}%
\settoheight{\Height}{$3$}\settodepth{\Depth}{$3$}\setlength{\Height}{-0.5\Height}\setlength{\Depth}{0.5\Depth}\addtolength{\Height}{\Depth}%
\put(  7.600,  0.900){\hspace*{\Width}\raisebox{\Height}{$3$}}%
%
\settowidth{\Width}{$4$}\setlength{\Width}{-0.5\Width}%
\settoheight{\Height}{$4$}\settodepth{\Depth}{$4$}\setlength{\Height}{-0.5\Height}\setlength{\Depth}{0.5\Depth}\addtolength{\Height}{\Depth}%
\put(  8.400,  0.900){\hspace*{\Width}\raisebox{\Height}{$4$}}%
%
\settowidth{\Width}{$5$}\setlength{\Width}{-0.5\Width}%
\settoheight{\Height}{$5$}\settodepth{\Depth}{$5$}\setlength{\Height}{-0.5\Height}\setlength{\Depth}{0.5\Depth}\addtolength{\Height}{\Depth}%
\put(  9.200,  0.900){\hspace*{\Width}\raisebox{\Height}{$5$}}%
%
\settowidth{\Width}{$y$}\setlength{\Width}{-0.5\Width}%
\settoheight{\Height}{$y$}\settodepth{\Depth}{$y$}\setlength{\Height}{-0.5\Height}\setlength{\Depth}{0.5\Depth}\addtolength{\Height}{\Depth}%
\put(  0.400,  0.300){\hspace*{\Width}\raisebox{\Height}{$y$}}%
%
\settowidth{\Width}{$-9$}\setlength{\Width}{-0.5\Width}%
\settoheight{\Height}{$-9$}\settodepth{\Depth}{$-9$}\setlength{\Height}{-0.5\Height}\setlength{\Depth}{0.5\Depth}\addtolength{\Height}{\Depth}%
\put(  1.200,  0.300){\hspace*{\Width}\raisebox{\Height}{$-9$}}%
%
\settowidth{\Width}{$-7$}\setlength{\Width}{-0.5\Width}%
\settoheight{\Height}{$-7$}\settodepth{\Depth}{$-7$}\setlength{\Height}{-0.5\Height}\setlength{\Depth}{0.5\Depth}\addtolength{\Height}{\Depth}%
\put(  2.000,  0.300){\hspace*{\Width}\raisebox{\Height}{$-7$}}%
%
\settowidth{\Width}{$-5$}\setlength{\Width}{-0.5\Width}%
\settoheight{\Height}{$-5$}\settodepth{\Depth}{$-5$}\setlength{\Height}{-0.5\Height}\setlength{\Depth}{0.5\Depth}\addtolength{\Height}{\Depth}%
\put(  2.800,  0.300){\hspace*{\Width}\raisebox{\Height}{$-5$}}%
%
\settowidth{\Width}{$-3$}\setlength{\Width}{-0.5\Width}%
\settoheight{\Height}{$-3$}\settodepth{\Depth}{$-3$}\setlength{\Height}{-0.5\Height}\setlength{\Depth}{0.5\Depth}\addtolength{\Height}{\Depth}%
\put(  3.600,  0.300){\hspace*{\Width}\raisebox{\Height}{$-3$}}%
%
\settowidth{\Width}{$-1$}\setlength{\Width}{-0.5\Width}%
\settoheight{\Height}{$-1$}\settodepth{\Depth}{$-1$}\setlength{\Height}{-0.5\Height}\setlength{\Depth}{0.5\Depth}\addtolength{\Height}{\Depth}%
\put(  4.400,  0.300){\hspace*{\Width}\raisebox{\Height}{$-1$}}%
%
\settowidth{\Width}{$1$}\setlength{\Width}{-0.5\Width}%
\settoheight{\Height}{$1$}\settodepth{\Depth}{$1$}\setlength{\Height}{-0.5\Height}\setlength{\Depth}{0.5\Depth}\addtolength{\Height}{\Depth}%
\put(  5.200,  0.300){\hspace*{\Width}\raisebox{\Height}{$1$}}%
%
\settowidth{\Width}{$3$}\setlength{\Width}{-0.5\Width}%
\settoheight{\Height}{$3$}\settodepth{\Depth}{$3$}\setlength{\Height}{-0.5\Height}\setlength{\Depth}{0.5\Depth}\addtolength{\Height}{\Depth}%
\put(  6.000,  0.300){\hspace*{\Width}\raisebox{\Height}{$3$}}%
%
\settowidth{\Width}{$5$}\setlength{\Width}{-0.5\Width}%
\settoheight{\Height}{$5$}\settodepth{\Depth}{$5$}\setlength{\Height}{-0.5\Height}\setlength{\Depth}{0.5\Depth}\addtolength{\Height}{\Depth}%
\put(  6.800,  0.300){\hspace*{\Width}\raisebox{\Height}{$5$}}%
%
\settowidth{\Width}{$7$}\setlength{\Width}{-0.5\Width}%
\settoheight{\Height}{$7$}\settodepth{\Depth}{$7$}\setlength{\Height}{-0.5\Height}\setlength{\Depth}{0.5\Depth}\addtolength{\Height}{\Depth}%
\put(  7.600,  0.300){\hspace*{\Width}\raisebox{\Height}{$7$}}%
%
\settowidth{\Width}{$9$}\setlength{\Width}{-0.5\Width}%
\settoheight{\Height}{$9$}\settodepth{\Depth}{$9$}\setlength{\Height}{-0.5\Height}\setlength{\Depth}{0.5\Depth}\addtolength{\Height}{\Depth}%
\put(  8.400,  0.300){\hspace*{\Width}\raisebox{\Height}{$9$}}%
%
\settowidth{\Width}{$11$}\setlength{\Width}{-0.5\Width}%
\settoheight{\Height}{$11$}\settodepth{\Depth}{$11$}\setlength{\Height}{-0.5\Height}\setlength{\Depth}{0.5\Depth}\addtolength{\Height}{\Depth}%
\put(  9.200,  0.300){\hspace*{\Width}\raisebox{\Height}{$11$}}%
%
\end{picture}}%}}
\putnotes{60}{6}{\scalebox{0.5}{%%% /Users/takatoosetsuo/polytech23.git/102-0424/presen/fig/graphpaper2.tex 
%%% Generator=200601.cdy 
{\unitlength=1cm%
\begin{picture}%
(12.4,12.4)(-6.2,-6.2)%
\linethickness{0.008in}%%
\Large\bf\boldmath%
\small%
\linethickness{0.006in}%%
\polyline(-6,6)(-6,-6)%
%
\linethickness{0.008in}%%
\linethickness{0.006in}%%
\polyline(-5,6)(-5,-6)%
%
\linethickness{0.008in}%%
\linethickness{0.006in}%%
\polyline(-4,6)(-4,-6)%
%
\linethickness{0.008in}%%
\linethickness{0.006in}%%
\polyline(-3,6)(-3,-6)%
%
\linethickness{0.008in}%%
\linethickness{0.006in}%%
\polyline(-2,6)(-2,-6)%
%
\linethickness{0.008in}%%
\linethickness{0.006in}%%
\polyline(-1,6)(-1,-6)%
%
\linethickness{0.008in}%%
\linethickness{0.006in}%%
\polyline(0,6)(0,-6)%
%
\linethickness{0.008in}%%
\linethickness{0.006in}%%
\polyline(1,6)(1,-6)%
%
\linethickness{0.008in}%%
\linethickness{0.006in}%%
\polyline(2,6)(2,-6)%
%
\linethickness{0.008in}%%
\linethickness{0.006in}%%
\polyline(3,6)(3,-6)%
%
\linethickness{0.008in}%%
\linethickness{0.006in}%%
\polyline(4,6)(4,-6)%
%
\linethickness{0.008in}%%
\linethickness{0.006in}%%
\polyline(5,6)(5,-6)%
%
\linethickness{0.008in}%%
\linethickness{0.006in}%%
\polyline(6,6)(6,-6)%
%
\linethickness{0.008in}%%
\linethickness{0.006in}%%
\polyline(-6,6)(6,6)%
%
\linethickness{0.008in}%%
\linethickness{0.006in}%%
\polyline(-6,5)(6,5)%
%
\linethickness{0.008in}%%
\linethickness{0.006in}%%
\polyline(-6,4)(6,4)%
%
\linethickness{0.008in}%%
\linethickness{0.006in}%%
\polyline(-6,3)(6,3)%
%
\linethickness{0.008in}%%
\linethickness{0.006in}%%
\polyline(-6,2)(6,2)%
%
\linethickness{0.008in}%%
\linethickness{0.006in}%%
\polyline(-6,1)(6,1)%
%
\linethickness{0.008in}%%
\linethickness{0.006in}%%
\polyline(-6,0)(6,0)%
%
\linethickness{0.008in}%%
\linethickness{0.006in}%%
\polyline(-6,-1)(6,-1)%
%
\linethickness{0.008in}%%
\linethickness{0.006in}%%
\polyline(-6,-2)(6,-2)%
%
\linethickness{0.008in}%%
\linethickness{0.006in}%%
\polyline(-6,-3)(6,-3)%
%
\linethickness{0.008in}%%
\linethickness{0.006in}%%
\polyline(-6,-4)(6,-4)%
%
\linethickness{0.008in}%%
\linethickness{0.006in}%%
\polyline(-6,-5)(6,-5)%
%
\linethickness{0.008in}%%
\linethickness{0.006in}%%
\polyline(-6,-6)(6,-6)%
%
\linethickness{0.008in}%%
\linethickness{0.004in}%%
\polyline(-5.5,-6)(-5.5,-5.901)\polyline(-5.5,-5.802)(-5.5,-5.702)\polyline(-5.5,-5.603)(-5.5,-5.504)%
\polyline(-5.5,-5.405)(-5.5,-5.306)\polyline(-5.5,-5.207)(-5.5,-5.107)\polyline(-5.5,-5.008)(-5.5,-4.909)%
\polyline(-5.5,-4.81)(-5.5,-4.711)\polyline(-5.5,-4.612)(-5.5,-4.512)\polyline(-5.5,-4.413)(-5.5,-4.314)%
\polyline(-5.5,-4.215)(-5.5,-4.116)\polyline(-5.5,-4.017)(-5.5,-3.917)\polyline(-5.5,-3.818)(-5.5,-3.719)%
\polyline(-5.5,-3.62)(-5.5,-3.521)\polyline(-5.5,-3.421)(-5.5,-3.322)\polyline(-5.5,-3.223)(-5.5,-3.124)%
\polyline(-5.5,-3.025)(-5.5,-2.926)\polyline(-5.5,-2.826)(-5.5,-2.727)\polyline(-5.5,-2.628)(-5.5,-2.529)%
\polyline(-5.5,-2.43)(-5.5,-2.331)\polyline(-5.5,-2.231)(-5.5,-2.132)\polyline(-5.5,-2.033)(-5.5,-1.934)%
\polyline(-5.5,-1.835)(-5.5,-1.736)\polyline(-5.5,-1.636)(-5.5,-1.537)\polyline(-5.5,-1.438)(-5.5,-1.339)%
\polyline(-5.5,-1.24)(-5.5,-1.14)\polyline(-5.5,-1.041)(-5.5,-0.942)\polyline(-5.5,-0.843)(-5.5,-0.744)%
\polyline(-5.5,-0.645)(-5.5,-0.545)\polyline(-5.5,-0.446)(-5.5,-0.347)\polyline(-5.5,-0.248)(-5.5,-0.149)%
\polyline(-5.5,-0.05)(-5.5,0.05)\polyline(-5.5,0.149)(-5.5,0.248)\polyline(-5.5,0.347)(-5.5,0.446)%
\polyline(-5.5,0.545)(-5.5,0.645)\polyline(-5.5,0.744)(-5.5,0.843)\polyline(-5.5,0.942)(-5.5,1.041)%
\polyline(-5.5,1.14)(-5.5,1.24)\polyline(-5.5,1.339)(-5.5,1.438)\polyline(-5.5,1.537)(-5.5,1.636)%
\polyline(-5.5,1.736)(-5.5,1.835)\polyline(-5.5,1.934)(-5.5,2.033)\polyline(-5.5,2.132)(-5.5,2.231)%
\polyline(-5.5,2.331)(-5.5,2.43)\polyline(-5.5,2.529)(-5.5,2.628)\polyline(-5.5,2.727)(-5.5,2.826)%
\polyline(-5.5,2.926)(-5.5,3.025)\polyline(-5.5,3.124)(-5.5,3.223)\polyline(-5.5,3.322)(-5.5,3.421)%
\polyline(-5.5,3.521)(-5.5,3.62)\polyline(-5.5,3.719)(-5.5,3.818)\polyline(-5.5,3.917)(-5.5,4.017)%
\polyline(-5.5,4.116)(-5.5,4.215)\polyline(-5.5,4.314)(-5.5,4.413)\polyline(-5.5,4.512)(-5.5,4.612)%
\polyline(-5.5,4.711)(-5.5,4.81)\polyline(-5.5,4.909)(-5.5,5.008)\polyline(-5.5,5.107)(-5.5,5.207)%
\polyline(-5.5,5.306)(-5.5,5.405)\polyline(-5.5,5.504)(-5.5,5.603)\polyline(-5.5,5.702)(-5.5,5.802)%
\polyline(-5.5,5.901)(-5.5,6)%
%
\polyline(-6,-5.5)(-5.901,-5.5)\polyline(-5.802,-5.5)(-5.702,-5.5)\polyline(-5.603,-5.5)(-5.504,-5.5)%
\polyline(-5.405,-5.5)(-5.306,-5.5)\polyline(-5.207,-5.5)(-5.107,-5.5)\polyline(-5.008,-5.5)(-4.909,-5.5)%
\polyline(-4.81,-5.5)(-4.711,-5.5)\polyline(-4.612,-5.5)(-4.512,-5.5)\polyline(-4.413,-5.5)(-4.314,-5.5)%
\polyline(-4.215,-5.5)(-4.116,-5.5)\polyline(-4.017,-5.5)(-3.917,-5.5)\polyline(-3.818,-5.5)(-3.719,-5.5)%
\polyline(-3.62,-5.5)(-3.521,-5.5)\polyline(-3.421,-5.5)(-3.322,-5.5)\polyline(-3.223,-5.5)(-3.124,-5.5)%
\polyline(-3.025,-5.5)(-2.926,-5.5)\polyline(-2.826,-5.5)(-2.727,-5.5)\polyline(-2.628,-5.5)(-2.529,-5.5)%
\polyline(-2.43,-5.5)(-2.331,-5.5)\polyline(-2.231,-5.5)(-2.132,-5.5)\polyline(-2.033,-5.5)(-1.934,-5.5)%
\polyline(-1.835,-5.5)(-1.736,-5.5)\polyline(-1.636,-5.5)(-1.537,-5.5)\polyline(-1.438,-5.5)(-1.339,-5.5)%
\polyline(-1.24,-5.5)(-1.14,-5.5)\polyline(-1.041,-5.5)(-0.942,-5.5)\polyline(-0.843,-5.5)(-0.744,-5.5)%
\polyline(-0.645,-5.5)(-0.545,-5.5)\polyline(-0.446,-5.5)(-0.347,-5.5)\polyline(-0.248,-5.5)(-0.149,-5.5)%
\polyline(-0.05,-5.5)(0.05,-5.5)\polyline(0.149,-5.5)(0.248,-5.5)\polyline(0.347,-5.5)(0.446,-5.5)%
\polyline(0.545,-5.5)(0.645,-5.5)\polyline(0.744,-5.5)(0.843,-5.5)\polyline(0.942,-5.5)(1.041,-5.5)%
\polyline(1.14,-5.5)(1.24,-5.5)\polyline(1.339,-5.5)(1.438,-5.5)\polyline(1.537,-5.5)(1.636,-5.5)%
\polyline(1.736,-5.5)(1.835,-5.5)\polyline(1.934,-5.5)(2.033,-5.5)\polyline(2.132,-5.5)(2.231,-5.5)%
\polyline(2.331,-5.5)(2.43,-5.5)\polyline(2.529,-5.5)(2.628,-5.5)\polyline(2.727,-5.5)(2.826,-5.5)%
\polyline(2.926,-5.5)(3.025,-5.5)\polyline(3.124,-5.5)(3.223,-5.5)\polyline(3.322,-5.5)(3.421,-5.5)%
\polyline(3.521,-5.5)(3.62,-5.5)\polyline(3.719,-5.5)(3.818,-5.5)\polyline(3.917,-5.5)(4.017,-5.5)%
\polyline(4.116,-5.5)(4.215,-5.5)\polyline(4.314,-5.5)(4.413,-5.5)\polyline(4.512,-5.5)(4.612,-5.5)%
\polyline(4.711,-5.5)(4.81,-5.5)\polyline(4.909,-5.5)(5.008,-5.5)\polyline(5.107,-5.5)(5.207,-5.5)%
\polyline(5.306,-5.5)(5.405,-5.5)\polyline(5.504,-5.5)(5.603,-5.5)\polyline(5.702,-5.5)(5.802,-5.5)%
\polyline(5.901,-5.5)(6,-5.5)%
%
\polyline(-4.5,-6)(-4.5,-5.901)\polyline(-4.5,-5.802)(-4.5,-5.702)\polyline(-4.5,-5.603)(-4.5,-5.504)%
\polyline(-4.5,-5.405)(-4.5,-5.306)\polyline(-4.5,-5.207)(-4.5,-5.107)\polyline(-4.5,-5.008)(-4.5,-4.909)%
\polyline(-4.5,-4.81)(-4.5,-4.711)\polyline(-4.5,-4.612)(-4.5,-4.512)\polyline(-4.5,-4.413)(-4.5,-4.314)%
\polyline(-4.5,-4.215)(-4.5,-4.116)\polyline(-4.5,-4.017)(-4.5,-3.917)\polyline(-4.5,-3.818)(-4.5,-3.719)%
\polyline(-4.5,-3.62)(-4.5,-3.521)\polyline(-4.5,-3.421)(-4.5,-3.322)\polyline(-4.5,-3.223)(-4.5,-3.124)%
\polyline(-4.5,-3.025)(-4.5,-2.926)\polyline(-4.5,-2.826)(-4.5,-2.727)\polyline(-4.5,-2.628)(-4.5,-2.529)%
\polyline(-4.5,-2.43)(-4.5,-2.331)\polyline(-4.5,-2.231)(-4.5,-2.132)\polyline(-4.5,-2.033)(-4.5,-1.934)%
\polyline(-4.5,-1.835)(-4.5,-1.736)\polyline(-4.5,-1.636)(-4.5,-1.537)\polyline(-4.5,-1.438)(-4.5,-1.339)%
\polyline(-4.5,-1.24)(-4.5,-1.14)\polyline(-4.5,-1.041)(-4.5,-0.942)\polyline(-4.5,-0.843)(-4.5,-0.744)%
\polyline(-4.5,-0.645)(-4.5,-0.545)\polyline(-4.5,-0.446)(-4.5,-0.347)\polyline(-4.5,-0.248)(-4.5,-0.149)%
\polyline(-4.5,-0.05)(-4.5,0.05)\polyline(-4.5,0.149)(-4.5,0.248)\polyline(-4.5,0.347)(-4.5,0.446)%
\polyline(-4.5,0.545)(-4.5,0.645)\polyline(-4.5,0.744)(-4.5,0.843)\polyline(-4.5,0.942)(-4.5,1.041)%
\polyline(-4.5,1.14)(-4.5,1.24)\polyline(-4.5,1.339)(-4.5,1.438)\polyline(-4.5,1.537)(-4.5,1.636)%
\polyline(-4.5,1.736)(-4.5,1.835)\polyline(-4.5,1.934)(-4.5,2.033)\polyline(-4.5,2.132)(-4.5,2.231)%
\polyline(-4.5,2.331)(-4.5,2.43)\polyline(-4.5,2.529)(-4.5,2.628)\polyline(-4.5,2.727)(-4.5,2.826)%
\polyline(-4.5,2.926)(-4.5,3.025)\polyline(-4.5,3.124)(-4.5,3.223)\polyline(-4.5,3.322)(-4.5,3.421)%
\polyline(-4.5,3.521)(-4.5,3.62)\polyline(-4.5,3.719)(-4.5,3.818)\polyline(-4.5,3.917)(-4.5,4.017)%
\polyline(-4.5,4.116)(-4.5,4.215)\polyline(-4.5,4.314)(-4.5,4.413)\polyline(-4.5,4.512)(-4.5,4.612)%
\polyline(-4.5,4.711)(-4.5,4.81)\polyline(-4.5,4.909)(-4.5,5.008)\polyline(-4.5,5.107)(-4.5,5.207)%
\polyline(-4.5,5.306)(-4.5,5.405)\polyline(-4.5,5.504)(-4.5,5.603)\polyline(-4.5,5.702)(-4.5,5.802)%
\polyline(-4.5,5.901)(-4.5,6)%
%
\polyline(-6,-4.5)(-5.901,-4.5)\polyline(-5.802,-4.5)(-5.702,-4.5)\polyline(-5.603,-4.5)(-5.504,-4.5)%
\polyline(-5.405,-4.5)(-5.306,-4.5)\polyline(-5.207,-4.5)(-5.107,-4.5)\polyline(-5.008,-4.5)(-4.909,-4.5)%
\polyline(-4.81,-4.5)(-4.711,-4.5)\polyline(-4.612,-4.5)(-4.512,-4.5)\polyline(-4.413,-4.5)(-4.314,-4.5)%
\polyline(-4.215,-4.5)(-4.116,-4.5)\polyline(-4.017,-4.5)(-3.917,-4.5)\polyline(-3.818,-4.5)(-3.719,-4.5)%
\polyline(-3.62,-4.5)(-3.521,-4.5)\polyline(-3.421,-4.5)(-3.322,-4.5)\polyline(-3.223,-4.5)(-3.124,-4.5)%
\polyline(-3.025,-4.5)(-2.926,-4.5)\polyline(-2.826,-4.5)(-2.727,-4.5)\polyline(-2.628,-4.5)(-2.529,-4.5)%
\polyline(-2.43,-4.5)(-2.331,-4.5)\polyline(-2.231,-4.5)(-2.132,-4.5)\polyline(-2.033,-4.5)(-1.934,-4.5)%
\polyline(-1.835,-4.5)(-1.736,-4.5)\polyline(-1.636,-4.5)(-1.537,-4.5)\polyline(-1.438,-4.5)(-1.339,-4.5)%
\polyline(-1.24,-4.5)(-1.14,-4.5)\polyline(-1.041,-4.5)(-0.942,-4.5)\polyline(-0.843,-4.5)(-0.744,-4.5)%
\polyline(-0.645,-4.5)(-0.545,-4.5)\polyline(-0.446,-4.5)(-0.347,-4.5)\polyline(-0.248,-4.5)(-0.149,-4.5)%
\polyline(-0.05,-4.5)(0.05,-4.5)\polyline(0.149,-4.5)(0.248,-4.5)\polyline(0.347,-4.5)(0.446,-4.5)%
\polyline(0.545,-4.5)(0.645,-4.5)\polyline(0.744,-4.5)(0.843,-4.5)\polyline(0.942,-4.5)(1.041,-4.5)%
\polyline(1.14,-4.5)(1.24,-4.5)\polyline(1.339,-4.5)(1.438,-4.5)\polyline(1.537,-4.5)(1.636,-4.5)%
\polyline(1.736,-4.5)(1.835,-4.5)\polyline(1.934,-4.5)(2.033,-4.5)\polyline(2.132,-4.5)(2.231,-4.5)%
\polyline(2.331,-4.5)(2.43,-4.5)\polyline(2.529,-4.5)(2.628,-4.5)\polyline(2.727,-4.5)(2.826,-4.5)%
\polyline(2.926,-4.5)(3.025,-4.5)\polyline(3.124,-4.5)(3.223,-4.5)\polyline(3.322,-4.5)(3.421,-4.5)%
\polyline(3.521,-4.5)(3.62,-4.5)\polyline(3.719,-4.5)(3.818,-4.5)\polyline(3.917,-4.5)(4.017,-4.5)%
\polyline(4.116,-4.5)(4.215,-4.5)\polyline(4.314,-4.5)(4.413,-4.5)\polyline(4.512,-4.5)(4.612,-4.5)%
\polyline(4.711,-4.5)(4.81,-4.5)\polyline(4.909,-4.5)(5.008,-4.5)\polyline(5.107,-4.5)(5.207,-4.5)%
\polyline(5.306,-4.5)(5.405,-4.5)\polyline(5.504,-4.5)(5.603,-4.5)\polyline(5.702,-4.5)(5.802,-4.5)%
\polyline(5.901,-4.5)(6,-4.5)%
%
\polyline(-3.5,-6)(-3.5,-5.901)\polyline(-3.5,-5.802)(-3.5,-5.702)\polyline(-3.5,-5.603)(-3.5,-5.504)%
\polyline(-3.5,-5.405)(-3.5,-5.306)\polyline(-3.5,-5.207)(-3.5,-5.107)\polyline(-3.5,-5.008)(-3.5,-4.909)%
\polyline(-3.5,-4.81)(-3.5,-4.711)\polyline(-3.5,-4.612)(-3.5,-4.512)\polyline(-3.5,-4.413)(-3.5,-4.314)%
\polyline(-3.5,-4.215)(-3.5,-4.116)\polyline(-3.5,-4.017)(-3.5,-3.917)\polyline(-3.5,-3.818)(-3.5,-3.719)%
\polyline(-3.5,-3.62)(-3.5,-3.521)\polyline(-3.5,-3.421)(-3.5,-3.322)\polyline(-3.5,-3.223)(-3.5,-3.124)%
\polyline(-3.5,-3.025)(-3.5,-2.926)\polyline(-3.5,-2.826)(-3.5,-2.727)\polyline(-3.5,-2.628)(-3.5,-2.529)%
\polyline(-3.5,-2.43)(-3.5,-2.331)\polyline(-3.5,-2.231)(-3.5,-2.132)\polyline(-3.5,-2.033)(-3.5,-1.934)%
\polyline(-3.5,-1.835)(-3.5,-1.736)\polyline(-3.5,-1.636)(-3.5,-1.537)\polyline(-3.5,-1.438)(-3.5,-1.339)%
\polyline(-3.5,-1.24)(-3.5,-1.14)\polyline(-3.5,-1.041)(-3.5,-0.942)\polyline(-3.5,-0.843)(-3.5,-0.744)%
\polyline(-3.5,-0.645)(-3.5,-0.545)\polyline(-3.5,-0.446)(-3.5,-0.347)\polyline(-3.5,-0.248)(-3.5,-0.149)%
\polyline(-3.5,-0.05)(-3.5,0.05)\polyline(-3.5,0.149)(-3.5,0.248)\polyline(-3.5,0.347)(-3.5,0.446)%
\polyline(-3.5,0.545)(-3.5,0.645)\polyline(-3.5,0.744)(-3.5,0.843)\polyline(-3.5,0.942)(-3.5,1.041)%
\polyline(-3.5,1.14)(-3.5,1.24)\polyline(-3.5,1.339)(-3.5,1.438)\polyline(-3.5,1.537)(-3.5,1.636)%
\polyline(-3.5,1.736)(-3.5,1.835)\polyline(-3.5,1.934)(-3.5,2.033)\polyline(-3.5,2.132)(-3.5,2.231)%
\polyline(-3.5,2.331)(-3.5,2.43)\polyline(-3.5,2.529)(-3.5,2.628)\polyline(-3.5,2.727)(-3.5,2.826)%
\polyline(-3.5,2.926)(-3.5,3.025)\polyline(-3.5,3.124)(-3.5,3.223)\polyline(-3.5,3.322)(-3.5,3.421)%
\polyline(-3.5,3.521)(-3.5,3.62)\polyline(-3.5,3.719)(-3.5,3.818)\polyline(-3.5,3.917)(-3.5,4.017)%
\polyline(-3.5,4.116)(-3.5,4.215)\polyline(-3.5,4.314)(-3.5,4.413)\polyline(-3.5,4.512)(-3.5,4.612)%
\polyline(-3.5,4.711)(-3.5,4.81)\polyline(-3.5,4.909)(-3.5,5.008)\polyline(-3.5,5.107)(-3.5,5.207)%
\polyline(-3.5,5.306)(-3.5,5.405)\polyline(-3.5,5.504)(-3.5,5.603)\polyline(-3.5,5.702)(-3.5,5.802)%
\polyline(-3.5,5.901)(-3.5,6)%
%
\polyline(-6,-3.5)(-5.901,-3.5)\polyline(-5.802,-3.5)(-5.702,-3.5)\polyline(-5.603,-3.5)(-5.504,-3.5)%
\polyline(-5.405,-3.5)(-5.306,-3.5)\polyline(-5.207,-3.5)(-5.107,-3.5)\polyline(-5.008,-3.5)(-4.909,-3.5)%
\polyline(-4.81,-3.5)(-4.711,-3.5)\polyline(-4.612,-3.5)(-4.512,-3.5)\polyline(-4.413,-3.5)(-4.314,-3.5)%
\polyline(-4.215,-3.5)(-4.116,-3.5)\polyline(-4.017,-3.5)(-3.917,-3.5)\polyline(-3.818,-3.5)(-3.719,-3.5)%
\polyline(-3.62,-3.5)(-3.521,-3.5)\polyline(-3.421,-3.5)(-3.322,-3.5)\polyline(-3.223,-3.5)(-3.124,-3.5)%
\polyline(-3.025,-3.5)(-2.926,-3.5)\polyline(-2.826,-3.5)(-2.727,-3.5)\polyline(-2.628,-3.5)(-2.529,-3.5)%
\polyline(-2.43,-3.5)(-2.331,-3.5)\polyline(-2.231,-3.5)(-2.132,-3.5)\polyline(-2.033,-3.5)(-1.934,-3.5)%
\polyline(-1.835,-3.5)(-1.736,-3.5)\polyline(-1.636,-3.5)(-1.537,-3.5)\polyline(-1.438,-3.5)(-1.339,-3.5)%
\polyline(-1.24,-3.5)(-1.14,-3.5)\polyline(-1.041,-3.5)(-0.942,-3.5)\polyline(-0.843,-3.5)(-0.744,-3.5)%
\polyline(-0.645,-3.5)(-0.545,-3.5)\polyline(-0.446,-3.5)(-0.347,-3.5)\polyline(-0.248,-3.5)(-0.149,-3.5)%
\polyline(-0.05,-3.5)(0.05,-3.5)\polyline(0.149,-3.5)(0.248,-3.5)\polyline(0.347,-3.5)(0.446,-3.5)%
\polyline(0.545,-3.5)(0.645,-3.5)\polyline(0.744,-3.5)(0.843,-3.5)\polyline(0.942,-3.5)(1.041,-3.5)%
\polyline(1.14,-3.5)(1.24,-3.5)\polyline(1.339,-3.5)(1.438,-3.5)\polyline(1.537,-3.5)(1.636,-3.5)%
\polyline(1.736,-3.5)(1.835,-3.5)\polyline(1.934,-3.5)(2.033,-3.5)\polyline(2.132,-3.5)(2.231,-3.5)%
\polyline(2.331,-3.5)(2.43,-3.5)\polyline(2.529,-3.5)(2.628,-3.5)\polyline(2.727,-3.5)(2.826,-3.5)%
\polyline(2.926,-3.5)(3.025,-3.5)\polyline(3.124,-3.5)(3.223,-3.5)\polyline(3.322,-3.5)(3.421,-3.5)%
\polyline(3.521,-3.5)(3.62,-3.5)\polyline(3.719,-3.5)(3.818,-3.5)\polyline(3.917,-3.5)(4.017,-3.5)%
\polyline(4.116,-3.5)(4.215,-3.5)\polyline(4.314,-3.5)(4.413,-3.5)\polyline(4.512,-3.5)(4.612,-3.5)%
\polyline(4.711,-3.5)(4.81,-3.5)\polyline(4.909,-3.5)(5.008,-3.5)\polyline(5.107,-3.5)(5.207,-3.5)%
\polyline(5.306,-3.5)(5.405,-3.5)\polyline(5.504,-3.5)(5.603,-3.5)\polyline(5.702,-3.5)(5.802,-3.5)%
\polyline(5.901,-3.5)(6,-3.5)%
%
\polyline(-2.5,-6)(-2.5,-5.901)\polyline(-2.5,-5.802)(-2.5,-5.702)\polyline(-2.5,-5.603)(-2.5,-5.504)%
\polyline(-2.5,-5.405)(-2.5,-5.306)\polyline(-2.5,-5.207)(-2.5,-5.107)\polyline(-2.5,-5.008)(-2.5,-4.909)%
\polyline(-2.5,-4.81)(-2.5,-4.711)\polyline(-2.5,-4.612)(-2.5,-4.512)\polyline(-2.5,-4.413)(-2.5,-4.314)%
\polyline(-2.5,-4.215)(-2.5,-4.116)\polyline(-2.5,-4.017)(-2.5,-3.917)\polyline(-2.5,-3.818)(-2.5,-3.719)%
\polyline(-2.5,-3.62)(-2.5,-3.521)\polyline(-2.5,-3.421)(-2.5,-3.322)\polyline(-2.5,-3.223)(-2.5,-3.124)%
\polyline(-2.5,-3.025)(-2.5,-2.926)\polyline(-2.5,-2.826)(-2.5,-2.727)\polyline(-2.5,-2.628)(-2.5,-2.529)%
\polyline(-2.5,-2.43)(-2.5,-2.331)\polyline(-2.5,-2.231)(-2.5,-2.132)\polyline(-2.5,-2.033)(-2.5,-1.934)%
\polyline(-2.5,-1.835)(-2.5,-1.736)\polyline(-2.5,-1.636)(-2.5,-1.537)\polyline(-2.5,-1.438)(-2.5,-1.339)%
\polyline(-2.5,-1.24)(-2.5,-1.14)\polyline(-2.5,-1.041)(-2.5,-0.942)\polyline(-2.5,-0.843)(-2.5,-0.744)%
\polyline(-2.5,-0.645)(-2.5,-0.545)\polyline(-2.5,-0.446)(-2.5,-0.347)\polyline(-2.5,-0.248)(-2.5,-0.149)%
\polyline(-2.5,-0.05)(-2.5,0.05)\polyline(-2.5,0.149)(-2.5,0.248)\polyline(-2.5,0.347)(-2.5,0.446)%
\polyline(-2.5,0.545)(-2.5,0.645)\polyline(-2.5,0.744)(-2.5,0.843)\polyline(-2.5,0.942)(-2.5,1.041)%
\polyline(-2.5,1.14)(-2.5,1.24)\polyline(-2.5,1.339)(-2.5,1.438)\polyline(-2.5,1.537)(-2.5,1.636)%
\polyline(-2.5,1.736)(-2.5,1.835)\polyline(-2.5,1.934)(-2.5,2.033)\polyline(-2.5,2.132)(-2.5,2.231)%
\polyline(-2.5,2.331)(-2.5,2.43)\polyline(-2.5,2.529)(-2.5,2.628)\polyline(-2.5,2.727)(-2.5,2.826)%
\polyline(-2.5,2.926)(-2.5,3.025)\polyline(-2.5,3.124)(-2.5,3.223)\polyline(-2.5,3.322)(-2.5,3.421)%
\polyline(-2.5,3.521)(-2.5,3.62)\polyline(-2.5,3.719)(-2.5,3.818)\polyline(-2.5,3.917)(-2.5,4.017)%
\polyline(-2.5,4.116)(-2.5,4.215)\polyline(-2.5,4.314)(-2.5,4.413)\polyline(-2.5,4.512)(-2.5,4.612)%
\polyline(-2.5,4.711)(-2.5,4.81)\polyline(-2.5,4.909)(-2.5,5.008)\polyline(-2.5,5.107)(-2.5,5.207)%
\polyline(-2.5,5.306)(-2.5,5.405)\polyline(-2.5,5.504)(-2.5,5.603)\polyline(-2.5,5.702)(-2.5,5.802)%
\polyline(-2.5,5.901)(-2.5,6)%
%
\polyline(-6,-2.5)(-5.901,-2.5)\polyline(-5.802,-2.5)(-5.702,-2.5)\polyline(-5.603,-2.5)(-5.504,-2.5)%
\polyline(-5.405,-2.5)(-5.306,-2.5)\polyline(-5.207,-2.5)(-5.107,-2.5)\polyline(-5.008,-2.5)(-4.909,-2.5)%
\polyline(-4.81,-2.5)(-4.711,-2.5)\polyline(-4.612,-2.5)(-4.512,-2.5)\polyline(-4.413,-2.5)(-4.314,-2.5)%
\polyline(-4.215,-2.5)(-4.116,-2.5)\polyline(-4.017,-2.5)(-3.917,-2.5)\polyline(-3.818,-2.5)(-3.719,-2.5)%
\polyline(-3.62,-2.5)(-3.521,-2.5)\polyline(-3.421,-2.5)(-3.322,-2.5)\polyline(-3.223,-2.5)(-3.124,-2.5)%
\polyline(-3.025,-2.5)(-2.926,-2.5)\polyline(-2.826,-2.5)(-2.727,-2.5)\polyline(-2.628,-2.5)(-2.529,-2.5)%
\polyline(-2.43,-2.5)(-2.331,-2.5)\polyline(-2.231,-2.5)(-2.132,-2.5)\polyline(-2.033,-2.5)(-1.934,-2.5)%
\polyline(-1.835,-2.5)(-1.736,-2.5)\polyline(-1.636,-2.5)(-1.537,-2.5)\polyline(-1.438,-2.5)(-1.339,-2.5)%
\polyline(-1.24,-2.5)(-1.14,-2.5)\polyline(-1.041,-2.5)(-0.942,-2.5)\polyline(-0.843,-2.5)(-0.744,-2.5)%
\polyline(-0.645,-2.5)(-0.545,-2.5)\polyline(-0.446,-2.5)(-0.347,-2.5)\polyline(-0.248,-2.5)(-0.149,-2.5)%
\polyline(-0.05,-2.5)(0.05,-2.5)\polyline(0.149,-2.5)(0.248,-2.5)\polyline(0.347,-2.5)(0.446,-2.5)%
\polyline(0.545,-2.5)(0.645,-2.5)\polyline(0.744,-2.5)(0.843,-2.5)\polyline(0.942,-2.5)(1.041,-2.5)%
\polyline(1.14,-2.5)(1.24,-2.5)\polyline(1.339,-2.5)(1.438,-2.5)\polyline(1.537,-2.5)(1.636,-2.5)%
\polyline(1.736,-2.5)(1.835,-2.5)\polyline(1.934,-2.5)(2.033,-2.5)\polyline(2.132,-2.5)(2.231,-2.5)%
\polyline(2.331,-2.5)(2.43,-2.5)\polyline(2.529,-2.5)(2.628,-2.5)\polyline(2.727,-2.5)(2.826,-2.5)%
\polyline(2.926,-2.5)(3.025,-2.5)\polyline(3.124,-2.5)(3.223,-2.5)\polyline(3.322,-2.5)(3.421,-2.5)%
\polyline(3.521,-2.5)(3.62,-2.5)\polyline(3.719,-2.5)(3.818,-2.5)\polyline(3.917,-2.5)(4.017,-2.5)%
\polyline(4.116,-2.5)(4.215,-2.5)\polyline(4.314,-2.5)(4.413,-2.5)\polyline(4.512,-2.5)(4.612,-2.5)%
\polyline(4.711,-2.5)(4.81,-2.5)\polyline(4.909,-2.5)(5.008,-2.5)\polyline(5.107,-2.5)(5.207,-2.5)%
\polyline(5.306,-2.5)(5.405,-2.5)\polyline(5.504,-2.5)(5.603,-2.5)\polyline(5.702,-2.5)(5.802,-2.5)%
\polyline(5.901,-2.5)(6,-2.5)%
%
\polyline(-1.5,-6)(-1.5,-5.901)\polyline(-1.5,-5.802)(-1.5,-5.702)\polyline(-1.5,-5.603)(-1.5,-5.504)%
\polyline(-1.5,-5.405)(-1.5,-5.306)\polyline(-1.5,-5.207)(-1.5,-5.107)\polyline(-1.5,-5.008)(-1.5,-4.909)%
\polyline(-1.5,-4.81)(-1.5,-4.711)\polyline(-1.5,-4.612)(-1.5,-4.512)\polyline(-1.5,-4.413)(-1.5,-4.314)%
\polyline(-1.5,-4.215)(-1.5,-4.116)\polyline(-1.5,-4.017)(-1.5,-3.917)\polyline(-1.5,-3.818)(-1.5,-3.719)%
\polyline(-1.5,-3.62)(-1.5,-3.521)\polyline(-1.5,-3.421)(-1.5,-3.322)\polyline(-1.5,-3.223)(-1.5,-3.124)%
\polyline(-1.5,-3.025)(-1.5,-2.926)\polyline(-1.5,-2.826)(-1.5,-2.727)\polyline(-1.5,-2.628)(-1.5,-2.529)%
\polyline(-1.5,-2.43)(-1.5,-2.331)\polyline(-1.5,-2.231)(-1.5,-2.132)\polyline(-1.5,-2.033)(-1.5,-1.934)%
\polyline(-1.5,-1.835)(-1.5,-1.736)\polyline(-1.5,-1.636)(-1.5,-1.537)\polyline(-1.5,-1.438)(-1.5,-1.339)%
\polyline(-1.5,-1.24)(-1.5,-1.14)\polyline(-1.5,-1.041)(-1.5,-0.942)\polyline(-1.5,-0.843)(-1.5,-0.744)%
\polyline(-1.5,-0.645)(-1.5,-0.545)\polyline(-1.5,-0.446)(-1.5,-0.347)\polyline(-1.5,-0.248)(-1.5,-0.149)%
\polyline(-1.5,-0.05)(-1.5,0.05)\polyline(-1.5,0.149)(-1.5,0.248)\polyline(-1.5,0.347)(-1.5,0.446)%
\polyline(-1.5,0.545)(-1.5,0.645)\polyline(-1.5,0.744)(-1.5,0.843)\polyline(-1.5,0.942)(-1.5,1.041)%
\polyline(-1.5,1.14)(-1.5,1.24)\polyline(-1.5,1.339)(-1.5,1.438)\polyline(-1.5,1.537)(-1.5,1.636)%
\polyline(-1.5,1.736)(-1.5,1.835)\polyline(-1.5,1.934)(-1.5,2.033)\polyline(-1.5,2.132)(-1.5,2.231)%
\polyline(-1.5,2.331)(-1.5,2.43)\polyline(-1.5,2.529)(-1.5,2.628)\polyline(-1.5,2.727)(-1.5,2.826)%
\polyline(-1.5,2.926)(-1.5,3.025)\polyline(-1.5,3.124)(-1.5,3.223)\polyline(-1.5,3.322)(-1.5,3.421)%
\polyline(-1.5,3.521)(-1.5,3.62)\polyline(-1.5,3.719)(-1.5,3.818)\polyline(-1.5,3.917)(-1.5,4.017)%
\polyline(-1.5,4.116)(-1.5,4.215)\polyline(-1.5,4.314)(-1.5,4.413)\polyline(-1.5,4.512)(-1.5,4.612)%
\polyline(-1.5,4.711)(-1.5,4.81)\polyline(-1.5,4.909)(-1.5,5.008)\polyline(-1.5,5.107)(-1.5,5.207)%
\polyline(-1.5,5.306)(-1.5,5.405)\polyline(-1.5,5.504)(-1.5,5.603)\polyline(-1.5,5.702)(-1.5,5.802)%
\polyline(-1.5,5.901)(-1.5,6)%
%
\polyline(-6,-1.5)(-5.901,-1.5)\polyline(-5.802,-1.5)(-5.702,-1.5)\polyline(-5.603,-1.5)(-5.504,-1.5)%
\polyline(-5.405,-1.5)(-5.306,-1.5)\polyline(-5.207,-1.5)(-5.107,-1.5)\polyline(-5.008,-1.5)(-4.909,-1.5)%
\polyline(-4.81,-1.5)(-4.711,-1.5)\polyline(-4.612,-1.5)(-4.512,-1.5)\polyline(-4.413,-1.5)(-4.314,-1.5)%
\polyline(-4.215,-1.5)(-4.116,-1.5)\polyline(-4.017,-1.5)(-3.917,-1.5)\polyline(-3.818,-1.5)(-3.719,-1.5)%
\polyline(-3.62,-1.5)(-3.521,-1.5)\polyline(-3.421,-1.5)(-3.322,-1.5)\polyline(-3.223,-1.5)(-3.124,-1.5)%
\polyline(-3.025,-1.5)(-2.926,-1.5)\polyline(-2.826,-1.5)(-2.727,-1.5)\polyline(-2.628,-1.5)(-2.529,-1.5)%
\polyline(-2.43,-1.5)(-2.331,-1.5)\polyline(-2.231,-1.5)(-2.132,-1.5)\polyline(-2.033,-1.5)(-1.934,-1.5)%
\polyline(-1.835,-1.5)(-1.736,-1.5)\polyline(-1.636,-1.5)(-1.537,-1.5)\polyline(-1.438,-1.5)(-1.339,-1.5)%
\polyline(-1.24,-1.5)(-1.14,-1.5)\polyline(-1.041,-1.5)(-0.942,-1.5)\polyline(-0.843,-1.5)(-0.744,-1.5)%
\polyline(-0.645,-1.5)(-0.545,-1.5)\polyline(-0.446,-1.5)(-0.347,-1.5)\polyline(-0.248,-1.5)(-0.149,-1.5)%
\polyline(-0.05,-1.5)(0.05,-1.5)\polyline(0.149,-1.5)(0.248,-1.5)\polyline(0.347,-1.5)(0.446,-1.5)%
\polyline(0.545,-1.5)(0.645,-1.5)\polyline(0.744,-1.5)(0.843,-1.5)\polyline(0.942,-1.5)(1.041,-1.5)%
\polyline(1.14,-1.5)(1.24,-1.5)\polyline(1.339,-1.5)(1.438,-1.5)\polyline(1.537,-1.5)(1.636,-1.5)%
\polyline(1.736,-1.5)(1.835,-1.5)\polyline(1.934,-1.5)(2.033,-1.5)\polyline(2.132,-1.5)(2.231,-1.5)%
\polyline(2.331,-1.5)(2.43,-1.5)\polyline(2.529,-1.5)(2.628,-1.5)\polyline(2.727,-1.5)(2.826,-1.5)%
\polyline(2.926,-1.5)(3.025,-1.5)\polyline(3.124,-1.5)(3.223,-1.5)\polyline(3.322,-1.5)(3.421,-1.5)%
\polyline(3.521,-1.5)(3.62,-1.5)\polyline(3.719,-1.5)(3.818,-1.5)\polyline(3.917,-1.5)(4.017,-1.5)%
\polyline(4.116,-1.5)(4.215,-1.5)\polyline(4.314,-1.5)(4.413,-1.5)\polyline(4.512,-1.5)(4.612,-1.5)%
\polyline(4.711,-1.5)(4.81,-1.5)\polyline(4.909,-1.5)(5.008,-1.5)\polyline(5.107,-1.5)(5.207,-1.5)%
\polyline(5.306,-1.5)(5.405,-1.5)\polyline(5.504,-1.5)(5.603,-1.5)\polyline(5.702,-1.5)(5.802,-1.5)%
\polyline(5.901,-1.5)(6,-1.5)%
%
\polyline(-0.5,-6)(-0.5,-5.901)\polyline(-0.5,-5.802)(-0.5,-5.702)\polyline(-0.5,-5.603)(-0.5,-5.504)%
\polyline(-0.5,-5.405)(-0.5,-5.306)\polyline(-0.5,-5.207)(-0.5,-5.107)\polyline(-0.5,-5.008)(-0.5,-4.909)%
\polyline(-0.5,-4.81)(-0.5,-4.711)\polyline(-0.5,-4.612)(-0.5,-4.512)\polyline(-0.5,-4.413)(-0.5,-4.314)%
\polyline(-0.5,-4.215)(-0.5,-4.116)\polyline(-0.5,-4.017)(-0.5,-3.917)\polyline(-0.5,-3.818)(-0.5,-3.719)%
\polyline(-0.5,-3.62)(-0.5,-3.521)\polyline(-0.5,-3.421)(-0.5,-3.322)\polyline(-0.5,-3.223)(-0.5,-3.124)%
\polyline(-0.5,-3.025)(-0.5,-2.926)\polyline(-0.5,-2.826)(-0.5,-2.727)\polyline(-0.5,-2.628)(-0.5,-2.529)%
\polyline(-0.5,-2.43)(-0.5,-2.331)\polyline(-0.5,-2.231)(-0.5,-2.132)\polyline(-0.5,-2.033)(-0.5,-1.934)%
\polyline(-0.5,-1.835)(-0.5,-1.736)\polyline(-0.5,-1.636)(-0.5,-1.537)\polyline(-0.5,-1.438)(-0.5,-1.339)%
\polyline(-0.5,-1.24)(-0.5,-1.14)\polyline(-0.5,-1.041)(-0.5,-0.942)\polyline(-0.5,-0.843)(-0.5,-0.744)%
\polyline(-0.5,-0.645)(-0.5,-0.545)\polyline(-0.5,-0.446)(-0.5,-0.347)\polyline(-0.5,-0.248)(-0.5,-0.149)%
\polyline(-0.5,-0.05)(-0.5,0.05)\polyline(-0.5,0.149)(-0.5,0.248)\polyline(-0.5,0.347)(-0.5,0.446)%
\polyline(-0.5,0.545)(-0.5,0.645)\polyline(-0.5,0.744)(-0.5,0.843)\polyline(-0.5,0.942)(-0.5,1.041)%
\polyline(-0.5,1.14)(-0.5,1.24)\polyline(-0.5,1.339)(-0.5,1.438)\polyline(-0.5,1.537)(-0.5,1.636)%
\polyline(-0.5,1.736)(-0.5,1.835)\polyline(-0.5,1.934)(-0.5,2.033)\polyline(-0.5,2.132)(-0.5,2.231)%
\polyline(-0.5,2.331)(-0.5,2.43)\polyline(-0.5,2.529)(-0.5,2.628)\polyline(-0.5,2.727)(-0.5,2.826)%
\polyline(-0.5,2.926)(-0.5,3.025)\polyline(-0.5,3.124)(-0.5,3.223)\polyline(-0.5,3.322)(-0.5,3.421)%
\polyline(-0.5,3.521)(-0.5,3.62)\polyline(-0.5,3.719)(-0.5,3.818)\polyline(-0.5,3.917)(-0.5,4.017)%
\polyline(-0.5,4.116)(-0.5,4.215)\polyline(-0.5,4.314)(-0.5,4.413)\polyline(-0.5,4.512)(-0.5,4.612)%
\polyline(-0.5,4.711)(-0.5,4.81)\polyline(-0.5,4.909)(-0.5,5.008)\polyline(-0.5,5.107)(-0.5,5.207)%
\polyline(-0.5,5.306)(-0.5,5.405)\polyline(-0.5,5.504)(-0.5,5.603)\polyline(-0.5,5.702)(-0.5,5.802)%
\polyline(-0.5,5.901)(-0.5,6)%
%
\polyline(-6,-0.5)(-5.901,-0.5)\polyline(-5.802,-0.5)(-5.702,-0.5)\polyline(-5.603,-0.5)(-5.504,-0.5)%
\polyline(-5.405,-0.5)(-5.306,-0.5)\polyline(-5.207,-0.5)(-5.107,-0.5)\polyline(-5.008,-0.5)(-4.909,-0.5)%
\polyline(-4.81,-0.5)(-4.711,-0.5)\polyline(-4.612,-0.5)(-4.512,-0.5)\polyline(-4.413,-0.5)(-4.314,-0.5)%
\polyline(-4.215,-0.5)(-4.116,-0.5)\polyline(-4.017,-0.5)(-3.917,-0.5)\polyline(-3.818,-0.5)(-3.719,-0.5)%
\polyline(-3.62,-0.5)(-3.521,-0.5)\polyline(-3.421,-0.5)(-3.322,-0.5)\polyline(-3.223,-0.5)(-3.124,-0.5)%
\polyline(-3.025,-0.5)(-2.926,-0.5)\polyline(-2.826,-0.5)(-2.727,-0.5)\polyline(-2.628,-0.5)(-2.529,-0.5)%
\polyline(-2.43,-0.5)(-2.331,-0.5)\polyline(-2.231,-0.5)(-2.132,-0.5)\polyline(-2.033,-0.5)(-1.934,-0.5)%
\polyline(-1.835,-0.5)(-1.736,-0.5)\polyline(-1.636,-0.5)(-1.537,-0.5)\polyline(-1.438,-0.5)(-1.339,-0.5)%
\polyline(-1.24,-0.5)(-1.14,-0.5)\polyline(-1.041,-0.5)(-0.942,-0.5)\polyline(-0.843,-0.5)(-0.744,-0.5)%
\polyline(-0.645,-0.5)(-0.545,-0.5)\polyline(-0.446,-0.5)(-0.347,-0.5)\polyline(-0.248,-0.5)(-0.149,-0.5)%
\polyline(-0.05,-0.5)(0.05,-0.5)\polyline(0.149,-0.5)(0.248,-0.5)\polyline(0.347,-0.5)(0.446,-0.5)%
\polyline(0.545,-0.5)(0.645,-0.5)\polyline(0.744,-0.5)(0.843,-0.5)\polyline(0.942,-0.5)(1.041,-0.5)%
\polyline(1.14,-0.5)(1.24,-0.5)\polyline(1.339,-0.5)(1.438,-0.5)\polyline(1.537,-0.5)(1.636,-0.5)%
\polyline(1.736,-0.5)(1.835,-0.5)\polyline(1.934,-0.5)(2.033,-0.5)\polyline(2.132,-0.5)(2.231,-0.5)%
\polyline(2.331,-0.5)(2.43,-0.5)\polyline(2.529,-0.5)(2.628,-0.5)\polyline(2.727,-0.5)(2.826,-0.5)%
\polyline(2.926,-0.5)(3.025,-0.5)\polyline(3.124,-0.5)(3.223,-0.5)\polyline(3.322,-0.5)(3.421,-0.5)%
\polyline(3.521,-0.5)(3.62,-0.5)\polyline(3.719,-0.5)(3.818,-0.5)\polyline(3.917,-0.5)(4.017,-0.5)%
\polyline(4.116,-0.5)(4.215,-0.5)\polyline(4.314,-0.5)(4.413,-0.5)\polyline(4.512,-0.5)(4.612,-0.5)%
\polyline(4.711,-0.5)(4.81,-0.5)\polyline(4.909,-0.5)(5.008,-0.5)\polyline(5.107,-0.5)(5.207,-0.5)%
\polyline(5.306,-0.5)(5.405,-0.5)\polyline(5.504,-0.5)(5.603,-0.5)\polyline(5.702,-0.5)(5.802,-0.5)%
\polyline(5.901,-0.5)(6,-0.5)%
%
\polyline(0.5,-6)(0.5,-5.901)\polyline(0.5,-5.802)(0.5,-5.702)\polyline(0.5,-5.603)(0.5,-5.504)%
\polyline(0.5,-5.405)(0.5,-5.306)\polyline(0.5,-5.207)(0.5,-5.107)\polyline(0.5,-5.008)(0.5,-4.909)%
\polyline(0.5,-4.81)(0.5,-4.711)\polyline(0.5,-4.612)(0.5,-4.512)\polyline(0.5,-4.413)(0.5,-4.314)%
\polyline(0.5,-4.215)(0.5,-4.116)\polyline(0.5,-4.017)(0.5,-3.917)\polyline(0.5,-3.818)(0.5,-3.719)%
\polyline(0.5,-3.62)(0.5,-3.521)\polyline(0.5,-3.421)(0.5,-3.322)\polyline(0.5,-3.223)(0.5,-3.124)%
\polyline(0.5,-3.025)(0.5,-2.926)\polyline(0.5,-2.826)(0.5,-2.727)\polyline(0.5,-2.628)(0.5,-2.529)%
\polyline(0.5,-2.43)(0.5,-2.331)\polyline(0.5,-2.231)(0.5,-2.132)\polyline(0.5,-2.033)(0.5,-1.934)%
\polyline(0.5,-1.835)(0.5,-1.736)\polyline(0.5,-1.636)(0.5,-1.537)\polyline(0.5,-1.438)(0.5,-1.339)%
\polyline(0.5,-1.24)(0.5,-1.14)\polyline(0.5,-1.041)(0.5,-0.942)\polyline(0.5,-0.843)(0.5,-0.744)%
\polyline(0.5,-0.645)(0.5,-0.545)\polyline(0.5,-0.446)(0.5,-0.347)\polyline(0.5,-0.248)(0.5,-0.149)%
\polyline(0.5,-0.05)(0.5,0.05)\polyline(0.5,0.149)(0.5,0.248)\polyline(0.5,0.347)(0.5,0.446)%
\polyline(0.5,0.545)(0.5,0.645)\polyline(0.5,0.744)(0.5,0.843)\polyline(0.5,0.942)(0.5,1.041)%
\polyline(0.5,1.14)(0.5,1.24)\polyline(0.5,1.339)(0.5,1.438)\polyline(0.5,1.537)(0.5,1.636)%
\polyline(0.5,1.736)(0.5,1.835)\polyline(0.5,1.934)(0.5,2.033)\polyline(0.5,2.132)(0.5,2.231)%
\polyline(0.5,2.331)(0.5,2.43)\polyline(0.5,2.529)(0.5,2.628)\polyline(0.5,2.727)(0.5,2.826)%
\polyline(0.5,2.926)(0.5,3.025)\polyline(0.5,3.124)(0.5,3.223)\polyline(0.5,3.322)(0.5,3.421)%
\polyline(0.5,3.521)(0.5,3.62)\polyline(0.5,3.719)(0.5,3.818)\polyline(0.5,3.917)(0.5,4.017)%
\polyline(0.5,4.116)(0.5,4.215)\polyline(0.5,4.314)(0.5,4.413)\polyline(0.5,4.512)(0.5,4.612)%
\polyline(0.5,4.711)(0.5,4.81)\polyline(0.5,4.909)(0.5,5.008)\polyline(0.5,5.107)(0.5,5.207)%
\polyline(0.5,5.306)(0.5,5.405)\polyline(0.5,5.504)(0.5,5.603)\polyline(0.5,5.702)(0.5,5.802)%
\polyline(0.5,5.901)(0.5,6)%
%
\polyline(-6,0.5)(-5.901,0.5)\polyline(-5.802,0.5)(-5.702,0.5)\polyline(-5.603,0.5)(-5.504,0.5)%
\polyline(-5.405,0.5)(-5.306,0.5)\polyline(-5.207,0.5)(-5.107,0.5)\polyline(-5.008,0.5)(-4.909,0.5)%
\polyline(-4.81,0.5)(-4.711,0.5)\polyline(-4.612,0.5)(-4.512,0.5)\polyline(-4.413,0.5)(-4.314,0.5)%
\polyline(-4.215,0.5)(-4.116,0.5)\polyline(-4.017,0.5)(-3.917,0.5)\polyline(-3.818,0.5)(-3.719,0.5)%
\polyline(-3.62,0.5)(-3.521,0.5)\polyline(-3.421,0.5)(-3.322,0.5)\polyline(-3.223,0.5)(-3.124,0.5)%
\polyline(-3.025,0.5)(-2.926,0.5)\polyline(-2.826,0.5)(-2.727,0.5)\polyline(-2.628,0.5)(-2.529,0.5)%
\polyline(-2.43,0.5)(-2.331,0.5)\polyline(-2.231,0.5)(-2.132,0.5)\polyline(-2.033,0.5)(-1.934,0.5)%
\polyline(-1.835,0.5)(-1.736,0.5)\polyline(-1.636,0.5)(-1.537,0.5)\polyline(-1.438,0.5)(-1.339,0.5)%
\polyline(-1.24,0.5)(-1.14,0.5)\polyline(-1.041,0.5)(-0.942,0.5)\polyline(-0.843,0.5)(-0.744,0.5)%
\polyline(-0.645,0.5)(-0.545,0.5)\polyline(-0.446,0.5)(-0.347,0.5)\polyline(-0.248,0.5)(-0.149,0.5)%
\polyline(-0.05,0.5)(0.05,0.5)\polyline(0.149,0.5)(0.248,0.5)\polyline(0.347,0.5)(0.446,0.5)%
\polyline(0.545,0.5)(0.645,0.5)\polyline(0.744,0.5)(0.843,0.5)\polyline(0.942,0.5)(1.041,0.5)%
\polyline(1.14,0.5)(1.24,0.5)\polyline(1.339,0.5)(1.438,0.5)\polyline(1.537,0.5)(1.636,0.5)%
\polyline(1.736,0.5)(1.835,0.5)\polyline(1.934,0.5)(2.033,0.5)\polyline(2.132,0.5)(2.231,0.5)%
\polyline(2.331,0.5)(2.43,0.5)\polyline(2.529,0.5)(2.628,0.5)\polyline(2.727,0.5)(2.826,0.5)%
\polyline(2.926,0.5)(3.025,0.5)\polyline(3.124,0.5)(3.223,0.5)\polyline(3.322,0.5)(3.421,0.5)%
\polyline(3.521,0.5)(3.62,0.5)\polyline(3.719,0.5)(3.818,0.5)\polyline(3.917,0.5)(4.017,0.5)%
\polyline(4.116,0.5)(4.215,0.5)\polyline(4.314,0.5)(4.413,0.5)\polyline(4.512,0.5)(4.612,0.5)%
\polyline(4.711,0.5)(4.81,0.5)\polyline(4.909,0.5)(5.008,0.5)\polyline(5.107,0.5)(5.207,0.5)%
\polyline(5.306,0.5)(5.405,0.5)\polyline(5.504,0.5)(5.603,0.5)\polyline(5.702,0.5)(5.802,0.5)%
\polyline(5.901,0.5)(6,0.5)%
%
\polyline(1.5,-6)(1.5,-5.901)\polyline(1.5,-5.802)(1.5,-5.702)\polyline(1.5,-5.603)(1.5,-5.504)%
\polyline(1.5,-5.405)(1.5,-5.306)\polyline(1.5,-5.207)(1.5,-5.107)\polyline(1.5,-5.008)(1.5,-4.909)%
\polyline(1.5,-4.81)(1.5,-4.711)\polyline(1.5,-4.612)(1.5,-4.512)\polyline(1.5,-4.413)(1.5,-4.314)%
\polyline(1.5,-4.215)(1.5,-4.116)\polyline(1.5,-4.017)(1.5,-3.917)\polyline(1.5,-3.818)(1.5,-3.719)%
\polyline(1.5,-3.62)(1.5,-3.521)\polyline(1.5,-3.421)(1.5,-3.322)\polyline(1.5,-3.223)(1.5,-3.124)%
\polyline(1.5,-3.025)(1.5,-2.926)\polyline(1.5,-2.826)(1.5,-2.727)\polyline(1.5,-2.628)(1.5,-2.529)%
\polyline(1.5,-2.43)(1.5,-2.331)\polyline(1.5,-2.231)(1.5,-2.132)\polyline(1.5,-2.033)(1.5,-1.934)%
\polyline(1.5,-1.835)(1.5,-1.736)\polyline(1.5,-1.636)(1.5,-1.537)\polyline(1.5,-1.438)(1.5,-1.339)%
\polyline(1.5,-1.24)(1.5,-1.14)\polyline(1.5,-1.041)(1.5,-0.942)\polyline(1.5,-0.843)(1.5,-0.744)%
\polyline(1.5,-0.645)(1.5,-0.545)\polyline(1.5,-0.446)(1.5,-0.347)\polyline(1.5,-0.248)(1.5,-0.149)%
\polyline(1.5,-0.05)(1.5,0.05)\polyline(1.5,0.149)(1.5,0.248)\polyline(1.5,0.347)(1.5,0.446)%
\polyline(1.5,0.545)(1.5,0.645)\polyline(1.5,0.744)(1.5,0.843)\polyline(1.5,0.942)(1.5,1.041)%
\polyline(1.5,1.14)(1.5,1.24)\polyline(1.5,1.339)(1.5,1.438)\polyline(1.5,1.537)(1.5,1.636)%
\polyline(1.5,1.736)(1.5,1.835)\polyline(1.5,1.934)(1.5,2.033)\polyline(1.5,2.132)(1.5,2.231)%
\polyline(1.5,2.331)(1.5,2.43)\polyline(1.5,2.529)(1.5,2.628)\polyline(1.5,2.727)(1.5,2.826)%
\polyline(1.5,2.926)(1.5,3.025)\polyline(1.5,3.124)(1.5,3.223)\polyline(1.5,3.322)(1.5,3.421)%
\polyline(1.5,3.521)(1.5,3.62)\polyline(1.5,3.719)(1.5,3.818)\polyline(1.5,3.917)(1.5,4.017)%
\polyline(1.5,4.116)(1.5,4.215)\polyline(1.5,4.314)(1.5,4.413)\polyline(1.5,4.512)(1.5,4.612)%
\polyline(1.5,4.711)(1.5,4.81)\polyline(1.5,4.909)(1.5,5.008)\polyline(1.5,5.107)(1.5,5.207)%
\polyline(1.5,5.306)(1.5,5.405)\polyline(1.5,5.504)(1.5,5.603)\polyline(1.5,5.702)(1.5,5.802)%
\polyline(1.5,5.901)(1.5,6)%
%
\polyline(-6,1.5)(-5.901,1.5)\polyline(-5.802,1.5)(-5.702,1.5)\polyline(-5.603,1.5)(-5.504,1.5)%
\polyline(-5.405,1.5)(-5.306,1.5)\polyline(-5.207,1.5)(-5.107,1.5)\polyline(-5.008,1.5)(-4.909,1.5)%
\polyline(-4.81,1.5)(-4.711,1.5)\polyline(-4.612,1.5)(-4.512,1.5)\polyline(-4.413,1.5)(-4.314,1.5)%
\polyline(-4.215,1.5)(-4.116,1.5)\polyline(-4.017,1.5)(-3.917,1.5)\polyline(-3.818,1.5)(-3.719,1.5)%
\polyline(-3.62,1.5)(-3.521,1.5)\polyline(-3.421,1.5)(-3.322,1.5)\polyline(-3.223,1.5)(-3.124,1.5)%
\polyline(-3.025,1.5)(-2.926,1.5)\polyline(-2.826,1.5)(-2.727,1.5)\polyline(-2.628,1.5)(-2.529,1.5)%
\polyline(-2.43,1.5)(-2.331,1.5)\polyline(-2.231,1.5)(-2.132,1.5)\polyline(-2.033,1.5)(-1.934,1.5)%
\polyline(-1.835,1.5)(-1.736,1.5)\polyline(-1.636,1.5)(-1.537,1.5)\polyline(-1.438,1.5)(-1.339,1.5)%
\polyline(-1.24,1.5)(-1.14,1.5)\polyline(-1.041,1.5)(-0.942,1.5)\polyline(-0.843,1.5)(-0.744,1.5)%
\polyline(-0.645,1.5)(-0.545,1.5)\polyline(-0.446,1.5)(-0.347,1.5)\polyline(-0.248,1.5)(-0.149,1.5)%
\polyline(-0.05,1.5)(0.05,1.5)\polyline(0.149,1.5)(0.248,1.5)\polyline(0.347,1.5)(0.446,1.5)%
\polyline(0.545,1.5)(0.645,1.5)\polyline(0.744,1.5)(0.843,1.5)\polyline(0.942,1.5)(1.041,1.5)%
\polyline(1.14,1.5)(1.24,1.5)\polyline(1.339,1.5)(1.438,1.5)\polyline(1.537,1.5)(1.636,1.5)%
\polyline(1.736,1.5)(1.835,1.5)\polyline(1.934,1.5)(2.033,1.5)\polyline(2.132,1.5)(2.231,1.5)%
\polyline(2.331,1.5)(2.43,1.5)\polyline(2.529,1.5)(2.628,1.5)\polyline(2.727,1.5)(2.826,1.5)%
\polyline(2.926,1.5)(3.025,1.5)\polyline(3.124,1.5)(3.223,1.5)\polyline(3.322,1.5)(3.421,1.5)%
\polyline(3.521,1.5)(3.62,1.5)\polyline(3.719,1.5)(3.818,1.5)\polyline(3.917,1.5)(4.017,1.5)%
\polyline(4.116,1.5)(4.215,1.5)\polyline(4.314,1.5)(4.413,1.5)\polyline(4.512,1.5)(4.612,1.5)%
\polyline(4.711,1.5)(4.81,1.5)\polyline(4.909,1.5)(5.008,1.5)\polyline(5.107,1.5)(5.207,1.5)%
\polyline(5.306,1.5)(5.405,1.5)\polyline(5.504,1.5)(5.603,1.5)\polyline(5.702,1.5)(5.802,1.5)%
\polyline(5.901,1.5)(6,1.5)%
%
\polyline(2.5,-6)(2.5,-5.901)\polyline(2.5,-5.802)(2.5,-5.702)\polyline(2.5,-5.603)(2.5,-5.504)%
\polyline(2.5,-5.405)(2.5,-5.306)\polyline(2.5,-5.207)(2.5,-5.107)\polyline(2.5,-5.008)(2.5,-4.909)%
\polyline(2.5,-4.81)(2.5,-4.711)\polyline(2.5,-4.612)(2.5,-4.512)\polyline(2.5,-4.413)(2.5,-4.314)%
\polyline(2.5,-4.215)(2.5,-4.116)\polyline(2.5,-4.017)(2.5,-3.917)\polyline(2.5,-3.818)(2.5,-3.719)%
\polyline(2.5,-3.62)(2.5,-3.521)\polyline(2.5,-3.421)(2.5,-3.322)\polyline(2.5,-3.223)(2.5,-3.124)%
\polyline(2.5,-3.025)(2.5,-2.926)\polyline(2.5,-2.826)(2.5,-2.727)\polyline(2.5,-2.628)(2.5,-2.529)%
\polyline(2.5,-2.43)(2.5,-2.331)\polyline(2.5,-2.231)(2.5,-2.132)\polyline(2.5,-2.033)(2.5,-1.934)%
\polyline(2.5,-1.835)(2.5,-1.736)\polyline(2.5,-1.636)(2.5,-1.537)\polyline(2.5,-1.438)(2.5,-1.339)%
\polyline(2.5,-1.24)(2.5,-1.14)\polyline(2.5,-1.041)(2.5,-0.942)\polyline(2.5,-0.843)(2.5,-0.744)%
\polyline(2.5,-0.645)(2.5,-0.545)\polyline(2.5,-0.446)(2.5,-0.347)\polyline(2.5,-0.248)(2.5,-0.149)%
\polyline(2.5,-0.05)(2.5,0.05)\polyline(2.5,0.149)(2.5,0.248)\polyline(2.5,0.347)(2.5,0.446)%
\polyline(2.5,0.545)(2.5,0.645)\polyline(2.5,0.744)(2.5,0.843)\polyline(2.5,0.942)(2.5,1.041)%
\polyline(2.5,1.14)(2.5,1.24)\polyline(2.5,1.339)(2.5,1.438)\polyline(2.5,1.537)(2.5,1.636)%
\polyline(2.5,1.736)(2.5,1.835)\polyline(2.5,1.934)(2.5,2.033)\polyline(2.5,2.132)(2.5,2.231)%
\polyline(2.5,2.331)(2.5,2.43)\polyline(2.5,2.529)(2.5,2.628)\polyline(2.5,2.727)(2.5,2.826)%
\polyline(2.5,2.926)(2.5,3.025)\polyline(2.5,3.124)(2.5,3.223)\polyline(2.5,3.322)(2.5,3.421)%
\polyline(2.5,3.521)(2.5,3.62)\polyline(2.5,3.719)(2.5,3.818)\polyline(2.5,3.917)(2.5,4.017)%
\polyline(2.5,4.116)(2.5,4.215)\polyline(2.5,4.314)(2.5,4.413)\polyline(2.5,4.512)(2.5,4.612)%
\polyline(2.5,4.711)(2.5,4.81)\polyline(2.5,4.909)(2.5,5.008)\polyline(2.5,5.107)(2.5,5.207)%
\polyline(2.5,5.306)(2.5,5.405)\polyline(2.5,5.504)(2.5,5.603)\polyline(2.5,5.702)(2.5,5.802)%
\polyline(2.5,5.901)(2.5,6)%
%
\polyline(-6,2.5)(-5.901,2.5)\polyline(-5.802,2.5)(-5.702,2.5)\polyline(-5.603,2.5)(-5.504,2.5)%
\polyline(-5.405,2.5)(-5.306,2.5)\polyline(-5.207,2.5)(-5.107,2.5)\polyline(-5.008,2.5)(-4.909,2.5)%
\polyline(-4.81,2.5)(-4.711,2.5)\polyline(-4.612,2.5)(-4.512,2.5)\polyline(-4.413,2.5)(-4.314,2.5)%
\polyline(-4.215,2.5)(-4.116,2.5)\polyline(-4.017,2.5)(-3.917,2.5)\polyline(-3.818,2.5)(-3.719,2.5)%
\polyline(-3.62,2.5)(-3.521,2.5)\polyline(-3.421,2.5)(-3.322,2.5)\polyline(-3.223,2.5)(-3.124,2.5)%
\polyline(-3.025,2.5)(-2.926,2.5)\polyline(-2.826,2.5)(-2.727,2.5)\polyline(-2.628,2.5)(-2.529,2.5)%
\polyline(-2.43,2.5)(-2.331,2.5)\polyline(-2.231,2.5)(-2.132,2.5)\polyline(-2.033,2.5)(-1.934,2.5)%
\polyline(-1.835,2.5)(-1.736,2.5)\polyline(-1.636,2.5)(-1.537,2.5)\polyline(-1.438,2.5)(-1.339,2.5)%
\polyline(-1.24,2.5)(-1.14,2.5)\polyline(-1.041,2.5)(-0.942,2.5)\polyline(-0.843,2.5)(-0.744,2.5)%
\polyline(-0.645,2.5)(-0.545,2.5)\polyline(-0.446,2.5)(-0.347,2.5)\polyline(-0.248,2.5)(-0.149,2.5)%
\polyline(-0.05,2.5)(0.05,2.5)\polyline(0.149,2.5)(0.248,2.5)\polyline(0.347,2.5)(0.446,2.5)%
\polyline(0.545,2.5)(0.645,2.5)\polyline(0.744,2.5)(0.843,2.5)\polyline(0.942,2.5)(1.041,2.5)%
\polyline(1.14,2.5)(1.24,2.5)\polyline(1.339,2.5)(1.438,2.5)\polyline(1.537,2.5)(1.636,2.5)%
\polyline(1.736,2.5)(1.835,2.5)\polyline(1.934,2.5)(2.033,2.5)\polyline(2.132,2.5)(2.231,2.5)%
\polyline(2.331,2.5)(2.43,2.5)\polyline(2.529,2.5)(2.628,2.5)\polyline(2.727,2.5)(2.826,2.5)%
\polyline(2.926,2.5)(3.025,2.5)\polyline(3.124,2.5)(3.223,2.5)\polyline(3.322,2.5)(3.421,2.5)%
\polyline(3.521,2.5)(3.62,2.5)\polyline(3.719,2.5)(3.818,2.5)\polyline(3.917,2.5)(4.017,2.5)%
\polyline(4.116,2.5)(4.215,2.5)\polyline(4.314,2.5)(4.413,2.5)\polyline(4.512,2.5)(4.612,2.5)%
\polyline(4.711,2.5)(4.81,2.5)\polyline(4.909,2.5)(5.008,2.5)\polyline(5.107,2.5)(5.207,2.5)%
\polyline(5.306,2.5)(5.405,2.5)\polyline(5.504,2.5)(5.603,2.5)\polyline(5.702,2.5)(5.802,2.5)%
\polyline(5.901,2.5)(6,2.5)%
%
\polyline(3.5,-6)(3.5,-5.901)\polyline(3.5,-5.802)(3.5,-5.702)\polyline(3.5,-5.603)(3.5,-5.504)%
\polyline(3.5,-5.405)(3.5,-5.306)\polyline(3.5,-5.207)(3.5,-5.107)\polyline(3.5,-5.008)(3.5,-4.909)%
\polyline(3.5,-4.81)(3.5,-4.711)\polyline(3.5,-4.612)(3.5,-4.512)\polyline(3.5,-4.413)(3.5,-4.314)%
\polyline(3.5,-4.215)(3.5,-4.116)\polyline(3.5,-4.017)(3.5,-3.917)\polyline(3.5,-3.818)(3.5,-3.719)%
\polyline(3.5,-3.62)(3.5,-3.521)\polyline(3.5,-3.421)(3.5,-3.322)\polyline(3.5,-3.223)(3.5,-3.124)%
\polyline(3.5,-3.025)(3.5,-2.926)\polyline(3.5,-2.826)(3.5,-2.727)\polyline(3.5,-2.628)(3.5,-2.529)%
\polyline(3.5,-2.43)(3.5,-2.331)\polyline(3.5,-2.231)(3.5,-2.132)\polyline(3.5,-2.033)(3.5,-1.934)%
\polyline(3.5,-1.835)(3.5,-1.736)\polyline(3.5,-1.636)(3.5,-1.537)\polyline(3.5,-1.438)(3.5,-1.339)%
\polyline(3.5,-1.24)(3.5,-1.14)\polyline(3.5,-1.041)(3.5,-0.942)\polyline(3.5,-0.843)(3.5,-0.744)%
\polyline(3.5,-0.645)(3.5,-0.545)\polyline(3.5,-0.446)(3.5,-0.347)\polyline(3.5,-0.248)(3.5,-0.149)%
\polyline(3.5,-0.05)(3.5,0.05)\polyline(3.5,0.149)(3.5,0.248)\polyline(3.5,0.347)(3.5,0.446)%
\polyline(3.5,0.545)(3.5,0.645)\polyline(3.5,0.744)(3.5,0.843)\polyline(3.5,0.942)(3.5,1.041)%
\polyline(3.5,1.14)(3.5,1.24)\polyline(3.5,1.339)(3.5,1.438)\polyline(3.5,1.537)(3.5,1.636)%
\polyline(3.5,1.736)(3.5,1.835)\polyline(3.5,1.934)(3.5,2.033)\polyline(3.5,2.132)(3.5,2.231)%
\polyline(3.5,2.331)(3.5,2.43)\polyline(3.5,2.529)(3.5,2.628)\polyline(3.5,2.727)(3.5,2.826)%
\polyline(3.5,2.926)(3.5,3.025)\polyline(3.5,3.124)(3.5,3.223)\polyline(3.5,3.322)(3.5,3.421)%
\polyline(3.5,3.521)(3.5,3.62)\polyline(3.5,3.719)(3.5,3.818)\polyline(3.5,3.917)(3.5,4.017)%
\polyline(3.5,4.116)(3.5,4.215)\polyline(3.5,4.314)(3.5,4.413)\polyline(3.5,4.512)(3.5,4.612)%
\polyline(3.5,4.711)(3.5,4.81)\polyline(3.5,4.909)(3.5,5.008)\polyline(3.5,5.107)(3.5,5.207)%
\polyline(3.5,5.306)(3.5,5.405)\polyline(3.5,5.504)(3.5,5.603)\polyline(3.5,5.702)(3.5,5.802)%
\polyline(3.5,5.901)(3.5,6)%
%
\polyline(-6,3.5)(-5.901,3.5)\polyline(-5.802,3.5)(-5.702,3.5)\polyline(-5.603,3.5)(-5.504,3.5)%
\polyline(-5.405,3.5)(-5.306,3.5)\polyline(-5.207,3.5)(-5.107,3.5)\polyline(-5.008,3.5)(-4.909,3.5)%
\polyline(-4.81,3.5)(-4.711,3.5)\polyline(-4.612,3.5)(-4.512,3.5)\polyline(-4.413,3.5)(-4.314,3.5)%
\polyline(-4.215,3.5)(-4.116,3.5)\polyline(-4.017,3.5)(-3.917,3.5)\polyline(-3.818,3.5)(-3.719,3.5)%
\polyline(-3.62,3.5)(-3.521,3.5)\polyline(-3.421,3.5)(-3.322,3.5)\polyline(-3.223,3.5)(-3.124,3.5)%
\polyline(-3.025,3.5)(-2.926,3.5)\polyline(-2.826,3.5)(-2.727,3.5)\polyline(-2.628,3.5)(-2.529,3.5)%
\polyline(-2.43,3.5)(-2.331,3.5)\polyline(-2.231,3.5)(-2.132,3.5)\polyline(-2.033,3.5)(-1.934,3.5)%
\polyline(-1.835,3.5)(-1.736,3.5)\polyline(-1.636,3.5)(-1.537,3.5)\polyline(-1.438,3.5)(-1.339,3.5)%
\polyline(-1.24,3.5)(-1.14,3.5)\polyline(-1.041,3.5)(-0.942,3.5)\polyline(-0.843,3.5)(-0.744,3.5)%
\polyline(-0.645,3.5)(-0.545,3.5)\polyline(-0.446,3.5)(-0.347,3.5)\polyline(-0.248,3.5)(-0.149,3.5)%
\polyline(-0.05,3.5)(0.05,3.5)\polyline(0.149,3.5)(0.248,3.5)\polyline(0.347,3.5)(0.446,3.5)%
\polyline(0.545,3.5)(0.645,3.5)\polyline(0.744,3.5)(0.843,3.5)\polyline(0.942,3.5)(1.041,3.5)%
\polyline(1.14,3.5)(1.24,3.5)\polyline(1.339,3.5)(1.438,3.5)\polyline(1.537,3.5)(1.636,3.5)%
\polyline(1.736,3.5)(1.835,3.5)\polyline(1.934,3.5)(2.033,3.5)\polyline(2.132,3.5)(2.231,3.5)%
\polyline(2.331,3.5)(2.43,3.5)\polyline(2.529,3.5)(2.628,3.5)\polyline(2.727,3.5)(2.826,3.5)%
\polyline(2.926,3.5)(3.025,3.5)\polyline(3.124,3.5)(3.223,3.5)\polyline(3.322,3.5)(3.421,3.5)%
\polyline(3.521,3.5)(3.62,3.5)\polyline(3.719,3.5)(3.818,3.5)\polyline(3.917,3.5)(4.017,3.5)%
\polyline(4.116,3.5)(4.215,3.5)\polyline(4.314,3.5)(4.413,3.5)\polyline(4.512,3.5)(4.612,3.5)%
\polyline(4.711,3.5)(4.81,3.5)\polyline(4.909,3.5)(5.008,3.5)\polyline(5.107,3.5)(5.207,3.5)%
\polyline(5.306,3.5)(5.405,3.5)\polyline(5.504,3.5)(5.603,3.5)\polyline(5.702,3.5)(5.802,3.5)%
\polyline(5.901,3.5)(6,3.5)%
%
\polyline(4.5,-6)(4.5,-5.901)\polyline(4.5,-5.802)(4.5,-5.702)\polyline(4.5,-5.603)(4.5,-5.504)%
\polyline(4.5,-5.405)(4.5,-5.306)\polyline(4.5,-5.207)(4.5,-5.107)\polyline(4.5,-5.008)(4.5,-4.909)%
\polyline(4.5,-4.81)(4.5,-4.711)\polyline(4.5,-4.612)(4.5,-4.512)\polyline(4.5,-4.413)(4.5,-4.314)%
\polyline(4.5,-4.215)(4.5,-4.116)\polyline(4.5,-4.017)(4.5,-3.917)\polyline(4.5,-3.818)(4.5,-3.719)%
\polyline(4.5,-3.62)(4.5,-3.521)\polyline(4.5,-3.421)(4.5,-3.322)\polyline(4.5,-3.223)(4.5,-3.124)%
\polyline(4.5,-3.025)(4.5,-2.926)\polyline(4.5,-2.826)(4.5,-2.727)\polyline(4.5,-2.628)(4.5,-2.529)%
\polyline(4.5,-2.43)(4.5,-2.331)\polyline(4.5,-2.231)(4.5,-2.132)\polyline(4.5,-2.033)(4.5,-1.934)%
\polyline(4.5,-1.835)(4.5,-1.736)\polyline(4.5,-1.636)(4.5,-1.537)\polyline(4.5,-1.438)(4.5,-1.339)%
\polyline(4.5,-1.24)(4.5,-1.14)\polyline(4.5,-1.041)(4.5,-0.942)\polyline(4.5,-0.843)(4.5,-0.744)%
\polyline(4.5,-0.645)(4.5,-0.545)\polyline(4.5,-0.446)(4.5,-0.347)\polyline(4.5,-0.248)(4.5,-0.149)%
\polyline(4.5,-0.05)(4.5,0.05)\polyline(4.5,0.149)(4.5,0.248)\polyline(4.5,0.347)(4.5,0.446)%
\polyline(4.5,0.545)(4.5,0.645)\polyline(4.5,0.744)(4.5,0.843)\polyline(4.5,0.942)(4.5,1.041)%
\polyline(4.5,1.14)(4.5,1.24)\polyline(4.5,1.339)(4.5,1.438)\polyline(4.5,1.537)(4.5,1.636)%
\polyline(4.5,1.736)(4.5,1.835)\polyline(4.5,1.934)(4.5,2.033)\polyline(4.5,2.132)(4.5,2.231)%
\polyline(4.5,2.331)(4.5,2.43)\polyline(4.5,2.529)(4.5,2.628)\polyline(4.5,2.727)(4.5,2.826)%
\polyline(4.5,2.926)(4.5,3.025)\polyline(4.5,3.124)(4.5,3.223)\polyline(4.5,3.322)(4.5,3.421)%
\polyline(4.5,3.521)(4.5,3.62)\polyline(4.5,3.719)(4.5,3.818)\polyline(4.5,3.917)(4.5,4.017)%
\polyline(4.5,4.116)(4.5,4.215)\polyline(4.5,4.314)(4.5,4.413)\polyline(4.5,4.512)(4.5,4.612)%
\polyline(4.5,4.711)(4.5,4.81)\polyline(4.5,4.909)(4.5,5.008)\polyline(4.5,5.107)(4.5,5.207)%
\polyline(4.5,5.306)(4.5,5.405)\polyline(4.5,5.504)(4.5,5.603)\polyline(4.5,5.702)(4.5,5.802)%
\polyline(4.5,5.901)(4.5,6)%
%
\polyline(-6,4.5)(-5.901,4.5)\polyline(-5.802,4.5)(-5.702,4.5)\polyline(-5.603,4.5)(-5.504,4.5)%
\polyline(-5.405,4.5)(-5.306,4.5)\polyline(-5.207,4.5)(-5.107,4.5)\polyline(-5.008,4.5)(-4.909,4.5)%
\polyline(-4.81,4.5)(-4.711,4.5)\polyline(-4.612,4.5)(-4.512,4.5)\polyline(-4.413,4.5)(-4.314,4.5)%
\polyline(-4.215,4.5)(-4.116,4.5)\polyline(-4.017,4.5)(-3.917,4.5)\polyline(-3.818,4.5)(-3.719,4.5)%
\polyline(-3.62,4.5)(-3.521,4.5)\polyline(-3.421,4.5)(-3.322,4.5)\polyline(-3.223,4.5)(-3.124,4.5)%
\polyline(-3.025,4.5)(-2.926,4.5)\polyline(-2.826,4.5)(-2.727,4.5)\polyline(-2.628,4.5)(-2.529,4.5)%
\polyline(-2.43,4.5)(-2.331,4.5)\polyline(-2.231,4.5)(-2.132,4.5)\polyline(-2.033,4.5)(-1.934,4.5)%
\polyline(-1.835,4.5)(-1.736,4.5)\polyline(-1.636,4.5)(-1.537,4.5)\polyline(-1.438,4.5)(-1.339,4.5)%
\polyline(-1.24,4.5)(-1.14,4.5)\polyline(-1.041,4.5)(-0.942,4.5)\polyline(-0.843,4.5)(-0.744,4.5)%
\polyline(-0.645,4.5)(-0.545,4.5)\polyline(-0.446,4.5)(-0.347,4.5)\polyline(-0.248,4.5)(-0.149,4.5)%
\polyline(-0.05,4.5)(0.05,4.5)\polyline(0.149,4.5)(0.248,4.5)\polyline(0.347,4.5)(0.446,4.5)%
\polyline(0.545,4.5)(0.645,4.5)\polyline(0.744,4.5)(0.843,4.5)\polyline(0.942,4.5)(1.041,4.5)%
\polyline(1.14,4.5)(1.24,4.5)\polyline(1.339,4.5)(1.438,4.5)\polyline(1.537,4.5)(1.636,4.5)%
\polyline(1.736,4.5)(1.835,4.5)\polyline(1.934,4.5)(2.033,4.5)\polyline(2.132,4.5)(2.231,4.5)%
\polyline(2.331,4.5)(2.43,4.5)\polyline(2.529,4.5)(2.628,4.5)\polyline(2.727,4.5)(2.826,4.5)%
\polyline(2.926,4.5)(3.025,4.5)\polyline(3.124,4.5)(3.223,4.5)\polyline(3.322,4.5)(3.421,4.5)%
\polyline(3.521,4.5)(3.62,4.5)\polyline(3.719,4.5)(3.818,4.5)\polyline(3.917,4.5)(4.017,4.5)%
\polyline(4.116,4.5)(4.215,4.5)\polyline(4.314,4.5)(4.413,4.5)\polyline(4.512,4.5)(4.612,4.5)%
\polyline(4.711,4.5)(4.81,4.5)\polyline(4.909,4.5)(5.008,4.5)\polyline(5.107,4.5)(5.207,4.5)%
\polyline(5.306,4.5)(5.405,4.5)\polyline(5.504,4.5)(5.603,4.5)\polyline(5.702,4.5)(5.802,4.5)%
\polyline(5.901,4.5)(6,4.5)%
%
\polyline(5.5,-6)(5.5,-5.901)\polyline(5.5,-5.802)(5.5,-5.702)\polyline(5.5,-5.603)(5.5,-5.504)%
\polyline(5.5,-5.405)(5.5,-5.306)\polyline(5.5,-5.207)(5.5,-5.107)\polyline(5.5,-5.008)(5.5,-4.909)%
\polyline(5.5,-4.81)(5.5,-4.711)\polyline(5.5,-4.612)(5.5,-4.512)\polyline(5.5,-4.413)(5.5,-4.314)%
\polyline(5.5,-4.215)(5.5,-4.116)\polyline(5.5,-4.017)(5.5,-3.917)\polyline(5.5,-3.818)(5.5,-3.719)%
\polyline(5.5,-3.62)(5.5,-3.521)\polyline(5.5,-3.421)(5.5,-3.322)\polyline(5.5,-3.223)(5.5,-3.124)%
\polyline(5.5,-3.025)(5.5,-2.926)\polyline(5.5,-2.826)(5.5,-2.727)\polyline(5.5,-2.628)(5.5,-2.529)%
\polyline(5.5,-2.43)(5.5,-2.331)\polyline(5.5,-2.231)(5.5,-2.132)\polyline(5.5,-2.033)(5.5,-1.934)%
\polyline(5.5,-1.835)(5.5,-1.736)\polyline(5.5,-1.636)(5.5,-1.537)\polyline(5.5,-1.438)(5.5,-1.339)%
\polyline(5.5,-1.24)(5.5,-1.14)\polyline(5.5,-1.041)(5.5,-0.942)\polyline(5.5,-0.843)(5.5,-0.744)%
\polyline(5.5,-0.645)(5.5,-0.545)\polyline(5.5,-0.446)(5.5,-0.347)\polyline(5.5,-0.248)(5.5,-0.149)%
\polyline(5.5,-0.05)(5.5,0.05)\polyline(5.5,0.149)(5.5,0.248)\polyline(5.5,0.347)(5.5,0.446)%
\polyline(5.5,0.545)(5.5,0.645)\polyline(5.5,0.744)(5.5,0.843)\polyline(5.5,0.942)(5.5,1.041)%
\polyline(5.5,1.14)(5.5,1.24)\polyline(5.5,1.339)(5.5,1.438)\polyline(5.5,1.537)(5.5,1.636)%
\polyline(5.5,1.736)(5.5,1.835)\polyline(5.5,1.934)(5.5,2.033)\polyline(5.5,2.132)(5.5,2.231)%
\polyline(5.5,2.331)(5.5,2.43)\polyline(5.5,2.529)(5.5,2.628)\polyline(5.5,2.727)(5.5,2.826)%
\polyline(5.5,2.926)(5.5,3.025)\polyline(5.5,3.124)(5.5,3.223)\polyline(5.5,3.322)(5.5,3.421)%
\polyline(5.5,3.521)(5.5,3.62)\polyline(5.5,3.719)(5.5,3.818)\polyline(5.5,3.917)(5.5,4.017)%
\polyline(5.5,4.116)(5.5,4.215)\polyline(5.5,4.314)(5.5,4.413)\polyline(5.5,4.512)(5.5,4.612)%
\polyline(5.5,4.711)(5.5,4.81)\polyline(5.5,4.909)(5.5,5.008)\polyline(5.5,5.107)(5.5,5.207)%
\polyline(5.5,5.306)(5.5,5.405)\polyline(5.5,5.504)(5.5,5.603)\polyline(5.5,5.702)(5.5,5.802)%
\polyline(5.5,5.901)(5.5,6)%
%
\polyline(-6,5.5)(-5.901,5.5)\polyline(-5.802,5.5)(-5.702,5.5)\polyline(-5.603,5.5)(-5.504,5.5)%
\polyline(-5.405,5.5)(-5.306,5.5)\polyline(-5.207,5.5)(-5.107,5.5)\polyline(-5.008,5.5)(-4.909,5.5)%
\polyline(-4.81,5.5)(-4.711,5.5)\polyline(-4.612,5.5)(-4.512,5.5)\polyline(-4.413,5.5)(-4.314,5.5)%
\polyline(-4.215,5.5)(-4.116,5.5)\polyline(-4.017,5.5)(-3.917,5.5)\polyline(-3.818,5.5)(-3.719,5.5)%
\polyline(-3.62,5.5)(-3.521,5.5)\polyline(-3.421,5.5)(-3.322,5.5)\polyline(-3.223,5.5)(-3.124,5.5)%
\polyline(-3.025,5.5)(-2.926,5.5)\polyline(-2.826,5.5)(-2.727,5.5)\polyline(-2.628,5.5)(-2.529,5.5)%
\polyline(-2.43,5.5)(-2.331,5.5)\polyline(-2.231,5.5)(-2.132,5.5)\polyline(-2.033,5.5)(-1.934,5.5)%
\polyline(-1.835,5.5)(-1.736,5.5)\polyline(-1.636,5.5)(-1.537,5.5)\polyline(-1.438,5.5)(-1.339,5.5)%
\polyline(-1.24,5.5)(-1.14,5.5)\polyline(-1.041,5.5)(-0.942,5.5)\polyline(-0.843,5.5)(-0.744,5.5)%
\polyline(-0.645,5.5)(-0.545,5.5)\polyline(-0.446,5.5)(-0.347,5.5)\polyline(-0.248,5.5)(-0.149,5.5)%
\polyline(-0.05,5.5)(0.05,5.5)\polyline(0.149,5.5)(0.248,5.5)\polyline(0.347,5.5)(0.446,5.5)%
\polyline(0.545,5.5)(0.645,5.5)\polyline(0.744,5.5)(0.843,5.5)\polyline(0.942,5.5)(1.041,5.5)%
\polyline(1.14,5.5)(1.24,5.5)\polyline(1.339,5.5)(1.438,5.5)\polyline(1.537,5.5)(1.636,5.5)%
\polyline(1.736,5.5)(1.835,5.5)\polyline(1.934,5.5)(2.033,5.5)\polyline(2.132,5.5)(2.231,5.5)%
\polyline(2.331,5.5)(2.43,5.5)\polyline(2.529,5.5)(2.628,5.5)\polyline(2.727,5.5)(2.826,5.5)%
\polyline(2.926,5.5)(3.025,5.5)\polyline(3.124,5.5)(3.223,5.5)\polyline(3.322,5.5)(3.421,5.5)%
\polyline(3.521,5.5)(3.62,5.5)\polyline(3.719,5.5)(3.818,5.5)\polyline(3.917,5.5)(4.017,5.5)%
\polyline(4.116,5.5)(4.215,5.5)\polyline(4.314,5.5)(4.413,5.5)\polyline(4.512,5.5)(4.612,5.5)%
\polyline(4.711,5.5)(4.81,5.5)\polyline(4.909,5.5)(5.008,5.5)\polyline(5.107,5.5)(5.207,5.5)%
\polyline(5.306,5.5)(5.405,5.5)\polyline(5.504,5.5)(5.603,5.5)\polyline(5.702,5.5)(5.802,5.5)%
\polyline(5.901,5.5)(6,5.5)%
%
\linethickness{0.008in}%%
\polyline(6,-0.05)(6,0.05)%
%
\settowidth{\Width}{$-6$}\setlength{\Width}{-0.5\Width}%
\settoheight{\Height}{$-6$}\settodepth{\Depth}{$-6$}\setlength{\Height}{-\Height}%
\put( -6.000, -0.100){\hspace*{\Width}\raisebox{\Height}{$-6$}}%
%
\polyline(-0.05,6)(0.05,6)%
%
\settowidth{\Width}{$-6$}\setlength{\Width}{-1\Width}%
\settoheight{\Height}{$-6$}\settodepth{\Depth}{$-6$}\setlength{\Height}{-0.5\Height}\setlength{\Depth}{0.5\Depth}\addtolength{\Height}{\Depth}%
\put( -0.100, -6.000){\hspace*{\Width}\raisebox{\Height}{$-6$}}%
%
\polyline(6,-0.05)(6,0.05)%
%
\settowidth{\Width}{$-5$}\setlength{\Width}{-0.5\Width}%
\settoheight{\Height}{$-5$}\settodepth{\Depth}{$-5$}\setlength{\Height}{-\Height}%
\put( -5.000, -0.100){\hspace*{\Width}\raisebox{\Height}{$-5$}}%
%
\polyline(-0.05,6)(0.05,6)%
%
\settowidth{\Width}{$-5$}\setlength{\Width}{-1\Width}%
\settoheight{\Height}{$-5$}\settodepth{\Depth}{$-5$}\setlength{\Height}{-0.5\Height}\setlength{\Depth}{0.5\Depth}\addtolength{\Height}{\Depth}%
\put( -0.100, -5.000){\hspace*{\Width}\raisebox{\Height}{$-5$}}%
%
\polyline(6,-0.05)(6,0.05)%
%
\settowidth{\Width}{$-4$}\setlength{\Width}{-0.5\Width}%
\settoheight{\Height}{$-4$}\settodepth{\Depth}{$-4$}\setlength{\Height}{-\Height}%
\put( -4.000, -0.100){\hspace*{\Width}\raisebox{\Height}{$-4$}}%
%
\polyline(-0.05,6)(0.05,6)%
%
\settowidth{\Width}{$-4$}\setlength{\Width}{-1\Width}%
\settoheight{\Height}{$-4$}\settodepth{\Depth}{$-4$}\setlength{\Height}{-0.5\Height}\setlength{\Depth}{0.5\Depth}\addtolength{\Height}{\Depth}%
\put( -0.100, -4.000){\hspace*{\Width}\raisebox{\Height}{$-4$}}%
%
\polyline(6,-0.05)(6,0.05)%
%
\settowidth{\Width}{$-3$}\setlength{\Width}{-0.5\Width}%
\settoheight{\Height}{$-3$}\settodepth{\Depth}{$-3$}\setlength{\Height}{-\Height}%
\put( -3.000, -0.100){\hspace*{\Width}\raisebox{\Height}{$-3$}}%
%
\polyline(-0.05,6)(0.05,6)%
%
\settowidth{\Width}{$-3$}\setlength{\Width}{-1\Width}%
\settoheight{\Height}{$-3$}\settodepth{\Depth}{$-3$}\setlength{\Height}{-0.5\Height}\setlength{\Depth}{0.5\Depth}\addtolength{\Height}{\Depth}%
\put( -0.100, -3.000){\hspace*{\Width}\raisebox{\Height}{$-3$}}%
%
\polyline(6,-0.05)(6,0.05)%
%
\settowidth{\Width}{$-2$}\setlength{\Width}{-0.5\Width}%
\settoheight{\Height}{$-2$}\settodepth{\Depth}{$-2$}\setlength{\Height}{-\Height}%
\put( -2.000, -0.100){\hspace*{\Width}\raisebox{\Height}{$-2$}}%
%
\polyline(-0.05,6)(0.05,6)%
%
\settowidth{\Width}{$-2$}\setlength{\Width}{-1\Width}%
\settoheight{\Height}{$-2$}\settodepth{\Depth}{$-2$}\setlength{\Height}{-0.5\Height}\setlength{\Depth}{0.5\Depth}\addtolength{\Height}{\Depth}%
\put( -0.100, -2.000){\hspace*{\Width}\raisebox{\Height}{$-2$}}%
%
\polyline(6,-0.05)(6,0.05)%
%
\settowidth{\Width}{$-1$}\setlength{\Width}{-0.5\Width}%
\settoheight{\Height}{$-1$}\settodepth{\Depth}{$-1$}\setlength{\Height}{-\Height}%
\put( -1.000, -0.100){\hspace*{\Width}\raisebox{\Height}{$-1$}}%
%
\polyline(-0.05,6)(0.05,6)%
%
\settowidth{\Width}{$-1$}\setlength{\Width}{-1\Width}%
\settoheight{\Height}{$-1$}\settodepth{\Depth}{$-1$}\setlength{\Height}{-0.5\Height}\setlength{\Depth}{0.5\Depth}\addtolength{\Height}{\Depth}%
\put( -0.100, -1.000){\hspace*{\Width}\raisebox{\Height}{$-1$}}%
%
\polyline(6,-0.05)(6,0.05)%
%
\settowidth{\Width}{$1$}\setlength{\Width}{-0.5\Width}%
\settoheight{\Height}{$1$}\settodepth{\Depth}{$1$}\setlength{\Height}{-\Height}%
\put(  1.000, -0.100){\hspace*{\Width}\raisebox{\Height}{$1$}}%
%
\polyline(-0.05,6)(0.05,6)%
%
\settowidth{\Width}{$1$}\setlength{\Width}{-1\Width}%
\settoheight{\Height}{$1$}\settodepth{\Depth}{$1$}\setlength{\Height}{-0.5\Height}\setlength{\Depth}{0.5\Depth}\addtolength{\Height}{\Depth}%
\put( -0.100,  1.000){\hspace*{\Width}\raisebox{\Height}{$1$}}%
%
\polyline(6,-0.05)(6,0.05)%
%
\settowidth{\Width}{$2$}\setlength{\Width}{-0.5\Width}%
\settoheight{\Height}{$2$}\settodepth{\Depth}{$2$}\setlength{\Height}{-\Height}%
\put(  2.000, -0.100){\hspace*{\Width}\raisebox{\Height}{$2$}}%
%
\polyline(-0.05,6)(0.05,6)%
%
\settowidth{\Width}{$2$}\setlength{\Width}{-1\Width}%
\settoheight{\Height}{$2$}\settodepth{\Depth}{$2$}\setlength{\Height}{-0.5\Height}\setlength{\Depth}{0.5\Depth}\addtolength{\Height}{\Depth}%
\put( -0.100,  2.000){\hspace*{\Width}\raisebox{\Height}{$2$}}%
%
\polyline(6,-0.05)(6,0.05)%
%
\settowidth{\Width}{$3$}\setlength{\Width}{-0.5\Width}%
\settoheight{\Height}{$3$}\settodepth{\Depth}{$3$}\setlength{\Height}{-\Height}%
\put(  3.000, -0.100){\hspace*{\Width}\raisebox{\Height}{$3$}}%
%
\polyline(-0.05,6)(0.05,6)%
%
\settowidth{\Width}{$3$}\setlength{\Width}{-1\Width}%
\settoheight{\Height}{$3$}\settodepth{\Depth}{$3$}\setlength{\Height}{-0.5\Height}\setlength{\Depth}{0.5\Depth}\addtolength{\Height}{\Depth}%
\put( -0.100,  3.000){\hspace*{\Width}\raisebox{\Height}{$3$}}%
%
\polyline(6,-0.05)(6,0.05)%
%
\settowidth{\Width}{$4$}\setlength{\Width}{-0.5\Width}%
\settoheight{\Height}{$4$}\settodepth{\Depth}{$4$}\setlength{\Height}{-\Height}%
\put(  4.000, -0.100){\hspace*{\Width}\raisebox{\Height}{$4$}}%
%
\polyline(-0.05,6)(0.05,6)%
%
\settowidth{\Width}{$4$}\setlength{\Width}{-1\Width}%
\settoheight{\Height}{$4$}\settodepth{\Depth}{$4$}\setlength{\Height}{-0.5\Height}\setlength{\Depth}{0.5\Depth}\addtolength{\Height}{\Depth}%
\put( -0.100,  4.000){\hspace*{\Width}\raisebox{\Height}{$4$}}%
%
\polyline(6,-0.05)(6,0.05)%
%
\settowidth{\Width}{$5$}\setlength{\Width}{-0.5\Width}%
\settoheight{\Height}{$5$}\settodepth{\Depth}{$5$}\setlength{\Height}{-\Height}%
\put(  5.000, -0.100){\hspace*{\Width}\raisebox{\Height}{$5$}}%
%
\polyline(-0.05,6)(0.05,6)%
%
\settowidth{\Width}{$5$}\setlength{\Width}{-1\Width}%
\settoheight{\Height}{$5$}\settodepth{\Depth}{$5$}\setlength{\Height}{-0.5\Height}\setlength{\Depth}{0.5\Depth}\addtolength{\Height}{\Depth}%
\put( -0.100,  5.000){\hspace*{\Width}\raisebox{\Height}{$5$}}%
%
\polyline(6,-0.05)(6,0.05)%
%
\settowidth{\Width}{$6$}\setlength{\Width}{-0.5\Width}%
\settoheight{\Height}{$6$}\settodepth{\Depth}{$6$}\setlength{\Height}{-\Height}%
\put(  6.000, -0.100){\hspace*{\Width}\raisebox{\Height}{$6$}}%
%
\polyline(-0.05,6)(0.05,6)%
%
\settowidth{\Width}{$6$}\setlength{\Width}{-1\Width}%
\settoheight{\Height}{$6$}\settodepth{\Depth}{$6$}\setlength{\Height}{-0.5\Height}\setlength{\Depth}{0.5\Depth}\addtolength{\Height}{\Depth}%
\put( -0.100,  6.000){\hspace*{\Width}\raisebox{\Height}{$6$}}%
%
\linethickness{0.002in}%%
{%
\color[cmyk]{0,1,1,0}%
\polygon*(-2.905,-5)(-2.906,-4.988)(-2.908,-4.976)(-2.912,-4.965)(-2.917,-4.954)(-2.923,-4.944)%
(-2.931,-4.935)(-2.939,-4.927)(-2.949,-4.92)(-2.96,-4.914)(-2.971,-4.91)(-2.982,-4.907)%
(-2.994,-4.905)(-3.006,-4.905)(-3.018,-4.907)(-3.029,-4.91)(-3.04,-4.914)(-3.051,-4.92)%
(-3.061,-4.927)(-3.069,-4.935)(-3.077,-4.944)(-3.083,-4.954)(-3.088,-4.965)(-3.092,-4.976)%
(-3.094,-4.988)(-3.095,-5)(-3.094,-5.012)(-3.092,-5.024)(-3.088,-5.035)(-3.083,-5.046)%
(-3.077,-5.056)(-3.069,-5.065)(-3.061,-5.073)(-3.051,-5.08)(-3.04,-5.086)(-3.029,-5.09)%
(-3.018,-5.093)(-3.006,-5.095)(-2.994,-5.095)(-2.982,-5.093)(-2.971,-5.09)(-2.96,-5.086)%
(-2.949,-5.08)(-2.939,-5.073)(-2.931,-5.065)(-2.923,-5.056)(-2.917,-5.046)(-2.912,-5.035)%
(-2.908,-5.024)(-2.906,-5.012)(-2.905,-5)(-2.905,-5)}%
\polyline(-2.905,-5)(-2.906,-4.988)(-2.908,-4.976)(-2.912,-4.965)(-2.917,-4.954)(-2.923,-4.944)%
(-2.931,-4.935)(-2.939,-4.927)(-2.949,-4.92)(-2.96,-4.914)(-2.971,-4.91)(-2.982,-4.907)%
(-2.994,-4.905)(-3.006,-4.905)(-3.018,-4.907)(-3.029,-4.91)(-3.04,-4.914)(-3.051,-4.92)%
(-3.061,-4.927)(-3.069,-4.935)(-3.077,-4.944)(-3.083,-4.954)(-3.088,-4.965)(-3.092,-4.976)%
(-3.094,-4.988)(-3.095,-5)(-3.094,-5.012)(-3.092,-5.024)(-3.088,-5.035)(-3.083,-5.046)%
(-3.077,-5.056)(-3.069,-5.065)(-3.061,-5.073)(-3.051,-5.08)(-3.04,-5.086)(-3.029,-5.09)%
(-3.018,-5.093)(-3.006,-5.095)(-2.994,-5.095)(-2.982,-5.093)(-2.971,-5.09)(-2.96,-5.086)%
(-2.949,-5.08)(-2.939,-5.073)(-2.931,-5.065)(-2.923,-5.056)(-2.917,-5.046)(-2.912,-5.035)%
(-2.908,-5.024)(-2.906,-5.012)(-2.905,-5)%
%
{%
\color[cmyk]{0,1,1,0}%
\polygon*(-1.905,-3)(-1.906,-2.988)(-1.908,-2.976)(-1.912,-2.965)(-1.917,-2.954)(-1.923,-2.944)%
(-1.931,-2.935)(-1.939,-2.927)(-1.949,-2.92)(-1.96,-2.914)(-1.971,-2.91)(-1.982,-2.907)%
(-1.994,-2.905)(-2.006,-2.905)(-2.018,-2.907)(-2.029,-2.91)(-2.04,-2.914)(-2.051,-2.92)%
(-2.061,-2.927)(-2.069,-2.935)(-2.077,-2.944)(-2.083,-2.954)(-2.088,-2.965)(-2.092,-2.976)%
(-2.094,-2.988)(-2.095,-3)(-2.094,-3.012)(-2.092,-3.024)(-2.088,-3.035)(-2.083,-3.046)%
(-2.077,-3.056)(-2.069,-3.065)(-2.061,-3.073)(-2.051,-3.08)(-2.04,-3.086)(-2.029,-3.09)%
(-2.018,-3.093)(-2.006,-3.095)(-1.994,-3.095)(-1.982,-3.093)(-1.971,-3.09)(-1.96,-3.086)%
(-1.949,-3.08)(-1.939,-3.073)(-1.931,-3.065)(-1.923,-3.056)(-1.917,-3.046)(-1.912,-3.035)%
(-1.908,-3.024)(-1.906,-3.012)(-1.905,-3)(-1.905,-3)}%
\polyline(-1.905,-3)(-1.906,-2.988)(-1.908,-2.976)(-1.912,-2.965)(-1.917,-2.954)(-1.923,-2.944)%
(-1.931,-2.935)(-1.939,-2.927)(-1.949,-2.92)(-1.96,-2.914)(-1.971,-2.91)(-1.982,-2.907)%
(-1.994,-2.905)(-2.006,-2.905)(-2.018,-2.907)(-2.029,-2.91)(-2.04,-2.914)(-2.051,-2.92)%
(-2.061,-2.927)(-2.069,-2.935)(-2.077,-2.944)(-2.083,-2.954)(-2.088,-2.965)(-2.092,-2.976)%
(-2.094,-2.988)(-2.095,-3)(-2.094,-3.012)(-2.092,-3.024)(-2.088,-3.035)(-2.083,-3.046)%
(-2.077,-3.056)(-2.069,-3.065)(-2.061,-3.073)(-2.051,-3.08)(-2.04,-3.086)(-2.029,-3.09)%
(-2.018,-3.093)(-2.006,-3.095)(-1.994,-3.095)(-1.982,-3.093)(-1.971,-3.09)(-1.96,-3.086)%
(-1.949,-3.08)(-1.939,-3.073)(-1.931,-3.065)(-1.923,-3.056)(-1.917,-3.046)(-1.912,-3.035)%
(-1.908,-3.024)(-1.906,-3.012)(-1.905,-3)%
%
{%
\color[cmyk]{0,1,1,0}%
\polygon*(-0.905,-1)(-0.906,-0.988)(-0.908,-0.976)(-0.912,-0.965)(-0.917,-0.954)(-0.923,-0.944)%
(-0.931,-0.935)(-0.939,-0.927)(-0.949,-0.92)(-0.96,-0.914)(-0.971,-0.91)(-0.982,-0.907)%
(-0.994,-0.905)(-1.006,-0.905)(-1.018,-0.907)(-1.029,-0.91)(-1.04,-0.914)(-1.051,-0.92)%
(-1.061,-0.927)(-1.069,-0.935)(-1.077,-0.944)(-1.083,-0.954)(-1.088,-0.965)(-1.092,-0.976)%
(-1.094,-0.988)(-1.095,-1)(-1.094,-1.012)(-1.092,-1.024)(-1.088,-1.035)(-1.083,-1.046)%
(-1.077,-1.056)(-1.069,-1.065)(-1.061,-1.073)(-1.051,-1.08)(-1.04,-1.086)(-1.029,-1.09)%
(-1.018,-1.093)(-1.006,-1.095)(-0.994,-1.095)(-0.982,-1.093)(-0.971,-1.09)(-0.96,-1.086)%
(-0.949,-1.08)(-0.939,-1.073)(-0.931,-1.065)(-0.923,-1.056)(-0.917,-1.046)(-0.912,-1.035)%
(-0.908,-1.024)(-0.906,-1.012)(-0.905,-1)(-0.905,-1)}%
\polyline(-0.905,-1)(-0.906,-0.988)(-0.908,-0.976)(-0.912,-0.965)(-0.917,-0.954)(-0.923,-0.944)%
(-0.931,-0.935)(-0.939,-0.927)(-0.949,-0.92)(-0.96,-0.914)(-0.971,-0.91)(-0.982,-0.907)%
(-0.994,-0.905)(-1.006,-0.905)(-1.018,-0.907)(-1.029,-0.91)(-1.04,-0.914)(-1.051,-0.92)%
(-1.061,-0.927)(-1.069,-0.935)(-1.077,-0.944)(-1.083,-0.954)(-1.088,-0.965)(-1.092,-0.976)%
(-1.094,-0.988)(-1.095,-1)(-1.094,-1.012)(-1.092,-1.024)(-1.088,-1.035)(-1.083,-1.046)%
(-1.077,-1.056)(-1.069,-1.065)(-1.061,-1.073)(-1.051,-1.08)(-1.04,-1.086)(-1.029,-1.09)%
(-1.018,-1.093)(-1.006,-1.095)(-0.994,-1.095)(-0.982,-1.093)(-0.971,-1.09)(-0.96,-1.086)%
(-0.949,-1.08)(-0.939,-1.073)(-0.931,-1.065)(-0.923,-1.056)(-0.917,-1.046)(-0.912,-1.035)%
(-0.908,-1.024)(-0.906,-1.012)(-0.905,-1)%
%
{%
\color[cmyk]{0,1,1,0}%
\polygon*(0.095,1)(0.094,1.012)(0.092,1.024)(0.088,1.035)(0.083,1.046)(0.077,1.056)%
(0.069,1.065)(0.061,1.073)(0.051,1.08)(0.04,1.086)(0.029,1.09)(0.018,1.093)(0.006,1.095)%
(-0.006,1.095)(-0.018,1.093)(-0.029,1.09)(-0.04,1.086)(-0.051,1.08)(-0.061,1.073)%
(-0.069,1.065)(-0.077,1.056)(-0.083,1.046)(-0.088,1.035)(-0.092,1.024)(-0.094,1.012)%
(-0.095,1)(-0.094,0.988)(-0.092,0.976)(-0.088,0.965)(-0.083,0.954)(-0.077,0.944)(-0.069,0.935)%
(-0.061,0.927)(-0.051,0.92)(-0.04,0.914)(-0.029,0.91)(-0.018,0.907)(-0.006,0.905)%
(0.006,0.905)(0.018,0.907)(0.029,0.91)(0.04,0.914)(0.051,0.92)(0.061,0.927)(0.069,0.935)%
(0.077,0.944)(0.083,0.954)(0.088,0.965)(0.092,0.976)(0.094,0.988)(0.095,1)(0.095,1)%
}%
\polyline(0.095,1)(0.094,1.012)(0.092,1.024)(0.088,1.035)(0.083,1.046)(0.077,1.056)%
(0.069,1.065)(0.061,1.073)(0.051,1.08)(0.04,1.086)(0.029,1.09)(0.018,1.093)(0.006,1.095)%
(-0.006,1.095)(-0.018,1.093)(-0.029,1.09)(-0.04,1.086)(-0.051,1.08)(-0.061,1.073)%
(-0.069,1.065)(-0.077,1.056)(-0.083,1.046)(-0.088,1.035)(-0.092,1.024)(-0.094,1.012)%
(-0.095,1)(-0.094,0.988)(-0.092,0.976)(-0.088,0.965)(-0.083,0.954)(-0.077,0.944)(-0.069,0.935)%
(-0.061,0.927)(-0.051,0.92)(-0.04,0.914)(-0.029,0.91)(-0.018,0.907)(-0.006,0.905)%
(0.006,0.905)(0.018,0.907)(0.029,0.91)(0.04,0.914)(0.051,0.92)(0.061,0.927)(0.069,0.935)%
(0.077,0.944)(0.083,0.954)(0.088,0.965)(0.092,0.976)(0.094,0.988)(0.095,1)%
%
{%
\color[cmyk]{0,1,1,0}%
\polygon*(1.095,3)(1.094,3.012)(1.092,3.024)(1.088,3.035)(1.083,3.046)(1.077,3.056)%
(1.069,3.065)(1.061,3.073)(1.051,3.08)(1.04,3.086)(1.029,3.09)(1.018,3.093)(1.006,3.095)%
(0.994,3.095)(0.982,3.093)(0.971,3.09)(0.96,3.086)(0.949,3.08)(0.939,3.073)(0.931,3.065)%
(0.923,3.056)(0.917,3.046)(0.912,3.035)(0.908,3.024)(0.906,3.012)(0.905,3)(0.906,2.988)%
(0.908,2.976)(0.912,2.965)(0.917,2.954)(0.923,2.944)(0.931,2.935)(0.939,2.927)(0.949,2.92)%
(0.96,2.914)(0.971,2.91)(0.982,2.907)(0.994,2.905)(1.006,2.905)(1.018,2.907)(1.029,2.91)%
(1.04,2.914)(1.051,2.92)(1.061,2.927)(1.069,2.935)(1.077,2.944)(1.083,2.954)(1.088,2.965)%
(1.092,2.976)(1.094,2.988)(1.095,3)(1.095,3)}%
\polyline(1.095,3)(1.094,3.012)(1.092,3.024)(1.088,3.035)(1.083,3.046)(1.077,3.056)%
(1.069,3.065)(1.061,3.073)(1.051,3.08)(1.04,3.086)(1.029,3.09)(1.018,3.093)(1.006,3.095)%
(0.994,3.095)(0.982,3.093)(0.971,3.09)(0.96,3.086)(0.949,3.08)(0.939,3.073)(0.931,3.065)%
(0.923,3.056)(0.917,3.046)(0.912,3.035)(0.908,3.024)(0.906,3.012)(0.905,3)(0.906,2.988)%
(0.908,2.976)(0.912,2.965)(0.917,2.954)(0.923,2.944)(0.931,2.935)(0.939,2.927)(0.949,2.92)%
(0.96,2.914)(0.971,2.91)(0.982,2.907)(0.994,2.905)(1.006,2.905)(1.018,2.907)(1.029,2.91)%
(1.04,2.914)(1.051,2.92)(1.061,2.927)(1.069,2.935)(1.077,2.944)(1.083,2.954)(1.088,2.965)%
(1.092,2.976)(1.094,2.988)(1.095,3)%
%
{%
\color[cmyk]{0,1,1,0}%
\polygon*(2.095,5)(2.094,5.012)(2.092,5.024)(2.088,5.035)(2.083,5.046)(2.077,5.056)%
(2.069,5.065)(2.061,5.073)(2.051,5.08)(2.04,5.086)(2.029,5.09)(2.018,5.093)(2.006,5.095)%
(1.994,5.095)(1.982,5.093)(1.971,5.09)(1.96,5.086)(1.949,5.08)(1.939,5.073)(1.931,5.065)%
(1.923,5.056)(1.917,5.046)(1.912,5.035)(1.908,5.024)(1.906,5.012)(1.905,5)(1.906,4.988)%
(1.908,4.976)(1.912,4.965)(1.917,4.954)(1.923,4.944)(1.931,4.935)(1.939,4.927)(1.949,4.92)%
(1.96,4.914)(1.971,4.91)(1.982,4.907)(1.994,4.905)(2.006,4.905)(2.018,4.907)(2.029,4.91)%
(2.04,4.914)(2.051,4.92)(2.061,4.927)(2.069,4.935)(2.077,4.944)(2.083,4.954)(2.088,4.965)%
(2.092,4.976)(2.094,4.988)(2.095,5)(2.095,5)}%
\polyline(2.095,5)(2.094,5.012)(2.092,5.024)(2.088,5.035)(2.083,5.046)(2.077,5.056)%
(2.069,5.065)(2.061,5.073)(2.051,5.08)(2.04,5.086)(2.029,5.09)(2.018,5.093)(2.006,5.095)%
(1.994,5.095)(1.982,5.093)(1.971,5.09)(1.96,5.086)(1.949,5.08)(1.939,5.073)(1.931,5.065)%
(1.923,5.056)(1.917,5.046)(1.912,5.035)(1.908,5.024)(1.906,5.012)(1.905,5)(1.906,4.988)%
(1.908,4.976)(1.912,4.965)(1.917,4.954)(1.923,4.944)(1.931,4.935)(1.939,4.927)(1.949,4.92)%
(1.96,4.914)(1.971,4.91)(1.982,4.907)(1.994,4.905)(2.006,4.905)(2.018,4.907)(2.029,4.91)%
(2.04,4.914)(2.051,4.92)(2.061,4.927)(2.069,4.935)(2.077,4.944)(2.083,4.954)(2.088,4.965)%
(2.092,4.976)(2.094,4.988)(2.095,5)%
%
\linethickness{0.008in}%%
\polyline(-6.2,0)(6.2,0)%
%
\polyline(0,-6.2)(0,6.2)%
%
\settowidth{\Width}{$x$}\setlength{\Width}{0\Width}%
\settoheight{\Height}{$x$}\settodepth{\Depth}{$x$}\setlength{\Height}{-0.5\Height}\setlength{\Depth}{0.5\Depth}\addtolength{\Height}{\Depth}%
\put(  6.250,  0.000){\hspace*{\Width}\raisebox{\Height}{$x$}}%
%
\settowidth{\Width}{$y$}\setlength{\Width}{-0.5\Width}%
\settoheight{\Height}{$y$}\settodepth{\Depth}{$y$}\setlength{\Height}{\Depth}%
\put(  0.000,  6.250){\hspace*{\Width}\raisebox{\Height}{$y$}}%
%
\settowidth{\Width}{O}\setlength{\Width}{-1\Width}%
\settoheight{\Height}{O}\settodepth{\Depth}{O}\setlength{\Height}{-\Height}%
\put( -0.050, -0.050){\hspace*{\Width}\raisebox{\Height}{O}}%
%
\end{picture}}%}}
\end{layer}


\sameslide

\vspace*{18mm}

\slidepage
\down
例)$y=2x+1$

\begin{layer}{120}{0}
\putnotese{70}{-3}{\scalebox{0.6}{%%% /Users/takatoosetsuo/polytech23.git/101-0417/presen/fig/table1b.tex 
%%% Generator=presen23101.cdy 
{\unitlength=1cm%
\begin{picture}%
(9.6,1.2)(0,0)%
\linethickness{0.008in}%%
\Large\bf\boldmath%
\small%
\polyline(0,1.2)(0,0)%
%
\polyline(0.8,1.2)(0.8,0)%
%
\polyline(1.6,1.2)(1.6,0)%
%
\polyline(2.4,1.2)(2.4,0)%
%
\polyline(3.2,1.2)(3.2,0)%
%
\polyline(4,1.2)(4,0)%
%
\polyline(4.8,1.2)(4.8,0)%
%
\polyline(5.6,1.2)(5.6,0)%
%
\polyline(6.4,1.2)(6.4,0)%
%
\polyline(7.2,1.2)(7.2,0)%
%
\polyline(8,1.2)(8,0)%
%
\polyline(8.8,1.2)(8.8,0)%
%
\polyline(9.6,1.2)(9.6,0)%
%
\polyline(0,1.2)(9.6,1.2)%
%
\polyline(0,0.6)(9.6,0.6)%
%
\polyline(0,0)(9.6,0)%
%
\settowidth{\Width}{$x$}\setlength{\Width}{-0.5\Width}%
\settoheight{\Height}{$x$}\settodepth{\Depth}{$x$}\setlength{\Height}{-0.5\Height}\setlength{\Depth}{0.5\Depth}\addtolength{\Height}{\Depth}%
\put(  0.400,  0.900){\hspace*{\Width}\raisebox{\Height}{$x$}}%
%
\settowidth{\Width}{$-5$}\setlength{\Width}{-0.5\Width}%
\settoheight{\Height}{$-5$}\settodepth{\Depth}{$-5$}\setlength{\Height}{-0.5\Height}\setlength{\Depth}{0.5\Depth}\addtolength{\Height}{\Depth}%
\put(  1.200,  0.900){\hspace*{\Width}\raisebox{\Height}{$-5$}}%
%
\settowidth{\Width}{$-4$}\setlength{\Width}{-0.5\Width}%
\settoheight{\Height}{$-4$}\settodepth{\Depth}{$-4$}\setlength{\Height}{-0.5\Height}\setlength{\Depth}{0.5\Depth}\addtolength{\Height}{\Depth}%
\put(  2.000,  0.900){\hspace*{\Width}\raisebox{\Height}{$-4$}}%
%
\settowidth{\Width}{$-3$}\setlength{\Width}{-0.5\Width}%
\settoheight{\Height}{$-3$}\settodepth{\Depth}{$-3$}\setlength{\Height}{-0.5\Height}\setlength{\Depth}{0.5\Depth}\addtolength{\Height}{\Depth}%
\put(  2.800,  0.900){\hspace*{\Width}\raisebox{\Height}{$-3$}}%
%
\settowidth{\Width}{$-2$}\setlength{\Width}{-0.5\Width}%
\settoheight{\Height}{$-2$}\settodepth{\Depth}{$-2$}\setlength{\Height}{-0.5\Height}\setlength{\Depth}{0.5\Depth}\addtolength{\Height}{\Depth}%
\put(  3.600,  0.900){\hspace*{\Width}\raisebox{\Height}{$-2$}}%
%
\settowidth{\Width}{$-1$}\setlength{\Width}{-0.5\Width}%
\settoheight{\Height}{$-1$}\settodepth{\Depth}{$-1$}\setlength{\Height}{-0.5\Height}\setlength{\Depth}{0.5\Depth}\addtolength{\Height}{\Depth}%
\put(  4.400,  0.900){\hspace*{\Width}\raisebox{\Height}{$-1$}}%
%
\settowidth{\Width}{$0$}\setlength{\Width}{-0.5\Width}%
\settoheight{\Height}{$0$}\settodepth{\Depth}{$0$}\setlength{\Height}{-0.5\Height}\setlength{\Depth}{0.5\Depth}\addtolength{\Height}{\Depth}%
\put(  5.200,  0.900){\hspace*{\Width}\raisebox{\Height}{$0$}}%
%
\settowidth{\Width}{$1$}\setlength{\Width}{-0.5\Width}%
\settoheight{\Height}{$1$}\settodepth{\Depth}{$1$}\setlength{\Height}{-0.5\Height}\setlength{\Depth}{0.5\Depth}\addtolength{\Height}{\Depth}%
\put(  6.000,  0.900){\hspace*{\Width}\raisebox{\Height}{$1$}}%
%
\settowidth{\Width}{$2$}\setlength{\Width}{-0.5\Width}%
\settoheight{\Height}{$2$}\settodepth{\Depth}{$2$}\setlength{\Height}{-0.5\Height}\setlength{\Depth}{0.5\Depth}\addtolength{\Height}{\Depth}%
\put(  6.800,  0.900){\hspace*{\Width}\raisebox{\Height}{$2$}}%
%
\settowidth{\Width}{$3$}\setlength{\Width}{-0.5\Width}%
\settoheight{\Height}{$3$}\settodepth{\Depth}{$3$}\setlength{\Height}{-0.5\Height}\setlength{\Depth}{0.5\Depth}\addtolength{\Height}{\Depth}%
\put(  7.600,  0.900){\hspace*{\Width}\raisebox{\Height}{$3$}}%
%
\settowidth{\Width}{$4$}\setlength{\Width}{-0.5\Width}%
\settoheight{\Height}{$4$}\settodepth{\Depth}{$4$}\setlength{\Height}{-0.5\Height}\setlength{\Depth}{0.5\Depth}\addtolength{\Height}{\Depth}%
\put(  8.400,  0.900){\hspace*{\Width}\raisebox{\Height}{$4$}}%
%
\settowidth{\Width}{$5$}\setlength{\Width}{-0.5\Width}%
\settoheight{\Height}{$5$}\settodepth{\Depth}{$5$}\setlength{\Height}{-0.5\Height}\setlength{\Depth}{0.5\Depth}\addtolength{\Height}{\Depth}%
\put(  9.200,  0.900){\hspace*{\Width}\raisebox{\Height}{$5$}}%
%
\settowidth{\Width}{$y$}\setlength{\Width}{-0.5\Width}%
\settoheight{\Height}{$y$}\settodepth{\Depth}{$y$}\setlength{\Height}{-0.5\Height}\setlength{\Depth}{0.5\Depth}\addtolength{\Height}{\Depth}%
\put(  0.400,  0.300){\hspace*{\Width}\raisebox{\Height}{$y$}}%
%
\settowidth{\Width}{$-9$}\setlength{\Width}{-0.5\Width}%
\settoheight{\Height}{$-9$}\settodepth{\Depth}{$-9$}\setlength{\Height}{-0.5\Height}\setlength{\Depth}{0.5\Depth}\addtolength{\Height}{\Depth}%
\put(  1.200,  0.300){\hspace*{\Width}\raisebox{\Height}{$-9$}}%
%
\settowidth{\Width}{$-7$}\setlength{\Width}{-0.5\Width}%
\settoheight{\Height}{$-7$}\settodepth{\Depth}{$-7$}\setlength{\Height}{-0.5\Height}\setlength{\Depth}{0.5\Depth}\addtolength{\Height}{\Depth}%
\put(  2.000,  0.300){\hspace*{\Width}\raisebox{\Height}{$-7$}}%
%
\settowidth{\Width}{$-5$}\setlength{\Width}{-0.5\Width}%
\settoheight{\Height}{$-5$}\settodepth{\Depth}{$-5$}\setlength{\Height}{-0.5\Height}\setlength{\Depth}{0.5\Depth}\addtolength{\Height}{\Depth}%
\put(  2.800,  0.300){\hspace*{\Width}\raisebox{\Height}{$-5$}}%
%
\settowidth{\Width}{$-3$}\setlength{\Width}{-0.5\Width}%
\settoheight{\Height}{$-3$}\settodepth{\Depth}{$-3$}\setlength{\Height}{-0.5\Height}\setlength{\Depth}{0.5\Depth}\addtolength{\Height}{\Depth}%
\put(  3.600,  0.300){\hspace*{\Width}\raisebox{\Height}{$-3$}}%
%
\settowidth{\Width}{$-1$}\setlength{\Width}{-0.5\Width}%
\settoheight{\Height}{$-1$}\settodepth{\Depth}{$-1$}\setlength{\Height}{-0.5\Height}\setlength{\Depth}{0.5\Depth}\addtolength{\Height}{\Depth}%
\put(  4.400,  0.300){\hspace*{\Width}\raisebox{\Height}{$-1$}}%
%
\settowidth{\Width}{$1$}\setlength{\Width}{-0.5\Width}%
\settoheight{\Height}{$1$}\settodepth{\Depth}{$1$}\setlength{\Height}{-0.5\Height}\setlength{\Depth}{0.5\Depth}\addtolength{\Height}{\Depth}%
\put(  5.200,  0.300){\hspace*{\Width}\raisebox{\Height}{$1$}}%
%
\settowidth{\Width}{$3$}\setlength{\Width}{-0.5\Width}%
\settoheight{\Height}{$3$}\settodepth{\Depth}{$3$}\setlength{\Height}{-0.5\Height}\setlength{\Depth}{0.5\Depth}\addtolength{\Height}{\Depth}%
\put(  6.000,  0.300){\hspace*{\Width}\raisebox{\Height}{$3$}}%
%
\settowidth{\Width}{$5$}\setlength{\Width}{-0.5\Width}%
\settoheight{\Height}{$5$}\settodepth{\Depth}{$5$}\setlength{\Height}{-0.5\Height}\setlength{\Depth}{0.5\Depth}\addtolength{\Height}{\Depth}%
\put(  6.800,  0.300){\hspace*{\Width}\raisebox{\Height}{$5$}}%
%
\settowidth{\Width}{$7$}\setlength{\Width}{-0.5\Width}%
\settoheight{\Height}{$7$}\settodepth{\Depth}{$7$}\setlength{\Height}{-0.5\Height}\setlength{\Depth}{0.5\Depth}\addtolength{\Height}{\Depth}%
\put(  7.600,  0.300){\hspace*{\Width}\raisebox{\Height}{$7$}}%
%
\settowidth{\Width}{$9$}\setlength{\Width}{-0.5\Width}%
\settoheight{\Height}{$9$}\settodepth{\Depth}{$9$}\setlength{\Height}{-0.5\Height}\setlength{\Depth}{0.5\Depth}\addtolength{\Height}{\Depth}%
\put(  8.400,  0.300){\hspace*{\Width}\raisebox{\Height}{$9$}}%
%
\settowidth{\Width}{$11$}\setlength{\Width}{-0.5\Width}%
\settoheight{\Height}{$11$}\settodepth{\Depth}{$11$}\setlength{\Height}{-0.5\Height}\setlength{\Depth}{0.5\Depth}\addtolength{\Height}{\Depth}%
\put(  9.200,  0.300){\hspace*{\Width}\raisebox{\Height}{$11$}}%
%
\end{picture}}%}}
\putnotes{60}{6}{\scalebox{0.5}{%%% /polytech.git/n101/fig/graphpaper3.tex 
%%% Generator=presen0601.cdy 
{\unitlength=1cm%
\begin{picture}%
(12.4,12.4)(-6.2,-6.2)%
\special{pn 8}%
%
\Large\bf\boldmath%
\small%
{%
\color[rgb]{0,0,0}%
\special{pn 4}%
\special{pa -2362 -2362}\special{pa -2362  2362}%
\special{fp}%
\special{pn 8}%
}%
{%
\color[rgb]{0,0,0}%
\special{pn 4}%
\special{pa -1969 -2362}\special{pa -1969  2362}%
\special{fp}%
\special{pn 8}%
}%
{%
\color[rgb]{0,0,0}%
\special{pn 4}%
\special{pa -1575 -2362}\special{pa -1575  2362}%
\special{fp}%
\special{pn 8}%
}%
{%
\color[rgb]{0,0,0}%
\special{pn 4}%
\special{pa -1181 -2362}\special{pa -1181  2362}%
\special{fp}%
\special{pn 8}%
}%
{%
\color[rgb]{0,0,0}%
\special{pn 4}%
\special{pa  -787 -2362}\special{pa  -787  2362}%
\special{fp}%
\special{pn 8}%
}%
{%
\color[rgb]{0,0,0}%
\special{pn 4}%
\special{pa  -394 -2362}\special{pa  -394  2362}%
\special{fp}%
\special{pn 8}%
}%
{%
\color[rgb]{0,0,0}%
\special{pn 4}%
\special{pa     0 -2362}\special{pa     0  2362}%
\special{fp}%
\special{pn 8}%
}%
{%
\color[rgb]{0,0,0}%
\special{pn 4}%
\special{pa   394 -2362}\special{pa   394  2362}%
\special{fp}%
\special{pn 8}%
}%
{%
\color[rgb]{0,0,0}%
\special{pn 4}%
\special{pa   787 -2362}\special{pa   787  2362}%
\special{fp}%
\special{pn 8}%
}%
{%
\color[rgb]{0,0,0}%
\special{pn 4}%
\special{pa  1181 -2362}\special{pa  1181  2362}%
\special{fp}%
\special{pn 8}%
}%
{%
\color[rgb]{0,0,0}%
\special{pn 4}%
\special{pa  1575 -2362}\special{pa  1575  2362}%
\special{fp}%
\special{pn 8}%
}%
{%
\color[rgb]{0,0,0}%
\special{pn 4}%
\special{pa  1969 -2362}\special{pa  1969  2362}%
\special{fp}%
\special{pn 8}%
}%
{%
\color[rgb]{0,0,0}%
\special{pn 4}%
\special{pa  2362 -2362}\special{pa  2362  2362}%
\special{fp}%
\special{pn 8}%
}%
{%
\color[rgb]{0,0,0}%
\special{pn 4}%
\special{pa -2362 -2362}\special{pa  2362 -2362}%
\special{fp}%
\special{pn 8}%
}%
{%
\color[rgb]{0,0,0}%
\special{pn 4}%
\special{pa -2362 -1969}\special{pa  2362 -1969}%
\special{fp}%
\special{pn 8}%
}%
{%
\color[rgb]{0,0,0}%
\special{pn 4}%
\special{pa -2362 -1575}\special{pa  2362 -1575}%
\special{fp}%
\special{pn 8}%
}%
{%
\color[rgb]{0,0,0}%
\special{pn 4}%
\special{pa -2362 -1181}\special{pa  2362 -1181}%
\special{fp}%
\special{pn 8}%
}%
{%
\color[rgb]{0,0,0}%
\special{pn 4}%
\special{pa -2362  -787}\special{pa  2362  -787}%
\special{fp}%
\special{pn 8}%
}%
{%
\color[rgb]{0,0,0}%
\special{pn 4}%
\special{pa -2362  -394}\special{pa  2362  -394}%
\special{fp}%
\special{pn 8}%
}%
{%
\color[rgb]{0,0,0}%
\special{pn 4}%
\special{pa -2362    -0}\special{pa  2362    -0}%
\special{fp}%
\special{pn 8}%
}%
{%
\color[rgb]{0,0,0}%
\special{pn 4}%
\special{pa -2362   394}\special{pa  2362   394}%
\special{fp}%
\special{pn 8}%
}%
{%
\color[rgb]{0,0,0}%
\special{pn 4}%
\special{pa -2362   787}\special{pa  2362   787}%
\special{fp}%
\special{pn 8}%
}%
{%
\color[rgb]{0,0,0}%
\special{pn 4}%
\special{pa -2362  1181}\special{pa  2362  1181}%
\special{fp}%
\special{pn 8}%
}%
{%
\color[rgb]{0,0,0}%
\special{pn 4}%
\special{pa -2362  1575}\special{pa  2362  1575}%
\special{fp}%
\special{pn 8}%
}%
{%
\color[rgb]{0,0,0}%
\special{pn 4}%
\special{pa -2362  1969}\special{pa  2362  1969}%
\special{fp}%
\special{pn 8}%
}%
{%
\color[rgb]{0,0,0}%
\special{pn 4}%
\special{pa -2362  2362}\special{pa  2362  2362}%
\special{fp}%
\special{pn 8}%
}%
\special{pn 4}%
{%
\color[rgb]{0,0,0}%
\special{pa -2165 2362}\special{pa -2165 2323}\special{fp}\special{pa -2165 2284}\special{pa -2165 2245}\special{fp}%
\special{pa -2165 2206}\special{pa -2165 2167}\special{fp}\special{pa -2165 2128}\special{pa -2165 2089}\special{fp}%
\special{pa -2165 2050}\special{pa -2165 2011}\special{fp}\special{pa -2165 1972}\special{pa -2165 1933}\special{fp}%
\special{pa -2165 1894}\special{pa -2165 1855}\special{fp}\special{pa -2165 1816}\special{pa -2165 1777}\special{fp}%
\special{pa -2165 1737}\special{pa -2165 1698}\special{fp}\special{pa -2165 1659}\special{pa -2165 1620}\special{fp}%
\special{pa -2165 1581}\special{pa -2165 1542}\special{fp}\special{pa -2165 1503}\special{pa -2165 1464}\special{fp}%
\special{pa -2165 1425}\special{pa -2165 1386}\special{fp}\special{pa -2165 1347}\special{pa -2165 1308}\special{fp}%
\special{pa -2165 1269}\special{pa -2165 1230}\special{fp}\special{pa -2165 1191}\special{pa -2165 1152}\special{fp}%
\special{pa -2165 1113}\special{pa -2165 1074}\special{fp}\special{pa -2165 1035}\special{pa -2165 996}\special{fp}%
\special{pa -2165 957}\special{pa -2165 918}\special{fp}\special{pa -2165 879}\special{pa -2165 839}\special{fp}%
\special{pa -2165 800}\special{pa -2165 761}\special{fp}\special{pa -2165 722}\special{pa -2165 683}\special{fp}%
\special{pa -2165 644}\special{pa -2165 605}\special{fp}\special{pa -2165 566}\special{pa -2165 527}\special{fp}%
\special{pa -2165 488}\special{pa -2165 449}\special{fp}\special{pa -2165 410}\special{pa -2165 371}\special{fp}%
\special{pa -2165 332}\special{pa -2165 293}\special{fp}\special{pa -2165 254}\special{pa -2165 215}\special{fp}%
\special{pa -2165 176}\special{pa -2165 137}\special{fp}\special{pa -2165 98}\special{pa -2165 59}\special{fp}%
\special{pa -2165 20}\special{pa -2165 -20}\special{fp}\special{pa -2165 -59}\special{pa -2165 -98}\special{fp}%
\special{pa -2165 -137}\special{pa -2165 -176}\special{fp}\special{pa -2165 -215}\special{pa -2165 -254}\special{fp}%
\special{pa -2165 -293}\special{pa -2165 -332}\special{fp}\special{pa -2165 -371}\special{pa -2165 -410}\special{fp}%
\special{pa -2165 -449}\special{pa -2165 -488}\special{fp}\special{pa -2165 -527}\special{pa -2165 -566}\special{fp}%
\special{pa -2165 -605}\special{pa -2165 -644}\special{fp}\special{pa -2165 -683}\special{pa -2165 -722}\special{fp}%
\special{pa -2165 -761}\special{pa -2165 -800}\special{fp}\special{pa -2165 -839}\special{pa -2165 -879}\special{fp}%
\special{pa -2165 -918}\special{pa -2165 -957}\special{fp}\special{pa -2165 -996}\special{pa -2165 -1035}\special{fp}%
\special{pa -2165 -1074}\special{pa -2165 -1113}\special{fp}\special{pa -2165 -1152}\special{pa -2165 -1191}\special{fp}%
\special{pa -2165 -1230}\special{pa -2165 -1269}\special{fp}\special{pa -2165 -1308}\special{pa -2165 -1347}\special{fp}%
\special{pa -2165 -1386}\special{pa -2165 -1425}\special{fp}\special{pa -2165 -1464}\special{pa -2165 -1503}\special{fp}%
\special{pa -2165 -1542}\special{pa -2165 -1581}\special{fp}\special{pa -2165 -1620}\special{pa -2165 -1659}\special{fp}%
\special{pa -2165 -1698}\special{pa -2165 -1737}\special{fp}\special{pa -2165 -1777}\special{pa -2165 -1816}\special{fp}%
\special{pa -2165 -1855}\special{pa -2165 -1894}\special{fp}\special{pa -2165 -1933}\special{pa -2165 -1972}\special{fp}%
\special{pa -2165 -2011}\special{pa -2165 -2050}\special{fp}\special{pa -2165 -2089}\special{pa -2165 -2128}\special{fp}%
\special{pa -2165 -2167}\special{pa -2165 -2206}\special{fp}\special{pa -2165 -2245}\special{pa -2165 -2284}\special{fp}%
\special{pa -2165 -2323}\special{pa -2165 -2362}\special{fp}%
%
}%
{%
\color[rgb]{0,0,0}%
\special{pa -2362 2165}\special{pa -2323 2165}\special{fp}\special{pa -2284 2165}\special{pa -2245 2165}\special{fp}%
\special{pa -2206 2165}\special{pa -2167 2165}\special{fp}\special{pa -2128 2165}\special{pa -2089 2165}\special{fp}%
\special{pa -2050 2165}\special{pa -2011 2165}\special{fp}\special{pa -1972 2165}\special{pa -1933 2165}\special{fp}%
\special{pa -1894 2165}\special{pa -1855 2165}\special{fp}\special{pa -1816 2165}\special{pa -1777 2165}\special{fp}%
\special{pa -1737 2165}\special{pa -1698 2165}\special{fp}\special{pa -1659 2165}\special{pa -1620 2165}\special{fp}%
\special{pa -1581 2165}\special{pa -1542 2165}\special{fp}\special{pa -1503 2165}\special{pa -1464 2165}\special{fp}%
\special{pa -1425 2165}\special{pa -1386 2165}\special{fp}\special{pa -1347 2165}\special{pa -1308 2165}\special{fp}%
\special{pa -1269 2165}\special{pa -1230 2165}\special{fp}\special{pa -1191 2165}\special{pa -1152 2165}\special{fp}%
\special{pa -1113 2165}\special{pa -1074 2165}\special{fp}\special{pa -1035 2165}\special{pa -996 2165}\special{fp}%
\special{pa -957 2165}\special{pa -918 2165}\special{fp}\special{pa -879 2165}\special{pa -839 2165}\special{fp}%
\special{pa -800 2165}\special{pa -761 2165}\special{fp}\special{pa -722 2165}\special{pa -683 2165}\special{fp}%
\special{pa -644 2165}\special{pa -605 2165}\special{fp}\special{pa -566 2165}\special{pa -527 2165}\special{fp}%
\special{pa -488 2165}\special{pa -449 2165}\special{fp}\special{pa -410 2165}\special{pa -371 2165}\special{fp}%
\special{pa -332 2165}\special{pa -293 2165}\special{fp}\special{pa -254 2165}\special{pa -215 2165}\special{fp}%
\special{pa -176 2165}\special{pa -137 2165}\special{fp}\special{pa -98 2165}\special{pa -59 2165}\special{fp}%
\special{pa -20 2165}\special{pa 20 2165}\special{fp}\special{pa 59 2165}\special{pa 98 2165}\special{fp}%
\special{pa 137 2165}\special{pa 176 2165}\special{fp}\special{pa 215 2165}\special{pa 254 2165}\special{fp}%
\special{pa 293 2165}\special{pa 332 2165}\special{fp}\special{pa 371 2165}\special{pa 410 2165}\special{fp}%
\special{pa 449 2165}\special{pa 488 2165}\special{fp}\special{pa 527 2165}\special{pa 566 2165}\special{fp}%
\special{pa 605 2165}\special{pa 644 2165}\special{fp}\special{pa 683 2165}\special{pa 722 2165}\special{fp}%
\special{pa 761 2165}\special{pa 800 2165}\special{fp}\special{pa 839 2165}\special{pa 879 2165}\special{fp}%
\special{pa 918 2165}\special{pa 957 2165}\special{fp}\special{pa 996 2165}\special{pa 1035 2165}\special{fp}%
\special{pa 1074 2165}\special{pa 1113 2165}\special{fp}\special{pa 1152 2165}\special{pa 1191 2165}\special{fp}%
\special{pa 1230 2165}\special{pa 1269 2165}\special{fp}\special{pa 1308 2165}\special{pa 1347 2165}\special{fp}%
\special{pa 1386 2165}\special{pa 1425 2165}\special{fp}\special{pa 1464 2165}\special{pa 1503 2165}\special{fp}%
\special{pa 1542 2165}\special{pa 1581 2165}\special{fp}\special{pa 1620 2165}\special{pa 1659 2165}\special{fp}%
\special{pa 1698 2165}\special{pa 1737 2165}\special{fp}\special{pa 1777 2165}\special{pa 1816 2165}\special{fp}%
\special{pa 1855 2165}\special{pa 1894 2165}\special{fp}\special{pa 1933 2165}\special{pa 1972 2165}\special{fp}%
\special{pa 2011 2165}\special{pa 2050 2165}\special{fp}\special{pa 2089 2165}\special{pa 2128 2165}\special{fp}%
\special{pa 2167 2165}\special{pa 2206 2165}\special{fp}\special{pa 2245 2165}\special{pa 2284 2165}\special{fp}%
\special{pa 2323 2165}\special{pa 2362 2165}\special{fp}%
%
}%
{%
\color[rgb]{0,0,0}%
\special{pa -1772 2362}\special{pa -1772 2323}\special{fp}\special{pa -1772 2284}\special{pa -1772 2245}\special{fp}%
\special{pa -1772 2206}\special{pa -1772 2167}\special{fp}\special{pa -1772 2128}\special{pa -1772 2089}\special{fp}%
\special{pa -1772 2050}\special{pa -1772 2011}\special{fp}\special{pa -1772 1972}\special{pa -1772 1933}\special{fp}%
\special{pa -1772 1894}\special{pa -1772 1855}\special{fp}\special{pa -1772 1816}\special{pa -1772 1777}\special{fp}%
\special{pa -1772 1737}\special{pa -1772 1698}\special{fp}\special{pa -1772 1659}\special{pa -1772 1620}\special{fp}%
\special{pa -1772 1581}\special{pa -1772 1542}\special{fp}\special{pa -1772 1503}\special{pa -1772 1464}\special{fp}%
\special{pa -1772 1425}\special{pa -1772 1386}\special{fp}\special{pa -1772 1347}\special{pa -1772 1308}\special{fp}%
\special{pa -1772 1269}\special{pa -1772 1230}\special{fp}\special{pa -1772 1191}\special{pa -1772 1152}\special{fp}%
\special{pa -1772 1113}\special{pa -1772 1074}\special{fp}\special{pa -1772 1035}\special{pa -1772 996}\special{fp}%
\special{pa -1772 957}\special{pa -1772 918}\special{fp}\special{pa -1772 879}\special{pa -1772 839}\special{fp}%
\special{pa -1772 800}\special{pa -1772 761}\special{fp}\special{pa -1772 722}\special{pa -1772 683}\special{fp}%
\special{pa -1772 644}\special{pa -1772 605}\special{fp}\special{pa -1772 566}\special{pa -1772 527}\special{fp}%
\special{pa -1772 488}\special{pa -1772 449}\special{fp}\special{pa -1772 410}\special{pa -1772 371}\special{fp}%
\special{pa -1772 332}\special{pa -1772 293}\special{fp}\special{pa -1772 254}\special{pa -1772 215}\special{fp}%
\special{pa -1772 176}\special{pa -1772 137}\special{fp}\special{pa -1772 98}\special{pa -1772 59}\special{fp}%
\special{pa -1772 20}\special{pa -1772 -20}\special{fp}\special{pa -1772 -59}\special{pa -1772 -98}\special{fp}%
\special{pa -1772 -137}\special{pa -1772 -176}\special{fp}\special{pa -1772 -215}\special{pa -1772 -254}\special{fp}%
\special{pa -1772 -293}\special{pa -1772 -332}\special{fp}\special{pa -1772 -371}\special{pa -1772 -410}\special{fp}%
\special{pa -1772 -449}\special{pa -1772 -488}\special{fp}\special{pa -1772 -527}\special{pa -1772 -566}\special{fp}%
\special{pa -1772 -605}\special{pa -1772 -644}\special{fp}\special{pa -1772 -683}\special{pa -1772 -722}\special{fp}%
\special{pa -1772 -761}\special{pa -1772 -800}\special{fp}\special{pa -1772 -839}\special{pa -1772 -879}\special{fp}%
\special{pa -1772 -918}\special{pa -1772 -957}\special{fp}\special{pa -1772 -996}\special{pa -1772 -1035}\special{fp}%
\special{pa -1772 -1074}\special{pa -1772 -1113}\special{fp}\special{pa -1772 -1152}\special{pa -1772 -1191}\special{fp}%
\special{pa -1772 -1230}\special{pa -1772 -1269}\special{fp}\special{pa -1772 -1308}\special{pa -1772 -1347}\special{fp}%
\special{pa -1772 -1386}\special{pa -1772 -1425}\special{fp}\special{pa -1772 -1464}\special{pa -1772 -1503}\special{fp}%
\special{pa -1772 -1542}\special{pa -1772 -1581}\special{fp}\special{pa -1772 -1620}\special{pa -1772 -1659}\special{fp}%
\special{pa -1772 -1698}\special{pa -1772 -1737}\special{fp}\special{pa -1772 -1777}\special{pa -1772 -1816}\special{fp}%
\special{pa -1772 -1855}\special{pa -1772 -1894}\special{fp}\special{pa -1772 -1933}\special{pa -1772 -1972}\special{fp}%
\special{pa -1772 -2011}\special{pa -1772 -2050}\special{fp}\special{pa -1772 -2089}\special{pa -1772 -2128}\special{fp}%
\special{pa -1772 -2167}\special{pa -1772 -2206}\special{fp}\special{pa -1772 -2245}\special{pa -1772 -2284}\special{fp}%
\special{pa -1772 -2323}\special{pa -1772 -2362}\special{fp}%
%
}%
{%
\color[rgb]{0,0,0}%
\special{pa -2362 1772}\special{pa -2323 1772}\special{fp}\special{pa -2284 1772}\special{pa -2245 1772}\special{fp}%
\special{pa -2206 1772}\special{pa -2167 1772}\special{fp}\special{pa -2128 1772}\special{pa -2089 1772}\special{fp}%
\special{pa -2050 1772}\special{pa -2011 1772}\special{fp}\special{pa -1972 1772}\special{pa -1933 1772}\special{fp}%
\special{pa -1894 1772}\special{pa -1855 1772}\special{fp}\special{pa -1816 1772}\special{pa -1777 1772}\special{fp}%
\special{pa -1737 1772}\special{pa -1698 1772}\special{fp}\special{pa -1659 1772}\special{pa -1620 1772}\special{fp}%
\special{pa -1581 1772}\special{pa -1542 1772}\special{fp}\special{pa -1503 1772}\special{pa -1464 1772}\special{fp}%
\special{pa -1425 1772}\special{pa -1386 1772}\special{fp}\special{pa -1347 1772}\special{pa -1308 1772}\special{fp}%
\special{pa -1269 1772}\special{pa -1230 1772}\special{fp}\special{pa -1191 1772}\special{pa -1152 1772}\special{fp}%
\special{pa -1113 1772}\special{pa -1074 1772}\special{fp}\special{pa -1035 1772}\special{pa -996 1772}\special{fp}%
\special{pa -957 1772}\special{pa -918 1772}\special{fp}\special{pa -879 1772}\special{pa -839 1772}\special{fp}%
\special{pa -800 1772}\special{pa -761 1772}\special{fp}\special{pa -722 1772}\special{pa -683 1772}\special{fp}%
\special{pa -644 1772}\special{pa -605 1772}\special{fp}\special{pa -566 1772}\special{pa -527 1772}\special{fp}%
\special{pa -488 1772}\special{pa -449 1772}\special{fp}\special{pa -410 1772}\special{pa -371 1772}\special{fp}%
\special{pa -332 1772}\special{pa -293 1772}\special{fp}\special{pa -254 1772}\special{pa -215 1772}\special{fp}%
\special{pa -176 1772}\special{pa -137 1772}\special{fp}\special{pa -98 1772}\special{pa -59 1772}\special{fp}%
\special{pa -20 1772}\special{pa 20 1772}\special{fp}\special{pa 59 1772}\special{pa 98 1772}\special{fp}%
\special{pa 137 1772}\special{pa 176 1772}\special{fp}\special{pa 215 1772}\special{pa 254 1772}\special{fp}%
\special{pa 293 1772}\special{pa 332 1772}\special{fp}\special{pa 371 1772}\special{pa 410 1772}\special{fp}%
\special{pa 449 1772}\special{pa 488 1772}\special{fp}\special{pa 527 1772}\special{pa 566 1772}\special{fp}%
\special{pa 605 1772}\special{pa 644 1772}\special{fp}\special{pa 683 1772}\special{pa 722 1772}\special{fp}%
\special{pa 761 1772}\special{pa 800 1772}\special{fp}\special{pa 839 1772}\special{pa 879 1772}\special{fp}%
\special{pa 918 1772}\special{pa 957 1772}\special{fp}\special{pa 996 1772}\special{pa 1035 1772}\special{fp}%
\special{pa 1074 1772}\special{pa 1113 1772}\special{fp}\special{pa 1152 1772}\special{pa 1191 1772}\special{fp}%
\special{pa 1230 1772}\special{pa 1269 1772}\special{fp}\special{pa 1308 1772}\special{pa 1347 1772}\special{fp}%
\special{pa 1386 1772}\special{pa 1425 1772}\special{fp}\special{pa 1464 1772}\special{pa 1503 1772}\special{fp}%
\special{pa 1542 1772}\special{pa 1581 1772}\special{fp}\special{pa 1620 1772}\special{pa 1659 1772}\special{fp}%
\special{pa 1698 1772}\special{pa 1737 1772}\special{fp}\special{pa 1777 1772}\special{pa 1816 1772}\special{fp}%
\special{pa 1855 1772}\special{pa 1894 1772}\special{fp}\special{pa 1933 1772}\special{pa 1972 1772}\special{fp}%
\special{pa 2011 1772}\special{pa 2050 1772}\special{fp}\special{pa 2089 1772}\special{pa 2128 1772}\special{fp}%
\special{pa 2167 1772}\special{pa 2206 1772}\special{fp}\special{pa 2245 1772}\special{pa 2284 1772}\special{fp}%
\special{pa 2323 1772}\special{pa 2362 1772}\special{fp}%
%
}%
{%
\color[rgb]{0,0,0}%
\special{pa -1378 2362}\special{pa -1378 2323}\special{fp}\special{pa -1378 2284}\special{pa -1378 2245}\special{fp}%
\special{pa -1378 2206}\special{pa -1378 2167}\special{fp}\special{pa -1378 2128}\special{pa -1378 2089}\special{fp}%
\special{pa -1378 2050}\special{pa -1378 2011}\special{fp}\special{pa -1378 1972}\special{pa -1378 1933}\special{fp}%
\special{pa -1378 1894}\special{pa -1378 1855}\special{fp}\special{pa -1378 1816}\special{pa -1378 1777}\special{fp}%
\special{pa -1378 1737}\special{pa -1378 1698}\special{fp}\special{pa -1378 1659}\special{pa -1378 1620}\special{fp}%
\special{pa -1378 1581}\special{pa -1378 1542}\special{fp}\special{pa -1378 1503}\special{pa -1378 1464}\special{fp}%
\special{pa -1378 1425}\special{pa -1378 1386}\special{fp}\special{pa -1378 1347}\special{pa -1378 1308}\special{fp}%
\special{pa -1378 1269}\special{pa -1378 1230}\special{fp}\special{pa -1378 1191}\special{pa -1378 1152}\special{fp}%
\special{pa -1378 1113}\special{pa -1378 1074}\special{fp}\special{pa -1378 1035}\special{pa -1378 996}\special{fp}%
\special{pa -1378 957}\special{pa -1378 918}\special{fp}\special{pa -1378 879}\special{pa -1378 839}\special{fp}%
\special{pa -1378 800}\special{pa -1378 761}\special{fp}\special{pa -1378 722}\special{pa -1378 683}\special{fp}%
\special{pa -1378 644}\special{pa -1378 605}\special{fp}\special{pa -1378 566}\special{pa -1378 527}\special{fp}%
\special{pa -1378 488}\special{pa -1378 449}\special{fp}\special{pa -1378 410}\special{pa -1378 371}\special{fp}%
\special{pa -1378 332}\special{pa -1378 293}\special{fp}\special{pa -1378 254}\special{pa -1378 215}\special{fp}%
\special{pa -1378 176}\special{pa -1378 137}\special{fp}\special{pa -1378 98}\special{pa -1378 59}\special{fp}%
\special{pa -1378 20}\special{pa -1378 -20}\special{fp}\special{pa -1378 -59}\special{pa -1378 -98}\special{fp}%
\special{pa -1378 -137}\special{pa -1378 -176}\special{fp}\special{pa -1378 -215}\special{pa -1378 -254}\special{fp}%
\special{pa -1378 -293}\special{pa -1378 -332}\special{fp}\special{pa -1378 -371}\special{pa -1378 -410}\special{fp}%
\special{pa -1378 -449}\special{pa -1378 -488}\special{fp}\special{pa -1378 -527}\special{pa -1378 -566}\special{fp}%
\special{pa -1378 -605}\special{pa -1378 -644}\special{fp}\special{pa -1378 -683}\special{pa -1378 -722}\special{fp}%
\special{pa -1378 -761}\special{pa -1378 -800}\special{fp}\special{pa -1378 -839}\special{pa -1378 -879}\special{fp}%
\special{pa -1378 -918}\special{pa -1378 -957}\special{fp}\special{pa -1378 -996}\special{pa -1378 -1035}\special{fp}%
\special{pa -1378 -1074}\special{pa -1378 -1113}\special{fp}\special{pa -1378 -1152}\special{pa -1378 -1191}\special{fp}%
\special{pa -1378 -1230}\special{pa -1378 -1269}\special{fp}\special{pa -1378 -1308}\special{pa -1378 -1347}\special{fp}%
\special{pa -1378 -1386}\special{pa -1378 -1425}\special{fp}\special{pa -1378 -1464}\special{pa -1378 -1503}\special{fp}%
\special{pa -1378 -1542}\special{pa -1378 -1581}\special{fp}\special{pa -1378 -1620}\special{pa -1378 -1659}\special{fp}%
\special{pa -1378 -1698}\special{pa -1378 -1737}\special{fp}\special{pa -1378 -1777}\special{pa -1378 -1816}\special{fp}%
\special{pa -1378 -1855}\special{pa -1378 -1894}\special{fp}\special{pa -1378 -1933}\special{pa -1378 -1972}\special{fp}%
\special{pa -1378 -2011}\special{pa -1378 -2050}\special{fp}\special{pa -1378 -2089}\special{pa -1378 -2128}\special{fp}%
\special{pa -1378 -2167}\special{pa -1378 -2206}\special{fp}\special{pa -1378 -2245}\special{pa -1378 -2284}\special{fp}%
\special{pa -1378 -2323}\special{pa -1378 -2362}\special{fp}%
%
}%
{%
\color[rgb]{0,0,0}%
\special{pa -2362 1378}\special{pa -2323 1378}\special{fp}\special{pa -2284 1378}\special{pa -2245 1378}\special{fp}%
\special{pa -2206 1378}\special{pa -2167 1378}\special{fp}\special{pa -2128 1378}\special{pa -2089 1378}\special{fp}%
\special{pa -2050 1378}\special{pa -2011 1378}\special{fp}\special{pa -1972 1378}\special{pa -1933 1378}\special{fp}%
\special{pa -1894 1378}\special{pa -1855 1378}\special{fp}\special{pa -1816 1378}\special{pa -1777 1378}\special{fp}%
\special{pa -1737 1378}\special{pa -1698 1378}\special{fp}\special{pa -1659 1378}\special{pa -1620 1378}\special{fp}%
\special{pa -1581 1378}\special{pa -1542 1378}\special{fp}\special{pa -1503 1378}\special{pa -1464 1378}\special{fp}%
\special{pa -1425 1378}\special{pa -1386 1378}\special{fp}\special{pa -1347 1378}\special{pa -1308 1378}\special{fp}%
\special{pa -1269 1378}\special{pa -1230 1378}\special{fp}\special{pa -1191 1378}\special{pa -1152 1378}\special{fp}%
\special{pa -1113 1378}\special{pa -1074 1378}\special{fp}\special{pa -1035 1378}\special{pa -996 1378}\special{fp}%
\special{pa -957 1378}\special{pa -918 1378}\special{fp}\special{pa -879 1378}\special{pa -839 1378}\special{fp}%
\special{pa -800 1378}\special{pa -761 1378}\special{fp}\special{pa -722 1378}\special{pa -683 1378}\special{fp}%
\special{pa -644 1378}\special{pa -605 1378}\special{fp}\special{pa -566 1378}\special{pa -527 1378}\special{fp}%
\special{pa -488 1378}\special{pa -449 1378}\special{fp}\special{pa -410 1378}\special{pa -371 1378}\special{fp}%
\special{pa -332 1378}\special{pa -293 1378}\special{fp}\special{pa -254 1378}\special{pa -215 1378}\special{fp}%
\special{pa -176 1378}\special{pa -137 1378}\special{fp}\special{pa -98 1378}\special{pa -59 1378}\special{fp}%
\special{pa -20 1378}\special{pa 20 1378}\special{fp}\special{pa 59 1378}\special{pa 98 1378}\special{fp}%
\special{pa 137 1378}\special{pa 176 1378}\special{fp}\special{pa 215 1378}\special{pa 254 1378}\special{fp}%
\special{pa 293 1378}\special{pa 332 1378}\special{fp}\special{pa 371 1378}\special{pa 410 1378}\special{fp}%
\special{pa 449 1378}\special{pa 488 1378}\special{fp}\special{pa 527 1378}\special{pa 566 1378}\special{fp}%
\special{pa 605 1378}\special{pa 644 1378}\special{fp}\special{pa 683 1378}\special{pa 722 1378}\special{fp}%
\special{pa 761 1378}\special{pa 800 1378}\special{fp}\special{pa 839 1378}\special{pa 879 1378}\special{fp}%
\special{pa 918 1378}\special{pa 957 1378}\special{fp}\special{pa 996 1378}\special{pa 1035 1378}\special{fp}%
\special{pa 1074 1378}\special{pa 1113 1378}\special{fp}\special{pa 1152 1378}\special{pa 1191 1378}\special{fp}%
\special{pa 1230 1378}\special{pa 1269 1378}\special{fp}\special{pa 1308 1378}\special{pa 1347 1378}\special{fp}%
\special{pa 1386 1378}\special{pa 1425 1378}\special{fp}\special{pa 1464 1378}\special{pa 1503 1378}\special{fp}%
\special{pa 1542 1378}\special{pa 1581 1378}\special{fp}\special{pa 1620 1378}\special{pa 1659 1378}\special{fp}%
\special{pa 1698 1378}\special{pa 1737 1378}\special{fp}\special{pa 1777 1378}\special{pa 1816 1378}\special{fp}%
\special{pa 1855 1378}\special{pa 1894 1378}\special{fp}\special{pa 1933 1378}\special{pa 1972 1378}\special{fp}%
\special{pa 2011 1378}\special{pa 2050 1378}\special{fp}\special{pa 2089 1378}\special{pa 2128 1378}\special{fp}%
\special{pa 2167 1378}\special{pa 2206 1378}\special{fp}\special{pa 2245 1378}\special{pa 2284 1378}\special{fp}%
\special{pa 2323 1378}\special{pa 2362 1378}\special{fp}%
%
}%
{%
\color[rgb]{0,0,0}%
\special{pa -984 2362}\special{pa -984 2323}\special{fp}\special{pa -984 2284}\special{pa -984 2245}\special{fp}%
\special{pa -984 2206}\special{pa -984 2167}\special{fp}\special{pa -984 2128}\special{pa -984 2089}\special{fp}%
\special{pa -984 2050}\special{pa -984 2011}\special{fp}\special{pa -984 1972}\special{pa -984 1933}\special{fp}%
\special{pa -984 1894}\special{pa -984 1855}\special{fp}\special{pa -984 1816}\special{pa -984 1777}\special{fp}%
\special{pa -984 1737}\special{pa -984 1698}\special{fp}\special{pa -984 1659}\special{pa -984 1620}\special{fp}%
\special{pa -984 1581}\special{pa -984 1542}\special{fp}\special{pa -984 1503}\special{pa -984 1464}\special{fp}%
\special{pa -984 1425}\special{pa -984 1386}\special{fp}\special{pa -984 1347}\special{pa -984 1308}\special{fp}%
\special{pa -984 1269}\special{pa -984 1230}\special{fp}\special{pa -984 1191}\special{pa -984 1152}\special{fp}%
\special{pa -984 1113}\special{pa -984 1074}\special{fp}\special{pa -984 1035}\special{pa -984 996}\special{fp}%
\special{pa -984 957}\special{pa -984 918}\special{fp}\special{pa -984 879}\special{pa -984 839}\special{fp}%
\special{pa -984 800}\special{pa -984 761}\special{fp}\special{pa -984 722}\special{pa -984 683}\special{fp}%
\special{pa -984 644}\special{pa -984 605}\special{fp}\special{pa -984 566}\special{pa -984 527}\special{fp}%
\special{pa -984 488}\special{pa -984 449}\special{fp}\special{pa -984 410}\special{pa -984 371}\special{fp}%
\special{pa -984 332}\special{pa -984 293}\special{fp}\special{pa -984 254}\special{pa -984 215}\special{fp}%
\special{pa -984 176}\special{pa -984 137}\special{fp}\special{pa -984 98}\special{pa -984 59}\special{fp}%
\special{pa -984 20}\special{pa -984 -20}\special{fp}\special{pa -984 -59}\special{pa -984 -98}\special{fp}%
\special{pa -984 -137}\special{pa -984 -176}\special{fp}\special{pa -984 -215}\special{pa -984 -254}\special{fp}%
\special{pa -984 -293}\special{pa -984 -332}\special{fp}\special{pa -984 -371}\special{pa -984 -410}\special{fp}%
\special{pa -984 -449}\special{pa -984 -488}\special{fp}\special{pa -984 -527}\special{pa -984 -566}\special{fp}%
\special{pa -984 -605}\special{pa -984 -644}\special{fp}\special{pa -984 -683}\special{pa -984 -722}\special{fp}%
\special{pa -984 -761}\special{pa -984 -800}\special{fp}\special{pa -984 -839}\special{pa -984 -879}\special{fp}%
\special{pa -984 -918}\special{pa -984 -957}\special{fp}\special{pa -984 -996}\special{pa -984 -1035}\special{fp}%
\special{pa -984 -1074}\special{pa -984 -1113}\special{fp}\special{pa -984 -1152}\special{pa -984 -1191}\special{fp}%
\special{pa -984 -1230}\special{pa -984 -1269}\special{fp}\special{pa -984 -1308}\special{pa -984 -1347}\special{fp}%
\special{pa -984 -1386}\special{pa -984 -1425}\special{fp}\special{pa -984 -1464}\special{pa -984 -1503}\special{fp}%
\special{pa -984 -1542}\special{pa -984 -1581}\special{fp}\special{pa -984 -1620}\special{pa -984 -1659}\special{fp}%
\special{pa -984 -1698}\special{pa -984 -1737}\special{fp}\special{pa -984 -1777}\special{pa -984 -1816}\special{fp}%
\special{pa -984 -1855}\special{pa -984 -1894}\special{fp}\special{pa -984 -1933}\special{pa -984 -1972}\special{fp}%
\special{pa -984 -2011}\special{pa -984 -2050}\special{fp}\special{pa -984 -2089}\special{pa -984 -2128}\special{fp}%
\special{pa -984 -2167}\special{pa -984 -2206}\special{fp}\special{pa -984 -2245}\special{pa -984 -2284}\special{fp}%
\special{pa -984 -2323}\special{pa -984 -2362}\special{fp}%
%
}%
{%
\color[rgb]{0,0,0}%
\special{pa -2362 984}\special{pa -2323 984}\special{fp}\special{pa -2284 984}\special{pa -2245 984}\special{fp}%
\special{pa -2206 984}\special{pa -2167 984}\special{fp}\special{pa -2128 984}\special{pa -2089 984}\special{fp}%
\special{pa -2050 984}\special{pa -2011 984}\special{fp}\special{pa -1972 984}\special{pa -1933 984}\special{fp}%
\special{pa -1894 984}\special{pa -1855 984}\special{fp}\special{pa -1816 984}\special{pa -1777 984}\special{fp}%
\special{pa -1737 984}\special{pa -1698 984}\special{fp}\special{pa -1659 984}\special{pa -1620 984}\special{fp}%
\special{pa -1581 984}\special{pa -1542 984}\special{fp}\special{pa -1503 984}\special{pa -1464 984}\special{fp}%
\special{pa -1425 984}\special{pa -1386 984}\special{fp}\special{pa -1347 984}\special{pa -1308 984}\special{fp}%
\special{pa -1269 984}\special{pa -1230 984}\special{fp}\special{pa -1191 984}\special{pa -1152 984}\special{fp}%
\special{pa -1113 984}\special{pa -1074 984}\special{fp}\special{pa -1035 984}\special{pa -996 984}\special{fp}%
\special{pa -957 984}\special{pa -918 984}\special{fp}\special{pa -879 984}\special{pa -839 984}\special{fp}%
\special{pa -800 984}\special{pa -761 984}\special{fp}\special{pa -722 984}\special{pa -683 984}\special{fp}%
\special{pa -644 984}\special{pa -605 984}\special{fp}\special{pa -566 984}\special{pa -527 984}\special{fp}%
\special{pa -488 984}\special{pa -449 984}\special{fp}\special{pa -410 984}\special{pa -371 984}\special{fp}%
\special{pa -332 984}\special{pa -293 984}\special{fp}\special{pa -254 984}\special{pa -215 984}\special{fp}%
\special{pa -176 984}\special{pa -137 984}\special{fp}\special{pa -98 984}\special{pa -59 984}\special{fp}%
\special{pa -20 984}\special{pa 20 984}\special{fp}\special{pa 59 984}\special{pa 98 984}\special{fp}%
\special{pa 137 984}\special{pa 176 984}\special{fp}\special{pa 215 984}\special{pa 254 984}\special{fp}%
\special{pa 293 984}\special{pa 332 984}\special{fp}\special{pa 371 984}\special{pa 410 984}\special{fp}%
\special{pa 449 984}\special{pa 488 984}\special{fp}\special{pa 527 984}\special{pa 566 984}\special{fp}%
\special{pa 605 984}\special{pa 644 984}\special{fp}\special{pa 683 984}\special{pa 722 984}\special{fp}%
\special{pa 761 984}\special{pa 800 984}\special{fp}\special{pa 839 984}\special{pa 879 984}\special{fp}%
\special{pa 918 984}\special{pa 957 984}\special{fp}\special{pa 996 984}\special{pa 1035 984}\special{fp}%
\special{pa 1074 984}\special{pa 1113 984}\special{fp}\special{pa 1152 984}\special{pa 1191 984}\special{fp}%
\special{pa 1230 984}\special{pa 1269 984}\special{fp}\special{pa 1308 984}\special{pa 1347 984}\special{fp}%
\special{pa 1386 984}\special{pa 1425 984}\special{fp}\special{pa 1464 984}\special{pa 1503 984}\special{fp}%
\special{pa 1542 984}\special{pa 1581 984}\special{fp}\special{pa 1620 984}\special{pa 1659 984}\special{fp}%
\special{pa 1698 984}\special{pa 1737 984}\special{fp}\special{pa 1777 984}\special{pa 1816 984}\special{fp}%
\special{pa 1855 984}\special{pa 1894 984}\special{fp}\special{pa 1933 984}\special{pa 1972 984}\special{fp}%
\special{pa 2011 984}\special{pa 2050 984}\special{fp}\special{pa 2089 984}\special{pa 2128 984}\special{fp}%
\special{pa 2167 984}\special{pa 2206 984}\special{fp}\special{pa 2245 984}\special{pa 2284 984}\special{fp}%
\special{pa 2323 984}\special{pa 2362 984}\special{fp}%
%
}%
{%
\color[rgb]{0,0,0}%
\special{pa -591 2362}\special{pa -591 2323}\special{fp}\special{pa -591 2284}\special{pa -591 2245}\special{fp}%
\special{pa -591 2206}\special{pa -591 2167}\special{fp}\special{pa -591 2128}\special{pa -591 2089}\special{fp}%
\special{pa -591 2050}\special{pa -591 2011}\special{fp}\special{pa -591 1972}\special{pa -591 1933}\special{fp}%
\special{pa -591 1894}\special{pa -591 1855}\special{fp}\special{pa -591 1816}\special{pa -591 1777}\special{fp}%
\special{pa -591 1737}\special{pa -591 1698}\special{fp}\special{pa -591 1659}\special{pa -591 1620}\special{fp}%
\special{pa -591 1581}\special{pa -591 1542}\special{fp}\special{pa -591 1503}\special{pa -591 1464}\special{fp}%
\special{pa -591 1425}\special{pa -591 1386}\special{fp}\special{pa -591 1347}\special{pa -591 1308}\special{fp}%
\special{pa -591 1269}\special{pa -591 1230}\special{fp}\special{pa -591 1191}\special{pa -591 1152}\special{fp}%
\special{pa -591 1113}\special{pa -591 1074}\special{fp}\special{pa -591 1035}\special{pa -591 996}\special{fp}%
\special{pa -591 957}\special{pa -591 918}\special{fp}\special{pa -591 879}\special{pa -591 839}\special{fp}%
\special{pa -591 800}\special{pa -591 761}\special{fp}\special{pa -591 722}\special{pa -591 683}\special{fp}%
\special{pa -591 644}\special{pa -591 605}\special{fp}\special{pa -591 566}\special{pa -591 527}\special{fp}%
\special{pa -591 488}\special{pa -591 449}\special{fp}\special{pa -591 410}\special{pa -591 371}\special{fp}%
\special{pa -591 332}\special{pa -591 293}\special{fp}\special{pa -591 254}\special{pa -591 215}\special{fp}%
\special{pa -591 176}\special{pa -591 137}\special{fp}\special{pa -591 98}\special{pa -591 59}\special{fp}%
\special{pa -591 20}\special{pa -591 -20}\special{fp}\special{pa -591 -59}\special{pa -591 -98}\special{fp}%
\special{pa -591 -137}\special{pa -591 -176}\special{fp}\special{pa -591 -215}\special{pa -591 -254}\special{fp}%
\special{pa -591 -293}\special{pa -591 -332}\special{fp}\special{pa -591 -371}\special{pa -591 -410}\special{fp}%
\special{pa -591 -449}\special{pa -591 -488}\special{fp}\special{pa -591 -527}\special{pa -591 -566}\special{fp}%
\special{pa -591 -605}\special{pa -591 -644}\special{fp}\special{pa -591 -683}\special{pa -591 -722}\special{fp}%
\special{pa -591 -761}\special{pa -591 -800}\special{fp}\special{pa -591 -839}\special{pa -591 -879}\special{fp}%
\special{pa -591 -918}\special{pa -591 -957}\special{fp}\special{pa -591 -996}\special{pa -591 -1035}\special{fp}%
\special{pa -591 -1074}\special{pa -591 -1113}\special{fp}\special{pa -591 -1152}\special{pa -591 -1191}\special{fp}%
\special{pa -591 -1230}\special{pa -591 -1269}\special{fp}\special{pa -591 -1308}\special{pa -591 -1347}\special{fp}%
\special{pa -591 -1386}\special{pa -591 -1425}\special{fp}\special{pa -591 -1464}\special{pa -591 -1503}\special{fp}%
\special{pa -591 -1542}\special{pa -591 -1581}\special{fp}\special{pa -591 -1620}\special{pa -591 -1659}\special{fp}%
\special{pa -591 -1698}\special{pa -591 -1737}\special{fp}\special{pa -591 -1777}\special{pa -591 -1816}\special{fp}%
\special{pa -591 -1855}\special{pa -591 -1894}\special{fp}\special{pa -591 -1933}\special{pa -591 -1972}\special{fp}%
\special{pa -591 -2011}\special{pa -591 -2050}\special{fp}\special{pa -591 -2089}\special{pa -591 -2128}\special{fp}%
\special{pa -591 -2167}\special{pa -591 -2206}\special{fp}\special{pa -591 -2245}\special{pa -591 -2284}\special{fp}%
\special{pa -591 -2323}\special{pa -591 -2362}\special{fp}%
%
}%
{%
\color[rgb]{0,0,0}%
\special{pa -2362 591}\special{pa -2323 591}\special{fp}\special{pa -2284 591}\special{pa -2245 591}\special{fp}%
\special{pa -2206 591}\special{pa -2167 591}\special{fp}\special{pa -2128 591}\special{pa -2089 591}\special{fp}%
\special{pa -2050 591}\special{pa -2011 591}\special{fp}\special{pa -1972 591}\special{pa -1933 591}\special{fp}%
\special{pa -1894 591}\special{pa -1855 591}\special{fp}\special{pa -1816 591}\special{pa -1777 591}\special{fp}%
\special{pa -1737 591}\special{pa -1698 591}\special{fp}\special{pa -1659 591}\special{pa -1620 591}\special{fp}%
\special{pa -1581 591}\special{pa -1542 591}\special{fp}\special{pa -1503 591}\special{pa -1464 591}\special{fp}%
\special{pa -1425 591}\special{pa -1386 591}\special{fp}\special{pa -1347 591}\special{pa -1308 591}\special{fp}%
\special{pa -1269 591}\special{pa -1230 591}\special{fp}\special{pa -1191 591}\special{pa -1152 591}\special{fp}%
\special{pa -1113 591}\special{pa -1074 591}\special{fp}\special{pa -1035 591}\special{pa -996 591}\special{fp}%
\special{pa -957 591}\special{pa -918 591}\special{fp}\special{pa -879 591}\special{pa -839 591}\special{fp}%
\special{pa -800 591}\special{pa -761 591}\special{fp}\special{pa -722 591}\special{pa -683 591}\special{fp}%
\special{pa -644 591}\special{pa -605 591}\special{fp}\special{pa -566 591}\special{pa -527 591}\special{fp}%
\special{pa -488 591}\special{pa -449 591}\special{fp}\special{pa -410 591}\special{pa -371 591}\special{fp}%
\special{pa -332 591}\special{pa -293 591}\special{fp}\special{pa -254 591}\special{pa -215 591}\special{fp}%
\special{pa -176 591}\special{pa -137 591}\special{fp}\special{pa -98 591}\special{pa -59 591}\special{fp}%
\special{pa -20 591}\special{pa 20 591}\special{fp}\special{pa 59 591}\special{pa 98 591}\special{fp}%
\special{pa 137 591}\special{pa 176 591}\special{fp}\special{pa 215 591}\special{pa 254 591}\special{fp}%
\special{pa 293 591}\special{pa 332 591}\special{fp}\special{pa 371 591}\special{pa 410 591}\special{fp}%
\special{pa 449 591}\special{pa 488 591}\special{fp}\special{pa 527 591}\special{pa 566 591}\special{fp}%
\special{pa 605 591}\special{pa 644 591}\special{fp}\special{pa 683 591}\special{pa 722 591}\special{fp}%
\special{pa 761 591}\special{pa 800 591}\special{fp}\special{pa 839 591}\special{pa 879 591}\special{fp}%
\special{pa 918 591}\special{pa 957 591}\special{fp}\special{pa 996 591}\special{pa 1035 591}\special{fp}%
\special{pa 1074 591}\special{pa 1113 591}\special{fp}\special{pa 1152 591}\special{pa 1191 591}\special{fp}%
\special{pa 1230 591}\special{pa 1269 591}\special{fp}\special{pa 1308 591}\special{pa 1347 591}\special{fp}%
\special{pa 1386 591}\special{pa 1425 591}\special{fp}\special{pa 1464 591}\special{pa 1503 591}\special{fp}%
\special{pa 1542 591}\special{pa 1581 591}\special{fp}\special{pa 1620 591}\special{pa 1659 591}\special{fp}%
\special{pa 1698 591}\special{pa 1737 591}\special{fp}\special{pa 1777 591}\special{pa 1816 591}\special{fp}%
\special{pa 1855 591}\special{pa 1894 591}\special{fp}\special{pa 1933 591}\special{pa 1972 591}\special{fp}%
\special{pa 2011 591}\special{pa 2050 591}\special{fp}\special{pa 2089 591}\special{pa 2128 591}\special{fp}%
\special{pa 2167 591}\special{pa 2206 591}\special{fp}\special{pa 2245 591}\special{pa 2284 591}\special{fp}%
\special{pa 2323 591}\special{pa 2362 591}\special{fp}%
%
}%
{%
\color[rgb]{0,0,0}%
\special{pa -197 2362}\special{pa -197 2323}\special{fp}\special{pa -197 2284}\special{pa -197 2245}\special{fp}%
\special{pa -197 2206}\special{pa -197 2167}\special{fp}\special{pa -197 2128}\special{pa -197 2089}\special{fp}%
\special{pa -197 2050}\special{pa -197 2011}\special{fp}\special{pa -197 1972}\special{pa -197 1933}\special{fp}%
\special{pa -197 1894}\special{pa -197 1855}\special{fp}\special{pa -197 1816}\special{pa -197 1777}\special{fp}%
\special{pa -197 1737}\special{pa -197 1698}\special{fp}\special{pa -197 1659}\special{pa -197 1620}\special{fp}%
\special{pa -197 1581}\special{pa -197 1542}\special{fp}\special{pa -197 1503}\special{pa -197 1464}\special{fp}%
\special{pa -197 1425}\special{pa -197 1386}\special{fp}\special{pa -197 1347}\special{pa -197 1308}\special{fp}%
\special{pa -197 1269}\special{pa -197 1230}\special{fp}\special{pa -197 1191}\special{pa -197 1152}\special{fp}%
\special{pa -197 1113}\special{pa -197 1074}\special{fp}\special{pa -197 1035}\special{pa -197 996}\special{fp}%
\special{pa -197 957}\special{pa -197 918}\special{fp}\special{pa -197 879}\special{pa -197 839}\special{fp}%
\special{pa -197 800}\special{pa -197 761}\special{fp}\special{pa -197 722}\special{pa -197 683}\special{fp}%
\special{pa -197 644}\special{pa -197 605}\special{fp}\special{pa -197 566}\special{pa -197 527}\special{fp}%
\special{pa -197 488}\special{pa -197 449}\special{fp}\special{pa -197 410}\special{pa -197 371}\special{fp}%
\special{pa -197 332}\special{pa -197 293}\special{fp}\special{pa -197 254}\special{pa -197 215}\special{fp}%
\special{pa -197 176}\special{pa -197 137}\special{fp}\special{pa -197 98}\special{pa -197 59}\special{fp}%
\special{pa -197 20}\special{pa -197 -20}\special{fp}\special{pa -197 -59}\special{pa -197 -98}\special{fp}%
\special{pa -197 -137}\special{pa -197 -176}\special{fp}\special{pa -197 -215}\special{pa -197 -254}\special{fp}%
\special{pa -197 -293}\special{pa -197 -332}\special{fp}\special{pa -197 -371}\special{pa -197 -410}\special{fp}%
\special{pa -197 -449}\special{pa -197 -488}\special{fp}\special{pa -197 -527}\special{pa -197 -566}\special{fp}%
\special{pa -197 -605}\special{pa -197 -644}\special{fp}\special{pa -197 -683}\special{pa -197 -722}\special{fp}%
\special{pa -197 -761}\special{pa -197 -800}\special{fp}\special{pa -197 -839}\special{pa -197 -879}\special{fp}%
\special{pa -197 -918}\special{pa -197 -957}\special{fp}\special{pa -197 -996}\special{pa -197 -1035}\special{fp}%
\special{pa -197 -1074}\special{pa -197 -1113}\special{fp}\special{pa -197 -1152}\special{pa -197 -1191}\special{fp}%
\special{pa -197 -1230}\special{pa -197 -1269}\special{fp}\special{pa -197 -1308}\special{pa -197 -1347}\special{fp}%
\special{pa -197 -1386}\special{pa -197 -1425}\special{fp}\special{pa -197 -1464}\special{pa -197 -1503}\special{fp}%
\special{pa -197 -1542}\special{pa -197 -1581}\special{fp}\special{pa -197 -1620}\special{pa -197 -1659}\special{fp}%
\special{pa -197 -1698}\special{pa -197 -1737}\special{fp}\special{pa -197 -1777}\special{pa -197 -1816}\special{fp}%
\special{pa -197 -1855}\special{pa -197 -1894}\special{fp}\special{pa -197 -1933}\special{pa -197 -1972}\special{fp}%
\special{pa -197 -2011}\special{pa -197 -2050}\special{fp}\special{pa -197 -2089}\special{pa -197 -2128}\special{fp}%
\special{pa -197 -2167}\special{pa -197 -2206}\special{fp}\special{pa -197 -2245}\special{pa -197 -2284}\special{fp}%
\special{pa -197 -2323}\special{pa -197 -2362}\special{fp}%
%
}%
{%
\color[rgb]{0,0,0}%
\special{pa -2362 197}\special{pa -2323 197}\special{fp}\special{pa -2284 197}\special{pa -2245 197}\special{fp}%
\special{pa -2206 197}\special{pa -2167 197}\special{fp}\special{pa -2128 197}\special{pa -2089 197}\special{fp}%
\special{pa -2050 197}\special{pa -2011 197}\special{fp}\special{pa -1972 197}\special{pa -1933 197}\special{fp}%
\special{pa -1894 197}\special{pa -1855 197}\special{fp}\special{pa -1816 197}\special{pa -1777 197}\special{fp}%
\special{pa -1737 197}\special{pa -1698 197}\special{fp}\special{pa -1659 197}\special{pa -1620 197}\special{fp}%
\special{pa -1581 197}\special{pa -1542 197}\special{fp}\special{pa -1503 197}\special{pa -1464 197}\special{fp}%
\special{pa -1425 197}\special{pa -1386 197}\special{fp}\special{pa -1347 197}\special{pa -1308 197}\special{fp}%
\special{pa -1269 197}\special{pa -1230 197}\special{fp}\special{pa -1191 197}\special{pa -1152 197}\special{fp}%
\special{pa -1113 197}\special{pa -1074 197}\special{fp}\special{pa -1035 197}\special{pa -996 197}\special{fp}%
\special{pa -957 197}\special{pa -918 197}\special{fp}\special{pa -879 197}\special{pa -839 197}\special{fp}%
\special{pa -800 197}\special{pa -761 197}\special{fp}\special{pa -722 197}\special{pa -683 197}\special{fp}%
\special{pa -644 197}\special{pa -605 197}\special{fp}\special{pa -566 197}\special{pa -527 197}\special{fp}%
\special{pa -488 197}\special{pa -449 197}\special{fp}\special{pa -410 197}\special{pa -371 197}\special{fp}%
\special{pa -332 197}\special{pa -293 197}\special{fp}\special{pa -254 197}\special{pa -215 197}\special{fp}%
\special{pa -176 197}\special{pa -137 197}\special{fp}\special{pa -98 197}\special{pa -59 197}\special{fp}%
\special{pa -20 197}\special{pa 20 197}\special{fp}\special{pa 59 197}\special{pa 98 197}\special{fp}%
\special{pa 137 197}\special{pa 176 197}\special{fp}\special{pa 215 197}\special{pa 254 197}\special{fp}%
\special{pa 293 197}\special{pa 332 197}\special{fp}\special{pa 371 197}\special{pa 410 197}\special{fp}%
\special{pa 449 197}\special{pa 488 197}\special{fp}\special{pa 527 197}\special{pa 566 197}\special{fp}%
\special{pa 605 197}\special{pa 644 197}\special{fp}\special{pa 683 197}\special{pa 722 197}\special{fp}%
\special{pa 761 197}\special{pa 800 197}\special{fp}\special{pa 839 197}\special{pa 879 197}\special{fp}%
\special{pa 918 197}\special{pa 957 197}\special{fp}\special{pa 996 197}\special{pa 1035 197}\special{fp}%
\special{pa 1074 197}\special{pa 1113 197}\special{fp}\special{pa 1152 197}\special{pa 1191 197}\special{fp}%
\special{pa 1230 197}\special{pa 1269 197}\special{fp}\special{pa 1308 197}\special{pa 1347 197}\special{fp}%
\special{pa 1386 197}\special{pa 1425 197}\special{fp}\special{pa 1464 197}\special{pa 1503 197}\special{fp}%
\special{pa 1542 197}\special{pa 1581 197}\special{fp}\special{pa 1620 197}\special{pa 1659 197}\special{fp}%
\special{pa 1698 197}\special{pa 1737 197}\special{fp}\special{pa 1777 197}\special{pa 1816 197}\special{fp}%
\special{pa 1855 197}\special{pa 1894 197}\special{fp}\special{pa 1933 197}\special{pa 1972 197}\special{fp}%
\special{pa 2011 197}\special{pa 2050 197}\special{fp}\special{pa 2089 197}\special{pa 2128 197}\special{fp}%
\special{pa 2167 197}\special{pa 2206 197}\special{fp}\special{pa 2245 197}\special{pa 2284 197}\special{fp}%
\special{pa 2323 197}\special{pa 2362 197}\special{fp}%
%
}%
{%
\color[rgb]{0,0,0}%
\special{pa 197 2362}\special{pa 197 2323}\special{fp}\special{pa 197 2284}\special{pa 197 2245}\special{fp}%
\special{pa 197 2206}\special{pa 197 2167}\special{fp}\special{pa 197 2128}\special{pa 197 2089}\special{fp}%
\special{pa 197 2050}\special{pa 197 2011}\special{fp}\special{pa 197 1972}\special{pa 197 1933}\special{fp}%
\special{pa 197 1894}\special{pa 197 1855}\special{fp}\special{pa 197 1816}\special{pa 197 1777}\special{fp}%
\special{pa 197 1737}\special{pa 197 1698}\special{fp}\special{pa 197 1659}\special{pa 197 1620}\special{fp}%
\special{pa 197 1581}\special{pa 197 1542}\special{fp}\special{pa 197 1503}\special{pa 197 1464}\special{fp}%
\special{pa 197 1425}\special{pa 197 1386}\special{fp}\special{pa 197 1347}\special{pa 197 1308}\special{fp}%
\special{pa 197 1269}\special{pa 197 1230}\special{fp}\special{pa 197 1191}\special{pa 197 1152}\special{fp}%
\special{pa 197 1113}\special{pa 197 1074}\special{fp}\special{pa 197 1035}\special{pa 197 996}\special{fp}%
\special{pa 197 957}\special{pa 197 918}\special{fp}\special{pa 197 879}\special{pa 197 839}\special{fp}%
\special{pa 197 800}\special{pa 197 761}\special{fp}\special{pa 197 722}\special{pa 197 683}\special{fp}%
\special{pa 197 644}\special{pa 197 605}\special{fp}\special{pa 197 566}\special{pa 197 527}\special{fp}%
\special{pa 197 488}\special{pa 197 449}\special{fp}\special{pa 197 410}\special{pa 197 371}\special{fp}%
\special{pa 197 332}\special{pa 197 293}\special{fp}\special{pa 197 254}\special{pa 197 215}\special{fp}%
\special{pa 197 176}\special{pa 197 137}\special{fp}\special{pa 197 98}\special{pa 197 59}\special{fp}%
\special{pa 197 20}\special{pa 197 -20}\special{fp}\special{pa 197 -59}\special{pa 197 -98}\special{fp}%
\special{pa 197 -137}\special{pa 197 -176}\special{fp}\special{pa 197 -215}\special{pa 197 -254}\special{fp}%
\special{pa 197 -293}\special{pa 197 -332}\special{fp}\special{pa 197 -371}\special{pa 197 -410}\special{fp}%
\special{pa 197 -449}\special{pa 197 -488}\special{fp}\special{pa 197 -527}\special{pa 197 -566}\special{fp}%
\special{pa 197 -605}\special{pa 197 -644}\special{fp}\special{pa 197 -683}\special{pa 197 -722}\special{fp}%
\special{pa 197 -761}\special{pa 197 -800}\special{fp}\special{pa 197 -839}\special{pa 197 -879}\special{fp}%
\special{pa 197 -918}\special{pa 197 -957}\special{fp}\special{pa 197 -996}\special{pa 197 -1035}\special{fp}%
\special{pa 197 -1074}\special{pa 197 -1113}\special{fp}\special{pa 197 -1152}\special{pa 197 -1191}\special{fp}%
\special{pa 197 -1230}\special{pa 197 -1269}\special{fp}\special{pa 197 -1308}\special{pa 197 -1347}\special{fp}%
\special{pa 197 -1386}\special{pa 197 -1425}\special{fp}\special{pa 197 -1464}\special{pa 197 -1503}\special{fp}%
\special{pa 197 -1542}\special{pa 197 -1581}\special{fp}\special{pa 197 -1620}\special{pa 197 -1659}\special{fp}%
\special{pa 197 -1698}\special{pa 197 -1737}\special{fp}\special{pa 197 -1777}\special{pa 197 -1816}\special{fp}%
\special{pa 197 -1855}\special{pa 197 -1894}\special{fp}\special{pa 197 -1933}\special{pa 197 -1972}\special{fp}%
\special{pa 197 -2011}\special{pa 197 -2050}\special{fp}\special{pa 197 -2089}\special{pa 197 -2128}\special{fp}%
\special{pa 197 -2167}\special{pa 197 -2206}\special{fp}\special{pa 197 -2245}\special{pa 197 -2284}\special{fp}%
\special{pa 197 -2323}\special{pa 197 -2362}\special{fp}%
%
}%
{%
\color[rgb]{0,0,0}%
\special{pa -2362 -197}\special{pa -2323 -197}\special{fp}\special{pa -2284 -197}\special{pa -2245 -197}\special{fp}%
\special{pa -2206 -197}\special{pa -2167 -197}\special{fp}\special{pa -2128 -197}\special{pa -2089 -197}\special{fp}%
\special{pa -2050 -197}\special{pa -2011 -197}\special{fp}\special{pa -1972 -197}\special{pa -1933 -197}\special{fp}%
\special{pa -1894 -197}\special{pa -1855 -197}\special{fp}\special{pa -1816 -197}\special{pa -1777 -197}\special{fp}%
\special{pa -1737 -197}\special{pa -1698 -197}\special{fp}\special{pa -1659 -197}\special{pa -1620 -197}\special{fp}%
\special{pa -1581 -197}\special{pa -1542 -197}\special{fp}\special{pa -1503 -197}\special{pa -1464 -197}\special{fp}%
\special{pa -1425 -197}\special{pa -1386 -197}\special{fp}\special{pa -1347 -197}\special{pa -1308 -197}\special{fp}%
\special{pa -1269 -197}\special{pa -1230 -197}\special{fp}\special{pa -1191 -197}\special{pa -1152 -197}\special{fp}%
\special{pa -1113 -197}\special{pa -1074 -197}\special{fp}\special{pa -1035 -197}\special{pa -996 -197}\special{fp}%
\special{pa -957 -197}\special{pa -918 -197}\special{fp}\special{pa -879 -197}\special{pa -839 -197}\special{fp}%
\special{pa -800 -197}\special{pa -761 -197}\special{fp}\special{pa -722 -197}\special{pa -683 -197}\special{fp}%
\special{pa -644 -197}\special{pa -605 -197}\special{fp}\special{pa -566 -197}\special{pa -527 -197}\special{fp}%
\special{pa -488 -197}\special{pa -449 -197}\special{fp}\special{pa -410 -197}\special{pa -371 -197}\special{fp}%
\special{pa -332 -197}\special{pa -293 -197}\special{fp}\special{pa -254 -197}\special{pa -215 -197}\special{fp}%
\special{pa -176 -197}\special{pa -137 -197}\special{fp}\special{pa -98 -197}\special{pa -59 -197}\special{fp}%
\special{pa -20 -197}\special{pa 20 -197}\special{fp}\special{pa 59 -197}\special{pa 98 -197}\special{fp}%
\special{pa 137 -197}\special{pa 176 -197}\special{fp}\special{pa 215 -197}\special{pa 254 -197}\special{fp}%
\special{pa 293 -197}\special{pa 332 -197}\special{fp}\special{pa 371 -197}\special{pa 410 -197}\special{fp}%
\special{pa 449 -197}\special{pa 488 -197}\special{fp}\special{pa 527 -197}\special{pa 566 -197}\special{fp}%
\special{pa 605 -197}\special{pa 644 -197}\special{fp}\special{pa 683 -197}\special{pa 722 -197}\special{fp}%
\special{pa 761 -197}\special{pa 800 -197}\special{fp}\special{pa 839 -197}\special{pa 879 -197}\special{fp}%
\special{pa 918 -197}\special{pa 957 -197}\special{fp}\special{pa 996 -197}\special{pa 1035 -197}\special{fp}%
\special{pa 1074 -197}\special{pa 1113 -197}\special{fp}\special{pa 1152 -197}\special{pa 1191 -197}\special{fp}%
\special{pa 1230 -197}\special{pa 1269 -197}\special{fp}\special{pa 1308 -197}\special{pa 1347 -197}\special{fp}%
\special{pa 1386 -197}\special{pa 1425 -197}\special{fp}\special{pa 1464 -197}\special{pa 1503 -197}\special{fp}%
\special{pa 1542 -197}\special{pa 1581 -197}\special{fp}\special{pa 1620 -197}\special{pa 1659 -197}\special{fp}%
\special{pa 1698 -197}\special{pa 1737 -197}\special{fp}\special{pa 1777 -197}\special{pa 1816 -197}\special{fp}%
\special{pa 1855 -197}\special{pa 1894 -197}\special{fp}\special{pa 1933 -197}\special{pa 1972 -197}\special{fp}%
\special{pa 2011 -197}\special{pa 2050 -197}\special{fp}\special{pa 2089 -197}\special{pa 2128 -197}\special{fp}%
\special{pa 2167 -197}\special{pa 2206 -197}\special{fp}\special{pa 2245 -197}\special{pa 2284 -197}\special{fp}%
\special{pa 2323 -197}\special{pa 2362 -197}\special{fp}%
%
}%
{%
\color[rgb]{0,0,0}%
\special{pa 591 2362}\special{pa 591 2323}\special{fp}\special{pa 591 2284}\special{pa 591 2245}\special{fp}%
\special{pa 591 2206}\special{pa 591 2167}\special{fp}\special{pa 591 2128}\special{pa 591 2089}\special{fp}%
\special{pa 591 2050}\special{pa 591 2011}\special{fp}\special{pa 591 1972}\special{pa 591 1933}\special{fp}%
\special{pa 591 1894}\special{pa 591 1855}\special{fp}\special{pa 591 1816}\special{pa 591 1777}\special{fp}%
\special{pa 591 1737}\special{pa 591 1698}\special{fp}\special{pa 591 1659}\special{pa 591 1620}\special{fp}%
\special{pa 591 1581}\special{pa 591 1542}\special{fp}\special{pa 591 1503}\special{pa 591 1464}\special{fp}%
\special{pa 591 1425}\special{pa 591 1386}\special{fp}\special{pa 591 1347}\special{pa 591 1308}\special{fp}%
\special{pa 591 1269}\special{pa 591 1230}\special{fp}\special{pa 591 1191}\special{pa 591 1152}\special{fp}%
\special{pa 591 1113}\special{pa 591 1074}\special{fp}\special{pa 591 1035}\special{pa 591 996}\special{fp}%
\special{pa 591 957}\special{pa 591 918}\special{fp}\special{pa 591 879}\special{pa 591 839}\special{fp}%
\special{pa 591 800}\special{pa 591 761}\special{fp}\special{pa 591 722}\special{pa 591 683}\special{fp}%
\special{pa 591 644}\special{pa 591 605}\special{fp}\special{pa 591 566}\special{pa 591 527}\special{fp}%
\special{pa 591 488}\special{pa 591 449}\special{fp}\special{pa 591 410}\special{pa 591 371}\special{fp}%
\special{pa 591 332}\special{pa 591 293}\special{fp}\special{pa 591 254}\special{pa 591 215}\special{fp}%
\special{pa 591 176}\special{pa 591 137}\special{fp}\special{pa 591 98}\special{pa 591 59}\special{fp}%
\special{pa 591 20}\special{pa 591 -20}\special{fp}\special{pa 591 -59}\special{pa 591 -98}\special{fp}%
\special{pa 591 -137}\special{pa 591 -176}\special{fp}\special{pa 591 -215}\special{pa 591 -254}\special{fp}%
\special{pa 591 -293}\special{pa 591 -332}\special{fp}\special{pa 591 -371}\special{pa 591 -410}\special{fp}%
\special{pa 591 -449}\special{pa 591 -488}\special{fp}\special{pa 591 -527}\special{pa 591 -566}\special{fp}%
\special{pa 591 -605}\special{pa 591 -644}\special{fp}\special{pa 591 -683}\special{pa 591 -722}\special{fp}%
\special{pa 591 -761}\special{pa 591 -800}\special{fp}\special{pa 591 -839}\special{pa 591 -879}\special{fp}%
\special{pa 591 -918}\special{pa 591 -957}\special{fp}\special{pa 591 -996}\special{pa 591 -1035}\special{fp}%
\special{pa 591 -1074}\special{pa 591 -1113}\special{fp}\special{pa 591 -1152}\special{pa 591 -1191}\special{fp}%
\special{pa 591 -1230}\special{pa 591 -1269}\special{fp}\special{pa 591 -1308}\special{pa 591 -1347}\special{fp}%
\special{pa 591 -1386}\special{pa 591 -1425}\special{fp}\special{pa 591 -1464}\special{pa 591 -1503}\special{fp}%
\special{pa 591 -1542}\special{pa 591 -1581}\special{fp}\special{pa 591 -1620}\special{pa 591 -1659}\special{fp}%
\special{pa 591 -1698}\special{pa 591 -1737}\special{fp}\special{pa 591 -1777}\special{pa 591 -1816}\special{fp}%
\special{pa 591 -1855}\special{pa 591 -1894}\special{fp}\special{pa 591 -1933}\special{pa 591 -1972}\special{fp}%
\special{pa 591 -2011}\special{pa 591 -2050}\special{fp}\special{pa 591 -2089}\special{pa 591 -2128}\special{fp}%
\special{pa 591 -2167}\special{pa 591 -2206}\special{fp}\special{pa 591 -2245}\special{pa 591 -2284}\special{fp}%
\special{pa 591 -2323}\special{pa 591 -2362}\special{fp}%
%
}%
{%
\color[rgb]{0,0,0}%
\special{pa -2362 -591}\special{pa -2323 -591}\special{fp}\special{pa -2284 -591}\special{pa -2245 -591}\special{fp}%
\special{pa -2206 -591}\special{pa -2167 -591}\special{fp}\special{pa -2128 -591}\special{pa -2089 -591}\special{fp}%
\special{pa -2050 -591}\special{pa -2011 -591}\special{fp}\special{pa -1972 -591}\special{pa -1933 -591}\special{fp}%
\special{pa -1894 -591}\special{pa -1855 -591}\special{fp}\special{pa -1816 -591}\special{pa -1777 -591}\special{fp}%
\special{pa -1737 -591}\special{pa -1698 -591}\special{fp}\special{pa -1659 -591}\special{pa -1620 -591}\special{fp}%
\special{pa -1581 -591}\special{pa -1542 -591}\special{fp}\special{pa -1503 -591}\special{pa -1464 -591}\special{fp}%
\special{pa -1425 -591}\special{pa -1386 -591}\special{fp}\special{pa -1347 -591}\special{pa -1308 -591}\special{fp}%
\special{pa -1269 -591}\special{pa -1230 -591}\special{fp}\special{pa -1191 -591}\special{pa -1152 -591}\special{fp}%
\special{pa -1113 -591}\special{pa -1074 -591}\special{fp}\special{pa -1035 -591}\special{pa -996 -591}\special{fp}%
\special{pa -957 -591}\special{pa -918 -591}\special{fp}\special{pa -879 -591}\special{pa -839 -591}\special{fp}%
\special{pa -800 -591}\special{pa -761 -591}\special{fp}\special{pa -722 -591}\special{pa -683 -591}\special{fp}%
\special{pa -644 -591}\special{pa -605 -591}\special{fp}\special{pa -566 -591}\special{pa -527 -591}\special{fp}%
\special{pa -488 -591}\special{pa -449 -591}\special{fp}\special{pa -410 -591}\special{pa -371 -591}\special{fp}%
\special{pa -332 -591}\special{pa -293 -591}\special{fp}\special{pa -254 -591}\special{pa -215 -591}\special{fp}%
\special{pa -176 -591}\special{pa -137 -591}\special{fp}\special{pa -98 -591}\special{pa -59 -591}\special{fp}%
\special{pa -20 -591}\special{pa 20 -591}\special{fp}\special{pa 59 -591}\special{pa 98 -591}\special{fp}%
\special{pa 137 -591}\special{pa 176 -591}\special{fp}\special{pa 215 -591}\special{pa 254 -591}\special{fp}%
\special{pa 293 -591}\special{pa 332 -591}\special{fp}\special{pa 371 -591}\special{pa 410 -591}\special{fp}%
\special{pa 449 -591}\special{pa 488 -591}\special{fp}\special{pa 527 -591}\special{pa 566 -591}\special{fp}%
\special{pa 605 -591}\special{pa 644 -591}\special{fp}\special{pa 683 -591}\special{pa 722 -591}\special{fp}%
\special{pa 761 -591}\special{pa 800 -591}\special{fp}\special{pa 839 -591}\special{pa 879 -591}\special{fp}%
\special{pa 918 -591}\special{pa 957 -591}\special{fp}\special{pa 996 -591}\special{pa 1035 -591}\special{fp}%
\special{pa 1074 -591}\special{pa 1113 -591}\special{fp}\special{pa 1152 -591}\special{pa 1191 -591}\special{fp}%
\special{pa 1230 -591}\special{pa 1269 -591}\special{fp}\special{pa 1308 -591}\special{pa 1347 -591}\special{fp}%
\special{pa 1386 -591}\special{pa 1425 -591}\special{fp}\special{pa 1464 -591}\special{pa 1503 -591}\special{fp}%
\special{pa 1542 -591}\special{pa 1581 -591}\special{fp}\special{pa 1620 -591}\special{pa 1659 -591}\special{fp}%
\special{pa 1698 -591}\special{pa 1737 -591}\special{fp}\special{pa 1777 -591}\special{pa 1816 -591}\special{fp}%
\special{pa 1855 -591}\special{pa 1894 -591}\special{fp}\special{pa 1933 -591}\special{pa 1972 -591}\special{fp}%
\special{pa 2011 -591}\special{pa 2050 -591}\special{fp}\special{pa 2089 -591}\special{pa 2128 -591}\special{fp}%
\special{pa 2167 -591}\special{pa 2206 -591}\special{fp}\special{pa 2245 -591}\special{pa 2284 -591}\special{fp}%
\special{pa 2323 -591}\special{pa 2362 -591}\special{fp}%
%
}%
{%
\color[rgb]{0,0,0}%
\special{pa 984 2362}\special{pa 984 2323}\special{fp}\special{pa 984 2284}\special{pa 984 2245}\special{fp}%
\special{pa 984 2206}\special{pa 984 2167}\special{fp}\special{pa 984 2128}\special{pa 984 2089}\special{fp}%
\special{pa 984 2050}\special{pa 984 2011}\special{fp}\special{pa 984 1972}\special{pa 984 1933}\special{fp}%
\special{pa 984 1894}\special{pa 984 1855}\special{fp}\special{pa 984 1816}\special{pa 984 1777}\special{fp}%
\special{pa 984 1737}\special{pa 984 1698}\special{fp}\special{pa 984 1659}\special{pa 984 1620}\special{fp}%
\special{pa 984 1581}\special{pa 984 1542}\special{fp}\special{pa 984 1503}\special{pa 984 1464}\special{fp}%
\special{pa 984 1425}\special{pa 984 1386}\special{fp}\special{pa 984 1347}\special{pa 984 1308}\special{fp}%
\special{pa 984 1269}\special{pa 984 1230}\special{fp}\special{pa 984 1191}\special{pa 984 1152}\special{fp}%
\special{pa 984 1113}\special{pa 984 1074}\special{fp}\special{pa 984 1035}\special{pa 984 996}\special{fp}%
\special{pa 984 957}\special{pa 984 918}\special{fp}\special{pa 984 879}\special{pa 984 839}\special{fp}%
\special{pa 984 800}\special{pa 984 761}\special{fp}\special{pa 984 722}\special{pa 984 683}\special{fp}%
\special{pa 984 644}\special{pa 984 605}\special{fp}\special{pa 984 566}\special{pa 984 527}\special{fp}%
\special{pa 984 488}\special{pa 984 449}\special{fp}\special{pa 984 410}\special{pa 984 371}\special{fp}%
\special{pa 984 332}\special{pa 984 293}\special{fp}\special{pa 984 254}\special{pa 984 215}\special{fp}%
\special{pa 984 176}\special{pa 984 137}\special{fp}\special{pa 984 98}\special{pa 984 59}\special{fp}%
\special{pa 984 20}\special{pa 984 -20}\special{fp}\special{pa 984 -59}\special{pa 984 -98}\special{fp}%
\special{pa 984 -137}\special{pa 984 -176}\special{fp}\special{pa 984 -215}\special{pa 984 -254}\special{fp}%
\special{pa 984 -293}\special{pa 984 -332}\special{fp}\special{pa 984 -371}\special{pa 984 -410}\special{fp}%
\special{pa 984 -449}\special{pa 984 -488}\special{fp}\special{pa 984 -527}\special{pa 984 -566}\special{fp}%
\special{pa 984 -605}\special{pa 984 -644}\special{fp}\special{pa 984 -683}\special{pa 984 -722}\special{fp}%
\special{pa 984 -761}\special{pa 984 -800}\special{fp}\special{pa 984 -839}\special{pa 984 -879}\special{fp}%
\special{pa 984 -918}\special{pa 984 -957}\special{fp}\special{pa 984 -996}\special{pa 984 -1035}\special{fp}%
\special{pa 984 -1074}\special{pa 984 -1113}\special{fp}\special{pa 984 -1152}\special{pa 984 -1191}\special{fp}%
\special{pa 984 -1230}\special{pa 984 -1269}\special{fp}\special{pa 984 -1308}\special{pa 984 -1347}\special{fp}%
\special{pa 984 -1386}\special{pa 984 -1425}\special{fp}\special{pa 984 -1464}\special{pa 984 -1503}\special{fp}%
\special{pa 984 -1542}\special{pa 984 -1581}\special{fp}\special{pa 984 -1620}\special{pa 984 -1659}\special{fp}%
\special{pa 984 -1698}\special{pa 984 -1737}\special{fp}\special{pa 984 -1777}\special{pa 984 -1816}\special{fp}%
\special{pa 984 -1855}\special{pa 984 -1894}\special{fp}\special{pa 984 -1933}\special{pa 984 -1972}\special{fp}%
\special{pa 984 -2011}\special{pa 984 -2050}\special{fp}\special{pa 984 -2089}\special{pa 984 -2128}\special{fp}%
\special{pa 984 -2167}\special{pa 984 -2206}\special{fp}\special{pa 984 -2245}\special{pa 984 -2284}\special{fp}%
\special{pa 984 -2323}\special{pa 984 -2362}\special{fp}%
%
}%
{%
\color[rgb]{0,0,0}%
\special{pa -2362 -984}\special{pa -2323 -984}\special{fp}\special{pa -2284 -984}\special{pa -2245 -984}\special{fp}%
\special{pa -2206 -984}\special{pa -2167 -984}\special{fp}\special{pa -2128 -984}\special{pa -2089 -984}\special{fp}%
\special{pa -2050 -984}\special{pa -2011 -984}\special{fp}\special{pa -1972 -984}\special{pa -1933 -984}\special{fp}%
\special{pa -1894 -984}\special{pa -1855 -984}\special{fp}\special{pa -1816 -984}\special{pa -1777 -984}\special{fp}%
\special{pa -1737 -984}\special{pa -1698 -984}\special{fp}\special{pa -1659 -984}\special{pa -1620 -984}\special{fp}%
\special{pa -1581 -984}\special{pa -1542 -984}\special{fp}\special{pa -1503 -984}\special{pa -1464 -984}\special{fp}%
\special{pa -1425 -984}\special{pa -1386 -984}\special{fp}\special{pa -1347 -984}\special{pa -1308 -984}\special{fp}%
\special{pa -1269 -984}\special{pa -1230 -984}\special{fp}\special{pa -1191 -984}\special{pa -1152 -984}\special{fp}%
\special{pa -1113 -984}\special{pa -1074 -984}\special{fp}\special{pa -1035 -984}\special{pa -996 -984}\special{fp}%
\special{pa -957 -984}\special{pa -918 -984}\special{fp}\special{pa -879 -984}\special{pa -839 -984}\special{fp}%
\special{pa -800 -984}\special{pa -761 -984}\special{fp}\special{pa -722 -984}\special{pa -683 -984}\special{fp}%
\special{pa -644 -984}\special{pa -605 -984}\special{fp}\special{pa -566 -984}\special{pa -527 -984}\special{fp}%
\special{pa -488 -984}\special{pa -449 -984}\special{fp}\special{pa -410 -984}\special{pa -371 -984}\special{fp}%
\special{pa -332 -984}\special{pa -293 -984}\special{fp}\special{pa -254 -984}\special{pa -215 -984}\special{fp}%
\special{pa -176 -984}\special{pa -137 -984}\special{fp}\special{pa -98 -984}\special{pa -59 -984}\special{fp}%
\special{pa -20 -984}\special{pa 20 -984}\special{fp}\special{pa 59 -984}\special{pa 98 -984}\special{fp}%
\special{pa 137 -984}\special{pa 176 -984}\special{fp}\special{pa 215 -984}\special{pa 254 -984}\special{fp}%
\special{pa 293 -984}\special{pa 332 -984}\special{fp}\special{pa 371 -984}\special{pa 410 -984}\special{fp}%
\special{pa 449 -984}\special{pa 488 -984}\special{fp}\special{pa 527 -984}\special{pa 566 -984}\special{fp}%
\special{pa 605 -984}\special{pa 644 -984}\special{fp}\special{pa 683 -984}\special{pa 722 -984}\special{fp}%
\special{pa 761 -984}\special{pa 800 -984}\special{fp}\special{pa 839 -984}\special{pa 879 -984}\special{fp}%
\special{pa 918 -984}\special{pa 957 -984}\special{fp}\special{pa 996 -984}\special{pa 1035 -984}\special{fp}%
\special{pa 1074 -984}\special{pa 1113 -984}\special{fp}\special{pa 1152 -984}\special{pa 1191 -984}\special{fp}%
\special{pa 1230 -984}\special{pa 1269 -984}\special{fp}\special{pa 1308 -984}\special{pa 1347 -984}\special{fp}%
\special{pa 1386 -984}\special{pa 1425 -984}\special{fp}\special{pa 1464 -984}\special{pa 1503 -984}\special{fp}%
\special{pa 1542 -984}\special{pa 1581 -984}\special{fp}\special{pa 1620 -984}\special{pa 1659 -984}\special{fp}%
\special{pa 1698 -984}\special{pa 1737 -984}\special{fp}\special{pa 1777 -984}\special{pa 1816 -984}\special{fp}%
\special{pa 1855 -984}\special{pa 1894 -984}\special{fp}\special{pa 1933 -984}\special{pa 1972 -984}\special{fp}%
\special{pa 2011 -984}\special{pa 2050 -984}\special{fp}\special{pa 2089 -984}\special{pa 2128 -984}\special{fp}%
\special{pa 2167 -984}\special{pa 2206 -984}\special{fp}\special{pa 2245 -984}\special{pa 2284 -984}\special{fp}%
\special{pa 2323 -984}\special{pa 2362 -984}\special{fp}%
%
}%
{%
\color[rgb]{0,0,0}%
\special{pa 1378 2362}\special{pa 1378 2323}\special{fp}\special{pa 1378 2284}\special{pa 1378 2245}\special{fp}%
\special{pa 1378 2206}\special{pa 1378 2167}\special{fp}\special{pa 1378 2128}\special{pa 1378 2089}\special{fp}%
\special{pa 1378 2050}\special{pa 1378 2011}\special{fp}\special{pa 1378 1972}\special{pa 1378 1933}\special{fp}%
\special{pa 1378 1894}\special{pa 1378 1855}\special{fp}\special{pa 1378 1816}\special{pa 1378 1777}\special{fp}%
\special{pa 1378 1737}\special{pa 1378 1698}\special{fp}\special{pa 1378 1659}\special{pa 1378 1620}\special{fp}%
\special{pa 1378 1581}\special{pa 1378 1542}\special{fp}\special{pa 1378 1503}\special{pa 1378 1464}\special{fp}%
\special{pa 1378 1425}\special{pa 1378 1386}\special{fp}\special{pa 1378 1347}\special{pa 1378 1308}\special{fp}%
\special{pa 1378 1269}\special{pa 1378 1230}\special{fp}\special{pa 1378 1191}\special{pa 1378 1152}\special{fp}%
\special{pa 1378 1113}\special{pa 1378 1074}\special{fp}\special{pa 1378 1035}\special{pa 1378 996}\special{fp}%
\special{pa 1378 957}\special{pa 1378 918}\special{fp}\special{pa 1378 879}\special{pa 1378 839}\special{fp}%
\special{pa 1378 800}\special{pa 1378 761}\special{fp}\special{pa 1378 722}\special{pa 1378 683}\special{fp}%
\special{pa 1378 644}\special{pa 1378 605}\special{fp}\special{pa 1378 566}\special{pa 1378 527}\special{fp}%
\special{pa 1378 488}\special{pa 1378 449}\special{fp}\special{pa 1378 410}\special{pa 1378 371}\special{fp}%
\special{pa 1378 332}\special{pa 1378 293}\special{fp}\special{pa 1378 254}\special{pa 1378 215}\special{fp}%
\special{pa 1378 176}\special{pa 1378 137}\special{fp}\special{pa 1378 98}\special{pa 1378 59}\special{fp}%
\special{pa 1378 20}\special{pa 1378 -20}\special{fp}\special{pa 1378 -59}\special{pa 1378 -98}\special{fp}%
\special{pa 1378 -137}\special{pa 1378 -176}\special{fp}\special{pa 1378 -215}\special{pa 1378 -254}\special{fp}%
\special{pa 1378 -293}\special{pa 1378 -332}\special{fp}\special{pa 1378 -371}\special{pa 1378 -410}\special{fp}%
\special{pa 1378 -449}\special{pa 1378 -488}\special{fp}\special{pa 1378 -527}\special{pa 1378 -566}\special{fp}%
\special{pa 1378 -605}\special{pa 1378 -644}\special{fp}\special{pa 1378 -683}\special{pa 1378 -722}\special{fp}%
\special{pa 1378 -761}\special{pa 1378 -800}\special{fp}\special{pa 1378 -839}\special{pa 1378 -879}\special{fp}%
\special{pa 1378 -918}\special{pa 1378 -957}\special{fp}\special{pa 1378 -996}\special{pa 1378 -1035}\special{fp}%
\special{pa 1378 -1074}\special{pa 1378 -1113}\special{fp}\special{pa 1378 -1152}\special{pa 1378 -1191}\special{fp}%
\special{pa 1378 -1230}\special{pa 1378 -1269}\special{fp}\special{pa 1378 -1308}\special{pa 1378 -1347}\special{fp}%
\special{pa 1378 -1386}\special{pa 1378 -1425}\special{fp}\special{pa 1378 -1464}\special{pa 1378 -1503}\special{fp}%
\special{pa 1378 -1542}\special{pa 1378 -1581}\special{fp}\special{pa 1378 -1620}\special{pa 1378 -1659}\special{fp}%
\special{pa 1378 -1698}\special{pa 1378 -1737}\special{fp}\special{pa 1378 -1777}\special{pa 1378 -1816}\special{fp}%
\special{pa 1378 -1855}\special{pa 1378 -1894}\special{fp}\special{pa 1378 -1933}\special{pa 1378 -1972}\special{fp}%
\special{pa 1378 -2011}\special{pa 1378 -2050}\special{fp}\special{pa 1378 -2089}\special{pa 1378 -2128}\special{fp}%
\special{pa 1378 -2167}\special{pa 1378 -2206}\special{fp}\special{pa 1378 -2245}\special{pa 1378 -2284}\special{fp}%
\special{pa 1378 -2323}\special{pa 1378 -2362}\special{fp}%
%
}%
{%
\color[rgb]{0,0,0}%
\special{pa -2362 -1378}\special{pa -2323 -1378}\special{fp}\special{pa -2284 -1378}\special{pa -2245 -1378}\special{fp}%
\special{pa -2206 -1378}\special{pa -2167 -1378}\special{fp}\special{pa -2128 -1378}\special{pa -2089 -1378}\special{fp}%
\special{pa -2050 -1378}\special{pa -2011 -1378}\special{fp}\special{pa -1972 -1378}\special{pa -1933 -1378}\special{fp}%
\special{pa -1894 -1378}\special{pa -1855 -1378}\special{fp}\special{pa -1816 -1378}\special{pa -1777 -1378}\special{fp}%
\special{pa -1737 -1378}\special{pa -1698 -1378}\special{fp}\special{pa -1659 -1378}\special{pa -1620 -1378}\special{fp}%
\special{pa -1581 -1378}\special{pa -1542 -1378}\special{fp}\special{pa -1503 -1378}\special{pa -1464 -1378}\special{fp}%
\special{pa -1425 -1378}\special{pa -1386 -1378}\special{fp}\special{pa -1347 -1378}\special{pa -1308 -1378}\special{fp}%
\special{pa -1269 -1378}\special{pa -1230 -1378}\special{fp}\special{pa -1191 -1378}\special{pa -1152 -1378}\special{fp}%
\special{pa -1113 -1378}\special{pa -1074 -1378}\special{fp}\special{pa -1035 -1378}\special{pa -996 -1378}\special{fp}%
\special{pa -957 -1378}\special{pa -918 -1378}\special{fp}\special{pa -879 -1378}\special{pa -839 -1378}\special{fp}%
\special{pa -800 -1378}\special{pa -761 -1378}\special{fp}\special{pa -722 -1378}\special{pa -683 -1378}\special{fp}%
\special{pa -644 -1378}\special{pa -605 -1378}\special{fp}\special{pa -566 -1378}\special{pa -527 -1378}\special{fp}%
\special{pa -488 -1378}\special{pa -449 -1378}\special{fp}\special{pa -410 -1378}\special{pa -371 -1378}\special{fp}%
\special{pa -332 -1378}\special{pa -293 -1378}\special{fp}\special{pa -254 -1378}\special{pa -215 -1378}\special{fp}%
\special{pa -176 -1378}\special{pa -137 -1378}\special{fp}\special{pa -98 -1378}\special{pa -59 -1378}\special{fp}%
\special{pa -20 -1378}\special{pa 20 -1378}\special{fp}\special{pa 59 -1378}\special{pa 98 -1378}\special{fp}%
\special{pa 137 -1378}\special{pa 176 -1378}\special{fp}\special{pa 215 -1378}\special{pa 254 -1378}\special{fp}%
\special{pa 293 -1378}\special{pa 332 -1378}\special{fp}\special{pa 371 -1378}\special{pa 410 -1378}\special{fp}%
\special{pa 449 -1378}\special{pa 488 -1378}\special{fp}\special{pa 527 -1378}\special{pa 566 -1378}\special{fp}%
\special{pa 605 -1378}\special{pa 644 -1378}\special{fp}\special{pa 683 -1378}\special{pa 722 -1378}\special{fp}%
\special{pa 761 -1378}\special{pa 800 -1378}\special{fp}\special{pa 839 -1378}\special{pa 879 -1378}\special{fp}%
\special{pa 918 -1378}\special{pa 957 -1378}\special{fp}\special{pa 996 -1378}\special{pa 1035 -1378}\special{fp}%
\special{pa 1074 -1378}\special{pa 1113 -1378}\special{fp}\special{pa 1152 -1378}\special{pa 1191 -1378}\special{fp}%
\special{pa 1230 -1378}\special{pa 1269 -1378}\special{fp}\special{pa 1308 -1378}\special{pa 1347 -1378}\special{fp}%
\special{pa 1386 -1378}\special{pa 1425 -1378}\special{fp}\special{pa 1464 -1378}\special{pa 1503 -1378}\special{fp}%
\special{pa 1542 -1378}\special{pa 1581 -1378}\special{fp}\special{pa 1620 -1378}\special{pa 1659 -1378}\special{fp}%
\special{pa 1698 -1378}\special{pa 1737 -1378}\special{fp}\special{pa 1777 -1378}\special{pa 1816 -1378}\special{fp}%
\special{pa 1855 -1378}\special{pa 1894 -1378}\special{fp}\special{pa 1933 -1378}\special{pa 1972 -1378}\special{fp}%
\special{pa 2011 -1378}\special{pa 2050 -1378}\special{fp}\special{pa 2089 -1378}\special{pa 2128 -1378}\special{fp}%
\special{pa 2167 -1378}\special{pa 2206 -1378}\special{fp}\special{pa 2245 -1378}\special{pa 2284 -1378}\special{fp}%
\special{pa 2323 -1378}\special{pa 2362 -1378}\special{fp}%
%
}%
{%
\color[rgb]{0,0,0}%
\special{pa 1772 2362}\special{pa 1772 2323}\special{fp}\special{pa 1772 2284}\special{pa 1772 2245}\special{fp}%
\special{pa 1772 2206}\special{pa 1772 2167}\special{fp}\special{pa 1772 2128}\special{pa 1772 2089}\special{fp}%
\special{pa 1772 2050}\special{pa 1772 2011}\special{fp}\special{pa 1772 1972}\special{pa 1772 1933}\special{fp}%
\special{pa 1772 1894}\special{pa 1772 1855}\special{fp}\special{pa 1772 1816}\special{pa 1772 1777}\special{fp}%
\special{pa 1772 1737}\special{pa 1772 1698}\special{fp}\special{pa 1772 1659}\special{pa 1772 1620}\special{fp}%
\special{pa 1772 1581}\special{pa 1772 1542}\special{fp}\special{pa 1772 1503}\special{pa 1772 1464}\special{fp}%
\special{pa 1772 1425}\special{pa 1772 1386}\special{fp}\special{pa 1772 1347}\special{pa 1772 1308}\special{fp}%
\special{pa 1772 1269}\special{pa 1772 1230}\special{fp}\special{pa 1772 1191}\special{pa 1772 1152}\special{fp}%
\special{pa 1772 1113}\special{pa 1772 1074}\special{fp}\special{pa 1772 1035}\special{pa 1772 996}\special{fp}%
\special{pa 1772 957}\special{pa 1772 918}\special{fp}\special{pa 1772 879}\special{pa 1772 839}\special{fp}%
\special{pa 1772 800}\special{pa 1772 761}\special{fp}\special{pa 1772 722}\special{pa 1772 683}\special{fp}%
\special{pa 1772 644}\special{pa 1772 605}\special{fp}\special{pa 1772 566}\special{pa 1772 527}\special{fp}%
\special{pa 1772 488}\special{pa 1772 449}\special{fp}\special{pa 1772 410}\special{pa 1772 371}\special{fp}%
\special{pa 1772 332}\special{pa 1772 293}\special{fp}\special{pa 1772 254}\special{pa 1772 215}\special{fp}%
\special{pa 1772 176}\special{pa 1772 137}\special{fp}\special{pa 1772 98}\special{pa 1772 59}\special{fp}%
\special{pa 1772 20}\special{pa 1772 -20}\special{fp}\special{pa 1772 -59}\special{pa 1772 -98}\special{fp}%
\special{pa 1772 -137}\special{pa 1772 -176}\special{fp}\special{pa 1772 -215}\special{pa 1772 -254}\special{fp}%
\special{pa 1772 -293}\special{pa 1772 -332}\special{fp}\special{pa 1772 -371}\special{pa 1772 -410}\special{fp}%
\special{pa 1772 -449}\special{pa 1772 -488}\special{fp}\special{pa 1772 -527}\special{pa 1772 -566}\special{fp}%
\special{pa 1772 -605}\special{pa 1772 -644}\special{fp}\special{pa 1772 -683}\special{pa 1772 -722}\special{fp}%
\special{pa 1772 -761}\special{pa 1772 -800}\special{fp}\special{pa 1772 -839}\special{pa 1772 -879}\special{fp}%
\special{pa 1772 -918}\special{pa 1772 -957}\special{fp}\special{pa 1772 -996}\special{pa 1772 -1035}\special{fp}%
\special{pa 1772 -1074}\special{pa 1772 -1113}\special{fp}\special{pa 1772 -1152}\special{pa 1772 -1191}\special{fp}%
\special{pa 1772 -1230}\special{pa 1772 -1269}\special{fp}\special{pa 1772 -1308}\special{pa 1772 -1347}\special{fp}%
\special{pa 1772 -1386}\special{pa 1772 -1425}\special{fp}\special{pa 1772 -1464}\special{pa 1772 -1503}\special{fp}%
\special{pa 1772 -1542}\special{pa 1772 -1581}\special{fp}\special{pa 1772 -1620}\special{pa 1772 -1659}\special{fp}%
\special{pa 1772 -1698}\special{pa 1772 -1737}\special{fp}\special{pa 1772 -1777}\special{pa 1772 -1816}\special{fp}%
\special{pa 1772 -1855}\special{pa 1772 -1894}\special{fp}\special{pa 1772 -1933}\special{pa 1772 -1972}\special{fp}%
\special{pa 1772 -2011}\special{pa 1772 -2050}\special{fp}\special{pa 1772 -2089}\special{pa 1772 -2128}\special{fp}%
\special{pa 1772 -2167}\special{pa 1772 -2206}\special{fp}\special{pa 1772 -2245}\special{pa 1772 -2284}\special{fp}%
\special{pa 1772 -2323}\special{pa 1772 -2362}\special{fp}%
%
}%
{%
\color[rgb]{0,0,0}%
\special{pa -2362 -1772}\special{pa -2323 -1772}\special{fp}\special{pa -2284 -1772}\special{pa -2245 -1772}\special{fp}%
\special{pa -2206 -1772}\special{pa -2167 -1772}\special{fp}\special{pa -2128 -1772}\special{pa -2089 -1772}\special{fp}%
\special{pa -2050 -1772}\special{pa -2011 -1772}\special{fp}\special{pa -1972 -1772}\special{pa -1933 -1772}\special{fp}%
\special{pa -1894 -1772}\special{pa -1855 -1772}\special{fp}\special{pa -1816 -1772}\special{pa -1777 -1772}\special{fp}%
\special{pa -1737 -1772}\special{pa -1698 -1772}\special{fp}\special{pa -1659 -1772}\special{pa -1620 -1772}\special{fp}%
\special{pa -1581 -1772}\special{pa -1542 -1772}\special{fp}\special{pa -1503 -1772}\special{pa -1464 -1772}\special{fp}%
\special{pa -1425 -1772}\special{pa -1386 -1772}\special{fp}\special{pa -1347 -1772}\special{pa -1308 -1772}\special{fp}%
\special{pa -1269 -1772}\special{pa -1230 -1772}\special{fp}\special{pa -1191 -1772}\special{pa -1152 -1772}\special{fp}%
\special{pa -1113 -1772}\special{pa -1074 -1772}\special{fp}\special{pa -1035 -1772}\special{pa -996 -1772}\special{fp}%
\special{pa -957 -1772}\special{pa -918 -1772}\special{fp}\special{pa -879 -1772}\special{pa -839 -1772}\special{fp}%
\special{pa -800 -1772}\special{pa -761 -1772}\special{fp}\special{pa -722 -1772}\special{pa -683 -1772}\special{fp}%
\special{pa -644 -1772}\special{pa -605 -1772}\special{fp}\special{pa -566 -1772}\special{pa -527 -1772}\special{fp}%
\special{pa -488 -1772}\special{pa -449 -1772}\special{fp}\special{pa -410 -1772}\special{pa -371 -1772}\special{fp}%
\special{pa -332 -1772}\special{pa -293 -1772}\special{fp}\special{pa -254 -1772}\special{pa -215 -1772}\special{fp}%
\special{pa -176 -1772}\special{pa -137 -1772}\special{fp}\special{pa -98 -1772}\special{pa -59 -1772}\special{fp}%
\special{pa -20 -1772}\special{pa 20 -1772}\special{fp}\special{pa 59 -1772}\special{pa 98 -1772}\special{fp}%
\special{pa 137 -1772}\special{pa 176 -1772}\special{fp}\special{pa 215 -1772}\special{pa 254 -1772}\special{fp}%
\special{pa 293 -1772}\special{pa 332 -1772}\special{fp}\special{pa 371 -1772}\special{pa 410 -1772}\special{fp}%
\special{pa 449 -1772}\special{pa 488 -1772}\special{fp}\special{pa 527 -1772}\special{pa 566 -1772}\special{fp}%
\special{pa 605 -1772}\special{pa 644 -1772}\special{fp}\special{pa 683 -1772}\special{pa 722 -1772}\special{fp}%
\special{pa 761 -1772}\special{pa 800 -1772}\special{fp}\special{pa 839 -1772}\special{pa 879 -1772}\special{fp}%
\special{pa 918 -1772}\special{pa 957 -1772}\special{fp}\special{pa 996 -1772}\special{pa 1035 -1772}\special{fp}%
\special{pa 1074 -1772}\special{pa 1113 -1772}\special{fp}\special{pa 1152 -1772}\special{pa 1191 -1772}\special{fp}%
\special{pa 1230 -1772}\special{pa 1269 -1772}\special{fp}\special{pa 1308 -1772}\special{pa 1347 -1772}\special{fp}%
\special{pa 1386 -1772}\special{pa 1425 -1772}\special{fp}\special{pa 1464 -1772}\special{pa 1503 -1772}\special{fp}%
\special{pa 1542 -1772}\special{pa 1581 -1772}\special{fp}\special{pa 1620 -1772}\special{pa 1659 -1772}\special{fp}%
\special{pa 1698 -1772}\special{pa 1737 -1772}\special{fp}\special{pa 1777 -1772}\special{pa 1816 -1772}\special{fp}%
\special{pa 1855 -1772}\special{pa 1894 -1772}\special{fp}\special{pa 1933 -1772}\special{pa 1972 -1772}\special{fp}%
\special{pa 2011 -1772}\special{pa 2050 -1772}\special{fp}\special{pa 2089 -1772}\special{pa 2128 -1772}\special{fp}%
\special{pa 2167 -1772}\special{pa 2206 -1772}\special{fp}\special{pa 2245 -1772}\special{pa 2284 -1772}\special{fp}%
\special{pa 2323 -1772}\special{pa 2362 -1772}\special{fp}%
%
}%
{%
\color[rgb]{0,0,0}%
\special{pa 2165 2362}\special{pa 2165 2323}\special{fp}\special{pa 2165 2284}\special{pa 2165 2245}\special{fp}%
\special{pa 2165 2206}\special{pa 2165 2167}\special{fp}\special{pa 2165 2128}\special{pa 2165 2089}\special{fp}%
\special{pa 2165 2050}\special{pa 2165 2011}\special{fp}\special{pa 2165 1972}\special{pa 2165 1933}\special{fp}%
\special{pa 2165 1894}\special{pa 2165 1855}\special{fp}\special{pa 2165 1816}\special{pa 2165 1777}\special{fp}%
\special{pa 2165 1737}\special{pa 2165 1698}\special{fp}\special{pa 2165 1659}\special{pa 2165 1620}\special{fp}%
\special{pa 2165 1581}\special{pa 2165 1542}\special{fp}\special{pa 2165 1503}\special{pa 2165 1464}\special{fp}%
\special{pa 2165 1425}\special{pa 2165 1386}\special{fp}\special{pa 2165 1347}\special{pa 2165 1308}\special{fp}%
\special{pa 2165 1269}\special{pa 2165 1230}\special{fp}\special{pa 2165 1191}\special{pa 2165 1152}\special{fp}%
\special{pa 2165 1113}\special{pa 2165 1074}\special{fp}\special{pa 2165 1035}\special{pa 2165 996}\special{fp}%
\special{pa 2165 957}\special{pa 2165 918}\special{fp}\special{pa 2165 879}\special{pa 2165 839}\special{fp}%
\special{pa 2165 800}\special{pa 2165 761}\special{fp}\special{pa 2165 722}\special{pa 2165 683}\special{fp}%
\special{pa 2165 644}\special{pa 2165 605}\special{fp}\special{pa 2165 566}\special{pa 2165 527}\special{fp}%
\special{pa 2165 488}\special{pa 2165 449}\special{fp}\special{pa 2165 410}\special{pa 2165 371}\special{fp}%
\special{pa 2165 332}\special{pa 2165 293}\special{fp}\special{pa 2165 254}\special{pa 2165 215}\special{fp}%
\special{pa 2165 176}\special{pa 2165 137}\special{fp}\special{pa 2165 98}\special{pa 2165 59}\special{fp}%
\special{pa 2165 20}\special{pa 2165 -20}\special{fp}\special{pa 2165 -59}\special{pa 2165 -98}\special{fp}%
\special{pa 2165 -137}\special{pa 2165 -176}\special{fp}\special{pa 2165 -215}\special{pa 2165 -254}\special{fp}%
\special{pa 2165 -293}\special{pa 2165 -332}\special{fp}\special{pa 2165 -371}\special{pa 2165 -410}\special{fp}%
\special{pa 2165 -449}\special{pa 2165 -488}\special{fp}\special{pa 2165 -527}\special{pa 2165 -566}\special{fp}%
\special{pa 2165 -605}\special{pa 2165 -644}\special{fp}\special{pa 2165 -683}\special{pa 2165 -722}\special{fp}%
\special{pa 2165 -761}\special{pa 2165 -800}\special{fp}\special{pa 2165 -839}\special{pa 2165 -879}\special{fp}%
\special{pa 2165 -918}\special{pa 2165 -957}\special{fp}\special{pa 2165 -996}\special{pa 2165 -1035}\special{fp}%
\special{pa 2165 -1074}\special{pa 2165 -1113}\special{fp}\special{pa 2165 -1152}\special{pa 2165 -1191}\special{fp}%
\special{pa 2165 -1230}\special{pa 2165 -1269}\special{fp}\special{pa 2165 -1308}\special{pa 2165 -1347}\special{fp}%
\special{pa 2165 -1386}\special{pa 2165 -1425}\special{fp}\special{pa 2165 -1464}\special{pa 2165 -1503}\special{fp}%
\special{pa 2165 -1542}\special{pa 2165 -1581}\special{fp}\special{pa 2165 -1620}\special{pa 2165 -1659}\special{fp}%
\special{pa 2165 -1698}\special{pa 2165 -1737}\special{fp}\special{pa 2165 -1777}\special{pa 2165 -1816}\special{fp}%
\special{pa 2165 -1855}\special{pa 2165 -1894}\special{fp}\special{pa 2165 -1933}\special{pa 2165 -1972}\special{fp}%
\special{pa 2165 -2011}\special{pa 2165 -2050}\special{fp}\special{pa 2165 -2089}\special{pa 2165 -2128}\special{fp}%
\special{pa 2165 -2167}\special{pa 2165 -2206}\special{fp}\special{pa 2165 -2245}\special{pa 2165 -2284}\special{fp}%
\special{pa 2165 -2323}\special{pa 2165 -2362}\special{fp}%
%
}%
{%
\color[rgb]{0,0,0}%
\special{pa -2362 -2165}\special{pa -2323 -2165}\special{fp}\special{pa -2284 -2165}\special{pa -2245 -2165}\special{fp}%
\special{pa -2206 -2165}\special{pa -2167 -2165}\special{fp}\special{pa -2128 -2165}\special{pa -2089 -2165}\special{fp}%
\special{pa -2050 -2165}\special{pa -2011 -2165}\special{fp}\special{pa -1972 -2165}\special{pa -1933 -2165}\special{fp}%
\special{pa -1894 -2165}\special{pa -1855 -2165}\special{fp}\special{pa -1816 -2165}\special{pa -1777 -2165}\special{fp}%
\special{pa -1737 -2165}\special{pa -1698 -2165}\special{fp}\special{pa -1659 -2165}\special{pa -1620 -2165}\special{fp}%
\special{pa -1581 -2165}\special{pa -1542 -2165}\special{fp}\special{pa -1503 -2165}\special{pa -1464 -2165}\special{fp}%
\special{pa -1425 -2165}\special{pa -1386 -2165}\special{fp}\special{pa -1347 -2165}\special{pa -1308 -2165}\special{fp}%
\special{pa -1269 -2165}\special{pa -1230 -2165}\special{fp}\special{pa -1191 -2165}\special{pa -1152 -2165}\special{fp}%
\special{pa -1113 -2165}\special{pa -1074 -2165}\special{fp}\special{pa -1035 -2165}\special{pa -996 -2165}\special{fp}%
\special{pa -957 -2165}\special{pa -918 -2165}\special{fp}\special{pa -879 -2165}\special{pa -839 -2165}\special{fp}%
\special{pa -800 -2165}\special{pa -761 -2165}\special{fp}\special{pa -722 -2165}\special{pa -683 -2165}\special{fp}%
\special{pa -644 -2165}\special{pa -605 -2165}\special{fp}\special{pa -566 -2165}\special{pa -527 -2165}\special{fp}%
\special{pa -488 -2165}\special{pa -449 -2165}\special{fp}\special{pa -410 -2165}\special{pa -371 -2165}\special{fp}%
\special{pa -332 -2165}\special{pa -293 -2165}\special{fp}\special{pa -254 -2165}\special{pa -215 -2165}\special{fp}%
\special{pa -176 -2165}\special{pa -137 -2165}\special{fp}\special{pa -98 -2165}\special{pa -59 -2165}\special{fp}%
\special{pa -20 -2165}\special{pa 20 -2165}\special{fp}\special{pa 59 -2165}\special{pa 98 -2165}\special{fp}%
\special{pa 137 -2165}\special{pa 176 -2165}\special{fp}\special{pa 215 -2165}\special{pa 254 -2165}\special{fp}%
\special{pa 293 -2165}\special{pa 332 -2165}\special{fp}\special{pa 371 -2165}\special{pa 410 -2165}\special{fp}%
\special{pa 449 -2165}\special{pa 488 -2165}\special{fp}\special{pa 527 -2165}\special{pa 566 -2165}\special{fp}%
\special{pa 605 -2165}\special{pa 644 -2165}\special{fp}\special{pa 683 -2165}\special{pa 722 -2165}\special{fp}%
\special{pa 761 -2165}\special{pa 800 -2165}\special{fp}\special{pa 839 -2165}\special{pa 879 -2165}\special{fp}%
\special{pa 918 -2165}\special{pa 957 -2165}\special{fp}\special{pa 996 -2165}\special{pa 1035 -2165}\special{fp}%
\special{pa 1074 -2165}\special{pa 1113 -2165}\special{fp}\special{pa 1152 -2165}\special{pa 1191 -2165}\special{fp}%
\special{pa 1230 -2165}\special{pa 1269 -2165}\special{fp}\special{pa 1308 -2165}\special{pa 1347 -2165}\special{fp}%
\special{pa 1386 -2165}\special{pa 1425 -2165}\special{fp}\special{pa 1464 -2165}\special{pa 1503 -2165}\special{fp}%
\special{pa 1542 -2165}\special{pa 1581 -2165}\special{fp}\special{pa 1620 -2165}\special{pa 1659 -2165}\special{fp}%
\special{pa 1698 -2165}\special{pa 1737 -2165}\special{fp}\special{pa 1777 -2165}\special{pa 1816 -2165}\special{fp}%
\special{pa 1855 -2165}\special{pa 1894 -2165}\special{fp}\special{pa 1933 -2165}\special{pa 1972 -2165}\special{fp}%
\special{pa 2011 -2165}\special{pa 2050 -2165}\special{fp}\special{pa 2089 -2165}\special{pa 2128 -2165}\special{fp}%
\special{pa 2167 -2165}\special{pa 2206 -2165}\special{fp}\special{pa 2245 -2165}\special{pa 2284 -2165}\special{fp}%
\special{pa 2323 -2165}\special{pa 2362 -2165}\special{fp}%
%
}%
\special{pn 8}%
{%
\color[rgb]{0,0,0}%
\special{pa  2362   -20}\special{pa  2362    20}%
\special{fp}%
}%
{%
\color[rgb]{0,0,0}%
\settowidth{\Width}{$-6$}\setlength{\Width}{-0.5\Width}%
\settoheight{\Height}{$-6$}\settodepth{\Depth}{$-6$}\setlength{\Height}{-\Height}%
\put(-6.0000000,-0.1000000){\hspace*{\Width}\raisebox{\Height}{$-6$}}%
%
}%
{%
\color[rgb]{0,0,0}%
\special{pa    20 -2362}\special{pa   -20 -2362}%
\special{fp}%
}%
{%
\color[rgb]{0,0,0}%
\settowidth{\Width}{$-6$}\setlength{\Width}{-1\Width}%
\settoheight{\Height}{$-6$}\settodepth{\Depth}{$-6$}\setlength{\Height}{-0.5\Height}\setlength{\Depth}{0.5\Depth}\addtolength{\Height}{\Depth}%
\put(-0.1000000,-6.0000000){\hspace*{\Width}\raisebox{\Height}{$-6$}}%
%
}%
{%
\color[rgb]{0,0,0}%
\special{pa  2362   -20}\special{pa  2362    20}%
\special{fp}%
}%
{%
\color[rgb]{0,0,0}%
\settowidth{\Width}{$-5$}\setlength{\Width}{-0.5\Width}%
\settoheight{\Height}{$-5$}\settodepth{\Depth}{$-5$}\setlength{\Height}{-\Height}%
\put(-5.0000000,-0.1000000){\hspace*{\Width}\raisebox{\Height}{$-5$}}%
%
}%
{%
\color[rgb]{0,0,0}%
\special{pa    20 -2362}\special{pa   -20 -2362}%
\special{fp}%
}%
{%
\color[rgb]{0,0,0}%
\settowidth{\Width}{$-5$}\setlength{\Width}{-1\Width}%
\settoheight{\Height}{$-5$}\settodepth{\Depth}{$-5$}\setlength{\Height}{-0.5\Height}\setlength{\Depth}{0.5\Depth}\addtolength{\Height}{\Depth}%
\put(-0.1000000,-5.0000000){\hspace*{\Width}\raisebox{\Height}{$-5$}}%
%
}%
{%
\color[rgb]{0,0,0}%
\special{pa  2362   -20}\special{pa  2362    20}%
\special{fp}%
}%
{%
\color[rgb]{0,0,0}%
\settowidth{\Width}{$-4$}\setlength{\Width}{-0.5\Width}%
\settoheight{\Height}{$-4$}\settodepth{\Depth}{$-4$}\setlength{\Height}{-\Height}%
\put(-4.0000000,-0.1000000){\hspace*{\Width}\raisebox{\Height}{$-4$}}%
%
}%
{%
\color[rgb]{0,0,0}%
\special{pa    20 -2362}\special{pa   -20 -2362}%
\special{fp}%
}%
{%
\color[rgb]{0,0,0}%
\settowidth{\Width}{$-4$}\setlength{\Width}{-1\Width}%
\settoheight{\Height}{$-4$}\settodepth{\Depth}{$-4$}\setlength{\Height}{-0.5\Height}\setlength{\Depth}{0.5\Depth}\addtolength{\Height}{\Depth}%
\put(-0.1000000,-4.0000000){\hspace*{\Width}\raisebox{\Height}{$-4$}}%
%
}%
{%
\color[rgb]{0,0,0}%
\special{pa  2362   -20}\special{pa  2362    20}%
\special{fp}%
}%
{%
\color[rgb]{0,0,0}%
\settowidth{\Width}{$-3$}\setlength{\Width}{-0.5\Width}%
\settoheight{\Height}{$-3$}\settodepth{\Depth}{$-3$}\setlength{\Height}{-\Height}%
\put(-3.0000000,-0.1000000){\hspace*{\Width}\raisebox{\Height}{$-3$}}%
%
}%
{%
\color[rgb]{0,0,0}%
\special{pa    20 -2362}\special{pa   -20 -2362}%
\special{fp}%
}%
{%
\color[rgb]{0,0,0}%
\settowidth{\Width}{$-3$}\setlength{\Width}{-1\Width}%
\settoheight{\Height}{$-3$}\settodepth{\Depth}{$-3$}\setlength{\Height}{-0.5\Height}\setlength{\Depth}{0.5\Depth}\addtolength{\Height}{\Depth}%
\put(-0.1000000,-3.0000000){\hspace*{\Width}\raisebox{\Height}{$-3$}}%
%
}%
{%
\color[rgb]{0,0,0}%
\special{pa  2362   -20}\special{pa  2362    20}%
\special{fp}%
}%
{%
\color[rgb]{0,0,0}%
\settowidth{\Width}{$-2$}\setlength{\Width}{-0.5\Width}%
\settoheight{\Height}{$-2$}\settodepth{\Depth}{$-2$}\setlength{\Height}{-\Height}%
\put(-2.0000000,-0.1000000){\hspace*{\Width}\raisebox{\Height}{$-2$}}%
%
}%
{%
\color[rgb]{0,0,0}%
\special{pa    20 -2362}\special{pa   -20 -2362}%
\special{fp}%
}%
{%
\color[rgb]{0,0,0}%
\settowidth{\Width}{$-2$}\setlength{\Width}{-1\Width}%
\settoheight{\Height}{$-2$}\settodepth{\Depth}{$-2$}\setlength{\Height}{-0.5\Height}\setlength{\Depth}{0.5\Depth}\addtolength{\Height}{\Depth}%
\put(-0.1000000,-2.0000000){\hspace*{\Width}\raisebox{\Height}{$-2$}}%
%
}%
{%
\color[rgb]{0,0,0}%
\special{pa  2362   -20}\special{pa  2362    20}%
\special{fp}%
}%
{%
\color[rgb]{0,0,0}%
\settowidth{\Width}{$-1$}\setlength{\Width}{-0.5\Width}%
\settoheight{\Height}{$-1$}\settodepth{\Depth}{$-1$}\setlength{\Height}{-\Height}%
\put(-1.0000000,-0.1000000){\hspace*{\Width}\raisebox{\Height}{$-1$}}%
%
}%
{%
\color[rgb]{0,0,0}%
\special{pa    20 -2362}\special{pa   -20 -2362}%
\special{fp}%
}%
{%
\color[rgb]{0,0,0}%
\settowidth{\Width}{$-1$}\setlength{\Width}{-1\Width}%
\settoheight{\Height}{$-1$}\settodepth{\Depth}{$-1$}\setlength{\Height}{-0.5\Height}\setlength{\Depth}{0.5\Depth}\addtolength{\Height}{\Depth}%
\put(-0.1000000,-1.0000000){\hspace*{\Width}\raisebox{\Height}{$-1$}}%
%
}%
{%
\color[rgb]{0,0,0}%
\special{pa  2362   -20}\special{pa  2362    20}%
\special{fp}%
}%
{%
\color[rgb]{0,0,0}%
\settowidth{\Width}{$1$}\setlength{\Width}{-0.5\Width}%
\settoheight{\Height}{$1$}\settodepth{\Depth}{$1$}\setlength{\Height}{-\Height}%
\put(1.0000000,-0.1000000){\hspace*{\Width}\raisebox{\Height}{$1$}}%
%
}%
{%
\color[rgb]{0,0,0}%
\special{pa    20 -2362}\special{pa   -20 -2362}%
\special{fp}%
}%
{%
\color[rgb]{0,0,0}%
\settowidth{\Width}{$1$}\setlength{\Width}{-1\Width}%
\settoheight{\Height}{$1$}\settodepth{\Depth}{$1$}\setlength{\Height}{-0.5\Height}\setlength{\Depth}{0.5\Depth}\addtolength{\Height}{\Depth}%
\put(-0.1000000,1.0000000){\hspace*{\Width}\raisebox{\Height}{$1$}}%
%
}%
{%
\color[rgb]{0,0,0}%
\special{pa  2362   -20}\special{pa  2362    20}%
\special{fp}%
}%
{%
\color[rgb]{0,0,0}%
\settowidth{\Width}{$2$}\setlength{\Width}{-0.5\Width}%
\settoheight{\Height}{$2$}\settodepth{\Depth}{$2$}\setlength{\Height}{-\Height}%
\put(2.0000000,-0.1000000){\hspace*{\Width}\raisebox{\Height}{$2$}}%
%
}%
{%
\color[rgb]{0,0,0}%
\special{pa    20 -2362}\special{pa   -20 -2362}%
\special{fp}%
}%
{%
\color[rgb]{0,0,0}%
\settowidth{\Width}{$2$}\setlength{\Width}{-1\Width}%
\settoheight{\Height}{$2$}\settodepth{\Depth}{$2$}\setlength{\Height}{-0.5\Height}\setlength{\Depth}{0.5\Depth}\addtolength{\Height}{\Depth}%
\put(-0.1000000,2.0000000){\hspace*{\Width}\raisebox{\Height}{$2$}}%
%
}%
{%
\color[rgb]{0,0,0}%
\special{pa  2362   -20}\special{pa  2362    20}%
\special{fp}%
}%
{%
\color[rgb]{0,0,0}%
\settowidth{\Width}{$3$}\setlength{\Width}{-0.5\Width}%
\settoheight{\Height}{$3$}\settodepth{\Depth}{$3$}\setlength{\Height}{-\Height}%
\put(3.0000000,-0.1000000){\hspace*{\Width}\raisebox{\Height}{$3$}}%
%
}%
{%
\color[rgb]{0,0,0}%
\special{pa    20 -2362}\special{pa   -20 -2362}%
\special{fp}%
}%
{%
\color[rgb]{0,0,0}%
\settowidth{\Width}{$3$}\setlength{\Width}{-1\Width}%
\settoheight{\Height}{$3$}\settodepth{\Depth}{$3$}\setlength{\Height}{-0.5\Height}\setlength{\Depth}{0.5\Depth}\addtolength{\Height}{\Depth}%
\put(-0.1000000,3.0000000){\hspace*{\Width}\raisebox{\Height}{$3$}}%
%
}%
{%
\color[rgb]{0,0,0}%
\special{pa  2362   -20}\special{pa  2362    20}%
\special{fp}%
}%
{%
\color[rgb]{0,0,0}%
\settowidth{\Width}{$4$}\setlength{\Width}{-0.5\Width}%
\settoheight{\Height}{$4$}\settodepth{\Depth}{$4$}\setlength{\Height}{-\Height}%
\put(4.0000000,-0.1000000){\hspace*{\Width}\raisebox{\Height}{$4$}}%
%
}%
{%
\color[rgb]{0,0,0}%
\special{pa    20 -2362}\special{pa   -20 -2362}%
\special{fp}%
}%
{%
\color[rgb]{0,0,0}%
\settowidth{\Width}{$4$}\setlength{\Width}{-1\Width}%
\settoheight{\Height}{$4$}\settodepth{\Depth}{$4$}\setlength{\Height}{-0.5\Height}\setlength{\Depth}{0.5\Depth}\addtolength{\Height}{\Depth}%
\put(-0.1000000,4.0000000){\hspace*{\Width}\raisebox{\Height}{$4$}}%
%
}%
{%
\color[rgb]{0,0,0}%
\special{pa  2362   -20}\special{pa  2362    20}%
\special{fp}%
}%
{%
\color[rgb]{0,0,0}%
\settowidth{\Width}{$5$}\setlength{\Width}{-0.5\Width}%
\settoheight{\Height}{$5$}\settodepth{\Depth}{$5$}\setlength{\Height}{-\Height}%
\put(5.0000000,-0.1000000){\hspace*{\Width}\raisebox{\Height}{$5$}}%
%
}%
{%
\color[rgb]{0,0,0}%
\special{pa    20 -2362}\special{pa   -20 -2362}%
\special{fp}%
}%
{%
\color[rgb]{0,0,0}%
\settowidth{\Width}{$5$}\setlength{\Width}{-1\Width}%
\settoheight{\Height}{$5$}\settodepth{\Depth}{$5$}\setlength{\Height}{-0.5\Height}\setlength{\Depth}{0.5\Depth}\addtolength{\Height}{\Depth}%
\put(-0.1000000,5.0000000){\hspace*{\Width}\raisebox{\Height}{$5$}}%
%
}%
{%
\color[rgb]{0,0,0}%
\special{pa  2362   -20}\special{pa  2362    20}%
\special{fp}%
}%
{%
\color[rgb]{0,0,0}%
\settowidth{\Width}{$6$}\setlength{\Width}{-0.5\Width}%
\settoheight{\Height}{$6$}\settodepth{\Depth}{$6$}\setlength{\Height}{-\Height}%
\put(6.0000000,-0.1000000){\hspace*{\Width}\raisebox{\Height}{$6$}}%
%
}%
{%
\color[rgb]{0,0,0}%
\special{pa    20 -2362}\special{pa   -20 -2362}%
\special{fp}%
}%
{%
\color[rgb]{0,0,0}%
\settowidth{\Width}{$6$}\setlength{\Width}{-1\Width}%
\settoheight{\Height}{$6$}\settodepth{\Depth}{$6$}\setlength{\Height}{-0.5\Height}\setlength{\Depth}{0.5\Depth}\addtolength{\Height}{\Depth}%
\put(-0.1000000,6.0000000){\hspace*{\Width}\raisebox{\Height}{$6$}}%
%
{%
\color[cmyk]{0,1,1,0}%
\special{pa -1157 1969}\special{pa -1159 1960}\special{pa -1164 1953}\special{pa -1171 1947}%
\special{pa -1179 1945}\special{pa -1188 1946}\special{pa -1195 1950}\special{pa -1201 1956}%
\special{pa -1204 1964}\special{pa -1204 1973}\special{pa -1201 1981}\special{pa -1195 1987}%
\special{pa -1188 1991}\special{pa -1179 1992}\special{pa -1171 1990}\special{pa -1164 1984}%
\special{pa -1159 1977}\special{pa -1157 1969}\special{pa -1157 1969}\special{sh 1}\special{ip}%
}%
}%
{%
\color[cmyk]{0,1,1,0}%
\special{pa -1157  1969}\special{pa -1159  1960}\special{pa -1164  1953}\special{pa -1171  1947}%
\special{pa -1179  1945}\special{pa -1188  1946}\special{pa -1195  1950}\special{pa -1201  1956}%
\special{pa -1204  1964}\special{pa -1204  1973}\special{pa -1201  1981}\special{pa -1195  1987}%
\special{pa -1188  1991}\special{pa -1179  1992}\special{pa -1171  1990}\special{pa -1164  1984}%
\special{pa -1159  1977}\special{pa -1157  1969}%
\special{fp}%
{%
\color[cmyk]{0,1,1,0}%
\special{pa -764 1181}\special{pa -765 1173}\special{pa -770 1165}\special{pa -777 1160}%
\special{pa -785 1158}\special{pa -794 1158}\special{pa -802 1162}\special{pa -807 1169}%
\special{pa -811 1177}\special{pa -811 1185}\special{pa -807 1194}\special{pa -802 1200}%
\special{pa -794 1204}\special{pa -785 1205}\special{pa -777 1202}\special{pa -770 1197}%
\special{pa -765 1190}\special{pa -764 1181}\special{pa -764 1181}\special{sh 1}\special{ip}%
}%
}%
{%
\color[cmyk]{0,1,1,0}%
\special{pa  -764  1181}\special{pa  -765  1173}\special{pa  -770  1165}\special{pa  -777  1160}%
\special{pa  -785  1158}\special{pa  -794  1158}\special{pa  -802  1162}\special{pa  -807  1169}%
\special{pa  -811  1177}\special{pa  -811  1185}\special{pa  -807  1194}\special{pa  -802  1200}%
\special{pa  -794  1204}\special{pa  -785  1205}\special{pa  -777  1202}\special{pa  -770  1197}%
\special{pa  -765  1190}\special{pa  -764  1181}%
\special{fp}%
{%
\color[cmyk]{0,1,1,0}%
\special{pa -370 394}\special{pa -372 385}\special{pa -376 378}\special{pa -383 373}%
\special{pa -392 370}\special{pa -400 371}\special{pa -408 375}\special{pa -414 381}%
\special{pa -417 389}\special{pa -417 398}\special{pa -414 406}\special{pa -408 413}%
\special{pa -400 416}\special{pa -392 417}\special{pa -383 415}\special{pa -376 410}%
\special{pa -372 402}\special{pa -370 394}\special{pa -370 394}\special{sh 1}\special{ip}%
}%
}%
{%
\color[cmyk]{0,1,1,0}%
\special{pa  -370   394}\special{pa  -372   385}\special{pa  -376   378}\special{pa  -383   373}%
\special{pa  -392   370}\special{pa  -400   371}\special{pa  -408   375}\special{pa  -414   381}%
\special{pa  -417   389}\special{pa  -417   398}\special{pa  -414   406}\special{pa  -408   413}%
\special{pa  -400   416}\special{pa  -392   417}\special{pa  -383   415}\special{pa  -376   410}%
\special{pa  -372   402}\special{pa  -370   394}%
\special{fp}%
{%
\color[cmyk]{0,1,1,0}%
\special{pa 24 -394}\special{pa 22 -402}\special{pa 17 -410}\special{pa 11 -415}\special{pa 2 -417}%
\special{pa -6 -416}\special{pa -14 -413}\special{pa -20 -406}\special{pa -23 -398}%
\special{pa -23 -389}\special{pa -20 -381}\special{pa -14 -375}\special{pa -6 -371}%
\special{pa 2 -370}\special{pa 11 -373}\special{pa 17 -378}\special{pa 22 -385}\special{pa 24 -394}%
\special{pa 24 -394}\special{sh 1}\special{ip}%
}%
}%
{%
\color[cmyk]{0,1,1,0}%
\special{pa    24  -394}\special{pa    22  -402}\special{pa    17  -410}\special{pa    11  -415}%
\special{pa     2  -417}\special{pa    -6  -416}\special{pa   -14  -413}\special{pa   -20  -406}%
\special{pa   -23  -398}\special{pa   -23  -389}\special{pa   -20  -381}\special{pa   -14  -375}%
\special{pa    -6  -371}\special{pa     2  -370}\special{pa    11  -373}\special{pa    17  -378}%
\special{pa    22  -385}\special{pa    24  -394}%
\special{fp}%
{%
\color[cmyk]{0,1,1,0}%
\special{pa 417 -1181}\special{pa 416 -1190}\special{pa 411 -1197}\special{pa 404 -1202}%
\special{pa 396 -1205}\special{pa 387 -1204}\special{pa 379 -1200}\special{pa 374 -1194}%
\special{pa 370 -1185}\special{pa 370 -1177}\special{pa 374 -1169}\special{pa 379 -1162}%
\special{pa 387 -1158}\special{pa 396 -1158}\special{pa 404 -1160}\special{pa 411 -1165}%
\special{pa 416 -1173}\special{pa 417 -1181}\special{pa 417 -1181}\special{sh 1}\special{ip}%
}%
}%
{%
\color[cmyk]{0,1,1,0}%
\special{pa   417 -1181}\special{pa   416 -1190}\special{pa   411 -1197}\special{pa   404 -1202}%
\special{pa   396 -1205}\special{pa   387 -1204}\special{pa   379 -1200}\special{pa   374 -1194}%
\special{pa   370 -1185}\special{pa   370 -1177}\special{pa   374 -1169}\special{pa   379 -1162}%
\special{pa   387 -1158}\special{pa   396 -1158}\special{pa   404 -1160}\special{pa   411 -1165}%
\special{pa   416 -1173}\special{pa   417 -1181}%
\special{fp}%
{%
\color[cmyk]{0,1,1,0}%
\special{pa 811 -1969}\special{pa 809 -1977}\special{pa 805 -1984}\special{pa 798 -1990}%
\special{pa 790 -1992}\special{pa 781 -1991}\special{pa 773 -1987}\special{pa 767 -1981}%
\special{pa 764 -1973}\special{pa 764 -1964}\special{pa 767 -1956}\special{pa 773 -1950}%
\special{pa 781 -1946}\special{pa 790 -1945}\special{pa 798 -1947}\special{pa 805 -1953}%
\special{pa 809 -1960}\special{pa 811 -1969}\special{pa 811 -1969}\special{sh 1}\special{ip}%
}%
}%
{%
\color[cmyk]{0,1,1,0}%
\special{pa   811 -1969}\special{pa   809 -1977}\special{pa   805 -1984}\special{pa   798 -1990}%
\special{pa   790 -1992}\special{pa   781 -1991}\special{pa   773 -1987}\special{pa   767 -1981}%
\special{pa   764 -1973}\special{pa   764 -1964}\special{pa   767 -1956}\special{pa   773 -1950}%
\special{pa   781 -1946}\special{pa   790 -1945}\special{pa   798 -1947}\special{pa   805 -1953}%
\special{pa   809 -1960}\special{pa   811 -1969}%
\special{fp}%
}%
{%
\color[cmyk]{0,1,1,0}%
\special{pa -1378  2362}\special{pa   984 -2362}%
\special{fp}%
}%
\special{pa -2441    -0}\special{pa  2441    -0}%
\special{fp}%
\special{pa     0  2441}\special{pa     0 -2441}%
\special{fp}%
\settowidth{\Width}{$x$}\setlength{\Width}{0\Width}%
\settoheight{\Height}{$x$}\settodepth{\Depth}{$x$}\setlength{\Height}{-0.5\Height}\setlength{\Depth}{0.5\Depth}\addtolength{\Height}{\Depth}%
\put(6.2500000,0.0000000){\hspace*{\Width}\raisebox{\Height}{$x$}}%
%
\settowidth{\Width}{$y$}\setlength{\Width}{-0.5\Width}%
\settoheight{\Height}{$y$}\settodepth{\Depth}{$y$}\setlength{\Height}{\Depth}%
\put(0.0000000,6.2500000){\hspace*{\Width}\raisebox{\Height}{$y$}}%
%
\settowidth{\Width}{O}\setlength{\Width}{-1\Width}%
\settoheight{\Height}{O}\settodepth{\Depth}{O}\setlength{\Height}{-\Height}%
\put(-0.0500000,-0.0500000){\hspace*{\Width}\raisebox{\Height}{O}}%
%
\end{picture}}%}}
\end{layer}


\sameslide

\vspace*{18mm}

\slidepage
\down
例)$y=2x+1$

\begin{layer}{120}{0}
\putnotese{70}{-3}{\scalebox{0.6}{%%% /Users/takatoosetsuo/polytech23.git/101-0417/presen/fig/table1b.tex 
%%% Generator=presen23101.cdy 
{\unitlength=1cm%
\begin{picture}%
(9.6,1.2)(0,0)%
\linethickness{0.008in}%%
\Large\bf\boldmath%
\small%
\polyline(0,1.2)(0,0)%
%
\polyline(0.8,1.2)(0.8,0)%
%
\polyline(1.6,1.2)(1.6,0)%
%
\polyline(2.4,1.2)(2.4,0)%
%
\polyline(3.2,1.2)(3.2,0)%
%
\polyline(4,1.2)(4,0)%
%
\polyline(4.8,1.2)(4.8,0)%
%
\polyline(5.6,1.2)(5.6,0)%
%
\polyline(6.4,1.2)(6.4,0)%
%
\polyline(7.2,1.2)(7.2,0)%
%
\polyline(8,1.2)(8,0)%
%
\polyline(8.8,1.2)(8.8,0)%
%
\polyline(9.6,1.2)(9.6,0)%
%
\polyline(0,1.2)(9.6,1.2)%
%
\polyline(0,0.6)(9.6,0.6)%
%
\polyline(0,0)(9.6,0)%
%
\settowidth{\Width}{$x$}\setlength{\Width}{-0.5\Width}%
\settoheight{\Height}{$x$}\settodepth{\Depth}{$x$}\setlength{\Height}{-0.5\Height}\setlength{\Depth}{0.5\Depth}\addtolength{\Height}{\Depth}%
\put(  0.400,  0.900){\hspace*{\Width}\raisebox{\Height}{$x$}}%
%
\settowidth{\Width}{$-5$}\setlength{\Width}{-0.5\Width}%
\settoheight{\Height}{$-5$}\settodepth{\Depth}{$-5$}\setlength{\Height}{-0.5\Height}\setlength{\Depth}{0.5\Depth}\addtolength{\Height}{\Depth}%
\put(  1.200,  0.900){\hspace*{\Width}\raisebox{\Height}{$-5$}}%
%
\settowidth{\Width}{$-4$}\setlength{\Width}{-0.5\Width}%
\settoheight{\Height}{$-4$}\settodepth{\Depth}{$-4$}\setlength{\Height}{-0.5\Height}\setlength{\Depth}{0.5\Depth}\addtolength{\Height}{\Depth}%
\put(  2.000,  0.900){\hspace*{\Width}\raisebox{\Height}{$-4$}}%
%
\settowidth{\Width}{$-3$}\setlength{\Width}{-0.5\Width}%
\settoheight{\Height}{$-3$}\settodepth{\Depth}{$-3$}\setlength{\Height}{-0.5\Height}\setlength{\Depth}{0.5\Depth}\addtolength{\Height}{\Depth}%
\put(  2.800,  0.900){\hspace*{\Width}\raisebox{\Height}{$-3$}}%
%
\settowidth{\Width}{$-2$}\setlength{\Width}{-0.5\Width}%
\settoheight{\Height}{$-2$}\settodepth{\Depth}{$-2$}\setlength{\Height}{-0.5\Height}\setlength{\Depth}{0.5\Depth}\addtolength{\Height}{\Depth}%
\put(  3.600,  0.900){\hspace*{\Width}\raisebox{\Height}{$-2$}}%
%
\settowidth{\Width}{$-1$}\setlength{\Width}{-0.5\Width}%
\settoheight{\Height}{$-1$}\settodepth{\Depth}{$-1$}\setlength{\Height}{-0.5\Height}\setlength{\Depth}{0.5\Depth}\addtolength{\Height}{\Depth}%
\put(  4.400,  0.900){\hspace*{\Width}\raisebox{\Height}{$-1$}}%
%
\settowidth{\Width}{$0$}\setlength{\Width}{-0.5\Width}%
\settoheight{\Height}{$0$}\settodepth{\Depth}{$0$}\setlength{\Height}{-0.5\Height}\setlength{\Depth}{0.5\Depth}\addtolength{\Height}{\Depth}%
\put(  5.200,  0.900){\hspace*{\Width}\raisebox{\Height}{$0$}}%
%
\settowidth{\Width}{$1$}\setlength{\Width}{-0.5\Width}%
\settoheight{\Height}{$1$}\settodepth{\Depth}{$1$}\setlength{\Height}{-0.5\Height}\setlength{\Depth}{0.5\Depth}\addtolength{\Height}{\Depth}%
\put(  6.000,  0.900){\hspace*{\Width}\raisebox{\Height}{$1$}}%
%
\settowidth{\Width}{$2$}\setlength{\Width}{-0.5\Width}%
\settoheight{\Height}{$2$}\settodepth{\Depth}{$2$}\setlength{\Height}{-0.5\Height}\setlength{\Depth}{0.5\Depth}\addtolength{\Height}{\Depth}%
\put(  6.800,  0.900){\hspace*{\Width}\raisebox{\Height}{$2$}}%
%
\settowidth{\Width}{$3$}\setlength{\Width}{-0.5\Width}%
\settoheight{\Height}{$3$}\settodepth{\Depth}{$3$}\setlength{\Height}{-0.5\Height}\setlength{\Depth}{0.5\Depth}\addtolength{\Height}{\Depth}%
\put(  7.600,  0.900){\hspace*{\Width}\raisebox{\Height}{$3$}}%
%
\settowidth{\Width}{$4$}\setlength{\Width}{-0.5\Width}%
\settoheight{\Height}{$4$}\settodepth{\Depth}{$4$}\setlength{\Height}{-0.5\Height}\setlength{\Depth}{0.5\Depth}\addtolength{\Height}{\Depth}%
\put(  8.400,  0.900){\hspace*{\Width}\raisebox{\Height}{$4$}}%
%
\settowidth{\Width}{$5$}\setlength{\Width}{-0.5\Width}%
\settoheight{\Height}{$5$}\settodepth{\Depth}{$5$}\setlength{\Height}{-0.5\Height}\setlength{\Depth}{0.5\Depth}\addtolength{\Height}{\Depth}%
\put(  9.200,  0.900){\hspace*{\Width}\raisebox{\Height}{$5$}}%
%
\settowidth{\Width}{$y$}\setlength{\Width}{-0.5\Width}%
\settoheight{\Height}{$y$}\settodepth{\Depth}{$y$}\setlength{\Height}{-0.5\Height}\setlength{\Depth}{0.5\Depth}\addtolength{\Height}{\Depth}%
\put(  0.400,  0.300){\hspace*{\Width}\raisebox{\Height}{$y$}}%
%
\settowidth{\Width}{$-9$}\setlength{\Width}{-0.5\Width}%
\settoheight{\Height}{$-9$}\settodepth{\Depth}{$-9$}\setlength{\Height}{-0.5\Height}\setlength{\Depth}{0.5\Depth}\addtolength{\Height}{\Depth}%
\put(  1.200,  0.300){\hspace*{\Width}\raisebox{\Height}{$-9$}}%
%
\settowidth{\Width}{$-7$}\setlength{\Width}{-0.5\Width}%
\settoheight{\Height}{$-7$}\settodepth{\Depth}{$-7$}\setlength{\Height}{-0.5\Height}\setlength{\Depth}{0.5\Depth}\addtolength{\Height}{\Depth}%
\put(  2.000,  0.300){\hspace*{\Width}\raisebox{\Height}{$-7$}}%
%
\settowidth{\Width}{$-5$}\setlength{\Width}{-0.5\Width}%
\settoheight{\Height}{$-5$}\settodepth{\Depth}{$-5$}\setlength{\Height}{-0.5\Height}\setlength{\Depth}{0.5\Depth}\addtolength{\Height}{\Depth}%
\put(  2.800,  0.300){\hspace*{\Width}\raisebox{\Height}{$-5$}}%
%
\settowidth{\Width}{$-3$}\setlength{\Width}{-0.5\Width}%
\settoheight{\Height}{$-3$}\settodepth{\Depth}{$-3$}\setlength{\Height}{-0.5\Height}\setlength{\Depth}{0.5\Depth}\addtolength{\Height}{\Depth}%
\put(  3.600,  0.300){\hspace*{\Width}\raisebox{\Height}{$-3$}}%
%
\settowidth{\Width}{$-1$}\setlength{\Width}{-0.5\Width}%
\settoheight{\Height}{$-1$}\settodepth{\Depth}{$-1$}\setlength{\Height}{-0.5\Height}\setlength{\Depth}{0.5\Depth}\addtolength{\Height}{\Depth}%
\put(  4.400,  0.300){\hspace*{\Width}\raisebox{\Height}{$-1$}}%
%
\settowidth{\Width}{$1$}\setlength{\Width}{-0.5\Width}%
\settoheight{\Height}{$1$}\settodepth{\Depth}{$1$}\setlength{\Height}{-0.5\Height}\setlength{\Depth}{0.5\Depth}\addtolength{\Height}{\Depth}%
\put(  5.200,  0.300){\hspace*{\Width}\raisebox{\Height}{$1$}}%
%
\settowidth{\Width}{$3$}\setlength{\Width}{-0.5\Width}%
\settoheight{\Height}{$3$}\settodepth{\Depth}{$3$}\setlength{\Height}{-0.5\Height}\setlength{\Depth}{0.5\Depth}\addtolength{\Height}{\Depth}%
\put(  6.000,  0.300){\hspace*{\Width}\raisebox{\Height}{$3$}}%
%
\settowidth{\Width}{$5$}\setlength{\Width}{-0.5\Width}%
\settoheight{\Height}{$5$}\settodepth{\Depth}{$5$}\setlength{\Height}{-0.5\Height}\setlength{\Depth}{0.5\Depth}\addtolength{\Height}{\Depth}%
\put(  6.800,  0.300){\hspace*{\Width}\raisebox{\Height}{$5$}}%
%
\settowidth{\Width}{$7$}\setlength{\Width}{-0.5\Width}%
\settoheight{\Height}{$7$}\settodepth{\Depth}{$7$}\setlength{\Height}{-0.5\Height}\setlength{\Depth}{0.5\Depth}\addtolength{\Height}{\Depth}%
\put(  7.600,  0.300){\hspace*{\Width}\raisebox{\Height}{$7$}}%
%
\settowidth{\Width}{$9$}\setlength{\Width}{-0.5\Width}%
\settoheight{\Height}{$9$}\settodepth{\Depth}{$9$}\setlength{\Height}{-0.5\Height}\setlength{\Depth}{0.5\Depth}\addtolength{\Height}{\Depth}%
\put(  8.400,  0.300){\hspace*{\Width}\raisebox{\Height}{$9$}}%
%
\settowidth{\Width}{$11$}\setlength{\Width}{-0.5\Width}%
\settoheight{\Height}{$11$}\settodepth{\Depth}{$11$}\setlength{\Height}{-0.5\Height}\setlength{\Depth}{0.5\Depth}\addtolength{\Height}{\Depth}%
\put(  9.200,  0.300){\hspace*{\Width}\raisebox{\Height}{$11$}}%
%
\end{picture}}%}}
\putnotes{60}{6}{\scalebox{0.5}{%%% /polytech.git/n101/fig/graphpaper3.tex 
%%% Generator=presen0601.cdy 
{\unitlength=1cm%
\begin{picture}%
(12.4,12.4)(-6.2,-6.2)%
\special{pn 8}%
%
\Large\bf\boldmath%
\small%
{%
\color[rgb]{0,0,0}%
\special{pn 4}%
\special{pa -2362 -2362}\special{pa -2362  2362}%
\special{fp}%
\special{pn 8}%
}%
{%
\color[rgb]{0,0,0}%
\special{pn 4}%
\special{pa -1969 -2362}\special{pa -1969  2362}%
\special{fp}%
\special{pn 8}%
}%
{%
\color[rgb]{0,0,0}%
\special{pn 4}%
\special{pa -1575 -2362}\special{pa -1575  2362}%
\special{fp}%
\special{pn 8}%
}%
{%
\color[rgb]{0,0,0}%
\special{pn 4}%
\special{pa -1181 -2362}\special{pa -1181  2362}%
\special{fp}%
\special{pn 8}%
}%
{%
\color[rgb]{0,0,0}%
\special{pn 4}%
\special{pa  -787 -2362}\special{pa  -787  2362}%
\special{fp}%
\special{pn 8}%
}%
{%
\color[rgb]{0,0,0}%
\special{pn 4}%
\special{pa  -394 -2362}\special{pa  -394  2362}%
\special{fp}%
\special{pn 8}%
}%
{%
\color[rgb]{0,0,0}%
\special{pn 4}%
\special{pa     0 -2362}\special{pa     0  2362}%
\special{fp}%
\special{pn 8}%
}%
{%
\color[rgb]{0,0,0}%
\special{pn 4}%
\special{pa   394 -2362}\special{pa   394  2362}%
\special{fp}%
\special{pn 8}%
}%
{%
\color[rgb]{0,0,0}%
\special{pn 4}%
\special{pa   787 -2362}\special{pa   787  2362}%
\special{fp}%
\special{pn 8}%
}%
{%
\color[rgb]{0,0,0}%
\special{pn 4}%
\special{pa  1181 -2362}\special{pa  1181  2362}%
\special{fp}%
\special{pn 8}%
}%
{%
\color[rgb]{0,0,0}%
\special{pn 4}%
\special{pa  1575 -2362}\special{pa  1575  2362}%
\special{fp}%
\special{pn 8}%
}%
{%
\color[rgb]{0,0,0}%
\special{pn 4}%
\special{pa  1969 -2362}\special{pa  1969  2362}%
\special{fp}%
\special{pn 8}%
}%
{%
\color[rgb]{0,0,0}%
\special{pn 4}%
\special{pa  2362 -2362}\special{pa  2362  2362}%
\special{fp}%
\special{pn 8}%
}%
{%
\color[rgb]{0,0,0}%
\special{pn 4}%
\special{pa -2362 -2362}\special{pa  2362 -2362}%
\special{fp}%
\special{pn 8}%
}%
{%
\color[rgb]{0,0,0}%
\special{pn 4}%
\special{pa -2362 -1969}\special{pa  2362 -1969}%
\special{fp}%
\special{pn 8}%
}%
{%
\color[rgb]{0,0,0}%
\special{pn 4}%
\special{pa -2362 -1575}\special{pa  2362 -1575}%
\special{fp}%
\special{pn 8}%
}%
{%
\color[rgb]{0,0,0}%
\special{pn 4}%
\special{pa -2362 -1181}\special{pa  2362 -1181}%
\special{fp}%
\special{pn 8}%
}%
{%
\color[rgb]{0,0,0}%
\special{pn 4}%
\special{pa -2362  -787}\special{pa  2362  -787}%
\special{fp}%
\special{pn 8}%
}%
{%
\color[rgb]{0,0,0}%
\special{pn 4}%
\special{pa -2362  -394}\special{pa  2362  -394}%
\special{fp}%
\special{pn 8}%
}%
{%
\color[rgb]{0,0,0}%
\special{pn 4}%
\special{pa -2362    -0}\special{pa  2362    -0}%
\special{fp}%
\special{pn 8}%
}%
{%
\color[rgb]{0,0,0}%
\special{pn 4}%
\special{pa -2362   394}\special{pa  2362   394}%
\special{fp}%
\special{pn 8}%
}%
{%
\color[rgb]{0,0,0}%
\special{pn 4}%
\special{pa -2362   787}\special{pa  2362   787}%
\special{fp}%
\special{pn 8}%
}%
{%
\color[rgb]{0,0,0}%
\special{pn 4}%
\special{pa -2362  1181}\special{pa  2362  1181}%
\special{fp}%
\special{pn 8}%
}%
{%
\color[rgb]{0,0,0}%
\special{pn 4}%
\special{pa -2362  1575}\special{pa  2362  1575}%
\special{fp}%
\special{pn 8}%
}%
{%
\color[rgb]{0,0,0}%
\special{pn 4}%
\special{pa -2362  1969}\special{pa  2362  1969}%
\special{fp}%
\special{pn 8}%
}%
{%
\color[rgb]{0,0,0}%
\special{pn 4}%
\special{pa -2362  2362}\special{pa  2362  2362}%
\special{fp}%
\special{pn 8}%
}%
\special{pn 4}%
{%
\color[rgb]{0,0,0}%
\special{pa -2165 2362}\special{pa -2165 2323}\special{fp}\special{pa -2165 2284}\special{pa -2165 2245}\special{fp}%
\special{pa -2165 2206}\special{pa -2165 2167}\special{fp}\special{pa -2165 2128}\special{pa -2165 2089}\special{fp}%
\special{pa -2165 2050}\special{pa -2165 2011}\special{fp}\special{pa -2165 1972}\special{pa -2165 1933}\special{fp}%
\special{pa -2165 1894}\special{pa -2165 1855}\special{fp}\special{pa -2165 1816}\special{pa -2165 1777}\special{fp}%
\special{pa -2165 1737}\special{pa -2165 1698}\special{fp}\special{pa -2165 1659}\special{pa -2165 1620}\special{fp}%
\special{pa -2165 1581}\special{pa -2165 1542}\special{fp}\special{pa -2165 1503}\special{pa -2165 1464}\special{fp}%
\special{pa -2165 1425}\special{pa -2165 1386}\special{fp}\special{pa -2165 1347}\special{pa -2165 1308}\special{fp}%
\special{pa -2165 1269}\special{pa -2165 1230}\special{fp}\special{pa -2165 1191}\special{pa -2165 1152}\special{fp}%
\special{pa -2165 1113}\special{pa -2165 1074}\special{fp}\special{pa -2165 1035}\special{pa -2165 996}\special{fp}%
\special{pa -2165 957}\special{pa -2165 918}\special{fp}\special{pa -2165 879}\special{pa -2165 839}\special{fp}%
\special{pa -2165 800}\special{pa -2165 761}\special{fp}\special{pa -2165 722}\special{pa -2165 683}\special{fp}%
\special{pa -2165 644}\special{pa -2165 605}\special{fp}\special{pa -2165 566}\special{pa -2165 527}\special{fp}%
\special{pa -2165 488}\special{pa -2165 449}\special{fp}\special{pa -2165 410}\special{pa -2165 371}\special{fp}%
\special{pa -2165 332}\special{pa -2165 293}\special{fp}\special{pa -2165 254}\special{pa -2165 215}\special{fp}%
\special{pa -2165 176}\special{pa -2165 137}\special{fp}\special{pa -2165 98}\special{pa -2165 59}\special{fp}%
\special{pa -2165 20}\special{pa -2165 -20}\special{fp}\special{pa -2165 -59}\special{pa -2165 -98}\special{fp}%
\special{pa -2165 -137}\special{pa -2165 -176}\special{fp}\special{pa -2165 -215}\special{pa -2165 -254}\special{fp}%
\special{pa -2165 -293}\special{pa -2165 -332}\special{fp}\special{pa -2165 -371}\special{pa -2165 -410}\special{fp}%
\special{pa -2165 -449}\special{pa -2165 -488}\special{fp}\special{pa -2165 -527}\special{pa -2165 -566}\special{fp}%
\special{pa -2165 -605}\special{pa -2165 -644}\special{fp}\special{pa -2165 -683}\special{pa -2165 -722}\special{fp}%
\special{pa -2165 -761}\special{pa -2165 -800}\special{fp}\special{pa -2165 -839}\special{pa -2165 -879}\special{fp}%
\special{pa -2165 -918}\special{pa -2165 -957}\special{fp}\special{pa -2165 -996}\special{pa -2165 -1035}\special{fp}%
\special{pa -2165 -1074}\special{pa -2165 -1113}\special{fp}\special{pa -2165 -1152}\special{pa -2165 -1191}\special{fp}%
\special{pa -2165 -1230}\special{pa -2165 -1269}\special{fp}\special{pa -2165 -1308}\special{pa -2165 -1347}\special{fp}%
\special{pa -2165 -1386}\special{pa -2165 -1425}\special{fp}\special{pa -2165 -1464}\special{pa -2165 -1503}\special{fp}%
\special{pa -2165 -1542}\special{pa -2165 -1581}\special{fp}\special{pa -2165 -1620}\special{pa -2165 -1659}\special{fp}%
\special{pa -2165 -1698}\special{pa -2165 -1737}\special{fp}\special{pa -2165 -1777}\special{pa -2165 -1816}\special{fp}%
\special{pa -2165 -1855}\special{pa -2165 -1894}\special{fp}\special{pa -2165 -1933}\special{pa -2165 -1972}\special{fp}%
\special{pa -2165 -2011}\special{pa -2165 -2050}\special{fp}\special{pa -2165 -2089}\special{pa -2165 -2128}\special{fp}%
\special{pa -2165 -2167}\special{pa -2165 -2206}\special{fp}\special{pa -2165 -2245}\special{pa -2165 -2284}\special{fp}%
\special{pa -2165 -2323}\special{pa -2165 -2362}\special{fp}%
%
}%
{%
\color[rgb]{0,0,0}%
\special{pa -2362 2165}\special{pa -2323 2165}\special{fp}\special{pa -2284 2165}\special{pa -2245 2165}\special{fp}%
\special{pa -2206 2165}\special{pa -2167 2165}\special{fp}\special{pa -2128 2165}\special{pa -2089 2165}\special{fp}%
\special{pa -2050 2165}\special{pa -2011 2165}\special{fp}\special{pa -1972 2165}\special{pa -1933 2165}\special{fp}%
\special{pa -1894 2165}\special{pa -1855 2165}\special{fp}\special{pa -1816 2165}\special{pa -1777 2165}\special{fp}%
\special{pa -1737 2165}\special{pa -1698 2165}\special{fp}\special{pa -1659 2165}\special{pa -1620 2165}\special{fp}%
\special{pa -1581 2165}\special{pa -1542 2165}\special{fp}\special{pa -1503 2165}\special{pa -1464 2165}\special{fp}%
\special{pa -1425 2165}\special{pa -1386 2165}\special{fp}\special{pa -1347 2165}\special{pa -1308 2165}\special{fp}%
\special{pa -1269 2165}\special{pa -1230 2165}\special{fp}\special{pa -1191 2165}\special{pa -1152 2165}\special{fp}%
\special{pa -1113 2165}\special{pa -1074 2165}\special{fp}\special{pa -1035 2165}\special{pa -996 2165}\special{fp}%
\special{pa -957 2165}\special{pa -918 2165}\special{fp}\special{pa -879 2165}\special{pa -839 2165}\special{fp}%
\special{pa -800 2165}\special{pa -761 2165}\special{fp}\special{pa -722 2165}\special{pa -683 2165}\special{fp}%
\special{pa -644 2165}\special{pa -605 2165}\special{fp}\special{pa -566 2165}\special{pa -527 2165}\special{fp}%
\special{pa -488 2165}\special{pa -449 2165}\special{fp}\special{pa -410 2165}\special{pa -371 2165}\special{fp}%
\special{pa -332 2165}\special{pa -293 2165}\special{fp}\special{pa -254 2165}\special{pa -215 2165}\special{fp}%
\special{pa -176 2165}\special{pa -137 2165}\special{fp}\special{pa -98 2165}\special{pa -59 2165}\special{fp}%
\special{pa -20 2165}\special{pa 20 2165}\special{fp}\special{pa 59 2165}\special{pa 98 2165}\special{fp}%
\special{pa 137 2165}\special{pa 176 2165}\special{fp}\special{pa 215 2165}\special{pa 254 2165}\special{fp}%
\special{pa 293 2165}\special{pa 332 2165}\special{fp}\special{pa 371 2165}\special{pa 410 2165}\special{fp}%
\special{pa 449 2165}\special{pa 488 2165}\special{fp}\special{pa 527 2165}\special{pa 566 2165}\special{fp}%
\special{pa 605 2165}\special{pa 644 2165}\special{fp}\special{pa 683 2165}\special{pa 722 2165}\special{fp}%
\special{pa 761 2165}\special{pa 800 2165}\special{fp}\special{pa 839 2165}\special{pa 879 2165}\special{fp}%
\special{pa 918 2165}\special{pa 957 2165}\special{fp}\special{pa 996 2165}\special{pa 1035 2165}\special{fp}%
\special{pa 1074 2165}\special{pa 1113 2165}\special{fp}\special{pa 1152 2165}\special{pa 1191 2165}\special{fp}%
\special{pa 1230 2165}\special{pa 1269 2165}\special{fp}\special{pa 1308 2165}\special{pa 1347 2165}\special{fp}%
\special{pa 1386 2165}\special{pa 1425 2165}\special{fp}\special{pa 1464 2165}\special{pa 1503 2165}\special{fp}%
\special{pa 1542 2165}\special{pa 1581 2165}\special{fp}\special{pa 1620 2165}\special{pa 1659 2165}\special{fp}%
\special{pa 1698 2165}\special{pa 1737 2165}\special{fp}\special{pa 1777 2165}\special{pa 1816 2165}\special{fp}%
\special{pa 1855 2165}\special{pa 1894 2165}\special{fp}\special{pa 1933 2165}\special{pa 1972 2165}\special{fp}%
\special{pa 2011 2165}\special{pa 2050 2165}\special{fp}\special{pa 2089 2165}\special{pa 2128 2165}\special{fp}%
\special{pa 2167 2165}\special{pa 2206 2165}\special{fp}\special{pa 2245 2165}\special{pa 2284 2165}\special{fp}%
\special{pa 2323 2165}\special{pa 2362 2165}\special{fp}%
%
}%
{%
\color[rgb]{0,0,0}%
\special{pa -1772 2362}\special{pa -1772 2323}\special{fp}\special{pa -1772 2284}\special{pa -1772 2245}\special{fp}%
\special{pa -1772 2206}\special{pa -1772 2167}\special{fp}\special{pa -1772 2128}\special{pa -1772 2089}\special{fp}%
\special{pa -1772 2050}\special{pa -1772 2011}\special{fp}\special{pa -1772 1972}\special{pa -1772 1933}\special{fp}%
\special{pa -1772 1894}\special{pa -1772 1855}\special{fp}\special{pa -1772 1816}\special{pa -1772 1777}\special{fp}%
\special{pa -1772 1737}\special{pa -1772 1698}\special{fp}\special{pa -1772 1659}\special{pa -1772 1620}\special{fp}%
\special{pa -1772 1581}\special{pa -1772 1542}\special{fp}\special{pa -1772 1503}\special{pa -1772 1464}\special{fp}%
\special{pa -1772 1425}\special{pa -1772 1386}\special{fp}\special{pa -1772 1347}\special{pa -1772 1308}\special{fp}%
\special{pa -1772 1269}\special{pa -1772 1230}\special{fp}\special{pa -1772 1191}\special{pa -1772 1152}\special{fp}%
\special{pa -1772 1113}\special{pa -1772 1074}\special{fp}\special{pa -1772 1035}\special{pa -1772 996}\special{fp}%
\special{pa -1772 957}\special{pa -1772 918}\special{fp}\special{pa -1772 879}\special{pa -1772 839}\special{fp}%
\special{pa -1772 800}\special{pa -1772 761}\special{fp}\special{pa -1772 722}\special{pa -1772 683}\special{fp}%
\special{pa -1772 644}\special{pa -1772 605}\special{fp}\special{pa -1772 566}\special{pa -1772 527}\special{fp}%
\special{pa -1772 488}\special{pa -1772 449}\special{fp}\special{pa -1772 410}\special{pa -1772 371}\special{fp}%
\special{pa -1772 332}\special{pa -1772 293}\special{fp}\special{pa -1772 254}\special{pa -1772 215}\special{fp}%
\special{pa -1772 176}\special{pa -1772 137}\special{fp}\special{pa -1772 98}\special{pa -1772 59}\special{fp}%
\special{pa -1772 20}\special{pa -1772 -20}\special{fp}\special{pa -1772 -59}\special{pa -1772 -98}\special{fp}%
\special{pa -1772 -137}\special{pa -1772 -176}\special{fp}\special{pa -1772 -215}\special{pa -1772 -254}\special{fp}%
\special{pa -1772 -293}\special{pa -1772 -332}\special{fp}\special{pa -1772 -371}\special{pa -1772 -410}\special{fp}%
\special{pa -1772 -449}\special{pa -1772 -488}\special{fp}\special{pa -1772 -527}\special{pa -1772 -566}\special{fp}%
\special{pa -1772 -605}\special{pa -1772 -644}\special{fp}\special{pa -1772 -683}\special{pa -1772 -722}\special{fp}%
\special{pa -1772 -761}\special{pa -1772 -800}\special{fp}\special{pa -1772 -839}\special{pa -1772 -879}\special{fp}%
\special{pa -1772 -918}\special{pa -1772 -957}\special{fp}\special{pa -1772 -996}\special{pa -1772 -1035}\special{fp}%
\special{pa -1772 -1074}\special{pa -1772 -1113}\special{fp}\special{pa -1772 -1152}\special{pa -1772 -1191}\special{fp}%
\special{pa -1772 -1230}\special{pa -1772 -1269}\special{fp}\special{pa -1772 -1308}\special{pa -1772 -1347}\special{fp}%
\special{pa -1772 -1386}\special{pa -1772 -1425}\special{fp}\special{pa -1772 -1464}\special{pa -1772 -1503}\special{fp}%
\special{pa -1772 -1542}\special{pa -1772 -1581}\special{fp}\special{pa -1772 -1620}\special{pa -1772 -1659}\special{fp}%
\special{pa -1772 -1698}\special{pa -1772 -1737}\special{fp}\special{pa -1772 -1777}\special{pa -1772 -1816}\special{fp}%
\special{pa -1772 -1855}\special{pa -1772 -1894}\special{fp}\special{pa -1772 -1933}\special{pa -1772 -1972}\special{fp}%
\special{pa -1772 -2011}\special{pa -1772 -2050}\special{fp}\special{pa -1772 -2089}\special{pa -1772 -2128}\special{fp}%
\special{pa -1772 -2167}\special{pa -1772 -2206}\special{fp}\special{pa -1772 -2245}\special{pa -1772 -2284}\special{fp}%
\special{pa -1772 -2323}\special{pa -1772 -2362}\special{fp}%
%
}%
{%
\color[rgb]{0,0,0}%
\special{pa -2362 1772}\special{pa -2323 1772}\special{fp}\special{pa -2284 1772}\special{pa -2245 1772}\special{fp}%
\special{pa -2206 1772}\special{pa -2167 1772}\special{fp}\special{pa -2128 1772}\special{pa -2089 1772}\special{fp}%
\special{pa -2050 1772}\special{pa -2011 1772}\special{fp}\special{pa -1972 1772}\special{pa -1933 1772}\special{fp}%
\special{pa -1894 1772}\special{pa -1855 1772}\special{fp}\special{pa -1816 1772}\special{pa -1777 1772}\special{fp}%
\special{pa -1737 1772}\special{pa -1698 1772}\special{fp}\special{pa -1659 1772}\special{pa -1620 1772}\special{fp}%
\special{pa -1581 1772}\special{pa -1542 1772}\special{fp}\special{pa -1503 1772}\special{pa -1464 1772}\special{fp}%
\special{pa -1425 1772}\special{pa -1386 1772}\special{fp}\special{pa -1347 1772}\special{pa -1308 1772}\special{fp}%
\special{pa -1269 1772}\special{pa -1230 1772}\special{fp}\special{pa -1191 1772}\special{pa -1152 1772}\special{fp}%
\special{pa -1113 1772}\special{pa -1074 1772}\special{fp}\special{pa -1035 1772}\special{pa -996 1772}\special{fp}%
\special{pa -957 1772}\special{pa -918 1772}\special{fp}\special{pa -879 1772}\special{pa -839 1772}\special{fp}%
\special{pa -800 1772}\special{pa -761 1772}\special{fp}\special{pa -722 1772}\special{pa -683 1772}\special{fp}%
\special{pa -644 1772}\special{pa -605 1772}\special{fp}\special{pa -566 1772}\special{pa -527 1772}\special{fp}%
\special{pa -488 1772}\special{pa -449 1772}\special{fp}\special{pa -410 1772}\special{pa -371 1772}\special{fp}%
\special{pa -332 1772}\special{pa -293 1772}\special{fp}\special{pa -254 1772}\special{pa -215 1772}\special{fp}%
\special{pa -176 1772}\special{pa -137 1772}\special{fp}\special{pa -98 1772}\special{pa -59 1772}\special{fp}%
\special{pa -20 1772}\special{pa 20 1772}\special{fp}\special{pa 59 1772}\special{pa 98 1772}\special{fp}%
\special{pa 137 1772}\special{pa 176 1772}\special{fp}\special{pa 215 1772}\special{pa 254 1772}\special{fp}%
\special{pa 293 1772}\special{pa 332 1772}\special{fp}\special{pa 371 1772}\special{pa 410 1772}\special{fp}%
\special{pa 449 1772}\special{pa 488 1772}\special{fp}\special{pa 527 1772}\special{pa 566 1772}\special{fp}%
\special{pa 605 1772}\special{pa 644 1772}\special{fp}\special{pa 683 1772}\special{pa 722 1772}\special{fp}%
\special{pa 761 1772}\special{pa 800 1772}\special{fp}\special{pa 839 1772}\special{pa 879 1772}\special{fp}%
\special{pa 918 1772}\special{pa 957 1772}\special{fp}\special{pa 996 1772}\special{pa 1035 1772}\special{fp}%
\special{pa 1074 1772}\special{pa 1113 1772}\special{fp}\special{pa 1152 1772}\special{pa 1191 1772}\special{fp}%
\special{pa 1230 1772}\special{pa 1269 1772}\special{fp}\special{pa 1308 1772}\special{pa 1347 1772}\special{fp}%
\special{pa 1386 1772}\special{pa 1425 1772}\special{fp}\special{pa 1464 1772}\special{pa 1503 1772}\special{fp}%
\special{pa 1542 1772}\special{pa 1581 1772}\special{fp}\special{pa 1620 1772}\special{pa 1659 1772}\special{fp}%
\special{pa 1698 1772}\special{pa 1737 1772}\special{fp}\special{pa 1777 1772}\special{pa 1816 1772}\special{fp}%
\special{pa 1855 1772}\special{pa 1894 1772}\special{fp}\special{pa 1933 1772}\special{pa 1972 1772}\special{fp}%
\special{pa 2011 1772}\special{pa 2050 1772}\special{fp}\special{pa 2089 1772}\special{pa 2128 1772}\special{fp}%
\special{pa 2167 1772}\special{pa 2206 1772}\special{fp}\special{pa 2245 1772}\special{pa 2284 1772}\special{fp}%
\special{pa 2323 1772}\special{pa 2362 1772}\special{fp}%
%
}%
{%
\color[rgb]{0,0,0}%
\special{pa -1378 2362}\special{pa -1378 2323}\special{fp}\special{pa -1378 2284}\special{pa -1378 2245}\special{fp}%
\special{pa -1378 2206}\special{pa -1378 2167}\special{fp}\special{pa -1378 2128}\special{pa -1378 2089}\special{fp}%
\special{pa -1378 2050}\special{pa -1378 2011}\special{fp}\special{pa -1378 1972}\special{pa -1378 1933}\special{fp}%
\special{pa -1378 1894}\special{pa -1378 1855}\special{fp}\special{pa -1378 1816}\special{pa -1378 1777}\special{fp}%
\special{pa -1378 1737}\special{pa -1378 1698}\special{fp}\special{pa -1378 1659}\special{pa -1378 1620}\special{fp}%
\special{pa -1378 1581}\special{pa -1378 1542}\special{fp}\special{pa -1378 1503}\special{pa -1378 1464}\special{fp}%
\special{pa -1378 1425}\special{pa -1378 1386}\special{fp}\special{pa -1378 1347}\special{pa -1378 1308}\special{fp}%
\special{pa -1378 1269}\special{pa -1378 1230}\special{fp}\special{pa -1378 1191}\special{pa -1378 1152}\special{fp}%
\special{pa -1378 1113}\special{pa -1378 1074}\special{fp}\special{pa -1378 1035}\special{pa -1378 996}\special{fp}%
\special{pa -1378 957}\special{pa -1378 918}\special{fp}\special{pa -1378 879}\special{pa -1378 839}\special{fp}%
\special{pa -1378 800}\special{pa -1378 761}\special{fp}\special{pa -1378 722}\special{pa -1378 683}\special{fp}%
\special{pa -1378 644}\special{pa -1378 605}\special{fp}\special{pa -1378 566}\special{pa -1378 527}\special{fp}%
\special{pa -1378 488}\special{pa -1378 449}\special{fp}\special{pa -1378 410}\special{pa -1378 371}\special{fp}%
\special{pa -1378 332}\special{pa -1378 293}\special{fp}\special{pa -1378 254}\special{pa -1378 215}\special{fp}%
\special{pa -1378 176}\special{pa -1378 137}\special{fp}\special{pa -1378 98}\special{pa -1378 59}\special{fp}%
\special{pa -1378 20}\special{pa -1378 -20}\special{fp}\special{pa -1378 -59}\special{pa -1378 -98}\special{fp}%
\special{pa -1378 -137}\special{pa -1378 -176}\special{fp}\special{pa -1378 -215}\special{pa -1378 -254}\special{fp}%
\special{pa -1378 -293}\special{pa -1378 -332}\special{fp}\special{pa -1378 -371}\special{pa -1378 -410}\special{fp}%
\special{pa -1378 -449}\special{pa -1378 -488}\special{fp}\special{pa -1378 -527}\special{pa -1378 -566}\special{fp}%
\special{pa -1378 -605}\special{pa -1378 -644}\special{fp}\special{pa -1378 -683}\special{pa -1378 -722}\special{fp}%
\special{pa -1378 -761}\special{pa -1378 -800}\special{fp}\special{pa -1378 -839}\special{pa -1378 -879}\special{fp}%
\special{pa -1378 -918}\special{pa -1378 -957}\special{fp}\special{pa -1378 -996}\special{pa -1378 -1035}\special{fp}%
\special{pa -1378 -1074}\special{pa -1378 -1113}\special{fp}\special{pa -1378 -1152}\special{pa -1378 -1191}\special{fp}%
\special{pa -1378 -1230}\special{pa -1378 -1269}\special{fp}\special{pa -1378 -1308}\special{pa -1378 -1347}\special{fp}%
\special{pa -1378 -1386}\special{pa -1378 -1425}\special{fp}\special{pa -1378 -1464}\special{pa -1378 -1503}\special{fp}%
\special{pa -1378 -1542}\special{pa -1378 -1581}\special{fp}\special{pa -1378 -1620}\special{pa -1378 -1659}\special{fp}%
\special{pa -1378 -1698}\special{pa -1378 -1737}\special{fp}\special{pa -1378 -1777}\special{pa -1378 -1816}\special{fp}%
\special{pa -1378 -1855}\special{pa -1378 -1894}\special{fp}\special{pa -1378 -1933}\special{pa -1378 -1972}\special{fp}%
\special{pa -1378 -2011}\special{pa -1378 -2050}\special{fp}\special{pa -1378 -2089}\special{pa -1378 -2128}\special{fp}%
\special{pa -1378 -2167}\special{pa -1378 -2206}\special{fp}\special{pa -1378 -2245}\special{pa -1378 -2284}\special{fp}%
\special{pa -1378 -2323}\special{pa -1378 -2362}\special{fp}%
%
}%
{%
\color[rgb]{0,0,0}%
\special{pa -2362 1378}\special{pa -2323 1378}\special{fp}\special{pa -2284 1378}\special{pa -2245 1378}\special{fp}%
\special{pa -2206 1378}\special{pa -2167 1378}\special{fp}\special{pa -2128 1378}\special{pa -2089 1378}\special{fp}%
\special{pa -2050 1378}\special{pa -2011 1378}\special{fp}\special{pa -1972 1378}\special{pa -1933 1378}\special{fp}%
\special{pa -1894 1378}\special{pa -1855 1378}\special{fp}\special{pa -1816 1378}\special{pa -1777 1378}\special{fp}%
\special{pa -1737 1378}\special{pa -1698 1378}\special{fp}\special{pa -1659 1378}\special{pa -1620 1378}\special{fp}%
\special{pa -1581 1378}\special{pa -1542 1378}\special{fp}\special{pa -1503 1378}\special{pa -1464 1378}\special{fp}%
\special{pa -1425 1378}\special{pa -1386 1378}\special{fp}\special{pa -1347 1378}\special{pa -1308 1378}\special{fp}%
\special{pa -1269 1378}\special{pa -1230 1378}\special{fp}\special{pa -1191 1378}\special{pa -1152 1378}\special{fp}%
\special{pa -1113 1378}\special{pa -1074 1378}\special{fp}\special{pa -1035 1378}\special{pa -996 1378}\special{fp}%
\special{pa -957 1378}\special{pa -918 1378}\special{fp}\special{pa -879 1378}\special{pa -839 1378}\special{fp}%
\special{pa -800 1378}\special{pa -761 1378}\special{fp}\special{pa -722 1378}\special{pa -683 1378}\special{fp}%
\special{pa -644 1378}\special{pa -605 1378}\special{fp}\special{pa -566 1378}\special{pa -527 1378}\special{fp}%
\special{pa -488 1378}\special{pa -449 1378}\special{fp}\special{pa -410 1378}\special{pa -371 1378}\special{fp}%
\special{pa -332 1378}\special{pa -293 1378}\special{fp}\special{pa -254 1378}\special{pa -215 1378}\special{fp}%
\special{pa -176 1378}\special{pa -137 1378}\special{fp}\special{pa -98 1378}\special{pa -59 1378}\special{fp}%
\special{pa -20 1378}\special{pa 20 1378}\special{fp}\special{pa 59 1378}\special{pa 98 1378}\special{fp}%
\special{pa 137 1378}\special{pa 176 1378}\special{fp}\special{pa 215 1378}\special{pa 254 1378}\special{fp}%
\special{pa 293 1378}\special{pa 332 1378}\special{fp}\special{pa 371 1378}\special{pa 410 1378}\special{fp}%
\special{pa 449 1378}\special{pa 488 1378}\special{fp}\special{pa 527 1378}\special{pa 566 1378}\special{fp}%
\special{pa 605 1378}\special{pa 644 1378}\special{fp}\special{pa 683 1378}\special{pa 722 1378}\special{fp}%
\special{pa 761 1378}\special{pa 800 1378}\special{fp}\special{pa 839 1378}\special{pa 879 1378}\special{fp}%
\special{pa 918 1378}\special{pa 957 1378}\special{fp}\special{pa 996 1378}\special{pa 1035 1378}\special{fp}%
\special{pa 1074 1378}\special{pa 1113 1378}\special{fp}\special{pa 1152 1378}\special{pa 1191 1378}\special{fp}%
\special{pa 1230 1378}\special{pa 1269 1378}\special{fp}\special{pa 1308 1378}\special{pa 1347 1378}\special{fp}%
\special{pa 1386 1378}\special{pa 1425 1378}\special{fp}\special{pa 1464 1378}\special{pa 1503 1378}\special{fp}%
\special{pa 1542 1378}\special{pa 1581 1378}\special{fp}\special{pa 1620 1378}\special{pa 1659 1378}\special{fp}%
\special{pa 1698 1378}\special{pa 1737 1378}\special{fp}\special{pa 1777 1378}\special{pa 1816 1378}\special{fp}%
\special{pa 1855 1378}\special{pa 1894 1378}\special{fp}\special{pa 1933 1378}\special{pa 1972 1378}\special{fp}%
\special{pa 2011 1378}\special{pa 2050 1378}\special{fp}\special{pa 2089 1378}\special{pa 2128 1378}\special{fp}%
\special{pa 2167 1378}\special{pa 2206 1378}\special{fp}\special{pa 2245 1378}\special{pa 2284 1378}\special{fp}%
\special{pa 2323 1378}\special{pa 2362 1378}\special{fp}%
%
}%
{%
\color[rgb]{0,0,0}%
\special{pa -984 2362}\special{pa -984 2323}\special{fp}\special{pa -984 2284}\special{pa -984 2245}\special{fp}%
\special{pa -984 2206}\special{pa -984 2167}\special{fp}\special{pa -984 2128}\special{pa -984 2089}\special{fp}%
\special{pa -984 2050}\special{pa -984 2011}\special{fp}\special{pa -984 1972}\special{pa -984 1933}\special{fp}%
\special{pa -984 1894}\special{pa -984 1855}\special{fp}\special{pa -984 1816}\special{pa -984 1777}\special{fp}%
\special{pa -984 1737}\special{pa -984 1698}\special{fp}\special{pa -984 1659}\special{pa -984 1620}\special{fp}%
\special{pa -984 1581}\special{pa -984 1542}\special{fp}\special{pa -984 1503}\special{pa -984 1464}\special{fp}%
\special{pa -984 1425}\special{pa -984 1386}\special{fp}\special{pa -984 1347}\special{pa -984 1308}\special{fp}%
\special{pa -984 1269}\special{pa -984 1230}\special{fp}\special{pa -984 1191}\special{pa -984 1152}\special{fp}%
\special{pa -984 1113}\special{pa -984 1074}\special{fp}\special{pa -984 1035}\special{pa -984 996}\special{fp}%
\special{pa -984 957}\special{pa -984 918}\special{fp}\special{pa -984 879}\special{pa -984 839}\special{fp}%
\special{pa -984 800}\special{pa -984 761}\special{fp}\special{pa -984 722}\special{pa -984 683}\special{fp}%
\special{pa -984 644}\special{pa -984 605}\special{fp}\special{pa -984 566}\special{pa -984 527}\special{fp}%
\special{pa -984 488}\special{pa -984 449}\special{fp}\special{pa -984 410}\special{pa -984 371}\special{fp}%
\special{pa -984 332}\special{pa -984 293}\special{fp}\special{pa -984 254}\special{pa -984 215}\special{fp}%
\special{pa -984 176}\special{pa -984 137}\special{fp}\special{pa -984 98}\special{pa -984 59}\special{fp}%
\special{pa -984 20}\special{pa -984 -20}\special{fp}\special{pa -984 -59}\special{pa -984 -98}\special{fp}%
\special{pa -984 -137}\special{pa -984 -176}\special{fp}\special{pa -984 -215}\special{pa -984 -254}\special{fp}%
\special{pa -984 -293}\special{pa -984 -332}\special{fp}\special{pa -984 -371}\special{pa -984 -410}\special{fp}%
\special{pa -984 -449}\special{pa -984 -488}\special{fp}\special{pa -984 -527}\special{pa -984 -566}\special{fp}%
\special{pa -984 -605}\special{pa -984 -644}\special{fp}\special{pa -984 -683}\special{pa -984 -722}\special{fp}%
\special{pa -984 -761}\special{pa -984 -800}\special{fp}\special{pa -984 -839}\special{pa -984 -879}\special{fp}%
\special{pa -984 -918}\special{pa -984 -957}\special{fp}\special{pa -984 -996}\special{pa -984 -1035}\special{fp}%
\special{pa -984 -1074}\special{pa -984 -1113}\special{fp}\special{pa -984 -1152}\special{pa -984 -1191}\special{fp}%
\special{pa -984 -1230}\special{pa -984 -1269}\special{fp}\special{pa -984 -1308}\special{pa -984 -1347}\special{fp}%
\special{pa -984 -1386}\special{pa -984 -1425}\special{fp}\special{pa -984 -1464}\special{pa -984 -1503}\special{fp}%
\special{pa -984 -1542}\special{pa -984 -1581}\special{fp}\special{pa -984 -1620}\special{pa -984 -1659}\special{fp}%
\special{pa -984 -1698}\special{pa -984 -1737}\special{fp}\special{pa -984 -1777}\special{pa -984 -1816}\special{fp}%
\special{pa -984 -1855}\special{pa -984 -1894}\special{fp}\special{pa -984 -1933}\special{pa -984 -1972}\special{fp}%
\special{pa -984 -2011}\special{pa -984 -2050}\special{fp}\special{pa -984 -2089}\special{pa -984 -2128}\special{fp}%
\special{pa -984 -2167}\special{pa -984 -2206}\special{fp}\special{pa -984 -2245}\special{pa -984 -2284}\special{fp}%
\special{pa -984 -2323}\special{pa -984 -2362}\special{fp}%
%
}%
{%
\color[rgb]{0,0,0}%
\special{pa -2362 984}\special{pa -2323 984}\special{fp}\special{pa -2284 984}\special{pa -2245 984}\special{fp}%
\special{pa -2206 984}\special{pa -2167 984}\special{fp}\special{pa -2128 984}\special{pa -2089 984}\special{fp}%
\special{pa -2050 984}\special{pa -2011 984}\special{fp}\special{pa -1972 984}\special{pa -1933 984}\special{fp}%
\special{pa -1894 984}\special{pa -1855 984}\special{fp}\special{pa -1816 984}\special{pa -1777 984}\special{fp}%
\special{pa -1737 984}\special{pa -1698 984}\special{fp}\special{pa -1659 984}\special{pa -1620 984}\special{fp}%
\special{pa -1581 984}\special{pa -1542 984}\special{fp}\special{pa -1503 984}\special{pa -1464 984}\special{fp}%
\special{pa -1425 984}\special{pa -1386 984}\special{fp}\special{pa -1347 984}\special{pa -1308 984}\special{fp}%
\special{pa -1269 984}\special{pa -1230 984}\special{fp}\special{pa -1191 984}\special{pa -1152 984}\special{fp}%
\special{pa -1113 984}\special{pa -1074 984}\special{fp}\special{pa -1035 984}\special{pa -996 984}\special{fp}%
\special{pa -957 984}\special{pa -918 984}\special{fp}\special{pa -879 984}\special{pa -839 984}\special{fp}%
\special{pa -800 984}\special{pa -761 984}\special{fp}\special{pa -722 984}\special{pa -683 984}\special{fp}%
\special{pa -644 984}\special{pa -605 984}\special{fp}\special{pa -566 984}\special{pa -527 984}\special{fp}%
\special{pa -488 984}\special{pa -449 984}\special{fp}\special{pa -410 984}\special{pa -371 984}\special{fp}%
\special{pa -332 984}\special{pa -293 984}\special{fp}\special{pa -254 984}\special{pa -215 984}\special{fp}%
\special{pa -176 984}\special{pa -137 984}\special{fp}\special{pa -98 984}\special{pa -59 984}\special{fp}%
\special{pa -20 984}\special{pa 20 984}\special{fp}\special{pa 59 984}\special{pa 98 984}\special{fp}%
\special{pa 137 984}\special{pa 176 984}\special{fp}\special{pa 215 984}\special{pa 254 984}\special{fp}%
\special{pa 293 984}\special{pa 332 984}\special{fp}\special{pa 371 984}\special{pa 410 984}\special{fp}%
\special{pa 449 984}\special{pa 488 984}\special{fp}\special{pa 527 984}\special{pa 566 984}\special{fp}%
\special{pa 605 984}\special{pa 644 984}\special{fp}\special{pa 683 984}\special{pa 722 984}\special{fp}%
\special{pa 761 984}\special{pa 800 984}\special{fp}\special{pa 839 984}\special{pa 879 984}\special{fp}%
\special{pa 918 984}\special{pa 957 984}\special{fp}\special{pa 996 984}\special{pa 1035 984}\special{fp}%
\special{pa 1074 984}\special{pa 1113 984}\special{fp}\special{pa 1152 984}\special{pa 1191 984}\special{fp}%
\special{pa 1230 984}\special{pa 1269 984}\special{fp}\special{pa 1308 984}\special{pa 1347 984}\special{fp}%
\special{pa 1386 984}\special{pa 1425 984}\special{fp}\special{pa 1464 984}\special{pa 1503 984}\special{fp}%
\special{pa 1542 984}\special{pa 1581 984}\special{fp}\special{pa 1620 984}\special{pa 1659 984}\special{fp}%
\special{pa 1698 984}\special{pa 1737 984}\special{fp}\special{pa 1777 984}\special{pa 1816 984}\special{fp}%
\special{pa 1855 984}\special{pa 1894 984}\special{fp}\special{pa 1933 984}\special{pa 1972 984}\special{fp}%
\special{pa 2011 984}\special{pa 2050 984}\special{fp}\special{pa 2089 984}\special{pa 2128 984}\special{fp}%
\special{pa 2167 984}\special{pa 2206 984}\special{fp}\special{pa 2245 984}\special{pa 2284 984}\special{fp}%
\special{pa 2323 984}\special{pa 2362 984}\special{fp}%
%
}%
{%
\color[rgb]{0,0,0}%
\special{pa -591 2362}\special{pa -591 2323}\special{fp}\special{pa -591 2284}\special{pa -591 2245}\special{fp}%
\special{pa -591 2206}\special{pa -591 2167}\special{fp}\special{pa -591 2128}\special{pa -591 2089}\special{fp}%
\special{pa -591 2050}\special{pa -591 2011}\special{fp}\special{pa -591 1972}\special{pa -591 1933}\special{fp}%
\special{pa -591 1894}\special{pa -591 1855}\special{fp}\special{pa -591 1816}\special{pa -591 1777}\special{fp}%
\special{pa -591 1737}\special{pa -591 1698}\special{fp}\special{pa -591 1659}\special{pa -591 1620}\special{fp}%
\special{pa -591 1581}\special{pa -591 1542}\special{fp}\special{pa -591 1503}\special{pa -591 1464}\special{fp}%
\special{pa -591 1425}\special{pa -591 1386}\special{fp}\special{pa -591 1347}\special{pa -591 1308}\special{fp}%
\special{pa -591 1269}\special{pa -591 1230}\special{fp}\special{pa -591 1191}\special{pa -591 1152}\special{fp}%
\special{pa -591 1113}\special{pa -591 1074}\special{fp}\special{pa -591 1035}\special{pa -591 996}\special{fp}%
\special{pa -591 957}\special{pa -591 918}\special{fp}\special{pa -591 879}\special{pa -591 839}\special{fp}%
\special{pa -591 800}\special{pa -591 761}\special{fp}\special{pa -591 722}\special{pa -591 683}\special{fp}%
\special{pa -591 644}\special{pa -591 605}\special{fp}\special{pa -591 566}\special{pa -591 527}\special{fp}%
\special{pa -591 488}\special{pa -591 449}\special{fp}\special{pa -591 410}\special{pa -591 371}\special{fp}%
\special{pa -591 332}\special{pa -591 293}\special{fp}\special{pa -591 254}\special{pa -591 215}\special{fp}%
\special{pa -591 176}\special{pa -591 137}\special{fp}\special{pa -591 98}\special{pa -591 59}\special{fp}%
\special{pa -591 20}\special{pa -591 -20}\special{fp}\special{pa -591 -59}\special{pa -591 -98}\special{fp}%
\special{pa -591 -137}\special{pa -591 -176}\special{fp}\special{pa -591 -215}\special{pa -591 -254}\special{fp}%
\special{pa -591 -293}\special{pa -591 -332}\special{fp}\special{pa -591 -371}\special{pa -591 -410}\special{fp}%
\special{pa -591 -449}\special{pa -591 -488}\special{fp}\special{pa -591 -527}\special{pa -591 -566}\special{fp}%
\special{pa -591 -605}\special{pa -591 -644}\special{fp}\special{pa -591 -683}\special{pa -591 -722}\special{fp}%
\special{pa -591 -761}\special{pa -591 -800}\special{fp}\special{pa -591 -839}\special{pa -591 -879}\special{fp}%
\special{pa -591 -918}\special{pa -591 -957}\special{fp}\special{pa -591 -996}\special{pa -591 -1035}\special{fp}%
\special{pa -591 -1074}\special{pa -591 -1113}\special{fp}\special{pa -591 -1152}\special{pa -591 -1191}\special{fp}%
\special{pa -591 -1230}\special{pa -591 -1269}\special{fp}\special{pa -591 -1308}\special{pa -591 -1347}\special{fp}%
\special{pa -591 -1386}\special{pa -591 -1425}\special{fp}\special{pa -591 -1464}\special{pa -591 -1503}\special{fp}%
\special{pa -591 -1542}\special{pa -591 -1581}\special{fp}\special{pa -591 -1620}\special{pa -591 -1659}\special{fp}%
\special{pa -591 -1698}\special{pa -591 -1737}\special{fp}\special{pa -591 -1777}\special{pa -591 -1816}\special{fp}%
\special{pa -591 -1855}\special{pa -591 -1894}\special{fp}\special{pa -591 -1933}\special{pa -591 -1972}\special{fp}%
\special{pa -591 -2011}\special{pa -591 -2050}\special{fp}\special{pa -591 -2089}\special{pa -591 -2128}\special{fp}%
\special{pa -591 -2167}\special{pa -591 -2206}\special{fp}\special{pa -591 -2245}\special{pa -591 -2284}\special{fp}%
\special{pa -591 -2323}\special{pa -591 -2362}\special{fp}%
%
}%
{%
\color[rgb]{0,0,0}%
\special{pa -2362 591}\special{pa -2323 591}\special{fp}\special{pa -2284 591}\special{pa -2245 591}\special{fp}%
\special{pa -2206 591}\special{pa -2167 591}\special{fp}\special{pa -2128 591}\special{pa -2089 591}\special{fp}%
\special{pa -2050 591}\special{pa -2011 591}\special{fp}\special{pa -1972 591}\special{pa -1933 591}\special{fp}%
\special{pa -1894 591}\special{pa -1855 591}\special{fp}\special{pa -1816 591}\special{pa -1777 591}\special{fp}%
\special{pa -1737 591}\special{pa -1698 591}\special{fp}\special{pa -1659 591}\special{pa -1620 591}\special{fp}%
\special{pa -1581 591}\special{pa -1542 591}\special{fp}\special{pa -1503 591}\special{pa -1464 591}\special{fp}%
\special{pa -1425 591}\special{pa -1386 591}\special{fp}\special{pa -1347 591}\special{pa -1308 591}\special{fp}%
\special{pa -1269 591}\special{pa -1230 591}\special{fp}\special{pa -1191 591}\special{pa -1152 591}\special{fp}%
\special{pa -1113 591}\special{pa -1074 591}\special{fp}\special{pa -1035 591}\special{pa -996 591}\special{fp}%
\special{pa -957 591}\special{pa -918 591}\special{fp}\special{pa -879 591}\special{pa -839 591}\special{fp}%
\special{pa -800 591}\special{pa -761 591}\special{fp}\special{pa -722 591}\special{pa -683 591}\special{fp}%
\special{pa -644 591}\special{pa -605 591}\special{fp}\special{pa -566 591}\special{pa -527 591}\special{fp}%
\special{pa -488 591}\special{pa -449 591}\special{fp}\special{pa -410 591}\special{pa -371 591}\special{fp}%
\special{pa -332 591}\special{pa -293 591}\special{fp}\special{pa -254 591}\special{pa -215 591}\special{fp}%
\special{pa -176 591}\special{pa -137 591}\special{fp}\special{pa -98 591}\special{pa -59 591}\special{fp}%
\special{pa -20 591}\special{pa 20 591}\special{fp}\special{pa 59 591}\special{pa 98 591}\special{fp}%
\special{pa 137 591}\special{pa 176 591}\special{fp}\special{pa 215 591}\special{pa 254 591}\special{fp}%
\special{pa 293 591}\special{pa 332 591}\special{fp}\special{pa 371 591}\special{pa 410 591}\special{fp}%
\special{pa 449 591}\special{pa 488 591}\special{fp}\special{pa 527 591}\special{pa 566 591}\special{fp}%
\special{pa 605 591}\special{pa 644 591}\special{fp}\special{pa 683 591}\special{pa 722 591}\special{fp}%
\special{pa 761 591}\special{pa 800 591}\special{fp}\special{pa 839 591}\special{pa 879 591}\special{fp}%
\special{pa 918 591}\special{pa 957 591}\special{fp}\special{pa 996 591}\special{pa 1035 591}\special{fp}%
\special{pa 1074 591}\special{pa 1113 591}\special{fp}\special{pa 1152 591}\special{pa 1191 591}\special{fp}%
\special{pa 1230 591}\special{pa 1269 591}\special{fp}\special{pa 1308 591}\special{pa 1347 591}\special{fp}%
\special{pa 1386 591}\special{pa 1425 591}\special{fp}\special{pa 1464 591}\special{pa 1503 591}\special{fp}%
\special{pa 1542 591}\special{pa 1581 591}\special{fp}\special{pa 1620 591}\special{pa 1659 591}\special{fp}%
\special{pa 1698 591}\special{pa 1737 591}\special{fp}\special{pa 1777 591}\special{pa 1816 591}\special{fp}%
\special{pa 1855 591}\special{pa 1894 591}\special{fp}\special{pa 1933 591}\special{pa 1972 591}\special{fp}%
\special{pa 2011 591}\special{pa 2050 591}\special{fp}\special{pa 2089 591}\special{pa 2128 591}\special{fp}%
\special{pa 2167 591}\special{pa 2206 591}\special{fp}\special{pa 2245 591}\special{pa 2284 591}\special{fp}%
\special{pa 2323 591}\special{pa 2362 591}\special{fp}%
%
}%
{%
\color[rgb]{0,0,0}%
\special{pa -197 2362}\special{pa -197 2323}\special{fp}\special{pa -197 2284}\special{pa -197 2245}\special{fp}%
\special{pa -197 2206}\special{pa -197 2167}\special{fp}\special{pa -197 2128}\special{pa -197 2089}\special{fp}%
\special{pa -197 2050}\special{pa -197 2011}\special{fp}\special{pa -197 1972}\special{pa -197 1933}\special{fp}%
\special{pa -197 1894}\special{pa -197 1855}\special{fp}\special{pa -197 1816}\special{pa -197 1777}\special{fp}%
\special{pa -197 1737}\special{pa -197 1698}\special{fp}\special{pa -197 1659}\special{pa -197 1620}\special{fp}%
\special{pa -197 1581}\special{pa -197 1542}\special{fp}\special{pa -197 1503}\special{pa -197 1464}\special{fp}%
\special{pa -197 1425}\special{pa -197 1386}\special{fp}\special{pa -197 1347}\special{pa -197 1308}\special{fp}%
\special{pa -197 1269}\special{pa -197 1230}\special{fp}\special{pa -197 1191}\special{pa -197 1152}\special{fp}%
\special{pa -197 1113}\special{pa -197 1074}\special{fp}\special{pa -197 1035}\special{pa -197 996}\special{fp}%
\special{pa -197 957}\special{pa -197 918}\special{fp}\special{pa -197 879}\special{pa -197 839}\special{fp}%
\special{pa -197 800}\special{pa -197 761}\special{fp}\special{pa -197 722}\special{pa -197 683}\special{fp}%
\special{pa -197 644}\special{pa -197 605}\special{fp}\special{pa -197 566}\special{pa -197 527}\special{fp}%
\special{pa -197 488}\special{pa -197 449}\special{fp}\special{pa -197 410}\special{pa -197 371}\special{fp}%
\special{pa -197 332}\special{pa -197 293}\special{fp}\special{pa -197 254}\special{pa -197 215}\special{fp}%
\special{pa -197 176}\special{pa -197 137}\special{fp}\special{pa -197 98}\special{pa -197 59}\special{fp}%
\special{pa -197 20}\special{pa -197 -20}\special{fp}\special{pa -197 -59}\special{pa -197 -98}\special{fp}%
\special{pa -197 -137}\special{pa -197 -176}\special{fp}\special{pa -197 -215}\special{pa -197 -254}\special{fp}%
\special{pa -197 -293}\special{pa -197 -332}\special{fp}\special{pa -197 -371}\special{pa -197 -410}\special{fp}%
\special{pa -197 -449}\special{pa -197 -488}\special{fp}\special{pa -197 -527}\special{pa -197 -566}\special{fp}%
\special{pa -197 -605}\special{pa -197 -644}\special{fp}\special{pa -197 -683}\special{pa -197 -722}\special{fp}%
\special{pa -197 -761}\special{pa -197 -800}\special{fp}\special{pa -197 -839}\special{pa -197 -879}\special{fp}%
\special{pa -197 -918}\special{pa -197 -957}\special{fp}\special{pa -197 -996}\special{pa -197 -1035}\special{fp}%
\special{pa -197 -1074}\special{pa -197 -1113}\special{fp}\special{pa -197 -1152}\special{pa -197 -1191}\special{fp}%
\special{pa -197 -1230}\special{pa -197 -1269}\special{fp}\special{pa -197 -1308}\special{pa -197 -1347}\special{fp}%
\special{pa -197 -1386}\special{pa -197 -1425}\special{fp}\special{pa -197 -1464}\special{pa -197 -1503}\special{fp}%
\special{pa -197 -1542}\special{pa -197 -1581}\special{fp}\special{pa -197 -1620}\special{pa -197 -1659}\special{fp}%
\special{pa -197 -1698}\special{pa -197 -1737}\special{fp}\special{pa -197 -1777}\special{pa -197 -1816}\special{fp}%
\special{pa -197 -1855}\special{pa -197 -1894}\special{fp}\special{pa -197 -1933}\special{pa -197 -1972}\special{fp}%
\special{pa -197 -2011}\special{pa -197 -2050}\special{fp}\special{pa -197 -2089}\special{pa -197 -2128}\special{fp}%
\special{pa -197 -2167}\special{pa -197 -2206}\special{fp}\special{pa -197 -2245}\special{pa -197 -2284}\special{fp}%
\special{pa -197 -2323}\special{pa -197 -2362}\special{fp}%
%
}%
{%
\color[rgb]{0,0,0}%
\special{pa -2362 197}\special{pa -2323 197}\special{fp}\special{pa -2284 197}\special{pa -2245 197}\special{fp}%
\special{pa -2206 197}\special{pa -2167 197}\special{fp}\special{pa -2128 197}\special{pa -2089 197}\special{fp}%
\special{pa -2050 197}\special{pa -2011 197}\special{fp}\special{pa -1972 197}\special{pa -1933 197}\special{fp}%
\special{pa -1894 197}\special{pa -1855 197}\special{fp}\special{pa -1816 197}\special{pa -1777 197}\special{fp}%
\special{pa -1737 197}\special{pa -1698 197}\special{fp}\special{pa -1659 197}\special{pa -1620 197}\special{fp}%
\special{pa -1581 197}\special{pa -1542 197}\special{fp}\special{pa -1503 197}\special{pa -1464 197}\special{fp}%
\special{pa -1425 197}\special{pa -1386 197}\special{fp}\special{pa -1347 197}\special{pa -1308 197}\special{fp}%
\special{pa -1269 197}\special{pa -1230 197}\special{fp}\special{pa -1191 197}\special{pa -1152 197}\special{fp}%
\special{pa -1113 197}\special{pa -1074 197}\special{fp}\special{pa -1035 197}\special{pa -996 197}\special{fp}%
\special{pa -957 197}\special{pa -918 197}\special{fp}\special{pa -879 197}\special{pa -839 197}\special{fp}%
\special{pa -800 197}\special{pa -761 197}\special{fp}\special{pa -722 197}\special{pa -683 197}\special{fp}%
\special{pa -644 197}\special{pa -605 197}\special{fp}\special{pa -566 197}\special{pa -527 197}\special{fp}%
\special{pa -488 197}\special{pa -449 197}\special{fp}\special{pa -410 197}\special{pa -371 197}\special{fp}%
\special{pa -332 197}\special{pa -293 197}\special{fp}\special{pa -254 197}\special{pa -215 197}\special{fp}%
\special{pa -176 197}\special{pa -137 197}\special{fp}\special{pa -98 197}\special{pa -59 197}\special{fp}%
\special{pa -20 197}\special{pa 20 197}\special{fp}\special{pa 59 197}\special{pa 98 197}\special{fp}%
\special{pa 137 197}\special{pa 176 197}\special{fp}\special{pa 215 197}\special{pa 254 197}\special{fp}%
\special{pa 293 197}\special{pa 332 197}\special{fp}\special{pa 371 197}\special{pa 410 197}\special{fp}%
\special{pa 449 197}\special{pa 488 197}\special{fp}\special{pa 527 197}\special{pa 566 197}\special{fp}%
\special{pa 605 197}\special{pa 644 197}\special{fp}\special{pa 683 197}\special{pa 722 197}\special{fp}%
\special{pa 761 197}\special{pa 800 197}\special{fp}\special{pa 839 197}\special{pa 879 197}\special{fp}%
\special{pa 918 197}\special{pa 957 197}\special{fp}\special{pa 996 197}\special{pa 1035 197}\special{fp}%
\special{pa 1074 197}\special{pa 1113 197}\special{fp}\special{pa 1152 197}\special{pa 1191 197}\special{fp}%
\special{pa 1230 197}\special{pa 1269 197}\special{fp}\special{pa 1308 197}\special{pa 1347 197}\special{fp}%
\special{pa 1386 197}\special{pa 1425 197}\special{fp}\special{pa 1464 197}\special{pa 1503 197}\special{fp}%
\special{pa 1542 197}\special{pa 1581 197}\special{fp}\special{pa 1620 197}\special{pa 1659 197}\special{fp}%
\special{pa 1698 197}\special{pa 1737 197}\special{fp}\special{pa 1777 197}\special{pa 1816 197}\special{fp}%
\special{pa 1855 197}\special{pa 1894 197}\special{fp}\special{pa 1933 197}\special{pa 1972 197}\special{fp}%
\special{pa 2011 197}\special{pa 2050 197}\special{fp}\special{pa 2089 197}\special{pa 2128 197}\special{fp}%
\special{pa 2167 197}\special{pa 2206 197}\special{fp}\special{pa 2245 197}\special{pa 2284 197}\special{fp}%
\special{pa 2323 197}\special{pa 2362 197}\special{fp}%
%
}%
{%
\color[rgb]{0,0,0}%
\special{pa 197 2362}\special{pa 197 2323}\special{fp}\special{pa 197 2284}\special{pa 197 2245}\special{fp}%
\special{pa 197 2206}\special{pa 197 2167}\special{fp}\special{pa 197 2128}\special{pa 197 2089}\special{fp}%
\special{pa 197 2050}\special{pa 197 2011}\special{fp}\special{pa 197 1972}\special{pa 197 1933}\special{fp}%
\special{pa 197 1894}\special{pa 197 1855}\special{fp}\special{pa 197 1816}\special{pa 197 1777}\special{fp}%
\special{pa 197 1737}\special{pa 197 1698}\special{fp}\special{pa 197 1659}\special{pa 197 1620}\special{fp}%
\special{pa 197 1581}\special{pa 197 1542}\special{fp}\special{pa 197 1503}\special{pa 197 1464}\special{fp}%
\special{pa 197 1425}\special{pa 197 1386}\special{fp}\special{pa 197 1347}\special{pa 197 1308}\special{fp}%
\special{pa 197 1269}\special{pa 197 1230}\special{fp}\special{pa 197 1191}\special{pa 197 1152}\special{fp}%
\special{pa 197 1113}\special{pa 197 1074}\special{fp}\special{pa 197 1035}\special{pa 197 996}\special{fp}%
\special{pa 197 957}\special{pa 197 918}\special{fp}\special{pa 197 879}\special{pa 197 839}\special{fp}%
\special{pa 197 800}\special{pa 197 761}\special{fp}\special{pa 197 722}\special{pa 197 683}\special{fp}%
\special{pa 197 644}\special{pa 197 605}\special{fp}\special{pa 197 566}\special{pa 197 527}\special{fp}%
\special{pa 197 488}\special{pa 197 449}\special{fp}\special{pa 197 410}\special{pa 197 371}\special{fp}%
\special{pa 197 332}\special{pa 197 293}\special{fp}\special{pa 197 254}\special{pa 197 215}\special{fp}%
\special{pa 197 176}\special{pa 197 137}\special{fp}\special{pa 197 98}\special{pa 197 59}\special{fp}%
\special{pa 197 20}\special{pa 197 -20}\special{fp}\special{pa 197 -59}\special{pa 197 -98}\special{fp}%
\special{pa 197 -137}\special{pa 197 -176}\special{fp}\special{pa 197 -215}\special{pa 197 -254}\special{fp}%
\special{pa 197 -293}\special{pa 197 -332}\special{fp}\special{pa 197 -371}\special{pa 197 -410}\special{fp}%
\special{pa 197 -449}\special{pa 197 -488}\special{fp}\special{pa 197 -527}\special{pa 197 -566}\special{fp}%
\special{pa 197 -605}\special{pa 197 -644}\special{fp}\special{pa 197 -683}\special{pa 197 -722}\special{fp}%
\special{pa 197 -761}\special{pa 197 -800}\special{fp}\special{pa 197 -839}\special{pa 197 -879}\special{fp}%
\special{pa 197 -918}\special{pa 197 -957}\special{fp}\special{pa 197 -996}\special{pa 197 -1035}\special{fp}%
\special{pa 197 -1074}\special{pa 197 -1113}\special{fp}\special{pa 197 -1152}\special{pa 197 -1191}\special{fp}%
\special{pa 197 -1230}\special{pa 197 -1269}\special{fp}\special{pa 197 -1308}\special{pa 197 -1347}\special{fp}%
\special{pa 197 -1386}\special{pa 197 -1425}\special{fp}\special{pa 197 -1464}\special{pa 197 -1503}\special{fp}%
\special{pa 197 -1542}\special{pa 197 -1581}\special{fp}\special{pa 197 -1620}\special{pa 197 -1659}\special{fp}%
\special{pa 197 -1698}\special{pa 197 -1737}\special{fp}\special{pa 197 -1777}\special{pa 197 -1816}\special{fp}%
\special{pa 197 -1855}\special{pa 197 -1894}\special{fp}\special{pa 197 -1933}\special{pa 197 -1972}\special{fp}%
\special{pa 197 -2011}\special{pa 197 -2050}\special{fp}\special{pa 197 -2089}\special{pa 197 -2128}\special{fp}%
\special{pa 197 -2167}\special{pa 197 -2206}\special{fp}\special{pa 197 -2245}\special{pa 197 -2284}\special{fp}%
\special{pa 197 -2323}\special{pa 197 -2362}\special{fp}%
%
}%
{%
\color[rgb]{0,0,0}%
\special{pa -2362 -197}\special{pa -2323 -197}\special{fp}\special{pa -2284 -197}\special{pa -2245 -197}\special{fp}%
\special{pa -2206 -197}\special{pa -2167 -197}\special{fp}\special{pa -2128 -197}\special{pa -2089 -197}\special{fp}%
\special{pa -2050 -197}\special{pa -2011 -197}\special{fp}\special{pa -1972 -197}\special{pa -1933 -197}\special{fp}%
\special{pa -1894 -197}\special{pa -1855 -197}\special{fp}\special{pa -1816 -197}\special{pa -1777 -197}\special{fp}%
\special{pa -1737 -197}\special{pa -1698 -197}\special{fp}\special{pa -1659 -197}\special{pa -1620 -197}\special{fp}%
\special{pa -1581 -197}\special{pa -1542 -197}\special{fp}\special{pa -1503 -197}\special{pa -1464 -197}\special{fp}%
\special{pa -1425 -197}\special{pa -1386 -197}\special{fp}\special{pa -1347 -197}\special{pa -1308 -197}\special{fp}%
\special{pa -1269 -197}\special{pa -1230 -197}\special{fp}\special{pa -1191 -197}\special{pa -1152 -197}\special{fp}%
\special{pa -1113 -197}\special{pa -1074 -197}\special{fp}\special{pa -1035 -197}\special{pa -996 -197}\special{fp}%
\special{pa -957 -197}\special{pa -918 -197}\special{fp}\special{pa -879 -197}\special{pa -839 -197}\special{fp}%
\special{pa -800 -197}\special{pa -761 -197}\special{fp}\special{pa -722 -197}\special{pa -683 -197}\special{fp}%
\special{pa -644 -197}\special{pa -605 -197}\special{fp}\special{pa -566 -197}\special{pa -527 -197}\special{fp}%
\special{pa -488 -197}\special{pa -449 -197}\special{fp}\special{pa -410 -197}\special{pa -371 -197}\special{fp}%
\special{pa -332 -197}\special{pa -293 -197}\special{fp}\special{pa -254 -197}\special{pa -215 -197}\special{fp}%
\special{pa -176 -197}\special{pa -137 -197}\special{fp}\special{pa -98 -197}\special{pa -59 -197}\special{fp}%
\special{pa -20 -197}\special{pa 20 -197}\special{fp}\special{pa 59 -197}\special{pa 98 -197}\special{fp}%
\special{pa 137 -197}\special{pa 176 -197}\special{fp}\special{pa 215 -197}\special{pa 254 -197}\special{fp}%
\special{pa 293 -197}\special{pa 332 -197}\special{fp}\special{pa 371 -197}\special{pa 410 -197}\special{fp}%
\special{pa 449 -197}\special{pa 488 -197}\special{fp}\special{pa 527 -197}\special{pa 566 -197}\special{fp}%
\special{pa 605 -197}\special{pa 644 -197}\special{fp}\special{pa 683 -197}\special{pa 722 -197}\special{fp}%
\special{pa 761 -197}\special{pa 800 -197}\special{fp}\special{pa 839 -197}\special{pa 879 -197}\special{fp}%
\special{pa 918 -197}\special{pa 957 -197}\special{fp}\special{pa 996 -197}\special{pa 1035 -197}\special{fp}%
\special{pa 1074 -197}\special{pa 1113 -197}\special{fp}\special{pa 1152 -197}\special{pa 1191 -197}\special{fp}%
\special{pa 1230 -197}\special{pa 1269 -197}\special{fp}\special{pa 1308 -197}\special{pa 1347 -197}\special{fp}%
\special{pa 1386 -197}\special{pa 1425 -197}\special{fp}\special{pa 1464 -197}\special{pa 1503 -197}\special{fp}%
\special{pa 1542 -197}\special{pa 1581 -197}\special{fp}\special{pa 1620 -197}\special{pa 1659 -197}\special{fp}%
\special{pa 1698 -197}\special{pa 1737 -197}\special{fp}\special{pa 1777 -197}\special{pa 1816 -197}\special{fp}%
\special{pa 1855 -197}\special{pa 1894 -197}\special{fp}\special{pa 1933 -197}\special{pa 1972 -197}\special{fp}%
\special{pa 2011 -197}\special{pa 2050 -197}\special{fp}\special{pa 2089 -197}\special{pa 2128 -197}\special{fp}%
\special{pa 2167 -197}\special{pa 2206 -197}\special{fp}\special{pa 2245 -197}\special{pa 2284 -197}\special{fp}%
\special{pa 2323 -197}\special{pa 2362 -197}\special{fp}%
%
}%
{%
\color[rgb]{0,0,0}%
\special{pa 591 2362}\special{pa 591 2323}\special{fp}\special{pa 591 2284}\special{pa 591 2245}\special{fp}%
\special{pa 591 2206}\special{pa 591 2167}\special{fp}\special{pa 591 2128}\special{pa 591 2089}\special{fp}%
\special{pa 591 2050}\special{pa 591 2011}\special{fp}\special{pa 591 1972}\special{pa 591 1933}\special{fp}%
\special{pa 591 1894}\special{pa 591 1855}\special{fp}\special{pa 591 1816}\special{pa 591 1777}\special{fp}%
\special{pa 591 1737}\special{pa 591 1698}\special{fp}\special{pa 591 1659}\special{pa 591 1620}\special{fp}%
\special{pa 591 1581}\special{pa 591 1542}\special{fp}\special{pa 591 1503}\special{pa 591 1464}\special{fp}%
\special{pa 591 1425}\special{pa 591 1386}\special{fp}\special{pa 591 1347}\special{pa 591 1308}\special{fp}%
\special{pa 591 1269}\special{pa 591 1230}\special{fp}\special{pa 591 1191}\special{pa 591 1152}\special{fp}%
\special{pa 591 1113}\special{pa 591 1074}\special{fp}\special{pa 591 1035}\special{pa 591 996}\special{fp}%
\special{pa 591 957}\special{pa 591 918}\special{fp}\special{pa 591 879}\special{pa 591 839}\special{fp}%
\special{pa 591 800}\special{pa 591 761}\special{fp}\special{pa 591 722}\special{pa 591 683}\special{fp}%
\special{pa 591 644}\special{pa 591 605}\special{fp}\special{pa 591 566}\special{pa 591 527}\special{fp}%
\special{pa 591 488}\special{pa 591 449}\special{fp}\special{pa 591 410}\special{pa 591 371}\special{fp}%
\special{pa 591 332}\special{pa 591 293}\special{fp}\special{pa 591 254}\special{pa 591 215}\special{fp}%
\special{pa 591 176}\special{pa 591 137}\special{fp}\special{pa 591 98}\special{pa 591 59}\special{fp}%
\special{pa 591 20}\special{pa 591 -20}\special{fp}\special{pa 591 -59}\special{pa 591 -98}\special{fp}%
\special{pa 591 -137}\special{pa 591 -176}\special{fp}\special{pa 591 -215}\special{pa 591 -254}\special{fp}%
\special{pa 591 -293}\special{pa 591 -332}\special{fp}\special{pa 591 -371}\special{pa 591 -410}\special{fp}%
\special{pa 591 -449}\special{pa 591 -488}\special{fp}\special{pa 591 -527}\special{pa 591 -566}\special{fp}%
\special{pa 591 -605}\special{pa 591 -644}\special{fp}\special{pa 591 -683}\special{pa 591 -722}\special{fp}%
\special{pa 591 -761}\special{pa 591 -800}\special{fp}\special{pa 591 -839}\special{pa 591 -879}\special{fp}%
\special{pa 591 -918}\special{pa 591 -957}\special{fp}\special{pa 591 -996}\special{pa 591 -1035}\special{fp}%
\special{pa 591 -1074}\special{pa 591 -1113}\special{fp}\special{pa 591 -1152}\special{pa 591 -1191}\special{fp}%
\special{pa 591 -1230}\special{pa 591 -1269}\special{fp}\special{pa 591 -1308}\special{pa 591 -1347}\special{fp}%
\special{pa 591 -1386}\special{pa 591 -1425}\special{fp}\special{pa 591 -1464}\special{pa 591 -1503}\special{fp}%
\special{pa 591 -1542}\special{pa 591 -1581}\special{fp}\special{pa 591 -1620}\special{pa 591 -1659}\special{fp}%
\special{pa 591 -1698}\special{pa 591 -1737}\special{fp}\special{pa 591 -1777}\special{pa 591 -1816}\special{fp}%
\special{pa 591 -1855}\special{pa 591 -1894}\special{fp}\special{pa 591 -1933}\special{pa 591 -1972}\special{fp}%
\special{pa 591 -2011}\special{pa 591 -2050}\special{fp}\special{pa 591 -2089}\special{pa 591 -2128}\special{fp}%
\special{pa 591 -2167}\special{pa 591 -2206}\special{fp}\special{pa 591 -2245}\special{pa 591 -2284}\special{fp}%
\special{pa 591 -2323}\special{pa 591 -2362}\special{fp}%
%
}%
{%
\color[rgb]{0,0,0}%
\special{pa -2362 -591}\special{pa -2323 -591}\special{fp}\special{pa -2284 -591}\special{pa -2245 -591}\special{fp}%
\special{pa -2206 -591}\special{pa -2167 -591}\special{fp}\special{pa -2128 -591}\special{pa -2089 -591}\special{fp}%
\special{pa -2050 -591}\special{pa -2011 -591}\special{fp}\special{pa -1972 -591}\special{pa -1933 -591}\special{fp}%
\special{pa -1894 -591}\special{pa -1855 -591}\special{fp}\special{pa -1816 -591}\special{pa -1777 -591}\special{fp}%
\special{pa -1737 -591}\special{pa -1698 -591}\special{fp}\special{pa -1659 -591}\special{pa -1620 -591}\special{fp}%
\special{pa -1581 -591}\special{pa -1542 -591}\special{fp}\special{pa -1503 -591}\special{pa -1464 -591}\special{fp}%
\special{pa -1425 -591}\special{pa -1386 -591}\special{fp}\special{pa -1347 -591}\special{pa -1308 -591}\special{fp}%
\special{pa -1269 -591}\special{pa -1230 -591}\special{fp}\special{pa -1191 -591}\special{pa -1152 -591}\special{fp}%
\special{pa -1113 -591}\special{pa -1074 -591}\special{fp}\special{pa -1035 -591}\special{pa -996 -591}\special{fp}%
\special{pa -957 -591}\special{pa -918 -591}\special{fp}\special{pa -879 -591}\special{pa -839 -591}\special{fp}%
\special{pa -800 -591}\special{pa -761 -591}\special{fp}\special{pa -722 -591}\special{pa -683 -591}\special{fp}%
\special{pa -644 -591}\special{pa -605 -591}\special{fp}\special{pa -566 -591}\special{pa -527 -591}\special{fp}%
\special{pa -488 -591}\special{pa -449 -591}\special{fp}\special{pa -410 -591}\special{pa -371 -591}\special{fp}%
\special{pa -332 -591}\special{pa -293 -591}\special{fp}\special{pa -254 -591}\special{pa -215 -591}\special{fp}%
\special{pa -176 -591}\special{pa -137 -591}\special{fp}\special{pa -98 -591}\special{pa -59 -591}\special{fp}%
\special{pa -20 -591}\special{pa 20 -591}\special{fp}\special{pa 59 -591}\special{pa 98 -591}\special{fp}%
\special{pa 137 -591}\special{pa 176 -591}\special{fp}\special{pa 215 -591}\special{pa 254 -591}\special{fp}%
\special{pa 293 -591}\special{pa 332 -591}\special{fp}\special{pa 371 -591}\special{pa 410 -591}\special{fp}%
\special{pa 449 -591}\special{pa 488 -591}\special{fp}\special{pa 527 -591}\special{pa 566 -591}\special{fp}%
\special{pa 605 -591}\special{pa 644 -591}\special{fp}\special{pa 683 -591}\special{pa 722 -591}\special{fp}%
\special{pa 761 -591}\special{pa 800 -591}\special{fp}\special{pa 839 -591}\special{pa 879 -591}\special{fp}%
\special{pa 918 -591}\special{pa 957 -591}\special{fp}\special{pa 996 -591}\special{pa 1035 -591}\special{fp}%
\special{pa 1074 -591}\special{pa 1113 -591}\special{fp}\special{pa 1152 -591}\special{pa 1191 -591}\special{fp}%
\special{pa 1230 -591}\special{pa 1269 -591}\special{fp}\special{pa 1308 -591}\special{pa 1347 -591}\special{fp}%
\special{pa 1386 -591}\special{pa 1425 -591}\special{fp}\special{pa 1464 -591}\special{pa 1503 -591}\special{fp}%
\special{pa 1542 -591}\special{pa 1581 -591}\special{fp}\special{pa 1620 -591}\special{pa 1659 -591}\special{fp}%
\special{pa 1698 -591}\special{pa 1737 -591}\special{fp}\special{pa 1777 -591}\special{pa 1816 -591}\special{fp}%
\special{pa 1855 -591}\special{pa 1894 -591}\special{fp}\special{pa 1933 -591}\special{pa 1972 -591}\special{fp}%
\special{pa 2011 -591}\special{pa 2050 -591}\special{fp}\special{pa 2089 -591}\special{pa 2128 -591}\special{fp}%
\special{pa 2167 -591}\special{pa 2206 -591}\special{fp}\special{pa 2245 -591}\special{pa 2284 -591}\special{fp}%
\special{pa 2323 -591}\special{pa 2362 -591}\special{fp}%
%
}%
{%
\color[rgb]{0,0,0}%
\special{pa 984 2362}\special{pa 984 2323}\special{fp}\special{pa 984 2284}\special{pa 984 2245}\special{fp}%
\special{pa 984 2206}\special{pa 984 2167}\special{fp}\special{pa 984 2128}\special{pa 984 2089}\special{fp}%
\special{pa 984 2050}\special{pa 984 2011}\special{fp}\special{pa 984 1972}\special{pa 984 1933}\special{fp}%
\special{pa 984 1894}\special{pa 984 1855}\special{fp}\special{pa 984 1816}\special{pa 984 1777}\special{fp}%
\special{pa 984 1737}\special{pa 984 1698}\special{fp}\special{pa 984 1659}\special{pa 984 1620}\special{fp}%
\special{pa 984 1581}\special{pa 984 1542}\special{fp}\special{pa 984 1503}\special{pa 984 1464}\special{fp}%
\special{pa 984 1425}\special{pa 984 1386}\special{fp}\special{pa 984 1347}\special{pa 984 1308}\special{fp}%
\special{pa 984 1269}\special{pa 984 1230}\special{fp}\special{pa 984 1191}\special{pa 984 1152}\special{fp}%
\special{pa 984 1113}\special{pa 984 1074}\special{fp}\special{pa 984 1035}\special{pa 984 996}\special{fp}%
\special{pa 984 957}\special{pa 984 918}\special{fp}\special{pa 984 879}\special{pa 984 839}\special{fp}%
\special{pa 984 800}\special{pa 984 761}\special{fp}\special{pa 984 722}\special{pa 984 683}\special{fp}%
\special{pa 984 644}\special{pa 984 605}\special{fp}\special{pa 984 566}\special{pa 984 527}\special{fp}%
\special{pa 984 488}\special{pa 984 449}\special{fp}\special{pa 984 410}\special{pa 984 371}\special{fp}%
\special{pa 984 332}\special{pa 984 293}\special{fp}\special{pa 984 254}\special{pa 984 215}\special{fp}%
\special{pa 984 176}\special{pa 984 137}\special{fp}\special{pa 984 98}\special{pa 984 59}\special{fp}%
\special{pa 984 20}\special{pa 984 -20}\special{fp}\special{pa 984 -59}\special{pa 984 -98}\special{fp}%
\special{pa 984 -137}\special{pa 984 -176}\special{fp}\special{pa 984 -215}\special{pa 984 -254}\special{fp}%
\special{pa 984 -293}\special{pa 984 -332}\special{fp}\special{pa 984 -371}\special{pa 984 -410}\special{fp}%
\special{pa 984 -449}\special{pa 984 -488}\special{fp}\special{pa 984 -527}\special{pa 984 -566}\special{fp}%
\special{pa 984 -605}\special{pa 984 -644}\special{fp}\special{pa 984 -683}\special{pa 984 -722}\special{fp}%
\special{pa 984 -761}\special{pa 984 -800}\special{fp}\special{pa 984 -839}\special{pa 984 -879}\special{fp}%
\special{pa 984 -918}\special{pa 984 -957}\special{fp}\special{pa 984 -996}\special{pa 984 -1035}\special{fp}%
\special{pa 984 -1074}\special{pa 984 -1113}\special{fp}\special{pa 984 -1152}\special{pa 984 -1191}\special{fp}%
\special{pa 984 -1230}\special{pa 984 -1269}\special{fp}\special{pa 984 -1308}\special{pa 984 -1347}\special{fp}%
\special{pa 984 -1386}\special{pa 984 -1425}\special{fp}\special{pa 984 -1464}\special{pa 984 -1503}\special{fp}%
\special{pa 984 -1542}\special{pa 984 -1581}\special{fp}\special{pa 984 -1620}\special{pa 984 -1659}\special{fp}%
\special{pa 984 -1698}\special{pa 984 -1737}\special{fp}\special{pa 984 -1777}\special{pa 984 -1816}\special{fp}%
\special{pa 984 -1855}\special{pa 984 -1894}\special{fp}\special{pa 984 -1933}\special{pa 984 -1972}\special{fp}%
\special{pa 984 -2011}\special{pa 984 -2050}\special{fp}\special{pa 984 -2089}\special{pa 984 -2128}\special{fp}%
\special{pa 984 -2167}\special{pa 984 -2206}\special{fp}\special{pa 984 -2245}\special{pa 984 -2284}\special{fp}%
\special{pa 984 -2323}\special{pa 984 -2362}\special{fp}%
%
}%
{%
\color[rgb]{0,0,0}%
\special{pa -2362 -984}\special{pa -2323 -984}\special{fp}\special{pa -2284 -984}\special{pa -2245 -984}\special{fp}%
\special{pa -2206 -984}\special{pa -2167 -984}\special{fp}\special{pa -2128 -984}\special{pa -2089 -984}\special{fp}%
\special{pa -2050 -984}\special{pa -2011 -984}\special{fp}\special{pa -1972 -984}\special{pa -1933 -984}\special{fp}%
\special{pa -1894 -984}\special{pa -1855 -984}\special{fp}\special{pa -1816 -984}\special{pa -1777 -984}\special{fp}%
\special{pa -1737 -984}\special{pa -1698 -984}\special{fp}\special{pa -1659 -984}\special{pa -1620 -984}\special{fp}%
\special{pa -1581 -984}\special{pa -1542 -984}\special{fp}\special{pa -1503 -984}\special{pa -1464 -984}\special{fp}%
\special{pa -1425 -984}\special{pa -1386 -984}\special{fp}\special{pa -1347 -984}\special{pa -1308 -984}\special{fp}%
\special{pa -1269 -984}\special{pa -1230 -984}\special{fp}\special{pa -1191 -984}\special{pa -1152 -984}\special{fp}%
\special{pa -1113 -984}\special{pa -1074 -984}\special{fp}\special{pa -1035 -984}\special{pa -996 -984}\special{fp}%
\special{pa -957 -984}\special{pa -918 -984}\special{fp}\special{pa -879 -984}\special{pa -839 -984}\special{fp}%
\special{pa -800 -984}\special{pa -761 -984}\special{fp}\special{pa -722 -984}\special{pa -683 -984}\special{fp}%
\special{pa -644 -984}\special{pa -605 -984}\special{fp}\special{pa -566 -984}\special{pa -527 -984}\special{fp}%
\special{pa -488 -984}\special{pa -449 -984}\special{fp}\special{pa -410 -984}\special{pa -371 -984}\special{fp}%
\special{pa -332 -984}\special{pa -293 -984}\special{fp}\special{pa -254 -984}\special{pa -215 -984}\special{fp}%
\special{pa -176 -984}\special{pa -137 -984}\special{fp}\special{pa -98 -984}\special{pa -59 -984}\special{fp}%
\special{pa -20 -984}\special{pa 20 -984}\special{fp}\special{pa 59 -984}\special{pa 98 -984}\special{fp}%
\special{pa 137 -984}\special{pa 176 -984}\special{fp}\special{pa 215 -984}\special{pa 254 -984}\special{fp}%
\special{pa 293 -984}\special{pa 332 -984}\special{fp}\special{pa 371 -984}\special{pa 410 -984}\special{fp}%
\special{pa 449 -984}\special{pa 488 -984}\special{fp}\special{pa 527 -984}\special{pa 566 -984}\special{fp}%
\special{pa 605 -984}\special{pa 644 -984}\special{fp}\special{pa 683 -984}\special{pa 722 -984}\special{fp}%
\special{pa 761 -984}\special{pa 800 -984}\special{fp}\special{pa 839 -984}\special{pa 879 -984}\special{fp}%
\special{pa 918 -984}\special{pa 957 -984}\special{fp}\special{pa 996 -984}\special{pa 1035 -984}\special{fp}%
\special{pa 1074 -984}\special{pa 1113 -984}\special{fp}\special{pa 1152 -984}\special{pa 1191 -984}\special{fp}%
\special{pa 1230 -984}\special{pa 1269 -984}\special{fp}\special{pa 1308 -984}\special{pa 1347 -984}\special{fp}%
\special{pa 1386 -984}\special{pa 1425 -984}\special{fp}\special{pa 1464 -984}\special{pa 1503 -984}\special{fp}%
\special{pa 1542 -984}\special{pa 1581 -984}\special{fp}\special{pa 1620 -984}\special{pa 1659 -984}\special{fp}%
\special{pa 1698 -984}\special{pa 1737 -984}\special{fp}\special{pa 1777 -984}\special{pa 1816 -984}\special{fp}%
\special{pa 1855 -984}\special{pa 1894 -984}\special{fp}\special{pa 1933 -984}\special{pa 1972 -984}\special{fp}%
\special{pa 2011 -984}\special{pa 2050 -984}\special{fp}\special{pa 2089 -984}\special{pa 2128 -984}\special{fp}%
\special{pa 2167 -984}\special{pa 2206 -984}\special{fp}\special{pa 2245 -984}\special{pa 2284 -984}\special{fp}%
\special{pa 2323 -984}\special{pa 2362 -984}\special{fp}%
%
}%
{%
\color[rgb]{0,0,0}%
\special{pa 1378 2362}\special{pa 1378 2323}\special{fp}\special{pa 1378 2284}\special{pa 1378 2245}\special{fp}%
\special{pa 1378 2206}\special{pa 1378 2167}\special{fp}\special{pa 1378 2128}\special{pa 1378 2089}\special{fp}%
\special{pa 1378 2050}\special{pa 1378 2011}\special{fp}\special{pa 1378 1972}\special{pa 1378 1933}\special{fp}%
\special{pa 1378 1894}\special{pa 1378 1855}\special{fp}\special{pa 1378 1816}\special{pa 1378 1777}\special{fp}%
\special{pa 1378 1737}\special{pa 1378 1698}\special{fp}\special{pa 1378 1659}\special{pa 1378 1620}\special{fp}%
\special{pa 1378 1581}\special{pa 1378 1542}\special{fp}\special{pa 1378 1503}\special{pa 1378 1464}\special{fp}%
\special{pa 1378 1425}\special{pa 1378 1386}\special{fp}\special{pa 1378 1347}\special{pa 1378 1308}\special{fp}%
\special{pa 1378 1269}\special{pa 1378 1230}\special{fp}\special{pa 1378 1191}\special{pa 1378 1152}\special{fp}%
\special{pa 1378 1113}\special{pa 1378 1074}\special{fp}\special{pa 1378 1035}\special{pa 1378 996}\special{fp}%
\special{pa 1378 957}\special{pa 1378 918}\special{fp}\special{pa 1378 879}\special{pa 1378 839}\special{fp}%
\special{pa 1378 800}\special{pa 1378 761}\special{fp}\special{pa 1378 722}\special{pa 1378 683}\special{fp}%
\special{pa 1378 644}\special{pa 1378 605}\special{fp}\special{pa 1378 566}\special{pa 1378 527}\special{fp}%
\special{pa 1378 488}\special{pa 1378 449}\special{fp}\special{pa 1378 410}\special{pa 1378 371}\special{fp}%
\special{pa 1378 332}\special{pa 1378 293}\special{fp}\special{pa 1378 254}\special{pa 1378 215}\special{fp}%
\special{pa 1378 176}\special{pa 1378 137}\special{fp}\special{pa 1378 98}\special{pa 1378 59}\special{fp}%
\special{pa 1378 20}\special{pa 1378 -20}\special{fp}\special{pa 1378 -59}\special{pa 1378 -98}\special{fp}%
\special{pa 1378 -137}\special{pa 1378 -176}\special{fp}\special{pa 1378 -215}\special{pa 1378 -254}\special{fp}%
\special{pa 1378 -293}\special{pa 1378 -332}\special{fp}\special{pa 1378 -371}\special{pa 1378 -410}\special{fp}%
\special{pa 1378 -449}\special{pa 1378 -488}\special{fp}\special{pa 1378 -527}\special{pa 1378 -566}\special{fp}%
\special{pa 1378 -605}\special{pa 1378 -644}\special{fp}\special{pa 1378 -683}\special{pa 1378 -722}\special{fp}%
\special{pa 1378 -761}\special{pa 1378 -800}\special{fp}\special{pa 1378 -839}\special{pa 1378 -879}\special{fp}%
\special{pa 1378 -918}\special{pa 1378 -957}\special{fp}\special{pa 1378 -996}\special{pa 1378 -1035}\special{fp}%
\special{pa 1378 -1074}\special{pa 1378 -1113}\special{fp}\special{pa 1378 -1152}\special{pa 1378 -1191}\special{fp}%
\special{pa 1378 -1230}\special{pa 1378 -1269}\special{fp}\special{pa 1378 -1308}\special{pa 1378 -1347}\special{fp}%
\special{pa 1378 -1386}\special{pa 1378 -1425}\special{fp}\special{pa 1378 -1464}\special{pa 1378 -1503}\special{fp}%
\special{pa 1378 -1542}\special{pa 1378 -1581}\special{fp}\special{pa 1378 -1620}\special{pa 1378 -1659}\special{fp}%
\special{pa 1378 -1698}\special{pa 1378 -1737}\special{fp}\special{pa 1378 -1777}\special{pa 1378 -1816}\special{fp}%
\special{pa 1378 -1855}\special{pa 1378 -1894}\special{fp}\special{pa 1378 -1933}\special{pa 1378 -1972}\special{fp}%
\special{pa 1378 -2011}\special{pa 1378 -2050}\special{fp}\special{pa 1378 -2089}\special{pa 1378 -2128}\special{fp}%
\special{pa 1378 -2167}\special{pa 1378 -2206}\special{fp}\special{pa 1378 -2245}\special{pa 1378 -2284}\special{fp}%
\special{pa 1378 -2323}\special{pa 1378 -2362}\special{fp}%
%
}%
{%
\color[rgb]{0,0,0}%
\special{pa -2362 -1378}\special{pa -2323 -1378}\special{fp}\special{pa -2284 -1378}\special{pa -2245 -1378}\special{fp}%
\special{pa -2206 -1378}\special{pa -2167 -1378}\special{fp}\special{pa -2128 -1378}\special{pa -2089 -1378}\special{fp}%
\special{pa -2050 -1378}\special{pa -2011 -1378}\special{fp}\special{pa -1972 -1378}\special{pa -1933 -1378}\special{fp}%
\special{pa -1894 -1378}\special{pa -1855 -1378}\special{fp}\special{pa -1816 -1378}\special{pa -1777 -1378}\special{fp}%
\special{pa -1737 -1378}\special{pa -1698 -1378}\special{fp}\special{pa -1659 -1378}\special{pa -1620 -1378}\special{fp}%
\special{pa -1581 -1378}\special{pa -1542 -1378}\special{fp}\special{pa -1503 -1378}\special{pa -1464 -1378}\special{fp}%
\special{pa -1425 -1378}\special{pa -1386 -1378}\special{fp}\special{pa -1347 -1378}\special{pa -1308 -1378}\special{fp}%
\special{pa -1269 -1378}\special{pa -1230 -1378}\special{fp}\special{pa -1191 -1378}\special{pa -1152 -1378}\special{fp}%
\special{pa -1113 -1378}\special{pa -1074 -1378}\special{fp}\special{pa -1035 -1378}\special{pa -996 -1378}\special{fp}%
\special{pa -957 -1378}\special{pa -918 -1378}\special{fp}\special{pa -879 -1378}\special{pa -839 -1378}\special{fp}%
\special{pa -800 -1378}\special{pa -761 -1378}\special{fp}\special{pa -722 -1378}\special{pa -683 -1378}\special{fp}%
\special{pa -644 -1378}\special{pa -605 -1378}\special{fp}\special{pa -566 -1378}\special{pa -527 -1378}\special{fp}%
\special{pa -488 -1378}\special{pa -449 -1378}\special{fp}\special{pa -410 -1378}\special{pa -371 -1378}\special{fp}%
\special{pa -332 -1378}\special{pa -293 -1378}\special{fp}\special{pa -254 -1378}\special{pa -215 -1378}\special{fp}%
\special{pa -176 -1378}\special{pa -137 -1378}\special{fp}\special{pa -98 -1378}\special{pa -59 -1378}\special{fp}%
\special{pa -20 -1378}\special{pa 20 -1378}\special{fp}\special{pa 59 -1378}\special{pa 98 -1378}\special{fp}%
\special{pa 137 -1378}\special{pa 176 -1378}\special{fp}\special{pa 215 -1378}\special{pa 254 -1378}\special{fp}%
\special{pa 293 -1378}\special{pa 332 -1378}\special{fp}\special{pa 371 -1378}\special{pa 410 -1378}\special{fp}%
\special{pa 449 -1378}\special{pa 488 -1378}\special{fp}\special{pa 527 -1378}\special{pa 566 -1378}\special{fp}%
\special{pa 605 -1378}\special{pa 644 -1378}\special{fp}\special{pa 683 -1378}\special{pa 722 -1378}\special{fp}%
\special{pa 761 -1378}\special{pa 800 -1378}\special{fp}\special{pa 839 -1378}\special{pa 879 -1378}\special{fp}%
\special{pa 918 -1378}\special{pa 957 -1378}\special{fp}\special{pa 996 -1378}\special{pa 1035 -1378}\special{fp}%
\special{pa 1074 -1378}\special{pa 1113 -1378}\special{fp}\special{pa 1152 -1378}\special{pa 1191 -1378}\special{fp}%
\special{pa 1230 -1378}\special{pa 1269 -1378}\special{fp}\special{pa 1308 -1378}\special{pa 1347 -1378}\special{fp}%
\special{pa 1386 -1378}\special{pa 1425 -1378}\special{fp}\special{pa 1464 -1378}\special{pa 1503 -1378}\special{fp}%
\special{pa 1542 -1378}\special{pa 1581 -1378}\special{fp}\special{pa 1620 -1378}\special{pa 1659 -1378}\special{fp}%
\special{pa 1698 -1378}\special{pa 1737 -1378}\special{fp}\special{pa 1777 -1378}\special{pa 1816 -1378}\special{fp}%
\special{pa 1855 -1378}\special{pa 1894 -1378}\special{fp}\special{pa 1933 -1378}\special{pa 1972 -1378}\special{fp}%
\special{pa 2011 -1378}\special{pa 2050 -1378}\special{fp}\special{pa 2089 -1378}\special{pa 2128 -1378}\special{fp}%
\special{pa 2167 -1378}\special{pa 2206 -1378}\special{fp}\special{pa 2245 -1378}\special{pa 2284 -1378}\special{fp}%
\special{pa 2323 -1378}\special{pa 2362 -1378}\special{fp}%
%
}%
{%
\color[rgb]{0,0,0}%
\special{pa 1772 2362}\special{pa 1772 2323}\special{fp}\special{pa 1772 2284}\special{pa 1772 2245}\special{fp}%
\special{pa 1772 2206}\special{pa 1772 2167}\special{fp}\special{pa 1772 2128}\special{pa 1772 2089}\special{fp}%
\special{pa 1772 2050}\special{pa 1772 2011}\special{fp}\special{pa 1772 1972}\special{pa 1772 1933}\special{fp}%
\special{pa 1772 1894}\special{pa 1772 1855}\special{fp}\special{pa 1772 1816}\special{pa 1772 1777}\special{fp}%
\special{pa 1772 1737}\special{pa 1772 1698}\special{fp}\special{pa 1772 1659}\special{pa 1772 1620}\special{fp}%
\special{pa 1772 1581}\special{pa 1772 1542}\special{fp}\special{pa 1772 1503}\special{pa 1772 1464}\special{fp}%
\special{pa 1772 1425}\special{pa 1772 1386}\special{fp}\special{pa 1772 1347}\special{pa 1772 1308}\special{fp}%
\special{pa 1772 1269}\special{pa 1772 1230}\special{fp}\special{pa 1772 1191}\special{pa 1772 1152}\special{fp}%
\special{pa 1772 1113}\special{pa 1772 1074}\special{fp}\special{pa 1772 1035}\special{pa 1772 996}\special{fp}%
\special{pa 1772 957}\special{pa 1772 918}\special{fp}\special{pa 1772 879}\special{pa 1772 839}\special{fp}%
\special{pa 1772 800}\special{pa 1772 761}\special{fp}\special{pa 1772 722}\special{pa 1772 683}\special{fp}%
\special{pa 1772 644}\special{pa 1772 605}\special{fp}\special{pa 1772 566}\special{pa 1772 527}\special{fp}%
\special{pa 1772 488}\special{pa 1772 449}\special{fp}\special{pa 1772 410}\special{pa 1772 371}\special{fp}%
\special{pa 1772 332}\special{pa 1772 293}\special{fp}\special{pa 1772 254}\special{pa 1772 215}\special{fp}%
\special{pa 1772 176}\special{pa 1772 137}\special{fp}\special{pa 1772 98}\special{pa 1772 59}\special{fp}%
\special{pa 1772 20}\special{pa 1772 -20}\special{fp}\special{pa 1772 -59}\special{pa 1772 -98}\special{fp}%
\special{pa 1772 -137}\special{pa 1772 -176}\special{fp}\special{pa 1772 -215}\special{pa 1772 -254}\special{fp}%
\special{pa 1772 -293}\special{pa 1772 -332}\special{fp}\special{pa 1772 -371}\special{pa 1772 -410}\special{fp}%
\special{pa 1772 -449}\special{pa 1772 -488}\special{fp}\special{pa 1772 -527}\special{pa 1772 -566}\special{fp}%
\special{pa 1772 -605}\special{pa 1772 -644}\special{fp}\special{pa 1772 -683}\special{pa 1772 -722}\special{fp}%
\special{pa 1772 -761}\special{pa 1772 -800}\special{fp}\special{pa 1772 -839}\special{pa 1772 -879}\special{fp}%
\special{pa 1772 -918}\special{pa 1772 -957}\special{fp}\special{pa 1772 -996}\special{pa 1772 -1035}\special{fp}%
\special{pa 1772 -1074}\special{pa 1772 -1113}\special{fp}\special{pa 1772 -1152}\special{pa 1772 -1191}\special{fp}%
\special{pa 1772 -1230}\special{pa 1772 -1269}\special{fp}\special{pa 1772 -1308}\special{pa 1772 -1347}\special{fp}%
\special{pa 1772 -1386}\special{pa 1772 -1425}\special{fp}\special{pa 1772 -1464}\special{pa 1772 -1503}\special{fp}%
\special{pa 1772 -1542}\special{pa 1772 -1581}\special{fp}\special{pa 1772 -1620}\special{pa 1772 -1659}\special{fp}%
\special{pa 1772 -1698}\special{pa 1772 -1737}\special{fp}\special{pa 1772 -1777}\special{pa 1772 -1816}\special{fp}%
\special{pa 1772 -1855}\special{pa 1772 -1894}\special{fp}\special{pa 1772 -1933}\special{pa 1772 -1972}\special{fp}%
\special{pa 1772 -2011}\special{pa 1772 -2050}\special{fp}\special{pa 1772 -2089}\special{pa 1772 -2128}\special{fp}%
\special{pa 1772 -2167}\special{pa 1772 -2206}\special{fp}\special{pa 1772 -2245}\special{pa 1772 -2284}\special{fp}%
\special{pa 1772 -2323}\special{pa 1772 -2362}\special{fp}%
%
}%
{%
\color[rgb]{0,0,0}%
\special{pa -2362 -1772}\special{pa -2323 -1772}\special{fp}\special{pa -2284 -1772}\special{pa -2245 -1772}\special{fp}%
\special{pa -2206 -1772}\special{pa -2167 -1772}\special{fp}\special{pa -2128 -1772}\special{pa -2089 -1772}\special{fp}%
\special{pa -2050 -1772}\special{pa -2011 -1772}\special{fp}\special{pa -1972 -1772}\special{pa -1933 -1772}\special{fp}%
\special{pa -1894 -1772}\special{pa -1855 -1772}\special{fp}\special{pa -1816 -1772}\special{pa -1777 -1772}\special{fp}%
\special{pa -1737 -1772}\special{pa -1698 -1772}\special{fp}\special{pa -1659 -1772}\special{pa -1620 -1772}\special{fp}%
\special{pa -1581 -1772}\special{pa -1542 -1772}\special{fp}\special{pa -1503 -1772}\special{pa -1464 -1772}\special{fp}%
\special{pa -1425 -1772}\special{pa -1386 -1772}\special{fp}\special{pa -1347 -1772}\special{pa -1308 -1772}\special{fp}%
\special{pa -1269 -1772}\special{pa -1230 -1772}\special{fp}\special{pa -1191 -1772}\special{pa -1152 -1772}\special{fp}%
\special{pa -1113 -1772}\special{pa -1074 -1772}\special{fp}\special{pa -1035 -1772}\special{pa -996 -1772}\special{fp}%
\special{pa -957 -1772}\special{pa -918 -1772}\special{fp}\special{pa -879 -1772}\special{pa -839 -1772}\special{fp}%
\special{pa -800 -1772}\special{pa -761 -1772}\special{fp}\special{pa -722 -1772}\special{pa -683 -1772}\special{fp}%
\special{pa -644 -1772}\special{pa -605 -1772}\special{fp}\special{pa -566 -1772}\special{pa -527 -1772}\special{fp}%
\special{pa -488 -1772}\special{pa -449 -1772}\special{fp}\special{pa -410 -1772}\special{pa -371 -1772}\special{fp}%
\special{pa -332 -1772}\special{pa -293 -1772}\special{fp}\special{pa -254 -1772}\special{pa -215 -1772}\special{fp}%
\special{pa -176 -1772}\special{pa -137 -1772}\special{fp}\special{pa -98 -1772}\special{pa -59 -1772}\special{fp}%
\special{pa -20 -1772}\special{pa 20 -1772}\special{fp}\special{pa 59 -1772}\special{pa 98 -1772}\special{fp}%
\special{pa 137 -1772}\special{pa 176 -1772}\special{fp}\special{pa 215 -1772}\special{pa 254 -1772}\special{fp}%
\special{pa 293 -1772}\special{pa 332 -1772}\special{fp}\special{pa 371 -1772}\special{pa 410 -1772}\special{fp}%
\special{pa 449 -1772}\special{pa 488 -1772}\special{fp}\special{pa 527 -1772}\special{pa 566 -1772}\special{fp}%
\special{pa 605 -1772}\special{pa 644 -1772}\special{fp}\special{pa 683 -1772}\special{pa 722 -1772}\special{fp}%
\special{pa 761 -1772}\special{pa 800 -1772}\special{fp}\special{pa 839 -1772}\special{pa 879 -1772}\special{fp}%
\special{pa 918 -1772}\special{pa 957 -1772}\special{fp}\special{pa 996 -1772}\special{pa 1035 -1772}\special{fp}%
\special{pa 1074 -1772}\special{pa 1113 -1772}\special{fp}\special{pa 1152 -1772}\special{pa 1191 -1772}\special{fp}%
\special{pa 1230 -1772}\special{pa 1269 -1772}\special{fp}\special{pa 1308 -1772}\special{pa 1347 -1772}\special{fp}%
\special{pa 1386 -1772}\special{pa 1425 -1772}\special{fp}\special{pa 1464 -1772}\special{pa 1503 -1772}\special{fp}%
\special{pa 1542 -1772}\special{pa 1581 -1772}\special{fp}\special{pa 1620 -1772}\special{pa 1659 -1772}\special{fp}%
\special{pa 1698 -1772}\special{pa 1737 -1772}\special{fp}\special{pa 1777 -1772}\special{pa 1816 -1772}\special{fp}%
\special{pa 1855 -1772}\special{pa 1894 -1772}\special{fp}\special{pa 1933 -1772}\special{pa 1972 -1772}\special{fp}%
\special{pa 2011 -1772}\special{pa 2050 -1772}\special{fp}\special{pa 2089 -1772}\special{pa 2128 -1772}\special{fp}%
\special{pa 2167 -1772}\special{pa 2206 -1772}\special{fp}\special{pa 2245 -1772}\special{pa 2284 -1772}\special{fp}%
\special{pa 2323 -1772}\special{pa 2362 -1772}\special{fp}%
%
}%
{%
\color[rgb]{0,0,0}%
\special{pa 2165 2362}\special{pa 2165 2323}\special{fp}\special{pa 2165 2284}\special{pa 2165 2245}\special{fp}%
\special{pa 2165 2206}\special{pa 2165 2167}\special{fp}\special{pa 2165 2128}\special{pa 2165 2089}\special{fp}%
\special{pa 2165 2050}\special{pa 2165 2011}\special{fp}\special{pa 2165 1972}\special{pa 2165 1933}\special{fp}%
\special{pa 2165 1894}\special{pa 2165 1855}\special{fp}\special{pa 2165 1816}\special{pa 2165 1777}\special{fp}%
\special{pa 2165 1737}\special{pa 2165 1698}\special{fp}\special{pa 2165 1659}\special{pa 2165 1620}\special{fp}%
\special{pa 2165 1581}\special{pa 2165 1542}\special{fp}\special{pa 2165 1503}\special{pa 2165 1464}\special{fp}%
\special{pa 2165 1425}\special{pa 2165 1386}\special{fp}\special{pa 2165 1347}\special{pa 2165 1308}\special{fp}%
\special{pa 2165 1269}\special{pa 2165 1230}\special{fp}\special{pa 2165 1191}\special{pa 2165 1152}\special{fp}%
\special{pa 2165 1113}\special{pa 2165 1074}\special{fp}\special{pa 2165 1035}\special{pa 2165 996}\special{fp}%
\special{pa 2165 957}\special{pa 2165 918}\special{fp}\special{pa 2165 879}\special{pa 2165 839}\special{fp}%
\special{pa 2165 800}\special{pa 2165 761}\special{fp}\special{pa 2165 722}\special{pa 2165 683}\special{fp}%
\special{pa 2165 644}\special{pa 2165 605}\special{fp}\special{pa 2165 566}\special{pa 2165 527}\special{fp}%
\special{pa 2165 488}\special{pa 2165 449}\special{fp}\special{pa 2165 410}\special{pa 2165 371}\special{fp}%
\special{pa 2165 332}\special{pa 2165 293}\special{fp}\special{pa 2165 254}\special{pa 2165 215}\special{fp}%
\special{pa 2165 176}\special{pa 2165 137}\special{fp}\special{pa 2165 98}\special{pa 2165 59}\special{fp}%
\special{pa 2165 20}\special{pa 2165 -20}\special{fp}\special{pa 2165 -59}\special{pa 2165 -98}\special{fp}%
\special{pa 2165 -137}\special{pa 2165 -176}\special{fp}\special{pa 2165 -215}\special{pa 2165 -254}\special{fp}%
\special{pa 2165 -293}\special{pa 2165 -332}\special{fp}\special{pa 2165 -371}\special{pa 2165 -410}\special{fp}%
\special{pa 2165 -449}\special{pa 2165 -488}\special{fp}\special{pa 2165 -527}\special{pa 2165 -566}\special{fp}%
\special{pa 2165 -605}\special{pa 2165 -644}\special{fp}\special{pa 2165 -683}\special{pa 2165 -722}\special{fp}%
\special{pa 2165 -761}\special{pa 2165 -800}\special{fp}\special{pa 2165 -839}\special{pa 2165 -879}\special{fp}%
\special{pa 2165 -918}\special{pa 2165 -957}\special{fp}\special{pa 2165 -996}\special{pa 2165 -1035}\special{fp}%
\special{pa 2165 -1074}\special{pa 2165 -1113}\special{fp}\special{pa 2165 -1152}\special{pa 2165 -1191}\special{fp}%
\special{pa 2165 -1230}\special{pa 2165 -1269}\special{fp}\special{pa 2165 -1308}\special{pa 2165 -1347}\special{fp}%
\special{pa 2165 -1386}\special{pa 2165 -1425}\special{fp}\special{pa 2165 -1464}\special{pa 2165 -1503}\special{fp}%
\special{pa 2165 -1542}\special{pa 2165 -1581}\special{fp}\special{pa 2165 -1620}\special{pa 2165 -1659}\special{fp}%
\special{pa 2165 -1698}\special{pa 2165 -1737}\special{fp}\special{pa 2165 -1777}\special{pa 2165 -1816}\special{fp}%
\special{pa 2165 -1855}\special{pa 2165 -1894}\special{fp}\special{pa 2165 -1933}\special{pa 2165 -1972}\special{fp}%
\special{pa 2165 -2011}\special{pa 2165 -2050}\special{fp}\special{pa 2165 -2089}\special{pa 2165 -2128}\special{fp}%
\special{pa 2165 -2167}\special{pa 2165 -2206}\special{fp}\special{pa 2165 -2245}\special{pa 2165 -2284}\special{fp}%
\special{pa 2165 -2323}\special{pa 2165 -2362}\special{fp}%
%
}%
{%
\color[rgb]{0,0,0}%
\special{pa -2362 -2165}\special{pa -2323 -2165}\special{fp}\special{pa -2284 -2165}\special{pa -2245 -2165}\special{fp}%
\special{pa -2206 -2165}\special{pa -2167 -2165}\special{fp}\special{pa -2128 -2165}\special{pa -2089 -2165}\special{fp}%
\special{pa -2050 -2165}\special{pa -2011 -2165}\special{fp}\special{pa -1972 -2165}\special{pa -1933 -2165}\special{fp}%
\special{pa -1894 -2165}\special{pa -1855 -2165}\special{fp}\special{pa -1816 -2165}\special{pa -1777 -2165}\special{fp}%
\special{pa -1737 -2165}\special{pa -1698 -2165}\special{fp}\special{pa -1659 -2165}\special{pa -1620 -2165}\special{fp}%
\special{pa -1581 -2165}\special{pa -1542 -2165}\special{fp}\special{pa -1503 -2165}\special{pa -1464 -2165}\special{fp}%
\special{pa -1425 -2165}\special{pa -1386 -2165}\special{fp}\special{pa -1347 -2165}\special{pa -1308 -2165}\special{fp}%
\special{pa -1269 -2165}\special{pa -1230 -2165}\special{fp}\special{pa -1191 -2165}\special{pa -1152 -2165}\special{fp}%
\special{pa -1113 -2165}\special{pa -1074 -2165}\special{fp}\special{pa -1035 -2165}\special{pa -996 -2165}\special{fp}%
\special{pa -957 -2165}\special{pa -918 -2165}\special{fp}\special{pa -879 -2165}\special{pa -839 -2165}\special{fp}%
\special{pa -800 -2165}\special{pa -761 -2165}\special{fp}\special{pa -722 -2165}\special{pa -683 -2165}\special{fp}%
\special{pa -644 -2165}\special{pa -605 -2165}\special{fp}\special{pa -566 -2165}\special{pa -527 -2165}\special{fp}%
\special{pa -488 -2165}\special{pa -449 -2165}\special{fp}\special{pa -410 -2165}\special{pa -371 -2165}\special{fp}%
\special{pa -332 -2165}\special{pa -293 -2165}\special{fp}\special{pa -254 -2165}\special{pa -215 -2165}\special{fp}%
\special{pa -176 -2165}\special{pa -137 -2165}\special{fp}\special{pa -98 -2165}\special{pa -59 -2165}\special{fp}%
\special{pa -20 -2165}\special{pa 20 -2165}\special{fp}\special{pa 59 -2165}\special{pa 98 -2165}\special{fp}%
\special{pa 137 -2165}\special{pa 176 -2165}\special{fp}\special{pa 215 -2165}\special{pa 254 -2165}\special{fp}%
\special{pa 293 -2165}\special{pa 332 -2165}\special{fp}\special{pa 371 -2165}\special{pa 410 -2165}\special{fp}%
\special{pa 449 -2165}\special{pa 488 -2165}\special{fp}\special{pa 527 -2165}\special{pa 566 -2165}\special{fp}%
\special{pa 605 -2165}\special{pa 644 -2165}\special{fp}\special{pa 683 -2165}\special{pa 722 -2165}\special{fp}%
\special{pa 761 -2165}\special{pa 800 -2165}\special{fp}\special{pa 839 -2165}\special{pa 879 -2165}\special{fp}%
\special{pa 918 -2165}\special{pa 957 -2165}\special{fp}\special{pa 996 -2165}\special{pa 1035 -2165}\special{fp}%
\special{pa 1074 -2165}\special{pa 1113 -2165}\special{fp}\special{pa 1152 -2165}\special{pa 1191 -2165}\special{fp}%
\special{pa 1230 -2165}\special{pa 1269 -2165}\special{fp}\special{pa 1308 -2165}\special{pa 1347 -2165}\special{fp}%
\special{pa 1386 -2165}\special{pa 1425 -2165}\special{fp}\special{pa 1464 -2165}\special{pa 1503 -2165}\special{fp}%
\special{pa 1542 -2165}\special{pa 1581 -2165}\special{fp}\special{pa 1620 -2165}\special{pa 1659 -2165}\special{fp}%
\special{pa 1698 -2165}\special{pa 1737 -2165}\special{fp}\special{pa 1777 -2165}\special{pa 1816 -2165}\special{fp}%
\special{pa 1855 -2165}\special{pa 1894 -2165}\special{fp}\special{pa 1933 -2165}\special{pa 1972 -2165}\special{fp}%
\special{pa 2011 -2165}\special{pa 2050 -2165}\special{fp}\special{pa 2089 -2165}\special{pa 2128 -2165}\special{fp}%
\special{pa 2167 -2165}\special{pa 2206 -2165}\special{fp}\special{pa 2245 -2165}\special{pa 2284 -2165}\special{fp}%
\special{pa 2323 -2165}\special{pa 2362 -2165}\special{fp}%
%
}%
\special{pn 8}%
{%
\color[rgb]{0,0,0}%
\special{pa  2362   -20}\special{pa  2362    20}%
\special{fp}%
}%
{%
\color[rgb]{0,0,0}%
\settowidth{\Width}{$-6$}\setlength{\Width}{-0.5\Width}%
\settoheight{\Height}{$-6$}\settodepth{\Depth}{$-6$}\setlength{\Height}{-\Height}%
\put(-6.0000000,-0.1000000){\hspace*{\Width}\raisebox{\Height}{$-6$}}%
%
}%
{%
\color[rgb]{0,0,0}%
\special{pa    20 -2362}\special{pa   -20 -2362}%
\special{fp}%
}%
{%
\color[rgb]{0,0,0}%
\settowidth{\Width}{$-6$}\setlength{\Width}{-1\Width}%
\settoheight{\Height}{$-6$}\settodepth{\Depth}{$-6$}\setlength{\Height}{-0.5\Height}\setlength{\Depth}{0.5\Depth}\addtolength{\Height}{\Depth}%
\put(-0.1000000,-6.0000000){\hspace*{\Width}\raisebox{\Height}{$-6$}}%
%
}%
{%
\color[rgb]{0,0,0}%
\special{pa  2362   -20}\special{pa  2362    20}%
\special{fp}%
}%
{%
\color[rgb]{0,0,0}%
\settowidth{\Width}{$-5$}\setlength{\Width}{-0.5\Width}%
\settoheight{\Height}{$-5$}\settodepth{\Depth}{$-5$}\setlength{\Height}{-\Height}%
\put(-5.0000000,-0.1000000){\hspace*{\Width}\raisebox{\Height}{$-5$}}%
%
}%
{%
\color[rgb]{0,0,0}%
\special{pa    20 -2362}\special{pa   -20 -2362}%
\special{fp}%
}%
{%
\color[rgb]{0,0,0}%
\settowidth{\Width}{$-5$}\setlength{\Width}{-1\Width}%
\settoheight{\Height}{$-5$}\settodepth{\Depth}{$-5$}\setlength{\Height}{-0.5\Height}\setlength{\Depth}{0.5\Depth}\addtolength{\Height}{\Depth}%
\put(-0.1000000,-5.0000000){\hspace*{\Width}\raisebox{\Height}{$-5$}}%
%
}%
{%
\color[rgb]{0,0,0}%
\special{pa  2362   -20}\special{pa  2362    20}%
\special{fp}%
}%
{%
\color[rgb]{0,0,0}%
\settowidth{\Width}{$-4$}\setlength{\Width}{-0.5\Width}%
\settoheight{\Height}{$-4$}\settodepth{\Depth}{$-4$}\setlength{\Height}{-\Height}%
\put(-4.0000000,-0.1000000){\hspace*{\Width}\raisebox{\Height}{$-4$}}%
%
}%
{%
\color[rgb]{0,0,0}%
\special{pa    20 -2362}\special{pa   -20 -2362}%
\special{fp}%
}%
{%
\color[rgb]{0,0,0}%
\settowidth{\Width}{$-4$}\setlength{\Width}{-1\Width}%
\settoheight{\Height}{$-4$}\settodepth{\Depth}{$-4$}\setlength{\Height}{-0.5\Height}\setlength{\Depth}{0.5\Depth}\addtolength{\Height}{\Depth}%
\put(-0.1000000,-4.0000000){\hspace*{\Width}\raisebox{\Height}{$-4$}}%
%
}%
{%
\color[rgb]{0,0,0}%
\special{pa  2362   -20}\special{pa  2362    20}%
\special{fp}%
}%
{%
\color[rgb]{0,0,0}%
\settowidth{\Width}{$-3$}\setlength{\Width}{-0.5\Width}%
\settoheight{\Height}{$-3$}\settodepth{\Depth}{$-3$}\setlength{\Height}{-\Height}%
\put(-3.0000000,-0.1000000){\hspace*{\Width}\raisebox{\Height}{$-3$}}%
%
}%
{%
\color[rgb]{0,0,0}%
\special{pa    20 -2362}\special{pa   -20 -2362}%
\special{fp}%
}%
{%
\color[rgb]{0,0,0}%
\settowidth{\Width}{$-3$}\setlength{\Width}{-1\Width}%
\settoheight{\Height}{$-3$}\settodepth{\Depth}{$-3$}\setlength{\Height}{-0.5\Height}\setlength{\Depth}{0.5\Depth}\addtolength{\Height}{\Depth}%
\put(-0.1000000,-3.0000000){\hspace*{\Width}\raisebox{\Height}{$-3$}}%
%
}%
{%
\color[rgb]{0,0,0}%
\special{pa  2362   -20}\special{pa  2362    20}%
\special{fp}%
}%
{%
\color[rgb]{0,0,0}%
\settowidth{\Width}{$-2$}\setlength{\Width}{-0.5\Width}%
\settoheight{\Height}{$-2$}\settodepth{\Depth}{$-2$}\setlength{\Height}{-\Height}%
\put(-2.0000000,-0.1000000){\hspace*{\Width}\raisebox{\Height}{$-2$}}%
%
}%
{%
\color[rgb]{0,0,0}%
\special{pa    20 -2362}\special{pa   -20 -2362}%
\special{fp}%
}%
{%
\color[rgb]{0,0,0}%
\settowidth{\Width}{$-2$}\setlength{\Width}{-1\Width}%
\settoheight{\Height}{$-2$}\settodepth{\Depth}{$-2$}\setlength{\Height}{-0.5\Height}\setlength{\Depth}{0.5\Depth}\addtolength{\Height}{\Depth}%
\put(-0.1000000,-2.0000000){\hspace*{\Width}\raisebox{\Height}{$-2$}}%
%
}%
{%
\color[rgb]{0,0,0}%
\special{pa  2362   -20}\special{pa  2362    20}%
\special{fp}%
}%
{%
\color[rgb]{0,0,0}%
\settowidth{\Width}{$-1$}\setlength{\Width}{-0.5\Width}%
\settoheight{\Height}{$-1$}\settodepth{\Depth}{$-1$}\setlength{\Height}{-\Height}%
\put(-1.0000000,-0.1000000){\hspace*{\Width}\raisebox{\Height}{$-1$}}%
%
}%
{%
\color[rgb]{0,0,0}%
\special{pa    20 -2362}\special{pa   -20 -2362}%
\special{fp}%
}%
{%
\color[rgb]{0,0,0}%
\settowidth{\Width}{$-1$}\setlength{\Width}{-1\Width}%
\settoheight{\Height}{$-1$}\settodepth{\Depth}{$-1$}\setlength{\Height}{-0.5\Height}\setlength{\Depth}{0.5\Depth}\addtolength{\Height}{\Depth}%
\put(-0.1000000,-1.0000000){\hspace*{\Width}\raisebox{\Height}{$-1$}}%
%
}%
{%
\color[rgb]{0,0,0}%
\special{pa  2362   -20}\special{pa  2362    20}%
\special{fp}%
}%
{%
\color[rgb]{0,0,0}%
\settowidth{\Width}{$1$}\setlength{\Width}{-0.5\Width}%
\settoheight{\Height}{$1$}\settodepth{\Depth}{$1$}\setlength{\Height}{-\Height}%
\put(1.0000000,-0.1000000){\hspace*{\Width}\raisebox{\Height}{$1$}}%
%
}%
{%
\color[rgb]{0,0,0}%
\special{pa    20 -2362}\special{pa   -20 -2362}%
\special{fp}%
}%
{%
\color[rgb]{0,0,0}%
\settowidth{\Width}{$1$}\setlength{\Width}{-1\Width}%
\settoheight{\Height}{$1$}\settodepth{\Depth}{$1$}\setlength{\Height}{-0.5\Height}\setlength{\Depth}{0.5\Depth}\addtolength{\Height}{\Depth}%
\put(-0.1000000,1.0000000){\hspace*{\Width}\raisebox{\Height}{$1$}}%
%
}%
{%
\color[rgb]{0,0,0}%
\special{pa  2362   -20}\special{pa  2362    20}%
\special{fp}%
}%
{%
\color[rgb]{0,0,0}%
\settowidth{\Width}{$2$}\setlength{\Width}{-0.5\Width}%
\settoheight{\Height}{$2$}\settodepth{\Depth}{$2$}\setlength{\Height}{-\Height}%
\put(2.0000000,-0.1000000){\hspace*{\Width}\raisebox{\Height}{$2$}}%
%
}%
{%
\color[rgb]{0,0,0}%
\special{pa    20 -2362}\special{pa   -20 -2362}%
\special{fp}%
}%
{%
\color[rgb]{0,0,0}%
\settowidth{\Width}{$2$}\setlength{\Width}{-1\Width}%
\settoheight{\Height}{$2$}\settodepth{\Depth}{$2$}\setlength{\Height}{-0.5\Height}\setlength{\Depth}{0.5\Depth}\addtolength{\Height}{\Depth}%
\put(-0.1000000,2.0000000){\hspace*{\Width}\raisebox{\Height}{$2$}}%
%
}%
{%
\color[rgb]{0,0,0}%
\special{pa  2362   -20}\special{pa  2362    20}%
\special{fp}%
}%
{%
\color[rgb]{0,0,0}%
\settowidth{\Width}{$3$}\setlength{\Width}{-0.5\Width}%
\settoheight{\Height}{$3$}\settodepth{\Depth}{$3$}\setlength{\Height}{-\Height}%
\put(3.0000000,-0.1000000){\hspace*{\Width}\raisebox{\Height}{$3$}}%
%
}%
{%
\color[rgb]{0,0,0}%
\special{pa    20 -2362}\special{pa   -20 -2362}%
\special{fp}%
}%
{%
\color[rgb]{0,0,0}%
\settowidth{\Width}{$3$}\setlength{\Width}{-1\Width}%
\settoheight{\Height}{$3$}\settodepth{\Depth}{$3$}\setlength{\Height}{-0.5\Height}\setlength{\Depth}{0.5\Depth}\addtolength{\Height}{\Depth}%
\put(-0.1000000,3.0000000){\hspace*{\Width}\raisebox{\Height}{$3$}}%
%
}%
{%
\color[rgb]{0,0,0}%
\special{pa  2362   -20}\special{pa  2362    20}%
\special{fp}%
}%
{%
\color[rgb]{0,0,0}%
\settowidth{\Width}{$4$}\setlength{\Width}{-0.5\Width}%
\settoheight{\Height}{$4$}\settodepth{\Depth}{$4$}\setlength{\Height}{-\Height}%
\put(4.0000000,-0.1000000){\hspace*{\Width}\raisebox{\Height}{$4$}}%
%
}%
{%
\color[rgb]{0,0,0}%
\special{pa    20 -2362}\special{pa   -20 -2362}%
\special{fp}%
}%
{%
\color[rgb]{0,0,0}%
\settowidth{\Width}{$4$}\setlength{\Width}{-1\Width}%
\settoheight{\Height}{$4$}\settodepth{\Depth}{$4$}\setlength{\Height}{-0.5\Height}\setlength{\Depth}{0.5\Depth}\addtolength{\Height}{\Depth}%
\put(-0.1000000,4.0000000){\hspace*{\Width}\raisebox{\Height}{$4$}}%
%
}%
{%
\color[rgb]{0,0,0}%
\special{pa  2362   -20}\special{pa  2362    20}%
\special{fp}%
}%
{%
\color[rgb]{0,0,0}%
\settowidth{\Width}{$5$}\setlength{\Width}{-0.5\Width}%
\settoheight{\Height}{$5$}\settodepth{\Depth}{$5$}\setlength{\Height}{-\Height}%
\put(5.0000000,-0.1000000){\hspace*{\Width}\raisebox{\Height}{$5$}}%
%
}%
{%
\color[rgb]{0,0,0}%
\special{pa    20 -2362}\special{pa   -20 -2362}%
\special{fp}%
}%
{%
\color[rgb]{0,0,0}%
\settowidth{\Width}{$5$}\setlength{\Width}{-1\Width}%
\settoheight{\Height}{$5$}\settodepth{\Depth}{$5$}\setlength{\Height}{-0.5\Height}\setlength{\Depth}{0.5\Depth}\addtolength{\Height}{\Depth}%
\put(-0.1000000,5.0000000){\hspace*{\Width}\raisebox{\Height}{$5$}}%
%
}%
{%
\color[rgb]{0,0,0}%
\special{pa  2362   -20}\special{pa  2362    20}%
\special{fp}%
}%
{%
\color[rgb]{0,0,0}%
\settowidth{\Width}{$6$}\setlength{\Width}{-0.5\Width}%
\settoheight{\Height}{$6$}\settodepth{\Depth}{$6$}\setlength{\Height}{-\Height}%
\put(6.0000000,-0.1000000){\hspace*{\Width}\raisebox{\Height}{$6$}}%
%
}%
{%
\color[rgb]{0,0,0}%
\special{pa    20 -2362}\special{pa   -20 -2362}%
\special{fp}%
}%
{%
\color[rgb]{0,0,0}%
\settowidth{\Width}{$6$}\setlength{\Width}{-1\Width}%
\settoheight{\Height}{$6$}\settodepth{\Depth}{$6$}\setlength{\Height}{-0.5\Height}\setlength{\Depth}{0.5\Depth}\addtolength{\Height}{\Depth}%
\put(-0.1000000,6.0000000){\hspace*{\Width}\raisebox{\Height}{$6$}}%
%
{%
\color[cmyk]{0,1,1,0}%
\special{pa -1157 1969}\special{pa -1159 1960}\special{pa -1164 1953}\special{pa -1171 1947}%
\special{pa -1179 1945}\special{pa -1188 1946}\special{pa -1195 1950}\special{pa -1201 1956}%
\special{pa -1204 1964}\special{pa -1204 1973}\special{pa -1201 1981}\special{pa -1195 1987}%
\special{pa -1188 1991}\special{pa -1179 1992}\special{pa -1171 1990}\special{pa -1164 1984}%
\special{pa -1159 1977}\special{pa -1157 1969}\special{pa -1157 1969}\special{sh 1}\special{ip}%
}%
}%
{%
\color[cmyk]{0,1,1,0}%
\special{pa -1157  1969}\special{pa -1159  1960}\special{pa -1164  1953}\special{pa -1171  1947}%
\special{pa -1179  1945}\special{pa -1188  1946}\special{pa -1195  1950}\special{pa -1201  1956}%
\special{pa -1204  1964}\special{pa -1204  1973}\special{pa -1201  1981}\special{pa -1195  1987}%
\special{pa -1188  1991}\special{pa -1179  1992}\special{pa -1171  1990}\special{pa -1164  1984}%
\special{pa -1159  1977}\special{pa -1157  1969}%
\special{fp}%
{%
\color[cmyk]{0,1,1,0}%
\special{pa -764 1181}\special{pa -765 1173}\special{pa -770 1165}\special{pa -777 1160}%
\special{pa -785 1158}\special{pa -794 1158}\special{pa -802 1162}\special{pa -807 1169}%
\special{pa -811 1177}\special{pa -811 1185}\special{pa -807 1194}\special{pa -802 1200}%
\special{pa -794 1204}\special{pa -785 1205}\special{pa -777 1202}\special{pa -770 1197}%
\special{pa -765 1190}\special{pa -764 1181}\special{pa -764 1181}\special{sh 1}\special{ip}%
}%
}%
{%
\color[cmyk]{0,1,1,0}%
\special{pa  -764  1181}\special{pa  -765  1173}\special{pa  -770  1165}\special{pa  -777  1160}%
\special{pa  -785  1158}\special{pa  -794  1158}\special{pa  -802  1162}\special{pa  -807  1169}%
\special{pa  -811  1177}\special{pa  -811  1185}\special{pa  -807  1194}\special{pa  -802  1200}%
\special{pa  -794  1204}\special{pa  -785  1205}\special{pa  -777  1202}\special{pa  -770  1197}%
\special{pa  -765  1190}\special{pa  -764  1181}%
\special{fp}%
{%
\color[cmyk]{0,1,1,0}%
\special{pa -370 394}\special{pa -372 385}\special{pa -376 378}\special{pa -383 373}%
\special{pa -392 370}\special{pa -400 371}\special{pa -408 375}\special{pa -414 381}%
\special{pa -417 389}\special{pa -417 398}\special{pa -414 406}\special{pa -408 413}%
\special{pa -400 416}\special{pa -392 417}\special{pa -383 415}\special{pa -376 410}%
\special{pa -372 402}\special{pa -370 394}\special{pa -370 394}\special{sh 1}\special{ip}%
}%
}%
{%
\color[cmyk]{0,1,1,0}%
\special{pa  -370   394}\special{pa  -372   385}\special{pa  -376   378}\special{pa  -383   373}%
\special{pa  -392   370}\special{pa  -400   371}\special{pa  -408   375}\special{pa  -414   381}%
\special{pa  -417   389}\special{pa  -417   398}\special{pa  -414   406}\special{pa  -408   413}%
\special{pa  -400   416}\special{pa  -392   417}\special{pa  -383   415}\special{pa  -376   410}%
\special{pa  -372   402}\special{pa  -370   394}%
\special{fp}%
{%
\color[cmyk]{0,1,1,0}%
\special{pa 24 -394}\special{pa 22 -402}\special{pa 17 -410}\special{pa 11 -415}\special{pa 2 -417}%
\special{pa -6 -416}\special{pa -14 -413}\special{pa -20 -406}\special{pa -23 -398}%
\special{pa -23 -389}\special{pa -20 -381}\special{pa -14 -375}\special{pa -6 -371}%
\special{pa 2 -370}\special{pa 11 -373}\special{pa 17 -378}\special{pa 22 -385}\special{pa 24 -394}%
\special{pa 24 -394}\special{sh 1}\special{ip}%
}%
}%
{%
\color[cmyk]{0,1,1,0}%
\special{pa    24  -394}\special{pa    22  -402}\special{pa    17  -410}\special{pa    11  -415}%
\special{pa     2  -417}\special{pa    -6  -416}\special{pa   -14  -413}\special{pa   -20  -406}%
\special{pa   -23  -398}\special{pa   -23  -389}\special{pa   -20  -381}\special{pa   -14  -375}%
\special{pa    -6  -371}\special{pa     2  -370}\special{pa    11  -373}\special{pa    17  -378}%
\special{pa    22  -385}\special{pa    24  -394}%
\special{fp}%
{%
\color[cmyk]{0,1,1,0}%
\special{pa 417 -1181}\special{pa 416 -1190}\special{pa 411 -1197}\special{pa 404 -1202}%
\special{pa 396 -1205}\special{pa 387 -1204}\special{pa 379 -1200}\special{pa 374 -1194}%
\special{pa 370 -1185}\special{pa 370 -1177}\special{pa 374 -1169}\special{pa 379 -1162}%
\special{pa 387 -1158}\special{pa 396 -1158}\special{pa 404 -1160}\special{pa 411 -1165}%
\special{pa 416 -1173}\special{pa 417 -1181}\special{pa 417 -1181}\special{sh 1}\special{ip}%
}%
}%
{%
\color[cmyk]{0,1,1,0}%
\special{pa   417 -1181}\special{pa   416 -1190}\special{pa   411 -1197}\special{pa   404 -1202}%
\special{pa   396 -1205}\special{pa   387 -1204}\special{pa   379 -1200}\special{pa   374 -1194}%
\special{pa   370 -1185}\special{pa   370 -1177}\special{pa   374 -1169}\special{pa   379 -1162}%
\special{pa   387 -1158}\special{pa   396 -1158}\special{pa   404 -1160}\special{pa   411 -1165}%
\special{pa   416 -1173}\special{pa   417 -1181}%
\special{fp}%
{%
\color[cmyk]{0,1,1,0}%
\special{pa 811 -1969}\special{pa 809 -1977}\special{pa 805 -1984}\special{pa 798 -1990}%
\special{pa 790 -1992}\special{pa 781 -1991}\special{pa 773 -1987}\special{pa 767 -1981}%
\special{pa 764 -1973}\special{pa 764 -1964}\special{pa 767 -1956}\special{pa 773 -1950}%
\special{pa 781 -1946}\special{pa 790 -1945}\special{pa 798 -1947}\special{pa 805 -1953}%
\special{pa 809 -1960}\special{pa 811 -1969}\special{pa 811 -1969}\special{sh 1}\special{ip}%
}%
}%
{%
\color[cmyk]{0,1,1,0}%
\special{pa   811 -1969}\special{pa   809 -1977}\special{pa   805 -1984}\special{pa   798 -1990}%
\special{pa   790 -1992}\special{pa   781 -1991}\special{pa   773 -1987}\special{pa   767 -1981}%
\special{pa   764 -1973}\special{pa   764 -1964}\special{pa   767 -1956}\special{pa   773 -1950}%
\special{pa   781 -1946}\special{pa   790 -1945}\special{pa   798 -1947}\special{pa   805 -1953}%
\special{pa   809 -1960}\special{pa   811 -1969}%
\special{fp}%
}%
{%
\color[cmyk]{0,1,1,0}%
\special{pa -1378  2362}\special{pa   984 -2362}%
\special{fp}%
}%
\special{pa -2441    -0}\special{pa  2441    -0}%
\special{fp}%
\special{pa     0  2441}\special{pa     0 -2441}%
\special{fp}%
\settowidth{\Width}{$x$}\setlength{\Width}{0\Width}%
\settoheight{\Height}{$x$}\settodepth{\Depth}{$x$}\setlength{\Height}{-0.5\Height}\setlength{\Depth}{0.5\Depth}\addtolength{\Height}{\Depth}%
\put(6.2500000,0.0000000){\hspace*{\Width}\raisebox{\Height}{$x$}}%
%
\settowidth{\Width}{$y$}\setlength{\Width}{-0.5\Width}%
\settoheight{\Height}{$y$}\settodepth{\Depth}{$y$}\setlength{\Height}{\Depth}%
\put(0.0000000,6.2500000){\hspace*{\Width}\raisebox{\Height}{$y$}}%
%
\settowidth{\Width}{O}\setlength{\Width}{-1\Width}%
\settoheight{\Height}{O}\settodepth{\Depth}{O}\setlength{\Height}{-\Height}%
\put(-0.0500000,-0.0500000){\hspace*{\Width}\raisebox{\Height}{O}}%
%
\end{picture}}%}}
\putnotese{100}{30}{\color{blue}傾き}
\putnotese{100}{36}{\color{blue}$y$切片}
\end{layer}


\sameslide

\vspace*{18mm}

\slidepage
\down
例)$y=2x+1$

\begin{layer}{120}{0}
\putnotese{70}{-3}{\scalebox{0.6}{%%% /Users/takatoosetsuo/polytech23.git/101-0417/presen/fig/table1b.tex 
%%% Generator=presen23101.cdy 
{\unitlength=1cm%
\begin{picture}%
(9.6,1.2)(0,0)%
\linethickness{0.008in}%%
\Large\bf\boldmath%
\small%
\polyline(0,1.2)(0,0)%
%
\polyline(0.8,1.2)(0.8,0)%
%
\polyline(1.6,1.2)(1.6,0)%
%
\polyline(2.4,1.2)(2.4,0)%
%
\polyline(3.2,1.2)(3.2,0)%
%
\polyline(4,1.2)(4,0)%
%
\polyline(4.8,1.2)(4.8,0)%
%
\polyline(5.6,1.2)(5.6,0)%
%
\polyline(6.4,1.2)(6.4,0)%
%
\polyline(7.2,1.2)(7.2,0)%
%
\polyline(8,1.2)(8,0)%
%
\polyline(8.8,1.2)(8.8,0)%
%
\polyline(9.6,1.2)(9.6,0)%
%
\polyline(0,1.2)(9.6,1.2)%
%
\polyline(0,0.6)(9.6,0.6)%
%
\polyline(0,0)(9.6,0)%
%
\settowidth{\Width}{$x$}\setlength{\Width}{-0.5\Width}%
\settoheight{\Height}{$x$}\settodepth{\Depth}{$x$}\setlength{\Height}{-0.5\Height}\setlength{\Depth}{0.5\Depth}\addtolength{\Height}{\Depth}%
\put(  0.400,  0.900){\hspace*{\Width}\raisebox{\Height}{$x$}}%
%
\settowidth{\Width}{$-5$}\setlength{\Width}{-0.5\Width}%
\settoheight{\Height}{$-5$}\settodepth{\Depth}{$-5$}\setlength{\Height}{-0.5\Height}\setlength{\Depth}{0.5\Depth}\addtolength{\Height}{\Depth}%
\put(  1.200,  0.900){\hspace*{\Width}\raisebox{\Height}{$-5$}}%
%
\settowidth{\Width}{$-4$}\setlength{\Width}{-0.5\Width}%
\settoheight{\Height}{$-4$}\settodepth{\Depth}{$-4$}\setlength{\Height}{-0.5\Height}\setlength{\Depth}{0.5\Depth}\addtolength{\Height}{\Depth}%
\put(  2.000,  0.900){\hspace*{\Width}\raisebox{\Height}{$-4$}}%
%
\settowidth{\Width}{$-3$}\setlength{\Width}{-0.5\Width}%
\settoheight{\Height}{$-3$}\settodepth{\Depth}{$-3$}\setlength{\Height}{-0.5\Height}\setlength{\Depth}{0.5\Depth}\addtolength{\Height}{\Depth}%
\put(  2.800,  0.900){\hspace*{\Width}\raisebox{\Height}{$-3$}}%
%
\settowidth{\Width}{$-2$}\setlength{\Width}{-0.5\Width}%
\settoheight{\Height}{$-2$}\settodepth{\Depth}{$-2$}\setlength{\Height}{-0.5\Height}\setlength{\Depth}{0.5\Depth}\addtolength{\Height}{\Depth}%
\put(  3.600,  0.900){\hspace*{\Width}\raisebox{\Height}{$-2$}}%
%
\settowidth{\Width}{$-1$}\setlength{\Width}{-0.5\Width}%
\settoheight{\Height}{$-1$}\settodepth{\Depth}{$-1$}\setlength{\Height}{-0.5\Height}\setlength{\Depth}{0.5\Depth}\addtolength{\Height}{\Depth}%
\put(  4.400,  0.900){\hspace*{\Width}\raisebox{\Height}{$-1$}}%
%
\settowidth{\Width}{$0$}\setlength{\Width}{-0.5\Width}%
\settoheight{\Height}{$0$}\settodepth{\Depth}{$0$}\setlength{\Height}{-0.5\Height}\setlength{\Depth}{0.5\Depth}\addtolength{\Height}{\Depth}%
\put(  5.200,  0.900){\hspace*{\Width}\raisebox{\Height}{$0$}}%
%
\settowidth{\Width}{$1$}\setlength{\Width}{-0.5\Width}%
\settoheight{\Height}{$1$}\settodepth{\Depth}{$1$}\setlength{\Height}{-0.5\Height}\setlength{\Depth}{0.5\Depth}\addtolength{\Height}{\Depth}%
\put(  6.000,  0.900){\hspace*{\Width}\raisebox{\Height}{$1$}}%
%
\settowidth{\Width}{$2$}\setlength{\Width}{-0.5\Width}%
\settoheight{\Height}{$2$}\settodepth{\Depth}{$2$}\setlength{\Height}{-0.5\Height}\setlength{\Depth}{0.5\Depth}\addtolength{\Height}{\Depth}%
\put(  6.800,  0.900){\hspace*{\Width}\raisebox{\Height}{$2$}}%
%
\settowidth{\Width}{$3$}\setlength{\Width}{-0.5\Width}%
\settoheight{\Height}{$3$}\settodepth{\Depth}{$3$}\setlength{\Height}{-0.5\Height}\setlength{\Depth}{0.5\Depth}\addtolength{\Height}{\Depth}%
\put(  7.600,  0.900){\hspace*{\Width}\raisebox{\Height}{$3$}}%
%
\settowidth{\Width}{$4$}\setlength{\Width}{-0.5\Width}%
\settoheight{\Height}{$4$}\settodepth{\Depth}{$4$}\setlength{\Height}{-0.5\Height}\setlength{\Depth}{0.5\Depth}\addtolength{\Height}{\Depth}%
\put(  8.400,  0.900){\hspace*{\Width}\raisebox{\Height}{$4$}}%
%
\settowidth{\Width}{$5$}\setlength{\Width}{-0.5\Width}%
\settoheight{\Height}{$5$}\settodepth{\Depth}{$5$}\setlength{\Height}{-0.5\Height}\setlength{\Depth}{0.5\Depth}\addtolength{\Height}{\Depth}%
\put(  9.200,  0.900){\hspace*{\Width}\raisebox{\Height}{$5$}}%
%
\settowidth{\Width}{$y$}\setlength{\Width}{-0.5\Width}%
\settoheight{\Height}{$y$}\settodepth{\Depth}{$y$}\setlength{\Height}{-0.5\Height}\setlength{\Depth}{0.5\Depth}\addtolength{\Height}{\Depth}%
\put(  0.400,  0.300){\hspace*{\Width}\raisebox{\Height}{$y$}}%
%
\settowidth{\Width}{$-9$}\setlength{\Width}{-0.5\Width}%
\settoheight{\Height}{$-9$}\settodepth{\Depth}{$-9$}\setlength{\Height}{-0.5\Height}\setlength{\Depth}{0.5\Depth}\addtolength{\Height}{\Depth}%
\put(  1.200,  0.300){\hspace*{\Width}\raisebox{\Height}{$-9$}}%
%
\settowidth{\Width}{$-7$}\setlength{\Width}{-0.5\Width}%
\settoheight{\Height}{$-7$}\settodepth{\Depth}{$-7$}\setlength{\Height}{-0.5\Height}\setlength{\Depth}{0.5\Depth}\addtolength{\Height}{\Depth}%
\put(  2.000,  0.300){\hspace*{\Width}\raisebox{\Height}{$-7$}}%
%
\settowidth{\Width}{$-5$}\setlength{\Width}{-0.5\Width}%
\settoheight{\Height}{$-5$}\settodepth{\Depth}{$-5$}\setlength{\Height}{-0.5\Height}\setlength{\Depth}{0.5\Depth}\addtolength{\Height}{\Depth}%
\put(  2.800,  0.300){\hspace*{\Width}\raisebox{\Height}{$-5$}}%
%
\settowidth{\Width}{$-3$}\setlength{\Width}{-0.5\Width}%
\settoheight{\Height}{$-3$}\settodepth{\Depth}{$-3$}\setlength{\Height}{-0.5\Height}\setlength{\Depth}{0.5\Depth}\addtolength{\Height}{\Depth}%
\put(  3.600,  0.300){\hspace*{\Width}\raisebox{\Height}{$-3$}}%
%
\settowidth{\Width}{$-1$}\setlength{\Width}{-0.5\Width}%
\settoheight{\Height}{$-1$}\settodepth{\Depth}{$-1$}\setlength{\Height}{-0.5\Height}\setlength{\Depth}{0.5\Depth}\addtolength{\Height}{\Depth}%
\put(  4.400,  0.300){\hspace*{\Width}\raisebox{\Height}{$-1$}}%
%
\settowidth{\Width}{$1$}\setlength{\Width}{-0.5\Width}%
\settoheight{\Height}{$1$}\settodepth{\Depth}{$1$}\setlength{\Height}{-0.5\Height}\setlength{\Depth}{0.5\Depth}\addtolength{\Height}{\Depth}%
\put(  5.200,  0.300){\hspace*{\Width}\raisebox{\Height}{$1$}}%
%
\settowidth{\Width}{$3$}\setlength{\Width}{-0.5\Width}%
\settoheight{\Height}{$3$}\settodepth{\Depth}{$3$}\setlength{\Height}{-0.5\Height}\setlength{\Depth}{0.5\Depth}\addtolength{\Height}{\Depth}%
\put(  6.000,  0.300){\hspace*{\Width}\raisebox{\Height}{$3$}}%
%
\settowidth{\Width}{$5$}\setlength{\Width}{-0.5\Width}%
\settoheight{\Height}{$5$}\settodepth{\Depth}{$5$}\setlength{\Height}{-0.5\Height}\setlength{\Depth}{0.5\Depth}\addtolength{\Height}{\Depth}%
\put(  6.800,  0.300){\hspace*{\Width}\raisebox{\Height}{$5$}}%
%
\settowidth{\Width}{$7$}\setlength{\Width}{-0.5\Width}%
\settoheight{\Height}{$7$}\settodepth{\Depth}{$7$}\setlength{\Height}{-0.5\Height}\setlength{\Depth}{0.5\Depth}\addtolength{\Height}{\Depth}%
\put(  7.600,  0.300){\hspace*{\Width}\raisebox{\Height}{$7$}}%
%
\settowidth{\Width}{$9$}\setlength{\Width}{-0.5\Width}%
\settoheight{\Height}{$9$}\settodepth{\Depth}{$9$}\setlength{\Height}{-0.5\Height}\setlength{\Depth}{0.5\Depth}\addtolength{\Height}{\Depth}%
\put(  8.400,  0.300){\hspace*{\Width}\raisebox{\Height}{$9$}}%
%
\settowidth{\Width}{$11$}\setlength{\Width}{-0.5\Width}%
\settoheight{\Height}{$11$}\settodepth{\Depth}{$11$}\setlength{\Height}{-0.5\Height}\setlength{\Depth}{0.5\Depth}\addtolength{\Height}{\Depth}%
\put(  9.200,  0.300){\hspace*{\Width}\raisebox{\Height}{$11$}}%
%
\end{picture}}%}}
\putnotes{60}{6}{\scalebox{0.5}{%%% /polytech.git/n101/fig/graphpaper3.tex 
%%% Generator=presen0601.cdy 
{\unitlength=1cm%
\begin{picture}%
(12.4,12.4)(-6.2,-6.2)%
\special{pn 8}%
%
\Large\bf\boldmath%
\small%
{%
\color[rgb]{0,0,0}%
\special{pn 4}%
\special{pa -2362 -2362}\special{pa -2362  2362}%
\special{fp}%
\special{pn 8}%
}%
{%
\color[rgb]{0,0,0}%
\special{pn 4}%
\special{pa -1969 -2362}\special{pa -1969  2362}%
\special{fp}%
\special{pn 8}%
}%
{%
\color[rgb]{0,0,0}%
\special{pn 4}%
\special{pa -1575 -2362}\special{pa -1575  2362}%
\special{fp}%
\special{pn 8}%
}%
{%
\color[rgb]{0,0,0}%
\special{pn 4}%
\special{pa -1181 -2362}\special{pa -1181  2362}%
\special{fp}%
\special{pn 8}%
}%
{%
\color[rgb]{0,0,0}%
\special{pn 4}%
\special{pa  -787 -2362}\special{pa  -787  2362}%
\special{fp}%
\special{pn 8}%
}%
{%
\color[rgb]{0,0,0}%
\special{pn 4}%
\special{pa  -394 -2362}\special{pa  -394  2362}%
\special{fp}%
\special{pn 8}%
}%
{%
\color[rgb]{0,0,0}%
\special{pn 4}%
\special{pa     0 -2362}\special{pa     0  2362}%
\special{fp}%
\special{pn 8}%
}%
{%
\color[rgb]{0,0,0}%
\special{pn 4}%
\special{pa   394 -2362}\special{pa   394  2362}%
\special{fp}%
\special{pn 8}%
}%
{%
\color[rgb]{0,0,0}%
\special{pn 4}%
\special{pa   787 -2362}\special{pa   787  2362}%
\special{fp}%
\special{pn 8}%
}%
{%
\color[rgb]{0,0,0}%
\special{pn 4}%
\special{pa  1181 -2362}\special{pa  1181  2362}%
\special{fp}%
\special{pn 8}%
}%
{%
\color[rgb]{0,0,0}%
\special{pn 4}%
\special{pa  1575 -2362}\special{pa  1575  2362}%
\special{fp}%
\special{pn 8}%
}%
{%
\color[rgb]{0,0,0}%
\special{pn 4}%
\special{pa  1969 -2362}\special{pa  1969  2362}%
\special{fp}%
\special{pn 8}%
}%
{%
\color[rgb]{0,0,0}%
\special{pn 4}%
\special{pa  2362 -2362}\special{pa  2362  2362}%
\special{fp}%
\special{pn 8}%
}%
{%
\color[rgb]{0,0,0}%
\special{pn 4}%
\special{pa -2362 -2362}\special{pa  2362 -2362}%
\special{fp}%
\special{pn 8}%
}%
{%
\color[rgb]{0,0,0}%
\special{pn 4}%
\special{pa -2362 -1969}\special{pa  2362 -1969}%
\special{fp}%
\special{pn 8}%
}%
{%
\color[rgb]{0,0,0}%
\special{pn 4}%
\special{pa -2362 -1575}\special{pa  2362 -1575}%
\special{fp}%
\special{pn 8}%
}%
{%
\color[rgb]{0,0,0}%
\special{pn 4}%
\special{pa -2362 -1181}\special{pa  2362 -1181}%
\special{fp}%
\special{pn 8}%
}%
{%
\color[rgb]{0,0,0}%
\special{pn 4}%
\special{pa -2362  -787}\special{pa  2362  -787}%
\special{fp}%
\special{pn 8}%
}%
{%
\color[rgb]{0,0,0}%
\special{pn 4}%
\special{pa -2362  -394}\special{pa  2362  -394}%
\special{fp}%
\special{pn 8}%
}%
{%
\color[rgb]{0,0,0}%
\special{pn 4}%
\special{pa -2362    -0}\special{pa  2362    -0}%
\special{fp}%
\special{pn 8}%
}%
{%
\color[rgb]{0,0,0}%
\special{pn 4}%
\special{pa -2362   394}\special{pa  2362   394}%
\special{fp}%
\special{pn 8}%
}%
{%
\color[rgb]{0,0,0}%
\special{pn 4}%
\special{pa -2362   787}\special{pa  2362   787}%
\special{fp}%
\special{pn 8}%
}%
{%
\color[rgb]{0,0,0}%
\special{pn 4}%
\special{pa -2362  1181}\special{pa  2362  1181}%
\special{fp}%
\special{pn 8}%
}%
{%
\color[rgb]{0,0,0}%
\special{pn 4}%
\special{pa -2362  1575}\special{pa  2362  1575}%
\special{fp}%
\special{pn 8}%
}%
{%
\color[rgb]{0,0,0}%
\special{pn 4}%
\special{pa -2362  1969}\special{pa  2362  1969}%
\special{fp}%
\special{pn 8}%
}%
{%
\color[rgb]{0,0,0}%
\special{pn 4}%
\special{pa -2362  2362}\special{pa  2362  2362}%
\special{fp}%
\special{pn 8}%
}%
\special{pn 4}%
{%
\color[rgb]{0,0,0}%
\special{pa -2165 2362}\special{pa -2165 2323}\special{fp}\special{pa -2165 2284}\special{pa -2165 2245}\special{fp}%
\special{pa -2165 2206}\special{pa -2165 2167}\special{fp}\special{pa -2165 2128}\special{pa -2165 2089}\special{fp}%
\special{pa -2165 2050}\special{pa -2165 2011}\special{fp}\special{pa -2165 1972}\special{pa -2165 1933}\special{fp}%
\special{pa -2165 1894}\special{pa -2165 1855}\special{fp}\special{pa -2165 1816}\special{pa -2165 1777}\special{fp}%
\special{pa -2165 1737}\special{pa -2165 1698}\special{fp}\special{pa -2165 1659}\special{pa -2165 1620}\special{fp}%
\special{pa -2165 1581}\special{pa -2165 1542}\special{fp}\special{pa -2165 1503}\special{pa -2165 1464}\special{fp}%
\special{pa -2165 1425}\special{pa -2165 1386}\special{fp}\special{pa -2165 1347}\special{pa -2165 1308}\special{fp}%
\special{pa -2165 1269}\special{pa -2165 1230}\special{fp}\special{pa -2165 1191}\special{pa -2165 1152}\special{fp}%
\special{pa -2165 1113}\special{pa -2165 1074}\special{fp}\special{pa -2165 1035}\special{pa -2165 996}\special{fp}%
\special{pa -2165 957}\special{pa -2165 918}\special{fp}\special{pa -2165 879}\special{pa -2165 839}\special{fp}%
\special{pa -2165 800}\special{pa -2165 761}\special{fp}\special{pa -2165 722}\special{pa -2165 683}\special{fp}%
\special{pa -2165 644}\special{pa -2165 605}\special{fp}\special{pa -2165 566}\special{pa -2165 527}\special{fp}%
\special{pa -2165 488}\special{pa -2165 449}\special{fp}\special{pa -2165 410}\special{pa -2165 371}\special{fp}%
\special{pa -2165 332}\special{pa -2165 293}\special{fp}\special{pa -2165 254}\special{pa -2165 215}\special{fp}%
\special{pa -2165 176}\special{pa -2165 137}\special{fp}\special{pa -2165 98}\special{pa -2165 59}\special{fp}%
\special{pa -2165 20}\special{pa -2165 -20}\special{fp}\special{pa -2165 -59}\special{pa -2165 -98}\special{fp}%
\special{pa -2165 -137}\special{pa -2165 -176}\special{fp}\special{pa -2165 -215}\special{pa -2165 -254}\special{fp}%
\special{pa -2165 -293}\special{pa -2165 -332}\special{fp}\special{pa -2165 -371}\special{pa -2165 -410}\special{fp}%
\special{pa -2165 -449}\special{pa -2165 -488}\special{fp}\special{pa -2165 -527}\special{pa -2165 -566}\special{fp}%
\special{pa -2165 -605}\special{pa -2165 -644}\special{fp}\special{pa -2165 -683}\special{pa -2165 -722}\special{fp}%
\special{pa -2165 -761}\special{pa -2165 -800}\special{fp}\special{pa -2165 -839}\special{pa -2165 -879}\special{fp}%
\special{pa -2165 -918}\special{pa -2165 -957}\special{fp}\special{pa -2165 -996}\special{pa -2165 -1035}\special{fp}%
\special{pa -2165 -1074}\special{pa -2165 -1113}\special{fp}\special{pa -2165 -1152}\special{pa -2165 -1191}\special{fp}%
\special{pa -2165 -1230}\special{pa -2165 -1269}\special{fp}\special{pa -2165 -1308}\special{pa -2165 -1347}\special{fp}%
\special{pa -2165 -1386}\special{pa -2165 -1425}\special{fp}\special{pa -2165 -1464}\special{pa -2165 -1503}\special{fp}%
\special{pa -2165 -1542}\special{pa -2165 -1581}\special{fp}\special{pa -2165 -1620}\special{pa -2165 -1659}\special{fp}%
\special{pa -2165 -1698}\special{pa -2165 -1737}\special{fp}\special{pa -2165 -1777}\special{pa -2165 -1816}\special{fp}%
\special{pa -2165 -1855}\special{pa -2165 -1894}\special{fp}\special{pa -2165 -1933}\special{pa -2165 -1972}\special{fp}%
\special{pa -2165 -2011}\special{pa -2165 -2050}\special{fp}\special{pa -2165 -2089}\special{pa -2165 -2128}\special{fp}%
\special{pa -2165 -2167}\special{pa -2165 -2206}\special{fp}\special{pa -2165 -2245}\special{pa -2165 -2284}\special{fp}%
\special{pa -2165 -2323}\special{pa -2165 -2362}\special{fp}%
%
}%
{%
\color[rgb]{0,0,0}%
\special{pa -2362 2165}\special{pa -2323 2165}\special{fp}\special{pa -2284 2165}\special{pa -2245 2165}\special{fp}%
\special{pa -2206 2165}\special{pa -2167 2165}\special{fp}\special{pa -2128 2165}\special{pa -2089 2165}\special{fp}%
\special{pa -2050 2165}\special{pa -2011 2165}\special{fp}\special{pa -1972 2165}\special{pa -1933 2165}\special{fp}%
\special{pa -1894 2165}\special{pa -1855 2165}\special{fp}\special{pa -1816 2165}\special{pa -1777 2165}\special{fp}%
\special{pa -1737 2165}\special{pa -1698 2165}\special{fp}\special{pa -1659 2165}\special{pa -1620 2165}\special{fp}%
\special{pa -1581 2165}\special{pa -1542 2165}\special{fp}\special{pa -1503 2165}\special{pa -1464 2165}\special{fp}%
\special{pa -1425 2165}\special{pa -1386 2165}\special{fp}\special{pa -1347 2165}\special{pa -1308 2165}\special{fp}%
\special{pa -1269 2165}\special{pa -1230 2165}\special{fp}\special{pa -1191 2165}\special{pa -1152 2165}\special{fp}%
\special{pa -1113 2165}\special{pa -1074 2165}\special{fp}\special{pa -1035 2165}\special{pa -996 2165}\special{fp}%
\special{pa -957 2165}\special{pa -918 2165}\special{fp}\special{pa -879 2165}\special{pa -839 2165}\special{fp}%
\special{pa -800 2165}\special{pa -761 2165}\special{fp}\special{pa -722 2165}\special{pa -683 2165}\special{fp}%
\special{pa -644 2165}\special{pa -605 2165}\special{fp}\special{pa -566 2165}\special{pa -527 2165}\special{fp}%
\special{pa -488 2165}\special{pa -449 2165}\special{fp}\special{pa -410 2165}\special{pa -371 2165}\special{fp}%
\special{pa -332 2165}\special{pa -293 2165}\special{fp}\special{pa -254 2165}\special{pa -215 2165}\special{fp}%
\special{pa -176 2165}\special{pa -137 2165}\special{fp}\special{pa -98 2165}\special{pa -59 2165}\special{fp}%
\special{pa -20 2165}\special{pa 20 2165}\special{fp}\special{pa 59 2165}\special{pa 98 2165}\special{fp}%
\special{pa 137 2165}\special{pa 176 2165}\special{fp}\special{pa 215 2165}\special{pa 254 2165}\special{fp}%
\special{pa 293 2165}\special{pa 332 2165}\special{fp}\special{pa 371 2165}\special{pa 410 2165}\special{fp}%
\special{pa 449 2165}\special{pa 488 2165}\special{fp}\special{pa 527 2165}\special{pa 566 2165}\special{fp}%
\special{pa 605 2165}\special{pa 644 2165}\special{fp}\special{pa 683 2165}\special{pa 722 2165}\special{fp}%
\special{pa 761 2165}\special{pa 800 2165}\special{fp}\special{pa 839 2165}\special{pa 879 2165}\special{fp}%
\special{pa 918 2165}\special{pa 957 2165}\special{fp}\special{pa 996 2165}\special{pa 1035 2165}\special{fp}%
\special{pa 1074 2165}\special{pa 1113 2165}\special{fp}\special{pa 1152 2165}\special{pa 1191 2165}\special{fp}%
\special{pa 1230 2165}\special{pa 1269 2165}\special{fp}\special{pa 1308 2165}\special{pa 1347 2165}\special{fp}%
\special{pa 1386 2165}\special{pa 1425 2165}\special{fp}\special{pa 1464 2165}\special{pa 1503 2165}\special{fp}%
\special{pa 1542 2165}\special{pa 1581 2165}\special{fp}\special{pa 1620 2165}\special{pa 1659 2165}\special{fp}%
\special{pa 1698 2165}\special{pa 1737 2165}\special{fp}\special{pa 1777 2165}\special{pa 1816 2165}\special{fp}%
\special{pa 1855 2165}\special{pa 1894 2165}\special{fp}\special{pa 1933 2165}\special{pa 1972 2165}\special{fp}%
\special{pa 2011 2165}\special{pa 2050 2165}\special{fp}\special{pa 2089 2165}\special{pa 2128 2165}\special{fp}%
\special{pa 2167 2165}\special{pa 2206 2165}\special{fp}\special{pa 2245 2165}\special{pa 2284 2165}\special{fp}%
\special{pa 2323 2165}\special{pa 2362 2165}\special{fp}%
%
}%
{%
\color[rgb]{0,0,0}%
\special{pa -1772 2362}\special{pa -1772 2323}\special{fp}\special{pa -1772 2284}\special{pa -1772 2245}\special{fp}%
\special{pa -1772 2206}\special{pa -1772 2167}\special{fp}\special{pa -1772 2128}\special{pa -1772 2089}\special{fp}%
\special{pa -1772 2050}\special{pa -1772 2011}\special{fp}\special{pa -1772 1972}\special{pa -1772 1933}\special{fp}%
\special{pa -1772 1894}\special{pa -1772 1855}\special{fp}\special{pa -1772 1816}\special{pa -1772 1777}\special{fp}%
\special{pa -1772 1737}\special{pa -1772 1698}\special{fp}\special{pa -1772 1659}\special{pa -1772 1620}\special{fp}%
\special{pa -1772 1581}\special{pa -1772 1542}\special{fp}\special{pa -1772 1503}\special{pa -1772 1464}\special{fp}%
\special{pa -1772 1425}\special{pa -1772 1386}\special{fp}\special{pa -1772 1347}\special{pa -1772 1308}\special{fp}%
\special{pa -1772 1269}\special{pa -1772 1230}\special{fp}\special{pa -1772 1191}\special{pa -1772 1152}\special{fp}%
\special{pa -1772 1113}\special{pa -1772 1074}\special{fp}\special{pa -1772 1035}\special{pa -1772 996}\special{fp}%
\special{pa -1772 957}\special{pa -1772 918}\special{fp}\special{pa -1772 879}\special{pa -1772 839}\special{fp}%
\special{pa -1772 800}\special{pa -1772 761}\special{fp}\special{pa -1772 722}\special{pa -1772 683}\special{fp}%
\special{pa -1772 644}\special{pa -1772 605}\special{fp}\special{pa -1772 566}\special{pa -1772 527}\special{fp}%
\special{pa -1772 488}\special{pa -1772 449}\special{fp}\special{pa -1772 410}\special{pa -1772 371}\special{fp}%
\special{pa -1772 332}\special{pa -1772 293}\special{fp}\special{pa -1772 254}\special{pa -1772 215}\special{fp}%
\special{pa -1772 176}\special{pa -1772 137}\special{fp}\special{pa -1772 98}\special{pa -1772 59}\special{fp}%
\special{pa -1772 20}\special{pa -1772 -20}\special{fp}\special{pa -1772 -59}\special{pa -1772 -98}\special{fp}%
\special{pa -1772 -137}\special{pa -1772 -176}\special{fp}\special{pa -1772 -215}\special{pa -1772 -254}\special{fp}%
\special{pa -1772 -293}\special{pa -1772 -332}\special{fp}\special{pa -1772 -371}\special{pa -1772 -410}\special{fp}%
\special{pa -1772 -449}\special{pa -1772 -488}\special{fp}\special{pa -1772 -527}\special{pa -1772 -566}\special{fp}%
\special{pa -1772 -605}\special{pa -1772 -644}\special{fp}\special{pa -1772 -683}\special{pa -1772 -722}\special{fp}%
\special{pa -1772 -761}\special{pa -1772 -800}\special{fp}\special{pa -1772 -839}\special{pa -1772 -879}\special{fp}%
\special{pa -1772 -918}\special{pa -1772 -957}\special{fp}\special{pa -1772 -996}\special{pa -1772 -1035}\special{fp}%
\special{pa -1772 -1074}\special{pa -1772 -1113}\special{fp}\special{pa -1772 -1152}\special{pa -1772 -1191}\special{fp}%
\special{pa -1772 -1230}\special{pa -1772 -1269}\special{fp}\special{pa -1772 -1308}\special{pa -1772 -1347}\special{fp}%
\special{pa -1772 -1386}\special{pa -1772 -1425}\special{fp}\special{pa -1772 -1464}\special{pa -1772 -1503}\special{fp}%
\special{pa -1772 -1542}\special{pa -1772 -1581}\special{fp}\special{pa -1772 -1620}\special{pa -1772 -1659}\special{fp}%
\special{pa -1772 -1698}\special{pa -1772 -1737}\special{fp}\special{pa -1772 -1777}\special{pa -1772 -1816}\special{fp}%
\special{pa -1772 -1855}\special{pa -1772 -1894}\special{fp}\special{pa -1772 -1933}\special{pa -1772 -1972}\special{fp}%
\special{pa -1772 -2011}\special{pa -1772 -2050}\special{fp}\special{pa -1772 -2089}\special{pa -1772 -2128}\special{fp}%
\special{pa -1772 -2167}\special{pa -1772 -2206}\special{fp}\special{pa -1772 -2245}\special{pa -1772 -2284}\special{fp}%
\special{pa -1772 -2323}\special{pa -1772 -2362}\special{fp}%
%
}%
{%
\color[rgb]{0,0,0}%
\special{pa -2362 1772}\special{pa -2323 1772}\special{fp}\special{pa -2284 1772}\special{pa -2245 1772}\special{fp}%
\special{pa -2206 1772}\special{pa -2167 1772}\special{fp}\special{pa -2128 1772}\special{pa -2089 1772}\special{fp}%
\special{pa -2050 1772}\special{pa -2011 1772}\special{fp}\special{pa -1972 1772}\special{pa -1933 1772}\special{fp}%
\special{pa -1894 1772}\special{pa -1855 1772}\special{fp}\special{pa -1816 1772}\special{pa -1777 1772}\special{fp}%
\special{pa -1737 1772}\special{pa -1698 1772}\special{fp}\special{pa -1659 1772}\special{pa -1620 1772}\special{fp}%
\special{pa -1581 1772}\special{pa -1542 1772}\special{fp}\special{pa -1503 1772}\special{pa -1464 1772}\special{fp}%
\special{pa -1425 1772}\special{pa -1386 1772}\special{fp}\special{pa -1347 1772}\special{pa -1308 1772}\special{fp}%
\special{pa -1269 1772}\special{pa -1230 1772}\special{fp}\special{pa -1191 1772}\special{pa -1152 1772}\special{fp}%
\special{pa -1113 1772}\special{pa -1074 1772}\special{fp}\special{pa -1035 1772}\special{pa -996 1772}\special{fp}%
\special{pa -957 1772}\special{pa -918 1772}\special{fp}\special{pa -879 1772}\special{pa -839 1772}\special{fp}%
\special{pa -800 1772}\special{pa -761 1772}\special{fp}\special{pa -722 1772}\special{pa -683 1772}\special{fp}%
\special{pa -644 1772}\special{pa -605 1772}\special{fp}\special{pa -566 1772}\special{pa -527 1772}\special{fp}%
\special{pa -488 1772}\special{pa -449 1772}\special{fp}\special{pa -410 1772}\special{pa -371 1772}\special{fp}%
\special{pa -332 1772}\special{pa -293 1772}\special{fp}\special{pa -254 1772}\special{pa -215 1772}\special{fp}%
\special{pa -176 1772}\special{pa -137 1772}\special{fp}\special{pa -98 1772}\special{pa -59 1772}\special{fp}%
\special{pa -20 1772}\special{pa 20 1772}\special{fp}\special{pa 59 1772}\special{pa 98 1772}\special{fp}%
\special{pa 137 1772}\special{pa 176 1772}\special{fp}\special{pa 215 1772}\special{pa 254 1772}\special{fp}%
\special{pa 293 1772}\special{pa 332 1772}\special{fp}\special{pa 371 1772}\special{pa 410 1772}\special{fp}%
\special{pa 449 1772}\special{pa 488 1772}\special{fp}\special{pa 527 1772}\special{pa 566 1772}\special{fp}%
\special{pa 605 1772}\special{pa 644 1772}\special{fp}\special{pa 683 1772}\special{pa 722 1772}\special{fp}%
\special{pa 761 1772}\special{pa 800 1772}\special{fp}\special{pa 839 1772}\special{pa 879 1772}\special{fp}%
\special{pa 918 1772}\special{pa 957 1772}\special{fp}\special{pa 996 1772}\special{pa 1035 1772}\special{fp}%
\special{pa 1074 1772}\special{pa 1113 1772}\special{fp}\special{pa 1152 1772}\special{pa 1191 1772}\special{fp}%
\special{pa 1230 1772}\special{pa 1269 1772}\special{fp}\special{pa 1308 1772}\special{pa 1347 1772}\special{fp}%
\special{pa 1386 1772}\special{pa 1425 1772}\special{fp}\special{pa 1464 1772}\special{pa 1503 1772}\special{fp}%
\special{pa 1542 1772}\special{pa 1581 1772}\special{fp}\special{pa 1620 1772}\special{pa 1659 1772}\special{fp}%
\special{pa 1698 1772}\special{pa 1737 1772}\special{fp}\special{pa 1777 1772}\special{pa 1816 1772}\special{fp}%
\special{pa 1855 1772}\special{pa 1894 1772}\special{fp}\special{pa 1933 1772}\special{pa 1972 1772}\special{fp}%
\special{pa 2011 1772}\special{pa 2050 1772}\special{fp}\special{pa 2089 1772}\special{pa 2128 1772}\special{fp}%
\special{pa 2167 1772}\special{pa 2206 1772}\special{fp}\special{pa 2245 1772}\special{pa 2284 1772}\special{fp}%
\special{pa 2323 1772}\special{pa 2362 1772}\special{fp}%
%
}%
{%
\color[rgb]{0,0,0}%
\special{pa -1378 2362}\special{pa -1378 2323}\special{fp}\special{pa -1378 2284}\special{pa -1378 2245}\special{fp}%
\special{pa -1378 2206}\special{pa -1378 2167}\special{fp}\special{pa -1378 2128}\special{pa -1378 2089}\special{fp}%
\special{pa -1378 2050}\special{pa -1378 2011}\special{fp}\special{pa -1378 1972}\special{pa -1378 1933}\special{fp}%
\special{pa -1378 1894}\special{pa -1378 1855}\special{fp}\special{pa -1378 1816}\special{pa -1378 1777}\special{fp}%
\special{pa -1378 1737}\special{pa -1378 1698}\special{fp}\special{pa -1378 1659}\special{pa -1378 1620}\special{fp}%
\special{pa -1378 1581}\special{pa -1378 1542}\special{fp}\special{pa -1378 1503}\special{pa -1378 1464}\special{fp}%
\special{pa -1378 1425}\special{pa -1378 1386}\special{fp}\special{pa -1378 1347}\special{pa -1378 1308}\special{fp}%
\special{pa -1378 1269}\special{pa -1378 1230}\special{fp}\special{pa -1378 1191}\special{pa -1378 1152}\special{fp}%
\special{pa -1378 1113}\special{pa -1378 1074}\special{fp}\special{pa -1378 1035}\special{pa -1378 996}\special{fp}%
\special{pa -1378 957}\special{pa -1378 918}\special{fp}\special{pa -1378 879}\special{pa -1378 839}\special{fp}%
\special{pa -1378 800}\special{pa -1378 761}\special{fp}\special{pa -1378 722}\special{pa -1378 683}\special{fp}%
\special{pa -1378 644}\special{pa -1378 605}\special{fp}\special{pa -1378 566}\special{pa -1378 527}\special{fp}%
\special{pa -1378 488}\special{pa -1378 449}\special{fp}\special{pa -1378 410}\special{pa -1378 371}\special{fp}%
\special{pa -1378 332}\special{pa -1378 293}\special{fp}\special{pa -1378 254}\special{pa -1378 215}\special{fp}%
\special{pa -1378 176}\special{pa -1378 137}\special{fp}\special{pa -1378 98}\special{pa -1378 59}\special{fp}%
\special{pa -1378 20}\special{pa -1378 -20}\special{fp}\special{pa -1378 -59}\special{pa -1378 -98}\special{fp}%
\special{pa -1378 -137}\special{pa -1378 -176}\special{fp}\special{pa -1378 -215}\special{pa -1378 -254}\special{fp}%
\special{pa -1378 -293}\special{pa -1378 -332}\special{fp}\special{pa -1378 -371}\special{pa -1378 -410}\special{fp}%
\special{pa -1378 -449}\special{pa -1378 -488}\special{fp}\special{pa -1378 -527}\special{pa -1378 -566}\special{fp}%
\special{pa -1378 -605}\special{pa -1378 -644}\special{fp}\special{pa -1378 -683}\special{pa -1378 -722}\special{fp}%
\special{pa -1378 -761}\special{pa -1378 -800}\special{fp}\special{pa -1378 -839}\special{pa -1378 -879}\special{fp}%
\special{pa -1378 -918}\special{pa -1378 -957}\special{fp}\special{pa -1378 -996}\special{pa -1378 -1035}\special{fp}%
\special{pa -1378 -1074}\special{pa -1378 -1113}\special{fp}\special{pa -1378 -1152}\special{pa -1378 -1191}\special{fp}%
\special{pa -1378 -1230}\special{pa -1378 -1269}\special{fp}\special{pa -1378 -1308}\special{pa -1378 -1347}\special{fp}%
\special{pa -1378 -1386}\special{pa -1378 -1425}\special{fp}\special{pa -1378 -1464}\special{pa -1378 -1503}\special{fp}%
\special{pa -1378 -1542}\special{pa -1378 -1581}\special{fp}\special{pa -1378 -1620}\special{pa -1378 -1659}\special{fp}%
\special{pa -1378 -1698}\special{pa -1378 -1737}\special{fp}\special{pa -1378 -1777}\special{pa -1378 -1816}\special{fp}%
\special{pa -1378 -1855}\special{pa -1378 -1894}\special{fp}\special{pa -1378 -1933}\special{pa -1378 -1972}\special{fp}%
\special{pa -1378 -2011}\special{pa -1378 -2050}\special{fp}\special{pa -1378 -2089}\special{pa -1378 -2128}\special{fp}%
\special{pa -1378 -2167}\special{pa -1378 -2206}\special{fp}\special{pa -1378 -2245}\special{pa -1378 -2284}\special{fp}%
\special{pa -1378 -2323}\special{pa -1378 -2362}\special{fp}%
%
}%
{%
\color[rgb]{0,0,0}%
\special{pa -2362 1378}\special{pa -2323 1378}\special{fp}\special{pa -2284 1378}\special{pa -2245 1378}\special{fp}%
\special{pa -2206 1378}\special{pa -2167 1378}\special{fp}\special{pa -2128 1378}\special{pa -2089 1378}\special{fp}%
\special{pa -2050 1378}\special{pa -2011 1378}\special{fp}\special{pa -1972 1378}\special{pa -1933 1378}\special{fp}%
\special{pa -1894 1378}\special{pa -1855 1378}\special{fp}\special{pa -1816 1378}\special{pa -1777 1378}\special{fp}%
\special{pa -1737 1378}\special{pa -1698 1378}\special{fp}\special{pa -1659 1378}\special{pa -1620 1378}\special{fp}%
\special{pa -1581 1378}\special{pa -1542 1378}\special{fp}\special{pa -1503 1378}\special{pa -1464 1378}\special{fp}%
\special{pa -1425 1378}\special{pa -1386 1378}\special{fp}\special{pa -1347 1378}\special{pa -1308 1378}\special{fp}%
\special{pa -1269 1378}\special{pa -1230 1378}\special{fp}\special{pa -1191 1378}\special{pa -1152 1378}\special{fp}%
\special{pa -1113 1378}\special{pa -1074 1378}\special{fp}\special{pa -1035 1378}\special{pa -996 1378}\special{fp}%
\special{pa -957 1378}\special{pa -918 1378}\special{fp}\special{pa -879 1378}\special{pa -839 1378}\special{fp}%
\special{pa -800 1378}\special{pa -761 1378}\special{fp}\special{pa -722 1378}\special{pa -683 1378}\special{fp}%
\special{pa -644 1378}\special{pa -605 1378}\special{fp}\special{pa -566 1378}\special{pa -527 1378}\special{fp}%
\special{pa -488 1378}\special{pa -449 1378}\special{fp}\special{pa -410 1378}\special{pa -371 1378}\special{fp}%
\special{pa -332 1378}\special{pa -293 1378}\special{fp}\special{pa -254 1378}\special{pa -215 1378}\special{fp}%
\special{pa -176 1378}\special{pa -137 1378}\special{fp}\special{pa -98 1378}\special{pa -59 1378}\special{fp}%
\special{pa -20 1378}\special{pa 20 1378}\special{fp}\special{pa 59 1378}\special{pa 98 1378}\special{fp}%
\special{pa 137 1378}\special{pa 176 1378}\special{fp}\special{pa 215 1378}\special{pa 254 1378}\special{fp}%
\special{pa 293 1378}\special{pa 332 1378}\special{fp}\special{pa 371 1378}\special{pa 410 1378}\special{fp}%
\special{pa 449 1378}\special{pa 488 1378}\special{fp}\special{pa 527 1378}\special{pa 566 1378}\special{fp}%
\special{pa 605 1378}\special{pa 644 1378}\special{fp}\special{pa 683 1378}\special{pa 722 1378}\special{fp}%
\special{pa 761 1378}\special{pa 800 1378}\special{fp}\special{pa 839 1378}\special{pa 879 1378}\special{fp}%
\special{pa 918 1378}\special{pa 957 1378}\special{fp}\special{pa 996 1378}\special{pa 1035 1378}\special{fp}%
\special{pa 1074 1378}\special{pa 1113 1378}\special{fp}\special{pa 1152 1378}\special{pa 1191 1378}\special{fp}%
\special{pa 1230 1378}\special{pa 1269 1378}\special{fp}\special{pa 1308 1378}\special{pa 1347 1378}\special{fp}%
\special{pa 1386 1378}\special{pa 1425 1378}\special{fp}\special{pa 1464 1378}\special{pa 1503 1378}\special{fp}%
\special{pa 1542 1378}\special{pa 1581 1378}\special{fp}\special{pa 1620 1378}\special{pa 1659 1378}\special{fp}%
\special{pa 1698 1378}\special{pa 1737 1378}\special{fp}\special{pa 1777 1378}\special{pa 1816 1378}\special{fp}%
\special{pa 1855 1378}\special{pa 1894 1378}\special{fp}\special{pa 1933 1378}\special{pa 1972 1378}\special{fp}%
\special{pa 2011 1378}\special{pa 2050 1378}\special{fp}\special{pa 2089 1378}\special{pa 2128 1378}\special{fp}%
\special{pa 2167 1378}\special{pa 2206 1378}\special{fp}\special{pa 2245 1378}\special{pa 2284 1378}\special{fp}%
\special{pa 2323 1378}\special{pa 2362 1378}\special{fp}%
%
}%
{%
\color[rgb]{0,0,0}%
\special{pa -984 2362}\special{pa -984 2323}\special{fp}\special{pa -984 2284}\special{pa -984 2245}\special{fp}%
\special{pa -984 2206}\special{pa -984 2167}\special{fp}\special{pa -984 2128}\special{pa -984 2089}\special{fp}%
\special{pa -984 2050}\special{pa -984 2011}\special{fp}\special{pa -984 1972}\special{pa -984 1933}\special{fp}%
\special{pa -984 1894}\special{pa -984 1855}\special{fp}\special{pa -984 1816}\special{pa -984 1777}\special{fp}%
\special{pa -984 1737}\special{pa -984 1698}\special{fp}\special{pa -984 1659}\special{pa -984 1620}\special{fp}%
\special{pa -984 1581}\special{pa -984 1542}\special{fp}\special{pa -984 1503}\special{pa -984 1464}\special{fp}%
\special{pa -984 1425}\special{pa -984 1386}\special{fp}\special{pa -984 1347}\special{pa -984 1308}\special{fp}%
\special{pa -984 1269}\special{pa -984 1230}\special{fp}\special{pa -984 1191}\special{pa -984 1152}\special{fp}%
\special{pa -984 1113}\special{pa -984 1074}\special{fp}\special{pa -984 1035}\special{pa -984 996}\special{fp}%
\special{pa -984 957}\special{pa -984 918}\special{fp}\special{pa -984 879}\special{pa -984 839}\special{fp}%
\special{pa -984 800}\special{pa -984 761}\special{fp}\special{pa -984 722}\special{pa -984 683}\special{fp}%
\special{pa -984 644}\special{pa -984 605}\special{fp}\special{pa -984 566}\special{pa -984 527}\special{fp}%
\special{pa -984 488}\special{pa -984 449}\special{fp}\special{pa -984 410}\special{pa -984 371}\special{fp}%
\special{pa -984 332}\special{pa -984 293}\special{fp}\special{pa -984 254}\special{pa -984 215}\special{fp}%
\special{pa -984 176}\special{pa -984 137}\special{fp}\special{pa -984 98}\special{pa -984 59}\special{fp}%
\special{pa -984 20}\special{pa -984 -20}\special{fp}\special{pa -984 -59}\special{pa -984 -98}\special{fp}%
\special{pa -984 -137}\special{pa -984 -176}\special{fp}\special{pa -984 -215}\special{pa -984 -254}\special{fp}%
\special{pa -984 -293}\special{pa -984 -332}\special{fp}\special{pa -984 -371}\special{pa -984 -410}\special{fp}%
\special{pa -984 -449}\special{pa -984 -488}\special{fp}\special{pa -984 -527}\special{pa -984 -566}\special{fp}%
\special{pa -984 -605}\special{pa -984 -644}\special{fp}\special{pa -984 -683}\special{pa -984 -722}\special{fp}%
\special{pa -984 -761}\special{pa -984 -800}\special{fp}\special{pa -984 -839}\special{pa -984 -879}\special{fp}%
\special{pa -984 -918}\special{pa -984 -957}\special{fp}\special{pa -984 -996}\special{pa -984 -1035}\special{fp}%
\special{pa -984 -1074}\special{pa -984 -1113}\special{fp}\special{pa -984 -1152}\special{pa -984 -1191}\special{fp}%
\special{pa -984 -1230}\special{pa -984 -1269}\special{fp}\special{pa -984 -1308}\special{pa -984 -1347}\special{fp}%
\special{pa -984 -1386}\special{pa -984 -1425}\special{fp}\special{pa -984 -1464}\special{pa -984 -1503}\special{fp}%
\special{pa -984 -1542}\special{pa -984 -1581}\special{fp}\special{pa -984 -1620}\special{pa -984 -1659}\special{fp}%
\special{pa -984 -1698}\special{pa -984 -1737}\special{fp}\special{pa -984 -1777}\special{pa -984 -1816}\special{fp}%
\special{pa -984 -1855}\special{pa -984 -1894}\special{fp}\special{pa -984 -1933}\special{pa -984 -1972}\special{fp}%
\special{pa -984 -2011}\special{pa -984 -2050}\special{fp}\special{pa -984 -2089}\special{pa -984 -2128}\special{fp}%
\special{pa -984 -2167}\special{pa -984 -2206}\special{fp}\special{pa -984 -2245}\special{pa -984 -2284}\special{fp}%
\special{pa -984 -2323}\special{pa -984 -2362}\special{fp}%
%
}%
{%
\color[rgb]{0,0,0}%
\special{pa -2362 984}\special{pa -2323 984}\special{fp}\special{pa -2284 984}\special{pa -2245 984}\special{fp}%
\special{pa -2206 984}\special{pa -2167 984}\special{fp}\special{pa -2128 984}\special{pa -2089 984}\special{fp}%
\special{pa -2050 984}\special{pa -2011 984}\special{fp}\special{pa -1972 984}\special{pa -1933 984}\special{fp}%
\special{pa -1894 984}\special{pa -1855 984}\special{fp}\special{pa -1816 984}\special{pa -1777 984}\special{fp}%
\special{pa -1737 984}\special{pa -1698 984}\special{fp}\special{pa -1659 984}\special{pa -1620 984}\special{fp}%
\special{pa -1581 984}\special{pa -1542 984}\special{fp}\special{pa -1503 984}\special{pa -1464 984}\special{fp}%
\special{pa -1425 984}\special{pa -1386 984}\special{fp}\special{pa -1347 984}\special{pa -1308 984}\special{fp}%
\special{pa -1269 984}\special{pa -1230 984}\special{fp}\special{pa -1191 984}\special{pa -1152 984}\special{fp}%
\special{pa -1113 984}\special{pa -1074 984}\special{fp}\special{pa -1035 984}\special{pa -996 984}\special{fp}%
\special{pa -957 984}\special{pa -918 984}\special{fp}\special{pa -879 984}\special{pa -839 984}\special{fp}%
\special{pa -800 984}\special{pa -761 984}\special{fp}\special{pa -722 984}\special{pa -683 984}\special{fp}%
\special{pa -644 984}\special{pa -605 984}\special{fp}\special{pa -566 984}\special{pa -527 984}\special{fp}%
\special{pa -488 984}\special{pa -449 984}\special{fp}\special{pa -410 984}\special{pa -371 984}\special{fp}%
\special{pa -332 984}\special{pa -293 984}\special{fp}\special{pa -254 984}\special{pa -215 984}\special{fp}%
\special{pa -176 984}\special{pa -137 984}\special{fp}\special{pa -98 984}\special{pa -59 984}\special{fp}%
\special{pa -20 984}\special{pa 20 984}\special{fp}\special{pa 59 984}\special{pa 98 984}\special{fp}%
\special{pa 137 984}\special{pa 176 984}\special{fp}\special{pa 215 984}\special{pa 254 984}\special{fp}%
\special{pa 293 984}\special{pa 332 984}\special{fp}\special{pa 371 984}\special{pa 410 984}\special{fp}%
\special{pa 449 984}\special{pa 488 984}\special{fp}\special{pa 527 984}\special{pa 566 984}\special{fp}%
\special{pa 605 984}\special{pa 644 984}\special{fp}\special{pa 683 984}\special{pa 722 984}\special{fp}%
\special{pa 761 984}\special{pa 800 984}\special{fp}\special{pa 839 984}\special{pa 879 984}\special{fp}%
\special{pa 918 984}\special{pa 957 984}\special{fp}\special{pa 996 984}\special{pa 1035 984}\special{fp}%
\special{pa 1074 984}\special{pa 1113 984}\special{fp}\special{pa 1152 984}\special{pa 1191 984}\special{fp}%
\special{pa 1230 984}\special{pa 1269 984}\special{fp}\special{pa 1308 984}\special{pa 1347 984}\special{fp}%
\special{pa 1386 984}\special{pa 1425 984}\special{fp}\special{pa 1464 984}\special{pa 1503 984}\special{fp}%
\special{pa 1542 984}\special{pa 1581 984}\special{fp}\special{pa 1620 984}\special{pa 1659 984}\special{fp}%
\special{pa 1698 984}\special{pa 1737 984}\special{fp}\special{pa 1777 984}\special{pa 1816 984}\special{fp}%
\special{pa 1855 984}\special{pa 1894 984}\special{fp}\special{pa 1933 984}\special{pa 1972 984}\special{fp}%
\special{pa 2011 984}\special{pa 2050 984}\special{fp}\special{pa 2089 984}\special{pa 2128 984}\special{fp}%
\special{pa 2167 984}\special{pa 2206 984}\special{fp}\special{pa 2245 984}\special{pa 2284 984}\special{fp}%
\special{pa 2323 984}\special{pa 2362 984}\special{fp}%
%
}%
{%
\color[rgb]{0,0,0}%
\special{pa -591 2362}\special{pa -591 2323}\special{fp}\special{pa -591 2284}\special{pa -591 2245}\special{fp}%
\special{pa -591 2206}\special{pa -591 2167}\special{fp}\special{pa -591 2128}\special{pa -591 2089}\special{fp}%
\special{pa -591 2050}\special{pa -591 2011}\special{fp}\special{pa -591 1972}\special{pa -591 1933}\special{fp}%
\special{pa -591 1894}\special{pa -591 1855}\special{fp}\special{pa -591 1816}\special{pa -591 1777}\special{fp}%
\special{pa -591 1737}\special{pa -591 1698}\special{fp}\special{pa -591 1659}\special{pa -591 1620}\special{fp}%
\special{pa -591 1581}\special{pa -591 1542}\special{fp}\special{pa -591 1503}\special{pa -591 1464}\special{fp}%
\special{pa -591 1425}\special{pa -591 1386}\special{fp}\special{pa -591 1347}\special{pa -591 1308}\special{fp}%
\special{pa -591 1269}\special{pa -591 1230}\special{fp}\special{pa -591 1191}\special{pa -591 1152}\special{fp}%
\special{pa -591 1113}\special{pa -591 1074}\special{fp}\special{pa -591 1035}\special{pa -591 996}\special{fp}%
\special{pa -591 957}\special{pa -591 918}\special{fp}\special{pa -591 879}\special{pa -591 839}\special{fp}%
\special{pa -591 800}\special{pa -591 761}\special{fp}\special{pa -591 722}\special{pa -591 683}\special{fp}%
\special{pa -591 644}\special{pa -591 605}\special{fp}\special{pa -591 566}\special{pa -591 527}\special{fp}%
\special{pa -591 488}\special{pa -591 449}\special{fp}\special{pa -591 410}\special{pa -591 371}\special{fp}%
\special{pa -591 332}\special{pa -591 293}\special{fp}\special{pa -591 254}\special{pa -591 215}\special{fp}%
\special{pa -591 176}\special{pa -591 137}\special{fp}\special{pa -591 98}\special{pa -591 59}\special{fp}%
\special{pa -591 20}\special{pa -591 -20}\special{fp}\special{pa -591 -59}\special{pa -591 -98}\special{fp}%
\special{pa -591 -137}\special{pa -591 -176}\special{fp}\special{pa -591 -215}\special{pa -591 -254}\special{fp}%
\special{pa -591 -293}\special{pa -591 -332}\special{fp}\special{pa -591 -371}\special{pa -591 -410}\special{fp}%
\special{pa -591 -449}\special{pa -591 -488}\special{fp}\special{pa -591 -527}\special{pa -591 -566}\special{fp}%
\special{pa -591 -605}\special{pa -591 -644}\special{fp}\special{pa -591 -683}\special{pa -591 -722}\special{fp}%
\special{pa -591 -761}\special{pa -591 -800}\special{fp}\special{pa -591 -839}\special{pa -591 -879}\special{fp}%
\special{pa -591 -918}\special{pa -591 -957}\special{fp}\special{pa -591 -996}\special{pa -591 -1035}\special{fp}%
\special{pa -591 -1074}\special{pa -591 -1113}\special{fp}\special{pa -591 -1152}\special{pa -591 -1191}\special{fp}%
\special{pa -591 -1230}\special{pa -591 -1269}\special{fp}\special{pa -591 -1308}\special{pa -591 -1347}\special{fp}%
\special{pa -591 -1386}\special{pa -591 -1425}\special{fp}\special{pa -591 -1464}\special{pa -591 -1503}\special{fp}%
\special{pa -591 -1542}\special{pa -591 -1581}\special{fp}\special{pa -591 -1620}\special{pa -591 -1659}\special{fp}%
\special{pa -591 -1698}\special{pa -591 -1737}\special{fp}\special{pa -591 -1777}\special{pa -591 -1816}\special{fp}%
\special{pa -591 -1855}\special{pa -591 -1894}\special{fp}\special{pa -591 -1933}\special{pa -591 -1972}\special{fp}%
\special{pa -591 -2011}\special{pa -591 -2050}\special{fp}\special{pa -591 -2089}\special{pa -591 -2128}\special{fp}%
\special{pa -591 -2167}\special{pa -591 -2206}\special{fp}\special{pa -591 -2245}\special{pa -591 -2284}\special{fp}%
\special{pa -591 -2323}\special{pa -591 -2362}\special{fp}%
%
}%
{%
\color[rgb]{0,0,0}%
\special{pa -2362 591}\special{pa -2323 591}\special{fp}\special{pa -2284 591}\special{pa -2245 591}\special{fp}%
\special{pa -2206 591}\special{pa -2167 591}\special{fp}\special{pa -2128 591}\special{pa -2089 591}\special{fp}%
\special{pa -2050 591}\special{pa -2011 591}\special{fp}\special{pa -1972 591}\special{pa -1933 591}\special{fp}%
\special{pa -1894 591}\special{pa -1855 591}\special{fp}\special{pa -1816 591}\special{pa -1777 591}\special{fp}%
\special{pa -1737 591}\special{pa -1698 591}\special{fp}\special{pa -1659 591}\special{pa -1620 591}\special{fp}%
\special{pa -1581 591}\special{pa -1542 591}\special{fp}\special{pa -1503 591}\special{pa -1464 591}\special{fp}%
\special{pa -1425 591}\special{pa -1386 591}\special{fp}\special{pa -1347 591}\special{pa -1308 591}\special{fp}%
\special{pa -1269 591}\special{pa -1230 591}\special{fp}\special{pa -1191 591}\special{pa -1152 591}\special{fp}%
\special{pa -1113 591}\special{pa -1074 591}\special{fp}\special{pa -1035 591}\special{pa -996 591}\special{fp}%
\special{pa -957 591}\special{pa -918 591}\special{fp}\special{pa -879 591}\special{pa -839 591}\special{fp}%
\special{pa -800 591}\special{pa -761 591}\special{fp}\special{pa -722 591}\special{pa -683 591}\special{fp}%
\special{pa -644 591}\special{pa -605 591}\special{fp}\special{pa -566 591}\special{pa -527 591}\special{fp}%
\special{pa -488 591}\special{pa -449 591}\special{fp}\special{pa -410 591}\special{pa -371 591}\special{fp}%
\special{pa -332 591}\special{pa -293 591}\special{fp}\special{pa -254 591}\special{pa -215 591}\special{fp}%
\special{pa -176 591}\special{pa -137 591}\special{fp}\special{pa -98 591}\special{pa -59 591}\special{fp}%
\special{pa -20 591}\special{pa 20 591}\special{fp}\special{pa 59 591}\special{pa 98 591}\special{fp}%
\special{pa 137 591}\special{pa 176 591}\special{fp}\special{pa 215 591}\special{pa 254 591}\special{fp}%
\special{pa 293 591}\special{pa 332 591}\special{fp}\special{pa 371 591}\special{pa 410 591}\special{fp}%
\special{pa 449 591}\special{pa 488 591}\special{fp}\special{pa 527 591}\special{pa 566 591}\special{fp}%
\special{pa 605 591}\special{pa 644 591}\special{fp}\special{pa 683 591}\special{pa 722 591}\special{fp}%
\special{pa 761 591}\special{pa 800 591}\special{fp}\special{pa 839 591}\special{pa 879 591}\special{fp}%
\special{pa 918 591}\special{pa 957 591}\special{fp}\special{pa 996 591}\special{pa 1035 591}\special{fp}%
\special{pa 1074 591}\special{pa 1113 591}\special{fp}\special{pa 1152 591}\special{pa 1191 591}\special{fp}%
\special{pa 1230 591}\special{pa 1269 591}\special{fp}\special{pa 1308 591}\special{pa 1347 591}\special{fp}%
\special{pa 1386 591}\special{pa 1425 591}\special{fp}\special{pa 1464 591}\special{pa 1503 591}\special{fp}%
\special{pa 1542 591}\special{pa 1581 591}\special{fp}\special{pa 1620 591}\special{pa 1659 591}\special{fp}%
\special{pa 1698 591}\special{pa 1737 591}\special{fp}\special{pa 1777 591}\special{pa 1816 591}\special{fp}%
\special{pa 1855 591}\special{pa 1894 591}\special{fp}\special{pa 1933 591}\special{pa 1972 591}\special{fp}%
\special{pa 2011 591}\special{pa 2050 591}\special{fp}\special{pa 2089 591}\special{pa 2128 591}\special{fp}%
\special{pa 2167 591}\special{pa 2206 591}\special{fp}\special{pa 2245 591}\special{pa 2284 591}\special{fp}%
\special{pa 2323 591}\special{pa 2362 591}\special{fp}%
%
}%
{%
\color[rgb]{0,0,0}%
\special{pa -197 2362}\special{pa -197 2323}\special{fp}\special{pa -197 2284}\special{pa -197 2245}\special{fp}%
\special{pa -197 2206}\special{pa -197 2167}\special{fp}\special{pa -197 2128}\special{pa -197 2089}\special{fp}%
\special{pa -197 2050}\special{pa -197 2011}\special{fp}\special{pa -197 1972}\special{pa -197 1933}\special{fp}%
\special{pa -197 1894}\special{pa -197 1855}\special{fp}\special{pa -197 1816}\special{pa -197 1777}\special{fp}%
\special{pa -197 1737}\special{pa -197 1698}\special{fp}\special{pa -197 1659}\special{pa -197 1620}\special{fp}%
\special{pa -197 1581}\special{pa -197 1542}\special{fp}\special{pa -197 1503}\special{pa -197 1464}\special{fp}%
\special{pa -197 1425}\special{pa -197 1386}\special{fp}\special{pa -197 1347}\special{pa -197 1308}\special{fp}%
\special{pa -197 1269}\special{pa -197 1230}\special{fp}\special{pa -197 1191}\special{pa -197 1152}\special{fp}%
\special{pa -197 1113}\special{pa -197 1074}\special{fp}\special{pa -197 1035}\special{pa -197 996}\special{fp}%
\special{pa -197 957}\special{pa -197 918}\special{fp}\special{pa -197 879}\special{pa -197 839}\special{fp}%
\special{pa -197 800}\special{pa -197 761}\special{fp}\special{pa -197 722}\special{pa -197 683}\special{fp}%
\special{pa -197 644}\special{pa -197 605}\special{fp}\special{pa -197 566}\special{pa -197 527}\special{fp}%
\special{pa -197 488}\special{pa -197 449}\special{fp}\special{pa -197 410}\special{pa -197 371}\special{fp}%
\special{pa -197 332}\special{pa -197 293}\special{fp}\special{pa -197 254}\special{pa -197 215}\special{fp}%
\special{pa -197 176}\special{pa -197 137}\special{fp}\special{pa -197 98}\special{pa -197 59}\special{fp}%
\special{pa -197 20}\special{pa -197 -20}\special{fp}\special{pa -197 -59}\special{pa -197 -98}\special{fp}%
\special{pa -197 -137}\special{pa -197 -176}\special{fp}\special{pa -197 -215}\special{pa -197 -254}\special{fp}%
\special{pa -197 -293}\special{pa -197 -332}\special{fp}\special{pa -197 -371}\special{pa -197 -410}\special{fp}%
\special{pa -197 -449}\special{pa -197 -488}\special{fp}\special{pa -197 -527}\special{pa -197 -566}\special{fp}%
\special{pa -197 -605}\special{pa -197 -644}\special{fp}\special{pa -197 -683}\special{pa -197 -722}\special{fp}%
\special{pa -197 -761}\special{pa -197 -800}\special{fp}\special{pa -197 -839}\special{pa -197 -879}\special{fp}%
\special{pa -197 -918}\special{pa -197 -957}\special{fp}\special{pa -197 -996}\special{pa -197 -1035}\special{fp}%
\special{pa -197 -1074}\special{pa -197 -1113}\special{fp}\special{pa -197 -1152}\special{pa -197 -1191}\special{fp}%
\special{pa -197 -1230}\special{pa -197 -1269}\special{fp}\special{pa -197 -1308}\special{pa -197 -1347}\special{fp}%
\special{pa -197 -1386}\special{pa -197 -1425}\special{fp}\special{pa -197 -1464}\special{pa -197 -1503}\special{fp}%
\special{pa -197 -1542}\special{pa -197 -1581}\special{fp}\special{pa -197 -1620}\special{pa -197 -1659}\special{fp}%
\special{pa -197 -1698}\special{pa -197 -1737}\special{fp}\special{pa -197 -1777}\special{pa -197 -1816}\special{fp}%
\special{pa -197 -1855}\special{pa -197 -1894}\special{fp}\special{pa -197 -1933}\special{pa -197 -1972}\special{fp}%
\special{pa -197 -2011}\special{pa -197 -2050}\special{fp}\special{pa -197 -2089}\special{pa -197 -2128}\special{fp}%
\special{pa -197 -2167}\special{pa -197 -2206}\special{fp}\special{pa -197 -2245}\special{pa -197 -2284}\special{fp}%
\special{pa -197 -2323}\special{pa -197 -2362}\special{fp}%
%
}%
{%
\color[rgb]{0,0,0}%
\special{pa -2362 197}\special{pa -2323 197}\special{fp}\special{pa -2284 197}\special{pa -2245 197}\special{fp}%
\special{pa -2206 197}\special{pa -2167 197}\special{fp}\special{pa -2128 197}\special{pa -2089 197}\special{fp}%
\special{pa -2050 197}\special{pa -2011 197}\special{fp}\special{pa -1972 197}\special{pa -1933 197}\special{fp}%
\special{pa -1894 197}\special{pa -1855 197}\special{fp}\special{pa -1816 197}\special{pa -1777 197}\special{fp}%
\special{pa -1737 197}\special{pa -1698 197}\special{fp}\special{pa -1659 197}\special{pa -1620 197}\special{fp}%
\special{pa -1581 197}\special{pa -1542 197}\special{fp}\special{pa -1503 197}\special{pa -1464 197}\special{fp}%
\special{pa -1425 197}\special{pa -1386 197}\special{fp}\special{pa -1347 197}\special{pa -1308 197}\special{fp}%
\special{pa -1269 197}\special{pa -1230 197}\special{fp}\special{pa -1191 197}\special{pa -1152 197}\special{fp}%
\special{pa -1113 197}\special{pa -1074 197}\special{fp}\special{pa -1035 197}\special{pa -996 197}\special{fp}%
\special{pa -957 197}\special{pa -918 197}\special{fp}\special{pa -879 197}\special{pa -839 197}\special{fp}%
\special{pa -800 197}\special{pa -761 197}\special{fp}\special{pa -722 197}\special{pa -683 197}\special{fp}%
\special{pa -644 197}\special{pa -605 197}\special{fp}\special{pa -566 197}\special{pa -527 197}\special{fp}%
\special{pa -488 197}\special{pa -449 197}\special{fp}\special{pa -410 197}\special{pa -371 197}\special{fp}%
\special{pa -332 197}\special{pa -293 197}\special{fp}\special{pa -254 197}\special{pa -215 197}\special{fp}%
\special{pa -176 197}\special{pa -137 197}\special{fp}\special{pa -98 197}\special{pa -59 197}\special{fp}%
\special{pa -20 197}\special{pa 20 197}\special{fp}\special{pa 59 197}\special{pa 98 197}\special{fp}%
\special{pa 137 197}\special{pa 176 197}\special{fp}\special{pa 215 197}\special{pa 254 197}\special{fp}%
\special{pa 293 197}\special{pa 332 197}\special{fp}\special{pa 371 197}\special{pa 410 197}\special{fp}%
\special{pa 449 197}\special{pa 488 197}\special{fp}\special{pa 527 197}\special{pa 566 197}\special{fp}%
\special{pa 605 197}\special{pa 644 197}\special{fp}\special{pa 683 197}\special{pa 722 197}\special{fp}%
\special{pa 761 197}\special{pa 800 197}\special{fp}\special{pa 839 197}\special{pa 879 197}\special{fp}%
\special{pa 918 197}\special{pa 957 197}\special{fp}\special{pa 996 197}\special{pa 1035 197}\special{fp}%
\special{pa 1074 197}\special{pa 1113 197}\special{fp}\special{pa 1152 197}\special{pa 1191 197}\special{fp}%
\special{pa 1230 197}\special{pa 1269 197}\special{fp}\special{pa 1308 197}\special{pa 1347 197}\special{fp}%
\special{pa 1386 197}\special{pa 1425 197}\special{fp}\special{pa 1464 197}\special{pa 1503 197}\special{fp}%
\special{pa 1542 197}\special{pa 1581 197}\special{fp}\special{pa 1620 197}\special{pa 1659 197}\special{fp}%
\special{pa 1698 197}\special{pa 1737 197}\special{fp}\special{pa 1777 197}\special{pa 1816 197}\special{fp}%
\special{pa 1855 197}\special{pa 1894 197}\special{fp}\special{pa 1933 197}\special{pa 1972 197}\special{fp}%
\special{pa 2011 197}\special{pa 2050 197}\special{fp}\special{pa 2089 197}\special{pa 2128 197}\special{fp}%
\special{pa 2167 197}\special{pa 2206 197}\special{fp}\special{pa 2245 197}\special{pa 2284 197}\special{fp}%
\special{pa 2323 197}\special{pa 2362 197}\special{fp}%
%
}%
{%
\color[rgb]{0,0,0}%
\special{pa 197 2362}\special{pa 197 2323}\special{fp}\special{pa 197 2284}\special{pa 197 2245}\special{fp}%
\special{pa 197 2206}\special{pa 197 2167}\special{fp}\special{pa 197 2128}\special{pa 197 2089}\special{fp}%
\special{pa 197 2050}\special{pa 197 2011}\special{fp}\special{pa 197 1972}\special{pa 197 1933}\special{fp}%
\special{pa 197 1894}\special{pa 197 1855}\special{fp}\special{pa 197 1816}\special{pa 197 1777}\special{fp}%
\special{pa 197 1737}\special{pa 197 1698}\special{fp}\special{pa 197 1659}\special{pa 197 1620}\special{fp}%
\special{pa 197 1581}\special{pa 197 1542}\special{fp}\special{pa 197 1503}\special{pa 197 1464}\special{fp}%
\special{pa 197 1425}\special{pa 197 1386}\special{fp}\special{pa 197 1347}\special{pa 197 1308}\special{fp}%
\special{pa 197 1269}\special{pa 197 1230}\special{fp}\special{pa 197 1191}\special{pa 197 1152}\special{fp}%
\special{pa 197 1113}\special{pa 197 1074}\special{fp}\special{pa 197 1035}\special{pa 197 996}\special{fp}%
\special{pa 197 957}\special{pa 197 918}\special{fp}\special{pa 197 879}\special{pa 197 839}\special{fp}%
\special{pa 197 800}\special{pa 197 761}\special{fp}\special{pa 197 722}\special{pa 197 683}\special{fp}%
\special{pa 197 644}\special{pa 197 605}\special{fp}\special{pa 197 566}\special{pa 197 527}\special{fp}%
\special{pa 197 488}\special{pa 197 449}\special{fp}\special{pa 197 410}\special{pa 197 371}\special{fp}%
\special{pa 197 332}\special{pa 197 293}\special{fp}\special{pa 197 254}\special{pa 197 215}\special{fp}%
\special{pa 197 176}\special{pa 197 137}\special{fp}\special{pa 197 98}\special{pa 197 59}\special{fp}%
\special{pa 197 20}\special{pa 197 -20}\special{fp}\special{pa 197 -59}\special{pa 197 -98}\special{fp}%
\special{pa 197 -137}\special{pa 197 -176}\special{fp}\special{pa 197 -215}\special{pa 197 -254}\special{fp}%
\special{pa 197 -293}\special{pa 197 -332}\special{fp}\special{pa 197 -371}\special{pa 197 -410}\special{fp}%
\special{pa 197 -449}\special{pa 197 -488}\special{fp}\special{pa 197 -527}\special{pa 197 -566}\special{fp}%
\special{pa 197 -605}\special{pa 197 -644}\special{fp}\special{pa 197 -683}\special{pa 197 -722}\special{fp}%
\special{pa 197 -761}\special{pa 197 -800}\special{fp}\special{pa 197 -839}\special{pa 197 -879}\special{fp}%
\special{pa 197 -918}\special{pa 197 -957}\special{fp}\special{pa 197 -996}\special{pa 197 -1035}\special{fp}%
\special{pa 197 -1074}\special{pa 197 -1113}\special{fp}\special{pa 197 -1152}\special{pa 197 -1191}\special{fp}%
\special{pa 197 -1230}\special{pa 197 -1269}\special{fp}\special{pa 197 -1308}\special{pa 197 -1347}\special{fp}%
\special{pa 197 -1386}\special{pa 197 -1425}\special{fp}\special{pa 197 -1464}\special{pa 197 -1503}\special{fp}%
\special{pa 197 -1542}\special{pa 197 -1581}\special{fp}\special{pa 197 -1620}\special{pa 197 -1659}\special{fp}%
\special{pa 197 -1698}\special{pa 197 -1737}\special{fp}\special{pa 197 -1777}\special{pa 197 -1816}\special{fp}%
\special{pa 197 -1855}\special{pa 197 -1894}\special{fp}\special{pa 197 -1933}\special{pa 197 -1972}\special{fp}%
\special{pa 197 -2011}\special{pa 197 -2050}\special{fp}\special{pa 197 -2089}\special{pa 197 -2128}\special{fp}%
\special{pa 197 -2167}\special{pa 197 -2206}\special{fp}\special{pa 197 -2245}\special{pa 197 -2284}\special{fp}%
\special{pa 197 -2323}\special{pa 197 -2362}\special{fp}%
%
}%
{%
\color[rgb]{0,0,0}%
\special{pa -2362 -197}\special{pa -2323 -197}\special{fp}\special{pa -2284 -197}\special{pa -2245 -197}\special{fp}%
\special{pa -2206 -197}\special{pa -2167 -197}\special{fp}\special{pa -2128 -197}\special{pa -2089 -197}\special{fp}%
\special{pa -2050 -197}\special{pa -2011 -197}\special{fp}\special{pa -1972 -197}\special{pa -1933 -197}\special{fp}%
\special{pa -1894 -197}\special{pa -1855 -197}\special{fp}\special{pa -1816 -197}\special{pa -1777 -197}\special{fp}%
\special{pa -1737 -197}\special{pa -1698 -197}\special{fp}\special{pa -1659 -197}\special{pa -1620 -197}\special{fp}%
\special{pa -1581 -197}\special{pa -1542 -197}\special{fp}\special{pa -1503 -197}\special{pa -1464 -197}\special{fp}%
\special{pa -1425 -197}\special{pa -1386 -197}\special{fp}\special{pa -1347 -197}\special{pa -1308 -197}\special{fp}%
\special{pa -1269 -197}\special{pa -1230 -197}\special{fp}\special{pa -1191 -197}\special{pa -1152 -197}\special{fp}%
\special{pa -1113 -197}\special{pa -1074 -197}\special{fp}\special{pa -1035 -197}\special{pa -996 -197}\special{fp}%
\special{pa -957 -197}\special{pa -918 -197}\special{fp}\special{pa -879 -197}\special{pa -839 -197}\special{fp}%
\special{pa -800 -197}\special{pa -761 -197}\special{fp}\special{pa -722 -197}\special{pa -683 -197}\special{fp}%
\special{pa -644 -197}\special{pa -605 -197}\special{fp}\special{pa -566 -197}\special{pa -527 -197}\special{fp}%
\special{pa -488 -197}\special{pa -449 -197}\special{fp}\special{pa -410 -197}\special{pa -371 -197}\special{fp}%
\special{pa -332 -197}\special{pa -293 -197}\special{fp}\special{pa -254 -197}\special{pa -215 -197}\special{fp}%
\special{pa -176 -197}\special{pa -137 -197}\special{fp}\special{pa -98 -197}\special{pa -59 -197}\special{fp}%
\special{pa -20 -197}\special{pa 20 -197}\special{fp}\special{pa 59 -197}\special{pa 98 -197}\special{fp}%
\special{pa 137 -197}\special{pa 176 -197}\special{fp}\special{pa 215 -197}\special{pa 254 -197}\special{fp}%
\special{pa 293 -197}\special{pa 332 -197}\special{fp}\special{pa 371 -197}\special{pa 410 -197}\special{fp}%
\special{pa 449 -197}\special{pa 488 -197}\special{fp}\special{pa 527 -197}\special{pa 566 -197}\special{fp}%
\special{pa 605 -197}\special{pa 644 -197}\special{fp}\special{pa 683 -197}\special{pa 722 -197}\special{fp}%
\special{pa 761 -197}\special{pa 800 -197}\special{fp}\special{pa 839 -197}\special{pa 879 -197}\special{fp}%
\special{pa 918 -197}\special{pa 957 -197}\special{fp}\special{pa 996 -197}\special{pa 1035 -197}\special{fp}%
\special{pa 1074 -197}\special{pa 1113 -197}\special{fp}\special{pa 1152 -197}\special{pa 1191 -197}\special{fp}%
\special{pa 1230 -197}\special{pa 1269 -197}\special{fp}\special{pa 1308 -197}\special{pa 1347 -197}\special{fp}%
\special{pa 1386 -197}\special{pa 1425 -197}\special{fp}\special{pa 1464 -197}\special{pa 1503 -197}\special{fp}%
\special{pa 1542 -197}\special{pa 1581 -197}\special{fp}\special{pa 1620 -197}\special{pa 1659 -197}\special{fp}%
\special{pa 1698 -197}\special{pa 1737 -197}\special{fp}\special{pa 1777 -197}\special{pa 1816 -197}\special{fp}%
\special{pa 1855 -197}\special{pa 1894 -197}\special{fp}\special{pa 1933 -197}\special{pa 1972 -197}\special{fp}%
\special{pa 2011 -197}\special{pa 2050 -197}\special{fp}\special{pa 2089 -197}\special{pa 2128 -197}\special{fp}%
\special{pa 2167 -197}\special{pa 2206 -197}\special{fp}\special{pa 2245 -197}\special{pa 2284 -197}\special{fp}%
\special{pa 2323 -197}\special{pa 2362 -197}\special{fp}%
%
}%
{%
\color[rgb]{0,0,0}%
\special{pa 591 2362}\special{pa 591 2323}\special{fp}\special{pa 591 2284}\special{pa 591 2245}\special{fp}%
\special{pa 591 2206}\special{pa 591 2167}\special{fp}\special{pa 591 2128}\special{pa 591 2089}\special{fp}%
\special{pa 591 2050}\special{pa 591 2011}\special{fp}\special{pa 591 1972}\special{pa 591 1933}\special{fp}%
\special{pa 591 1894}\special{pa 591 1855}\special{fp}\special{pa 591 1816}\special{pa 591 1777}\special{fp}%
\special{pa 591 1737}\special{pa 591 1698}\special{fp}\special{pa 591 1659}\special{pa 591 1620}\special{fp}%
\special{pa 591 1581}\special{pa 591 1542}\special{fp}\special{pa 591 1503}\special{pa 591 1464}\special{fp}%
\special{pa 591 1425}\special{pa 591 1386}\special{fp}\special{pa 591 1347}\special{pa 591 1308}\special{fp}%
\special{pa 591 1269}\special{pa 591 1230}\special{fp}\special{pa 591 1191}\special{pa 591 1152}\special{fp}%
\special{pa 591 1113}\special{pa 591 1074}\special{fp}\special{pa 591 1035}\special{pa 591 996}\special{fp}%
\special{pa 591 957}\special{pa 591 918}\special{fp}\special{pa 591 879}\special{pa 591 839}\special{fp}%
\special{pa 591 800}\special{pa 591 761}\special{fp}\special{pa 591 722}\special{pa 591 683}\special{fp}%
\special{pa 591 644}\special{pa 591 605}\special{fp}\special{pa 591 566}\special{pa 591 527}\special{fp}%
\special{pa 591 488}\special{pa 591 449}\special{fp}\special{pa 591 410}\special{pa 591 371}\special{fp}%
\special{pa 591 332}\special{pa 591 293}\special{fp}\special{pa 591 254}\special{pa 591 215}\special{fp}%
\special{pa 591 176}\special{pa 591 137}\special{fp}\special{pa 591 98}\special{pa 591 59}\special{fp}%
\special{pa 591 20}\special{pa 591 -20}\special{fp}\special{pa 591 -59}\special{pa 591 -98}\special{fp}%
\special{pa 591 -137}\special{pa 591 -176}\special{fp}\special{pa 591 -215}\special{pa 591 -254}\special{fp}%
\special{pa 591 -293}\special{pa 591 -332}\special{fp}\special{pa 591 -371}\special{pa 591 -410}\special{fp}%
\special{pa 591 -449}\special{pa 591 -488}\special{fp}\special{pa 591 -527}\special{pa 591 -566}\special{fp}%
\special{pa 591 -605}\special{pa 591 -644}\special{fp}\special{pa 591 -683}\special{pa 591 -722}\special{fp}%
\special{pa 591 -761}\special{pa 591 -800}\special{fp}\special{pa 591 -839}\special{pa 591 -879}\special{fp}%
\special{pa 591 -918}\special{pa 591 -957}\special{fp}\special{pa 591 -996}\special{pa 591 -1035}\special{fp}%
\special{pa 591 -1074}\special{pa 591 -1113}\special{fp}\special{pa 591 -1152}\special{pa 591 -1191}\special{fp}%
\special{pa 591 -1230}\special{pa 591 -1269}\special{fp}\special{pa 591 -1308}\special{pa 591 -1347}\special{fp}%
\special{pa 591 -1386}\special{pa 591 -1425}\special{fp}\special{pa 591 -1464}\special{pa 591 -1503}\special{fp}%
\special{pa 591 -1542}\special{pa 591 -1581}\special{fp}\special{pa 591 -1620}\special{pa 591 -1659}\special{fp}%
\special{pa 591 -1698}\special{pa 591 -1737}\special{fp}\special{pa 591 -1777}\special{pa 591 -1816}\special{fp}%
\special{pa 591 -1855}\special{pa 591 -1894}\special{fp}\special{pa 591 -1933}\special{pa 591 -1972}\special{fp}%
\special{pa 591 -2011}\special{pa 591 -2050}\special{fp}\special{pa 591 -2089}\special{pa 591 -2128}\special{fp}%
\special{pa 591 -2167}\special{pa 591 -2206}\special{fp}\special{pa 591 -2245}\special{pa 591 -2284}\special{fp}%
\special{pa 591 -2323}\special{pa 591 -2362}\special{fp}%
%
}%
{%
\color[rgb]{0,0,0}%
\special{pa -2362 -591}\special{pa -2323 -591}\special{fp}\special{pa -2284 -591}\special{pa -2245 -591}\special{fp}%
\special{pa -2206 -591}\special{pa -2167 -591}\special{fp}\special{pa -2128 -591}\special{pa -2089 -591}\special{fp}%
\special{pa -2050 -591}\special{pa -2011 -591}\special{fp}\special{pa -1972 -591}\special{pa -1933 -591}\special{fp}%
\special{pa -1894 -591}\special{pa -1855 -591}\special{fp}\special{pa -1816 -591}\special{pa -1777 -591}\special{fp}%
\special{pa -1737 -591}\special{pa -1698 -591}\special{fp}\special{pa -1659 -591}\special{pa -1620 -591}\special{fp}%
\special{pa -1581 -591}\special{pa -1542 -591}\special{fp}\special{pa -1503 -591}\special{pa -1464 -591}\special{fp}%
\special{pa -1425 -591}\special{pa -1386 -591}\special{fp}\special{pa -1347 -591}\special{pa -1308 -591}\special{fp}%
\special{pa -1269 -591}\special{pa -1230 -591}\special{fp}\special{pa -1191 -591}\special{pa -1152 -591}\special{fp}%
\special{pa -1113 -591}\special{pa -1074 -591}\special{fp}\special{pa -1035 -591}\special{pa -996 -591}\special{fp}%
\special{pa -957 -591}\special{pa -918 -591}\special{fp}\special{pa -879 -591}\special{pa -839 -591}\special{fp}%
\special{pa -800 -591}\special{pa -761 -591}\special{fp}\special{pa -722 -591}\special{pa -683 -591}\special{fp}%
\special{pa -644 -591}\special{pa -605 -591}\special{fp}\special{pa -566 -591}\special{pa -527 -591}\special{fp}%
\special{pa -488 -591}\special{pa -449 -591}\special{fp}\special{pa -410 -591}\special{pa -371 -591}\special{fp}%
\special{pa -332 -591}\special{pa -293 -591}\special{fp}\special{pa -254 -591}\special{pa -215 -591}\special{fp}%
\special{pa -176 -591}\special{pa -137 -591}\special{fp}\special{pa -98 -591}\special{pa -59 -591}\special{fp}%
\special{pa -20 -591}\special{pa 20 -591}\special{fp}\special{pa 59 -591}\special{pa 98 -591}\special{fp}%
\special{pa 137 -591}\special{pa 176 -591}\special{fp}\special{pa 215 -591}\special{pa 254 -591}\special{fp}%
\special{pa 293 -591}\special{pa 332 -591}\special{fp}\special{pa 371 -591}\special{pa 410 -591}\special{fp}%
\special{pa 449 -591}\special{pa 488 -591}\special{fp}\special{pa 527 -591}\special{pa 566 -591}\special{fp}%
\special{pa 605 -591}\special{pa 644 -591}\special{fp}\special{pa 683 -591}\special{pa 722 -591}\special{fp}%
\special{pa 761 -591}\special{pa 800 -591}\special{fp}\special{pa 839 -591}\special{pa 879 -591}\special{fp}%
\special{pa 918 -591}\special{pa 957 -591}\special{fp}\special{pa 996 -591}\special{pa 1035 -591}\special{fp}%
\special{pa 1074 -591}\special{pa 1113 -591}\special{fp}\special{pa 1152 -591}\special{pa 1191 -591}\special{fp}%
\special{pa 1230 -591}\special{pa 1269 -591}\special{fp}\special{pa 1308 -591}\special{pa 1347 -591}\special{fp}%
\special{pa 1386 -591}\special{pa 1425 -591}\special{fp}\special{pa 1464 -591}\special{pa 1503 -591}\special{fp}%
\special{pa 1542 -591}\special{pa 1581 -591}\special{fp}\special{pa 1620 -591}\special{pa 1659 -591}\special{fp}%
\special{pa 1698 -591}\special{pa 1737 -591}\special{fp}\special{pa 1777 -591}\special{pa 1816 -591}\special{fp}%
\special{pa 1855 -591}\special{pa 1894 -591}\special{fp}\special{pa 1933 -591}\special{pa 1972 -591}\special{fp}%
\special{pa 2011 -591}\special{pa 2050 -591}\special{fp}\special{pa 2089 -591}\special{pa 2128 -591}\special{fp}%
\special{pa 2167 -591}\special{pa 2206 -591}\special{fp}\special{pa 2245 -591}\special{pa 2284 -591}\special{fp}%
\special{pa 2323 -591}\special{pa 2362 -591}\special{fp}%
%
}%
{%
\color[rgb]{0,0,0}%
\special{pa 984 2362}\special{pa 984 2323}\special{fp}\special{pa 984 2284}\special{pa 984 2245}\special{fp}%
\special{pa 984 2206}\special{pa 984 2167}\special{fp}\special{pa 984 2128}\special{pa 984 2089}\special{fp}%
\special{pa 984 2050}\special{pa 984 2011}\special{fp}\special{pa 984 1972}\special{pa 984 1933}\special{fp}%
\special{pa 984 1894}\special{pa 984 1855}\special{fp}\special{pa 984 1816}\special{pa 984 1777}\special{fp}%
\special{pa 984 1737}\special{pa 984 1698}\special{fp}\special{pa 984 1659}\special{pa 984 1620}\special{fp}%
\special{pa 984 1581}\special{pa 984 1542}\special{fp}\special{pa 984 1503}\special{pa 984 1464}\special{fp}%
\special{pa 984 1425}\special{pa 984 1386}\special{fp}\special{pa 984 1347}\special{pa 984 1308}\special{fp}%
\special{pa 984 1269}\special{pa 984 1230}\special{fp}\special{pa 984 1191}\special{pa 984 1152}\special{fp}%
\special{pa 984 1113}\special{pa 984 1074}\special{fp}\special{pa 984 1035}\special{pa 984 996}\special{fp}%
\special{pa 984 957}\special{pa 984 918}\special{fp}\special{pa 984 879}\special{pa 984 839}\special{fp}%
\special{pa 984 800}\special{pa 984 761}\special{fp}\special{pa 984 722}\special{pa 984 683}\special{fp}%
\special{pa 984 644}\special{pa 984 605}\special{fp}\special{pa 984 566}\special{pa 984 527}\special{fp}%
\special{pa 984 488}\special{pa 984 449}\special{fp}\special{pa 984 410}\special{pa 984 371}\special{fp}%
\special{pa 984 332}\special{pa 984 293}\special{fp}\special{pa 984 254}\special{pa 984 215}\special{fp}%
\special{pa 984 176}\special{pa 984 137}\special{fp}\special{pa 984 98}\special{pa 984 59}\special{fp}%
\special{pa 984 20}\special{pa 984 -20}\special{fp}\special{pa 984 -59}\special{pa 984 -98}\special{fp}%
\special{pa 984 -137}\special{pa 984 -176}\special{fp}\special{pa 984 -215}\special{pa 984 -254}\special{fp}%
\special{pa 984 -293}\special{pa 984 -332}\special{fp}\special{pa 984 -371}\special{pa 984 -410}\special{fp}%
\special{pa 984 -449}\special{pa 984 -488}\special{fp}\special{pa 984 -527}\special{pa 984 -566}\special{fp}%
\special{pa 984 -605}\special{pa 984 -644}\special{fp}\special{pa 984 -683}\special{pa 984 -722}\special{fp}%
\special{pa 984 -761}\special{pa 984 -800}\special{fp}\special{pa 984 -839}\special{pa 984 -879}\special{fp}%
\special{pa 984 -918}\special{pa 984 -957}\special{fp}\special{pa 984 -996}\special{pa 984 -1035}\special{fp}%
\special{pa 984 -1074}\special{pa 984 -1113}\special{fp}\special{pa 984 -1152}\special{pa 984 -1191}\special{fp}%
\special{pa 984 -1230}\special{pa 984 -1269}\special{fp}\special{pa 984 -1308}\special{pa 984 -1347}\special{fp}%
\special{pa 984 -1386}\special{pa 984 -1425}\special{fp}\special{pa 984 -1464}\special{pa 984 -1503}\special{fp}%
\special{pa 984 -1542}\special{pa 984 -1581}\special{fp}\special{pa 984 -1620}\special{pa 984 -1659}\special{fp}%
\special{pa 984 -1698}\special{pa 984 -1737}\special{fp}\special{pa 984 -1777}\special{pa 984 -1816}\special{fp}%
\special{pa 984 -1855}\special{pa 984 -1894}\special{fp}\special{pa 984 -1933}\special{pa 984 -1972}\special{fp}%
\special{pa 984 -2011}\special{pa 984 -2050}\special{fp}\special{pa 984 -2089}\special{pa 984 -2128}\special{fp}%
\special{pa 984 -2167}\special{pa 984 -2206}\special{fp}\special{pa 984 -2245}\special{pa 984 -2284}\special{fp}%
\special{pa 984 -2323}\special{pa 984 -2362}\special{fp}%
%
}%
{%
\color[rgb]{0,0,0}%
\special{pa -2362 -984}\special{pa -2323 -984}\special{fp}\special{pa -2284 -984}\special{pa -2245 -984}\special{fp}%
\special{pa -2206 -984}\special{pa -2167 -984}\special{fp}\special{pa -2128 -984}\special{pa -2089 -984}\special{fp}%
\special{pa -2050 -984}\special{pa -2011 -984}\special{fp}\special{pa -1972 -984}\special{pa -1933 -984}\special{fp}%
\special{pa -1894 -984}\special{pa -1855 -984}\special{fp}\special{pa -1816 -984}\special{pa -1777 -984}\special{fp}%
\special{pa -1737 -984}\special{pa -1698 -984}\special{fp}\special{pa -1659 -984}\special{pa -1620 -984}\special{fp}%
\special{pa -1581 -984}\special{pa -1542 -984}\special{fp}\special{pa -1503 -984}\special{pa -1464 -984}\special{fp}%
\special{pa -1425 -984}\special{pa -1386 -984}\special{fp}\special{pa -1347 -984}\special{pa -1308 -984}\special{fp}%
\special{pa -1269 -984}\special{pa -1230 -984}\special{fp}\special{pa -1191 -984}\special{pa -1152 -984}\special{fp}%
\special{pa -1113 -984}\special{pa -1074 -984}\special{fp}\special{pa -1035 -984}\special{pa -996 -984}\special{fp}%
\special{pa -957 -984}\special{pa -918 -984}\special{fp}\special{pa -879 -984}\special{pa -839 -984}\special{fp}%
\special{pa -800 -984}\special{pa -761 -984}\special{fp}\special{pa -722 -984}\special{pa -683 -984}\special{fp}%
\special{pa -644 -984}\special{pa -605 -984}\special{fp}\special{pa -566 -984}\special{pa -527 -984}\special{fp}%
\special{pa -488 -984}\special{pa -449 -984}\special{fp}\special{pa -410 -984}\special{pa -371 -984}\special{fp}%
\special{pa -332 -984}\special{pa -293 -984}\special{fp}\special{pa -254 -984}\special{pa -215 -984}\special{fp}%
\special{pa -176 -984}\special{pa -137 -984}\special{fp}\special{pa -98 -984}\special{pa -59 -984}\special{fp}%
\special{pa -20 -984}\special{pa 20 -984}\special{fp}\special{pa 59 -984}\special{pa 98 -984}\special{fp}%
\special{pa 137 -984}\special{pa 176 -984}\special{fp}\special{pa 215 -984}\special{pa 254 -984}\special{fp}%
\special{pa 293 -984}\special{pa 332 -984}\special{fp}\special{pa 371 -984}\special{pa 410 -984}\special{fp}%
\special{pa 449 -984}\special{pa 488 -984}\special{fp}\special{pa 527 -984}\special{pa 566 -984}\special{fp}%
\special{pa 605 -984}\special{pa 644 -984}\special{fp}\special{pa 683 -984}\special{pa 722 -984}\special{fp}%
\special{pa 761 -984}\special{pa 800 -984}\special{fp}\special{pa 839 -984}\special{pa 879 -984}\special{fp}%
\special{pa 918 -984}\special{pa 957 -984}\special{fp}\special{pa 996 -984}\special{pa 1035 -984}\special{fp}%
\special{pa 1074 -984}\special{pa 1113 -984}\special{fp}\special{pa 1152 -984}\special{pa 1191 -984}\special{fp}%
\special{pa 1230 -984}\special{pa 1269 -984}\special{fp}\special{pa 1308 -984}\special{pa 1347 -984}\special{fp}%
\special{pa 1386 -984}\special{pa 1425 -984}\special{fp}\special{pa 1464 -984}\special{pa 1503 -984}\special{fp}%
\special{pa 1542 -984}\special{pa 1581 -984}\special{fp}\special{pa 1620 -984}\special{pa 1659 -984}\special{fp}%
\special{pa 1698 -984}\special{pa 1737 -984}\special{fp}\special{pa 1777 -984}\special{pa 1816 -984}\special{fp}%
\special{pa 1855 -984}\special{pa 1894 -984}\special{fp}\special{pa 1933 -984}\special{pa 1972 -984}\special{fp}%
\special{pa 2011 -984}\special{pa 2050 -984}\special{fp}\special{pa 2089 -984}\special{pa 2128 -984}\special{fp}%
\special{pa 2167 -984}\special{pa 2206 -984}\special{fp}\special{pa 2245 -984}\special{pa 2284 -984}\special{fp}%
\special{pa 2323 -984}\special{pa 2362 -984}\special{fp}%
%
}%
{%
\color[rgb]{0,0,0}%
\special{pa 1378 2362}\special{pa 1378 2323}\special{fp}\special{pa 1378 2284}\special{pa 1378 2245}\special{fp}%
\special{pa 1378 2206}\special{pa 1378 2167}\special{fp}\special{pa 1378 2128}\special{pa 1378 2089}\special{fp}%
\special{pa 1378 2050}\special{pa 1378 2011}\special{fp}\special{pa 1378 1972}\special{pa 1378 1933}\special{fp}%
\special{pa 1378 1894}\special{pa 1378 1855}\special{fp}\special{pa 1378 1816}\special{pa 1378 1777}\special{fp}%
\special{pa 1378 1737}\special{pa 1378 1698}\special{fp}\special{pa 1378 1659}\special{pa 1378 1620}\special{fp}%
\special{pa 1378 1581}\special{pa 1378 1542}\special{fp}\special{pa 1378 1503}\special{pa 1378 1464}\special{fp}%
\special{pa 1378 1425}\special{pa 1378 1386}\special{fp}\special{pa 1378 1347}\special{pa 1378 1308}\special{fp}%
\special{pa 1378 1269}\special{pa 1378 1230}\special{fp}\special{pa 1378 1191}\special{pa 1378 1152}\special{fp}%
\special{pa 1378 1113}\special{pa 1378 1074}\special{fp}\special{pa 1378 1035}\special{pa 1378 996}\special{fp}%
\special{pa 1378 957}\special{pa 1378 918}\special{fp}\special{pa 1378 879}\special{pa 1378 839}\special{fp}%
\special{pa 1378 800}\special{pa 1378 761}\special{fp}\special{pa 1378 722}\special{pa 1378 683}\special{fp}%
\special{pa 1378 644}\special{pa 1378 605}\special{fp}\special{pa 1378 566}\special{pa 1378 527}\special{fp}%
\special{pa 1378 488}\special{pa 1378 449}\special{fp}\special{pa 1378 410}\special{pa 1378 371}\special{fp}%
\special{pa 1378 332}\special{pa 1378 293}\special{fp}\special{pa 1378 254}\special{pa 1378 215}\special{fp}%
\special{pa 1378 176}\special{pa 1378 137}\special{fp}\special{pa 1378 98}\special{pa 1378 59}\special{fp}%
\special{pa 1378 20}\special{pa 1378 -20}\special{fp}\special{pa 1378 -59}\special{pa 1378 -98}\special{fp}%
\special{pa 1378 -137}\special{pa 1378 -176}\special{fp}\special{pa 1378 -215}\special{pa 1378 -254}\special{fp}%
\special{pa 1378 -293}\special{pa 1378 -332}\special{fp}\special{pa 1378 -371}\special{pa 1378 -410}\special{fp}%
\special{pa 1378 -449}\special{pa 1378 -488}\special{fp}\special{pa 1378 -527}\special{pa 1378 -566}\special{fp}%
\special{pa 1378 -605}\special{pa 1378 -644}\special{fp}\special{pa 1378 -683}\special{pa 1378 -722}\special{fp}%
\special{pa 1378 -761}\special{pa 1378 -800}\special{fp}\special{pa 1378 -839}\special{pa 1378 -879}\special{fp}%
\special{pa 1378 -918}\special{pa 1378 -957}\special{fp}\special{pa 1378 -996}\special{pa 1378 -1035}\special{fp}%
\special{pa 1378 -1074}\special{pa 1378 -1113}\special{fp}\special{pa 1378 -1152}\special{pa 1378 -1191}\special{fp}%
\special{pa 1378 -1230}\special{pa 1378 -1269}\special{fp}\special{pa 1378 -1308}\special{pa 1378 -1347}\special{fp}%
\special{pa 1378 -1386}\special{pa 1378 -1425}\special{fp}\special{pa 1378 -1464}\special{pa 1378 -1503}\special{fp}%
\special{pa 1378 -1542}\special{pa 1378 -1581}\special{fp}\special{pa 1378 -1620}\special{pa 1378 -1659}\special{fp}%
\special{pa 1378 -1698}\special{pa 1378 -1737}\special{fp}\special{pa 1378 -1777}\special{pa 1378 -1816}\special{fp}%
\special{pa 1378 -1855}\special{pa 1378 -1894}\special{fp}\special{pa 1378 -1933}\special{pa 1378 -1972}\special{fp}%
\special{pa 1378 -2011}\special{pa 1378 -2050}\special{fp}\special{pa 1378 -2089}\special{pa 1378 -2128}\special{fp}%
\special{pa 1378 -2167}\special{pa 1378 -2206}\special{fp}\special{pa 1378 -2245}\special{pa 1378 -2284}\special{fp}%
\special{pa 1378 -2323}\special{pa 1378 -2362}\special{fp}%
%
}%
{%
\color[rgb]{0,0,0}%
\special{pa -2362 -1378}\special{pa -2323 -1378}\special{fp}\special{pa -2284 -1378}\special{pa -2245 -1378}\special{fp}%
\special{pa -2206 -1378}\special{pa -2167 -1378}\special{fp}\special{pa -2128 -1378}\special{pa -2089 -1378}\special{fp}%
\special{pa -2050 -1378}\special{pa -2011 -1378}\special{fp}\special{pa -1972 -1378}\special{pa -1933 -1378}\special{fp}%
\special{pa -1894 -1378}\special{pa -1855 -1378}\special{fp}\special{pa -1816 -1378}\special{pa -1777 -1378}\special{fp}%
\special{pa -1737 -1378}\special{pa -1698 -1378}\special{fp}\special{pa -1659 -1378}\special{pa -1620 -1378}\special{fp}%
\special{pa -1581 -1378}\special{pa -1542 -1378}\special{fp}\special{pa -1503 -1378}\special{pa -1464 -1378}\special{fp}%
\special{pa -1425 -1378}\special{pa -1386 -1378}\special{fp}\special{pa -1347 -1378}\special{pa -1308 -1378}\special{fp}%
\special{pa -1269 -1378}\special{pa -1230 -1378}\special{fp}\special{pa -1191 -1378}\special{pa -1152 -1378}\special{fp}%
\special{pa -1113 -1378}\special{pa -1074 -1378}\special{fp}\special{pa -1035 -1378}\special{pa -996 -1378}\special{fp}%
\special{pa -957 -1378}\special{pa -918 -1378}\special{fp}\special{pa -879 -1378}\special{pa -839 -1378}\special{fp}%
\special{pa -800 -1378}\special{pa -761 -1378}\special{fp}\special{pa -722 -1378}\special{pa -683 -1378}\special{fp}%
\special{pa -644 -1378}\special{pa -605 -1378}\special{fp}\special{pa -566 -1378}\special{pa -527 -1378}\special{fp}%
\special{pa -488 -1378}\special{pa -449 -1378}\special{fp}\special{pa -410 -1378}\special{pa -371 -1378}\special{fp}%
\special{pa -332 -1378}\special{pa -293 -1378}\special{fp}\special{pa -254 -1378}\special{pa -215 -1378}\special{fp}%
\special{pa -176 -1378}\special{pa -137 -1378}\special{fp}\special{pa -98 -1378}\special{pa -59 -1378}\special{fp}%
\special{pa -20 -1378}\special{pa 20 -1378}\special{fp}\special{pa 59 -1378}\special{pa 98 -1378}\special{fp}%
\special{pa 137 -1378}\special{pa 176 -1378}\special{fp}\special{pa 215 -1378}\special{pa 254 -1378}\special{fp}%
\special{pa 293 -1378}\special{pa 332 -1378}\special{fp}\special{pa 371 -1378}\special{pa 410 -1378}\special{fp}%
\special{pa 449 -1378}\special{pa 488 -1378}\special{fp}\special{pa 527 -1378}\special{pa 566 -1378}\special{fp}%
\special{pa 605 -1378}\special{pa 644 -1378}\special{fp}\special{pa 683 -1378}\special{pa 722 -1378}\special{fp}%
\special{pa 761 -1378}\special{pa 800 -1378}\special{fp}\special{pa 839 -1378}\special{pa 879 -1378}\special{fp}%
\special{pa 918 -1378}\special{pa 957 -1378}\special{fp}\special{pa 996 -1378}\special{pa 1035 -1378}\special{fp}%
\special{pa 1074 -1378}\special{pa 1113 -1378}\special{fp}\special{pa 1152 -1378}\special{pa 1191 -1378}\special{fp}%
\special{pa 1230 -1378}\special{pa 1269 -1378}\special{fp}\special{pa 1308 -1378}\special{pa 1347 -1378}\special{fp}%
\special{pa 1386 -1378}\special{pa 1425 -1378}\special{fp}\special{pa 1464 -1378}\special{pa 1503 -1378}\special{fp}%
\special{pa 1542 -1378}\special{pa 1581 -1378}\special{fp}\special{pa 1620 -1378}\special{pa 1659 -1378}\special{fp}%
\special{pa 1698 -1378}\special{pa 1737 -1378}\special{fp}\special{pa 1777 -1378}\special{pa 1816 -1378}\special{fp}%
\special{pa 1855 -1378}\special{pa 1894 -1378}\special{fp}\special{pa 1933 -1378}\special{pa 1972 -1378}\special{fp}%
\special{pa 2011 -1378}\special{pa 2050 -1378}\special{fp}\special{pa 2089 -1378}\special{pa 2128 -1378}\special{fp}%
\special{pa 2167 -1378}\special{pa 2206 -1378}\special{fp}\special{pa 2245 -1378}\special{pa 2284 -1378}\special{fp}%
\special{pa 2323 -1378}\special{pa 2362 -1378}\special{fp}%
%
}%
{%
\color[rgb]{0,0,0}%
\special{pa 1772 2362}\special{pa 1772 2323}\special{fp}\special{pa 1772 2284}\special{pa 1772 2245}\special{fp}%
\special{pa 1772 2206}\special{pa 1772 2167}\special{fp}\special{pa 1772 2128}\special{pa 1772 2089}\special{fp}%
\special{pa 1772 2050}\special{pa 1772 2011}\special{fp}\special{pa 1772 1972}\special{pa 1772 1933}\special{fp}%
\special{pa 1772 1894}\special{pa 1772 1855}\special{fp}\special{pa 1772 1816}\special{pa 1772 1777}\special{fp}%
\special{pa 1772 1737}\special{pa 1772 1698}\special{fp}\special{pa 1772 1659}\special{pa 1772 1620}\special{fp}%
\special{pa 1772 1581}\special{pa 1772 1542}\special{fp}\special{pa 1772 1503}\special{pa 1772 1464}\special{fp}%
\special{pa 1772 1425}\special{pa 1772 1386}\special{fp}\special{pa 1772 1347}\special{pa 1772 1308}\special{fp}%
\special{pa 1772 1269}\special{pa 1772 1230}\special{fp}\special{pa 1772 1191}\special{pa 1772 1152}\special{fp}%
\special{pa 1772 1113}\special{pa 1772 1074}\special{fp}\special{pa 1772 1035}\special{pa 1772 996}\special{fp}%
\special{pa 1772 957}\special{pa 1772 918}\special{fp}\special{pa 1772 879}\special{pa 1772 839}\special{fp}%
\special{pa 1772 800}\special{pa 1772 761}\special{fp}\special{pa 1772 722}\special{pa 1772 683}\special{fp}%
\special{pa 1772 644}\special{pa 1772 605}\special{fp}\special{pa 1772 566}\special{pa 1772 527}\special{fp}%
\special{pa 1772 488}\special{pa 1772 449}\special{fp}\special{pa 1772 410}\special{pa 1772 371}\special{fp}%
\special{pa 1772 332}\special{pa 1772 293}\special{fp}\special{pa 1772 254}\special{pa 1772 215}\special{fp}%
\special{pa 1772 176}\special{pa 1772 137}\special{fp}\special{pa 1772 98}\special{pa 1772 59}\special{fp}%
\special{pa 1772 20}\special{pa 1772 -20}\special{fp}\special{pa 1772 -59}\special{pa 1772 -98}\special{fp}%
\special{pa 1772 -137}\special{pa 1772 -176}\special{fp}\special{pa 1772 -215}\special{pa 1772 -254}\special{fp}%
\special{pa 1772 -293}\special{pa 1772 -332}\special{fp}\special{pa 1772 -371}\special{pa 1772 -410}\special{fp}%
\special{pa 1772 -449}\special{pa 1772 -488}\special{fp}\special{pa 1772 -527}\special{pa 1772 -566}\special{fp}%
\special{pa 1772 -605}\special{pa 1772 -644}\special{fp}\special{pa 1772 -683}\special{pa 1772 -722}\special{fp}%
\special{pa 1772 -761}\special{pa 1772 -800}\special{fp}\special{pa 1772 -839}\special{pa 1772 -879}\special{fp}%
\special{pa 1772 -918}\special{pa 1772 -957}\special{fp}\special{pa 1772 -996}\special{pa 1772 -1035}\special{fp}%
\special{pa 1772 -1074}\special{pa 1772 -1113}\special{fp}\special{pa 1772 -1152}\special{pa 1772 -1191}\special{fp}%
\special{pa 1772 -1230}\special{pa 1772 -1269}\special{fp}\special{pa 1772 -1308}\special{pa 1772 -1347}\special{fp}%
\special{pa 1772 -1386}\special{pa 1772 -1425}\special{fp}\special{pa 1772 -1464}\special{pa 1772 -1503}\special{fp}%
\special{pa 1772 -1542}\special{pa 1772 -1581}\special{fp}\special{pa 1772 -1620}\special{pa 1772 -1659}\special{fp}%
\special{pa 1772 -1698}\special{pa 1772 -1737}\special{fp}\special{pa 1772 -1777}\special{pa 1772 -1816}\special{fp}%
\special{pa 1772 -1855}\special{pa 1772 -1894}\special{fp}\special{pa 1772 -1933}\special{pa 1772 -1972}\special{fp}%
\special{pa 1772 -2011}\special{pa 1772 -2050}\special{fp}\special{pa 1772 -2089}\special{pa 1772 -2128}\special{fp}%
\special{pa 1772 -2167}\special{pa 1772 -2206}\special{fp}\special{pa 1772 -2245}\special{pa 1772 -2284}\special{fp}%
\special{pa 1772 -2323}\special{pa 1772 -2362}\special{fp}%
%
}%
{%
\color[rgb]{0,0,0}%
\special{pa -2362 -1772}\special{pa -2323 -1772}\special{fp}\special{pa -2284 -1772}\special{pa -2245 -1772}\special{fp}%
\special{pa -2206 -1772}\special{pa -2167 -1772}\special{fp}\special{pa -2128 -1772}\special{pa -2089 -1772}\special{fp}%
\special{pa -2050 -1772}\special{pa -2011 -1772}\special{fp}\special{pa -1972 -1772}\special{pa -1933 -1772}\special{fp}%
\special{pa -1894 -1772}\special{pa -1855 -1772}\special{fp}\special{pa -1816 -1772}\special{pa -1777 -1772}\special{fp}%
\special{pa -1737 -1772}\special{pa -1698 -1772}\special{fp}\special{pa -1659 -1772}\special{pa -1620 -1772}\special{fp}%
\special{pa -1581 -1772}\special{pa -1542 -1772}\special{fp}\special{pa -1503 -1772}\special{pa -1464 -1772}\special{fp}%
\special{pa -1425 -1772}\special{pa -1386 -1772}\special{fp}\special{pa -1347 -1772}\special{pa -1308 -1772}\special{fp}%
\special{pa -1269 -1772}\special{pa -1230 -1772}\special{fp}\special{pa -1191 -1772}\special{pa -1152 -1772}\special{fp}%
\special{pa -1113 -1772}\special{pa -1074 -1772}\special{fp}\special{pa -1035 -1772}\special{pa -996 -1772}\special{fp}%
\special{pa -957 -1772}\special{pa -918 -1772}\special{fp}\special{pa -879 -1772}\special{pa -839 -1772}\special{fp}%
\special{pa -800 -1772}\special{pa -761 -1772}\special{fp}\special{pa -722 -1772}\special{pa -683 -1772}\special{fp}%
\special{pa -644 -1772}\special{pa -605 -1772}\special{fp}\special{pa -566 -1772}\special{pa -527 -1772}\special{fp}%
\special{pa -488 -1772}\special{pa -449 -1772}\special{fp}\special{pa -410 -1772}\special{pa -371 -1772}\special{fp}%
\special{pa -332 -1772}\special{pa -293 -1772}\special{fp}\special{pa -254 -1772}\special{pa -215 -1772}\special{fp}%
\special{pa -176 -1772}\special{pa -137 -1772}\special{fp}\special{pa -98 -1772}\special{pa -59 -1772}\special{fp}%
\special{pa -20 -1772}\special{pa 20 -1772}\special{fp}\special{pa 59 -1772}\special{pa 98 -1772}\special{fp}%
\special{pa 137 -1772}\special{pa 176 -1772}\special{fp}\special{pa 215 -1772}\special{pa 254 -1772}\special{fp}%
\special{pa 293 -1772}\special{pa 332 -1772}\special{fp}\special{pa 371 -1772}\special{pa 410 -1772}\special{fp}%
\special{pa 449 -1772}\special{pa 488 -1772}\special{fp}\special{pa 527 -1772}\special{pa 566 -1772}\special{fp}%
\special{pa 605 -1772}\special{pa 644 -1772}\special{fp}\special{pa 683 -1772}\special{pa 722 -1772}\special{fp}%
\special{pa 761 -1772}\special{pa 800 -1772}\special{fp}\special{pa 839 -1772}\special{pa 879 -1772}\special{fp}%
\special{pa 918 -1772}\special{pa 957 -1772}\special{fp}\special{pa 996 -1772}\special{pa 1035 -1772}\special{fp}%
\special{pa 1074 -1772}\special{pa 1113 -1772}\special{fp}\special{pa 1152 -1772}\special{pa 1191 -1772}\special{fp}%
\special{pa 1230 -1772}\special{pa 1269 -1772}\special{fp}\special{pa 1308 -1772}\special{pa 1347 -1772}\special{fp}%
\special{pa 1386 -1772}\special{pa 1425 -1772}\special{fp}\special{pa 1464 -1772}\special{pa 1503 -1772}\special{fp}%
\special{pa 1542 -1772}\special{pa 1581 -1772}\special{fp}\special{pa 1620 -1772}\special{pa 1659 -1772}\special{fp}%
\special{pa 1698 -1772}\special{pa 1737 -1772}\special{fp}\special{pa 1777 -1772}\special{pa 1816 -1772}\special{fp}%
\special{pa 1855 -1772}\special{pa 1894 -1772}\special{fp}\special{pa 1933 -1772}\special{pa 1972 -1772}\special{fp}%
\special{pa 2011 -1772}\special{pa 2050 -1772}\special{fp}\special{pa 2089 -1772}\special{pa 2128 -1772}\special{fp}%
\special{pa 2167 -1772}\special{pa 2206 -1772}\special{fp}\special{pa 2245 -1772}\special{pa 2284 -1772}\special{fp}%
\special{pa 2323 -1772}\special{pa 2362 -1772}\special{fp}%
%
}%
{%
\color[rgb]{0,0,0}%
\special{pa 2165 2362}\special{pa 2165 2323}\special{fp}\special{pa 2165 2284}\special{pa 2165 2245}\special{fp}%
\special{pa 2165 2206}\special{pa 2165 2167}\special{fp}\special{pa 2165 2128}\special{pa 2165 2089}\special{fp}%
\special{pa 2165 2050}\special{pa 2165 2011}\special{fp}\special{pa 2165 1972}\special{pa 2165 1933}\special{fp}%
\special{pa 2165 1894}\special{pa 2165 1855}\special{fp}\special{pa 2165 1816}\special{pa 2165 1777}\special{fp}%
\special{pa 2165 1737}\special{pa 2165 1698}\special{fp}\special{pa 2165 1659}\special{pa 2165 1620}\special{fp}%
\special{pa 2165 1581}\special{pa 2165 1542}\special{fp}\special{pa 2165 1503}\special{pa 2165 1464}\special{fp}%
\special{pa 2165 1425}\special{pa 2165 1386}\special{fp}\special{pa 2165 1347}\special{pa 2165 1308}\special{fp}%
\special{pa 2165 1269}\special{pa 2165 1230}\special{fp}\special{pa 2165 1191}\special{pa 2165 1152}\special{fp}%
\special{pa 2165 1113}\special{pa 2165 1074}\special{fp}\special{pa 2165 1035}\special{pa 2165 996}\special{fp}%
\special{pa 2165 957}\special{pa 2165 918}\special{fp}\special{pa 2165 879}\special{pa 2165 839}\special{fp}%
\special{pa 2165 800}\special{pa 2165 761}\special{fp}\special{pa 2165 722}\special{pa 2165 683}\special{fp}%
\special{pa 2165 644}\special{pa 2165 605}\special{fp}\special{pa 2165 566}\special{pa 2165 527}\special{fp}%
\special{pa 2165 488}\special{pa 2165 449}\special{fp}\special{pa 2165 410}\special{pa 2165 371}\special{fp}%
\special{pa 2165 332}\special{pa 2165 293}\special{fp}\special{pa 2165 254}\special{pa 2165 215}\special{fp}%
\special{pa 2165 176}\special{pa 2165 137}\special{fp}\special{pa 2165 98}\special{pa 2165 59}\special{fp}%
\special{pa 2165 20}\special{pa 2165 -20}\special{fp}\special{pa 2165 -59}\special{pa 2165 -98}\special{fp}%
\special{pa 2165 -137}\special{pa 2165 -176}\special{fp}\special{pa 2165 -215}\special{pa 2165 -254}\special{fp}%
\special{pa 2165 -293}\special{pa 2165 -332}\special{fp}\special{pa 2165 -371}\special{pa 2165 -410}\special{fp}%
\special{pa 2165 -449}\special{pa 2165 -488}\special{fp}\special{pa 2165 -527}\special{pa 2165 -566}\special{fp}%
\special{pa 2165 -605}\special{pa 2165 -644}\special{fp}\special{pa 2165 -683}\special{pa 2165 -722}\special{fp}%
\special{pa 2165 -761}\special{pa 2165 -800}\special{fp}\special{pa 2165 -839}\special{pa 2165 -879}\special{fp}%
\special{pa 2165 -918}\special{pa 2165 -957}\special{fp}\special{pa 2165 -996}\special{pa 2165 -1035}\special{fp}%
\special{pa 2165 -1074}\special{pa 2165 -1113}\special{fp}\special{pa 2165 -1152}\special{pa 2165 -1191}\special{fp}%
\special{pa 2165 -1230}\special{pa 2165 -1269}\special{fp}\special{pa 2165 -1308}\special{pa 2165 -1347}\special{fp}%
\special{pa 2165 -1386}\special{pa 2165 -1425}\special{fp}\special{pa 2165 -1464}\special{pa 2165 -1503}\special{fp}%
\special{pa 2165 -1542}\special{pa 2165 -1581}\special{fp}\special{pa 2165 -1620}\special{pa 2165 -1659}\special{fp}%
\special{pa 2165 -1698}\special{pa 2165 -1737}\special{fp}\special{pa 2165 -1777}\special{pa 2165 -1816}\special{fp}%
\special{pa 2165 -1855}\special{pa 2165 -1894}\special{fp}\special{pa 2165 -1933}\special{pa 2165 -1972}\special{fp}%
\special{pa 2165 -2011}\special{pa 2165 -2050}\special{fp}\special{pa 2165 -2089}\special{pa 2165 -2128}\special{fp}%
\special{pa 2165 -2167}\special{pa 2165 -2206}\special{fp}\special{pa 2165 -2245}\special{pa 2165 -2284}\special{fp}%
\special{pa 2165 -2323}\special{pa 2165 -2362}\special{fp}%
%
}%
{%
\color[rgb]{0,0,0}%
\special{pa -2362 -2165}\special{pa -2323 -2165}\special{fp}\special{pa -2284 -2165}\special{pa -2245 -2165}\special{fp}%
\special{pa -2206 -2165}\special{pa -2167 -2165}\special{fp}\special{pa -2128 -2165}\special{pa -2089 -2165}\special{fp}%
\special{pa -2050 -2165}\special{pa -2011 -2165}\special{fp}\special{pa -1972 -2165}\special{pa -1933 -2165}\special{fp}%
\special{pa -1894 -2165}\special{pa -1855 -2165}\special{fp}\special{pa -1816 -2165}\special{pa -1777 -2165}\special{fp}%
\special{pa -1737 -2165}\special{pa -1698 -2165}\special{fp}\special{pa -1659 -2165}\special{pa -1620 -2165}\special{fp}%
\special{pa -1581 -2165}\special{pa -1542 -2165}\special{fp}\special{pa -1503 -2165}\special{pa -1464 -2165}\special{fp}%
\special{pa -1425 -2165}\special{pa -1386 -2165}\special{fp}\special{pa -1347 -2165}\special{pa -1308 -2165}\special{fp}%
\special{pa -1269 -2165}\special{pa -1230 -2165}\special{fp}\special{pa -1191 -2165}\special{pa -1152 -2165}\special{fp}%
\special{pa -1113 -2165}\special{pa -1074 -2165}\special{fp}\special{pa -1035 -2165}\special{pa -996 -2165}\special{fp}%
\special{pa -957 -2165}\special{pa -918 -2165}\special{fp}\special{pa -879 -2165}\special{pa -839 -2165}\special{fp}%
\special{pa -800 -2165}\special{pa -761 -2165}\special{fp}\special{pa -722 -2165}\special{pa -683 -2165}\special{fp}%
\special{pa -644 -2165}\special{pa -605 -2165}\special{fp}\special{pa -566 -2165}\special{pa -527 -2165}\special{fp}%
\special{pa -488 -2165}\special{pa -449 -2165}\special{fp}\special{pa -410 -2165}\special{pa -371 -2165}\special{fp}%
\special{pa -332 -2165}\special{pa -293 -2165}\special{fp}\special{pa -254 -2165}\special{pa -215 -2165}\special{fp}%
\special{pa -176 -2165}\special{pa -137 -2165}\special{fp}\special{pa -98 -2165}\special{pa -59 -2165}\special{fp}%
\special{pa -20 -2165}\special{pa 20 -2165}\special{fp}\special{pa 59 -2165}\special{pa 98 -2165}\special{fp}%
\special{pa 137 -2165}\special{pa 176 -2165}\special{fp}\special{pa 215 -2165}\special{pa 254 -2165}\special{fp}%
\special{pa 293 -2165}\special{pa 332 -2165}\special{fp}\special{pa 371 -2165}\special{pa 410 -2165}\special{fp}%
\special{pa 449 -2165}\special{pa 488 -2165}\special{fp}\special{pa 527 -2165}\special{pa 566 -2165}\special{fp}%
\special{pa 605 -2165}\special{pa 644 -2165}\special{fp}\special{pa 683 -2165}\special{pa 722 -2165}\special{fp}%
\special{pa 761 -2165}\special{pa 800 -2165}\special{fp}\special{pa 839 -2165}\special{pa 879 -2165}\special{fp}%
\special{pa 918 -2165}\special{pa 957 -2165}\special{fp}\special{pa 996 -2165}\special{pa 1035 -2165}\special{fp}%
\special{pa 1074 -2165}\special{pa 1113 -2165}\special{fp}\special{pa 1152 -2165}\special{pa 1191 -2165}\special{fp}%
\special{pa 1230 -2165}\special{pa 1269 -2165}\special{fp}\special{pa 1308 -2165}\special{pa 1347 -2165}\special{fp}%
\special{pa 1386 -2165}\special{pa 1425 -2165}\special{fp}\special{pa 1464 -2165}\special{pa 1503 -2165}\special{fp}%
\special{pa 1542 -2165}\special{pa 1581 -2165}\special{fp}\special{pa 1620 -2165}\special{pa 1659 -2165}\special{fp}%
\special{pa 1698 -2165}\special{pa 1737 -2165}\special{fp}\special{pa 1777 -2165}\special{pa 1816 -2165}\special{fp}%
\special{pa 1855 -2165}\special{pa 1894 -2165}\special{fp}\special{pa 1933 -2165}\special{pa 1972 -2165}\special{fp}%
\special{pa 2011 -2165}\special{pa 2050 -2165}\special{fp}\special{pa 2089 -2165}\special{pa 2128 -2165}\special{fp}%
\special{pa 2167 -2165}\special{pa 2206 -2165}\special{fp}\special{pa 2245 -2165}\special{pa 2284 -2165}\special{fp}%
\special{pa 2323 -2165}\special{pa 2362 -2165}\special{fp}%
%
}%
\special{pn 8}%
{%
\color[rgb]{0,0,0}%
\special{pa  2362   -20}\special{pa  2362    20}%
\special{fp}%
}%
{%
\color[rgb]{0,0,0}%
\settowidth{\Width}{$-6$}\setlength{\Width}{-0.5\Width}%
\settoheight{\Height}{$-6$}\settodepth{\Depth}{$-6$}\setlength{\Height}{-\Height}%
\put(-6.0000000,-0.1000000){\hspace*{\Width}\raisebox{\Height}{$-6$}}%
%
}%
{%
\color[rgb]{0,0,0}%
\special{pa    20 -2362}\special{pa   -20 -2362}%
\special{fp}%
}%
{%
\color[rgb]{0,0,0}%
\settowidth{\Width}{$-6$}\setlength{\Width}{-1\Width}%
\settoheight{\Height}{$-6$}\settodepth{\Depth}{$-6$}\setlength{\Height}{-0.5\Height}\setlength{\Depth}{0.5\Depth}\addtolength{\Height}{\Depth}%
\put(-0.1000000,-6.0000000){\hspace*{\Width}\raisebox{\Height}{$-6$}}%
%
}%
{%
\color[rgb]{0,0,0}%
\special{pa  2362   -20}\special{pa  2362    20}%
\special{fp}%
}%
{%
\color[rgb]{0,0,0}%
\settowidth{\Width}{$-5$}\setlength{\Width}{-0.5\Width}%
\settoheight{\Height}{$-5$}\settodepth{\Depth}{$-5$}\setlength{\Height}{-\Height}%
\put(-5.0000000,-0.1000000){\hspace*{\Width}\raisebox{\Height}{$-5$}}%
%
}%
{%
\color[rgb]{0,0,0}%
\special{pa    20 -2362}\special{pa   -20 -2362}%
\special{fp}%
}%
{%
\color[rgb]{0,0,0}%
\settowidth{\Width}{$-5$}\setlength{\Width}{-1\Width}%
\settoheight{\Height}{$-5$}\settodepth{\Depth}{$-5$}\setlength{\Height}{-0.5\Height}\setlength{\Depth}{0.5\Depth}\addtolength{\Height}{\Depth}%
\put(-0.1000000,-5.0000000){\hspace*{\Width}\raisebox{\Height}{$-5$}}%
%
}%
{%
\color[rgb]{0,0,0}%
\special{pa  2362   -20}\special{pa  2362    20}%
\special{fp}%
}%
{%
\color[rgb]{0,0,0}%
\settowidth{\Width}{$-4$}\setlength{\Width}{-0.5\Width}%
\settoheight{\Height}{$-4$}\settodepth{\Depth}{$-4$}\setlength{\Height}{-\Height}%
\put(-4.0000000,-0.1000000){\hspace*{\Width}\raisebox{\Height}{$-4$}}%
%
}%
{%
\color[rgb]{0,0,0}%
\special{pa    20 -2362}\special{pa   -20 -2362}%
\special{fp}%
}%
{%
\color[rgb]{0,0,0}%
\settowidth{\Width}{$-4$}\setlength{\Width}{-1\Width}%
\settoheight{\Height}{$-4$}\settodepth{\Depth}{$-4$}\setlength{\Height}{-0.5\Height}\setlength{\Depth}{0.5\Depth}\addtolength{\Height}{\Depth}%
\put(-0.1000000,-4.0000000){\hspace*{\Width}\raisebox{\Height}{$-4$}}%
%
}%
{%
\color[rgb]{0,0,0}%
\special{pa  2362   -20}\special{pa  2362    20}%
\special{fp}%
}%
{%
\color[rgb]{0,0,0}%
\settowidth{\Width}{$-3$}\setlength{\Width}{-0.5\Width}%
\settoheight{\Height}{$-3$}\settodepth{\Depth}{$-3$}\setlength{\Height}{-\Height}%
\put(-3.0000000,-0.1000000){\hspace*{\Width}\raisebox{\Height}{$-3$}}%
%
}%
{%
\color[rgb]{0,0,0}%
\special{pa    20 -2362}\special{pa   -20 -2362}%
\special{fp}%
}%
{%
\color[rgb]{0,0,0}%
\settowidth{\Width}{$-3$}\setlength{\Width}{-1\Width}%
\settoheight{\Height}{$-3$}\settodepth{\Depth}{$-3$}\setlength{\Height}{-0.5\Height}\setlength{\Depth}{0.5\Depth}\addtolength{\Height}{\Depth}%
\put(-0.1000000,-3.0000000){\hspace*{\Width}\raisebox{\Height}{$-3$}}%
%
}%
{%
\color[rgb]{0,0,0}%
\special{pa  2362   -20}\special{pa  2362    20}%
\special{fp}%
}%
{%
\color[rgb]{0,0,0}%
\settowidth{\Width}{$-2$}\setlength{\Width}{-0.5\Width}%
\settoheight{\Height}{$-2$}\settodepth{\Depth}{$-2$}\setlength{\Height}{-\Height}%
\put(-2.0000000,-0.1000000){\hspace*{\Width}\raisebox{\Height}{$-2$}}%
%
}%
{%
\color[rgb]{0,0,0}%
\special{pa    20 -2362}\special{pa   -20 -2362}%
\special{fp}%
}%
{%
\color[rgb]{0,0,0}%
\settowidth{\Width}{$-2$}\setlength{\Width}{-1\Width}%
\settoheight{\Height}{$-2$}\settodepth{\Depth}{$-2$}\setlength{\Height}{-0.5\Height}\setlength{\Depth}{0.5\Depth}\addtolength{\Height}{\Depth}%
\put(-0.1000000,-2.0000000){\hspace*{\Width}\raisebox{\Height}{$-2$}}%
%
}%
{%
\color[rgb]{0,0,0}%
\special{pa  2362   -20}\special{pa  2362    20}%
\special{fp}%
}%
{%
\color[rgb]{0,0,0}%
\settowidth{\Width}{$-1$}\setlength{\Width}{-0.5\Width}%
\settoheight{\Height}{$-1$}\settodepth{\Depth}{$-1$}\setlength{\Height}{-\Height}%
\put(-1.0000000,-0.1000000){\hspace*{\Width}\raisebox{\Height}{$-1$}}%
%
}%
{%
\color[rgb]{0,0,0}%
\special{pa    20 -2362}\special{pa   -20 -2362}%
\special{fp}%
}%
{%
\color[rgb]{0,0,0}%
\settowidth{\Width}{$-1$}\setlength{\Width}{-1\Width}%
\settoheight{\Height}{$-1$}\settodepth{\Depth}{$-1$}\setlength{\Height}{-0.5\Height}\setlength{\Depth}{0.5\Depth}\addtolength{\Height}{\Depth}%
\put(-0.1000000,-1.0000000){\hspace*{\Width}\raisebox{\Height}{$-1$}}%
%
}%
{%
\color[rgb]{0,0,0}%
\special{pa  2362   -20}\special{pa  2362    20}%
\special{fp}%
}%
{%
\color[rgb]{0,0,0}%
\settowidth{\Width}{$1$}\setlength{\Width}{-0.5\Width}%
\settoheight{\Height}{$1$}\settodepth{\Depth}{$1$}\setlength{\Height}{-\Height}%
\put(1.0000000,-0.1000000){\hspace*{\Width}\raisebox{\Height}{$1$}}%
%
}%
{%
\color[rgb]{0,0,0}%
\special{pa    20 -2362}\special{pa   -20 -2362}%
\special{fp}%
}%
{%
\color[rgb]{0,0,0}%
\settowidth{\Width}{$1$}\setlength{\Width}{-1\Width}%
\settoheight{\Height}{$1$}\settodepth{\Depth}{$1$}\setlength{\Height}{-0.5\Height}\setlength{\Depth}{0.5\Depth}\addtolength{\Height}{\Depth}%
\put(-0.1000000,1.0000000){\hspace*{\Width}\raisebox{\Height}{$1$}}%
%
}%
{%
\color[rgb]{0,0,0}%
\special{pa  2362   -20}\special{pa  2362    20}%
\special{fp}%
}%
{%
\color[rgb]{0,0,0}%
\settowidth{\Width}{$2$}\setlength{\Width}{-0.5\Width}%
\settoheight{\Height}{$2$}\settodepth{\Depth}{$2$}\setlength{\Height}{-\Height}%
\put(2.0000000,-0.1000000){\hspace*{\Width}\raisebox{\Height}{$2$}}%
%
}%
{%
\color[rgb]{0,0,0}%
\special{pa    20 -2362}\special{pa   -20 -2362}%
\special{fp}%
}%
{%
\color[rgb]{0,0,0}%
\settowidth{\Width}{$2$}\setlength{\Width}{-1\Width}%
\settoheight{\Height}{$2$}\settodepth{\Depth}{$2$}\setlength{\Height}{-0.5\Height}\setlength{\Depth}{0.5\Depth}\addtolength{\Height}{\Depth}%
\put(-0.1000000,2.0000000){\hspace*{\Width}\raisebox{\Height}{$2$}}%
%
}%
{%
\color[rgb]{0,0,0}%
\special{pa  2362   -20}\special{pa  2362    20}%
\special{fp}%
}%
{%
\color[rgb]{0,0,0}%
\settowidth{\Width}{$3$}\setlength{\Width}{-0.5\Width}%
\settoheight{\Height}{$3$}\settodepth{\Depth}{$3$}\setlength{\Height}{-\Height}%
\put(3.0000000,-0.1000000){\hspace*{\Width}\raisebox{\Height}{$3$}}%
%
}%
{%
\color[rgb]{0,0,0}%
\special{pa    20 -2362}\special{pa   -20 -2362}%
\special{fp}%
}%
{%
\color[rgb]{0,0,0}%
\settowidth{\Width}{$3$}\setlength{\Width}{-1\Width}%
\settoheight{\Height}{$3$}\settodepth{\Depth}{$3$}\setlength{\Height}{-0.5\Height}\setlength{\Depth}{0.5\Depth}\addtolength{\Height}{\Depth}%
\put(-0.1000000,3.0000000){\hspace*{\Width}\raisebox{\Height}{$3$}}%
%
}%
{%
\color[rgb]{0,0,0}%
\special{pa  2362   -20}\special{pa  2362    20}%
\special{fp}%
}%
{%
\color[rgb]{0,0,0}%
\settowidth{\Width}{$4$}\setlength{\Width}{-0.5\Width}%
\settoheight{\Height}{$4$}\settodepth{\Depth}{$4$}\setlength{\Height}{-\Height}%
\put(4.0000000,-0.1000000){\hspace*{\Width}\raisebox{\Height}{$4$}}%
%
}%
{%
\color[rgb]{0,0,0}%
\special{pa    20 -2362}\special{pa   -20 -2362}%
\special{fp}%
}%
{%
\color[rgb]{0,0,0}%
\settowidth{\Width}{$4$}\setlength{\Width}{-1\Width}%
\settoheight{\Height}{$4$}\settodepth{\Depth}{$4$}\setlength{\Height}{-0.5\Height}\setlength{\Depth}{0.5\Depth}\addtolength{\Height}{\Depth}%
\put(-0.1000000,4.0000000){\hspace*{\Width}\raisebox{\Height}{$4$}}%
%
}%
{%
\color[rgb]{0,0,0}%
\special{pa  2362   -20}\special{pa  2362    20}%
\special{fp}%
}%
{%
\color[rgb]{0,0,0}%
\settowidth{\Width}{$5$}\setlength{\Width}{-0.5\Width}%
\settoheight{\Height}{$5$}\settodepth{\Depth}{$5$}\setlength{\Height}{-\Height}%
\put(5.0000000,-0.1000000){\hspace*{\Width}\raisebox{\Height}{$5$}}%
%
}%
{%
\color[rgb]{0,0,0}%
\special{pa    20 -2362}\special{pa   -20 -2362}%
\special{fp}%
}%
{%
\color[rgb]{0,0,0}%
\settowidth{\Width}{$5$}\setlength{\Width}{-1\Width}%
\settoheight{\Height}{$5$}\settodepth{\Depth}{$5$}\setlength{\Height}{-0.5\Height}\setlength{\Depth}{0.5\Depth}\addtolength{\Height}{\Depth}%
\put(-0.1000000,5.0000000){\hspace*{\Width}\raisebox{\Height}{$5$}}%
%
}%
{%
\color[rgb]{0,0,0}%
\special{pa  2362   -20}\special{pa  2362    20}%
\special{fp}%
}%
{%
\color[rgb]{0,0,0}%
\settowidth{\Width}{$6$}\setlength{\Width}{-0.5\Width}%
\settoheight{\Height}{$6$}\settodepth{\Depth}{$6$}\setlength{\Height}{-\Height}%
\put(6.0000000,-0.1000000){\hspace*{\Width}\raisebox{\Height}{$6$}}%
%
}%
{%
\color[rgb]{0,0,0}%
\special{pa    20 -2362}\special{pa   -20 -2362}%
\special{fp}%
}%
{%
\color[rgb]{0,0,0}%
\settowidth{\Width}{$6$}\setlength{\Width}{-1\Width}%
\settoheight{\Height}{$6$}\settodepth{\Depth}{$6$}\setlength{\Height}{-0.5\Height}\setlength{\Depth}{0.5\Depth}\addtolength{\Height}{\Depth}%
\put(-0.1000000,6.0000000){\hspace*{\Width}\raisebox{\Height}{$6$}}%
%
{%
\color[cmyk]{0,1,1,0}%
\special{pa -1157 1969}\special{pa -1159 1960}\special{pa -1164 1953}\special{pa -1171 1947}%
\special{pa -1179 1945}\special{pa -1188 1946}\special{pa -1195 1950}\special{pa -1201 1956}%
\special{pa -1204 1964}\special{pa -1204 1973}\special{pa -1201 1981}\special{pa -1195 1987}%
\special{pa -1188 1991}\special{pa -1179 1992}\special{pa -1171 1990}\special{pa -1164 1984}%
\special{pa -1159 1977}\special{pa -1157 1969}\special{pa -1157 1969}\special{sh 1}\special{ip}%
}%
}%
{%
\color[cmyk]{0,1,1,0}%
\special{pa -1157  1969}\special{pa -1159  1960}\special{pa -1164  1953}\special{pa -1171  1947}%
\special{pa -1179  1945}\special{pa -1188  1946}\special{pa -1195  1950}\special{pa -1201  1956}%
\special{pa -1204  1964}\special{pa -1204  1973}\special{pa -1201  1981}\special{pa -1195  1987}%
\special{pa -1188  1991}\special{pa -1179  1992}\special{pa -1171  1990}\special{pa -1164  1984}%
\special{pa -1159  1977}\special{pa -1157  1969}%
\special{fp}%
{%
\color[cmyk]{0,1,1,0}%
\special{pa -764 1181}\special{pa -765 1173}\special{pa -770 1165}\special{pa -777 1160}%
\special{pa -785 1158}\special{pa -794 1158}\special{pa -802 1162}\special{pa -807 1169}%
\special{pa -811 1177}\special{pa -811 1185}\special{pa -807 1194}\special{pa -802 1200}%
\special{pa -794 1204}\special{pa -785 1205}\special{pa -777 1202}\special{pa -770 1197}%
\special{pa -765 1190}\special{pa -764 1181}\special{pa -764 1181}\special{sh 1}\special{ip}%
}%
}%
{%
\color[cmyk]{0,1,1,0}%
\special{pa  -764  1181}\special{pa  -765  1173}\special{pa  -770  1165}\special{pa  -777  1160}%
\special{pa  -785  1158}\special{pa  -794  1158}\special{pa  -802  1162}\special{pa  -807  1169}%
\special{pa  -811  1177}\special{pa  -811  1185}\special{pa  -807  1194}\special{pa  -802  1200}%
\special{pa  -794  1204}\special{pa  -785  1205}\special{pa  -777  1202}\special{pa  -770  1197}%
\special{pa  -765  1190}\special{pa  -764  1181}%
\special{fp}%
{%
\color[cmyk]{0,1,1,0}%
\special{pa -370 394}\special{pa -372 385}\special{pa -376 378}\special{pa -383 373}%
\special{pa -392 370}\special{pa -400 371}\special{pa -408 375}\special{pa -414 381}%
\special{pa -417 389}\special{pa -417 398}\special{pa -414 406}\special{pa -408 413}%
\special{pa -400 416}\special{pa -392 417}\special{pa -383 415}\special{pa -376 410}%
\special{pa -372 402}\special{pa -370 394}\special{pa -370 394}\special{sh 1}\special{ip}%
}%
}%
{%
\color[cmyk]{0,1,1,0}%
\special{pa  -370   394}\special{pa  -372   385}\special{pa  -376   378}\special{pa  -383   373}%
\special{pa  -392   370}\special{pa  -400   371}\special{pa  -408   375}\special{pa  -414   381}%
\special{pa  -417   389}\special{pa  -417   398}\special{pa  -414   406}\special{pa  -408   413}%
\special{pa  -400   416}\special{pa  -392   417}\special{pa  -383   415}\special{pa  -376   410}%
\special{pa  -372   402}\special{pa  -370   394}%
\special{fp}%
{%
\color[cmyk]{0,1,1,0}%
\special{pa 24 -394}\special{pa 22 -402}\special{pa 17 -410}\special{pa 11 -415}\special{pa 2 -417}%
\special{pa -6 -416}\special{pa -14 -413}\special{pa -20 -406}\special{pa -23 -398}%
\special{pa -23 -389}\special{pa -20 -381}\special{pa -14 -375}\special{pa -6 -371}%
\special{pa 2 -370}\special{pa 11 -373}\special{pa 17 -378}\special{pa 22 -385}\special{pa 24 -394}%
\special{pa 24 -394}\special{sh 1}\special{ip}%
}%
}%
{%
\color[cmyk]{0,1,1,0}%
\special{pa    24  -394}\special{pa    22  -402}\special{pa    17  -410}\special{pa    11  -415}%
\special{pa     2  -417}\special{pa    -6  -416}\special{pa   -14  -413}\special{pa   -20  -406}%
\special{pa   -23  -398}\special{pa   -23  -389}\special{pa   -20  -381}\special{pa   -14  -375}%
\special{pa    -6  -371}\special{pa     2  -370}\special{pa    11  -373}\special{pa    17  -378}%
\special{pa    22  -385}\special{pa    24  -394}%
\special{fp}%
{%
\color[cmyk]{0,1,1,0}%
\special{pa 417 -1181}\special{pa 416 -1190}\special{pa 411 -1197}\special{pa 404 -1202}%
\special{pa 396 -1205}\special{pa 387 -1204}\special{pa 379 -1200}\special{pa 374 -1194}%
\special{pa 370 -1185}\special{pa 370 -1177}\special{pa 374 -1169}\special{pa 379 -1162}%
\special{pa 387 -1158}\special{pa 396 -1158}\special{pa 404 -1160}\special{pa 411 -1165}%
\special{pa 416 -1173}\special{pa 417 -1181}\special{pa 417 -1181}\special{sh 1}\special{ip}%
}%
}%
{%
\color[cmyk]{0,1,1,0}%
\special{pa   417 -1181}\special{pa   416 -1190}\special{pa   411 -1197}\special{pa   404 -1202}%
\special{pa   396 -1205}\special{pa   387 -1204}\special{pa   379 -1200}\special{pa   374 -1194}%
\special{pa   370 -1185}\special{pa   370 -1177}\special{pa   374 -1169}\special{pa   379 -1162}%
\special{pa   387 -1158}\special{pa   396 -1158}\special{pa   404 -1160}\special{pa   411 -1165}%
\special{pa   416 -1173}\special{pa   417 -1181}%
\special{fp}%
{%
\color[cmyk]{0,1,1,0}%
\special{pa 811 -1969}\special{pa 809 -1977}\special{pa 805 -1984}\special{pa 798 -1990}%
\special{pa 790 -1992}\special{pa 781 -1991}\special{pa 773 -1987}\special{pa 767 -1981}%
\special{pa 764 -1973}\special{pa 764 -1964}\special{pa 767 -1956}\special{pa 773 -1950}%
\special{pa 781 -1946}\special{pa 790 -1945}\special{pa 798 -1947}\special{pa 805 -1953}%
\special{pa 809 -1960}\special{pa 811 -1969}\special{pa 811 -1969}\special{sh 1}\special{ip}%
}%
}%
{%
\color[cmyk]{0,1,1,0}%
\special{pa   811 -1969}\special{pa   809 -1977}\special{pa   805 -1984}\special{pa   798 -1990}%
\special{pa   790 -1992}\special{pa   781 -1991}\special{pa   773 -1987}\special{pa   767 -1981}%
\special{pa   764 -1973}\special{pa   764 -1964}\special{pa   767 -1956}\special{pa   773 -1950}%
\special{pa   781 -1946}\special{pa   790 -1945}\special{pa   798 -1947}\special{pa   805 -1953}%
\special{pa   809 -1960}\special{pa   811 -1969}%
\special{fp}%
}%
{%
\color[cmyk]{0,1,1,0}%
\special{pa -1378  2362}\special{pa   984 -2362}%
\special{fp}%
}%
\special{pa -2441    -0}\special{pa  2441    -0}%
\special{fp}%
\special{pa     0  2441}\special{pa     0 -2441}%
\special{fp}%
\settowidth{\Width}{$x$}\setlength{\Width}{0\Width}%
\settoheight{\Height}{$x$}\settodepth{\Depth}{$x$}\setlength{\Height}{-0.5\Height}\setlength{\Depth}{0.5\Depth}\addtolength{\Height}{\Depth}%
\put(6.2500000,0.0000000){\hspace*{\Width}\raisebox{\Height}{$x$}}%
%
\settowidth{\Width}{$y$}\setlength{\Width}{-0.5\Width}%
\settoheight{\Height}{$y$}\settodepth{\Depth}{$y$}\setlength{\Height}{\Depth}%
\put(0.0000000,6.2500000){\hspace*{\Width}\raisebox{\Height}{$y$}}%
%
\settowidth{\Width}{O}\setlength{\Width}{-1\Width}%
\settoheight{\Height}{O}\settodepth{\Depth}{O}\setlength{\Height}{-\Height}%
\put(-0.0500000,-0.0500000){\hspace*{\Width}\raisebox{\Height}{O}}%
%
\end{picture}}%}}
\putnotese{100}{30}{\color{blue}傾き\ \ $2$}
\putnotese{100}{36}{\color{blue}$y$切片\ $1$}
\end{layer}


\newslide{1次関数のグラフ}

\vspace*{18mm}

%%repeat=3
\slidepage
\seteda{90}
\begin{itemize}
\item
「2.関数のグラフ」を用いて,次の1次関数のグラフをかこう.
\\
\eda{$y=3x+3$} \\
\eda{$y=10-2x$} \\
\eda{$y=2x+2$} \\
\eda{$y=\bunsuu{1}{2}x+1$}
\item
[課題]\monban 傾きとy切片を答えよ.
\end{itemize}
%%%%%%%%%%%%

%%%%%%%%%%%%%%%%%%%%

\newslide{2次関数のグラフ(基本形)}

\vspace*{18mm}

\slidepage
\down
「2.関数のグラフ」で$y=x^2,\ y=-x^2$をかこう.

\begin{layer}{120}{0}
\end{layer}

\begin{itemize}
\item
$y=x^2$
\end{itemize}
%%%%%%%%%%%%

%%%%%%%%%%%%%%%%%%%%


\sameslide

\vspace*{18mm}

\slidepage
\down
「2.関数のグラフ」で$y=x^2,\ y=-x^2$をかこう.

\begin{layer}{120}{0}
\putnotese{55}{5}{\scalebox{0.7}{%%% /Users/takatoosetsuo/polytech23.git/102-0424/presen/fig/parabola1.tex 
%%% Generator=presen23102.cdy 
{\unitlength=7mm%
\begin{picture}%
(10,10)(-5,-5)%
\linethickness{0.008in}%%
\Large\bf\boldmath%
\small%
\linethickness{0.012in}%%
\polyline(-2.235,5)(-2.2,4.84)(-2,4)(-1.8,3.24)(-1.6,2.56)(-1.4,1.96)(-1.2,1.44)(-1,1)%
(-0.8,0.64)(-0.6,0.36)(-0.4,0.16)(-0.2,0.04)(0,0)(0.2,0.04)(0.4,0.16)(0.6,0.36)(0.8,0.64)%
(1,1)(1.2,1.44)(1.4,1.96)(1.6,2.56)(1.8,3.24)(2,4)(2.2,4.84)(2.235,5)%
%
\linethickness{0.008in}%%
\polyline(-4,-0.071)(-4,0.071)%
%
\settowidth{\Width}{$-4$}\setlength{\Width}{-0.5\Width}%
\settoheight{\Height}{$-4$}\settodepth{\Depth}{$-4$}\setlength{\Height}{-\Height}%
\put( -4.000, -0.143){\hspace*{\Width}\raisebox{\Height}{$-4$}}%
%
\polyline(-3,-0.071)(-3,0.071)%
%
\settowidth{\Width}{$-3$}\setlength{\Width}{-0.5\Width}%
\settoheight{\Height}{$-3$}\settodepth{\Depth}{$-3$}\setlength{\Height}{-\Height}%
\put( -3.000, -0.143){\hspace*{\Width}\raisebox{\Height}{$-3$}}%
%
\polyline(-2,-0.071)(-2,0.071)%
%
\settowidth{\Width}{$-2$}\setlength{\Width}{-0.5\Width}%
\settoheight{\Height}{$-2$}\settodepth{\Depth}{$-2$}\setlength{\Height}{-\Height}%
\put( -2.000, -0.143){\hspace*{\Width}\raisebox{\Height}{$-2$}}%
%
\polyline(-1,-0.071)(-1,0.071)%
%
\settowidth{\Width}{$-1$}\setlength{\Width}{-0.5\Width}%
\settoheight{\Height}{$-1$}\settodepth{\Depth}{$-1$}\setlength{\Height}{-\Height}%
\put( -1.000, -0.143){\hspace*{\Width}\raisebox{\Height}{$-1$}}%
%
\polyline(1,-0.071)(1,0.071)%
%
\settowidth{\Width}{$1$}\setlength{\Width}{-0.5\Width}%
\settoheight{\Height}{$1$}\settodepth{\Depth}{$1$}\setlength{\Height}{-\Height}%
\put(  1.000, -0.143){\hspace*{\Width}\raisebox{\Height}{$1$}}%
%
\polyline(2,-0.071)(2,0.071)%
%
\settowidth{\Width}{$2$}\setlength{\Width}{-0.5\Width}%
\settoheight{\Height}{$2$}\settodepth{\Depth}{$2$}\setlength{\Height}{-\Height}%
\put(  2.000, -0.143){\hspace*{\Width}\raisebox{\Height}{$2$}}%
%
\polyline(3,-0.071)(3,0.071)%
%
\settowidth{\Width}{$3$}\setlength{\Width}{-0.5\Width}%
\settoheight{\Height}{$3$}\settodepth{\Depth}{$3$}\setlength{\Height}{-\Height}%
\put(  3.000, -0.143){\hspace*{\Width}\raisebox{\Height}{$3$}}%
%
\polyline(4,-0.071)(4,0.071)%
%
\settowidth{\Width}{$4$}\setlength{\Width}{-0.5\Width}%
\settoheight{\Height}{$4$}\settodepth{\Depth}{$4$}\setlength{\Height}{-\Height}%
\put(  4.000, -0.143){\hspace*{\Width}\raisebox{\Height}{$4$}}%
%
\polyline(-0.071,-4)(0.071,-4)%
%
\settowidth{\Width}{$-4$}\setlength{\Width}{-1\Width}%
\settoheight{\Height}{$-4$}\settodepth{\Depth}{$-4$}\setlength{\Height}{-0.5\Height}\setlength{\Depth}{0.5\Depth}\addtolength{\Height}{\Depth}%
\put( -0.143, -4.000){\hspace*{\Width}\raisebox{\Height}{$-4$}}%
%
\polyline(-0.071,-3)(0.071,-3)%
%
\settowidth{\Width}{$-3$}\setlength{\Width}{-1\Width}%
\settoheight{\Height}{$-3$}\settodepth{\Depth}{$-3$}\setlength{\Height}{-0.5\Height}\setlength{\Depth}{0.5\Depth}\addtolength{\Height}{\Depth}%
\put( -0.143, -3.000){\hspace*{\Width}\raisebox{\Height}{$-3$}}%
%
\polyline(-0.071,-2)(0.071,-2)%
%
\settowidth{\Width}{$-2$}\setlength{\Width}{-1\Width}%
\settoheight{\Height}{$-2$}\settodepth{\Depth}{$-2$}\setlength{\Height}{-0.5\Height}\setlength{\Depth}{0.5\Depth}\addtolength{\Height}{\Depth}%
\put( -0.143, -2.000){\hspace*{\Width}\raisebox{\Height}{$-2$}}%
%
\polyline(-0.071,-1)(0.071,-1)%
%
\settowidth{\Width}{$-1$}\setlength{\Width}{-1\Width}%
\settoheight{\Height}{$-1$}\settodepth{\Depth}{$-1$}\setlength{\Height}{-0.5\Height}\setlength{\Depth}{0.5\Depth}\addtolength{\Height}{\Depth}%
\put( -0.143, -1.000){\hspace*{\Width}\raisebox{\Height}{$-1$}}%
%
\polyline(-0.071,1)(0.071,1)%
%
\settowidth{\Width}{$1$}\setlength{\Width}{-1\Width}%
\settoheight{\Height}{$1$}\settodepth{\Depth}{$1$}\setlength{\Height}{-0.5\Height}\setlength{\Depth}{0.5\Depth}\addtolength{\Height}{\Depth}%
\put( -0.143,  1.000){\hspace*{\Width}\raisebox{\Height}{$1$}}%
%
\polyline(-0.071,2)(0.071,2)%
%
\settowidth{\Width}{$2$}\setlength{\Width}{-1\Width}%
\settoheight{\Height}{$2$}\settodepth{\Depth}{$2$}\setlength{\Height}{-0.5\Height}\setlength{\Depth}{0.5\Depth}\addtolength{\Height}{\Depth}%
\put( -0.143,  2.000){\hspace*{\Width}\raisebox{\Height}{$2$}}%
%
\polyline(-0.071,3)(0.071,3)%
%
\settowidth{\Width}{$3$}\setlength{\Width}{-1\Width}%
\settoheight{\Height}{$3$}\settodepth{\Depth}{$3$}\setlength{\Height}{-0.5\Height}\setlength{\Depth}{0.5\Depth}\addtolength{\Height}{\Depth}%
\put( -0.143,  3.000){\hspace*{\Width}\raisebox{\Height}{$3$}}%
%
\polyline(-0.071,4)(0.071,4)%
%
\settowidth{\Width}{$4$}\setlength{\Width}{-1\Width}%
\settoheight{\Height}{$4$}\settodepth{\Depth}{$4$}\setlength{\Height}{-0.5\Height}\setlength{\Depth}{0.5\Depth}\addtolength{\Height}{\Depth}%
\put( -0.143,  4.000){\hspace*{\Width}\raisebox{\Height}{$4$}}%
%
\polyline(-5,0)(5,0)%
%
\polyline(0,-5)(0,5)%
%
\settowidth{\Width}{$x$}\setlength{\Width}{0\Width}%
\settoheight{\Height}{$x$}\settodepth{\Depth}{$x$}\setlength{\Height}{-0.5\Height}\setlength{\Depth}{0.5\Depth}\addtolength{\Height}{\Depth}%
\put(  5.071,  0.000){\hspace*{\Width}\raisebox{\Height}{$x$}}%
%
\settowidth{\Width}{$y$}\setlength{\Width}{-0.5\Width}%
\settoheight{\Height}{$y$}\settodepth{\Depth}{$y$}\setlength{\Height}{\Depth}%
\put(  0.000,  5.071){\hspace*{\Width}\raisebox{\Height}{$y$}}%
%
\settowidth{\Width}{O}\setlength{\Width}{-1\Width}%
\settoheight{\Height}{O}\settodepth{\Depth}{O}\setlength{\Height}{-\Height}%
\put( -0.071, -0.071){\hspace*{\Width}\raisebox{\Height}{O}}%
%
\end{picture}}%}}
\end{layer}

\begin{itemize}
\item
$y=x^2$
\end{itemize}

\sameslide

\vspace*{18mm}

\slidepage
\down
「2.関数のグラフ」で$y=x^2,\ y=-x^2$をかこう.

\begin{layer}{120}{0}
\putnotese{55}{5}{\scalebox{0.7}{%%% /Users/takatoosetsuo/polytech23.git/102-0424/presen/fig/parabola1.tex 
%%% Generator=presen23102.cdy 
{\unitlength=7mm%
\begin{picture}%
(10,10)(-5,-5)%
\linethickness{0.008in}%%
\Large\bf\boldmath%
\small%
\linethickness{0.012in}%%
\polyline(-2.235,5)(-2.2,4.84)(-2,4)(-1.8,3.24)(-1.6,2.56)(-1.4,1.96)(-1.2,1.44)(-1,1)%
(-0.8,0.64)(-0.6,0.36)(-0.4,0.16)(-0.2,0.04)(0,0)(0.2,0.04)(0.4,0.16)(0.6,0.36)(0.8,0.64)%
(1,1)(1.2,1.44)(1.4,1.96)(1.6,2.56)(1.8,3.24)(2,4)(2.2,4.84)(2.235,5)%
%
\linethickness{0.008in}%%
\polyline(-4,-0.071)(-4,0.071)%
%
\settowidth{\Width}{$-4$}\setlength{\Width}{-0.5\Width}%
\settoheight{\Height}{$-4$}\settodepth{\Depth}{$-4$}\setlength{\Height}{-\Height}%
\put( -4.000, -0.143){\hspace*{\Width}\raisebox{\Height}{$-4$}}%
%
\polyline(-3,-0.071)(-3,0.071)%
%
\settowidth{\Width}{$-3$}\setlength{\Width}{-0.5\Width}%
\settoheight{\Height}{$-3$}\settodepth{\Depth}{$-3$}\setlength{\Height}{-\Height}%
\put( -3.000, -0.143){\hspace*{\Width}\raisebox{\Height}{$-3$}}%
%
\polyline(-2,-0.071)(-2,0.071)%
%
\settowidth{\Width}{$-2$}\setlength{\Width}{-0.5\Width}%
\settoheight{\Height}{$-2$}\settodepth{\Depth}{$-2$}\setlength{\Height}{-\Height}%
\put( -2.000, -0.143){\hspace*{\Width}\raisebox{\Height}{$-2$}}%
%
\polyline(-1,-0.071)(-1,0.071)%
%
\settowidth{\Width}{$-1$}\setlength{\Width}{-0.5\Width}%
\settoheight{\Height}{$-1$}\settodepth{\Depth}{$-1$}\setlength{\Height}{-\Height}%
\put( -1.000, -0.143){\hspace*{\Width}\raisebox{\Height}{$-1$}}%
%
\polyline(1,-0.071)(1,0.071)%
%
\settowidth{\Width}{$1$}\setlength{\Width}{-0.5\Width}%
\settoheight{\Height}{$1$}\settodepth{\Depth}{$1$}\setlength{\Height}{-\Height}%
\put(  1.000, -0.143){\hspace*{\Width}\raisebox{\Height}{$1$}}%
%
\polyline(2,-0.071)(2,0.071)%
%
\settowidth{\Width}{$2$}\setlength{\Width}{-0.5\Width}%
\settoheight{\Height}{$2$}\settodepth{\Depth}{$2$}\setlength{\Height}{-\Height}%
\put(  2.000, -0.143){\hspace*{\Width}\raisebox{\Height}{$2$}}%
%
\polyline(3,-0.071)(3,0.071)%
%
\settowidth{\Width}{$3$}\setlength{\Width}{-0.5\Width}%
\settoheight{\Height}{$3$}\settodepth{\Depth}{$3$}\setlength{\Height}{-\Height}%
\put(  3.000, -0.143){\hspace*{\Width}\raisebox{\Height}{$3$}}%
%
\polyline(4,-0.071)(4,0.071)%
%
\settowidth{\Width}{$4$}\setlength{\Width}{-0.5\Width}%
\settoheight{\Height}{$4$}\settodepth{\Depth}{$4$}\setlength{\Height}{-\Height}%
\put(  4.000, -0.143){\hspace*{\Width}\raisebox{\Height}{$4$}}%
%
\polyline(-0.071,-4)(0.071,-4)%
%
\settowidth{\Width}{$-4$}\setlength{\Width}{-1\Width}%
\settoheight{\Height}{$-4$}\settodepth{\Depth}{$-4$}\setlength{\Height}{-0.5\Height}\setlength{\Depth}{0.5\Depth}\addtolength{\Height}{\Depth}%
\put( -0.143, -4.000){\hspace*{\Width}\raisebox{\Height}{$-4$}}%
%
\polyline(-0.071,-3)(0.071,-3)%
%
\settowidth{\Width}{$-3$}\setlength{\Width}{-1\Width}%
\settoheight{\Height}{$-3$}\settodepth{\Depth}{$-3$}\setlength{\Height}{-0.5\Height}\setlength{\Depth}{0.5\Depth}\addtolength{\Height}{\Depth}%
\put( -0.143, -3.000){\hspace*{\Width}\raisebox{\Height}{$-3$}}%
%
\polyline(-0.071,-2)(0.071,-2)%
%
\settowidth{\Width}{$-2$}\setlength{\Width}{-1\Width}%
\settoheight{\Height}{$-2$}\settodepth{\Depth}{$-2$}\setlength{\Height}{-0.5\Height}\setlength{\Depth}{0.5\Depth}\addtolength{\Height}{\Depth}%
\put( -0.143, -2.000){\hspace*{\Width}\raisebox{\Height}{$-2$}}%
%
\polyline(-0.071,-1)(0.071,-1)%
%
\settowidth{\Width}{$-1$}\setlength{\Width}{-1\Width}%
\settoheight{\Height}{$-1$}\settodepth{\Depth}{$-1$}\setlength{\Height}{-0.5\Height}\setlength{\Depth}{0.5\Depth}\addtolength{\Height}{\Depth}%
\put( -0.143, -1.000){\hspace*{\Width}\raisebox{\Height}{$-1$}}%
%
\polyline(-0.071,1)(0.071,1)%
%
\settowidth{\Width}{$1$}\setlength{\Width}{-1\Width}%
\settoheight{\Height}{$1$}\settodepth{\Depth}{$1$}\setlength{\Height}{-0.5\Height}\setlength{\Depth}{0.5\Depth}\addtolength{\Height}{\Depth}%
\put( -0.143,  1.000){\hspace*{\Width}\raisebox{\Height}{$1$}}%
%
\polyline(-0.071,2)(0.071,2)%
%
\settowidth{\Width}{$2$}\setlength{\Width}{-1\Width}%
\settoheight{\Height}{$2$}\settodepth{\Depth}{$2$}\setlength{\Height}{-0.5\Height}\setlength{\Depth}{0.5\Depth}\addtolength{\Height}{\Depth}%
\put( -0.143,  2.000){\hspace*{\Width}\raisebox{\Height}{$2$}}%
%
\polyline(-0.071,3)(0.071,3)%
%
\settowidth{\Width}{$3$}\setlength{\Width}{-1\Width}%
\settoheight{\Height}{$3$}\settodepth{\Depth}{$3$}\setlength{\Height}{-0.5\Height}\setlength{\Depth}{0.5\Depth}\addtolength{\Height}{\Depth}%
\put( -0.143,  3.000){\hspace*{\Width}\raisebox{\Height}{$3$}}%
%
\polyline(-0.071,4)(0.071,4)%
%
\settowidth{\Width}{$4$}\setlength{\Width}{-1\Width}%
\settoheight{\Height}{$4$}\settodepth{\Depth}{$4$}\setlength{\Height}{-0.5\Height}\setlength{\Depth}{0.5\Depth}\addtolength{\Height}{\Depth}%
\put( -0.143,  4.000){\hspace*{\Width}\raisebox{\Height}{$4$}}%
%
\polyline(-5,0)(5,0)%
%
\polyline(0,-5)(0,5)%
%
\settowidth{\Width}{$x$}\setlength{\Width}{0\Width}%
\settoheight{\Height}{$x$}\settodepth{\Depth}{$x$}\setlength{\Height}{-0.5\Height}\setlength{\Depth}{0.5\Depth}\addtolength{\Height}{\Depth}%
\put(  5.071,  0.000){\hspace*{\Width}\raisebox{\Height}{$x$}}%
%
\settowidth{\Width}{$y$}\setlength{\Width}{-0.5\Width}%
\settoheight{\Height}{$y$}\settodepth{\Depth}{$y$}\setlength{\Height}{\Depth}%
\put(  0.000,  5.071){\hspace*{\Width}\raisebox{\Height}{$y$}}%
%
\settowidth{\Width}{O}\setlength{\Width}{-1\Width}%
\settoheight{\Height}{O}\settodepth{\Depth}{O}\setlength{\Height}{-\Height}%
\put( -0.071, -0.071){\hspace*{\Width}\raisebox{\Height}{O}}%
%
\end{picture}}%}}
\arrowlineseg{80}{30}{20}{-30}
\putnotee{97}{40}{頂点}
\arrowlineseg{80}{20}{20}{0}
\putnotee{100}{20}{軸}
\end{layer}

\begin{itemize}
\item
$y=x^2$
\end{itemize}

\sameslide

\vspace*{18mm}

\slidepage
\down
「2.関数のグラフ」で$y=x^2,\ y=-x^2$をかこう.

\begin{layer}{120}{0}
\putnotese{55}{5}{\scalebox{0.7}{%%% /Users/takatoosetsuo/polytech23.git/102-0424/presen/fig/parabola1.tex 
%%% Generator=presen23102.cdy 
{\unitlength=7mm%
\begin{picture}%
(10,10)(-5,-5)%
\linethickness{0.008in}%%
\Large\bf\boldmath%
\small%
\linethickness{0.012in}%%
\polyline(-2.235,5)(-2.2,4.84)(-2,4)(-1.8,3.24)(-1.6,2.56)(-1.4,1.96)(-1.2,1.44)(-1,1)%
(-0.8,0.64)(-0.6,0.36)(-0.4,0.16)(-0.2,0.04)(0,0)(0.2,0.04)(0.4,0.16)(0.6,0.36)(0.8,0.64)%
(1,1)(1.2,1.44)(1.4,1.96)(1.6,2.56)(1.8,3.24)(2,4)(2.2,4.84)(2.235,5)%
%
\linethickness{0.008in}%%
\polyline(-4,-0.071)(-4,0.071)%
%
\settowidth{\Width}{$-4$}\setlength{\Width}{-0.5\Width}%
\settoheight{\Height}{$-4$}\settodepth{\Depth}{$-4$}\setlength{\Height}{-\Height}%
\put( -4.000, -0.143){\hspace*{\Width}\raisebox{\Height}{$-4$}}%
%
\polyline(-3,-0.071)(-3,0.071)%
%
\settowidth{\Width}{$-3$}\setlength{\Width}{-0.5\Width}%
\settoheight{\Height}{$-3$}\settodepth{\Depth}{$-3$}\setlength{\Height}{-\Height}%
\put( -3.000, -0.143){\hspace*{\Width}\raisebox{\Height}{$-3$}}%
%
\polyline(-2,-0.071)(-2,0.071)%
%
\settowidth{\Width}{$-2$}\setlength{\Width}{-0.5\Width}%
\settoheight{\Height}{$-2$}\settodepth{\Depth}{$-2$}\setlength{\Height}{-\Height}%
\put( -2.000, -0.143){\hspace*{\Width}\raisebox{\Height}{$-2$}}%
%
\polyline(-1,-0.071)(-1,0.071)%
%
\settowidth{\Width}{$-1$}\setlength{\Width}{-0.5\Width}%
\settoheight{\Height}{$-1$}\settodepth{\Depth}{$-1$}\setlength{\Height}{-\Height}%
\put( -1.000, -0.143){\hspace*{\Width}\raisebox{\Height}{$-1$}}%
%
\polyline(1,-0.071)(1,0.071)%
%
\settowidth{\Width}{$1$}\setlength{\Width}{-0.5\Width}%
\settoheight{\Height}{$1$}\settodepth{\Depth}{$1$}\setlength{\Height}{-\Height}%
\put(  1.000, -0.143){\hspace*{\Width}\raisebox{\Height}{$1$}}%
%
\polyline(2,-0.071)(2,0.071)%
%
\settowidth{\Width}{$2$}\setlength{\Width}{-0.5\Width}%
\settoheight{\Height}{$2$}\settodepth{\Depth}{$2$}\setlength{\Height}{-\Height}%
\put(  2.000, -0.143){\hspace*{\Width}\raisebox{\Height}{$2$}}%
%
\polyline(3,-0.071)(3,0.071)%
%
\settowidth{\Width}{$3$}\setlength{\Width}{-0.5\Width}%
\settoheight{\Height}{$3$}\settodepth{\Depth}{$3$}\setlength{\Height}{-\Height}%
\put(  3.000, -0.143){\hspace*{\Width}\raisebox{\Height}{$3$}}%
%
\polyline(4,-0.071)(4,0.071)%
%
\settowidth{\Width}{$4$}\setlength{\Width}{-0.5\Width}%
\settoheight{\Height}{$4$}\settodepth{\Depth}{$4$}\setlength{\Height}{-\Height}%
\put(  4.000, -0.143){\hspace*{\Width}\raisebox{\Height}{$4$}}%
%
\polyline(-0.071,-4)(0.071,-4)%
%
\settowidth{\Width}{$-4$}\setlength{\Width}{-1\Width}%
\settoheight{\Height}{$-4$}\settodepth{\Depth}{$-4$}\setlength{\Height}{-0.5\Height}\setlength{\Depth}{0.5\Depth}\addtolength{\Height}{\Depth}%
\put( -0.143, -4.000){\hspace*{\Width}\raisebox{\Height}{$-4$}}%
%
\polyline(-0.071,-3)(0.071,-3)%
%
\settowidth{\Width}{$-3$}\setlength{\Width}{-1\Width}%
\settoheight{\Height}{$-3$}\settodepth{\Depth}{$-3$}\setlength{\Height}{-0.5\Height}\setlength{\Depth}{0.5\Depth}\addtolength{\Height}{\Depth}%
\put( -0.143, -3.000){\hspace*{\Width}\raisebox{\Height}{$-3$}}%
%
\polyline(-0.071,-2)(0.071,-2)%
%
\settowidth{\Width}{$-2$}\setlength{\Width}{-1\Width}%
\settoheight{\Height}{$-2$}\settodepth{\Depth}{$-2$}\setlength{\Height}{-0.5\Height}\setlength{\Depth}{0.5\Depth}\addtolength{\Height}{\Depth}%
\put( -0.143, -2.000){\hspace*{\Width}\raisebox{\Height}{$-2$}}%
%
\polyline(-0.071,-1)(0.071,-1)%
%
\settowidth{\Width}{$-1$}\setlength{\Width}{-1\Width}%
\settoheight{\Height}{$-1$}\settodepth{\Depth}{$-1$}\setlength{\Height}{-0.5\Height}\setlength{\Depth}{0.5\Depth}\addtolength{\Height}{\Depth}%
\put( -0.143, -1.000){\hspace*{\Width}\raisebox{\Height}{$-1$}}%
%
\polyline(-0.071,1)(0.071,1)%
%
\settowidth{\Width}{$1$}\setlength{\Width}{-1\Width}%
\settoheight{\Height}{$1$}\settodepth{\Depth}{$1$}\setlength{\Height}{-0.5\Height}\setlength{\Depth}{0.5\Depth}\addtolength{\Height}{\Depth}%
\put( -0.143,  1.000){\hspace*{\Width}\raisebox{\Height}{$1$}}%
%
\polyline(-0.071,2)(0.071,2)%
%
\settowidth{\Width}{$2$}\setlength{\Width}{-1\Width}%
\settoheight{\Height}{$2$}\settodepth{\Depth}{$2$}\setlength{\Height}{-0.5\Height}\setlength{\Depth}{0.5\Depth}\addtolength{\Height}{\Depth}%
\put( -0.143,  2.000){\hspace*{\Width}\raisebox{\Height}{$2$}}%
%
\polyline(-0.071,3)(0.071,3)%
%
\settowidth{\Width}{$3$}\setlength{\Width}{-1\Width}%
\settoheight{\Height}{$3$}\settodepth{\Depth}{$3$}\setlength{\Height}{-0.5\Height}\setlength{\Depth}{0.5\Depth}\addtolength{\Height}{\Depth}%
\put( -0.143,  3.000){\hspace*{\Width}\raisebox{\Height}{$3$}}%
%
\polyline(-0.071,4)(0.071,4)%
%
\settowidth{\Width}{$4$}\setlength{\Width}{-1\Width}%
\settoheight{\Height}{$4$}\settodepth{\Depth}{$4$}\setlength{\Height}{-0.5\Height}\setlength{\Depth}{0.5\Depth}\addtolength{\Height}{\Depth}%
\put( -0.143,  4.000){\hspace*{\Width}\raisebox{\Height}{$4$}}%
%
\polyline(-5,0)(5,0)%
%
\polyline(0,-5)(0,5)%
%
\settowidth{\Width}{$x$}\setlength{\Width}{0\Width}%
\settoheight{\Height}{$x$}\settodepth{\Depth}{$x$}\setlength{\Height}{-0.5\Height}\setlength{\Depth}{0.5\Depth}\addtolength{\Height}{\Depth}%
\put(  5.071,  0.000){\hspace*{\Width}\raisebox{\Height}{$x$}}%
%
\settowidth{\Width}{$y$}\setlength{\Width}{-0.5\Width}%
\settoheight{\Height}{$y$}\settodepth{\Depth}{$y$}\setlength{\Height}{\Depth}%
\put(  0.000,  5.071){\hspace*{\Width}\raisebox{\Height}{$y$}}%
%
\settowidth{\Width}{O}\setlength{\Width}{-1\Width}%
\settoheight{\Height}{O}\settodepth{\Depth}{O}\setlength{\Height}{-\Height}%
\put( -0.071, -0.071){\hspace*{\Width}\raisebox{\Height}{O}}%
%
\end{picture}}%}}
\arrowlineseg{80}{30}{20}{-30}
\putnotee{97}{40}{頂点}
\arrowlineseg{80}{20}{20}{0}
\putnotee{100}{20}{軸}
\end{layer}

\begin{itemize}
\item
$y=x^2$
\item
[] 軸は$x=0$($y$軸)
\end{itemize}

\sameslide

\vspace*{18mm}

\slidepage
\down
「2.関数のグラフ」で$y=x^2,\ y=-x^2$をかこう.

\begin{layer}{120}{0}
\putnotese{55}{5}{\scalebox{0.7}{%%% /Users/takatoosetsuo/polytech23.git/102-0424/presen/fig/parabola1.tex 
%%% Generator=presen23102.cdy 
{\unitlength=7mm%
\begin{picture}%
(10,10)(-5,-5)%
\linethickness{0.008in}%%
\Large\bf\boldmath%
\small%
\linethickness{0.012in}%%
\polyline(-2.235,5)(-2.2,4.84)(-2,4)(-1.8,3.24)(-1.6,2.56)(-1.4,1.96)(-1.2,1.44)(-1,1)%
(-0.8,0.64)(-0.6,0.36)(-0.4,0.16)(-0.2,0.04)(0,0)(0.2,0.04)(0.4,0.16)(0.6,0.36)(0.8,0.64)%
(1,1)(1.2,1.44)(1.4,1.96)(1.6,2.56)(1.8,3.24)(2,4)(2.2,4.84)(2.235,5)%
%
\linethickness{0.008in}%%
\polyline(-4,-0.071)(-4,0.071)%
%
\settowidth{\Width}{$-4$}\setlength{\Width}{-0.5\Width}%
\settoheight{\Height}{$-4$}\settodepth{\Depth}{$-4$}\setlength{\Height}{-\Height}%
\put( -4.000, -0.143){\hspace*{\Width}\raisebox{\Height}{$-4$}}%
%
\polyline(-3,-0.071)(-3,0.071)%
%
\settowidth{\Width}{$-3$}\setlength{\Width}{-0.5\Width}%
\settoheight{\Height}{$-3$}\settodepth{\Depth}{$-3$}\setlength{\Height}{-\Height}%
\put( -3.000, -0.143){\hspace*{\Width}\raisebox{\Height}{$-3$}}%
%
\polyline(-2,-0.071)(-2,0.071)%
%
\settowidth{\Width}{$-2$}\setlength{\Width}{-0.5\Width}%
\settoheight{\Height}{$-2$}\settodepth{\Depth}{$-2$}\setlength{\Height}{-\Height}%
\put( -2.000, -0.143){\hspace*{\Width}\raisebox{\Height}{$-2$}}%
%
\polyline(-1,-0.071)(-1,0.071)%
%
\settowidth{\Width}{$-1$}\setlength{\Width}{-0.5\Width}%
\settoheight{\Height}{$-1$}\settodepth{\Depth}{$-1$}\setlength{\Height}{-\Height}%
\put( -1.000, -0.143){\hspace*{\Width}\raisebox{\Height}{$-1$}}%
%
\polyline(1,-0.071)(1,0.071)%
%
\settowidth{\Width}{$1$}\setlength{\Width}{-0.5\Width}%
\settoheight{\Height}{$1$}\settodepth{\Depth}{$1$}\setlength{\Height}{-\Height}%
\put(  1.000, -0.143){\hspace*{\Width}\raisebox{\Height}{$1$}}%
%
\polyline(2,-0.071)(2,0.071)%
%
\settowidth{\Width}{$2$}\setlength{\Width}{-0.5\Width}%
\settoheight{\Height}{$2$}\settodepth{\Depth}{$2$}\setlength{\Height}{-\Height}%
\put(  2.000, -0.143){\hspace*{\Width}\raisebox{\Height}{$2$}}%
%
\polyline(3,-0.071)(3,0.071)%
%
\settowidth{\Width}{$3$}\setlength{\Width}{-0.5\Width}%
\settoheight{\Height}{$3$}\settodepth{\Depth}{$3$}\setlength{\Height}{-\Height}%
\put(  3.000, -0.143){\hspace*{\Width}\raisebox{\Height}{$3$}}%
%
\polyline(4,-0.071)(4,0.071)%
%
\settowidth{\Width}{$4$}\setlength{\Width}{-0.5\Width}%
\settoheight{\Height}{$4$}\settodepth{\Depth}{$4$}\setlength{\Height}{-\Height}%
\put(  4.000, -0.143){\hspace*{\Width}\raisebox{\Height}{$4$}}%
%
\polyline(-0.071,-4)(0.071,-4)%
%
\settowidth{\Width}{$-4$}\setlength{\Width}{-1\Width}%
\settoheight{\Height}{$-4$}\settodepth{\Depth}{$-4$}\setlength{\Height}{-0.5\Height}\setlength{\Depth}{0.5\Depth}\addtolength{\Height}{\Depth}%
\put( -0.143, -4.000){\hspace*{\Width}\raisebox{\Height}{$-4$}}%
%
\polyline(-0.071,-3)(0.071,-3)%
%
\settowidth{\Width}{$-3$}\setlength{\Width}{-1\Width}%
\settoheight{\Height}{$-3$}\settodepth{\Depth}{$-3$}\setlength{\Height}{-0.5\Height}\setlength{\Depth}{0.5\Depth}\addtolength{\Height}{\Depth}%
\put( -0.143, -3.000){\hspace*{\Width}\raisebox{\Height}{$-3$}}%
%
\polyline(-0.071,-2)(0.071,-2)%
%
\settowidth{\Width}{$-2$}\setlength{\Width}{-1\Width}%
\settoheight{\Height}{$-2$}\settodepth{\Depth}{$-2$}\setlength{\Height}{-0.5\Height}\setlength{\Depth}{0.5\Depth}\addtolength{\Height}{\Depth}%
\put( -0.143, -2.000){\hspace*{\Width}\raisebox{\Height}{$-2$}}%
%
\polyline(-0.071,-1)(0.071,-1)%
%
\settowidth{\Width}{$-1$}\setlength{\Width}{-1\Width}%
\settoheight{\Height}{$-1$}\settodepth{\Depth}{$-1$}\setlength{\Height}{-0.5\Height}\setlength{\Depth}{0.5\Depth}\addtolength{\Height}{\Depth}%
\put( -0.143, -1.000){\hspace*{\Width}\raisebox{\Height}{$-1$}}%
%
\polyline(-0.071,1)(0.071,1)%
%
\settowidth{\Width}{$1$}\setlength{\Width}{-1\Width}%
\settoheight{\Height}{$1$}\settodepth{\Depth}{$1$}\setlength{\Height}{-0.5\Height}\setlength{\Depth}{0.5\Depth}\addtolength{\Height}{\Depth}%
\put( -0.143,  1.000){\hspace*{\Width}\raisebox{\Height}{$1$}}%
%
\polyline(-0.071,2)(0.071,2)%
%
\settowidth{\Width}{$2$}\setlength{\Width}{-1\Width}%
\settoheight{\Height}{$2$}\settodepth{\Depth}{$2$}\setlength{\Height}{-0.5\Height}\setlength{\Depth}{0.5\Depth}\addtolength{\Height}{\Depth}%
\put( -0.143,  2.000){\hspace*{\Width}\raisebox{\Height}{$2$}}%
%
\polyline(-0.071,3)(0.071,3)%
%
\settowidth{\Width}{$3$}\setlength{\Width}{-1\Width}%
\settoheight{\Height}{$3$}\settodepth{\Depth}{$3$}\setlength{\Height}{-0.5\Height}\setlength{\Depth}{0.5\Depth}\addtolength{\Height}{\Depth}%
\put( -0.143,  3.000){\hspace*{\Width}\raisebox{\Height}{$3$}}%
%
\polyline(-0.071,4)(0.071,4)%
%
\settowidth{\Width}{$4$}\setlength{\Width}{-1\Width}%
\settoheight{\Height}{$4$}\settodepth{\Depth}{$4$}\setlength{\Height}{-0.5\Height}\setlength{\Depth}{0.5\Depth}\addtolength{\Height}{\Depth}%
\put( -0.143,  4.000){\hspace*{\Width}\raisebox{\Height}{$4$}}%
%
\polyline(-5,0)(5,0)%
%
\polyline(0,-5)(0,5)%
%
\settowidth{\Width}{$x$}\setlength{\Width}{0\Width}%
\settoheight{\Height}{$x$}\settodepth{\Depth}{$x$}\setlength{\Height}{-0.5\Height}\setlength{\Depth}{0.5\Depth}\addtolength{\Height}{\Depth}%
\put(  5.071,  0.000){\hspace*{\Width}\raisebox{\Height}{$x$}}%
%
\settowidth{\Width}{$y$}\setlength{\Width}{-0.5\Width}%
\settoheight{\Height}{$y$}\settodepth{\Depth}{$y$}\setlength{\Height}{\Depth}%
\put(  0.000,  5.071){\hspace*{\Width}\raisebox{\Height}{$y$}}%
%
\settowidth{\Width}{O}\setlength{\Width}{-1\Width}%
\settoheight{\Height}{O}\settodepth{\Depth}{O}\setlength{\Height}{-\Height}%
\put( -0.071, -0.071){\hspace*{\Width}\raisebox{\Height}{O}}%
%
\end{picture}}%}}
\arrowlineseg{80}{30}{20}{-30}
\putnotee{97}{40}{頂点}
\arrowlineseg{80}{20}{20}{0}
\putnotee{100}{20}{軸}
\end{layer}

\begin{itemize}
\item
$y=x^2$
\item
[] 軸は$x=0$($y$軸)
\item
[] 頂点は$(0,\ 0)$
\end{itemize}

\sameslide

\vspace*{18mm}

\slidepage
\down
「2.関数のグラフ」で$y=x^2,\ y=-x^2$をかこう.

\begin{layer}{120}{0}
\putnotese{55}{5}{\scalebox{0.7}{%%% /Users/takatoosetsuo/polytech23.git/102-0424/presen/fig/parabola1.tex 
%%% Generator=presen23102.cdy 
{\unitlength=7mm%
\begin{picture}%
(10,10)(-5,-5)%
\linethickness{0.008in}%%
\Large\bf\boldmath%
\small%
\linethickness{0.012in}%%
\polyline(-2.235,5)(-2.2,4.84)(-2,4)(-1.8,3.24)(-1.6,2.56)(-1.4,1.96)(-1.2,1.44)(-1,1)%
(-0.8,0.64)(-0.6,0.36)(-0.4,0.16)(-0.2,0.04)(0,0)(0.2,0.04)(0.4,0.16)(0.6,0.36)(0.8,0.64)%
(1,1)(1.2,1.44)(1.4,1.96)(1.6,2.56)(1.8,3.24)(2,4)(2.2,4.84)(2.235,5)%
%
\linethickness{0.008in}%%
\polyline(-4,-0.071)(-4,0.071)%
%
\settowidth{\Width}{$-4$}\setlength{\Width}{-0.5\Width}%
\settoheight{\Height}{$-4$}\settodepth{\Depth}{$-4$}\setlength{\Height}{-\Height}%
\put( -4.000, -0.143){\hspace*{\Width}\raisebox{\Height}{$-4$}}%
%
\polyline(-3,-0.071)(-3,0.071)%
%
\settowidth{\Width}{$-3$}\setlength{\Width}{-0.5\Width}%
\settoheight{\Height}{$-3$}\settodepth{\Depth}{$-3$}\setlength{\Height}{-\Height}%
\put( -3.000, -0.143){\hspace*{\Width}\raisebox{\Height}{$-3$}}%
%
\polyline(-2,-0.071)(-2,0.071)%
%
\settowidth{\Width}{$-2$}\setlength{\Width}{-0.5\Width}%
\settoheight{\Height}{$-2$}\settodepth{\Depth}{$-2$}\setlength{\Height}{-\Height}%
\put( -2.000, -0.143){\hspace*{\Width}\raisebox{\Height}{$-2$}}%
%
\polyline(-1,-0.071)(-1,0.071)%
%
\settowidth{\Width}{$-1$}\setlength{\Width}{-0.5\Width}%
\settoheight{\Height}{$-1$}\settodepth{\Depth}{$-1$}\setlength{\Height}{-\Height}%
\put( -1.000, -0.143){\hspace*{\Width}\raisebox{\Height}{$-1$}}%
%
\polyline(1,-0.071)(1,0.071)%
%
\settowidth{\Width}{$1$}\setlength{\Width}{-0.5\Width}%
\settoheight{\Height}{$1$}\settodepth{\Depth}{$1$}\setlength{\Height}{-\Height}%
\put(  1.000, -0.143){\hspace*{\Width}\raisebox{\Height}{$1$}}%
%
\polyline(2,-0.071)(2,0.071)%
%
\settowidth{\Width}{$2$}\setlength{\Width}{-0.5\Width}%
\settoheight{\Height}{$2$}\settodepth{\Depth}{$2$}\setlength{\Height}{-\Height}%
\put(  2.000, -0.143){\hspace*{\Width}\raisebox{\Height}{$2$}}%
%
\polyline(3,-0.071)(3,0.071)%
%
\settowidth{\Width}{$3$}\setlength{\Width}{-0.5\Width}%
\settoheight{\Height}{$3$}\settodepth{\Depth}{$3$}\setlength{\Height}{-\Height}%
\put(  3.000, -0.143){\hspace*{\Width}\raisebox{\Height}{$3$}}%
%
\polyline(4,-0.071)(4,0.071)%
%
\settowidth{\Width}{$4$}\setlength{\Width}{-0.5\Width}%
\settoheight{\Height}{$4$}\settodepth{\Depth}{$4$}\setlength{\Height}{-\Height}%
\put(  4.000, -0.143){\hspace*{\Width}\raisebox{\Height}{$4$}}%
%
\polyline(-0.071,-4)(0.071,-4)%
%
\settowidth{\Width}{$-4$}\setlength{\Width}{-1\Width}%
\settoheight{\Height}{$-4$}\settodepth{\Depth}{$-4$}\setlength{\Height}{-0.5\Height}\setlength{\Depth}{0.5\Depth}\addtolength{\Height}{\Depth}%
\put( -0.143, -4.000){\hspace*{\Width}\raisebox{\Height}{$-4$}}%
%
\polyline(-0.071,-3)(0.071,-3)%
%
\settowidth{\Width}{$-3$}\setlength{\Width}{-1\Width}%
\settoheight{\Height}{$-3$}\settodepth{\Depth}{$-3$}\setlength{\Height}{-0.5\Height}\setlength{\Depth}{0.5\Depth}\addtolength{\Height}{\Depth}%
\put( -0.143, -3.000){\hspace*{\Width}\raisebox{\Height}{$-3$}}%
%
\polyline(-0.071,-2)(0.071,-2)%
%
\settowidth{\Width}{$-2$}\setlength{\Width}{-1\Width}%
\settoheight{\Height}{$-2$}\settodepth{\Depth}{$-2$}\setlength{\Height}{-0.5\Height}\setlength{\Depth}{0.5\Depth}\addtolength{\Height}{\Depth}%
\put( -0.143, -2.000){\hspace*{\Width}\raisebox{\Height}{$-2$}}%
%
\polyline(-0.071,-1)(0.071,-1)%
%
\settowidth{\Width}{$-1$}\setlength{\Width}{-1\Width}%
\settoheight{\Height}{$-1$}\settodepth{\Depth}{$-1$}\setlength{\Height}{-0.5\Height}\setlength{\Depth}{0.5\Depth}\addtolength{\Height}{\Depth}%
\put( -0.143, -1.000){\hspace*{\Width}\raisebox{\Height}{$-1$}}%
%
\polyline(-0.071,1)(0.071,1)%
%
\settowidth{\Width}{$1$}\setlength{\Width}{-1\Width}%
\settoheight{\Height}{$1$}\settodepth{\Depth}{$1$}\setlength{\Height}{-0.5\Height}\setlength{\Depth}{0.5\Depth}\addtolength{\Height}{\Depth}%
\put( -0.143,  1.000){\hspace*{\Width}\raisebox{\Height}{$1$}}%
%
\polyline(-0.071,2)(0.071,2)%
%
\settowidth{\Width}{$2$}\setlength{\Width}{-1\Width}%
\settoheight{\Height}{$2$}\settodepth{\Depth}{$2$}\setlength{\Height}{-0.5\Height}\setlength{\Depth}{0.5\Depth}\addtolength{\Height}{\Depth}%
\put( -0.143,  2.000){\hspace*{\Width}\raisebox{\Height}{$2$}}%
%
\polyline(-0.071,3)(0.071,3)%
%
\settowidth{\Width}{$3$}\setlength{\Width}{-1\Width}%
\settoheight{\Height}{$3$}\settodepth{\Depth}{$3$}\setlength{\Height}{-0.5\Height}\setlength{\Depth}{0.5\Depth}\addtolength{\Height}{\Depth}%
\put( -0.143,  3.000){\hspace*{\Width}\raisebox{\Height}{$3$}}%
%
\polyline(-0.071,4)(0.071,4)%
%
\settowidth{\Width}{$4$}\setlength{\Width}{-1\Width}%
\settoheight{\Height}{$4$}\settodepth{\Depth}{$4$}\setlength{\Height}{-0.5\Height}\setlength{\Depth}{0.5\Depth}\addtolength{\Height}{\Depth}%
\put( -0.143,  4.000){\hspace*{\Width}\raisebox{\Height}{$4$}}%
%
\polyline(-5,0)(5,0)%
%
\polyline(0,-5)(0,5)%
%
\settowidth{\Width}{$x$}\setlength{\Width}{0\Width}%
\settoheight{\Height}{$x$}\settodepth{\Depth}{$x$}\setlength{\Height}{-0.5\Height}\setlength{\Depth}{0.5\Depth}\addtolength{\Height}{\Depth}%
\put(  5.071,  0.000){\hspace*{\Width}\raisebox{\Height}{$x$}}%
%
\settowidth{\Width}{$y$}\setlength{\Width}{-0.5\Width}%
\settoheight{\Height}{$y$}\settodepth{\Depth}{$y$}\setlength{\Height}{\Depth}%
\put(  0.000,  5.071){\hspace*{\Width}\raisebox{\Height}{$y$}}%
%
\settowidth{\Width}{O}\setlength{\Width}{-1\Width}%
\settoheight{\Height}{O}\settodepth{\Depth}{O}\setlength{\Height}{-\Height}%
\put( -0.071, -0.071){\hspace*{\Width}\raisebox{\Height}{O}}%
%
\end{picture}}%}}
\arrowlineseg{80}{30}{20}{-30}
\putnotee{97}{40}{頂点}
\arrowlineseg{80}{20}{20}{0}
\putnotee{100}{20}{軸}
\end{layer}

\begin{itemize}
\item
$y=x^2$
\item
[] 軸は$x=0$($y$軸)
\item
[] 頂点は$(0,\ 0)$
\item
[] 下に凸
\end{itemize}

\sameslide

\vspace*{18mm}

\slidepage
\down
「2.関数のグラフ」で$y=x^2,\ y=-x^2$をかこう.

\begin{layer}{120}{0}
\putnotese{55}{5}{\scalebox{0.7}{%%% /Users/takatoosetsuo/polytech23.git/102-0424/presen/fig/parabola1.tex 
%%% Generator=presen23102.cdy 
{\unitlength=7mm%
\begin{picture}%
(10,10)(-5,-5)%
\linethickness{0.008in}%%
\Large\bf\boldmath%
\small%
\linethickness{0.012in}%%
\polyline(-2.235,5)(-2.2,4.84)(-2,4)(-1.8,3.24)(-1.6,2.56)(-1.4,1.96)(-1.2,1.44)(-1,1)%
(-0.8,0.64)(-0.6,0.36)(-0.4,0.16)(-0.2,0.04)(0,0)(0.2,0.04)(0.4,0.16)(0.6,0.36)(0.8,0.64)%
(1,1)(1.2,1.44)(1.4,1.96)(1.6,2.56)(1.8,3.24)(2,4)(2.2,4.84)(2.235,5)%
%
\linethickness{0.008in}%%
\polyline(-4,-0.071)(-4,0.071)%
%
\settowidth{\Width}{$-4$}\setlength{\Width}{-0.5\Width}%
\settoheight{\Height}{$-4$}\settodepth{\Depth}{$-4$}\setlength{\Height}{-\Height}%
\put( -4.000, -0.143){\hspace*{\Width}\raisebox{\Height}{$-4$}}%
%
\polyline(-3,-0.071)(-3,0.071)%
%
\settowidth{\Width}{$-3$}\setlength{\Width}{-0.5\Width}%
\settoheight{\Height}{$-3$}\settodepth{\Depth}{$-3$}\setlength{\Height}{-\Height}%
\put( -3.000, -0.143){\hspace*{\Width}\raisebox{\Height}{$-3$}}%
%
\polyline(-2,-0.071)(-2,0.071)%
%
\settowidth{\Width}{$-2$}\setlength{\Width}{-0.5\Width}%
\settoheight{\Height}{$-2$}\settodepth{\Depth}{$-2$}\setlength{\Height}{-\Height}%
\put( -2.000, -0.143){\hspace*{\Width}\raisebox{\Height}{$-2$}}%
%
\polyline(-1,-0.071)(-1,0.071)%
%
\settowidth{\Width}{$-1$}\setlength{\Width}{-0.5\Width}%
\settoheight{\Height}{$-1$}\settodepth{\Depth}{$-1$}\setlength{\Height}{-\Height}%
\put( -1.000, -0.143){\hspace*{\Width}\raisebox{\Height}{$-1$}}%
%
\polyline(1,-0.071)(1,0.071)%
%
\settowidth{\Width}{$1$}\setlength{\Width}{-0.5\Width}%
\settoheight{\Height}{$1$}\settodepth{\Depth}{$1$}\setlength{\Height}{-\Height}%
\put(  1.000, -0.143){\hspace*{\Width}\raisebox{\Height}{$1$}}%
%
\polyline(2,-0.071)(2,0.071)%
%
\settowidth{\Width}{$2$}\setlength{\Width}{-0.5\Width}%
\settoheight{\Height}{$2$}\settodepth{\Depth}{$2$}\setlength{\Height}{-\Height}%
\put(  2.000, -0.143){\hspace*{\Width}\raisebox{\Height}{$2$}}%
%
\polyline(3,-0.071)(3,0.071)%
%
\settowidth{\Width}{$3$}\setlength{\Width}{-0.5\Width}%
\settoheight{\Height}{$3$}\settodepth{\Depth}{$3$}\setlength{\Height}{-\Height}%
\put(  3.000, -0.143){\hspace*{\Width}\raisebox{\Height}{$3$}}%
%
\polyline(4,-0.071)(4,0.071)%
%
\settowidth{\Width}{$4$}\setlength{\Width}{-0.5\Width}%
\settoheight{\Height}{$4$}\settodepth{\Depth}{$4$}\setlength{\Height}{-\Height}%
\put(  4.000, -0.143){\hspace*{\Width}\raisebox{\Height}{$4$}}%
%
\polyline(-0.071,-4)(0.071,-4)%
%
\settowidth{\Width}{$-4$}\setlength{\Width}{-1\Width}%
\settoheight{\Height}{$-4$}\settodepth{\Depth}{$-4$}\setlength{\Height}{-0.5\Height}\setlength{\Depth}{0.5\Depth}\addtolength{\Height}{\Depth}%
\put( -0.143, -4.000){\hspace*{\Width}\raisebox{\Height}{$-4$}}%
%
\polyline(-0.071,-3)(0.071,-3)%
%
\settowidth{\Width}{$-3$}\setlength{\Width}{-1\Width}%
\settoheight{\Height}{$-3$}\settodepth{\Depth}{$-3$}\setlength{\Height}{-0.5\Height}\setlength{\Depth}{0.5\Depth}\addtolength{\Height}{\Depth}%
\put( -0.143, -3.000){\hspace*{\Width}\raisebox{\Height}{$-3$}}%
%
\polyline(-0.071,-2)(0.071,-2)%
%
\settowidth{\Width}{$-2$}\setlength{\Width}{-1\Width}%
\settoheight{\Height}{$-2$}\settodepth{\Depth}{$-2$}\setlength{\Height}{-0.5\Height}\setlength{\Depth}{0.5\Depth}\addtolength{\Height}{\Depth}%
\put( -0.143, -2.000){\hspace*{\Width}\raisebox{\Height}{$-2$}}%
%
\polyline(-0.071,-1)(0.071,-1)%
%
\settowidth{\Width}{$-1$}\setlength{\Width}{-1\Width}%
\settoheight{\Height}{$-1$}\settodepth{\Depth}{$-1$}\setlength{\Height}{-0.5\Height}\setlength{\Depth}{0.5\Depth}\addtolength{\Height}{\Depth}%
\put( -0.143, -1.000){\hspace*{\Width}\raisebox{\Height}{$-1$}}%
%
\polyline(-0.071,1)(0.071,1)%
%
\settowidth{\Width}{$1$}\setlength{\Width}{-1\Width}%
\settoheight{\Height}{$1$}\settodepth{\Depth}{$1$}\setlength{\Height}{-0.5\Height}\setlength{\Depth}{0.5\Depth}\addtolength{\Height}{\Depth}%
\put( -0.143,  1.000){\hspace*{\Width}\raisebox{\Height}{$1$}}%
%
\polyline(-0.071,2)(0.071,2)%
%
\settowidth{\Width}{$2$}\setlength{\Width}{-1\Width}%
\settoheight{\Height}{$2$}\settodepth{\Depth}{$2$}\setlength{\Height}{-0.5\Height}\setlength{\Depth}{0.5\Depth}\addtolength{\Height}{\Depth}%
\put( -0.143,  2.000){\hspace*{\Width}\raisebox{\Height}{$2$}}%
%
\polyline(-0.071,3)(0.071,3)%
%
\settowidth{\Width}{$3$}\setlength{\Width}{-1\Width}%
\settoheight{\Height}{$3$}\settodepth{\Depth}{$3$}\setlength{\Height}{-0.5\Height}\setlength{\Depth}{0.5\Depth}\addtolength{\Height}{\Depth}%
\put( -0.143,  3.000){\hspace*{\Width}\raisebox{\Height}{$3$}}%
%
\polyline(-0.071,4)(0.071,4)%
%
\settowidth{\Width}{$4$}\setlength{\Width}{-1\Width}%
\settoheight{\Height}{$4$}\settodepth{\Depth}{$4$}\setlength{\Height}{-0.5\Height}\setlength{\Depth}{0.5\Depth}\addtolength{\Height}{\Depth}%
\put( -0.143,  4.000){\hspace*{\Width}\raisebox{\Height}{$4$}}%
%
\polyline(-5,0)(5,0)%
%
\polyline(0,-5)(0,5)%
%
\settowidth{\Width}{$x$}\setlength{\Width}{0\Width}%
\settoheight{\Height}{$x$}\settodepth{\Depth}{$x$}\setlength{\Height}{-0.5\Height}\setlength{\Depth}{0.5\Depth}\addtolength{\Height}{\Depth}%
\put(  5.071,  0.000){\hspace*{\Width}\raisebox{\Height}{$x$}}%
%
\settowidth{\Width}{$y$}\setlength{\Width}{-0.5\Width}%
\settoheight{\Height}{$y$}\settodepth{\Depth}{$y$}\setlength{\Height}{\Depth}%
\put(  0.000,  5.071){\hspace*{\Width}\raisebox{\Height}{$y$}}%
%
\settowidth{\Width}{O}\setlength{\Width}{-1\Width}%
\settoheight{\Height}{O}\settodepth{\Depth}{O}\setlength{\Height}{-\Height}%
\put( -0.071, -0.071){\hspace*{\Width}\raisebox{\Height}{O}}%
%
\end{picture}}%}}
\arrowlineseg{80}{30}{20}{-30}
\putnotee{97}{40}{頂点}
\arrowlineseg{80}{20}{20}{0}
\putnotee{100}{20}{軸}
\end{layer}

\begin{itemize}
\item
$y=x^2$
\item
[] 軸は$x=0$($y$軸)
\item
[] 頂点は$(0,\ 0)$
\item
[] 下に凸
\item
$y=-x^2$
\end{itemize}

\sameslide

\vspace*{18mm}

\slidepage
\down
「2.関数のグラフ」で$y=x^2,\ y=-x^2$をかこう.

\begin{layer}{120}{0}
\putnotese{55}{5}{\scalebox{0.7}{%%% /Users/takatoosetsuo/polytech23.git/102-0424/presen/fig/parabola1.tex 
%%% Generator=presen23102.cdy 
{\unitlength=7mm%
\begin{picture}%
(10,10)(-5,-5)%
\linethickness{0.008in}%%
\Large\bf\boldmath%
\small%
\linethickness{0.012in}%%
\polyline(-2.235,5)(-2.2,4.84)(-2,4)(-1.8,3.24)(-1.6,2.56)(-1.4,1.96)(-1.2,1.44)(-1,1)%
(-0.8,0.64)(-0.6,0.36)(-0.4,0.16)(-0.2,0.04)(0,0)(0.2,0.04)(0.4,0.16)(0.6,0.36)(0.8,0.64)%
(1,1)(1.2,1.44)(1.4,1.96)(1.6,2.56)(1.8,3.24)(2,4)(2.2,4.84)(2.235,5)%
%
\linethickness{0.008in}%%
\polyline(-4,-0.071)(-4,0.071)%
%
\settowidth{\Width}{$-4$}\setlength{\Width}{-0.5\Width}%
\settoheight{\Height}{$-4$}\settodepth{\Depth}{$-4$}\setlength{\Height}{-\Height}%
\put( -4.000, -0.143){\hspace*{\Width}\raisebox{\Height}{$-4$}}%
%
\polyline(-3,-0.071)(-3,0.071)%
%
\settowidth{\Width}{$-3$}\setlength{\Width}{-0.5\Width}%
\settoheight{\Height}{$-3$}\settodepth{\Depth}{$-3$}\setlength{\Height}{-\Height}%
\put( -3.000, -0.143){\hspace*{\Width}\raisebox{\Height}{$-3$}}%
%
\polyline(-2,-0.071)(-2,0.071)%
%
\settowidth{\Width}{$-2$}\setlength{\Width}{-0.5\Width}%
\settoheight{\Height}{$-2$}\settodepth{\Depth}{$-2$}\setlength{\Height}{-\Height}%
\put( -2.000, -0.143){\hspace*{\Width}\raisebox{\Height}{$-2$}}%
%
\polyline(-1,-0.071)(-1,0.071)%
%
\settowidth{\Width}{$-1$}\setlength{\Width}{-0.5\Width}%
\settoheight{\Height}{$-1$}\settodepth{\Depth}{$-1$}\setlength{\Height}{-\Height}%
\put( -1.000, -0.143){\hspace*{\Width}\raisebox{\Height}{$-1$}}%
%
\polyline(1,-0.071)(1,0.071)%
%
\settowidth{\Width}{$1$}\setlength{\Width}{-0.5\Width}%
\settoheight{\Height}{$1$}\settodepth{\Depth}{$1$}\setlength{\Height}{-\Height}%
\put(  1.000, -0.143){\hspace*{\Width}\raisebox{\Height}{$1$}}%
%
\polyline(2,-0.071)(2,0.071)%
%
\settowidth{\Width}{$2$}\setlength{\Width}{-0.5\Width}%
\settoheight{\Height}{$2$}\settodepth{\Depth}{$2$}\setlength{\Height}{-\Height}%
\put(  2.000, -0.143){\hspace*{\Width}\raisebox{\Height}{$2$}}%
%
\polyline(3,-0.071)(3,0.071)%
%
\settowidth{\Width}{$3$}\setlength{\Width}{-0.5\Width}%
\settoheight{\Height}{$3$}\settodepth{\Depth}{$3$}\setlength{\Height}{-\Height}%
\put(  3.000, -0.143){\hspace*{\Width}\raisebox{\Height}{$3$}}%
%
\polyline(4,-0.071)(4,0.071)%
%
\settowidth{\Width}{$4$}\setlength{\Width}{-0.5\Width}%
\settoheight{\Height}{$4$}\settodepth{\Depth}{$4$}\setlength{\Height}{-\Height}%
\put(  4.000, -0.143){\hspace*{\Width}\raisebox{\Height}{$4$}}%
%
\polyline(-0.071,-4)(0.071,-4)%
%
\settowidth{\Width}{$-4$}\setlength{\Width}{-1\Width}%
\settoheight{\Height}{$-4$}\settodepth{\Depth}{$-4$}\setlength{\Height}{-0.5\Height}\setlength{\Depth}{0.5\Depth}\addtolength{\Height}{\Depth}%
\put( -0.143, -4.000){\hspace*{\Width}\raisebox{\Height}{$-4$}}%
%
\polyline(-0.071,-3)(0.071,-3)%
%
\settowidth{\Width}{$-3$}\setlength{\Width}{-1\Width}%
\settoheight{\Height}{$-3$}\settodepth{\Depth}{$-3$}\setlength{\Height}{-0.5\Height}\setlength{\Depth}{0.5\Depth}\addtolength{\Height}{\Depth}%
\put( -0.143, -3.000){\hspace*{\Width}\raisebox{\Height}{$-3$}}%
%
\polyline(-0.071,-2)(0.071,-2)%
%
\settowidth{\Width}{$-2$}\setlength{\Width}{-1\Width}%
\settoheight{\Height}{$-2$}\settodepth{\Depth}{$-2$}\setlength{\Height}{-0.5\Height}\setlength{\Depth}{0.5\Depth}\addtolength{\Height}{\Depth}%
\put( -0.143, -2.000){\hspace*{\Width}\raisebox{\Height}{$-2$}}%
%
\polyline(-0.071,-1)(0.071,-1)%
%
\settowidth{\Width}{$-1$}\setlength{\Width}{-1\Width}%
\settoheight{\Height}{$-1$}\settodepth{\Depth}{$-1$}\setlength{\Height}{-0.5\Height}\setlength{\Depth}{0.5\Depth}\addtolength{\Height}{\Depth}%
\put( -0.143, -1.000){\hspace*{\Width}\raisebox{\Height}{$-1$}}%
%
\polyline(-0.071,1)(0.071,1)%
%
\settowidth{\Width}{$1$}\setlength{\Width}{-1\Width}%
\settoheight{\Height}{$1$}\settodepth{\Depth}{$1$}\setlength{\Height}{-0.5\Height}\setlength{\Depth}{0.5\Depth}\addtolength{\Height}{\Depth}%
\put( -0.143,  1.000){\hspace*{\Width}\raisebox{\Height}{$1$}}%
%
\polyline(-0.071,2)(0.071,2)%
%
\settowidth{\Width}{$2$}\setlength{\Width}{-1\Width}%
\settoheight{\Height}{$2$}\settodepth{\Depth}{$2$}\setlength{\Height}{-0.5\Height}\setlength{\Depth}{0.5\Depth}\addtolength{\Height}{\Depth}%
\put( -0.143,  2.000){\hspace*{\Width}\raisebox{\Height}{$2$}}%
%
\polyline(-0.071,3)(0.071,3)%
%
\settowidth{\Width}{$3$}\setlength{\Width}{-1\Width}%
\settoheight{\Height}{$3$}\settodepth{\Depth}{$3$}\setlength{\Height}{-0.5\Height}\setlength{\Depth}{0.5\Depth}\addtolength{\Height}{\Depth}%
\put( -0.143,  3.000){\hspace*{\Width}\raisebox{\Height}{$3$}}%
%
\polyline(-0.071,4)(0.071,4)%
%
\settowidth{\Width}{$4$}\setlength{\Width}{-1\Width}%
\settoheight{\Height}{$4$}\settodepth{\Depth}{$4$}\setlength{\Height}{-0.5\Height}\setlength{\Depth}{0.5\Depth}\addtolength{\Height}{\Depth}%
\put( -0.143,  4.000){\hspace*{\Width}\raisebox{\Height}{$4$}}%
%
\polyline(-5,0)(5,0)%
%
\polyline(0,-5)(0,5)%
%
\settowidth{\Width}{$x$}\setlength{\Width}{0\Width}%
\settoheight{\Height}{$x$}\settodepth{\Depth}{$x$}\setlength{\Height}{-0.5\Height}\setlength{\Depth}{0.5\Depth}\addtolength{\Height}{\Depth}%
\put(  5.071,  0.000){\hspace*{\Width}\raisebox{\Height}{$x$}}%
%
\settowidth{\Width}{$y$}\setlength{\Width}{-0.5\Width}%
\settoheight{\Height}{$y$}\settodepth{\Depth}{$y$}\setlength{\Height}{\Depth}%
\put(  0.000,  5.071){\hspace*{\Width}\raisebox{\Height}{$y$}}%
%
\settowidth{\Width}{O}\setlength{\Width}{-1\Width}%
\settoheight{\Height}{O}\settodepth{\Depth}{O}\setlength{\Height}{-\Height}%
\put( -0.071, -0.071){\hspace*{\Width}\raisebox{\Height}{O}}%
%
\end{picture}}%}}
\arrowlineseg{80}{30}{20}{-30}
\putnotee{97}{40}{頂点}
\arrowlineseg{80}{20}{20}{0}
\putnotee{100}{20}{軸}
\putnotese{55}{5}{\scalebox{0.7}{%%% /polytech22.git/102-0418/presen/fig/parabola2.tex 
%%% Generator=presen22102.cdy 
{\unitlength=7mm%
\begin{picture}%
(10,10)(-5,-5)%
\linethickness{0.008in}%%
\Large\bf\boldmath%
\small%
{%
\color[cmyk]{0,1,1,0}%
\polyline(-2.23478,-5.00000)(-2.20000,-4.84000)(-2.00000,-4.00000)(-1.80000,-3.24000)%
(-1.60000,-2.56000)(-1.40000,-1.96000)(-1.20000,-1.44000)(-1.00000,-1.00000)(-0.80000,-0.64000)%
(-0.60000,-0.36000)(-0.40000,-0.16000)(-0.20000,-0.04000)(0.00000,0.00000)(0.20000,-0.04000)%
(0.40000,-0.16000)(0.60000,-0.36000)(0.80000,-0.64000)(1.00000,-1.00000)(1.20000,-1.44000)%
(1.40000,-1.96000)(1.60000,-2.56000)(1.80000,-3.24000)(2.00000,-4.00000)(2.20000,-4.84000)%
(2.23478,-5.00000)%
%
}%
\polyline(-4.00000,0.07143)(-4.00000,-0.07143)%
%
\settowidth{\Width}{$-4$}\setlength{\Width}{-0.5\Width}%
\settoheight{\Height}{$-4$}\settodepth{\Depth}{$-4$}\setlength{\Height}{-\Height}%
\put(-4.0000000,-0.1428571){\hspace*{\Width}\raisebox{\Height}{$-4$}}%
%
\polyline(-3.00000,0.07143)(-3.00000,-0.07143)%
%
\settowidth{\Width}{$-3$}\setlength{\Width}{-0.5\Width}%
\settoheight{\Height}{$-3$}\settodepth{\Depth}{$-3$}\setlength{\Height}{-\Height}%
\put(-3.0000000,-0.1428571){\hspace*{\Width}\raisebox{\Height}{$-3$}}%
%
\polyline(-2.00000,0.07143)(-2.00000,-0.07143)%
%
\settowidth{\Width}{$-2$}\setlength{\Width}{-0.5\Width}%
\settoheight{\Height}{$-2$}\settodepth{\Depth}{$-2$}\setlength{\Height}{-\Height}%
\put(-2.0000000,-0.1428571){\hspace*{\Width}\raisebox{\Height}{$-2$}}%
%
\polyline(-1.00000,0.07143)(-1.00000,-0.07143)%
%
\settowidth{\Width}{$-1$}\setlength{\Width}{-0.5\Width}%
\settoheight{\Height}{$-1$}\settodepth{\Depth}{$-1$}\setlength{\Height}{-\Height}%
\put(-1.0000000,-0.1428571){\hspace*{\Width}\raisebox{\Height}{$-1$}}%
%
\polyline(1.00000,0.07143)(1.00000,-0.07143)%
%
\settowidth{\Width}{$1$}\setlength{\Width}{-0.5\Width}%
\settoheight{\Height}{$1$}\settodepth{\Depth}{$1$}\setlength{\Height}{-\Height}%
\put(1.0000000,-0.1428571){\hspace*{\Width}\raisebox{\Height}{$1$}}%
%
\polyline(2.00000,0.07143)(2.00000,-0.07143)%
%
\settowidth{\Width}{$2$}\setlength{\Width}{-0.5\Width}%
\settoheight{\Height}{$2$}\settodepth{\Depth}{$2$}\setlength{\Height}{-\Height}%
\put(2.0000000,-0.1428571){\hspace*{\Width}\raisebox{\Height}{$2$}}%
%
\polyline(3.00000,0.07143)(3.00000,-0.07143)%
%
\settowidth{\Width}{$3$}\setlength{\Width}{-0.5\Width}%
\settoheight{\Height}{$3$}\settodepth{\Depth}{$3$}\setlength{\Height}{-\Height}%
\put(3.0000000,-0.1428571){\hspace*{\Width}\raisebox{\Height}{$3$}}%
%
\polyline(4.00000,0.07143)(4.00000,-0.07143)%
%
\settowidth{\Width}{$4$}\setlength{\Width}{-0.5\Width}%
\settoheight{\Height}{$4$}\settodepth{\Depth}{$4$}\setlength{\Height}{-\Height}%
\put(4.0000000,-0.1428571){\hspace*{\Width}\raisebox{\Height}{$4$}}%
%
\polyline(0.07143,-4.00000)(-0.07143,-4.00000)%
%
\settowidth{\Width}{$-4$}\setlength{\Width}{-1\Width}%
\settoheight{\Height}{$-4$}\settodepth{\Depth}{$-4$}\setlength{\Height}{-0.5\Height}\setlength{\Depth}{0.5\Depth}\addtolength{\Height}{\Depth}%
\put(-0.1428571,-4.0000000){\hspace*{\Width}\raisebox{\Height}{$-4$}}%
%
\polyline(0.07143,-3.00000)(-0.07143,-3.00000)%
%
\settowidth{\Width}{$-3$}\setlength{\Width}{-1\Width}%
\settoheight{\Height}{$-3$}\settodepth{\Depth}{$-3$}\setlength{\Height}{-0.5\Height}\setlength{\Depth}{0.5\Depth}\addtolength{\Height}{\Depth}%
\put(-0.1428571,-3.0000000){\hspace*{\Width}\raisebox{\Height}{$-3$}}%
%
\polyline(0.07143,-2.00000)(-0.07143,-2.00000)%
%
\settowidth{\Width}{$-2$}\setlength{\Width}{-1\Width}%
\settoheight{\Height}{$-2$}\settodepth{\Depth}{$-2$}\setlength{\Height}{-0.5\Height}\setlength{\Depth}{0.5\Depth}\addtolength{\Height}{\Depth}%
\put(-0.1428571,-2.0000000){\hspace*{\Width}\raisebox{\Height}{$-2$}}%
%
\polyline(0.07143,-1.00000)(-0.07143,-1.00000)%
%
\settowidth{\Width}{$-1$}\setlength{\Width}{-1\Width}%
\settoheight{\Height}{$-1$}\settodepth{\Depth}{$-1$}\setlength{\Height}{-0.5\Height}\setlength{\Depth}{0.5\Depth}\addtolength{\Height}{\Depth}%
\put(-0.1428571,-1.0000000){\hspace*{\Width}\raisebox{\Height}{$-1$}}%
%
\polyline(0.07143,1.00000)(-0.07143,1.00000)%
%
\settowidth{\Width}{$1$}\setlength{\Width}{-1\Width}%
\settoheight{\Height}{$1$}\settodepth{\Depth}{$1$}\setlength{\Height}{-0.5\Height}\setlength{\Depth}{0.5\Depth}\addtolength{\Height}{\Depth}%
\put(-0.1428571,1.0000000){\hspace*{\Width}\raisebox{\Height}{$1$}}%
%
\polyline(0.07143,2.00000)(-0.07143,2.00000)%
%
\settowidth{\Width}{$2$}\setlength{\Width}{-1\Width}%
\settoheight{\Height}{$2$}\settodepth{\Depth}{$2$}\setlength{\Height}{-0.5\Height}\setlength{\Depth}{0.5\Depth}\addtolength{\Height}{\Depth}%
\put(-0.1428571,2.0000000){\hspace*{\Width}\raisebox{\Height}{$2$}}%
%
\polyline(0.07143,3.00000)(-0.07143,3.00000)%
%
\settowidth{\Width}{$3$}\setlength{\Width}{-1\Width}%
\settoheight{\Height}{$3$}\settodepth{\Depth}{$3$}\setlength{\Height}{-0.5\Height}\setlength{\Depth}{0.5\Depth}\addtolength{\Height}{\Depth}%
\put(-0.1428571,3.0000000){\hspace*{\Width}\raisebox{\Height}{$3$}}%
%
\polyline(0.07143,4.00000)(-0.07143,4.00000)%
%
\settowidth{\Width}{$4$}\setlength{\Width}{-1\Width}%
\settoheight{\Height}{$4$}\settodepth{\Depth}{$4$}\setlength{\Height}{-0.5\Height}\setlength{\Depth}{0.5\Depth}\addtolength{\Height}{\Depth}%
\put(-0.1428571,4.0000000){\hspace*{\Width}\raisebox{\Height}{$4$}}%
%
\polyline(-5.00000,0.00000)(5.00000,0.00000)%
%
\polyline(0.00000,-5.00000)(0.00000,5.00000)%
%
\settowidth{\Width}{$x$}\setlength{\Width}{0\Width}%
\settoheight{\Height}{$x$}\settodepth{\Depth}{$x$}\setlength{\Height}{-0.5\Height}\setlength{\Depth}{0.5\Depth}\addtolength{\Height}{\Depth}%
\put(5.0714286,0.0000000){\hspace*{\Width}\raisebox{\Height}{$x$}}%
%
\settowidth{\Width}{$y$}\setlength{\Width}{-0.5\Width}%
\settoheight{\Height}{$y$}\settodepth{\Depth}{$y$}\setlength{\Height}{\Depth}%
\put(0.0000000,5.0714286){\hspace*{\Width}\raisebox{\Height}{$y$}}%
%
\settowidth{\Width}{O}\setlength{\Width}{-1\Width}%
\settoheight{\Height}{O}\settodepth{\Depth}{O}\setlength{\Height}{-\Height}%
\put(-0.0714286,-0.0714286){\hspace*{\Width}\raisebox{\Height}{O}}%
%
\end{picture}}%}}
\end{layer}

\begin{itemize}
\item
$y=x^2$
\item
[] 軸は$x=0$($y$軸)
\item
[] 頂点は$(0,\ 0)$
\item
[] 下に凸
\item
$y=-x^2$
\end{itemize}

\sameslide

\vspace*{18mm}

\slidepage
\down
「2.関数のグラフ」で$y=x^2,\ y=-x^2$をかこう.

\begin{layer}{120}{0}
\putnotese{55}{5}{\scalebox{0.7}{%%% /Users/takatoosetsuo/polytech23.git/102-0424/presen/fig/parabola1.tex 
%%% Generator=presen23102.cdy 
{\unitlength=7mm%
\begin{picture}%
(10,10)(-5,-5)%
\linethickness{0.008in}%%
\Large\bf\boldmath%
\small%
\linethickness{0.012in}%%
\polyline(-2.235,5)(-2.2,4.84)(-2,4)(-1.8,3.24)(-1.6,2.56)(-1.4,1.96)(-1.2,1.44)(-1,1)%
(-0.8,0.64)(-0.6,0.36)(-0.4,0.16)(-0.2,0.04)(0,0)(0.2,0.04)(0.4,0.16)(0.6,0.36)(0.8,0.64)%
(1,1)(1.2,1.44)(1.4,1.96)(1.6,2.56)(1.8,3.24)(2,4)(2.2,4.84)(2.235,5)%
%
\linethickness{0.008in}%%
\polyline(-4,-0.071)(-4,0.071)%
%
\settowidth{\Width}{$-4$}\setlength{\Width}{-0.5\Width}%
\settoheight{\Height}{$-4$}\settodepth{\Depth}{$-4$}\setlength{\Height}{-\Height}%
\put( -4.000, -0.143){\hspace*{\Width}\raisebox{\Height}{$-4$}}%
%
\polyline(-3,-0.071)(-3,0.071)%
%
\settowidth{\Width}{$-3$}\setlength{\Width}{-0.5\Width}%
\settoheight{\Height}{$-3$}\settodepth{\Depth}{$-3$}\setlength{\Height}{-\Height}%
\put( -3.000, -0.143){\hspace*{\Width}\raisebox{\Height}{$-3$}}%
%
\polyline(-2,-0.071)(-2,0.071)%
%
\settowidth{\Width}{$-2$}\setlength{\Width}{-0.5\Width}%
\settoheight{\Height}{$-2$}\settodepth{\Depth}{$-2$}\setlength{\Height}{-\Height}%
\put( -2.000, -0.143){\hspace*{\Width}\raisebox{\Height}{$-2$}}%
%
\polyline(-1,-0.071)(-1,0.071)%
%
\settowidth{\Width}{$-1$}\setlength{\Width}{-0.5\Width}%
\settoheight{\Height}{$-1$}\settodepth{\Depth}{$-1$}\setlength{\Height}{-\Height}%
\put( -1.000, -0.143){\hspace*{\Width}\raisebox{\Height}{$-1$}}%
%
\polyline(1,-0.071)(1,0.071)%
%
\settowidth{\Width}{$1$}\setlength{\Width}{-0.5\Width}%
\settoheight{\Height}{$1$}\settodepth{\Depth}{$1$}\setlength{\Height}{-\Height}%
\put(  1.000, -0.143){\hspace*{\Width}\raisebox{\Height}{$1$}}%
%
\polyline(2,-0.071)(2,0.071)%
%
\settowidth{\Width}{$2$}\setlength{\Width}{-0.5\Width}%
\settoheight{\Height}{$2$}\settodepth{\Depth}{$2$}\setlength{\Height}{-\Height}%
\put(  2.000, -0.143){\hspace*{\Width}\raisebox{\Height}{$2$}}%
%
\polyline(3,-0.071)(3,0.071)%
%
\settowidth{\Width}{$3$}\setlength{\Width}{-0.5\Width}%
\settoheight{\Height}{$3$}\settodepth{\Depth}{$3$}\setlength{\Height}{-\Height}%
\put(  3.000, -0.143){\hspace*{\Width}\raisebox{\Height}{$3$}}%
%
\polyline(4,-0.071)(4,0.071)%
%
\settowidth{\Width}{$4$}\setlength{\Width}{-0.5\Width}%
\settoheight{\Height}{$4$}\settodepth{\Depth}{$4$}\setlength{\Height}{-\Height}%
\put(  4.000, -0.143){\hspace*{\Width}\raisebox{\Height}{$4$}}%
%
\polyline(-0.071,-4)(0.071,-4)%
%
\settowidth{\Width}{$-4$}\setlength{\Width}{-1\Width}%
\settoheight{\Height}{$-4$}\settodepth{\Depth}{$-4$}\setlength{\Height}{-0.5\Height}\setlength{\Depth}{0.5\Depth}\addtolength{\Height}{\Depth}%
\put( -0.143, -4.000){\hspace*{\Width}\raisebox{\Height}{$-4$}}%
%
\polyline(-0.071,-3)(0.071,-3)%
%
\settowidth{\Width}{$-3$}\setlength{\Width}{-1\Width}%
\settoheight{\Height}{$-3$}\settodepth{\Depth}{$-3$}\setlength{\Height}{-0.5\Height}\setlength{\Depth}{0.5\Depth}\addtolength{\Height}{\Depth}%
\put( -0.143, -3.000){\hspace*{\Width}\raisebox{\Height}{$-3$}}%
%
\polyline(-0.071,-2)(0.071,-2)%
%
\settowidth{\Width}{$-2$}\setlength{\Width}{-1\Width}%
\settoheight{\Height}{$-2$}\settodepth{\Depth}{$-2$}\setlength{\Height}{-0.5\Height}\setlength{\Depth}{0.5\Depth}\addtolength{\Height}{\Depth}%
\put( -0.143, -2.000){\hspace*{\Width}\raisebox{\Height}{$-2$}}%
%
\polyline(-0.071,-1)(0.071,-1)%
%
\settowidth{\Width}{$-1$}\setlength{\Width}{-1\Width}%
\settoheight{\Height}{$-1$}\settodepth{\Depth}{$-1$}\setlength{\Height}{-0.5\Height}\setlength{\Depth}{0.5\Depth}\addtolength{\Height}{\Depth}%
\put( -0.143, -1.000){\hspace*{\Width}\raisebox{\Height}{$-1$}}%
%
\polyline(-0.071,1)(0.071,1)%
%
\settowidth{\Width}{$1$}\setlength{\Width}{-1\Width}%
\settoheight{\Height}{$1$}\settodepth{\Depth}{$1$}\setlength{\Height}{-0.5\Height}\setlength{\Depth}{0.5\Depth}\addtolength{\Height}{\Depth}%
\put( -0.143,  1.000){\hspace*{\Width}\raisebox{\Height}{$1$}}%
%
\polyline(-0.071,2)(0.071,2)%
%
\settowidth{\Width}{$2$}\setlength{\Width}{-1\Width}%
\settoheight{\Height}{$2$}\settodepth{\Depth}{$2$}\setlength{\Height}{-0.5\Height}\setlength{\Depth}{0.5\Depth}\addtolength{\Height}{\Depth}%
\put( -0.143,  2.000){\hspace*{\Width}\raisebox{\Height}{$2$}}%
%
\polyline(-0.071,3)(0.071,3)%
%
\settowidth{\Width}{$3$}\setlength{\Width}{-1\Width}%
\settoheight{\Height}{$3$}\settodepth{\Depth}{$3$}\setlength{\Height}{-0.5\Height}\setlength{\Depth}{0.5\Depth}\addtolength{\Height}{\Depth}%
\put( -0.143,  3.000){\hspace*{\Width}\raisebox{\Height}{$3$}}%
%
\polyline(-0.071,4)(0.071,4)%
%
\settowidth{\Width}{$4$}\setlength{\Width}{-1\Width}%
\settoheight{\Height}{$4$}\settodepth{\Depth}{$4$}\setlength{\Height}{-0.5\Height}\setlength{\Depth}{0.5\Depth}\addtolength{\Height}{\Depth}%
\put( -0.143,  4.000){\hspace*{\Width}\raisebox{\Height}{$4$}}%
%
\polyline(-5,0)(5,0)%
%
\polyline(0,-5)(0,5)%
%
\settowidth{\Width}{$x$}\setlength{\Width}{0\Width}%
\settoheight{\Height}{$x$}\settodepth{\Depth}{$x$}\setlength{\Height}{-0.5\Height}\setlength{\Depth}{0.5\Depth}\addtolength{\Height}{\Depth}%
\put(  5.071,  0.000){\hspace*{\Width}\raisebox{\Height}{$x$}}%
%
\settowidth{\Width}{$y$}\setlength{\Width}{-0.5\Width}%
\settoheight{\Height}{$y$}\settodepth{\Depth}{$y$}\setlength{\Height}{\Depth}%
\put(  0.000,  5.071){\hspace*{\Width}\raisebox{\Height}{$y$}}%
%
\settowidth{\Width}{O}\setlength{\Width}{-1\Width}%
\settoheight{\Height}{O}\settodepth{\Depth}{O}\setlength{\Height}{-\Height}%
\put( -0.071, -0.071){\hspace*{\Width}\raisebox{\Height}{O}}%
%
\end{picture}}%}}
\arrowlineseg{80}{30}{20}{-30}
\putnotee{97}{40}{頂点}
\arrowlineseg{80}{20}{20}{0}
\putnotee{100}{20}{軸}
\putnotese{55}{5}{\scalebox{0.7}{%%% /polytech22.git/102-0418/presen/fig/parabola2.tex 
%%% Generator=presen22102.cdy 
{\unitlength=7mm%
\begin{picture}%
(10,10)(-5,-5)%
\linethickness{0.008in}%%
\Large\bf\boldmath%
\small%
{%
\color[cmyk]{0,1,1,0}%
\polyline(-2.23478,-5.00000)(-2.20000,-4.84000)(-2.00000,-4.00000)(-1.80000,-3.24000)%
(-1.60000,-2.56000)(-1.40000,-1.96000)(-1.20000,-1.44000)(-1.00000,-1.00000)(-0.80000,-0.64000)%
(-0.60000,-0.36000)(-0.40000,-0.16000)(-0.20000,-0.04000)(0.00000,0.00000)(0.20000,-0.04000)%
(0.40000,-0.16000)(0.60000,-0.36000)(0.80000,-0.64000)(1.00000,-1.00000)(1.20000,-1.44000)%
(1.40000,-1.96000)(1.60000,-2.56000)(1.80000,-3.24000)(2.00000,-4.00000)(2.20000,-4.84000)%
(2.23478,-5.00000)%
%
}%
\polyline(-4.00000,0.07143)(-4.00000,-0.07143)%
%
\settowidth{\Width}{$-4$}\setlength{\Width}{-0.5\Width}%
\settoheight{\Height}{$-4$}\settodepth{\Depth}{$-4$}\setlength{\Height}{-\Height}%
\put(-4.0000000,-0.1428571){\hspace*{\Width}\raisebox{\Height}{$-4$}}%
%
\polyline(-3.00000,0.07143)(-3.00000,-0.07143)%
%
\settowidth{\Width}{$-3$}\setlength{\Width}{-0.5\Width}%
\settoheight{\Height}{$-3$}\settodepth{\Depth}{$-3$}\setlength{\Height}{-\Height}%
\put(-3.0000000,-0.1428571){\hspace*{\Width}\raisebox{\Height}{$-3$}}%
%
\polyline(-2.00000,0.07143)(-2.00000,-0.07143)%
%
\settowidth{\Width}{$-2$}\setlength{\Width}{-0.5\Width}%
\settoheight{\Height}{$-2$}\settodepth{\Depth}{$-2$}\setlength{\Height}{-\Height}%
\put(-2.0000000,-0.1428571){\hspace*{\Width}\raisebox{\Height}{$-2$}}%
%
\polyline(-1.00000,0.07143)(-1.00000,-0.07143)%
%
\settowidth{\Width}{$-1$}\setlength{\Width}{-0.5\Width}%
\settoheight{\Height}{$-1$}\settodepth{\Depth}{$-1$}\setlength{\Height}{-\Height}%
\put(-1.0000000,-0.1428571){\hspace*{\Width}\raisebox{\Height}{$-1$}}%
%
\polyline(1.00000,0.07143)(1.00000,-0.07143)%
%
\settowidth{\Width}{$1$}\setlength{\Width}{-0.5\Width}%
\settoheight{\Height}{$1$}\settodepth{\Depth}{$1$}\setlength{\Height}{-\Height}%
\put(1.0000000,-0.1428571){\hspace*{\Width}\raisebox{\Height}{$1$}}%
%
\polyline(2.00000,0.07143)(2.00000,-0.07143)%
%
\settowidth{\Width}{$2$}\setlength{\Width}{-0.5\Width}%
\settoheight{\Height}{$2$}\settodepth{\Depth}{$2$}\setlength{\Height}{-\Height}%
\put(2.0000000,-0.1428571){\hspace*{\Width}\raisebox{\Height}{$2$}}%
%
\polyline(3.00000,0.07143)(3.00000,-0.07143)%
%
\settowidth{\Width}{$3$}\setlength{\Width}{-0.5\Width}%
\settoheight{\Height}{$3$}\settodepth{\Depth}{$3$}\setlength{\Height}{-\Height}%
\put(3.0000000,-0.1428571){\hspace*{\Width}\raisebox{\Height}{$3$}}%
%
\polyline(4.00000,0.07143)(4.00000,-0.07143)%
%
\settowidth{\Width}{$4$}\setlength{\Width}{-0.5\Width}%
\settoheight{\Height}{$4$}\settodepth{\Depth}{$4$}\setlength{\Height}{-\Height}%
\put(4.0000000,-0.1428571){\hspace*{\Width}\raisebox{\Height}{$4$}}%
%
\polyline(0.07143,-4.00000)(-0.07143,-4.00000)%
%
\settowidth{\Width}{$-4$}\setlength{\Width}{-1\Width}%
\settoheight{\Height}{$-4$}\settodepth{\Depth}{$-4$}\setlength{\Height}{-0.5\Height}\setlength{\Depth}{0.5\Depth}\addtolength{\Height}{\Depth}%
\put(-0.1428571,-4.0000000){\hspace*{\Width}\raisebox{\Height}{$-4$}}%
%
\polyline(0.07143,-3.00000)(-0.07143,-3.00000)%
%
\settowidth{\Width}{$-3$}\setlength{\Width}{-1\Width}%
\settoheight{\Height}{$-3$}\settodepth{\Depth}{$-3$}\setlength{\Height}{-0.5\Height}\setlength{\Depth}{0.5\Depth}\addtolength{\Height}{\Depth}%
\put(-0.1428571,-3.0000000){\hspace*{\Width}\raisebox{\Height}{$-3$}}%
%
\polyline(0.07143,-2.00000)(-0.07143,-2.00000)%
%
\settowidth{\Width}{$-2$}\setlength{\Width}{-1\Width}%
\settoheight{\Height}{$-2$}\settodepth{\Depth}{$-2$}\setlength{\Height}{-0.5\Height}\setlength{\Depth}{0.5\Depth}\addtolength{\Height}{\Depth}%
\put(-0.1428571,-2.0000000){\hspace*{\Width}\raisebox{\Height}{$-2$}}%
%
\polyline(0.07143,-1.00000)(-0.07143,-1.00000)%
%
\settowidth{\Width}{$-1$}\setlength{\Width}{-1\Width}%
\settoheight{\Height}{$-1$}\settodepth{\Depth}{$-1$}\setlength{\Height}{-0.5\Height}\setlength{\Depth}{0.5\Depth}\addtolength{\Height}{\Depth}%
\put(-0.1428571,-1.0000000){\hspace*{\Width}\raisebox{\Height}{$-1$}}%
%
\polyline(0.07143,1.00000)(-0.07143,1.00000)%
%
\settowidth{\Width}{$1$}\setlength{\Width}{-1\Width}%
\settoheight{\Height}{$1$}\settodepth{\Depth}{$1$}\setlength{\Height}{-0.5\Height}\setlength{\Depth}{0.5\Depth}\addtolength{\Height}{\Depth}%
\put(-0.1428571,1.0000000){\hspace*{\Width}\raisebox{\Height}{$1$}}%
%
\polyline(0.07143,2.00000)(-0.07143,2.00000)%
%
\settowidth{\Width}{$2$}\setlength{\Width}{-1\Width}%
\settoheight{\Height}{$2$}\settodepth{\Depth}{$2$}\setlength{\Height}{-0.5\Height}\setlength{\Depth}{0.5\Depth}\addtolength{\Height}{\Depth}%
\put(-0.1428571,2.0000000){\hspace*{\Width}\raisebox{\Height}{$2$}}%
%
\polyline(0.07143,3.00000)(-0.07143,3.00000)%
%
\settowidth{\Width}{$3$}\setlength{\Width}{-1\Width}%
\settoheight{\Height}{$3$}\settodepth{\Depth}{$3$}\setlength{\Height}{-0.5\Height}\setlength{\Depth}{0.5\Depth}\addtolength{\Height}{\Depth}%
\put(-0.1428571,3.0000000){\hspace*{\Width}\raisebox{\Height}{$3$}}%
%
\polyline(0.07143,4.00000)(-0.07143,4.00000)%
%
\settowidth{\Width}{$4$}\setlength{\Width}{-1\Width}%
\settoheight{\Height}{$4$}\settodepth{\Depth}{$4$}\setlength{\Height}{-0.5\Height}\setlength{\Depth}{0.5\Depth}\addtolength{\Height}{\Depth}%
\put(-0.1428571,4.0000000){\hspace*{\Width}\raisebox{\Height}{$4$}}%
%
\polyline(-5.00000,0.00000)(5.00000,0.00000)%
%
\polyline(0.00000,-5.00000)(0.00000,5.00000)%
%
\settowidth{\Width}{$x$}\setlength{\Width}{0\Width}%
\settoheight{\Height}{$x$}\settodepth{\Depth}{$x$}\setlength{\Height}{-0.5\Height}\setlength{\Depth}{0.5\Depth}\addtolength{\Height}{\Depth}%
\put(5.0714286,0.0000000){\hspace*{\Width}\raisebox{\Height}{$x$}}%
%
\settowidth{\Width}{$y$}\setlength{\Width}{-0.5\Width}%
\settoheight{\Height}{$y$}\settodepth{\Depth}{$y$}\setlength{\Height}{\Depth}%
\put(0.0000000,5.0714286){\hspace*{\Width}\raisebox{\Height}{$y$}}%
%
\settowidth{\Width}{O}\setlength{\Width}{-1\Width}%
\settoheight{\Height}{O}\settodepth{\Depth}{O}\setlength{\Height}{-\Height}%
\put(-0.0714286,-0.0714286){\hspace*{\Width}\raisebox{\Height}{O}}%
%
\end{picture}}%}}
\end{layer}

\begin{itemize}
\item
$y=x^2$
\item
[] 軸は$x=0$($y$軸)
\item
[] 頂点は$(0,\ 0)$
\item
[] 下に凸
\item
$y=-x^2$
\item
[] 上に凸
\end{itemize}

\newslide{2次関数のグラフ2}

\vspace*{18mm}

\slidepage

\begin{layer}{120}{0}
\putnotese{75}{18}{\scalebox{0.8}{%%% /Users/takatoosetsuo/polytech23.git/102-0424/presen/fig/idou0.tex 
%%% Generator=presen23102.cdy 
{\unitlength=1cm%
\begin{picture}%
(7,5.5)(-3.5,-0.5)%
\linethickness{0.008in}%%
\Large\bf\boldmath%
\small%
\linethickness{0.012in}%%
\polyline(-2.236,5)(-2.1,4.41)(-1.96,3.842)(-1.82,3.312)(-1.68,2.822)(-1.54,2.372)%
(-1.4,1.96)(-1.26,1.588)(-1.12,1.254)(-0.98,0.96)(-0.84,0.706)(-0.7,0.49)(-0.56,0.314)%
(-0.42,0.176)(-0.28,0.078)(-0.14,0.02)(0,0)(0.14,0.02)(0.28,0.078)(0.42,0.176)(0.56,0.314)%
(0.7,0.49)(0.84,0.706)(0.98,0.96)(1.12,1.254)(1.26,1.588)(1.4,1.96)(1.54,2.372)(1.68,2.822)%
(1.82,3.312)(1.96,3.842)(2.1,4.41)(2.236,5)%
%
\linethickness{0.008in}%%
\polyline(-3.5,0)(3.5,0)%
%
\polyline(0,-0.5)(0,5)%
%
\settowidth{\Width}{$x$}\setlength{\Width}{0\Width}%
\settoheight{\Height}{$x$}\settodepth{\Depth}{$x$}\setlength{\Height}{-0.5\Height}\setlength{\Depth}{0.5\Depth}\addtolength{\Height}{\Depth}%
\put(  3.550,  0.000){\hspace*{\Width}\raisebox{\Height}{$x$}}%
%
\settowidth{\Width}{$y$}\setlength{\Width}{-0.5\Width}%
\settoheight{\Height}{$y$}\settodepth{\Depth}{$y$}\setlength{\Height}{\Depth}%
\put(  0.000,  5.050){\hspace*{\Width}\raisebox{\Height}{$y$}}%
%
\settowidth{\Width}{O}\setlength{\Width}{-1\Width}%
\settoheight{\Height}{O}\settodepth{\Depth}{O}\setlength{\Height}{-\Height}%
\put( -0.050, -0.050){\hspace*{\Width}\raisebox{\Height}{O}}%
%
\end{picture}}%}}
\end{layer}

カッコ内の定数を変えたときのグラフをかこう.
\begin{enumerate}[(1)]
\item
$y=ax^2$(定数$a$)\vspace{-2mm}
\end{enumerate}
%%%%%%%%%%%%

%%%%%%%%%%%%%%%%%%%%


\sameslide

\vspace*{18mm}

\slidepage

\begin{layer}{120}{0}
\putnotese{75}{18}{\scalebox{0.8}{%%% /polytech22.git/102-0418/presen/fig/idou1.tex 
%%% Generator=presen22102.cdy 
{\unitlength=1cm%
\begin{picture}%
(5.5,5.5)(-2,-0.5)%
\linethickness{0.008in}%%
\Large\bf\boldmath%
\small%
\linethickness{0.012in}%%
\polyline(-2.00000,4.00000)(-1.89000,3.57210)(-1.78000,3.16840)(-1.67000,2.78890)%
(-1.56000,2.43360)(-1.45000,2.10250)(-1.34000,1.79560)(-1.23000,1.51290)(-1.12000,1.25440)%
(-1.01000,1.02010)(-0.90000,0.81000)(-0.79000,0.62410)(-0.68000,0.46240)(-0.57000,0.32490)%
(-0.46000,0.21160)(-0.35000,0.12250)(-0.24000,0.05760)(-0.13000,0.01690)(-0.02000,0.00040)%
(0.09000,0.00810)(0.20000,0.04000)(0.31000,0.09610)(0.42000,0.17640)(0.53000,0.28090)%
(0.64000,0.40960)(0.75000,0.56250)(0.86000,0.73960)(0.97000,0.94090)(1.08000,1.16640)%
(1.19000,1.41610)(1.30000,1.69000)(1.41000,1.98810)(1.52000,2.31040)(1.63000,2.65690)%
(1.74000,3.02760)(1.85000,3.42250)(1.96000,3.84160)(2.07000,4.28490)(2.18000,4.75240)%
(2.23539,5.00000)%
%
\linethickness{0.008in}%%
{%
\color[cmyk]{0,1,1,0}%
\linethickness{0.012in}%%
\polyline(-1.58056,5.00000)(-1.56000,4.86720)(-1.45000,4.20500)(-1.34000,3.59120)%
(-1.23000,3.02580)(-1.12000,2.50880)(-1.01000,2.04020)(-0.90000,1.62000)(-0.79000,1.24820)%
(-0.68000,0.92480)(-0.57000,0.64980)(-0.46000,0.42320)(-0.35000,0.24500)(-0.24000,0.11520)%
(-0.13000,0.03380)(-0.02000,0.00080)(0.09000,0.01620)(0.20000,0.08000)(0.31000,0.19220)%
(0.42000,0.35280)(0.53000,0.56180)(0.64000,0.81920)(0.75000,1.12500)(0.86000,1.47920)%
(0.97000,1.88180)(1.08000,2.33280)(1.19000,2.83220)(1.30000,3.38000)(1.41000,3.97620)%
(1.52000,4.62080)(1.58019,5.00000)%
%
\linethickness{0.008in}%%
}%
\polyline(-2.00000,0.00000)(3.50000,0.00000)%
%
\polyline(0.00000,-0.50000)(0.00000,5.00000)%
%
\settowidth{\Width}{$x$}\setlength{\Width}{0\Width}%
\settoheight{\Height}{$x$}\settodepth{\Depth}{$x$}\setlength{\Height}{-0.5\Height}\setlength{\Depth}{0.5\Depth}\addtolength{\Height}{\Depth}%
\put(3.5500000,0.0000000){\hspace*{\Width}\raisebox{\Height}{$x$}}%
%
\settowidth{\Width}{$y$}\setlength{\Width}{-0.5\Width}%
\settoheight{\Height}{$y$}\settodepth{\Depth}{$y$}\setlength{\Height}{\Depth}%
\put(0.0000000,5.0500000){\hspace*{\Width}\raisebox{\Height}{$y$}}%
%
\settowidth{\Width}{O}\setlength{\Width}{-1\Width}%
\settoheight{\Height}{O}\settodepth{\Depth}{O}\setlength{\Height}{-\Height}%
\put(-0.0500000,-0.0500000){\hspace*{\Width}\raisebox{\Height}{O}}%
%
\end{picture}}%}}
\end{layer}

カッコ内の定数を変えたときのグラフをかこう.
\begin{enumerate}[(1)]
\item
$y=ax^2$(定数$a$)\vspace{-2mm}
\item
[] {\color{blue}開き(増え方)が変わる}\vspace{-2mm}
\end{enumerate}

\sameslide

\vspace*{18mm}

\slidepage

\begin{layer}{120}{0}
\putnotese{75}{18}{\scalebox{0.8}{%%% /Users/takatoosetsuo/polytech23.git/102-0424/presen/fig/idou0.tex 
%%% Generator=presen23102.cdy 
{\unitlength=1cm%
\begin{picture}%
(7,5.5)(-3.5,-0.5)%
\linethickness{0.008in}%%
\Large\bf\boldmath%
\small%
\linethickness{0.012in}%%
\polyline(-2.236,5)(-2.1,4.41)(-1.96,3.842)(-1.82,3.312)(-1.68,2.822)(-1.54,2.372)%
(-1.4,1.96)(-1.26,1.588)(-1.12,1.254)(-0.98,0.96)(-0.84,0.706)(-0.7,0.49)(-0.56,0.314)%
(-0.42,0.176)(-0.28,0.078)(-0.14,0.02)(0,0)(0.14,0.02)(0.28,0.078)(0.42,0.176)(0.56,0.314)%
(0.7,0.49)(0.84,0.706)(0.98,0.96)(1.12,1.254)(1.26,1.588)(1.4,1.96)(1.54,2.372)(1.68,2.822)%
(1.82,3.312)(1.96,3.842)(2.1,4.41)(2.236,5)%
%
\linethickness{0.008in}%%
\polyline(-3.5,0)(3.5,0)%
%
\polyline(0,-0.5)(0,5)%
%
\settowidth{\Width}{$x$}\setlength{\Width}{0\Width}%
\settoheight{\Height}{$x$}\settodepth{\Depth}{$x$}\setlength{\Height}{-0.5\Height}\setlength{\Depth}{0.5\Depth}\addtolength{\Height}{\Depth}%
\put(  3.550,  0.000){\hspace*{\Width}\raisebox{\Height}{$x$}}%
%
\settowidth{\Width}{$y$}\setlength{\Width}{-0.5\Width}%
\settoheight{\Height}{$y$}\settodepth{\Depth}{$y$}\setlength{\Height}{\Depth}%
\put(  0.000,  5.050){\hspace*{\Width}\raisebox{\Height}{$y$}}%
%
\settowidth{\Width}{O}\setlength{\Width}{-1\Width}%
\settoheight{\Height}{O}\settodepth{\Depth}{O}\setlength{\Height}{-\Height}%
\put( -0.050, -0.050){\hspace*{\Width}\raisebox{\Height}{O}}%
%
\end{picture}}%}}
\end{layer}

カッコ内の定数を変えたときのグラフをかこう.
\begin{enumerate}[(1)]
\item
$y=ax^2$(定数$a$)\vspace{-2mm}
\item
[] {\color{blue}開き(増え方)が変わる}\vspace{-2mm}
\item
$y=ax^2+c$(定数$c$)\vspace{-2mm}
\end{enumerate}

\sameslide

\vspace*{18mm}

\slidepage

\begin{layer}{120}{0}
\putnotese{75}{18}{\scalebox{0.8}{%%% /polytech22.git/102-0418/presen/fig/idou2.tex 
%%% Generator=presen22102.cdy 
{\unitlength=1cm%
\begin{picture}%
(5.5,5.5)(-2,-0.5)%
\linethickness{0.008in}%%
\Large\bf\boldmath%
\small%
\linethickness{0.012in}%%
\polyline(-2.00000,4.00000)(-1.89000,3.57210)(-1.78000,3.16840)(-1.67000,2.78890)%
(-1.56000,2.43360)(-1.45000,2.10250)(-1.34000,1.79560)(-1.23000,1.51290)(-1.12000,1.25440)%
(-1.01000,1.02010)(-0.90000,0.81000)(-0.79000,0.62410)(-0.68000,0.46240)(-0.57000,0.32490)%
(-0.46000,0.21160)(-0.35000,0.12250)(-0.24000,0.05760)(-0.13000,0.01690)(-0.02000,0.00040)%
(0.09000,0.00810)(0.20000,0.04000)(0.31000,0.09610)(0.42000,0.17640)(0.53000,0.28090)%
(0.64000,0.40960)(0.75000,0.56250)(0.86000,0.73960)(0.97000,0.94090)(1.08000,1.16640)%
(1.19000,1.41610)(1.30000,1.69000)(1.41000,1.98810)(1.52000,2.31040)(1.63000,2.65690)%
(1.74000,3.02760)(1.85000,3.42250)(1.96000,3.84160)(2.07000,4.28490)(2.18000,4.75240)%
(2.23539,5.00000)%
%
\linethickness{0.008in}%%
{%
\color[cmyk]{0,1,1,0}%
\linethickness{0.012in}%%
\polyline(-1.73119,5.00000)(-1.67000,4.78890)(-1.56000,4.43360)(-1.45000,4.10250)%
(-1.34000,3.79560)(-1.23000,3.51290)(-1.12000,3.25440)(-1.01000,3.02010)(-0.90000,2.81000)%
(-0.79000,2.62410)(-0.68000,2.46240)(-0.57000,2.32490)(-0.46000,2.21160)(-0.35000,2.12250)%
(-0.24000,2.05760)(-0.13000,2.01690)(-0.02000,2.00040)(0.09000,2.00810)(0.20000,2.04000)%
(0.31000,2.09610)(0.42000,2.17640)(0.53000,2.28090)(0.64000,2.40960)(0.75000,2.56250)%
(0.86000,2.73960)(0.97000,2.94090)(1.08000,3.16640)(1.19000,3.41610)(1.30000,3.69000)%
(1.41000,3.98810)(1.52000,4.31040)(1.63000,4.65690)(1.73181,5.00000)%
%
\linethickness{0.008in}%%
}%
\polyline(-2.00000,0.00000)(3.50000,0.00000)%
%
\polyline(0.00000,-0.50000)(0.00000,5.00000)%
%
\settowidth{\Width}{$x$}\setlength{\Width}{0\Width}%
\settoheight{\Height}{$x$}\settodepth{\Depth}{$x$}\setlength{\Height}{-0.5\Height}\setlength{\Depth}{0.5\Depth}\addtolength{\Height}{\Depth}%
\put(3.5500000,0.0000000){\hspace*{\Width}\raisebox{\Height}{$x$}}%
%
\settowidth{\Width}{$y$}\setlength{\Width}{-0.5\Width}%
\settoheight{\Height}{$y$}\settodepth{\Depth}{$y$}\setlength{\Height}{\Depth}%
\put(0.0000000,5.0500000){\hspace*{\Width}\raisebox{\Height}{$y$}}%
%
\settowidth{\Width}{O}\setlength{\Width}{-1\Width}%
\settoheight{\Height}{O}\settodepth{\Depth}{O}\setlength{\Height}{-\Height}%
\put(-0.0500000,-0.0500000){\hspace*{\Width}\raisebox{\Height}{O}}%
%
\end{picture}}%}}
\end{layer}

カッコ内の定数を変えたときのグラフをかこう.
\begin{enumerate}[(1)]
\item
$y=ax^2$(定数$a$)\vspace{-2mm}
\item
[] {\color{blue}開き(増え方)が変わる}\vspace{-2mm}
\item
$y=ax^2+c$(定数$c$)\vspace{-2mm}
\item
[] {\color{blue}縦方向に$c$だけ平行移動}\vspace{-2mm}
\end{enumerate}

\sameslide

\vspace*{18mm}

\slidepage

\begin{layer}{120}{0}
\putnotese{75}{18}{\scalebox{0.8}{%%% /Users/takatoosetsuo/polytech23.git/102-0424/presen/fig/idou0.tex 
%%% Generator=presen23102.cdy 
{\unitlength=1cm%
\begin{picture}%
(7,5.5)(-3.5,-0.5)%
\linethickness{0.008in}%%
\Large\bf\boldmath%
\small%
\linethickness{0.012in}%%
\polyline(-2.236,5)(-2.1,4.41)(-1.96,3.842)(-1.82,3.312)(-1.68,2.822)(-1.54,2.372)%
(-1.4,1.96)(-1.26,1.588)(-1.12,1.254)(-0.98,0.96)(-0.84,0.706)(-0.7,0.49)(-0.56,0.314)%
(-0.42,0.176)(-0.28,0.078)(-0.14,0.02)(0,0)(0.14,0.02)(0.28,0.078)(0.42,0.176)(0.56,0.314)%
(0.7,0.49)(0.84,0.706)(0.98,0.96)(1.12,1.254)(1.26,1.588)(1.4,1.96)(1.54,2.372)(1.68,2.822)%
(1.82,3.312)(1.96,3.842)(2.1,4.41)(2.236,5)%
%
\linethickness{0.008in}%%
\polyline(-3.5,0)(3.5,0)%
%
\polyline(0,-0.5)(0,5)%
%
\settowidth{\Width}{$x$}\setlength{\Width}{0\Width}%
\settoheight{\Height}{$x$}\settodepth{\Depth}{$x$}\setlength{\Height}{-0.5\Height}\setlength{\Depth}{0.5\Depth}\addtolength{\Height}{\Depth}%
\put(  3.550,  0.000){\hspace*{\Width}\raisebox{\Height}{$x$}}%
%
\settowidth{\Width}{$y$}\setlength{\Width}{-0.5\Width}%
\settoheight{\Height}{$y$}\settodepth{\Depth}{$y$}\setlength{\Height}{\Depth}%
\put(  0.000,  5.050){\hspace*{\Width}\raisebox{\Height}{$y$}}%
%
\settowidth{\Width}{O}\setlength{\Width}{-1\Width}%
\settoheight{\Height}{O}\settodepth{\Depth}{O}\setlength{\Height}{-\Height}%
\put( -0.050, -0.050){\hspace*{\Width}\raisebox{\Height}{O}}%
%
\end{picture}}%}}
\end{layer}

カッコ内の定数を変えたときのグラフをかこう.
\begin{enumerate}[(1)]
\item
$y=ax^2$(定数$a$)\vspace{-2mm}
\item
[] {\color{blue}開き(増え方)が変わる}\vspace{-2mm}
\item
$y=ax^2+c$(定数$c$)\vspace{-2mm}
\item
[] {\color{blue}縦方向に$c$だけ平行移動}\vspace{-2mm}
\item
$y=a(x-b)^2$(定数$b$)\vspace{-2mm}
\end{enumerate}

\sameslide

\vspace*{18mm}

\slidepage

\begin{layer}{120}{0}
\putnotese{75}{18}{\scalebox{0.8}{%%% /Users/takatoosetsuo/polytech23.git/102-0424/presen/fig/idou3.tex 
%%% Generator=presen23102.cdy 
{\unitlength=1cm%
\begin{picture}%
(7,5.5)(-3.5,-0.5)%
\linethickness{0.008in}%%
\Large\bf\boldmath%
\small%
\linethickness{0.012in}%%
\polyline(-2.236,5)(-2.1,4.41)(-1.96,3.842)(-1.82,3.312)(-1.68,2.822)(-1.54,2.372)%
(-1.4,1.96)(-1.26,1.588)(-1.12,1.254)(-0.98,0.96)(-0.84,0.706)(-0.7,0.49)(-0.56,0.314)%
(-0.42,0.176)(-0.28,0.078)(-0.14,0.02)(0,0)(0.14,0.02)(0.28,0.078)(0.42,0.176)(0.56,0.314)%
(0.7,0.49)(0.84,0.706)(0.98,0.96)(1.12,1.254)(1.26,1.588)(1.4,1.96)(1.54,2.372)(1.68,2.822)%
(1.82,3.312)(1.96,3.842)(2.1,4.41)(2.236,5)%
%
\linethickness{0.008in}%%
{%
\color[cmyk]{0,1,1,0}%
\linethickness{0.012in}%%
\polyline(-1.235,5)(-1.12,4.494)(-0.98,3.92)(-0.84,3.386)(-0.7,2.89)(-0.56,2.434)%
(-0.42,2.016)(-0.28,1.638)(-0.14,1.3)(0,1)(0.14,0.74)(0.28,0.518)(0.42,0.336)(0.56,0.194)%
(0.7,0.09)(0.84,0.026)(0.98,0)(1.12,0.014)(1.26,0.068)(1.4,0.16)(1.54,0.292)(1.68,0.462)%
(1.82,0.672)(1.96,0.922)(2.1,1.21)(2.24,1.538)(2.38,1.904)(2.52,2.31)(2.66,2.756)%
(2.8,3.24)(2.94,3.764)(3.08,4.326)(3.22,4.928)(3.236,5)%
%
\linethickness{0.008in}%%
}%
\polyline(-3.5,0)(3.5,0)%
%
\polyline(0,-0.5)(0,5)%
%
\settowidth{\Width}{$x$}\setlength{\Width}{0\Width}%
\settoheight{\Height}{$x$}\settodepth{\Depth}{$x$}\setlength{\Height}{-0.5\Height}\setlength{\Depth}{0.5\Depth}\addtolength{\Height}{\Depth}%
\put(  3.550,  0.000){\hspace*{\Width}\raisebox{\Height}{$x$}}%
%
\settowidth{\Width}{$y$}\setlength{\Width}{-0.5\Width}%
\settoheight{\Height}{$y$}\settodepth{\Depth}{$y$}\setlength{\Height}{\Depth}%
\put(  0.000,  5.050){\hspace*{\Width}\raisebox{\Height}{$y$}}%
%
\settowidth{\Width}{O}\setlength{\Width}{-1\Width}%
\settoheight{\Height}{O}\settodepth{\Depth}{O}\setlength{\Height}{-\Height}%
\put( -0.050, -0.050){\hspace*{\Width}\raisebox{\Height}{O}}%
%
\end{picture}}%}}
\end{layer}

カッコ内の定数を変えたときのグラフをかこう.
\begin{enumerate}[(1)]
\item
$y=ax^2$(定数$a$)\vspace{-2mm}
\item
[] {\color{blue}開き(増え方)が変わる}\vspace{-2mm}
\item
$y=ax^2+c$(定数$c$)\vspace{-2mm}
\item
[] {\color{blue}縦方向に$c$だけ平行移動}\vspace{-2mm}
\item
$y=a(x-b)^2$(定数$b$)\vspace{-2mm}
\item
[] {\color{blue}横方向に$b$だけ平行移動}\vspace{-2mm}
\end{enumerate}

\sameslide

\vspace*{18mm}

\slidepage

\begin{layer}{120}{0}
\putnotese{75}{18}{\scalebox{0.8}{%%% /Users/takatoosetsuo/polytech23.git/102-0424/presen/fig/idou0.tex 
%%% Generator=presen23102.cdy 
{\unitlength=1cm%
\begin{picture}%
(7,5.5)(-3.5,-0.5)%
\linethickness{0.008in}%%
\Large\bf\boldmath%
\small%
\linethickness{0.012in}%%
\polyline(-2.236,5)(-2.1,4.41)(-1.96,3.842)(-1.82,3.312)(-1.68,2.822)(-1.54,2.372)%
(-1.4,1.96)(-1.26,1.588)(-1.12,1.254)(-0.98,0.96)(-0.84,0.706)(-0.7,0.49)(-0.56,0.314)%
(-0.42,0.176)(-0.28,0.078)(-0.14,0.02)(0,0)(0.14,0.02)(0.28,0.078)(0.42,0.176)(0.56,0.314)%
(0.7,0.49)(0.84,0.706)(0.98,0.96)(1.12,1.254)(1.26,1.588)(1.4,1.96)(1.54,2.372)(1.68,2.822)%
(1.82,3.312)(1.96,3.842)(2.1,4.41)(2.236,5)%
%
\linethickness{0.008in}%%
\polyline(-3.5,0)(3.5,0)%
%
\polyline(0,-0.5)(0,5)%
%
\settowidth{\Width}{$x$}\setlength{\Width}{0\Width}%
\settoheight{\Height}{$x$}\settodepth{\Depth}{$x$}\setlength{\Height}{-0.5\Height}\setlength{\Depth}{0.5\Depth}\addtolength{\Height}{\Depth}%
\put(  3.550,  0.000){\hspace*{\Width}\raisebox{\Height}{$x$}}%
%
\settowidth{\Width}{$y$}\setlength{\Width}{-0.5\Width}%
\settoheight{\Height}{$y$}\settodepth{\Depth}{$y$}\setlength{\Height}{\Depth}%
\put(  0.000,  5.050){\hspace*{\Width}\raisebox{\Height}{$y$}}%
%
\settowidth{\Width}{O}\setlength{\Width}{-1\Width}%
\settoheight{\Height}{O}\settodepth{\Depth}{O}\setlength{\Height}{-\Height}%
\put( -0.050, -0.050){\hspace*{\Width}\raisebox{\Height}{O}}%
%
\end{picture}}%}}
\end{layer}

カッコ内の定数を変えたときのグラフをかこう.
\begin{enumerate}[(1)]
\item
$y=ax^2$(定数$a$)\vspace{-2mm}
\item
[] {\color{blue}開き(増え方)が変わる}\vspace{-2mm}
\item
$y=ax^2+c$(定数$c$)\vspace{-2mm}
\item
[] {\color{blue}縦方向に$c$だけ平行移動}\vspace{-2mm}
\item
$y=a(x-b)^2$(定数$b$)\vspace{-2mm}
\item
[] {\color{blue}横方向に$b$だけ平行移動}\vspace{-2mm}
\item
$y=a(x-b)^2+c$(定数$b,\ c$)\vspace{-2mm}
\end{enumerate}

\sameslide

\vspace*{18mm}

\slidepage

\begin{layer}{120}{0}
\putnotese{75}{18}{\scalebox{0.8}{%%% /polytech22.git/102-0418/presen/fig/idou4.tex 
%%% Generator=presen22102.cdy 
{\unitlength=1cm%
\begin{picture}%
(5.5,5.5)(-2,-0.5)%
\linethickness{0.008in}%%
\Large\bf\boldmath%
\small%
\linethickness{0.012in}%%
\polyline(-2.00000,4.00000)(-1.89000,3.57210)(-1.78000,3.16840)(-1.67000,2.78890)%
(-1.56000,2.43360)(-1.45000,2.10250)(-1.34000,1.79560)(-1.23000,1.51290)(-1.12000,1.25440)%
(-1.01000,1.02010)(-0.90000,0.81000)(-0.79000,0.62410)(-0.68000,0.46240)(-0.57000,0.32490)%
(-0.46000,0.21160)(-0.35000,0.12250)(-0.24000,0.05760)(-0.13000,0.01690)(-0.02000,0.00040)%
(0.09000,0.00810)(0.20000,0.04000)(0.31000,0.09610)(0.42000,0.17640)(0.53000,0.28090)%
(0.64000,0.40960)(0.75000,0.56250)(0.86000,0.73960)(0.97000,0.94090)(1.08000,1.16640)%
(1.19000,1.41610)(1.30000,1.69000)(1.41000,1.98810)(1.52000,2.31040)(1.63000,2.65690)%
(1.74000,3.02760)(1.85000,3.42250)(1.96000,3.84160)(2.07000,4.28490)(2.18000,4.75240)%
(2.23539,5.00000)%
%
\linethickness{0.008in}%%
{%
\color[cmyk]{0,1,1,0}%
\linethickness{0.012in}%%
\polyline(-0.73118,5.00000)(-0.68000,4.82240)(-0.57000,4.46490)(-0.46000,4.13160)%
(-0.35000,3.82250)(-0.24000,3.53760)(-0.13000,3.27690)(-0.02000,3.04040)(0.09000,2.82810)%
(0.20000,2.64000)(0.31000,2.47610)(0.42000,2.33640)(0.53000,2.22090)(0.64000,2.12960)%
(0.75000,2.06250)(0.86000,2.01960)(0.97000,2.00090)(1.08000,2.00640)(1.19000,2.03610)%
(1.30000,2.09000)(1.41000,2.16810)(1.52000,2.27040)(1.63000,2.39690)(1.74000,2.54760)%
(1.85000,2.72250)(1.96000,2.92160)(2.07000,3.14490)(2.18000,3.39240)(2.29000,3.66410)%
(2.40000,3.96000)(2.51000,4.28010)(2.62000,4.62440)(2.73000,4.99290)(2.73199,5.00000)%
%
\linethickness{0.008in}%%
}%
\polyline(-2.00000,0.00000)(3.50000,0.00000)%
%
\polyline(0.00000,-0.50000)(0.00000,5.00000)%
%
\settowidth{\Width}{$x$}\setlength{\Width}{0\Width}%
\settoheight{\Height}{$x$}\settodepth{\Depth}{$x$}\setlength{\Height}{-0.5\Height}\setlength{\Depth}{0.5\Depth}\addtolength{\Height}{\Depth}%
\put(3.5500000,0.0000000){\hspace*{\Width}\raisebox{\Height}{$x$}}%
%
\settowidth{\Width}{$y$}\setlength{\Width}{-0.5\Width}%
\settoheight{\Height}{$y$}\settodepth{\Depth}{$y$}\setlength{\Height}{\Depth}%
\put(0.0000000,5.0500000){\hspace*{\Width}\raisebox{\Height}{$y$}}%
%
\settowidth{\Width}{O}\setlength{\Width}{-1\Width}%
\settoheight{\Height}{O}\settodepth{\Depth}{O}\setlength{\Height}{-\Height}%
\put(-0.0500000,-0.0500000){\hspace*{\Width}\raisebox{\Height}{O}}%
%
\end{picture}}%}}
\end{layer}

カッコ内の定数を変えたときのグラフをかこう.
\begin{enumerate}[(1)]
\item
$y=ax^2$(定数$a$)\vspace{-2mm}
\item
[] {\color{blue}開き(増え方)が変わる}\vspace{-2mm}
\item
$y=ax^2+c$(定数$c$)\vspace{-2mm}
\item
[] {\color{blue}縦方向に$c$だけ平行移動}\vspace{-2mm}
\item
$y=a(x-b)^2$(定数$b$)\vspace{-2mm}
\item
[] {\color{blue}横方向に$b$だけ平行移動}\vspace{-2mm}
\item
$y=a(x-b)^2+c$(定数$b,\ c$)\vspace{-2mm}
\item
[] {\color{blue}頂点の座標は $(b,\ c)$}
\end{enumerate}

\newslide{課題 2次関数のグラフ}

\vspace*{18mm}

%%repeat=2
\slidepage
\seteda{90}
\begin{enumerate}[(1)]
\item
[課題]\monban 「2.関数のグラフ」を用いて,次の2次関数のグラフをかけ.
また,$y=x^2$のグラフをどのように移動(変形)したかを答えよ.\\
\eda{$y=2x^2$}\\
\eda{$y=x^2+1$}\\
\eda{$y=(x-3)^2$}\\
\eda{$y=(x+1)^2$}\\
%%\eda{$y=-(x-2)^2$}
\end{enumerate}
%%%%%%%%%%%%

%%%%%%%%%%%%%%%%%%%%

\newslide{2次関数のグラフ3}

\vspace*{18mm}

\slidepage

\begin{layer}{120}{0}
\end{layer}

\begin{itemize}
\item
$y=x^2+2bx+c$
\end{itemize}
%%%%%%%%%%%%

%%%%%%%%%%%%%%%%%%%%


\sameslide

\vspace*{18mm}

\slidepage

\begin{layer}{120}{0}
\end{layer}

\begin{itemize}
\item
$y=x^2+2bx+c$\hfill{\color{blue}$\Longrightarrow (x+b)^2+d$の形に変形}
\end{itemize}

\sameslide

\vspace*{18mm}

\slidepage

\begin{layer}{120}{0}
\putnoten{90}{0}{{\color{red}$(x^2+2bx+b^2)+d$}}
\putnoten{90}{5}{{\color{red}\rotatebox[origin=c]{90}{$=$}}}
\end{layer}

\begin{itemize}
\item
$y=x^2+2bx+c$\hfill{\color{blue}$\Longrightarrow (x+b)^2+d$の形に変形}
\end{itemize}

\sameslide

\vspace*{18mm}

\slidepage

\begin{layer}{120}{0}
\putnoten{90}{0}{{\color{red}$(x^2+2bx+b^2)+d$}}
\putnoten{90}{5}{{\color{red}\rotatebox[origin=c]{90}{$=$}}}
\end{layer}

\begin{itemize}
\item
$y=x^2+2bx+c$\hfill{\color{blue}$\Longrightarrow (x+b)^2+d$の形に変形}
\item
[(例)] $y=x^2-2x+3$
\end{itemize}

\sameslide

\vspace*{18mm}

\slidepage

\begin{layer}{120}{0}
\putnoten{90}{0}{{\color{red}$(x^2+2bx+b^2)+d$}}
\putnoten{90}{5}{{\color{red}\rotatebox[origin=c]{90}{$=$}}}
\end{layer}

\begin{itemize}
\item
$y=x^2+2bx+c$\hfill{\color{blue}$\Longrightarrow (x+b)^2+d$の形に変形}
\item
[(例)] $y=x^2-2x+3$
\\ $\phantom{y}=(x^2-2x+1)-1+3$
\end{itemize}

\sameslide

\vspace*{18mm}

\slidepage

\begin{layer}{120}{0}
\putnoten{90}{0}{{\color{red}$(x^2+2bx+b^2)+d$}}
\putnoten{90}{5}{{\color{red}\rotatebox[origin=c]{90}{$=$}}}
\end{layer}

\begin{itemize}
\item
$y=x^2+2bx+c$\hfill{\color{blue}$\Longrightarrow (x+b)^2+d$の形に変形}
\item
[(例)] $y=x^2-2x+3$
\\ $\phantom{y}=(x^2-2x+1)-1+3$
\\$\phantom{y}=(x-1)^2+2$
\end{itemize}

\sameslide

\vspace*{18mm}

\slidepage

\begin{layer}{120}{0}
\putnoten{90}{0}{{\color{red}$(x^2+2bx+b^2)+d$}}
\putnoten{90}{5}{{\color{red}\rotatebox[origin=c]{90}{$=$}}}
\end{layer}

\begin{itemize}
\item
$y=x^2+2bx+c$\hfill{\color{blue}$\Longrightarrow (x+b)^2+d$の形に変形}
\item
[(例)] $y=x^2-2x+3$
\\ $\phantom{y}=(x^2-2x+1)-1+3$
\\$\phantom{y}=(x-1)^2+2$
\item
[(例)] $y=-x^2-4x+1$
\end{itemize}

\sameslide

\vspace*{18mm}

\slidepage

\begin{layer}{120}{0}
\putnoten{90}{0}{{\color{red}$(x^2+2bx+b^2)+d$}}
\putnoten{90}{5}{{\color{red}\rotatebox[origin=c]{90}{$=$}}}
\end{layer}

\begin{itemize}
\item
$y=x^2+2bx+c$\hfill{\color{blue}$\Longrightarrow (x+b)^2+d$の形に変形}
\item
[(例)] $y=x^2-2x+3$
\\ $\phantom{y}=(x^2-2x+1)-1+3$
\\$\phantom{y}=(x-1)^2+2$
\item
[(例)] $y=-x^2-4x+1$
\\ $\phantom{y}=-(x^2+4x)+1$
\end{itemize}

\sameslide

\vspace*{18mm}

\slidepage

\begin{layer}{120}{0}
\putnoten{90}{0}{{\color{red}$(x^2+2bx+b^2)+d$}}
\putnoten{90}{5}{{\color{red}\rotatebox[origin=c]{90}{$=$}}}
\end{layer}

\begin{itemize}
\item
$y=x^2+2bx+c$\hfill{\color{blue}$\Longrightarrow (x+b)^2+d$の形に変形}
\item
[(例)] $y=x^2-2x+3$
\\ $\phantom{y}=(x^2-2x+1)-1+3$
\\$\phantom{y}=(x-1)^2+2$
\item
[(例)] $y=-x^2-4x+1$
\\ $\phantom{y}=-(x^2+4x)+1$
\\ $\phantom{y}=-\bigl((x+2)^2-4\bigr)+1$
\end{itemize}

\sameslide

\vspace*{18mm}

\slidepage

\begin{layer}{120}{0}
\putnoten{90}{0}{{\color{red}$(x^2+2bx+b^2)+d$}}
\putnoten{90}{5}{{\color{red}\rotatebox[origin=c]{90}{$=$}}}
\putnotese{75}{20}{\scalebox{0.8}{%%% /polytech22.git/102-0418/presen/fig/idou6.tex 
%%% Generator=presen22102.cdy 
{\unitlength=1cm%
\begin{picture}%
(5.5,6)(-2,-0.5)%
\linethickness{0.008in}%%
\Large\bf\boldmath%
\small%
{%
\color[cmyk]{0,1,1,0}%
\linethickness{0.012in}%%
\polyline(-0.34510,-0.50000)(-0.24000,-0.01760)(-0.13000,0.46310)(-0.02000,0.91960)%
(0.09000,1.35190)(0.20000,1.76000)(0.31000,2.14390)(0.42000,2.50360)(0.53000,2.83910)%
(0.64000,3.15040)(0.75000,3.43750)(0.86000,3.70040)(0.97000,3.93910)(1.08000,4.15360)%
(1.19000,4.34390)(1.30000,4.51000)(1.41000,4.65190)(1.52000,4.76960)(1.63000,4.86310)%
(1.74000,4.93240)(1.85000,4.97750)(1.96000,4.99840)(2.07000,4.99510)(2.18000,4.96760)%
(2.29000,4.91590)(2.40000,4.84000)(2.51000,4.73990)(2.62000,4.61560)(2.73000,4.46710)%
(2.84000,4.29440)(2.95000,4.09750)(3.06000,3.87640)(3.17000,3.63110)(3.28000,3.36160)%
(3.39000,3.06790)(3.50000,2.75000)%
%
\linethickness{0.008in}%%
}%
\polyline(2.00000,0.00000)(2.00000,0.09859)\polyline(2.00000,0.19718)(2.00000,0.29577)%
\polyline(2.00000,0.39437)(2.00000,0.49296)\polyline(2.00000,0.59155)(2.00000,0.69014)%
\polyline(2.00000,0.78873)(2.00000,0.88732)\polyline(2.00000,0.98592)(2.00000,1.08451)%
\polyline(2.00000,1.18310)(2.00000,1.28169)\polyline(2.00000,1.38028)(2.00000,1.47887)%
\polyline(2.00000,1.57746)(2.00000,1.67606)\polyline(2.00000,1.77465)(2.00000,1.87324)%
\polyline(2.00000,1.97183)(2.00000,2.07042)\polyline(2.00000,2.16901)(2.00000,2.26761)%
\polyline(2.00000,2.36620)(2.00000,2.46479)\polyline(2.00000,2.56338)(2.00000,2.66197)%
\polyline(2.00000,2.76056)(2.00000,2.85915)\polyline(2.00000,2.95775)(2.00000,3.05634)%
\polyline(2.00000,3.15493)(2.00000,3.25352)\polyline(2.00000,3.35211)(2.00000,3.45070)%
\polyline(2.00000,3.54930)(2.00000,3.64789)\polyline(2.00000,3.74648)(2.00000,3.84507)%
\polyline(2.00000,3.94366)(2.00000,4.04225)\polyline(2.00000,4.14085)(2.00000,4.23944)%
\polyline(2.00000,4.33803)(2.00000,4.43662)\polyline(2.00000,4.53521)(2.00000,4.63380)%
\polyline(2.00000,4.73239)(2.00000,4.83099)\polyline(2.00000,4.92958)(2.00000,5.00000)(1.97183,5.00000)%
\polyline(1.87324,5.00000)(1.77465,5.00000)\polyline(1.67606,5.00000)(1.57746,5.00000)%
\polyline(1.47887,5.00000)(1.38028,5.00000)\polyline(1.28169,5.00000)(1.18310,5.00000)%
\polyline(1.08451,5.00000)(0.98592,5.00000)\polyline(0.88732,5.00000)(0.78873,5.00000)%
\polyline(0.69014,5.00000)(0.59155,5.00000)\polyline(0.49296,5.00000)(0.39437,5.00000)%
\polyline(0.29577,5.00000)(0.19718,5.00000)\polyline(0.09859,5.00000)(0.00000,5.00000)%
%
%
\polyline(2.00000,0.05000)(2.00000,-0.05000)%
%
\settowidth{\Width}{$2$}\setlength{\Width}{-0.5\Width}%
\settoheight{\Height}{$2$}\settodepth{\Depth}{$2$}\setlength{\Height}{-\Height}%
\put(2.0000000,-0.1000000){\hspace*{\Width}\raisebox{\Height}{$2$}}%
%
\polyline(0.05000,5.00000)(-0.05000,5.00000)%
%
\settowidth{\Width}{$5$}\setlength{\Width}{-1\Width}%
\settoheight{\Height}{$5$}\settodepth{\Depth}{$5$}\setlength{\Height}{-0.5\Height}\setlength{\Depth}{0.5\Depth}\addtolength{\Height}{\Depth}%
\put(-0.1000000,5.0000000){\hspace*{\Width}\raisebox{\Height}{$5$}}%
%
\polyline(-2.00000,0.00000)(3.50000,0.00000)%
%
\polyline(0.00000,-0.50000)(0.00000,5.50000)%
%
\settowidth{\Width}{$x$}\setlength{\Width}{0\Width}%
\settoheight{\Height}{$x$}\settodepth{\Depth}{$x$}\setlength{\Height}{-0.5\Height}\setlength{\Depth}{0.5\Depth}\addtolength{\Height}{\Depth}%
\put(3.5500000,0.0000000){\hspace*{\Width}\raisebox{\Height}{$x$}}%
%
\settowidth{\Width}{$y$}\setlength{\Width}{-0.5\Width}%
\settoheight{\Height}{$y$}\settodepth{\Depth}{$y$}\setlength{\Height}{\Depth}%
\put(0.0000000,5.5500000){\hspace*{\Width}\raisebox{\Height}{$y$}}%
%
\settowidth{\Width}{O}\setlength{\Width}{-1\Width}%
\settoheight{\Height}{O}\settodepth{\Depth}{O}\setlength{\Height}{-\Height}%
\put(-0.0500000,-0.0500000){\hspace*{\Width}\raisebox{\Height}{O}}%
%
\end{picture}}%}}
\end{layer}

\begin{itemize}
\item
$y=x^2+2bx+c$\hfill{\color{blue}$\Longrightarrow (x+b)^2+d$の形に変形}
\item
[(例)] $y=x^2-2x+3$
\\ $\phantom{y}=(x^2-2x+1)-1+3$
\\$\phantom{y}=(x-1)^2+2$
\item
[(例)] $y=-x^2-4x+1$
\\ $\phantom{y}=-(x^2+4x)+1$
\\ $\phantom{y}=-\bigl((x+2)^2-4\bigr)+1$
\\$\phantom{y}=-(x+2)^2+5$
\end{itemize}

\newslide{課題(2次関数のグラフ)}

\vspace*{18mm}

%%repeat=9
\slidepage
\seteda{60}
\begin{itemize}
\item
[課題]\monban $a(x+b)^2+c$の形に変形せよ.\\
\eda{$y=x^2+4x-5$}\\
\eda{$y=x^2-2x-1$}\\
\eda{$y=-x^2-4x+1$}\\
\eda{$y=x^2+x+1$}
\end{itemize}
%%%%%%%%%%%%

%%%%%%%%%%%%%%%%%%%%

\mainslide{2次方程式}


\slidepage[m]
%%%%%%%%%%%%

%%%%%%%%%%%%%%%%%%%%

\newslide{2次式の因数分解}

\vspace*{18mm}

\slidepage
\begin{enumerate}[(1)]
\item
$x^2-a^2=(x+a)(x-a)$\\
\hspace*{2zw}$x^2-9$
\end{enumerate}
%%%%%%%%%%%%

%%%%%%%%%%%%%%%%%%%%


\sameslide

\vspace*{18mm}

\slidepage
\begin{enumerate}[(1)]
\item
$x^2-a^2=(x+a)(x-a)$\\
\hspace*{2zw}$x^2-9$
$=x^2-3^2$
\end{enumerate}

\sameslide

\vspace*{18mm}

\slidepage
\begin{enumerate}[(1)]
\item
$x^2-a^2=(x+a)(x-a)$\\
\hspace*{2zw}$x^2-9$
$=x^2-3^2$
$=(x+3)(x-3)$
\end{enumerate}

\sameslide

\vspace*{18mm}

\slidepage
\begin{enumerate}[(1)]
\item
$x^2-a^2=(x+a)(x-a)$\\
\hspace*{2zw}$x^2-9$
$=x^2-3^2$
$=(x+3)(x-3)$
\item
$x^2+2ax+a^2=(x+a)^2$\\
\hspace*{2zw}$x^2+4x+4$
\end{enumerate}

\sameslide

\vspace*{18mm}

\slidepage
\begin{enumerate}[(1)]
\item
$x^2-a^2=(x+a)(x-a)$\\
\hspace*{2zw}$x^2-9$
$=x^2-3^2$
$=(x+3)(x-3)$
\item
$x^2+2ax+a^2=(x+a)^2$\\
\hspace*{2zw}$x^2+4x+4$
$=x^2+2\cdot 2+2^2$
\end{enumerate}

\sameslide

\vspace*{18mm}

\slidepage
\begin{enumerate}[(1)]
\item
$x^2-a^2=(x+a)(x-a)$\\
\hspace*{2zw}$x^2-9$
$=x^2-3^2$
$=(x+3)(x-3)$
\item
$x^2+2ax+a^2=(x+a)^2$\\
\hspace*{2zw}$x^2+4x+4$
$=x^2+2\cdot 2+2^2$
$=(x+2)^2$
\end{enumerate}

\sameslide

\vspace*{18mm}

\slidepage
\begin{enumerate}[(1)]
\item
$x^2-a^2=(x+a)(x-a)$\\
\hspace*{2zw}$x^2-9$
$=x^2-3^2$
$=(x+3)(x-3)$
\item
$x^2+2ax+a^2=(x+a)^2$\\
\hspace*{2zw}$x^2+4x+4$
$=x^2+2\cdot 2+2^2$
$=(x+2)^2$
\item
$x^2+(a+b)x+ab=(x+a)(x+b)$\\
\end{enumerate}

\sameslide

\vspace*{18mm}

\slidepage
\begin{enumerate}[(1)]
\item
$x^2-a^2=(x+a)(x-a)$\\
\hspace*{2zw}$x^2-9$
$=x^2-3^2$
$=(x+3)(x-3)$
\item
$x^2+2ax+a^2=(x+a)^2$\\
\hspace*{2zw}$x^2+4x+4$
$=x^2+2\cdot 2+2^2$
$=(x+2)^2$
\item
$x^2+(a+b)x+ab=(x+a)(x+b)$\\
\hspace*{2zw}$x^2+5x+6$
\end{enumerate}

\sameslide

\vspace*{18mm}

\slidepage
\begin{enumerate}[(1)]
\item
$x^2-a^2=(x+a)(x-a)$\\
\hspace*{2zw}$x^2-9$
$=x^2-3^2$
$=(x+3)(x-3)$
\item
$x^2+2ax+a^2=(x+a)^2$\\
\hspace*{2zw}$x^2+4x+4$
$=x^2+2\cdot 2+2^2$
$=(x+2)^2$
\item
$x^2+(a+b)x+ab=(x+a)(x+b)$\\
\hspace*{2zw}$x^2+5x+6$
$=(x+2)(x+3)$\\
\end{enumerate}

\sameslide

\vspace*{18mm}

\slidepage
\begin{enumerate}[(1)]
\item
$x^2-a^2=(x+a)(x-a)$\\
\hspace*{2zw}$x^2-9$
$=x^2-3^2$
$=(x+3)(x-3)$
\item
$x^2+2ax+a^2=(x+a)^2$\\
\hspace*{2zw}$x^2+4x+4$
$=x^2+2\cdot 2+2^2$
$=(x+2)^2$
\item
$x^2+(a+b)x+ab=(x+a)(x+b)$\\
\hspace*{2zw}$x^2+5x+6$
$=(x+2)(x+3)$\\
\hspace*{2zw}$x^2-6x+8$
\end{enumerate}

\sameslide

\vspace*{18mm}

\slidepage
\begin{enumerate}[(1)]
\item
$x^2-a^2=(x+a)(x-a)$\\
\hspace*{2zw}$x^2-9$
$=x^2-3^2$
$=(x+3)(x-3)$
\item
$x^2+2ax+a^2=(x+a)^2$\\
\hspace*{2zw}$x^2+4x+4$
$=x^2+2\cdot 2+2^2$
$=(x+2)^2$
\item
$x^2+(a+b)x+ab=(x+a)(x+b)$\\
\hspace*{2zw}$x^2+5x+6$
$=(x+2)(x+3)$\\
\hspace*{2zw}$x^2-6x+8$
$=(x-2)(x-4)$
\end{enumerate}

\newslide{2次方程式(因数分解)}

\vspace*{18mm}

\slidepage
\begin{itemize}
\item
「$AB=0$ならば$A=0$または$B=0$」を用いる.\vspace{-2mm}
\item
[(例) ]$x^2-9=0$\\
\end{itemize}
%%%%%%%%%%%%

%%%%%%%%%%%%%%%%%%%%


\sameslide

\vspace*{18mm}

\slidepage
\begin{itemize}
\item
「$AB=0$ならば$A=0$または$B=0$」を用いる.\vspace{-2mm}
\item
[(例) ]$x^2-9=0$\\
\hspace*{1zw}$\Longleftrightarrow\ (x+3)(x-3)=0$\\
\end{itemize}

\sameslide

\vspace*{18mm}

\slidepage
\begin{itemize}
\item
「$AB=0$ならば$A=0$または$B=0$」を用いる.\vspace{-2mm}
\item
[(例) ]$x^2-9=0$\\
\hspace*{1zw}$\Longleftrightarrow\ (x+3)(x-3)=0$\\
\hspace*{1zw}$\Longleftrightarrow\ x=-3\ \mbox{(または)}\ x= 3$\\
\end{itemize}

\sameslide

\vspace*{18mm}

\slidepage
\begin{itemize}
\item
「$AB=0$ならば$A=0$または$B=0$」を用いる.\vspace{-2mm}
\item
[(例) ]$x^2-9=0$\\
\hspace*{1zw}$\Longleftrightarrow\ (x+3)(x-3)=0$\\
\hspace*{1zw}$\Longleftrightarrow\ x=-3\ \mbox{(または)}\ x= 3$\\
\hspace*{1zw}$\Longleftrightarrow\ x=\pm 3\ \ \mbox{\color{blue} と書く}$\vspace{-2mm}
\end{itemize}

\sameslide

\vspace*{18mm}

\slidepage
\begin{itemize}
\item
「$AB=0$ならば$A=0$または$B=0$」を用いる.\vspace{-2mm}
\item
[(例) ]$x^2-9=0$\\
\hspace*{1zw}$\Longleftrightarrow\ (x+3)(x-3)=0$\\
\hspace*{1zw}$\Longleftrightarrow\ x=-3\ \mbox{(または)}\ x= 3$\\
\hspace*{1zw}$\Longleftrightarrow\ x=\pm 3\ \ \mbox{\color{blue} と書く}$\vspace{-2mm}
\item
[課題]\monban 次の方程式を解け.\seteda{60}\\
\eda{$x^2-49=0$}\eda{$x^2-2x+1=0$}\\
\eda{$x^2-7x+12=0$}\eda{$x^2-x-20=0$}
\end{itemize}

\newslide{平方根}

\vspace*{18mm}

\slidepage
\begin{itemize}
\item
2乗して$4$になる数($x^2=4$となる$x$)
\end{itemize}
%%%%%%%%%%%%

%%%%%%%%%%%%%%%%%%%%


\sameslide

\vspace*{18mm}

\slidepage
\begin{itemize}
\item
2乗して$4$になる数($x^2=4$となる$x$)
\item
[]\hspace*{2zw}$\Longrightarrow\ 2,\ -2$の2つがある.
\end{itemize}

\sameslide

\vspace*{18mm}

\slidepage
\begin{itemize}
\item
2乗して$4$になる数($x^2=4$となる$x$)
\item
[]\hspace*{2zw}$\Longrightarrow\ 2,\ -2$の2つがある.
\item
このうち,正の方の$2$を$\sqrt{4}$とかく
\end{itemize}

\sameslide

\vspace*{18mm}

\slidepage
\begin{itemize}
\item
2乗して$4$になる数($x^2=4$となる$x$)
\item
[]\hspace*{2zw}$\Longrightarrow\ 2,\ -2$の2つがある.
\item
このうち,正の方の$2$を$\sqrt{4}$とかく
\item
正の数$a$について,2乗して$a$になる数のうち正の方をを$\sqrt{a}$とかく
\end{itemize}

\sameslide

\vspace*{18mm}

\slidepage
\begin{itemize}
\item
2乗して$4$になる数($x^2=4$となる$x$)
\item
[]\hspace*{2zw}$\Longrightarrow\ 2,\ -2$の2つがある.
\item
このうち,正の方の$2$を$\sqrt{4}$とかく
\item
正の数$a$について,2乗して$a$になる数のうち正の方をを$\sqrt{a}$とかく
\item
[]\hspace*{4zw}$(\sqrt{a})^2=a,\ (-\sqrt{a})^2=a$
\end{itemize}

\newslide{平方根の性質}

\vspace*{18mm}

\slidepage
\begin{itemize}
\item
$a>0$のとき,$\sqrt{a^2}=a$\\
\end{itemize}
%%%%%%%%%%%%

%%%%%%%%%%%%%%%%%%%%


\sameslide

\vspace*{18mm}

\slidepage
\begin{itemize}
\item
$a>0$のとき,$\sqrt{a^2}=a$\\
\hspace*{2zw}2乗して$4^2(=16)$になるのは$4$と$-4$\\
\end{itemize}

\sameslide

\vspace*{18mm}

\slidepage
\begin{itemize}
\item
$a>0$のとき,$\sqrt{a^2}=a$\\
\hspace*{2zw}2乗して$4^2(=16)$になるのは$4$と$-4$\\
\hspace*{2zw}正の方をとって,$\sqrt{4^2}=4$\vspace{-2mm}
\end{itemize}

\sameslide

\vspace*{18mm}

\slidepage
\begin{itemize}
\item
$a>0$のとき,$\sqrt{a^2}=a$\\
\hspace*{2zw}2乗して$4^2(=16)$になるのは$4$と$-4$\\
\hspace*{2zw}正の方をとって,$\sqrt{4^2}=4$\vspace{-2mm}
\item
[]$a<0$のとき,$\sqrt{a^2}=?$
\end{itemize}

\sameslide

\vspace*{18mm}

\slidepage
\begin{itemize}
\item
$a>0$のとき,$\sqrt{a^2}=a$\\
\hspace*{2zw}2乗して$4^2(=16)$になるのは$4$と$-4$\\
\hspace*{2zw}正の方をとって,$\sqrt{4^2}=4$\vspace{-2mm}
\item
[]$a<0$のとき,$\sqrt{a^2}=?$
\\\hspace*{2zw}2乗して$(-4)^2$になるのも$4$と$-4$
\end{itemize}

\sameslide

\vspace*{18mm}

\slidepage
\begin{itemize}
\item
$a>0$のとき,$\sqrt{a^2}=a$\\
\hspace*{2zw}2乗して$4^2(=16)$になるのは$4$と$-4$\\
\hspace*{2zw}正の方をとって,$\sqrt{4^2}=4$\vspace{-2mm}
\item
[]$a<0$のとき,$\sqrt{a^2}=?$
\\\hspace*{2zw}2乗して$(-4)^2$になるのも$4$と$-4$
\\\hspace*{2zw}正の方をとって,$\sqrt{(-4)^2}=4$\vspace{-2mm}
\end{itemize}

\sameslide

\vspace*{18mm}

\slidepage
\begin{itemize}
\item
$a>0$のとき,$\sqrt{a^2}=a$\\
\hspace*{2zw}2乗して$4^2(=16)$になるのは$4$と$-4$\\
\hspace*{2zw}正の方をとって,$\sqrt{4^2}=4$\vspace{-2mm}
\item
[]$a<0$のとき,$\sqrt{a^2}=-a$
\\\hspace*{2zw}2乗して$(-4)^2$になるのも$4$と$-4$
\\\hspace*{2zw}正の方をとって,$\sqrt{(-4)^2}=4$\vspace{-2mm}
\end{itemize}

\sameslide

\vspace*{18mm}

\slidepage
\begin{itemize}
\item
$a>0$のとき,$\sqrt{a^2}=a$\\
\hspace*{2zw}2乗して$4^2(=16)$になるのは$4$と$-4$\\
\hspace*{2zw}正の方をとって,$\sqrt{4^2}=4$\vspace{-2mm}
\item
[]$a<0$のとき,$\sqrt{a^2}=-a$
\\\hspace*{2zw}2乗して$(-4)^2$になるのも$4$と$-4$
\\\hspace*{2zw}正の方をとって,$\sqrt{(-4)^2}=4$\vspace{-2mm}
\item
{\color{red}$\sqrt{a^2}=|a|$}\vspace{-2mm}
\end{itemize}

\sameslide

\vspace*{18mm}

\slidepage
\begin{itemize}
\item
$a>0$のとき,$\sqrt{a^2}=a$\\
\hspace*{2zw}2乗して$4^2(=16)$になるのは$4$と$-4$\\
\hspace*{2zw}正の方をとって,$\sqrt{4^2}=4$\vspace{-2mm}
\item
[]$a<0$のとき,$\sqrt{a^2}=-a$
\\\hspace*{2zw}2乗して$(-4)^2$になるのも$4$と$-4$
\\\hspace*{2zw}正の方をとって,$\sqrt{(-4)^2}=4$\vspace{-2mm}
\item
{\color{red}$\sqrt{a^2}=|a|$}\vspace{-2mm}
\item
$b>0$のとき,$\sqrt{a^2b}=|a|\sqrt{b}$
\end{itemize}

\newslide{課題 平方根}

\vspace*{18mm}

%%repeat=6
\slidepage
\begin{itemize}
\item
[課題]\monban 次の数を根号を用いないで表せ\seteda{55}\hfill TextP17\vspace{2mm}\\
\eda{$-\sqrt{64}$}\eda{$\sqrt{\bunsuu{4}{9}}$}\\
\eda{$\bigl(-\sqrt{11}\bigr)^2$}\eda{$-\bigl(-\sqrt{3}\bigr)^2$}\\
\item
[課題]\monban 次を計算せよ($\sqrt{\phantom{5}}$の中を簡単にせよ)\seteda{55}\\
\eda{$-\sqrt{12}$}\eda{$\sqrt{18}$}\\
\eda{$\sqrt{27}-\sqrt{3}$}\eda{$\sqrt{100}\sqrt{8}$}
\end{itemize}
%%%%%%%%%%%%

%%%%%%%%%%%%%%%%%%%%

\newslide{2次方程式(平方完成)}

\vspace*{18mm}

\slidepage

\begin{layer}{120}{0}
\end{layer}

\begin{itemize}
\item
平方完成\\
\hspace*{1zw}$x^2+6x+2=$
\end{itemize}
%%%%%%%%%%%%

%%%%%%%%%%%%%%%%%%%%


\sameslide

\vspace*{18mm}

\slidepage

\begin{layer}{120}{0}
\putnotee{40}{10}{\normalsize\color{blue}$(x+a)^2=x^2+2ax+a^2$}
\end{layer}

\begin{itemize}
\item
平方完成\\
\hspace*{1zw}$x^2+6x+2=$
\end{itemize}

\sameslide

\vspace*{18mm}

\slidepage

\begin{layer}{120}{0}
\putnotee{40}{10}{\normalsize\color{blue}$(x+a)^2=x^2+2ax+a^2$}
\end{layer}

\begin{itemize}
\item
平方完成\\
\hspace*{1zw}$x^2+6x+2=$
$(x^2+6x+9)-9+2=$
\end{itemize}

\sameslide

\vspace*{18mm}

\slidepage

\begin{layer}{120}{0}
\putnotee{40}{10}{\normalsize\color{blue}$(x+a)^2=x^2+2ax+a^2$}
\end{layer}

\begin{itemize}
\item
平方完成\\
\hspace*{1zw}$x^2+6x+2=$
$(x^2+6x+9)-9+2=$
$(x+3)^2-7$\vspace{-2mm}
\end{itemize}

\sameslide

\vspace*{18mm}

\slidepage

\begin{layer}{120}{0}
\putnotee{40}{10}{\normalsize\color{blue}$(x+a)^2=x^2+2ax+a^2$}
\end{layer}

\begin{itemize}
\item
平方完成\\
\hspace*{1zw}$x^2+6x+2=$
$(x^2+6x+9)-9+2=$
$(x+3)^2-7$\vspace{-2mm}
\item
2次方程式$x^2+6x+2=0$
\end{itemize}

\sameslide

\vspace*{18mm}

\slidepage

\begin{layer}{120}{0}
\putnotee{40}{10}{\normalsize\color{blue}$(x+a)^2=x^2+2ax+a^2$}
\end{layer}

\begin{itemize}
\item
平方完成\\
\hspace*{1zw}$x^2+6x+2=$
$(x^2+6x+9)-9+2=$
$(x+3)^2-7$\vspace{-2mm}
\item
2次方程式$x^2+6x+2=0$
\\\hspace*{5zw}$(x+3)^2-7=0$
\end{itemize}

\sameslide

\vspace*{18mm}

\slidepage

\begin{layer}{120}{0}
\putnotee{40}{10}{\normalsize\color{blue}$(x+a)^2=x^2+2ax+a^2$}
\end{layer}

\begin{itemize}
\item
平方完成\\
\hspace*{1zw}$x^2+6x+2=$
$(x^2+6x+9)-9+2=$
$(x+3)^2-7$\vspace{-2mm}
\item
2次方程式$x^2+6x+2=0$
\\\hspace*{5zw}$(x+3)^2-7=0$
\\\hspace*{5zw}$(x+3)^2=7$
\end{itemize}

\sameslide

\vspace*{18mm}

\slidepage

\begin{layer}{120}{0}
\putnotee{40}{10}{\normalsize\color{blue}$(x+a)^2=x^2+2ax+a^2$}
\end{layer}

\begin{itemize}
\item
平方完成\\
\hspace*{1zw}$x^2+6x+2=$
$(x^2+6x+9)-9+2=$
$(x+3)^2-7$\vspace{-2mm}
\item
2次方程式$x^2+6x+2=0$
\\\hspace*{5zw}$(x+3)^2-7=0$
\\\hspace*{5zw}$(x+3)^2=7$
\\\hspace*{5zw}$x+3=\sqrt{7},\ -\sqrt{7}$
\\\Ltab{5.5zw}{合わせて}$x+3=\pm \sqrt{7}$
\end{itemize}

\sameslide

\vspace*{18mm}

\slidepage

\begin{layer}{120}{0}
\putnotee{40}{10}{\normalsize\color{blue}$(x+a)^2=x^2+2ax+a^2$}
\end{layer}

\begin{itemize}
\item
平方完成\\
\hspace*{1zw}$x^2+6x+2=$
$(x^2+6x+9)-9+2=$
$(x+3)^2-7$\vspace{-2mm}
\item
2次方程式$x^2+6x+2=0$
\\\hspace*{5zw}$(x+3)^2-7=0$
\\\hspace*{5zw}$(x+3)^2=7$
\\\hspace*{5zw}$x+3=\sqrt{7},\ -\sqrt{7}$
\\\Ltab{5.5zw}{合わせて}$x+3=\pm \sqrt{7}$
\\\hspace*{5zw}$x=-3\pm \sqrt{7}$
\end{itemize}

\newslide{解の公式1}

\vspace*{18mm}

\slidepage
\begin{itemize}
\item
$x^2+2ax+b=0$
\end{itemize}
%%%%%%%%%%%%

%%%%%%%%%%%%%%%%%%%%


\sameslide

\vspace*{18mm}

\slidepage
\begin{itemize}
\item
$x^2+2ax+b=0$
\\$(x+a)^2-a^2+b=0$
\end{itemize}

\sameslide

\vspace*{18mm}

\slidepage
\begin{itemize}
\item
$x^2+2ax+b=0$
\\$(x+a)^2-a^2+b=0$
\\$(x+a)^2=a^2-b$
\end{itemize}

\sameslide

\vspace*{18mm}

\slidepage
\begin{itemize}
\item
$x^2+2ax+b=0$
\\$(x+a)^2-a^2+b=0$
\\$(x+a)^2=a^2-b$
\\$x+a=\pm \sqrt{a^2-b}$
\end{itemize}

\sameslide

\vspace*{18mm}

\slidepage
\begin{itemize}
\item
$x^2+2ax+b=0$
\\$(x+a)^2-a^2+b=0$
\\$(x+a)^2=a^2-b$
\\$x+a=\pm \sqrt{a^2-b}$
\vspace{2mm}\\よって \fbox{\color{red}$x=-a\pm \sqrt{a^2-b}$}
\end{itemize}

\sameslide

\vspace*{18mm}

\slidepage
\begin{itemize}
\item
$x^2+2ax+b=0$
\\$(x+a)^2-a^2+b=0$
\\$(x+a)^2=a^2-b$
\\$x+a=\pm \sqrt{a^2-b}$
\vspace{2mm}\\よって \fbox{\color{red}$x=-a\pm \sqrt{a^2-b}$}
\item
[課題]\monban 次の2次方程式を解け.\seteda{57}\\
\eda{$x^2+4x+2=0$}\eda{$x^2+2x-2=0$}\\
\eda{$x^2-6x+1=0$}\eda{$x^2-8x+2=0$}
\end{itemize}

\newslide{解の公式}

\vspace*{18mm}

\slidepage
\begin{itemize}
\item
2次方程式$ax^2+bx+c=0$の解は\vspace{1mm}\\
\hspace*{3zw}\fbox{$x=\bunsuu{-b\pm\sqrt{b^2-4ac}}{2a}$}\vspace{-2mm}
\end{itemize}

\sameslide

\vspace*{18mm}

\slidepage
\begin{itemize}
\item
2次方程式$ax^2+bx+c=0$の解は\vspace{1mm}\\
\hspace*{3zw}\fbox{$x=\bunsuu{-b\pm\sqrt{b^2-4ac}}{2a}$}\vspace{-2mm}
\item
[]例)$2x^2-5x+1=0$\\
\end{itemize}

\sameslide

\vspace*{18mm}

\slidepage
\begin{itemize}
\item
2次方程式$ax^2+bx+c=0$の解は\vspace{1mm}\\
\hspace*{3zw}\fbox{$x=\bunsuu{-b\pm\sqrt{b^2-4ac}}{2a}$}\vspace{-2mm}
\item
[]例)$2x^2-5x+1=0$\\
\hspace*{3zw}$x=\bunsuu{5\pm \sqrt{5^2-4\cdot 2\cdot1}}{2\cdot 2}$
\end{itemize}

\sameslide

\vspace*{18mm}

\slidepage
\begin{itemize}
\item
2次方程式$ax^2+bx+c=0$の解は\vspace{1mm}\\
\hspace*{3zw}\fbox{$x=\bunsuu{-b\pm\sqrt{b^2-4ac}}{2a}$}\vspace{-2mm}
\item
[]例)$2x^2-5x+1=0$\\
\hspace*{3zw}$x=\bunsuu{5\pm \sqrt{5^2-4\cdot 2\cdot1}}{2\cdot 2}$
$=\bunsuu{5\pm \sqrt{17}}{4}$\vspace{-2mm}
\end{itemize}

\sameslide

\vspace*{18mm}

\slidepage
\begin{itemize}
\item
2次方程式$ax^2+bx+c=0$の解は\vspace{1mm}\\
\hspace*{3zw}\fbox{$x=\bunsuu{-b\pm\sqrt{b^2-4ac}}{2a}$}\vspace{-2mm}
\item
[]例)$2x^2-5x+1=0$\\
\hspace*{3zw}$x=\bunsuu{5\pm \sqrt{5^2-4\cdot 2\cdot1}}{2\cdot 2}$
$=\bunsuu{5\pm \sqrt{17}}{4}$\vspace{-2mm}
\item
[課題] \monban $ax^2+bx+c=0$より $x^2+\dfrac{b}{a}x+\dfrac{c}{a}=0$\\
 これを用いて上の公式を導け
\end{itemize}
\label{pageend}\mbox{}

\end{document}
