\documentclass[a4paper,10pt]{ujarticle}

\usepackage{ketpic,ketlayer}

\usepackage{graphicx}

\usepackage{emath}
\usepackage{emathMw,emathEy}

\setmargin{15}{15}{10}{20}

\pagestyle{empty}

\columnsep=0.5cm
\columnseprule=0.5pt

\newcommand{\hako}[4][-1]{%
\setcounter{ketpicctra}{#2}%
\divide\value{ketpicctra} by 2%
\setcounter{ketpicctrb}{#3}%
\divide\value{ketpicctrb} by 2%
\setcounter{ketpicctrc}{\theketpicctrb}%
\addtocounter{ketpicctrc}{#1}%
\def\kettmp{
\begin{picture}%
(#2, #3)(0,0)%
\settowidth{\Width}{#4}\setlength{\Width}{-0.5\Width}%
\settoheight{\Height}{#4}\settodepth{\Depth}{#4}\setlength{\Height}{-0.5\Height}\setlength{\Depth}{0.5\Depth}\addtolength{\Height}{\Depth}%
\put(\theketpicctra,\theketpicctrb){\hspace*{\Width}\raisebox{\Height}{#4}}%
\end{picture}%
}%
{\unitlength=1mm%
\raisebox{-\theketpicctrc mm}{\fbox{\kettmp}}%
}
}

\begin{document}

\noindent
\twocolumn[
{\tabcolsep=0mm
\Ctab{180mm}{{\Large 確認テスト}}\vspace{-3mm}%

%\begin{tabular}[b]{|c|c|}\hline
%\Ctab{2cm}{教室番号}\mbox{}& \Ctab{2cm}{座席番号}\mbox{}\\\hline
%\raisebox{8mm}{\mbox{}}& \\\hline
%\end{tabular}%
%{\large 
%\hspace{1.64cm}(\sentaku{春}・秋)学期 定期試験
%}%
\hspace{0.82cm}%
\hfill 2018年6月4日\vspace{1mm}% 第3時限(\ 1/2\ 枚)\vspace{1mm}

\begin{tabular}{|c|c|c|c|c|c|c|}\hline
\Ctab{4.1cm}{科 目}\mbox{} & \Ctab{2.46cm}{担 当 者}\mbox{}%
& \Ctab{2cm}{科}\mbox{} & \Ctab{0.984cm}{学年}\mbox{}%
& \Ctab{2cm}{番 号}\mbox{} & \Ctab{4.4cm}{氏    名}\mbox{}%
& \Ctab{2.2cm}{評 点}\mbox{}\\\hline
\raisebox{3 mm}{$\mathstrut$}\raisebox{-3mm}{$\mathstrut$}%
数 学 1& 高 遠 & 生産技術 & 1
 & & & \\\hline
\end{tabular}
}
\vspace{1mm}

\begin{center}
%%% /Users/takatoosetsuo/Dropbox/2018polytec/lecture/0618/fig/sincos.tex 
%%% Generator=fig0618.cdy 
{\unitlength=8.171206mm%
\begin{picture}%
(13.08,4.2)(-6.54,-2.1)%
\special{pn 8}%
%
\small%
\special{pn 12}%
\special{pa -2104  -311}\special{pa -2062  -319}\special{pa -2020  -322}\special{pa -1978  -319}%
\special{pa -1936  -310}\special{pa -1894  -297}\special{pa -1851  -278}\special{pa -1809  -254}%
\special{pa -1767  -227}\special{pa -1725  -195}\special{pa -1683  -160}\special{pa -1641  -122}%
\special{pa -1599   -82}\special{pa -1557   -41}\special{pa -1515     1}\special{pa -1473    43}%
\special{pa -1431    84}\special{pa -1389   124}\special{pa -1347   162}\special{pa -1304   197}%
\special{pa -1262   228}\special{pa -1220   256}\special{pa -1178   279}\special{pa -1136   298}%
\special{pa -1094   311}\special{pa -1052   319}\special{pa -1010   322}\special{pa  -968   319}%
\special{pa  -926   311}\special{pa  -884   297}\special{pa  -842   278}\special{pa  -799   255}%
\special{pa  -757   227}\special{pa  -715   195}\special{pa  -673   160}\special{pa  -631   123}%
\special{pa  -589    83}\special{pa  -547    42}\special{pa  -505    -0}\special{pa  -463   -42}%
\special{pa  -421   -84}\special{pa  -379  -123}\special{pa  -337  -161}\special{pa  -295  -196}%
\special{pa  -252  -228}\special{pa  -210  -255}\special{pa  -168  -279}\special{pa  -126  -297}%
\special{pa   -84  -311}\special{pa   -42  -319}\special{pa     0  -322}\special{pa    42  -319}%
\special{pa    84  -311}\special{pa   126  -297}\special{pa   168  -279}\special{pa   210  -255}%
\special{pa   252  -228}\special{pa   295  -196}\special{pa   337  -161}\special{pa   379  -123}%
\special{pa   421   -84}\special{pa   463   -42}\special{pa   505    -0}\special{pa   547    42}%
\special{pa   589    83}\special{pa   631   123}\special{pa   673   160}\special{pa   715   195}%
\special{pa   757   227}\special{pa   799   255}\special{pa   842   278}\special{pa   884   297}%
\special{pa   926   311}\special{pa   968   319}\special{pa  1010   322}\special{pa  1052   319}%
\special{pa  1094   311}\special{pa  1136   298}\special{pa  1178   279}\special{pa  1220   256}%
\special{pa  1262   228}\special{pa  1304   197}\special{pa  1347   162}\special{pa  1389   124}%
\special{pa  1431    84}\special{pa  1473    43}\special{pa  1515     1}\special{pa  1557   -41}%
\special{pa  1599   -82}\special{pa  1641  -122}\special{pa  1683  -160}\special{pa  1725  -195}%
\special{pa  1767  -227}\special{pa  1809  -254}\special{pa  1851  -278}\special{pa  1894  -297}%
\special{pa  1936  -310}\special{pa  1978  -319}\special{pa  2020  -322}\special{pa  2062  -319}%
\special{pa  2104  -311}%
\special{fp}%
\special{pn 8}%
\special{pn 12}%
\special{pa -2104   319}\special{pa -2062   321}\special{pa -2020   322}\special{pa -1978   321}%
\special{pa -1936   319}\special{pa -1894   315}\special{pa -1851   311}\special{pa -1809   304}%
\special{pa -1767   297}\special{pa -1725   288}\special{pa -1683   278}\special{pa -1641   267}%
\special{pa -1599   255}\special{pa -1557   241}\special{pa -1515   227}\special{pa -1473   212}%
\special{pa -1431   195}\special{pa -1389   178}\special{pa -1347   160}\special{pa -1304   142}%
\special{pa -1262   123}\special{pa -1220   103}\special{pa -1178    83}\special{pa -1136    62}%
\special{pa -1094    42}\special{pa -1052    21}\special{pa -1010    -0}\special{pa  -968   -21}%
\special{pa  -926   -42}\special{pa  -884   -63}\special{pa  -842   -84}\special{pa  -799  -104}%
\special{pa  -757  -123}\special{pa  -715  -143}\special{pa  -673  -161}\special{pa  -631  -179}%
\special{pa  -589  -196}\special{pa  -547  -212}\special{pa  -505  -228}\special{pa  -463  -242}%
\special{pa  -421  -255}\special{pa  -379  -268}\special{pa  -337  -279}\special{pa  -295  -289}%
\special{pa  -252  -297}\special{pa  -210  -305}\special{pa  -168  -311}\special{pa  -126  -316}%
\special{pa   -84  -319}\special{pa   -42  -321}\special{pa     0  -322}\special{pa    42  -321}%
\special{pa    84  -319}\special{pa   126  -316}\special{pa   168  -311}\special{pa   210  -305}%
\special{pa   252  -297}\special{pa   295  -289}\special{pa   337  -279}\special{pa   379  -268}%
\special{pa   421  -255}\special{pa   463  -242}\special{pa   505  -228}\special{pa   547  -212}%
\special{pa   589  -196}\special{pa   631  -179}\special{pa   673  -161}\special{pa   715  -143}%
\special{pa   757  -123}\special{pa   799  -104}\special{pa   842   -84}\special{pa   884   -63}%
\special{pa   926   -42}\special{pa   968   -21}\special{pa  1010    -0}\special{pa  1052    21}%
\special{pa  1094    42}\special{pa  1136    62}\special{pa  1178    83}\special{pa  1220   103}%
\special{pa  1262   123}\special{pa  1304   142}\special{pa  1347   160}\special{pa  1389   178}%
\special{pa  1431   195}\special{pa  1473   212}\special{pa  1515   227}\special{pa  1557   241}%
\special{pa  1599   255}\special{pa  1641   267}\special{pa  1683   278}\special{pa  1725   288}%
\special{pa  1767   297}\special{pa  1809   304}\special{pa  1851   311}\special{pa  1894   315}%
\special{pa  1936   319}\special{pa  1978   321}\special{pa  2020   322}\special{pa  2062   321}%
\special{pa  2104   319}%
\special{fp}%
\special{pn 8}%
\special{pn 4}%
\special{pa -2021 643}\special{pa -2021 604}\special{fp}\special{pa -2021 565}\special{pa -2021 526}\special{fp}%
\special{pa -2021 487}\special{pa -2021 448}\special{fp}\special{pa -2021 409}\special{pa -2021 370}\special{fp}%
\special{pa -2021 331}\special{pa -2021 292}\special{fp}\special{pa -2021 253}\special{pa -2021 214}\special{fp}%
\special{pa -2021 175}\special{pa -2021 136}\special{fp}\special{pa -2021 97}\special{pa -2021 58}\special{fp}%
\special{pa -2021 19}\special{pa -2021 -19}\special{fp}\special{pa -2021 -58}\special{pa -2021 -97}\special{fp}%
\special{pa -2021 -136}\special{pa -2021 -175}\special{fp}\special{pa -2021 -214}\special{pa -2021 -253}\special{fp}%
\special{pa -2021 -292}\special{pa -2021 -331}\special{fp}\special{pa -2021 -370}\special{pa -2021 -409}\special{fp}%
\special{pa -2021 -448}\special{pa -2021 -487}\special{fp}\special{pa -2021 -526}\special{pa -2021 -565}\special{fp}%
\special{pa -2021 -604}\special{pa -2021 -643}\special{fp}%
%
\special{pa -1769 643}\special{pa -1769 604}\special{fp}\special{pa -1769 565}\special{pa -1769 526}\special{fp}%
\special{pa -1769 487}\special{pa -1769 448}\special{fp}\special{pa -1769 409}\special{pa -1769 370}\special{fp}%
\special{pa -1769 331}\special{pa -1769 292}\special{fp}\special{pa -1769 253}\special{pa -1769 214}\special{fp}%
\special{pa -1769 175}\special{pa -1769 136}\special{fp}\special{pa -1769 97}\special{pa -1769 58}\special{fp}%
\special{pa -1769 19}\special{pa -1769 -19}\special{fp}\special{pa -1769 -58}\special{pa -1769 -97}\special{fp}%
\special{pa -1769 -136}\special{pa -1769 -175}\special{fp}\special{pa -1769 -214}\special{pa -1769 -253}\special{fp}%
\special{pa -1769 -292}\special{pa -1769 -331}\special{fp}\special{pa -1769 -370}\special{pa -1769 -409}\special{fp}%
\special{pa -1769 -448}\special{pa -1769 -487}\special{fp}\special{pa -1769 -526}\special{pa -1769 -565}\special{fp}%
\special{pa -1769 -604}\special{pa -1769 -643}\special{fp}%
%
\special{pa -1516 643}\special{pa -1516 604}\special{fp}\special{pa -1516 565}\special{pa -1516 526}\special{fp}%
\special{pa -1516 487}\special{pa -1516 448}\special{fp}\special{pa -1516 409}\special{pa -1516 370}\special{fp}%
\special{pa -1516 331}\special{pa -1516 292}\special{fp}\special{pa -1516 253}\special{pa -1516 214}\special{fp}%
\special{pa -1516 175}\special{pa -1516 136}\special{fp}\special{pa -1516 97}\special{pa -1516 58}\special{fp}%
\special{pa -1516 19}\special{pa -1516 -19}\special{fp}\special{pa -1516 -58}\special{pa -1516 -97}\special{fp}%
\special{pa -1516 -136}\special{pa -1516 -175}\special{fp}\special{pa -1516 -214}\special{pa -1516 -253}\special{fp}%
\special{pa -1516 -292}\special{pa -1516 -331}\special{fp}\special{pa -1516 -370}\special{pa -1516 -409}\special{fp}%
\special{pa -1516 -448}\special{pa -1516 -487}\special{fp}\special{pa -1516 -526}\special{pa -1516 -565}\special{fp}%
\special{pa -1516 -604}\special{pa -1516 -643}\special{fp}%
%
\special{pa -1263 643}\special{pa -1263 604}\special{fp}\special{pa -1263 565}\special{pa -1263 526}\special{fp}%
\special{pa -1263 487}\special{pa -1263 448}\special{fp}\special{pa -1263 409}\special{pa -1263 370}\special{fp}%
\special{pa -1263 331}\special{pa -1263 292}\special{fp}\special{pa -1263 253}\special{pa -1263 214}\special{fp}%
\special{pa -1263 175}\special{pa -1263 136}\special{fp}\special{pa -1263 97}\special{pa -1263 58}\special{fp}%
\special{pa -1263 19}\special{pa -1263 -19}\special{fp}\special{pa -1263 -58}\special{pa -1263 -97}\special{fp}%
\special{pa -1263 -136}\special{pa -1263 -175}\special{fp}\special{pa -1263 -214}\special{pa -1263 -253}\special{fp}%
\special{pa -1263 -292}\special{pa -1263 -331}\special{fp}\special{pa -1263 -370}\special{pa -1263 -409}\special{fp}%
\special{pa -1263 -448}\special{pa -1263 -487}\special{fp}\special{pa -1263 -526}\special{pa -1263 -565}\special{fp}%
\special{pa -1263 -604}\special{pa -1263 -643}\special{fp}%
%
\special{pa -1011 643}\special{pa -1011 604}\special{fp}\special{pa -1011 565}\special{pa -1011 526}\special{fp}%
\special{pa -1011 487}\special{pa -1011 448}\special{fp}\special{pa -1011 409}\special{pa -1011 370}\special{fp}%
\special{pa -1011 331}\special{pa -1011 292}\special{fp}\special{pa -1011 253}\special{pa -1011 214}\special{fp}%
\special{pa -1011 175}\special{pa -1011 136}\special{fp}\special{pa -1011 97}\special{pa -1011 58}\special{fp}%
\special{pa -1011 19}\special{pa -1011 -19}\special{fp}\special{pa -1011 -58}\special{pa -1011 -97}\special{fp}%
\special{pa -1011 -136}\special{pa -1011 -175}\special{fp}\special{pa -1011 -214}\special{pa -1011 -253}\special{fp}%
\special{pa -1011 -292}\special{pa -1011 -331}\special{fp}\special{pa -1011 -370}\special{pa -1011 -409}\special{fp}%
\special{pa -1011 -448}\special{pa -1011 -487}\special{fp}\special{pa -1011 -526}\special{pa -1011 -565}\special{fp}%
\special{pa -1011 -604}\special{pa -1011 -643}\special{fp}%
%
\special{pa -758 643}\special{pa -758 604}\special{fp}\special{pa -758 565}\special{pa -758 526}\special{fp}%
\special{pa -758 487}\special{pa -758 448}\special{fp}\special{pa -758 409}\special{pa -758 370}\special{fp}%
\special{pa -758 331}\special{pa -758 292}\special{fp}\special{pa -758 253}\special{pa -758 214}\special{fp}%
\special{pa -758 175}\special{pa -758 136}\special{fp}\special{pa -758 97}\special{pa -758 58}\special{fp}%
\special{pa -758 19}\special{pa -758 -19}\special{fp}\special{pa -758 -58}\special{pa -758 -97}\special{fp}%
\special{pa -758 -136}\special{pa -758 -175}\special{fp}\special{pa -758 -214}\special{pa -758 -253}\special{fp}%
\special{pa -758 -292}\special{pa -758 -331}\special{fp}\special{pa -758 -370}\special{pa -758 -409}\special{fp}%
\special{pa -758 -448}\special{pa -758 -487}\special{fp}\special{pa -758 -526}\special{pa -758 -565}\special{fp}%
\special{pa -758 -604}\special{pa -758 -643}\special{fp}%
%
\special{pa -505 643}\special{pa -505 604}\special{fp}\special{pa -505 565}\special{pa -505 526}\special{fp}%
\special{pa -505 487}\special{pa -505 448}\special{fp}\special{pa -505 409}\special{pa -505 370}\special{fp}%
\special{pa -505 331}\special{pa -505 292}\special{fp}\special{pa -505 253}\special{pa -505 214}\special{fp}%
\special{pa -505 175}\special{pa -505 136}\special{fp}\special{pa -505 97}\special{pa -505 58}\special{fp}%
\special{pa -505 19}\special{pa -505 -19}\special{fp}\special{pa -505 -58}\special{pa -505 -97}\special{fp}%
\special{pa -505 -136}\special{pa -505 -175}\special{fp}\special{pa -505 -214}\special{pa -505 -253}\special{fp}%
\special{pa -505 -292}\special{pa -505 -331}\special{fp}\special{pa -505 -370}\special{pa -505 -409}\special{fp}%
\special{pa -505 -448}\special{pa -505 -487}\special{fp}\special{pa -505 -526}\special{pa -505 -565}\special{fp}%
\special{pa -505 -604}\special{pa -505 -643}\special{fp}%
%
\special{pa -253 643}\special{pa -253 604}\special{fp}\special{pa -253 565}\special{pa -253 526}\special{fp}%
\special{pa -253 487}\special{pa -253 448}\special{fp}\special{pa -253 409}\special{pa -253 370}\special{fp}%
\special{pa -253 331}\special{pa -253 292}\special{fp}\special{pa -253 253}\special{pa -253 214}\special{fp}%
\special{pa -253 175}\special{pa -253 136}\special{fp}\special{pa -253 97}\special{pa -253 58}\special{fp}%
\special{pa -253 19}\special{pa -253 -19}\special{fp}\special{pa -253 -58}\special{pa -253 -97}\special{fp}%
\special{pa -253 -136}\special{pa -253 -175}\special{fp}\special{pa -253 -214}\special{pa -253 -253}\special{fp}%
\special{pa -253 -292}\special{pa -253 -331}\special{fp}\special{pa -253 -370}\special{pa -253 -409}\special{fp}%
\special{pa -253 -448}\special{pa -253 -487}\special{fp}\special{pa -253 -526}\special{pa -253 -565}\special{fp}%
\special{pa -253 -604}\special{pa -253 -643}\special{fp}%
%
\special{pa 0 643}\special{pa 0 604}\special{fp}\special{pa 0 565}\special{pa 0 526}\special{fp}%
\special{pa 0 487}\special{pa 0 448}\special{fp}\special{pa 0 409}\special{pa 0 370}\special{fp}%
\special{pa 0 331}\special{pa 0 292}\special{fp}\special{pa 0 253}\special{pa 0 214}\special{fp}%
\special{pa 0 175}\special{pa 0 136}\special{fp}\special{pa 0 97}\special{pa 0 58}\special{fp}%
\special{pa 0 19}\special{pa 0 -19}\special{fp}\special{pa 0 -58}\special{pa 0 -97}\special{fp}%
\special{pa 0 -136}\special{pa 0 -175}\special{fp}\special{pa 0 -214}\special{pa 0 -253}\special{fp}%
\special{pa 0 -292}\special{pa 0 -331}\special{fp}\special{pa 0 -370}\special{pa 0 -409}\special{fp}%
\special{pa 0 -448}\special{pa 0 -487}\special{fp}\special{pa 0 -526}\special{pa 0 -565}\special{fp}%
\special{pa 0 -604}\special{pa 0 -643}\special{fp}%
%
\special{pa 253 643}\special{pa 253 604}\special{fp}\special{pa 253 565}\special{pa 253 526}\special{fp}%
\special{pa 253 487}\special{pa 253 448}\special{fp}\special{pa 253 409}\special{pa 253 370}\special{fp}%
\special{pa 253 331}\special{pa 253 292}\special{fp}\special{pa 253 253}\special{pa 253 214}\special{fp}%
\special{pa 253 175}\special{pa 253 136}\special{fp}\special{pa 253 97}\special{pa 253 58}\special{fp}%
\special{pa 253 19}\special{pa 253 -19}\special{fp}\special{pa 253 -58}\special{pa 253 -97}\special{fp}%
\special{pa 253 -136}\special{pa 253 -175}\special{fp}\special{pa 253 -214}\special{pa 253 -253}\special{fp}%
\special{pa 253 -292}\special{pa 253 -331}\special{fp}\special{pa 253 -370}\special{pa 253 -409}\special{fp}%
\special{pa 253 -448}\special{pa 253 -487}\special{fp}\special{pa 253 -526}\special{pa 253 -565}\special{fp}%
\special{pa 253 -604}\special{pa 253 -643}\special{fp}%
%
\special{pa 505 643}\special{pa 505 604}\special{fp}\special{pa 505 565}\special{pa 505 526}\special{fp}%
\special{pa 505 487}\special{pa 505 448}\special{fp}\special{pa 505 409}\special{pa 505 370}\special{fp}%
\special{pa 505 331}\special{pa 505 292}\special{fp}\special{pa 505 253}\special{pa 505 214}\special{fp}%
\special{pa 505 175}\special{pa 505 136}\special{fp}\special{pa 505 97}\special{pa 505 58}\special{fp}%
\special{pa 505 19}\special{pa 505 -19}\special{fp}\special{pa 505 -58}\special{pa 505 -97}\special{fp}%
\special{pa 505 -136}\special{pa 505 -175}\special{fp}\special{pa 505 -214}\special{pa 505 -253}\special{fp}%
\special{pa 505 -292}\special{pa 505 -331}\special{fp}\special{pa 505 -370}\special{pa 505 -409}\special{fp}%
\special{pa 505 -448}\special{pa 505 -487}\special{fp}\special{pa 505 -526}\special{pa 505 -565}\special{fp}%
\special{pa 505 -604}\special{pa 505 -643}\special{fp}%
%
\special{pa 758 643}\special{pa 758 604}\special{fp}\special{pa 758 565}\special{pa 758 526}\special{fp}%
\special{pa 758 487}\special{pa 758 448}\special{fp}\special{pa 758 409}\special{pa 758 370}\special{fp}%
\special{pa 758 331}\special{pa 758 292}\special{fp}\special{pa 758 253}\special{pa 758 214}\special{fp}%
\special{pa 758 175}\special{pa 758 136}\special{fp}\special{pa 758 97}\special{pa 758 58}\special{fp}%
\special{pa 758 19}\special{pa 758 -19}\special{fp}\special{pa 758 -58}\special{pa 758 -97}\special{fp}%
\special{pa 758 -136}\special{pa 758 -175}\special{fp}\special{pa 758 -214}\special{pa 758 -253}\special{fp}%
\special{pa 758 -292}\special{pa 758 -331}\special{fp}\special{pa 758 -370}\special{pa 758 -409}\special{fp}%
\special{pa 758 -448}\special{pa 758 -487}\special{fp}\special{pa 758 -526}\special{pa 758 -565}\special{fp}%
\special{pa 758 -604}\special{pa 758 -643}\special{fp}%
%
\special{pa 1011 643}\special{pa 1011 604}\special{fp}\special{pa 1011 565}\special{pa 1011 526}\special{fp}%
\special{pa 1011 487}\special{pa 1011 448}\special{fp}\special{pa 1011 409}\special{pa 1011 370}\special{fp}%
\special{pa 1011 331}\special{pa 1011 292}\special{fp}\special{pa 1011 253}\special{pa 1011 214}\special{fp}%
\special{pa 1011 175}\special{pa 1011 136}\special{fp}\special{pa 1011 97}\special{pa 1011 58}\special{fp}%
\special{pa 1011 19}\special{pa 1011 -19}\special{fp}\special{pa 1011 -58}\special{pa 1011 -97}\special{fp}%
\special{pa 1011 -136}\special{pa 1011 -175}\special{fp}\special{pa 1011 -214}\special{pa 1011 -253}\special{fp}%
\special{pa 1011 -292}\special{pa 1011 -331}\special{fp}\special{pa 1011 -370}\special{pa 1011 -409}\special{fp}%
\special{pa 1011 -448}\special{pa 1011 -487}\special{fp}\special{pa 1011 -526}\special{pa 1011 -565}\special{fp}%
\special{pa 1011 -604}\special{pa 1011 -643}\special{fp}%
%
\special{pa 1263 643}\special{pa 1263 604}\special{fp}\special{pa 1263 565}\special{pa 1263 526}\special{fp}%
\special{pa 1263 487}\special{pa 1263 448}\special{fp}\special{pa 1263 409}\special{pa 1263 370}\special{fp}%
\special{pa 1263 331}\special{pa 1263 292}\special{fp}\special{pa 1263 253}\special{pa 1263 214}\special{fp}%
\special{pa 1263 175}\special{pa 1263 136}\special{fp}\special{pa 1263 97}\special{pa 1263 58}\special{fp}%
\special{pa 1263 19}\special{pa 1263 -19}\special{fp}\special{pa 1263 -58}\special{pa 1263 -97}\special{fp}%
\special{pa 1263 -136}\special{pa 1263 -175}\special{fp}\special{pa 1263 -214}\special{pa 1263 -253}\special{fp}%
\special{pa 1263 -292}\special{pa 1263 -331}\special{fp}\special{pa 1263 -370}\special{pa 1263 -409}\special{fp}%
\special{pa 1263 -448}\special{pa 1263 -487}\special{fp}\special{pa 1263 -526}\special{pa 1263 -565}\special{fp}%
\special{pa 1263 -604}\special{pa 1263 -643}\special{fp}%
%
\special{pa 1516 643}\special{pa 1516 604}\special{fp}\special{pa 1516 565}\special{pa 1516 526}\special{fp}%
\special{pa 1516 487}\special{pa 1516 448}\special{fp}\special{pa 1516 409}\special{pa 1516 370}\special{fp}%
\special{pa 1516 331}\special{pa 1516 292}\special{fp}\special{pa 1516 253}\special{pa 1516 214}\special{fp}%
\special{pa 1516 175}\special{pa 1516 136}\special{fp}\special{pa 1516 97}\special{pa 1516 58}\special{fp}%
\special{pa 1516 19}\special{pa 1516 -19}\special{fp}\special{pa 1516 -58}\special{pa 1516 -97}\special{fp}%
\special{pa 1516 -136}\special{pa 1516 -175}\special{fp}\special{pa 1516 -214}\special{pa 1516 -253}\special{fp}%
\special{pa 1516 -292}\special{pa 1516 -331}\special{fp}\special{pa 1516 -370}\special{pa 1516 -409}\special{fp}%
\special{pa 1516 -448}\special{pa 1516 -487}\special{fp}\special{pa 1516 -526}\special{pa 1516 -565}\special{fp}%
\special{pa 1516 -604}\special{pa 1516 -643}\special{fp}%
%
\special{pa 1769 643}\special{pa 1769 604}\special{fp}\special{pa 1769 565}\special{pa 1769 526}\special{fp}%
\special{pa 1769 487}\special{pa 1769 448}\special{fp}\special{pa 1769 409}\special{pa 1769 370}\special{fp}%
\special{pa 1769 331}\special{pa 1769 292}\special{fp}\special{pa 1769 253}\special{pa 1769 214}\special{fp}%
\special{pa 1769 175}\special{pa 1769 136}\special{fp}\special{pa 1769 97}\special{pa 1769 58}\special{fp}%
\special{pa 1769 19}\special{pa 1769 -19}\special{fp}\special{pa 1769 -58}\special{pa 1769 -97}\special{fp}%
\special{pa 1769 -136}\special{pa 1769 -175}\special{fp}\special{pa 1769 -214}\special{pa 1769 -253}\special{fp}%
\special{pa 1769 -292}\special{pa 1769 -331}\special{fp}\special{pa 1769 -370}\special{pa 1769 -409}\special{fp}%
\special{pa 1769 -448}\special{pa 1769 -487}\special{fp}\special{pa 1769 -526}\special{pa 1769 -565}\special{fp}%
\special{pa 1769 -604}\special{pa 1769 -643}\special{fp}%
%
\special{pa 2021 643}\special{pa 2021 604}\special{fp}\special{pa 2021 565}\special{pa 2021 526}\special{fp}%
\special{pa 2021 487}\special{pa 2021 448}\special{fp}\special{pa 2021 409}\special{pa 2021 370}\special{fp}%
\special{pa 2021 331}\special{pa 2021 292}\special{fp}\special{pa 2021 253}\special{pa 2021 214}\special{fp}%
\special{pa 2021 175}\special{pa 2021 136}\special{fp}\special{pa 2021 97}\special{pa 2021 58}\special{fp}%
\special{pa 2021 19}\special{pa 2021 -19}\special{fp}\special{pa 2021 -58}\special{pa 2021 -97}\special{fp}%
\special{pa 2021 -136}\special{pa 2021 -175}\special{fp}\special{pa 2021 -214}\special{pa 2021 -253}\special{fp}%
\special{pa 2021 -292}\special{pa 2021 -331}\special{fp}\special{pa 2021 -370}\special{pa 2021 -409}\special{fp}%
\special{pa 2021 -448}\special{pa 2021 -487}\special{fp}\special{pa 2021 -526}\special{pa 2021 -565}\special{fp}%
\special{pa 2021 -604}\special{pa 2021 -643}\special{fp}%
%
\special{pa -2021 643}\special{pa -1982 643}\special{fp}\special{pa -1943 643}\special{pa -1904 643}\special{fp}%
\special{pa -1864 643}\special{pa -1825 643}\special{fp}\special{pa -1786 643}\special{pa -1747 643}\special{fp}%
\special{pa -1707 643}\special{pa -1668 643}\special{fp}\special{pa -1629 643}\special{pa -1590 643}\special{fp}%
\special{pa -1550 643}\special{pa -1511 643}\special{fp}\special{pa -1472 643}\special{pa -1433 643}\special{fp}%
\special{pa -1393 643}\special{pa -1354 643}\special{fp}\special{pa -1315 643}\special{pa -1276 643}\special{fp}%
\special{pa -1236 643}\special{pa -1197 643}\special{fp}\special{pa -1158 643}\special{pa -1119 643}\special{fp}%
\special{pa -1079 643}\special{pa -1040 643}\special{fp}\special{pa -1001 643}\special{pa -962 643}\special{fp}%
\special{pa -922 643}\special{pa -883 643}\special{fp}\special{pa -844 643}\special{pa -805 643}\special{fp}%
\special{pa -765 643}\special{pa -726 643}\special{fp}\special{pa -687 643}\special{pa -648 643}\special{fp}%
\special{pa -608 643}\special{pa -569 643}\special{fp}\special{pa -530 643}\special{pa -491 643}\special{fp}%
\special{pa -451 643}\special{pa -412 643}\special{fp}\special{pa -373 643}\special{pa -334 643}\special{fp}%
\special{pa -294 643}\special{pa -255 643}\special{fp}\special{pa -216 643}\special{pa -177 643}\special{fp}%
\special{pa -137 643}\special{pa -98 643}\special{fp}\special{pa -59 643}\special{pa -20 643}\special{fp}%
\special{pa 20 643}\special{pa 59 643}\special{fp}\special{pa 98 643}\special{pa 137 643}\special{fp}%
\special{pa 177 643}\special{pa 216 643}\special{fp}\special{pa 255 643}\special{pa 294 643}\special{fp}%
\special{pa 334 643}\special{pa 373 643}\special{fp}\special{pa 412 643}\special{pa 451 643}\special{fp}%
\special{pa 491 643}\special{pa 530 643}\special{fp}\special{pa 569 643}\special{pa 608 643}\special{fp}%
\special{pa 648 643}\special{pa 687 643}\special{fp}\special{pa 726 643}\special{pa 765 643}\special{fp}%
\special{pa 805 643}\special{pa 844 643}\special{fp}\special{pa 883 643}\special{pa 922 643}\special{fp}%
\special{pa 962 643}\special{pa 1001 643}\special{fp}\special{pa 1040 643}\special{pa 1079 643}\special{fp}%
\special{pa 1119 643}\special{pa 1158 643}\special{fp}\special{pa 1197 643}\special{pa 1236 643}\special{fp}%
\special{pa 1276 643}\special{pa 1315 643}\special{fp}\special{pa 1354 643}\special{pa 1393 643}\special{fp}%
\special{pa 1433 643}\special{pa 1472 643}\special{fp}\special{pa 1511 643}\special{pa 1550 643}\special{fp}%
\special{pa 1590 643}\special{pa 1629 643}\special{fp}\special{pa 1668 643}\special{pa 1707 643}\special{fp}%
\special{pa 1747 643}\special{pa 1786 643}\special{fp}\special{pa 1825 643}\special{pa 1864 643}\special{fp}%
\special{pa 1904 643}\special{pa 1943 643}\special{fp}\special{pa 1982 643}\special{pa 2021 643}\special{fp}%
%
%
\special{pa -2021 483}\special{pa -1982 483}\special{fp}\special{pa -1943 483}\special{pa -1904 483}\special{fp}%
\special{pa -1864 483}\special{pa -1825 483}\special{fp}\special{pa -1786 483}\special{pa -1747 483}\special{fp}%
\special{pa -1707 483}\special{pa -1668 483}\special{fp}\special{pa -1629 483}\special{pa -1590 483}\special{fp}%
\special{pa -1550 483}\special{pa -1511 483}\special{fp}\special{pa -1472 483}\special{pa -1433 483}\special{fp}%
\special{pa -1393 483}\special{pa -1354 483}\special{fp}\special{pa -1315 483}\special{pa -1276 483}\special{fp}%
\special{pa -1236 483}\special{pa -1197 483}\special{fp}\special{pa -1158 483}\special{pa -1119 483}\special{fp}%
\special{pa -1079 483}\special{pa -1040 483}\special{fp}\special{pa -1001 483}\special{pa -962 483}\special{fp}%
\special{pa -922 483}\special{pa -883 483}\special{fp}\special{pa -844 483}\special{pa -805 483}\special{fp}%
\special{pa -765 483}\special{pa -726 483}\special{fp}\special{pa -687 483}\special{pa -648 483}\special{fp}%
\special{pa -608 483}\special{pa -569 483}\special{fp}\special{pa -530 483}\special{pa -491 483}\special{fp}%
\special{pa -451 483}\special{pa -412 483}\special{fp}\special{pa -373 483}\special{pa -334 483}\special{fp}%
\special{pa -294 483}\special{pa -255 483}\special{fp}\special{pa -216 483}\special{pa -177 483}\special{fp}%
\special{pa -137 483}\special{pa -98 483}\special{fp}\special{pa -59 483}\special{pa -20 483}\special{fp}%
\special{pa 20 483}\special{pa 59 483}\special{fp}\special{pa 98 483}\special{pa 137 483}\special{fp}%
\special{pa 177 483}\special{pa 216 483}\special{fp}\special{pa 255 483}\special{pa 294 483}\special{fp}%
\special{pa 334 483}\special{pa 373 483}\special{fp}\special{pa 412 483}\special{pa 451 483}\special{fp}%
\special{pa 491 483}\special{pa 530 483}\special{fp}\special{pa 569 483}\special{pa 608 483}\special{fp}%
\special{pa 648 483}\special{pa 687 483}\special{fp}\special{pa 726 483}\special{pa 765 483}\special{fp}%
\special{pa 805 483}\special{pa 844 483}\special{fp}\special{pa 883 483}\special{pa 922 483}\special{fp}%
\special{pa 962 483}\special{pa 1001 483}\special{fp}\special{pa 1040 483}\special{pa 1079 483}\special{fp}%
\special{pa 1119 483}\special{pa 1158 483}\special{fp}\special{pa 1197 483}\special{pa 1236 483}\special{fp}%
\special{pa 1276 483}\special{pa 1315 483}\special{fp}\special{pa 1354 483}\special{pa 1393 483}\special{fp}%
\special{pa 1433 483}\special{pa 1472 483}\special{fp}\special{pa 1511 483}\special{pa 1550 483}\special{fp}%
\special{pa 1590 483}\special{pa 1629 483}\special{fp}\special{pa 1668 483}\special{pa 1707 483}\special{fp}%
\special{pa 1747 483}\special{pa 1786 483}\special{fp}\special{pa 1825 483}\special{pa 1864 483}\special{fp}%
\special{pa 1904 483}\special{pa 1943 483}\special{fp}\special{pa 1982 483}\special{pa 2021 483}\special{fp}%
%
%
\special{pa -2021 322}\special{pa -1982 322}\special{fp}\special{pa -1943 322}\special{pa -1904 322}\special{fp}%
\special{pa -1864 322}\special{pa -1825 322}\special{fp}\special{pa -1786 322}\special{pa -1747 322}\special{fp}%
\special{pa -1707 322}\special{pa -1668 322}\special{fp}\special{pa -1629 322}\special{pa -1590 322}\special{fp}%
\special{pa -1550 322}\special{pa -1511 322}\special{fp}\special{pa -1472 322}\special{pa -1433 322}\special{fp}%
\special{pa -1393 322}\special{pa -1354 322}\special{fp}\special{pa -1315 322}\special{pa -1276 322}\special{fp}%
\special{pa -1236 322}\special{pa -1197 322}\special{fp}\special{pa -1158 322}\special{pa -1119 322}\special{fp}%
\special{pa -1079 322}\special{pa -1040 322}\special{fp}\special{pa -1001 322}\special{pa -962 322}\special{fp}%
\special{pa -922 322}\special{pa -883 322}\special{fp}\special{pa -844 322}\special{pa -805 322}\special{fp}%
\special{pa -765 322}\special{pa -726 322}\special{fp}\special{pa -687 322}\special{pa -648 322}\special{fp}%
\special{pa -608 322}\special{pa -569 322}\special{fp}\special{pa -530 322}\special{pa -491 322}\special{fp}%
\special{pa -451 322}\special{pa -412 322}\special{fp}\special{pa -373 322}\special{pa -334 322}\special{fp}%
\special{pa -294 322}\special{pa -255 322}\special{fp}\special{pa -216 322}\special{pa -177 322}\special{fp}%
\special{pa -137 322}\special{pa -98 322}\special{fp}\special{pa -59 322}\special{pa -20 322}\special{fp}%
\special{pa 20 322}\special{pa 59 322}\special{fp}\special{pa 98 322}\special{pa 137 322}\special{fp}%
\special{pa 177 322}\special{pa 216 322}\special{fp}\special{pa 255 322}\special{pa 294 322}\special{fp}%
\special{pa 334 322}\special{pa 373 322}\special{fp}\special{pa 412 322}\special{pa 451 322}\special{fp}%
\special{pa 491 322}\special{pa 530 322}\special{fp}\special{pa 569 322}\special{pa 608 322}\special{fp}%
\special{pa 648 322}\special{pa 687 322}\special{fp}\special{pa 726 322}\special{pa 765 322}\special{fp}%
\special{pa 805 322}\special{pa 844 322}\special{fp}\special{pa 883 322}\special{pa 922 322}\special{fp}%
\special{pa 962 322}\special{pa 1001 322}\special{fp}\special{pa 1040 322}\special{pa 1079 322}\special{fp}%
\special{pa 1119 322}\special{pa 1158 322}\special{fp}\special{pa 1197 322}\special{pa 1236 322}\special{fp}%
\special{pa 1276 322}\special{pa 1315 322}\special{fp}\special{pa 1354 322}\special{pa 1393 322}\special{fp}%
\special{pa 1433 322}\special{pa 1472 322}\special{fp}\special{pa 1511 322}\special{pa 1550 322}\special{fp}%
\special{pa 1590 322}\special{pa 1629 322}\special{fp}\special{pa 1668 322}\special{pa 1707 322}\special{fp}%
\special{pa 1747 322}\special{pa 1786 322}\special{fp}\special{pa 1825 322}\special{pa 1864 322}\special{fp}%
\special{pa 1904 322}\special{pa 1943 322}\special{fp}\special{pa 1982 322}\special{pa 2021 322}\special{fp}%
%
%
\special{pa -2021 161}\special{pa -1982 161}\special{fp}\special{pa -1943 161}\special{pa -1904 161}\special{fp}%
\special{pa -1864 161}\special{pa -1825 161}\special{fp}\special{pa -1786 161}\special{pa -1747 161}\special{fp}%
\special{pa -1707 161}\special{pa -1668 161}\special{fp}\special{pa -1629 161}\special{pa -1590 161}\special{fp}%
\special{pa -1550 161}\special{pa -1511 161}\special{fp}\special{pa -1472 161}\special{pa -1433 161}\special{fp}%
\special{pa -1393 161}\special{pa -1354 161}\special{fp}\special{pa -1315 161}\special{pa -1276 161}\special{fp}%
\special{pa -1236 161}\special{pa -1197 161}\special{fp}\special{pa -1158 161}\special{pa -1119 161}\special{fp}%
\special{pa -1079 161}\special{pa -1040 161}\special{fp}\special{pa -1001 161}\special{pa -962 161}\special{fp}%
\special{pa -922 161}\special{pa -883 161}\special{fp}\special{pa -844 161}\special{pa -805 161}\special{fp}%
\special{pa -765 161}\special{pa -726 161}\special{fp}\special{pa -687 161}\special{pa -648 161}\special{fp}%
\special{pa -608 161}\special{pa -569 161}\special{fp}\special{pa -530 161}\special{pa -491 161}\special{fp}%
\special{pa -451 161}\special{pa -412 161}\special{fp}\special{pa -373 161}\special{pa -334 161}\special{fp}%
\special{pa -294 161}\special{pa -255 161}\special{fp}\special{pa -216 161}\special{pa -177 161}\special{fp}%
\special{pa -137 161}\special{pa -98 161}\special{fp}\special{pa -59 161}\special{pa -20 161}\special{fp}%
\special{pa 20 161}\special{pa 59 161}\special{fp}\special{pa 98 161}\special{pa 137 161}\special{fp}%
\special{pa 177 161}\special{pa 216 161}\special{fp}\special{pa 255 161}\special{pa 294 161}\special{fp}%
\special{pa 334 161}\special{pa 373 161}\special{fp}\special{pa 412 161}\special{pa 451 161}\special{fp}%
\special{pa 491 161}\special{pa 530 161}\special{fp}\special{pa 569 161}\special{pa 608 161}\special{fp}%
\special{pa 648 161}\special{pa 687 161}\special{fp}\special{pa 726 161}\special{pa 765 161}\special{fp}%
\special{pa 805 161}\special{pa 844 161}\special{fp}\special{pa 883 161}\special{pa 922 161}\special{fp}%
\special{pa 962 161}\special{pa 1001 161}\special{fp}\special{pa 1040 161}\special{pa 1079 161}\special{fp}%
\special{pa 1119 161}\special{pa 1158 161}\special{fp}\special{pa 1197 161}\special{pa 1236 161}\special{fp}%
\special{pa 1276 161}\special{pa 1315 161}\special{fp}\special{pa 1354 161}\special{pa 1393 161}\special{fp}%
\special{pa 1433 161}\special{pa 1472 161}\special{fp}\special{pa 1511 161}\special{pa 1550 161}\special{fp}%
\special{pa 1590 161}\special{pa 1629 161}\special{fp}\special{pa 1668 161}\special{pa 1707 161}\special{fp}%
\special{pa 1747 161}\special{pa 1786 161}\special{fp}\special{pa 1825 161}\special{pa 1864 161}\special{fp}%
\special{pa 1904 161}\special{pa 1943 161}\special{fp}\special{pa 1982 161}\special{pa 2021 161}\special{fp}%
%
%
\special{pa -2021 0}\special{pa -1982 0}\special{fp}\special{pa -1943 0}\special{pa -1904 0}\special{fp}%
\special{pa -1864 0}\special{pa -1825 0}\special{fp}\special{pa -1786 0}\special{pa -1747 0}\special{fp}%
\special{pa -1707 0}\special{pa -1668 0}\special{fp}\special{pa -1629 0}\special{pa -1590 0}\special{fp}%
\special{pa -1550 0}\special{pa -1511 0}\special{fp}\special{pa -1472 0}\special{pa -1433 0}\special{fp}%
\special{pa -1393 0}\special{pa -1354 0}\special{fp}\special{pa -1315 0}\special{pa -1276 0}\special{fp}%
\special{pa -1236 0}\special{pa -1197 0}\special{fp}\special{pa -1158 0}\special{pa -1119 0}\special{fp}%
\special{pa -1079 0}\special{pa -1040 0}\special{fp}\special{pa -1001 0}\special{pa -962 0}\special{fp}%
\special{pa -922 0}\special{pa -883 0}\special{fp}\special{pa -844 0}\special{pa -805 0}\special{fp}%
\special{pa -765 0}\special{pa -726 0}\special{fp}\special{pa -687 0}\special{pa -648 0}\special{fp}%
\special{pa -608 0}\special{pa -569 0}\special{fp}\special{pa -530 0}\special{pa -491 0}\special{fp}%
\special{pa -451 0}\special{pa -412 0}\special{fp}\special{pa -373 0}\special{pa -334 0}\special{fp}%
\special{pa -294 0}\special{pa -255 0}\special{fp}\special{pa -216 0}\special{pa -177 0}\special{fp}%
\special{pa -137 0}\special{pa -98 0}\special{fp}\special{pa -59 0}\special{pa -20 0}\special{fp}%
\special{pa 20 0}\special{pa 59 0}\special{fp}\special{pa 98 0}\special{pa 137 0}\special{fp}%
\special{pa 177 0}\special{pa 216 0}\special{fp}\special{pa 255 0}\special{pa 294 0}\special{fp}%
\special{pa 334 0}\special{pa 373 0}\special{fp}\special{pa 412 0}\special{pa 451 0}\special{fp}%
\special{pa 491 0}\special{pa 530 0}\special{fp}\special{pa 569 0}\special{pa 608 0}\special{fp}%
\special{pa 648 0}\special{pa 687 0}\special{fp}\special{pa 726 0}\special{pa 765 0}\special{fp}%
\special{pa 805 0}\special{pa 844 0}\special{fp}\special{pa 883 0}\special{pa 922 0}\special{fp}%
\special{pa 962 0}\special{pa 1001 0}\special{fp}\special{pa 1040 0}\special{pa 1079 0}\special{fp}%
\special{pa 1119 0}\special{pa 1158 0}\special{fp}\special{pa 1197 0}\special{pa 1236 0}\special{fp}%
\special{pa 1276 0}\special{pa 1315 0}\special{fp}\special{pa 1354 0}\special{pa 1393 0}\special{fp}%
\special{pa 1433 0}\special{pa 1472 0}\special{fp}\special{pa 1511 0}\special{pa 1550 0}\special{fp}%
\special{pa 1590 0}\special{pa 1629 0}\special{fp}\special{pa 1668 0}\special{pa 1707 0}\special{fp}%
\special{pa 1747 0}\special{pa 1786 0}\special{fp}\special{pa 1825 0}\special{pa 1864 0}\special{fp}%
\special{pa 1904 0}\special{pa 1943 0}\special{fp}\special{pa 1982 0}\special{pa 2021 0}\special{fp}%
%
%
\special{pa -2021 -161}\special{pa -1982 -161}\special{fp}\special{pa -1943 -161}\special{pa -1904 -161}\special{fp}%
\special{pa -1864 -161}\special{pa -1825 -161}\special{fp}\special{pa -1786 -161}\special{pa -1747 -161}\special{fp}%
\special{pa -1707 -161}\special{pa -1668 -161}\special{fp}\special{pa -1629 -161}\special{pa -1590 -161}\special{fp}%
\special{pa -1550 -161}\special{pa -1511 -161}\special{fp}\special{pa -1472 -161}\special{pa -1433 -161}\special{fp}%
\special{pa -1393 -161}\special{pa -1354 -161}\special{fp}\special{pa -1315 -161}\special{pa -1276 -161}\special{fp}%
\special{pa -1236 -161}\special{pa -1197 -161}\special{fp}\special{pa -1158 -161}\special{pa -1119 -161}\special{fp}%
\special{pa -1079 -161}\special{pa -1040 -161}\special{fp}\special{pa -1001 -161}\special{pa -962 -161}\special{fp}%
\special{pa -922 -161}\special{pa -883 -161}\special{fp}\special{pa -844 -161}\special{pa -805 -161}\special{fp}%
\special{pa -765 -161}\special{pa -726 -161}\special{fp}\special{pa -687 -161}\special{pa -648 -161}\special{fp}%
\special{pa -608 -161}\special{pa -569 -161}\special{fp}\special{pa -530 -161}\special{pa -491 -161}\special{fp}%
\special{pa -451 -161}\special{pa -412 -161}\special{fp}\special{pa -373 -161}\special{pa -334 -161}\special{fp}%
\special{pa -294 -161}\special{pa -255 -161}\special{fp}\special{pa -216 -161}\special{pa -177 -161}\special{fp}%
\special{pa -137 -161}\special{pa -98 -161}\special{fp}\special{pa -59 -161}\special{pa -20 -161}\special{fp}%
\special{pa 20 -161}\special{pa 59 -161}\special{fp}\special{pa 98 -161}\special{pa 137 -161}\special{fp}%
\special{pa 177 -161}\special{pa 216 -161}\special{fp}\special{pa 255 -161}\special{pa 294 -161}\special{fp}%
\special{pa 334 -161}\special{pa 373 -161}\special{fp}\special{pa 412 -161}\special{pa 451 -161}\special{fp}%
\special{pa 491 -161}\special{pa 530 -161}\special{fp}\special{pa 569 -161}\special{pa 608 -161}\special{fp}%
\special{pa 648 -161}\special{pa 687 -161}\special{fp}\special{pa 726 -161}\special{pa 765 -161}\special{fp}%
\special{pa 805 -161}\special{pa 844 -161}\special{fp}\special{pa 883 -161}\special{pa 922 -161}\special{fp}%
\special{pa 962 -161}\special{pa 1001 -161}\special{fp}\special{pa 1040 -161}\special{pa 1079 -161}\special{fp}%
\special{pa 1119 -161}\special{pa 1158 -161}\special{fp}\special{pa 1197 -161}\special{pa 1236 -161}\special{fp}%
\special{pa 1276 -161}\special{pa 1315 -161}\special{fp}\special{pa 1354 -161}\special{pa 1393 -161}\special{fp}%
\special{pa 1433 -161}\special{pa 1472 -161}\special{fp}\special{pa 1511 -161}\special{pa 1550 -161}\special{fp}%
\special{pa 1590 -161}\special{pa 1629 -161}\special{fp}\special{pa 1668 -161}\special{pa 1707 -161}\special{fp}%
\special{pa 1747 -161}\special{pa 1786 -161}\special{fp}\special{pa 1825 -161}\special{pa 1864 -161}\special{fp}%
\special{pa 1904 -161}\special{pa 1943 -161}\special{fp}\special{pa 1982 -161}\special{pa 2021 -161}\special{fp}%
%
%
\special{pa -2021 -322}\special{pa -1982 -322}\special{fp}\special{pa -1943 -322}\special{pa -1904 -322}\special{fp}%
\special{pa -1864 -322}\special{pa -1825 -322}\special{fp}\special{pa -1786 -322}\special{pa -1747 -322}\special{fp}%
\special{pa -1707 -322}\special{pa -1668 -322}\special{fp}\special{pa -1629 -322}\special{pa -1590 -322}\special{fp}%
\special{pa -1550 -322}\special{pa -1511 -322}\special{fp}\special{pa -1472 -322}\special{pa -1433 -322}\special{fp}%
\special{pa -1393 -322}\special{pa -1354 -322}\special{fp}\special{pa -1315 -322}\special{pa -1276 -322}\special{fp}%
\special{pa -1236 -322}\special{pa -1197 -322}\special{fp}\special{pa -1158 -322}\special{pa -1119 -322}\special{fp}%
\special{pa -1079 -322}\special{pa -1040 -322}\special{fp}\special{pa -1001 -322}\special{pa -962 -322}\special{fp}%
\special{pa -922 -322}\special{pa -883 -322}\special{fp}\special{pa -844 -322}\special{pa -805 -322}\special{fp}%
\special{pa -765 -322}\special{pa -726 -322}\special{fp}\special{pa -687 -322}\special{pa -648 -322}\special{fp}%
\special{pa -608 -322}\special{pa -569 -322}\special{fp}\special{pa -530 -322}\special{pa -491 -322}\special{fp}%
\special{pa -451 -322}\special{pa -412 -322}\special{fp}\special{pa -373 -322}\special{pa -334 -322}\special{fp}%
\special{pa -294 -322}\special{pa -255 -322}\special{fp}\special{pa -216 -322}\special{pa -177 -322}\special{fp}%
\special{pa -137 -322}\special{pa -98 -322}\special{fp}\special{pa -59 -322}\special{pa -20 -322}\special{fp}%
\special{pa 20 -322}\special{pa 59 -322}\special{fp}\special{pa 98 -322}\special{pa 137 -322}\special{fp}%
\special{pa 177 -322}\special{pa 216 -322}\special{fp}\special{pa 255 -322}\special{pa 294 -322}\special{fp}%
\special{pa 334 -322}\special{pa 373 -322}\special{fp}\special{pa 412 -322}\special{pa 451 -322}\special{fp}%
\special{pa 491 -322}\special{pa 530 -322}\special{fp}\special{pa 569 -322}\special{pa 608 -322}\special{fp}%
\special{pa 648 -322}\special{pa 687 -322}\special{fp}\special{pa 726 -322}\special{pa 765 -322}\special{fp}%
\special{pa 805 -322}\special{pa 844 -322}\special{fp}\special{pa 883 -322}\special{pa 922 -322}\special{fp}%
\special{pa 962 -322}\special{pa 1001 -322}\special{fp}\special{pa 1040 -322}\special{pa 1079 -322}\special{fp}%
\special{pa 1119 -322}\special{pa 1158 -322}\special{fp}\special{pa 1197 -322}\special{pa 1236 -322}\special{fp}%
\special{pa 1276 -322}\special{pa 1315 -322}\special{fp}\special{pa 1354 -322}\special{pa 1393 -322}\special{fp}%
\special{pa 1433 -322}\special{pa 1472 -322}\special{fp}\special{pa 1511 -322}\special{pa 1550 -322}\special{fp}%
\special{pa 1590 -322}\special{pa 1629 -322}\special{fp}\special{pa 1668 -322}\special{pa 1707 -322}\special{fp}%
\special{pa 1747 -322}\special{pa 1786 -322}\special{fp}\special{pa 1825 -322}\special{pa 1864 -322}\special{fp}%
\special{pa 1904 -322}\special{pa 1943 -322}\special{fp}\special{pa 1982 -322}\special{pa 2021 -322}\special{fp}%
%
%
\special{pa -2021 -483}\special{pa -1982 -483}\special{fp}\special{pa -1943 -483}\special{pa -1904 -483}\special{fp}%
\special{pa -1864 -483}\special{pa -1825 -483}\special{fp}\special{pa -1786 -483}\special{pa -1747 -483}\special{fp}%
\special{pa -1707 -483}\special{pa -1668 -483}\special{fp}\special{pa -1629 -483}\special{pa -1590 -483}\special{fp}%
\special{pa -1550 -483}\special{pa -1511 -483}\special{fp}\special{pa -1472 -483}\special{pa -1433 -483}\special{fp}%
\special{pa -1393 -483}\special{pa -1354 -483}\special{fp}\special{pa -1315 -483}\special{pa -1276 -483}\special{fp}%
\special{pa -1236 -483}\special{pa -1197 -483}\special{fp}\special{pa -1158 -483}\special{pa -1119 -483}\special{fp}%
\special{pa -1079 -483}\special{pa -1040 -483}\special{fp}\special{pa -1001 -483}\special{pa -962 -483}\special{fp}%
\special{pa -922 -483}\special{pa -883 -483}\special{fp}\special{pa -844 -483}\special{pa -805 -483}\special{fp}%
\special{pa -765 -483}\special{pa -726 -483}\special{fp}\special{pa -687 -483}\special{pa -648 -483}\special{fp}%
\special{pa -608 -483}\special{pa -569 -483}\special{fp}\special{pa -530 -483}\special{pa -491 -483}\special{fp}%
\special{pa -451 -483}\special{pa -412 -483}\special{fp}\special{pa -373 -483}\special{pa -334 -483}\special{fp}%
\special{pa -294 -483}\special{pa -255 -483}\special{fp}\special{pa -216 -483}\special{pa -177 -483}\special{fp}%
\special{pa -137 -483}\special{pa -98 -483}\special{fp}\special{pa -59 -483}\special{pa -20 -483}\special{fp}%
\special{pa 20 -483}\special{pa 59 -483}\special{fp}\special{pa 98 -483}\special{pa 137 -483}\special{fp}%
\special{pa 177 -483}\special{pa 216 -483}\special{fp}\special{pa 255 -483}\special{pa 294 -483}\special{fp}%
\special{pa 334 -483}\special{pa 373 -483}\special{fp}\special{pa 412 -483}\special{pa 451 -483}\special{fp}%
\special{pa 491 -483}\special{pa 530 -483}\special{fp}\special{pa 569 -483}\special{pa 608 -483}\special{fp}%
\special{pa 648 -483}\special{pa 687 -483}\special{fp}\special{pa 726 -483}\special{pa 765 -483}\special{fp}%
\special{pa 805 -483}\special{pa 844 -483}\special{fp}\special{pa 883 -483}\special{pa 922 -483}\special{fp}%
\special{pa 962 -483}\special{pa 1001 -483}\special{fp}\special{pa 1040 -483}\special{pa 1079 -483}\special{fp}%
\special{pa 1119 -483}\special{pa 1158 -483}\special{fp}\special{pa 1197 -483}\special{pa 1236 -483}\special{fp}%
\special{pa 1276 -483}\special{pa 1315 -483}\special{fp}\special{pa 1354 -483}\special{pa 1393 -483}\special{fp}%
\special{pa 1433 -483}\special{pa 1472 -483}\special{fp}\special{pa 1511 -483}\special{pa 1550 -483}\special{fp}%
\special{pa 1590 -483}\special{pa 1629 -483}\special{fp}\special{pa 1668 -483}\special{pa 1707 -483}\special{fp}%
\special{pa 1747 -483}\special{pa 1786 -483}\special{fp}\special{pa 1825 -483}\special{pa 1864 -483}\special{fp}%
\special{pa 1904 -483}\special{pa 1943 -483}\special{fp}\special{pa 1982 -483}\special{pa 2021 -483}\special{fp}%
%
%
\special{pa -2021 -643}\special{pa -1982 -643}\special{fp}\special{pa -1943 -643}\special{pa -1904 -643}\special{fp}%
\special{pa -1864 -643}\special{pa -1825 -643}\special{fp}\special{pa -1786 -643}\special{pa -1747 -643}\special{fp}%
\special{pa -1707 -643}\special{pa -1668 -643}\special{fp}\special{pa -1629 -643}\special{pa -1590 -643}\special{fp}%
\special{pa -1550 -643}\special{pa -1511 -643}\special{fp}\special{pa -1472 -643}\special{pa -1433 -643}\special{fp}%
\special{pa -1393 -643}\special{pa -1354 -643}\special{fp}\special{pa -1315 -643}\special{pa -1276 -643}\special{fp}%
\special{pa -1236 -643}\special{pa -1197 -643}\special{fp}\special{pa -1158 -643}\special{pa -1119 -643}\special{fp}%
\special{pa -1079 -643}\special{pa -1040 -643}\special{fp}\special{pa -1001 -643}\special{pa -962 -643}\special{fp}%
\special{pa -922 -643}\special{pa -883 -643}\special{fp}\special{pa -844 -643}\special{pa -805 -643}\special{fp}%
\special{pa -765 -643}\special{pa -726 -643}\special{fp}\special{pa -687 -643}\special{pa -648 -643}\special{fp}%
\special{pa -608 -643}\special{pa -569 -643}\special{fp}\special{pa -530 -643}\special{pa -491 -643}\special{fp}%
\special{pa -451 -643}\special{pa -412 -643}\special{fp}\special{pa -373 -643}\special{pa -334 -643}\special{fp}%
\special{pa -294 -643}\special{pa -255 -643}\special{fp}\special{pa -216 -643}\special{pa -177 -643}\special{fp}%
\special{pa -137 -643}\special{pa -98 -643}\special{fp}\special{pa -59 -643}\special{pa -20 -643}\special{fp}%
\special{pa 20 -643}\special{pa 59 -643}\special{fp}\special{pa 98 -643}\special{pa 137 -643}\special{fp}%
\special{pa 177 -643}\special{pa 216 -643}\special{fp}\special{pa 255 -643}\special{pa 294 -643}\special{fp}%
\special{pa 334 -643}\special{pa 373 -643}\special{fp}\special{pa 412 -643}\special{pa 451 -643}\special{fp}%
\special{pa 491 -643}\special{pa 530 -643}\special{fp}\special{pa 569 -643}\special{pa 608 -643}\special{fp}%
\special{pa 648 -643}\special{pa 687 -643}\special{fp}\special{pa 726 -643}\special{pa 765 -643}\special{fp}%
\special{pa 805 -643}\special{pa 844 -643}\special{fp}\special{pa 883 -643}\special{pa 922 -643}\special{fp}%
\special{pa 962 -643}\special{pa 1001 -643}\special{fp}\special{pa 1040 -643}\special{pa 1079 -643}\special{fp}%
\special{pa 1119 -643}\special{pa 1158 -643}\special{fp}\special{pa 1197 -643}\special{pa 1236 -643}\special{fp}%
\special{pa 1276 -643}\special{pa 1315 -643}\special{fp}\special{pa 1354 -643}\special{pa 1393 -643}\special{fp}%
\special{pa 1433 -643}\special{pa 1472 -643}\special{fp}\special{pa 1511 -643}\special{pa 1550 -643}\special{fp}%
\special{pa 1590 -643}\special{pa 1629 -643}\special{fp}\special{pa 1668 -643}\special{pa 1707 -643}\special{fp}%
\special{pa 1747 -643}\special{pa 1786 -643}\special{fp}\special{pa 1825 -643}\special{pa 1864 -643}\special{fp}%
\special{pa 1904 -643}\special{pa 1943 -643}\special{fp}\special{pa 1982 -643}\special{pa 2021 -643}\special{fp}%
%
%
\special{pn 8}%
\special{pa -2021   -20}\special{pa -2021    20}%
\special{fp}%
\settowidth{\Width}{$-2\pi$}\setlength{\Width}{-0.5\Width}%
\settoheight{\Height}{$-2\pi$}\settodepth{\Depth}{$-2\pi$}\setlength{\Height}{-\Height}%
\put(-6.2831850,-0.1223810){\hspace*{\Width}\raisebox{\Height}{$-2\pi$}}%
%
%
\special{pa -1011   -20}\special{pa -1011    20}%
\special{fp}%
\settowidth{\Width}{$-\pi$}\setlength{\Width}{-0.5\Width}%
\settoheight{\Height}{$-\pi$}\settodepth{\Depth}{$-\pi$}\setlength{\Height}{-\Height}%
\put(-3.1415930,-0.1223810){\hspace*{\Width}\raisebox{\Height}{$-\pi$}}%
%
%
\special{pa  1011   -20}\special{pa  1011    20}%
\special{fp}%
\settowidth{\Width}{$\pi$}\setlength{\Width}{-0.5\Width}%
\settoheight{\Height}{$\pi$}\settodepth{\Depth}{$\pi$}\setlength{\Height}{-\Height}%
\put(3.1415930,-0.1223810){\hspace*{\Width}\raisebox{\Height}{$\pi$}}%
%
%
\special{pa  2021   -20}\special{pa  2021    20}%
\special{fp}%
\settowidth{\Width}{$2\pi$}\setlength{\Width}{-0.5\Width}%
\settoheight{\Height}{$2\pi$}\settodepth{\Depth}{$2\pi$}\setlength{\Height}{-\Height}%
\put(6.2831850,-0.1223810){\hspace*{\Width}\raisebox{\Height}{$2\pi$}}%
%
%
\special{pa    20   643}\special{pa   -20   643}%
\special{fp}%
\settowidth{\Width}{$-2$}\setlength{\Width}{-1\Width}%
\settoheight{\Height}{$-2$}\settodepth{\Depth}{$-2$}\setlength{\Height}{-0.5\Height}\setlength{\Depth}{0.5\Depth}\addtolength{\Height}{\Depth}%
\put(-0.1223810,-2.0000000){\hspace*{\Width}\raisebox{\Height}{$-2$}}%
%
%
\special{pa    20   322}\special{pa   -20   322}%
\special{fp}%
\settowidth{\Width}{$-1$}\setlength{\Width}{-1\Width}%
\settoheight{\Height}{$-1$}\settodepth{\Depth}{$-1$}\setlength{\Height}{-0.5\Height}\setlength{\Depth}{0.5\Depth}\addtolength{\Height}{\Depth}%
\put(-0.1223810,-1.0000000){\hspace*{\Width}\raisebox{\Height}{$-1$}}%
%
%
\special{pa    20  -322}\special{pa   -20  -322}%
\special{fp}%
\settowidth{\Width}{$1$}\setlength{\Width}{-1\Width}%
\settoheight{\Height}{$1$}\settodepth{\Depth}{$1$}\setlength{\Height}{-0.5\Height}\setlength{\Depth}{0.5\Depth}\addtolength{\Height}{\Depth}%
\put(-0.1223810,1.0000000){\hspace*{\Width}\raisebox{\Height}{$1$}}%
%
%
\special{pa    20  -643}\special{pa   -20  -643}%
\special{fp}%
\settowidth{\Width}{$2$}\setlength{\Width}{-1\Width}%
\settoheight{\Height}{$2$}\settodepth{\Depth}{$2$}\setlength{\Height}{-0.5\Height}\setlength{\Depth}{0.5\Depth}\addtolength{\Height}{\Depth}%
\put(-0.1223810,2.0000000){\hspace*{\Width}\raisebox{\Height}{$2$}}%
%
%
\special{pa -2104    -0}\special{pa  2104    -0}%
\special{fp}%
\special{pa     0   676}\special{pa     0  -676}%
\special{fp}%
\settowidth{\Width}{$x$}\setlength{\Width}{0\Width}%
\settoheight{\Height}{$x$}\settodepth{\Depth}{$x$}\setlength{\Height}{-0.5\Height}\setlength{\Depth}{0.5\Depth}\addtolength{\Height}{\Depth}%
\put(6.6011905,0.0000000){\hspace*{\Width}\raisebox{\Height}{$x$}}%
%
\settowidth{\Width}{$y$}\setlength{\Width}{-0.5\Width}%
\settoheight{\Height}{$y$}\settodepth{\Depth}{$y$}\setlength{\Height}{\Depth}%
\put(0.0000000,2.1611905){\hspace*{\Width}\raisebox{\Height}{$y$}}%
%
\settowidth{\Width}{O}\setlength{\Width}{0\Width}%
\settoheight{\Height}{O}\settodepth{\Depth}{O}\setlength{\Height}{-\Height}%
\put(0.0611905,-0.0611905){\hspace*{\Width}\raisebox{\Height}{O}}%
%
\end{picture}}%
\end{center}

\vspace{2mm}

]

\small

\begin{Enumerate}[{\bf \large 1}]

\item 上の図は$y=\sin x$と$y=\cos x$のグラフである.
\begin{enumerate}[(1)]
\item いずれも,振幅は\hako{10}{5}{},周期は\hako{10}{5}{}である.
\item $y=\cos x$は,$y=\sin x$と比べて位相が\;\hako{10}{5}{}だけ\\
\hako{20}{6}{}いる.
\item $y=2\sin x$の振幅は\hako{10}{5}{},周期は\hako{10}{5}{}である.
\item $y=\sin 2x$の振幅は\hako{10}{5}{},周期は\hako{10}{5}{}である.
\item $y=\cos\bigl(x+\dfrac{\pi}{4}\bigr)$は,$y=\cos x$より位相が\;\hako{10}{5}{}だけ\;
\hako{20}{6}{}いる.

\item $y=2\sin x$のグラフを描き入れよ.
\item $y=\sin 2x$のグラフを描き入れよ.
\item $y=\cos\bigl(x+\dfrac{\pi}{4}\bigr)$のグラフを描き入れよ.
\end{enumerate}

\vspace{2zw}


\item 直線について,次の問いに答えよ.
\begin{enumerate}[(1)]
\item 原点を通り,傾きが$2$の直線(1)の方程式を求めよ.\vspace{3zw}
\item 直線(1)に平行で,$y$切片が$3$の直線(2)の方程式を求めよ.\vspace{3zw}
\item 直線(1)に垂直で,原点を通る直線(3)の方程式を求めよ.\vspace{3zw}
\item 直線(2)と直線(3)の交点の座標Pを求めよ.\vspace{3zw}
\item 原点Oと点Pの距離を求めよ.\vspace{3zw}
\end{enumerate}

\newpage

\item 円について,次の問いに答えよ.
\begin{enumerate}[(1)]
\item 原点Oを中心とし,半径が$2$の円(1)の方程式を求めよ.\vspace{2zw}
\item 点$(1,2)$を中心とし,半径が$2$の円(2)の方程式を求めよ.\vspace{2zw}
\item (2)の方程式を展開してまとめると\\
$\hako{6}{5}{}\;x^2+\hako{6}{5}{}\;y^2+\hako{6}{5}{}\;x+\hako{6}{5}{}\;y+\hako{6}{5}{}=0$\vspace{2zw}
\item Oと$\mathrm{A}(1,\sqrt{3})$との距離は\;\;\hako{10}{5}{}である.\\
Aは円(1)の\;\hako{10}{5}{}にある.(内部,周上,外部)\vspace{2zw}
\item 原点Oと点Aを通る直線の傾きを求めよ.\vspace{2zw}
\item Aを通り,(5)の直線と垂直な直線の方程式を求めよ.\vspace{5zw}
\end{enumerate}

\item 2次曲線について,次の問いに答えよ.
\begin{enumerate}[(1)]
\item 放物線$y^2=x$の焦点$\mathrm{F}_1$を求めよ.\vspace{2zw}
\item $\mathrm{F}_1$から出た光線は放物線で反射するとどうなるか.\vspace{2zw}
\item 楕円$\dfrac{x^2}{6}+\dfrac{y^2}{2}=1$の焦点$\mathrm{F}_2,\mathrm{F}'_2$を求めよ.\vspace{2zw}
\item $\mathrm{F}_2$から出た光線は楕円で反射するとどうなるか.\vspace{2zw}
\item 双曲線$\dfrac{x^2}{4}-\dfrac{y^2}{1}=1$の焦点$\mathrm{F}_3,\mathrm{F}'_3$を求めよ.\vspace{2zw}
\item 上の双曲線の漸近線の方程式を求めよ.
\end{enumerate}


\end{Enumerate}

\end{document}