\documentclass[dvipdfmx,a4paper,10pt]{ujarticle}

\usepackage{pict2e}
\usepackage{ketpic2e,ketlayer2e}
\usepackage{graphicx,color}


\usepackage{emath}
\usepackage{emathMw,emathEy}

\setmargin{15}{15}{10}{20}

\pagestyle{empty}

\columnsep=0.5cm
\columnseprule=0.5pt

\newcommand{\hako}[4][-1]{%
\setcounter{ketpicctra}{#2}%
\divide\value{ketpicctra} by 2%
\setcounter{ketpicctrb}{#3}%
\divide\value{ketpicctrb} by 2%
\setcounter{ketpicctrc}{\theketpicctrb}%
\addtocounter{ketpicctrc}{#1}%
\def\kettmp{
\begin{picture}%
(#2, #3)(0,0)%
\settowidth{\Width}{#4}\setlength{\Width}{-0.5\Width}%
\settoheight{\Height}{#4}\settodepth{\Depth}{#4}\setlength{\Height}{-0.5\Height}\setlength{\Depth}{0.5\Depth}\addtolength{\Height}{\Depth}%
\put(\theketpicctra,\theketpicctrb){\hspace*{\Width}\raisebox{\Height}{#4}}%
\end{picture}%
}%
{\unitlength=1mm%
\raisebox{-\theketpicctrc mm}{\fbox{\kettmp}}%
}
}

\begin{document}

\noindent
\twocolumn[
{\tabcolsep=0mm
\Ctab{180mm}{{\Large 定期試験}}\vspace{-3mm}%

%\begin{tabular}[b]{|c|c|}\hline
%\Ctab{2cm}{教室番号}\mbox{}& \Ctab{2cm}{座席番号}\mbox{}\\\hline
%\raisebox{8mm}{\mbox{}}& \\\hline
%\end{tabular}%
%{\large 
%\hspace{1.64cm}(\sentaku{春}・秋)学期 定期試験
%}%
\hspace{0.82cm}%
\hfill 2023年6月19日\vspace{1mm}% 第3時限(\ 1/1\ 枚)\vspace{1mm}e

\begin{tabular}{|c|c|c|c|c|c|c|}\hline
\Ctab{4.1cm}{科 目}\mbox{} & \Ctab{2.46cm}{担 当 者}\mbox{}%
& \Ctab{2cm}{科}\mbox{} & \Ctab{0.984cm}{学年}\mbox{}%
& \Ctab{2cm}{番 号}\mbox{} & \Ctab{4.4cm}{氏    名}\mbox{}%
& \Ctab{2.2cm}{評 点}\mbox{}\\\hline
\raisebox{3 mm}{$\mathstrut$}\raisebox{-3mm}{$\mathstrut$}%
基礎数学& 高 遠 & 生産技術 & 1
 & & & \\\hline
\end{tabular}
}
\vspace{1mm}

\begin{center}
%%% /Users/takatoosetsuo/Dropbox/2018polytec/lecture/0618/fig/sincos.tex 
%%% Generator=fig0618.cdy 
{\unitlength=8.171206mm%
\begin{picture}%
(13.08,4.2)(-6.54,-2.1)%
\special{pn 8}%
%
\small%
\special{pn 12}%
\special{pa -2104  -311}\special{pa -2062  -319}\special{pa -2020  -322}\special{pa -1978  -319}%
\special{pa -1936  -310}\special{pa -1894  -297}\special{pa -1851  -278}\special{pa -1809  -254}%
\special{pa -1767  -227}\special{pa -1725  -195}\special{pa -1683  -160}\special{pa -1641  -122}%
\special{pa -1599   -82}\special{pa -1557   -41}\special{pa -1515     1}\special{pa -1473    43}%
\special{pa -1431    84}\special{pa -1389   124}\special{pa -1347   162}\special{pa -1304   197}%
\special{pa -1262   228}\special{pa -1220   256}\special{pa -1178   279}\special{pa -1136   298}%
\special{pa -1094   311}\special{pa -1052   319}\special{pa -1010   322}\special{pa  -968   319}%
\special{pa  -926   311}\special{pa  -884   297}\special{pa  -842   278}\special{pa  -799   255}%
\special{pa  -757   227}\special{pa  -715   195}\special{pa  -673   160}\special{pa  -631   123}%
\special{pa  -589    83}\special{pa  -547    42}\special{pa  -505    -0}\special{pa  -463   -42}%
\special{pa  -421   -84}\special{pa  -379  -123}\special{pa  -337  -161}\special{pa  -295  -196}%
\special{pa  -252  -228}\special{pa  -210  -255}\special{pa  -168  -279}\special{pa  -126  -297}%
\special{pa   -84  -311}\special{pa   -42  -319}\special{pa     0  -322}\special{pa    42  -319}%
\special{pa    84  -311}\special{pa   126  -297}\special{pa   168  -279}\special{pa   210  -255}%
\special{pa   252  -228}\special{pa   295  -196}\special{pa   337  -161}\special{pa   379  -123}%
\special{pa   421   -84}\special{pa   463   -42}\special{pa   505    -0}\special{pa   547    42}%
\special{pa   589    83}\special{pa   631   123}\special{pa   673   160}\special{pa   715   195}%
\special{pa   757   227}\special{pa   799   255}\special{pa   842   278}\special{pa   884   297}%
\special{pa   926   311}\special{pa   968   319}\special{pa  1010   322}\special{pa  1052   319}%
\special{pa  1094   311}\special{pa  1136   298}\special{pa  1178   279}\special{pa  1220   256}%
\special{pa  1262   228}\special{pa  1304   197}\special{pa  1347   162}\special{pa  1389   124}%
\special{pa  1431    84}\special{pa  1473    43}\special{pa  1515     1}\special{pa  1557   -41}%
\special{pa  1599   -82}\special{pa  1641  -122}\special{pa  1683  -160}\special{pa  1725  -195}%
\special{pa  1767  -227}\special{pa  1809  -254}\special{pa  1851  -278}\special{pa  1894  -297}%
\special{pa  1936  -310}\special{pa  1978  -319}\special{pa  2020  -322}\special{pa  2062  -319}%
\special{pa  2104  -311}%
\special{fp}%
\special{pn 8}%
\special{pn 12}%
\special{pa -2104   319}\special{pa -2062   321}\special{pa -2020   322}\special{pa -1978   321}%
\special{pa -1936   319}\special{pa -1894   315}\special{pa -1851   311}\special{pa -1809   304}%
\special{pa -1767   297}\special{pa -1725   288}\special{pa -1683   278}\special{pa -1641   267}%
\special{pa -1599   255}\special{pa -1557   241}\special{pa -1515   227}\special{pa -1473   212}%
\special{pa -1431   195}\special{pa -1389   178}\special{pa -1347   160}\special{pa -1304   142}%
\special{pa -1262   123}\special{pa -1220   103}\special{pa -1178    83}\special{pa -1136    62}%
\special{pa -1094    42}\special{pa -1052    21}\special{pa -1010    -0}\special{pa  -968   -21}%
\special{pa  -926   -42}\special{pa  -884   -63}\special{pa  -842   -84}\special{pa  -799  -104}%
\special{pa  -757  -123}\special{pa  -715  -143}\special{pa  -673  -161}\special{pa  -631  -179}%
\special{pa  -589  -196}\special{pa  -547  -212}\special{pa  -505  -228}\special{pa  -463  -242}%
\special{pa  -421  -255}\special{pa  -379  -268}\special{pa  -337  -279}\special{pa  -295  -289}%
\special{pa  -252  -297}\special{pa  -210  -305}\special{pa  -168  -311}\special{pa  -126  -316}%
\special{pa   -84  -319}\special{pa   -42  -321}\special{pa     0  -322}\special{pa    42  -321}%
\special{pa    84  -319}\special{pa   126  -316}\special{pa   168  -311}\special{pa   210  -305}%
\special{pa   252  -297}\special{pa   295  -289}\special{pa   337  -279}\special{pa   379  -268}%
\special{pa   421  -255}\special{pa   463  -242}\special{pa   505  -228}\special{pa   547  -212}%
\special{pa   589  -196}\special{pa   631  -179}\special{pa   673  -161}\special{pa   715  -143}%
\special{pa   757  -123}\special{pa   799  -104}\special{pa   842   -84}\special{pa   884   -63}%
\special{pa   926   -42}\special{pa   968   -21}\special{pa  1010    -0}\special{pa  1052    21}%
\special{pa  1094    42}\special{pa  1136    62}\special{pa  1178    83}\special{pa  1220   103}%
\special{pa  1262   123}\special{pa  1304   142}\special{pa  1347   160}\special{pa  1389   178}%
\special{pa  1431   195}\special{pa  1473   212}\special{pa  1515   227}\special{pa  1557   241}%
\special{pa  1599   255}\special{pa  1641   267}\special{pa  1683   278}\special{pa  1725   288}%
\special{pa  1767   297}\special{pa  1809   304}\special{pa  1851   311}\special{pa  1894   315}%
\special{pa  1936   319}\special{pa  1978   321}\special{pa  2020   322}\special{pa  2062   321}%
\special{pa  2104   319}%
\special{fp}%
\special{pn 8}%
\special{pn 4}%
\special{pa -2021 643}\special{pa -2021 604}\special{fp}\special{pa -2021 565}\special{pa -2021 526}\special{fp}%
\special{pa -2021 487}\special{pa -2021 448}\special{fp}\special{pa -2021 409}\special{pa -2021 370}\special{fp}%
\special{pa -2021 331}\special{pa -2021 292}\special{fp}\special{pa -2021 253}\special{pa -2021 214}\special{fp}%
\special{pa -2021 175}\special{pa -2021 136}\special{fp}\special{pa -2021 97}\special{pa -2021 58}\special{fp}%
\special{pa -2021 19}\special{pa -2021 -19}\special{fp}\special{pa -2021 -58}\special{pa -2021 -97}\special{fp}%
\special{pa -2021 -136}\special{pa -2021 -175}\special{fp}\special{pa -2021 -214}\special{pa -2021 -253}\special{fp}%
\special{pa -2021 -292}\special{pa -2021 -331}\special{fp}\special{pa -2021 -370}\special{pa -2021 -409}\special{fp}%
\special{pa -2021 -448}\special{pa -2021 -487}\special{fp}\special{pa -2021 -526}\special{pa -2021 -565}\special{fp}%
\special{pa -2021 -604}\special{pa -2021 -643}\special{fp}%
%
\special{pa -1769 643}\special{pa -1769 604}\special{fp}\special{pa -1769 565}\special{pa -1769 526}\special{fp}%
\special{pa -1769 487}\special{pa -1769 448}\special{fp}\special{pa -1769 409}\special{pa -1769 370}\special{fp}%
\special{pa -1769 331}\special{pa -1769 292}\special{fp}\special{pa -1769 253}\special{pa -1769 214}\special{fp}%
\special{pa -1769 175}\special{pa -1769 136}\special{fp}\special{pa -1769 97}\special{pa -1769 58}\special{fp}%
\special{pa -1769 19}\special{pa -1769 -19}\special{fp}\special{pa -1769 -58}\special{pa -1769 -97}\special{fp}%
\special{pa -1769 -136}\special{pa -1769 -175}\special{fp}\special{pa -1769 -214}\special{pa -1769 -253}\special{fp}%
\special{pa -1769 -292}\special{pa -1769 -331}\special{fp}\special{pa -1769 -370}\special{pa -1769 -409}\special{fp}%
\special{pa -1769 -448}\special{pa -1769 -487}\special{fp}\special{pa -1769 -526}\special{pa -1769 -565}\special{fp}%
\special{pa -1769 -604}\special{pa -1769 -643}\special{fp}%
%
\special{pa -1516 643}\special{pa -1516 604}\special{fp}\special{pa -1516 565}\special{pa -1516 526}\special{fp}%
\special{pa -1516 487}\special{pa -1516 448}\special{fp}\special{pa -1516 409}\special{pa -1516 370}\special{fp}%
\special{pa -1516 331}\special{pa -1516 292}\special{fp}\special{pa -1516 253}\special{pa -1516 214}\special{fp}%
\special{pa -1516 175}\special{pa -1516 136}\special{fp}\special{pa -1516 97}\special{pa -1516 58}\special{fp}%
\special{pa -1516 19}\special{pa -1516 -19}\special{fp}\special{pa -1516 -58}\special{pa -1516 -97}\special{fp}%
\special{pa -1516 -136}\special{pa -1516 -175}\special{fp}\special{pa -1516 -214}\special{pa -1516 -253}\special{fp}%
\special{pa -1516 -292}\special{pa -1516 -331}\special{fp}\special{pa -1516 -370}\special{pa -1516 -409}\special{fp}%
\special{pa -1516 -448}\special{pa -1516 -487}\special{fp}\special{pa -1516 -526}\special{pa -1516 -565}\special{fp}%
\special{pa -1516 -604}\special{pa -1516 -643}\special{fp}%
%
\special{pa -1263 643}\special{pa -1263 604}\special{fp}\special{pa -1263 565}\special{pa -1263 526}\special{fp}%
\special{pa -1263 487}\special{pa -1263 448}\special{fp}\special{pa -1263 409}\special{pa -1263 370}\special{fp}%
\special{pa -1263 331}\special{pa -1263 292}\special{fp}\special{pa -1263 253}\special{pa -1263 214}\special{fp}%
\special{pa -1263 175}\special{pa -1263 136}\special{fp}\special{pa -1263 97}\special{pa -1263 58}\special{fp}%
\special{pa -1263 19}\special{pa -1263 -19}\special{fp}\special{pa -1263 -58}\special{pa -1263 -97}\special{fp}%
\special{pa -1263 -136}\special{pa -1263 -175}\special{fp}\special{pa -1263 -214}\special{pa -1263 -253}\special{fp}%
\special{pa -1263 -292}\special{pa -1263 -331}\special{fp}\special{pa -1263 -370}\special{pa -1263 -409}\special{fp}%
\special{pa -1263 -448}\special{pa -1263 -487}\special{fp}\special{pa -1263 -526}\special{pa -1263 -565}\special{fp}%
\special{pa -1263 -604}\special{pa -1263 -643}\special{fp}%
%
\special{pa -1011 643}\special{pa -1011 604}\special{fp}\special{pa -1011 565}\special{pa -1011 526}\special{fp}%
\special{pa -1011 487}\special{pa -1011 448}\special{fp}\special{pa -1011 409}\special{pa -1011 370}\special{fp}%
\special{pa -1011 331}\special{pa -1011 292}\special{fp}\special{pa -1011 253}\special{pa -1011 214}\special{fp}%
\special{pa -1011 175}\special{pa -1011 136}\special{fp}\special{pa -1011 97}\special{pa -1011 58}\special{fp}%
\special{pa -1011 19}\special{pa -1011 -19}\special{fp}\special{pa -1011 -58}\special{pa -1011 -97}\special{fp}%
\special{pa -1011 -136}\special{pa -1011 -175}\special{fp}\special{pa -1011 -214}\special{pa -1011 -253}\special{fp}%
\special{pa -1011 -292}\special{pa -1011 -331}\special{fp}\special{pa -1011 -370}\special{pa -1011 -409}\special{fp}%
\special{pa -1011 -448}\special{pa -1011 -487}\special{fp}\special{pa -1011 -526}\special{pa -1011 -565}\special{fp}%
\special{pa -1011 -604}\special{pa -1011 -643}\special{fp}%
%
\special{pa -758 643}\special{pa -758 604}\special{fp}\special{pa -758 565}\special{pa -758 526}\special{fp}%
\special{pa -758 487}\special{pa -758 448}\special{fp}\special{pa -758 409}\special{pa -758 370}\special{fp}%
\special{pa -758 331}\special{pa -758 292}\special{fp}\special{pa -758 253}\special{pa -758 214}\special{fp}%
\special{pa -758 175}\special{pa -758 136}\special{fp}\special{pa -758 97}\special{pa -758 58}\special{fp}%
\special{pa -758 19}\special{pa -758 -19}\special{fp}\special{pa -758 -58}\special{pa -758 -97}\special{fp}%
\special{pa -758 -136}\special{pa -758 -175}\special{fp}\special{pa -758 -214}\special{pa -758 -253}\special{fp}%
\special{pa -758 -292}\special{pa -758 -331}\special{fp}\special{pa -758 -370}\special{pa -758 -409}\special{fp}%
\special{pa -758 -448}\special{pa -758 -487}\special{fp}\special{pa -758 -526}\special{pa -758 -565}\special{fp}%
\special{pa -758 -604}\special{pa -758 -643}\special{fp}%
%
\special{pa -505 643}\special{pa -505 604}\special{fp}\special{pa -505 565}\special{pa -505 526}\special{fp}%
\special{pa -505 487}\special{pa -505 448}\special{fp}\special{pa -505 409}\special{pa -505 370}\special{fp}%
\special{pa -505 331}\special{pa -505 292}\special{fp}\special{pa -505 253}\special{pa -505 214}\special{fp}%
\special{pa -505 175}\special{pa -505 136}\special{fp}\special{pa -505 97}\special{pa -505 58}\special{fp}%
\special{pa -505 19}\special{pa -505 -19}\special{fp}\special{pa -505 -58}\special{pa -505 -97}\special{fp}%
\special{pa -505 -136}\special{pa -505 -175}\special{fp}\special{pa -505 -214}\special{pa -505 -253}\special{fp}%
\special{pa -505 -292}\special{pa -505 -331}\special{fp}\special{pa -505 -370}\special{pa -505 -409}\special{fp}%
\special{pa -505 -448}\special{pa -505 -487}\special{fp}\special{pa -505 -526}\special{pa -505 -565}\special{fp}%
\special{pa -505 -604}\special{pa -505 -643}\special{fp}%
%
\special{pa -253 643}\special{pa -253 604}\special{fp}\special{pa -253 565}\special{pa -253 526}\special{fp}%
\special{pa -253 487}\special{pa -253 448}\special{fp}\special{pa -253 409}\special{pa -253 370}\special{fp}%
\special{pa -253 331}\special{pa -253 292}\special{fp}\special{pa -253 253}\special{pa -253 214}\special{fp}%
\special{pa -253 175}\special{pa -253 136}\special{fp}\special{pa -253 97}\special{pa -253 58}\special{fp}%
\special{pa -253 19}\special{pa -253 -19}\special{fp}\special{pa -253 -58}\special{pa -253 -97}\special{fp}%
\special{pa -253 -136}\special{pa -253 -175}\special{fp}\special{pa -253 -214}\special{pa -253 -253}\special{fp}%
\special{pa -253 -292}\special{pa -253 -331}\special{fp}\special{pa -253 -370}\special{pa -253 -409}\special{fp}%
\special{pa -253 -448}\special{pa -253 -487}\special{fp}\special{pa -253 -526}\special{pa -253 -565}\special{fp}%
\special{pa -253 -604}\special{pa -253 -643}\special{fp}%
%
\special{pa 0 643}\special{pa 0 604}\special{fp}\special{pa 0 565}\special{pa 0 526}\special{fp}%
\special{pa 0 487}\special{pa 0 448}\special{fp}\special{pa 0 409}\special{pa 0 370}\special{fp}%
\special{pa 0 331}\special{pa 0 292}\special{fp}\special{pa 0 253}\special{pa 0 214}\special{fp}%
\special{pa 0 175}\special{pa 0 136}\special{fp}\special{pa 0 97}\special{pa 0 58}\special{fp}%
\special{pa 0 19}\special{pa 0 -19}\special{fp}\special{pa 0 -58}\special{pa 0 -97}\special{fp}%
\special{pa 0 -136}\special{pa 0 -175}\special{fp}\special{pa 0 -214}\special{pa 0 -253}\special{fp}%
\special{pa 0 -292}\special{pa 0 -331}\special{fp}\special{pa 0 -370}\special{pa 0 -409}\special{fp}%
\special{pa 0 -448}\special{pa 0 -487}\special{fp}\special{pa 0 -526}\special{pa 0 -565}\special{fp}%
\special{pa 0 -604}\special{pa 0 -643}\special{fp}%
%
\special{pa 253 643}\special{pa 253 604}\special{fp}\special{pa 253 565}\special{pa 253 526}\special{fp}%
\special{pa 253 487}\special{pa 253 448}\special{fp}\special{pa 253 409}\special{pa 253 370}\special{fp}%
\special{pa 253 331}\special{pa 253 292}\special{fp}\special{pa 253 253}\special{pa 253 214}\special{fp}%
\special{pa 253 175}\special{pa 253 136}\special{fp}\special{pa 253 97}\special{pa 253 58}\special{fp}%
\special{pa 253 19}\special{pa 253 -19}\special{fp}\special{pa 253 -58}\special{pa 253 -97}\special{fp}%
\special{pa 253 -136}\special{pa 253 -175}\special{fp}\special{pa 253 -214}\special{pa 253 -253}\special{fp}%
\special{pa 253 -292}\special{pa 253 -331}\special{fp}\special{pa 253 -370}\special{pa 253 -409}\special{fp}%
\special{pa 253 -448}\special{pa 253 -487}\special{fp}\special{pa 253 -526}\special{pa 253 -565}\special{fp}%
\special{pa 253 -604}\special{pa 253 -643}\special{fp}%
%
\special{pa 505 643}\special{pa 505 604}\special{fp}\special{pa 505 565}\special{pa 505 526}\special{fp}%
\special{pa 505 487}\special{pa 505 448}\special{fp}\special{pa 505 409}\special{pa 505 370}\special{fp}%
\special{pa 505 331}\special{pa 505 292}\special{fp}\special{pa 505 253}\special{pa 505 214}\special{fp}%
\special{pa 505 175}\special{pa 505 136}\special{fp}\special{pa 505 97}\special{pa 505 58}\special{fp}%
\special{pa 505 19}\special{pa 505 -19}\special{fp}\special{pa 505 -58}\special{pa 505 -97}\special{fp}%
\special{pa 505 -136}\special{pa 505 -175}\special{fp}\special{pa 505 -214}\special{pa 505 -253}\special{fp}%
\special{pa 505 -292}\special{pa 505 -331}\special{fp}\special{pa 505 -370}\special{pa 505 -409}\special{fp}%
\special{pa 505 -448}\special{pa 505 -487}\special{fp}\special{pa 505 -526}\special{pa 505 -565}\special{fp}%
\special{pa 505 -604}\special{pa 505 -643}\special{fp}%
%
\special{pa 758 643}\special{pa 758 604}\special{fp}\special{pa 758 565}\special{pa 758 526}\special{fp}%
\special{pa 758 487}\special{pa 758 448}\special{fp}\special{pa 758 409}\special{pa 758 370}\special{fp}%
\special{pa 758 331}\special{pa 758 292}\special{fp}\special{pa 758 253}\special{pa 758 214}\special{fp}%
\special{pa 758 175}\special{pa 758 136}\special{fp}\special{pa 758 97}\special{pa 758 58}\special{fp}%
\special{pa 758 19}\special{pa 758 -19}\special{fp}\special{pa 758 -58}\special{pa 758 -97}\special{fp}%
\special{pa 758 -136}\special{pa 758 -175}\special{fp}\special{pa 758 -214}\special{pa 758 -253}\special{fp}%
\special{pa 758 -292}\special{pa 758 -331}\special{fp}\special{pa 758 -370}\special{pa 758 -409}\special{fp}%
\special{pa 758 -448}\special{pa 758 -487}\special{fp}\special{pa 758 -526}\special{pa 758 -565}\special{fp}%
\special{pa 758 -604}\special{pa 758 -643}\special{fp}%
%
\special{pa 1011 643}\special{pa 1011 604}\special{fp}\special{pa 1011 565}\special{pa 1011 526}\special{fp}%
\special{pa 1011 487}\special{pa 1011 448}\special{fp}\special{pa 1011 409}\special{pa 1011 370}\special{fp}%
\special{pa 1011 331}\special{pa 1011 292}\special{fp}\special{pa 1011 253}\special{pa 1011 214}\special{fp}%
\special{pa 1011 175}\special{pa 1011 136}\special{fp}\special{pa 1011 97}\special{pa 1011 58}\special{fp}%
\special{pa 1011 19}\special{pa 1011 -19}\special{fp}\special{pa 1011 -58}\special{pa 1011 -97}\special{fp}%
\special{pa 1011 -136}\special{pa 1011 -175}\special{fp}\special{pa 1011 -214}\special{pa 1011 -253}\special{fp}%
\special{pa 1011 -292}\special{pa 1011 -331}\special{fp}\special{pa 1011 -370}\special{pa 1011 -409}\special{fp}%
\special{pa 1011 -448}\special{pa 1011 -487}\special{fp}\special{pa 1011 -526}\special{pa 1011 -565}\special{fp}%
\special{pa 1011 -604}\special{pa 1011 -643}\special{fp}%
%
\special{pa 1263 643}\special{pa 1263 604}\special{fp}\special{pa 1263 565}\special{pa 1263 526}\special{fp}%
\special{pa 1263 487}\special{pa 1263 448}\special{fp}\special{pa 1263 409}\special{pa 1263 370}\special{fp}%
\special{pa 1263 331}\special{pa 1263 292}\special{fp}\special{pa 1263 253}\special{pa 1263 214}\special{fp}%
\special{pa 1263 175}\special{pa 1263 136}\special{fp}\special{pa 1263 97}\special{pa 1263 58}\special{fp}%
\special{pa 1263 19}\special{pa 1263 -19}\special{fp}\special{pa 1263 -58}\special{pa 1263 -97}\special{fp}%
\special{pa 1263 -136}\special{pa 1263 -175}\special{fp}\special{pa 1263 -214}\special{pa 1263 -253}\special{fp}%
\special{pa 1263 -292}\special{pa 1263 -331}\special{fp}\special{pa 1263 -370}\special{pa 1263 -409}\special{fp}%
\special{pa 1263 -448}\special{pa 1263 -487}\special{fp}\special{pa 1263 -526}\special{pa 1263 -565}\special{fp}%
\special{pa 1263 -604}\special{pa 1263 -643}\special{fp}%
%
\special{pa 1516 643}\special{pa 1516 604}\special{fp}\special{pa 1516 565}\special{pa 1516 526}\special{fp}%
\special{pa 1516 487}\special{pa 1516 448}\special{fp}\special{pa 1516 409}\special{pa 1516 370}\special{fp}%
\special{pa 1516 331}\special{pa 1516 292}\special{fp}\special{pa 1516 253}\special{pa 1516 214}\special{fp}%
\special{pa 1516 175}\special{pa 1516 136}\special{fp}\special{pa 1516 97}\special{pa 1516 58}\special{fp}%
\special{pa 1516 19}\special{pa 1516 -19}\special{fp}\special{pa 1516 -58}\special{pa 1516 -97}\special{fp}%
\special{pa 1516 -136}\special{pa 1516 -175}\special{fp}\special{pa 1516 -214}\special{pa 1516 -253}\special{fp}%
\special{pa 1516 -292}\special{pa 1516 -331}\special{fp}\special{pa 1516 -370}\special{pa 1516 -409}\special{fp}%
\special{pa 1516 -448}\special{pa 1516 -487}\special{fp}\special{pa 1516 -526}\special{pa 1516 -565}\special{fp}%
\special{pa 1516 -604}\special{pa 1516 -643}\special{fp}%
%
\special{pa 1769 643}\special{pa 1769 604}\special{fp}\special{pa 1769 565}\special{pa 1769 526}\special{fp}%
\special{pa 1769 487}\special{pa 1769 448}\special{fp}\special{pa 1769 409}\special{pa 1769 370}\special{fp}%
\special{pa 1769 331}\special{pa 1769 292}\special{fp}\special{pa 1769 253}\special{pa 1769 214}\special{fp}%
\special{pa 1769 175}\special{pa 1769 136}\special{fp}\special{pa 1769 97}\special{pa 1769 58}\special{fp}%
\special{pa 1769 19}\special{pa 1769 -19}\special{fp}\special{pa 1769 -58}\special{pa 1769 -97}\special{fp}%
\special{pa 1769 -136}\special{pa 1769 -175}\special{fp}\special{pa 1769 -214}\special{pa 1769 -253}\special{fp}%
\special{pa 1769 -292}\special{pa 1769 -331}\special{fp}\special{pa 1769 -370}\special{pa 1769 -409}\special{fp}%
\special{pa 1769 -448}\special{pa 1769 -487}\special{fp}\special{pa 1769 -526}\special{pa 1769 -565}\special{fp}%
\special{pa 1769 -604}\special{pa 1769 -643}\special{fp}%
%
\special{pa 2021 643}\special{pa 2021 604}\special{fp}\special{pa 2021 565}\special{pa 2021 526}\special{fp}%
\special{pa 2021 487}\special{pa 2021 448}\special{fp}\special{pa 2021 409}\special{pa 2021 370}\special{fp}%
\special{pa 2021 331}\special{pa 2021 292}\special{fp}\special{pa 2021 253}\special{pa 2021 214}\special{fp}%
\special{pa 2021 175}\special{pa 2021 136}\special{fp}\special{pa 2021 97}\special{pa 2021 58}\special{fp}%
\special{pa 2021 19}\special{pa 2021 -19}\special{fp}\special{pa 2021 -58}\special{pa 2021 -97}\special{fp}%
\special{pa 2021 -136}\special{pa 2021 -175}\special{fp}\special{pa 2021 -214}\special{pa 2021 -253}\special{fp}%
\special{pa 2021 -292}\special{pa 2021 -331}\special{fp}\special{pa 2021 -370}\special{pa 2021 -409}\special{fp}%
\special{pa 2021 -448}\special{pa 2021 -487}\special{fp}\special{pa 2021 -526}\special{pa 2021 -565}\special{fp}%
\special{pa 2021 -604}\special{pa 2021 -643}\special{fp}%
%
\special{pa -2021 643}\special{pa -1982 643}\special{fp}\special{pa -1943 643}\special{pa -1904 643}\special{fp}%
\special{pa -1864 643}\special{pa -1825 643}\special{fp}\special{pa -1786 643}\special{pa -1747 643}\special{fp}%
\special{pa -1707 643}\special{pa -1668 643}\special{fp}\special{pa -1629 643}\special{pa -1590 643}\special{fp}%
\special{pa -1550 643}\special{pa -1511 643}\special{fp}\special{pa -1472 643}\special{pa -1433 643}\special{fp}%
\special{pa -1393 643}\special{pa -1354 643}\special{fp}\special{pa -1315 643}\special{pa -1276 643}\special{fp}%
\special{pa -1236 643}\special{pa -1197 643}\special{fp}\special{pa -1158 643}\special{pa -1119 643}\special{fp}%
\special{pa -1079 643}\special{pa -1040 643}\special{fp}\special{pa -1001 643}\special{pa -962 643}\special{fp}%
\special{pa -922 643}\special{pa -883 643}\special{fp}\special{pa -844 643}\special{pa -805 643}\special{fp}%
\special{pa -765 643}\special{pa -726 643}\special{fp}\special{pa -687 643}\special{pa -648 643}\special{fp}%
\special{pa -608 643}\special{pa -569 643}\special{fp}\special{pa -530 643}\special{pa -491 643}\special{fp}%
\special{pa -451 643}\special{pa -412 643}\special{fp}\special{pa -373 643}\special{pa -334 643}\special{fp}%
\special{pa -294 643}\special{pa -255 643}\special{fp}\special{pa -216 643}\special{pa -177 643}\special{fp}%
\special{pa -137 643}\special{pa -98 643}\special{fp}\special{pa -59 643}\special{pa -20 643}\special{fp}%
\special{pa 20 643}\special{pa 59 643}\special{fp}\special{pa 98 643}\special{pa 137 643}\special{fp}%
\special{pa 177 643}\special{pa 216 643}\special{fp}\special{pa 255 643}\special{pa 294 643}\special{fp}%
\special{pa 334 643}\special{pa 373 643}\special{fp}\special{pa 412 643}\special{pa 451 643}\special{fp}%
\special{pa 491 643}\special{pa 530 643}\special{fp}\special{pa 569 643}\special{pa 608 643}\special{fp}%
\special{pa 648 643}\special{pa 687 643}\special{fp}\special{pa 726 643}\special{pa 765 643}\special{fp}%
\special{pa 805 643}\special{pa 844 643}\special{fp}\special{pa 883 643}\special{pa 922 643}\special{fp}%
\special{pa 962 643}\special{pa 1001 643}\special{fp}\special{pa 1040 643}\special{pa 1079 643}\special{fp}%
\special{pa 1119 643}\special{pa 1158 643}\special{fp}\special{pa 1197 643}\special{pa 1236 643}\special{fp}%
\special{pa 1276 643}\special{pa 1315 643}\special{fp}\special{pa 1354 643}\special{pa 1393 643}\special{fp}%
\special{pa 1433 643}\special{pa 1472 643}\special{fp}\special{pa 1511 643}\special{pa 1550 643}\special{fp}%
\special{pa 1590 643}\special{pa 1629 643}\special{fp}\special{pa 1668 643}\special{pa 1707 643}\special{fp}%
\special{pa 1747 643}\special{pa 1786 643}\special{fp}\special{pa 1825 643}\special{pa 1864 643}\special{fp}%
\special{pa 1904 643}\special{pa 1943 643}\special{fp}\special{pa 1982 643}\special{pa 2021 643}\special{fp}%
%
%
\special{pa -2021 483}\special{pa -1982 483}\special{fp}\special{pa -1943 483}\special{pa -1904 483}\special{fp}%
\special{pa -1864 483}\special{pa -1825 483}\special{fp}\special{pa -1786 483}\special{pa -1747 483}\special{fp}%
\special{pa -1707 483}\special{pa -1668 483}\special{fp}\special{pa -1629 483}\special{pa -1590 483}\special{fp}%
\special{pa -1550 483}\special{pa -1511 483}\special{fp}\special{pa -1472 483}\special{pa -1433 483}\special{fp}%
\special{pa -1393 483}\special{pa -1354 483}\special{fp}\special{pa -1315 483}\special{pa -1276 483}\special{fp}%
\special{pa -1236 483}\special{pa -1197 483}\special{fp}\special{pa -1158 483}\special{pa -1119 483}\special{fp}%
\special{pa -1079 483}\special{pa -1040 483}\special{fp}\special{pa -1001 483}\special{pa -962 483}\special{fp}%
\special{pa -922 483}\special{pa -883 483}\special{fp}\special{pa -844 483}\special{pa -805 483}\special{fp}%
\special{pa -765 483}\special{pa -726 483}\special{fp}\special{pa -687 483}\special{pa -648 483}\special{fp}%
\special{pa -608 483}\special{pa -569 483}\special{fp}\special{pa -530 483}\special{pa -491 483}\special{fp}%
\special{pa -451 483}\special{pa -412 483}\special{fp}\special{pa -373 483}\special{pa -334 483}\special{fp}%
\special{pa -294 483}\special{pa -255 483}\special{fp}\special{pa -216 483}\special{pa -177 483}\special{fp}%
\special{pa -137 483}\special{pa -98 483}\special{fp}\special{pa -59 483}\special{pa -20 483}\special{fp}%
\special{pa 20 483}\special{pa 59 483}\special{fp}\special{pa 98 483}\special{pa 137 483}\special{fp}%
\special{pa 177 483}\special{pa 216 483}\special{fp}\special{pa 255 483}\special{pa 294 483}\special{fp}%
\special{pa 334 483}\special{pa 373 483}\special{fp}\special{pa 412 483}\special{pa 451 483}\special{fp}%
\special{pa 491 483}\special{pa 530 483}\special{fp}\special{pa 569 483}\special{pa 608 483}\special{fp}%
\special{pa 648 483}\special{pa 687 483}\special{fp}\special{pa 726 483}\special{pa 765 483}\special{fp}%
\special{pa 805 483}\special{pa 844 483}\special{fp}\special{pa 883 483}\special{pa 922 483}\special{fp}%
\special{pa 962 483}\special{pa 1001 483}\special{fp}\special{pa 1040 483}\special{pa 1079 483}\special{fp}%
\special{pa 1119 483}\special{pa 1158 483}\special{fp}\special{pa 1197 483}\special{pa 1236 483}\special{fp}%
\special{pa 1276 483}\special{pa 1315 483}\special{fp}\special{pa 1354 483}\special{pa 1393 483}\special{fp}%
\special{pa 1433 483}\special{pa 1472 483}\special{fp}\special{pa 1511 483}\special{pa 1550 483}\special{fp}%
\special{pa 1590 483}\special{pa 1629 483}\special{fp}\special{pa 1668 483}\special{pa 1707 483}\special{fp}%
\special{pa 1747 483}\special{pa 1786 483}\special{fp}\special{pa 1825 483}\special{pa 1864 483}\special{fp}%
\special{pa 1904 483}\special{pa 1943 483}\special{fp}\special{pa 1982 483}\special{pa 2021 483}\special{fp}%
%
%
\special{pa -2021 322}\special{pa -1982 322}\special{fp}\special{pa -1943 322}\special{pa -1904 322}\special{fp}%
\special{pa -1864 322}\special{pa -1825 322}\special{fp}\special{pa -1786 322}\special{pa -1747 322}\special{fp}%
\special{pa -1707 322}\special{pa -1668 322}\special{fp}\special{pa -1629 322}\special{pa -1590 322}\special{fp}%
\special{pa -1550 322}\special{pa -1511 322}\special{fp}\special{pa -1472 322}\special{pa -1433 322}\special{fp}%
\special{pa -1393 322}\special{pa -1354 322}\special{fp}\special{pa -1315 322}\special{pa -1276 322}\special{fp}%
\special{pa -1236 322}\special{pa -1197 322}\special{fp}\special{pa -1158 322}\special{pa -1119 322}\special{fp}%
\special{pa -1079 322}\special{pa -1040 322}\special{fp}\special{pa -1001 322}\special{pa -962 322}\special{fp}%
\special{pa -922 322}\special{pa -883 322}\special{fp}\special{pa -844 322}\special{pa -805 322}\special{fp}%
\special{pa -765 322}\special{pa -726 322}\special{fp}\special{pa -687 322}\special{pa -648 322}\special{fp}%
\special{pa -608 322}\special{pa -569 322}\special{fp}\special{pa -530 322}\special{pa -491 322}\special{fp}%
\special{pa -451 322}\special{pa -412 322}\special{fp}\special{pa -373 322}\special{pa -334 322}\special{fp}%
\special{pa -294 322}\special{pa -255 322}\special{fp}\special{pa -216 322}\special{pa -177 322}\special{fp}%
\special{pa -137 322}\special{pa -98 322}\special{fp}\special{pa -59 322}\special{pa -20 322}\special{fp}%
\special{pa 20 322}\special{pa 59 322}\special{fp}\special{pa 98 322}\special{pa 137 322}\special{fp}%
\special{pa 177 322}\special{pa 216 322}\special{fp}\special{pa 255 322}\special{pa 294 322}\special{fp}%
\special{pa 334 322}\special{pa 373 322}\special{fp}\special{pa 412 322}\special{pa 451 322}\special{fp}%
\special{pa 491 322}\special{pa 530 322}\special{fp}\special{pa 569 322}\special{pa 608 322}\special{fp}%
\special{pa 648 322}\special{pa 687 322}\special{fp}\special{pa 726 322}\special{pa 765 322}\special{fp}%
\special{pa 805 322}\special{pa 844 322}\special{fp}\special{pa 883 322}\special{pa 922 322}\special{fp}%
\special{pa 962 322}\special{pa 1001 322}\special{fp}\special{pa 1040 322}\special{pa 1079 322}\special{fp}%
\special{pa 1119 322}\special{pa 1158 322}\special{fp}\special{pa 1197 322}\special{pa 1236 322}\special{fp}%
\special{pa 1276 322}\special{pa 1315 322}\special{fp}\special{pa 1354 322}\special{pa 1393 322}\special{fp}%
\special{pa 1433 322}\special{pa 1472 322}\special{fp}\special{pa 1511 322}\special{pa 1550 322}\special{fp}%
\special{pa 1590 322}\special{pa 1629 322}\special{fp}\special{pa 1668 322}\special{pa 1707 322}\special{fp}%
\special{pa 1747 322}\special{pa 1786 322}\special{fp}\special{pa 1825 322}\special{pa 1864 322}\special{fp}%
\special{pa 1904 322}\special{pa 1943 322}\special{fp}\special{pa 1982 322}\special{pa 2021 322}\special{fp}%
%
%
\special{pa -2021 161}\special{pa -1982 161}\special{fp}\special{pa -1943 161}\special{pa -1904 161}\special{fp}%
\special{pa -1864 161}\special{pa -1825 161}\special{fp}\special{pa -1786 161}\special{pa -1747 161}\special{fp}%
\special{pa -1707 161}\special{pa -1668 161}\special{fp}\special{pa -1629 161}\special{pa -1590 161}\special{fp}%
\special{pa -1550 161}\special{pa -1511 161}\special{fp}\special{pa -1472 161}\special{pa -1433 161}\special{fp}%
\special{pa -1393 161}\special{pa -1354 161}\special{fp}\special{pa -1315 161}\special{pa -1276 161}\special{fp}%
\special{pa -1236 161}\special{pa -1197 161}\special{fp}\special{pa -1158 161}\special{pa -1119 161}\special{fp}%
\special{pa -1079 161}\special{pa -1040 161}\special{fp}\special{pa -1001 161}\special{pa -962 161}\special{fp}%
\special{pa -922 161}\special{pa -883 161}\special{fp}\special{pa -844 161}\special{pa -805 161}\special{fp}%
\special{pa -765 161}\special{pa -726 161}\special{fp}\special{pa -687 161}\special{pa -648 161}\special{fp}%
\special{pa -608 161}\special{pa -569 161}\special{fp}\special{pa -530 161}\special{pa -491 161}\special{fp}%
\special{pa -451 161}\special{pa -412 161}\special{fp}\special{pa -373 161}\special{pa -334 161}\special{fp}%
\special{pa -294 161}\special{pa -255 161}\special{fp}\special{pa -216 161}\special{pa -177 161}\special{fp}%
\special{pa -137 161}\special{pa -98 161}\special{fp}\special{pa -59 161}\special{pa -20 161}\special{fp}%
\special{pa 20 161}\special{pa 59 161}\special{fp}\special{pa 98 161}\special{pa 137 161}\special{fp}%
\special{pa 177 161}\special{pa 216 161}\special{fp}\special{pa 255 161}\special{pa 294 161}\special{fp}%
\special{pa 334 161}\special{pa 373 161}\special{fp}\special{pa 412 161}\special{pa 451 161}\special{fp}%
\special{pa 491 161}\special{pa 530 161}\special{fp}\special{pa 569 161}\special{pa 608 161}\special{fp}%
\special{pa 648 161}\special{pa 687 161}\special{fp}\special{pa 726 161}\special{pa 765 161}\special{fp}%
\special{pa 805 161}\special{pa 844 161}\special{fp}\special{pa 883 161}\special{pa 922 161}\special{fp}%
\special{pa 962 161}\special{pa 1001 161}\special{fp}\special{pa 1040 161}\special{pa 1079 161}\special{fp}%
\special{pa 1119 161}\special{pa 1158 161}\special{fp}\special{pa 1197 161}\special{pa 1236 161}\special{fp}%
\special{pa 1276 161}\special{pa 1315 161}\special{fp}\special{pa 1354 161}\special{pa 1393 161}\special{fp}%
\special{pa 1433 161}\special{pa 1472 161}\special{fp}\special{pa 1511 161}\special{pa 1550 161}\special{fp}%
\special{pa 1590 161}\special{pa 1629 161}\special{fp}\special{pa 1668 161}\special{pa 1707 161}\special{fp}%
\special{pa 1747 161}\special{pa 1786 161}\special{fp}\special{pa 1825 161}\special{pa 1864 161}\special{fp}%
\special{pa 1904 161}\special{pa 1943 161}\special{fp}\special{pa 1982 161}\special{pa 2021 161}\special{fp}%
%
%
\special{pa -2021 0}\special{pa -1982 0}\special{fp}\special{pa -1943 0}\special{pa -1904 0}\special{fp}%
\special{pa -1864 0}\special{pa -1825 0}\special{fp}\special{pa -1786 0}\special{pa -1747 0}\special{fp}%
\special{pa -1707 0}\special{pa -1668 0}\special{fp}\special{pa -1629 0}\special{pa -1590 0}\special{fp}%
\special{pa -1550 0}\special{pa -1511 0}\special{fp}\special{pa -1472 0}\special{pa -1433 0}\special{fp}%
\special{pa -1393 0}\special{pa -1354 0}\special{fp}\special{pa -1315 0}\special{pa -1276 0}\special{fp}%
\special{pa -1236 0}\special{pa -1197 0}\special{fp}\special{pa -1158 0}\special{pa -1119 0}\special{fp}%
\special{pa -1079 0}\special{pa -1040 0}\special{fp}\special{pa -1001 0}\special{pa -962 0}\special{fp}%
\special{pa -922 0}\special{pa -883 0}\special{fp}\special{pa -844 0}\special{pa -805 0}\special{fp}%
\special{pa -765 0}\special{pa -726 0}\special{fp}\special{pa -687 0}\special{pa -648 0}\special{fp}%
\special{pa -608 0}\special{pa -569 0}\special{fp}\special{pa -530 0}\special{pa -491 0}\special{fp}%
\special{pa -451 0}\special{pa -412 0}\special{fp}\special{pa -373 0}\special{pa -334 0}\special{fp}%
\special{pa -294 0}\special{pa -255 0}\special{fp}\special{pa -216 0}\special{pa -177 0}\special{fp}%
\special{pa -137 0}\special{pa -98 0}\special{fp}\special{pa -59 0}\special{pa -20 0}\special{fp}%
\special{pa 20 0}\special{pa 59 0}\special{fp}\special{pa 98 0}\special{pa 137 0}\special{fp}%
\special{pa 177 0}\special{pa 216 0}\special{fp}\special{pa 255 0}\special{pa 294 0}\special{fp}%
\special{pa 334 0}\special{pa 373 0}\special{fp}\special{pa 412 0}\special{pa 451 0}\special{fp}%
\special{pa 491 0}\special{pa 530 0}\special{fp}\special{pa 569 0}\special{pa 608 0}\special{fp}%
\special{pa 648 0}\special{pa 687 0}\special{fp}\special{pa 726 0}\special{pa 765 0}\special{fp}%
\special{pa 805 0}\special{pa 844 0}\special{fp}\special{pa 883 0}\special{pa 922 0}\special{fp}%
\special{pa 962 0}\special{pa 1001 0}\special{fp}\special{pa 1040 0}\special{pa 1079 0}\special{fp}%
\special{pa 1119 0}\special{pa 1158 0}\special{fp}\special{pa 1197 0}\special{pa 1236 0}\special{fp}%
\special{pa 1276 0}\special{pa 1315 0}\special{fp}\special{pa 1354 0}\special{pa 1393 0}\special{fp}%
\special{pa 1433 0}\special{pa 1472 0}\special{fp}\special{pa 1511 0}\special{pa 1550 0}\special{fp}%
\special{pa 1590 0}\special{pa 1629 0}\special{fp}\special{pa 1668 0}\special{pa 1707 0}\special{fp}%
\special{pa 1747 0}\special{pa 1786 0}\special{fp}\special{pa 1825 0}\special{pa 1864 0}\special{fp}%
\special{pa 1904 0}\special{pa 1943 0}\special{fp}\special{pa 1982 0}\special{pa 2021 0}\special{fp}%
%
%
\special{pa -2021 -161}\special{pa -1982 -161}\special{fp}\special{pa -1943 -161}\special{pa -1904 -161}\special{fp}%
\special{pa -1864 -161}\special{pa -1825 -161}\special{fp}\special{pa -1786 -161}\special{pa -1747 -161}\special{fp}%
\special{pa -1707 -161}\special{pa -1668 -161}\special{fp}\special{pa -1629 -161}\special{pa -1590 -161}\special{fp}%
\special{pa -1550 -161}\special{pa -1511 -161}\special{fp}\special{pa -1472 -161}\special{pa -1433 -161}\special{fp}%
\special{pa -1393 -161}\special{pa -1354 -161}\special{fp}\special{pa -1315 -161}\special{pa -1276 -161}\special{fp}%
\special{pa -1236 -161}\special{pa -1197 -161}\special{fp}\special{pa -1158 -161}\special{pa -1119 -161}\special{fp}%
\special{pa -1079 -161}\special{pa -1040 -161}\special{fp}\special{pa -1001 -161}\special{pa -962 -161}\special{fp}%
\special{pa -922 -161}\special{pa -883 -161}\special{fp}\special{pa -844 -161}\special{pa -805 -161}\special{fp}%
\special{pa -765 -161}\special{pa -726 -161}\special{fp}\special{pa -687 -161}\special{pa -648 -161}\special{fp}%
\special{pa -608 -161}\special{pa -569 -161}\special{fp}\special{pa -530 -161}\special{pa -491 -161}\special{fp}%
\special{pa -451 -161}\special{pa -412 -161}\special{fp}\special{pa -373 -161}\special{pa -334 -161}\special{fp}%
\special{pa -294 -161}\special{pa -255 -161}\special{fp}\special{pa -216 -161}\special{pa -177 -161}\special{fp}%
\special{pa -137 -161}\special{pa -98 -161}\special{fp}\special{pa -59 -161}\special{pa -20 -161}\special{fp}%
\special{pa 20 -161}\special{pa 59 -161}\special{fp}\special{pa 98 -161}\special{pa 137 -161}\special{fp}%
\special{pa 177 -161}\special{pa 216 -161}\special{fp}\special{pa 255 -161}\special{pa 294 -161}\special{fp}%
\special{pa 334 -161}\special{pa 373 -161}\special{fp}\special{pa 412 -161}\special{pa 451 -161}\special{fp}%
\special{pa 491 -161}\special{pa 530 -161}\special{fp}\special{pa 569 -161}\special{pa 608 -161}\special{fp}%
\special{pa 648 -161}\special{pa 687 -161}\special{fp}\special{pa 726 -161}\special{pa 765 -161}\special{fp}%
\special{pa 805 -161}\special{pa 844 -161}\special{fp}\special{pa 883 -161}\special{pa 922 -161}\special{fp}%
\special{pa 962 -161}\special{pa 1001 -161}\special{fp}\special{pa 1040 -161}\special{pa 1079 -161}\special{fp}%
\special{pa 1119 -161}\special{pa 1158 -161}\special{fp}\special{pa 1197 -161}\special{pa 1236 -161}\special{fp}%
\special{pa 1276 -161}\special{pa 1315 -161}\special{fp}\special{pa 1354 -161}\special{pa 1393 -161}\special{fp}%
\special{pa 1433 -161}\special{pa 1472 -161}\special{fp}\special{pa 1511 -161}\special{pa 1550 -161}\special{fp}%
\special{pa 1590 -161}\special{pa 1629 -161}\special{fp}\special{pa 1668 -161}\special{pa 1707 -161}\special{fp}%
\special{pa 1747 -161}\special{pa 1786 -161}\special{fp}\special{pa 1825 -161}\special{pa 1864 -161}\special{fp}%
\special{pa 1904 -161}\special{pa 1943 -161}\special{fp}\special{pa 1982 -161}\special{pa 2021 -161}\special{fp}%
%
%
\special{pa -2021 -322}\special{pa -1982 -322}\special{fp}\special{pa -1943 -322}\special{pa -1904 -322}\special{fp}%
\special{pa -1864 -322}\special{pa -1825 -322}\special{fp}\special{pa -1786 -322}\special{pa -1747 -322}\special{fp}%
\special{pa -1707 -322}\special{pa -1668 -322}\special{fp}\special{pa -1629 -322}\special{pa -1590 -322}\special{fp}%
\special{pa -1550 -322}\special{pa -1511 -322}\special{fp}\special{pa -1472 -322}\special{pa -1433 -322}\special{fp}%
\special{pa -1393 -322}\special{pa -1354 -322}\special{fp}\special{pa -1315 -322}\special{pa -1276 -322}\special{fp}%
\special{pa -1236 -322}\special{pa -1197 -322}\special{fp}\special{pa -1158 -322}\special{pa -1119 -322}\special{fp}%
\special{pa -1079 -322}\special{pa -1040 -322}\special{fp}\special{pa -1001 -322}\special{pa -962 -322}\special{fp}%
\special{pa -922 -322}\special{pa -883 -322}\special{fp}\special{pa -844 -322}\special{pa -805 -322}\special{fp}%
\special{pa -765 -322}\special{pa -726 -322}\special{fp}\special{pa -687 -322}\special{pa -648 -322}\special{fp}%
\special{pa -608 -322}\special{pa -569 -322}\special{fp}\special{pa -530 -322}\special{pa -491 -322}\special{fp}%
\special{pa -451 -322}\special{pa -412 -322}\special{fp}\special{pa -373 -322}\special{pa -334 -322}\special{fp}%
\special{pa -294 -322}\special{pa -255 -322}\special{fp}\special{pa -216 -322}\special{pa -177 -322}\special{fp}%
\special{pa -137 -322}\special{pa -98 -322}\special{fp}\special{pa -59 -322}\special{pa -20 -322}\special{fp}%
\special{pa 20 -322}\special{pa 59 -322}\special{fp}\special{pa 98 -322}\special{pa 137 -322}\special{fp}%
\special{pa 177 -322}\special{pa 216 -322}\special{fp}\special{pa 255 -322}\special{pa 294 -322}\special{fp}%
\special{pa 334 -322}\special{pa 373 -322}\special{fp}\special{pa 412 -322}\special{pa 451 -322}\special{fp}%
\special{pa 491 -322}\special{pa 530 -322}\special{fp}\special{pa 569 -322}\special{pa 608 -322}\special{fp}%
\special{pa 648 -322}\special{pa 687 -322}\special{fp}\special{pa 726 -322}\special{pa 765 -322}\special{fp}%
\special{pa 805 -322}\special{pa 844 -322}\special{fp}\special{pa 883 -322}\special{pa 922 -322}\special{fp}%
\special{pa 962 -322}\special{pa 1001 -322}\special{fp}\special{pa 1040 -322}\special{pa 1079 -322}\special{fp}%
\special{pa 1119 -322}\special{pa 1158 -322}\special{fp}\special{pa 1197 -322}\special{pa 1236 -322}\special{fp}%
\special{pa 1276 -322}\special{pa 1315 -322}\special{fp}\special{pa 1354 -322}\special{pa 1393 -322}\special{fp}%
\special{pa 1433 -322}\special{pa 1472 -322}\special{fp}\special{pa 1511 -322}\special{pa 1550 -322}\special{fp}%
\special{pa 1590 -322}\special{pa 1629 -322}\special{fp}\special{pa 1668 -322}\special{pa 1707 -322}\special{fp}%
\special{pa 1747 -322}\special{pa 1786 -322}\special{fp}\special{pa 1825 -322}\special{pa 1864 -322}\special{fp}%
\special{pa 1904 -322}\special{pa 1943 -322}\special{fp}\special{pa 1982 -322}\special{pa 2021 -322}\special{fp}%
%
%
\special{pa -2021 -483}\special{pa -1982 -483}\special{fp}\special{pa -1943 -483}\special{pa -1904 -483}\special{fp}%
\special{pa -1864 -483}\special{pa -1825 -483}\special{fp}\special{pa -1786 -483}\special{pa -1747 -483}\special{fp}%
\special{pa -1707 -483}\special{pa -1668 -483}\special{fp}\special{pa -1629 -483}\special{pa -1590 -483}\special{fp}%
\special{pa -1550 -483}\special{pa -1511 -483}\special{fp}\special{pa -1472 -483}\special{pa -1433 -483}\special{fp}%
\special{pa -1393 -483}\special{pa -1354 -483}\special{fp}\special{pa -1315 -483}\special{pa -1276 -483}\special{fp}%
\special{pa -1236 -483}\special{pa -1197 -483}\special{fp}\special{pa -1158 -483}\special{pa -1119 -483}\special{fp}%
\special{pa -1079 -483}\special{pa -1040 -483}\special{fp}\special{pa -1001 -483}\special{pa -962 -483}\special{fp}%
\special{pa -922 -483}\special{pa -883 -483}\special{fp}\special{pa -844 -483}\special{pa -805 -483}\special{fp}%
\special{pa -765 -483}\special{pa -726 -483}\special{fp}\special{pa -687 -483}\special{pa -648 -483}\special{fp}%
\special{pa -608 -483}\special{pa -569 -483}\special{fp}\special{pa -530 -483}\special{pa -491 -483}\special{fp}%
\special{pa -451 -483}\special{pa -412 -483}\special{fp}\special{pa -373 -483}\special{pa -334 -483}\special{fp}%
\special{pa -294 -483}\special{pa -255 -483}\special{fp}\special{pa -216 -483}\special{pa -177 -483}\special{fp}%
\special{pa -137 -483}\special{pa -98 -483}\special{fp}\special{pa -59 -483}\special{pa -20 -483}\special{fp}%
\special{pa 20 -483}\special{pa 59 -483}\special{fp}\special{pa 98 -483}\special{pa 137 -483}\special{fp}%
\special{pa 177 -483}\special{pa 216 -483}\special{fp}\special{pa 255 -483}\special{pa 294 -483}\special{fp}%
\special{pa 334 -483}\special{pa 373 -483}\special{fp}\special{pa 412 -483}\special{pa 451 -483}\special{fp}%
\special{pa 491 -483}\special{pa 530 -483}\special{fp}\special{pa 569 -483}\special{pa 608 -483}\special{fp}%
\special{pa 648 -483}\special{pa 687 -483}\special{fp}\special{pa 726 -483}\special{pa 765 -483}\special{fp}%
\special{pa 805 -483}\special{pa 844 -483}\special{fp}\special{pa 883 -483}\special{pa 922 -483}\special{fp}%
\special{pa 962 -483}\special{pa 1001 -483}\special{fp}\special{pa 1040 -483}\special{pa 1079 -483}\special{fp}%
\special{pa 1119 -483}\special{pa 1158 -483}\special{fp}\special{pa 1197 -483}\special{pa 1236 -483}\special{fp}%
\special{pa 1276 -483}\special{pa 1315 -483}\special{fp}\special{pa 1354 -483}\special{pa 1393 -483}\special{fp}%
\special{pa 1433 -483}\special{pa 1472 -483}\special{fp}\special{pa 1511 -483}\special{pa 1550 -483}\special{fp}%
\special{pa 1590 -483}\special{pa 1629 -483}\special{fp}\special{pa 1668 -483}\special{pa 1707 -483}\special{fp}%
\special{pa 1747 -483}\special{pa 1786 -483}\special{fp}\special{pa 1825 -483}\special{pa 1864 -483}\special{fp}%
\special{pa 1904 -483}\special{pa 1943 -483}\special{fp}\special{pa 1982 -483}\special{pa 2021 -483}\special{fp}%
%
%
\special{pa -2021 -643}\special{pa -1982 -643}\special{fp}\special{pa -1943 -643}\special{pa -1904 -643}\special{fp}%
\special{pa -1864 -643}\special{pa -1825 -643}\special{fp}\special{pa -1786 -643}\special{pa -1747 -643}\special{fp}%
\special{pa -1707 -643}\special{pa -1668 -643}\special{fp}\special{pa -1629 -643}\special{pa -1590 -643}\special{fp}%
\special{pa -1550 -643}\special{pa -1511 -643}\special{fp}\special{pa -1472 -643}\special{pa -1433 -643}\special{fp}%
\special{pa -1393 -643}\special{pa -1354 -643}\special{fp}\special{pa -1315 -643}\special{pa -1276 -643}\special{fp}%
\special{pa -1236 -643}\special{pa -1197 -643}\special{fp}\special{pa -1158 -643}\special{pa -1119 -643}\special{fp}%
\special{pa -1079 -643}\special{pa -1040 -643}\special{fp}\special{pa -1001 -643}\special{pa -962 -643}\special{fp}%
\special{pa -922 -643}\special{pa -883 -643}\special{fp}\special{pa -844 -643}\special{pa -805 -643}\special{fp}%
\special{pa -765 -643}\special{pa -726 -643}\special{fp}\special{pa -687 -643}\special{pa -648 -643}\special{fp}%
\special{pa -608 -643}\special{pa -569 -643}\special{fp}\special{pa -530 -643}\special{pa -491 -643}\special{fp}%
\special{pa -451 -643}\special{pa -412 -643}\special{fp}\special{pa -373 -643}\special{pa -334 -643}\special{fp}%
\special{pa -294 -643}\special{pa -255 -643}\special{fp}\special{pa -216 -643}\special{pa -177 -643}\special{fp}%
\special{pa -137 -643}\special{pa -98 -643}\special{fp}\special{pa -59 -643}\special{pa -20 -643}\special{fp}%
\special{pa 20 -643}\special{pa 59 -643}\special{fp}\special{pa 98 -643}\special{pa 137 -643}\special{fp}%
\special{pa 177 -643}\special{pa 216 -643}\special{fp}\special{pa 255 -643}\special{pa 294 -643}\special{fp}%
\special{pa 334 -643}\special{pa 373 -643}\special{fp}\special{pa 412 -643}\special{pa 451 -643}\special{fp}%
\special{pa 491 -643}\special{pa 530 -643}\special{fp}\special{pa 569 -643}\special{pa 608 -643}\special{fp}%
\special{pa 648 -643}\special{pa 687 -643}\special{fp}\special{pa 726 -643}\special{pa 765 -643}\special{fp}%
\special{pa 805 -643}\special{pa 844 -643}\special{fp}\special{pa 883 -643}\special{pa 922 -643}\special{fp}%
\special{pa 962 -643}\special{pa 1001 -643}\special{fp}\special{pa 1040 -643}\special{pa 1079 -643}\special{fp}%
\special{pa 1119 -643}\special{pa 1158 -643}\special{fp}\special{pa 1197 -643}\special{pa 1236 -643}\special{fp}%
\special{pa 1276 -643}\special{pa 1315 -643}\special{fp}\special{pa 1354 -643}\special{pa 1393 -643}\special{fp}%
\special{pa 1433 -643}\special{pa 1472 -643}\special{fp}\special{pa 1511 -643}\special{pa 1550 -643}\special{fp}%
\special{pa 1590 -643}\special{pa 1629 -643}\special{fp}\special{pa 1668 -643}\special{pa 1707 -643}\special{fp}%
\special{pa 1747 -643}\special{pa 1786 -643}\special{fp}\special{pa 1825 -643}\special{pa 1864 -643}\special{fp}%
\special{pa 1904 -643}\special{pa 1943 -643}\special{fp}\special{pa 1982 -643}\special{pa 2021 -643}\special{fp}%
%
%
\special{pn 8}%
\special{pa -2021   -20}\special{pa -2021    20}%
\special{fp}%
\settowidth{\Width}{$-2\pi$}\setlength{\Width}{-0.5\Width}%
\settoheight{\Height}{$-2\pi$}\settodepth{\Depth}{$-2\pi$}\setlength{\Height}{-\Height}%
\put(-6.2831850,-0.1223810){\hspace*{\Width}\raisebox{\Height}{$-2\pi$}}%
%
%
\special{pa -1011   -20}\special{pa -1011    20}%
\special{fp}%
\settowidth{\Width}{$-\pi$}\setlength{\Width}{-0.5\Width}%
\settoheight{\Height}{$-\pi$}\settodepth{\Depth}{$-\pi$}\setlength{\Height}{-\Height}%
\put(-3.1415930,-0.1223810){\hspace*{\Width}\raisebox{\Height}{$-\pi$}}%
%
%
\special{pa  1011   -20}\special{pa  1011    20}%
\special{fp}%
\settowidth{\Width}{$\pi$}\setlength{\Width}{-0.5\Width}%
\settoheight{\Height}{$\pi$}\settodepth{\Depth}{$\pi$}\setlength{\Height}{-\Height}%
\put(3.1415930,-0.1223810){\hspace*{\Width}\raisebox{\Height}{$\pi$}}%
%
%
\special{pa  2021   -20}\special{pa  2021    20}%
\special{fp}%
\settowidth{\Width}{$2\pi$}\setlength{\Width}{-0.5\Width}%
\settoheight{\Height}{$2\pi$}\settodepth{\Depth}{$2\pi$}\setlength{\Height}{-\Height}%
\put(6.2831850,-0.1223810){\hspace*{\Width}\raisebox{\Height}{$2\pi$}}%
%
%
\special{pa    20   643}\special{pa   -20   643}%
\special{fp}%
\settowidth{\Width}{$-2$}\setlength{\Width}{-1\Width}%
\settoheight{\Height}{$-2$}\settodepth{\Depth}{$-2$}\setlength{\Height}{-0.5\Height}\setlength{\Depth}{0.5\Depth}\addtolength{\Height}{\Depth}%
\put(-0.1223810,-2.0000000){\hspace*{\Width}\raisebox{\Height}{$-2$}}%
%
%
\special{pa    20   322}\special{pa   -20   322}%
\special{fp}%
\settowidth{\Width}{$-1$}\setlength{\Width}{-1\Width}%
\settoheight{\Height}{$-1$}\settodepth{\Depth}{$-1$}\setlength{\Height}{-0.5\Height}\setlength{\Depth}{0.5\Depth}\addtolength{\Height}{\Depth}%
\put(-0.1223810,-1.0000000){\hspace*{\Width}\raisebox{\Height}{$-1$}}%
%
%
\special{pa    20  -322}\special{pa   -20  -322}%
\special{fp}%
\settowidth{\Width}{$1$}\setlength{\Width}{-1\Width}%
\settoheight{\Height}{$1$}\settodepth{\Depth}{$1$}\setlength{\Height}{-0.5\Height}\setlength{\Depth}{0.5\Depth}\addtolength{\Height}{\Depth}%
\put(-0.1223810,1.0000000){\hspace*{\Width}\raisebox{\Height}{$1$}}%
%
%
\special{pa    20  -643}\special{pa   -20  -643}%
\special{fp}%
\settowidth{\Width}{$2$}\setlength{\Width}{-1\Width}%
\settoheight{\Height}{$2$}\settodepth{\Depth}{$2$}\setlength{\Height}{-0.5\Height}\setlength{\Depth}{0.5\Depth}\addtolength{\Height}{\Depth}%
\put(-0.1223810,2.0000000){\hspace*{\Width}\raisebox{\Height}{$2$}}%
%
%
\special{pa -2104    -0}\special{pa  2104    -0}%
\special{fp}%
\special{pa     0   676}\special{pa     0  -676}%
\special{fp}%
\settowidth{\Width}{$x$}\setlength{\Width}{0\Width}%
\settoheight{\Height}{$x$}\settodepth{\Depth}{$x$}\setlength{\Height}{-0.5\Height}\setlength{\Depth}{0.5\Depth}\addtolength{\Height}{\Depth}%
\put(6.6011905,0.0000000){\hspace*{\Width}\raisebox{\Height}{$x$}}%
%
\settowidth{\Width}{$y$}\setlength{\Width}{-0.5\Width}%
\settoheight{\Height}{$y$}\settodepth{\Depth}{$y$}\setlength{\Height}{\Depth}%
\put(0.0000000,2.1611905){\hspace*{\Width}\raisebox{\Height}{$y$}}%
%
\settowidth{\Width}{O}\setlength{\Width}{0\Width}%
\settoheight{\Height}{O}\settodepth{\Depth}{O}\setlength{\Height}{-\Height}%
\put(0.0611905,-0.0611905){\hspace*{\Width}\raisebox{\Height}{O}}%
%
\end{picture}}%
\end{center}

%\vspace{2mm}

]

%\small

\begin{Enumerate}[{\bf \large 1}]
%各2 36
\item 上の図は$y=\cos x$と$y=\cos\dfrac{1}{2}x$のグラフである.
\begin{enumerate}[(1)]
\item $y=\cos x$の振幅は\hako{10}{5}{},周期は\hako{10}{5}{}である.
\item $y=\cos\dfrac{1}{2}x$の振幅は\hako{10}{5}{},周期は\hako{10}{5}{}である.
\item $y=2\cos x$の振幅は\hako{10}{5}{},周期は\hako{10}{5}{}である.\item $y=2\cos x$のグラフをかき入れよ.
\item $\dfrac{\pi}{5}\;(\mathrm{rad})=\hako{10}{6}{\,}\;\degree$\ ,$120\degree=\hako{10}{6}{}\;(\mathrm{rad})$
\end{enumerate}

\item $a>0,\ a\neqq1$とする.

\begin{layer}{100}{0}
\putnotese{43}{0}{%%% /Users/takatoosetsuo/Dropbox/2018polytec/lecture/0618/fig/explog.tex 
%%% Generator=fig0618.cdy 
{\unitlength=4mm%
\begin{picture}%
(10,10)(-5,-5)%
\special{pn 8}%
%
\small%
\scriptsize%
\special{pn 12}%
\special{pa  -787    -5}\special{pa  -772    -5}\special{pa  -756    -6}\special{pa  -740    -6}%
\special{pa  -724    -6}\special{pa  -709    -7}\special{pa  -693    -7}\special{pa  -677    -8}%
\special{pa  -661    -9}\special{pa  -646    -9}\special{pa  -630   -10}\special{pa  -614   -11}%
\special{pa  -598   -11}\special{pa  -583   -12}\special{pa  -567   -13}\special{pa  -551   -14}%
\special{pa  -535   -15}\special{pa  -520   -16}\special{pa  -504   -17}\special{pa  -488   -18}%
\special{pa  -472   -20}\special{pa  -457   -21}\special{pa  -441   -23}\special{pa  -425   -24}%
\special{pa  -409   -26}\special{pa  -394   -28}\special{pa  -378   -30}\special{pa  -362   -32}%
\special{pa  -346   -34}\special{pa  -331   -37}\special{pa  -315   -39}\special{pa  -299   -42}%
\special{pa  -283   -45}\special{pa  -268   -48}\special{pa  -252   -52}\special{pa  -236   -56}%
\special{pa  -220   -60}\special{pa  -205   -64}\special{pa  -189   -69}\special{pa  -173   -73}%
\special{pa  -157   -79}\special{pa  -142   -84}\special{pa  -126   -90}\special{pa  -110   -97}%
\special{pa   -94  -104}\special{pa   -79  -111}\special{pa   -63  -119}\special{pa   -47  -128}%
\special{pa   -31  -137}\special{pa   -16  -147}\special{pa     0  -157}\special{pa    16  -169}%
\special{pa    31  -181}\special{pa    47  -194}\special{pa    63  -208}\special{pa    79  -223}%
\special{pa    94  -239}\special{pa   110  -256}\special{pa   126  -274}\special{pa   142  -294}%
\special{pa   157  -315}\special{pa   173  -338}\special{pa   189  -362}\special{pa   205  -388}%
\special{pa   220  -416}\special{pa   236  -445}\special{pa   252  -477}\special{pa   268  -512}%
\special{pa   283  -548}\special{pa   299  -588}\special{pa   315  -630}\special{pa   331  -675}%
\special{pa   346  -724}\special{pa   362  -776}\special{pa   366  -787}%
\special{fp}%
\special{pn 8}%
\special{pn 12}%
\special{pa  -787    -1}\special{pa  -772    -1}\special{pa  -756    -1}\special{pa  -740    -1}%
\special{pa  -724    -1}\special{pa  -709    -1}\special{pa  -693    -1}\special{pa  -677    -1}%
\special{pa  -661    -2}\special{pa  -646    -2}\special{pa  -630    -2}\special{pa  -614    -2}%
\special{pa  -598    -2}\special{pa  -583    -3}\special{pa  -567    -3}\special{pa  -551    -3}%
\special{pa  -535    -4}\special{pa  -520    -4}\special{pa  -504    -5}\special{pa  -488    -5}%
\special{pa  -472    -6}\special{pa  -457    -7}\special{pa  -441    -7}\special{pa  -425    -8}%
\special{pa  -409    -9}\special{pa  -394   -10}\special{pa  -378   -11}\special{pa  -362   -13}%
\special{pa  -346   -14}\special{pa  -331   -16}\special{pa  -315   -17}\special{pa  -299   -20}%
\special{pa  -283   -22}\special{pa  -268   -24}\special{pa  -252   -27}\special{pa  -236   -30}%
\special{pa  -220   -34}\special{pa  -205   -38}\special{pa  -189   -42}\special{pa  -173   -47}%
\special{pa  -157   -52}\special{pa  -142   -59}\special{pa  -126   -65}\special{pa  -110   -73}%
\special{pa   -94   -81}\special{pa   -79   -91}\special{pa   -63  -101}\special{pa   -47  -113}%
\special{pa   -31  -126}\special{pa   -16  -141}\special{pa     0  -157}\special{pa    16  -176}%
\special{pa    31  -196}\special{pa    47  -219}\special{pa    63  -244}\special{pa    79  -273}%
\special{pa    94  -304}\special{pa   110  -340}\special{pa   126  -379}\special{pa   142  -423}%
\special{pa   157  -472}\special{pa   173  -527}\special{pa   189  -589}\special{pa   205  -657}%
\special{pa   220  -733}\special{pa   231  -787}%
\special{fp}%
\special{pn 8}%
\special{pn 12}%
\special{pa     7   787}\special{pa     9   640}\special{pa    17   502}\special{pa    25   417}%
\special{pa    33   355}\special{pa    41   306}\special{pa    49   267}\special{pa    57   233}%
\special{pa    64   203}\special{pa    72   177}\special{pa    80   153}\special{pa    88   132}%
\special{pa    96   113}\special{pa   104    95}\special{pa   112    78}\special{pa   119    63}%
\special{pa   127    48}\special{pa   135    35}\special{pa   143    22}\special{pa   151    10}%
\special{pa   159    -2}\special{pa   167   -13}\special{pa   174   -23}\special{pa   182   -33}%
\special{pa   190   -43}\special{pa   198   -52}\special{pa   206   -61}\special{pa   214   -69}%
\special{pa   222   -78}\special{pa   229   -86}\special{pa   237   -93}\special{pa   245  -101}%
\special{pa   253  -108}\special{pa   261  -115}\special{pa   269  -121}\special{pa   277  -128}%
\special{pa   284  -134}\special{pa   292  -141}\special{pa   300  -147}\special{pa   308  -152}%
\special{pa   316  -158}\special{pa   324  -164}\special{pa   332  -169}\special{pa   339  -175}%
\special{pa   347  -180}\special{pa   355  -185}\special{pa   363  -190}\special{pa   371  -195}%
\special{pa   379  -199}\special{pa   387  -204}\special{pa   394  -209}\special{pa   402  -213}%
\special{pa   410  -218}\special{pa   418  -222}\special{pa   426  -226}\special{pa   434  -230}%
\special{pa   442  -234}\special{pa   449  -238}\special{pa   457  -242}\special{pa   465  -246}%
\special{pa   473  -250}\special{pa   481  -254}\special{pa   489  -257}\special{pa   497  -261}%
\special{pa   505  -265}\special{pa   512  -268}\special{pa   520  -271}\special{pa   528  -275}%
\special{pa   536  -278}\special{pa   544  -282}\special{pa   552  -285}\special{pa   560  -288}%
\special{pa   567  -291}\special{pa   575  -294}\special{pa   583  -297}\special{pa   591  -300}%
\special{pa   599  -303}\special{pa   607  -306}\special{pa   615  -309}\special{pa   622  -312}%
\special{pa   630  -315}\special{pa   638  -318}\special{pa   646  -321}\special{pa   654  -323}%
\special{pa   662  -326}\special{pa   670  -329}\special{pa   677  -331}\special{pa   685  -334}%
\special{pa   693  -337}\special{pa   701  -339}\special{pa   709  -342}\special{pa   717  -344}%
\special{pa   725  -347}\special{pa   732  -349}\special{pa   740  -352}\special{pa   748  -354}%
\special{pa   756  -356}\special{pa   764  -359}\special{pa   772  -361}\special{pa   780  -363}%
\special{pa   787  -366}%
\special{fp}%
\special{pn 8}%
\special{pn 12}%
\special{pa    32   787}\special{pa    33   751}\special{pa    41   607}\special{pa    49   509}%
\special{pa    57   438}\special{pa    64   385}\special{pa    72   343}\special{pa    80   309}%
\special{pa    88   282}\special{pa    96   259}\special{pa   104   239}\special{pa   112   222}%
\special{pa   119   208}\special{pa   127   195}\special{pa   135   183}\special{pa   143   173}%
\special{pa   151   164}\special{pa   159   156}\special{pa   167   149}\special{pa   174   142}%
\special{pa   182   136}\special{pa   190   130}\special{pa   198   125}\special{pa   206   120}%
\special{pa   214   116}\special{pa   222   112}\special{pa   229   108}\special{pa   237   104}%
\special{pa   245   101}\special{pa   253    98}\special{pa   261    95}\special{pa   269    92}%
\special{pa   277    90}\special{pa   284    87}\special{pa   292    85}\special{pa   300    83}%
\special{pa   308    81}\special{pa   316    79}\special{pa   324    77}\special{pa   332    75}%
\special{pa   339    73}\special{pa   347    71}\special{pa   355    70}\special{pa   363    68}%
\special{pa   371    67}\special{pa   379    65}\special{pa   387    64}\special{pa   394    63}%
\special{pa   402    62}\special{pa   410    60}\special{pa   418    59}\special{pa   426    58}%
\special{pa   434    57}\special{pa   442    56}\special{pa   449    55}\special{pa   457    54}%
\special{pa   465    53}\special{pa   473    52}\special{pa   481    52}\special{pa   489    51}%
\special{pa   497    50}\special{pa   505    49}\special{pa   512    48}\special{pa   520    48}%
\special{pa   528    47}\special{pa   536    46}\special{pa   544    46}\special{pa   552    45}%
\special{pa   560    44}\special{pa   567    44}\special{pa   575    43}\special{pa   583    43}%
\special{pa   591    42}\special{pa   599    41}\special{pa   607    41}\special{pa   615    40}%
\special{pa   622    40}\special{pa   630    39}\special{pa   638    39}\special{pa   646    38}%
\special{pa   654    38}\special{pa   662    37}\special{pa   670    37}\special{pa   677    37}%
\special{pa   685    36}\special{pa   693    36}\special{pa   701    35}\special{pa   709    35}%
\special{pa   717    35}\special{pa   725    34}\special{pa   732    34}\special{pa   740    34}%
\special{pa   748    33}\special{pa   756    33}\special{pa   764    32}\special{pa   772    32}%
\special{pa   780    32}\special{pa   787    31}%
\special{fp}%
\special{pn 8}%
\special{pn 4}%
\special{pa -787 157}\special{pa -787 120}\special{fp}\special{pa -787 82}\special{pa -787 44}\special{fp}%
\special{pa -787 6}\special{pa -787 -31}\special{fp}\special{pa -787 -69}\special{pa -787 -107}\special{fp}%
\special{pa -787 -145}\special{pa -787 -183}\special{fp}\special{pa -787 -220}\special{pa -787 -258}\special{fp}%
\special{pa -787 -296}\special{pa -787 -334}\special{fp}\special{pa -787 -372}\special{pa -787 -409}\special{fp}%
\special{pa -787 -447}\special{pa -787 -485}\special{fp}\special{pa -787 -523}\special{pa -787 -561}\special{fp}%
\special{pa -787 -598}\special{pa -787 -636}\special{fp}\special{pa -787 -674}\special{pa -787 -712}\special{fp}%
\special{pa -787 -750}\special{pa -787 -787}\special{fp}%
%
\special{pa -630 157}\special{pa -630 120}\special{fp}\special{pa -630 82}\special{pa -630 44}\special{fp}%
\special{pa -630 6}\special{pa -630 -31}\special{fp}\special{pa -630 -69}\special{pa -630 -107}\special{fp}%
\special{pa -630 -145}\special{pa -630 -183}\special{fp}\special{pa -630 -220}\special{pa -630 -258}\special{fp}%
\special{pa -630 -296}\special{pa -630 -334}\special{fp}\special{pa -630 -372}\special{pa -630 -409}\special{fp}%
\special{pa -630 -447}\special{pa -630 -485}\special{fp}\special{pa -630 -523}\special{pa -630 -561}\special{fp}%
\special{pa -630 -598}\special{pa -630 -636}\special{fp}\special{pa -630 -674}\special{pa -630 -712}\special{fp}%
\special{pa -630 -750}\special{pa -630 -787}\special{fp}%
%
\special{pa -472 157}\special{pa -472 120}\special{fp}\special{pa -472 82}\special{pa -472 44}\special{fp}%
\special{pa -472 6}\special{pa -472 -31}\special{fp}\special{pa -472 -69}\special{pa -472 -107}\special{fp}%
\special{pa -472 -145}\special{pa -472 -183}\special{fp}\special{pa -472 -220}\special{pa -472 -258}\special{fp}%
\special{pa -472 -296}\special{pa -472 -334}\special{fp}\special{pa -472 -372}\special{pa -472 -409}\special{fp}%
\special{pa -472 -447}\special{pa -472 -485}\special{fp}\special{pa -472 -523}\special{pa -472 -561}\special{fp}%
\special{pa -472 -598}\special{pa -472 -636}\special{fp}\special{pa -472 -674}\special{pa -472 -712}\special{fp}%
\special{pa -472 -750}\special{pa -472 -787}\special{fp}%
%
\special{pa -315 157}\special{pa -315 120}\special{fp}\special{pa -315 82}\special{pa -315 44}\special{fp}%
\special{pa -315 6}\special{pa -315 -31}\special{fp}\special{pa -315 -69}\special{pa -315 -107}\special{fp}%
\special{pa -315 -145}\special{pa -315 -183}\special{fp}\special{pa -315 -220}\special{pa -315 -258}\special{fp}%
\special{pa -315 -296}\special{pa -315 -334}\special{fp}\special{pa -315 -372}\special{pa -315 -409}\special{fp}%
\special{pa -315 -447}\special{pa -315 -485}\special{fp}\special{pa -315 -523}\special{pa -315 -561}\special{fp}%
\special{pa -315 -598}\special{pa -315 -636}\special{fp}\special{pa -315 -674}\special{pa -315 -712}\special{fp}%
\special{pa -315 -750}\special{pa -315 -787}\special{fp}%
%
\special{pa -157 157}\special{pa -157 120}\special{fp}\special{pa -157 82}\special{pa -157 44}\special{fp}%
\special{pa -157 6}\special{pa -157 -31}\special{fp}\special{pa -157 -69}\special{pa -157 -107}\special{fp}%
\special{pa -157 -145}\special{pa -157 -183}\special{fp}\special{pa -157 -220}\special{pa -157 -258}\special{fp}%
\special{pa -157 -296}\special{pa -157 -334}\special{fp}\special{pa -157 -372}\special{pa -157 -409}\special{fp}%
\special{pa -157 -447}\special{pa -157 -485}\special{fp}\special{pa -157 -523}\special{pa -157 -561}\special{fp}%
\special{pa -157 -598}\special{pa -157 -636}\special{fp}\special{pa -157 -674}\special{pa -157 -712}\special{fp}%
\special{pa -157 -750}\special{pa -157 -787}\special{fp}%
%
\special{pa 157 787}\special{pa 157 749}\special{fp}\special{pa 157 711}\special{pa 157 672}\special{fp}%
\special{pa 157 634}\special{pa 157 595}\special{fp}\special{pa 157 557}\special{pa 157 519}\special{fp}%
\special{pa 157 480}\special{pa 157 442}\special{fp}\special{pa 157 403}\special{pa 157 365}\special{fp}%
\special{pa 157 326}\special{pa 157 288}\special{fp}\special{pa 157 250}\special{pa 157 211}\special{fp}%
\special{pa 157 173}\special{pa 157 134}\special{fp}\special{pa 157 96}\special{pa 157 58}\special{fp}%
\special{pa 157 19}\special{pa 157 -19}\special{fp}\special{pa 157 -58}\special{pa 157 -96}\special{fp}%
\special{pa 157 -134}\special{pa 157 -173}\special{fp}\special{pa 157 -211}\special{pa 157 -250}\special{fp}%
\special{pa 157 -288}\special{pa 157 -326}\special{fp}\special{pa 157 -365}\special{pa 157 -403}\special{fp}%
\special{pa 157 -442}\special{pa 157 -480}\special{fp}\special{pa 157 -519}\special{pa 157 -557}\special{fp}%
\special{pa 157 -595}\special{pa 157 -634}\special{fp}\special{pa 157 -672}\special{pa 157 -711}\special{fp}%
\special{pa 157 -749}\special{pa 157 -787}\special{fp}%
%
\special{pa 315 787}\special{pa 315 749}\special{fp}\special{pa 315 711}\special{pa 315 672}\special{fp}%
\special{pa 315 634}\special{pa 315 595}\special{fp}\special{pa 315 557}\special{pa 315 519}\special{fp}%
\special{pa 315 480}\special{pa 315 442}\special{fp}\special{pa 315 403}\special{pa 315 365}\special{fp}%
\special{pa 315 326}\special{pa 315 288}\special{fp}\special{pa 315 250}\special{pa 315 211}\special{fp}%
\special{pa 315 173}\special{pa 315 134}\special{fp}\special{pa 315 96}\special{pa 315 58}\special{fp}%
\special{pa 315 19}\special{pa 315 -19}\special{fp}\special{pa 315 -58}\special{pa 315 -96}\special{fp}%
\special{pa 315 -134}\special{pa 315 -173}\special{fp}\special{pa 315 -211}\special{pa 315 -250}\special{fp}%
\special{pa 315 -288}\special{pa 315 -326}\special{fp}\special{pa 315 -365}\special{pa 315 -403}\special{fp}%
\special{pa 315 -442}\special{pa 315 -480}\special{fp}\special{pa 315 -519}\special{pa 315 -557}\special{fp}%
\special{pa 315 -595}\special{pa 315 -634}\special{fp}\special{pa 315 -672}\special{pa 315 -711}\special{fp}%
\special{pa 315 -749}\special{pa 315 -787}\special{fp}%
%
\special{pa 472 787}\special{pa 472 749}\special{fp}\special{pa 472 711}\special{pa 472 672}\special{fp}%
\special{pa 472 634}\special{pa 472 595}\special{fp}\special{pa 472 557}\special{pa 472 519}\special{fp}%
\special{pa 472 480}\special{pa 472 442}\special{fp}\special{pa 472 403}\special{pa 472 365}\special{fp}%
\special{pa 472 326}\special{pa 472 288}\special{fp}\special{pa 472 250}\special{pa 472 211}\special{fp}%
\special{pa 472 173}\special{pa 472 134}\special{fp}\special{pa 472 96}\special{pa 472 58}\special{fp}%
\special{pa 472 19}\special{pa 472 -19}\special{fp}\special{pa 472 -58}\special{pa 472 -96}\special{fp}%
\special{pa 472 -134}\special{pa 472 -173}\special{fp}\special{pa 472 -211}\special{pa 472 -250}\special{fp}%
\special{pa 472 -288}\special{pa 472 -326}\special{fp}\special{pa 472 -365}\special{pa 472 -403}\special{fp}%
\special{pa 472 -442}\special{pa 472 -480}\special{fp}\special{pa 472 -519}\special{pa 472 -557}\special{fp}%
\special{pa 472 -595}\special{pa 472 -634}\special{fp}\special{pa 472 -672}\special{pa 472 -711}\special{fp}%
\special{pa 472 -749}\special{pa 472 -787}\special{fp}%
%
\special{pa 630 787}\special{pa 630 749}\special{fp}\special{pa 630 711}\special{pa 630 672}\special{fp}%
\special{pa 630 634}\special{pa 630 595}\special{fp}\special{pa 630 557}\special{pa 630 519}\special{fp}%
\special{pa 630 480}\special{pa 630 442}\special{fp}\special{pa 630 403}\special{pa 630 365}\special{fp}%
\special{pa 630 326}\special{pa 630 288}\special{fp}\special{pa 630 250}\special{pa 630 211}\special{fp}%
\special{pa 630 173}\special{pa 630 134}\special{fp}\special{pa 630 96}\special{pa 630 58}\special{fp}%
\special{pa 630 19}\special{pa 630 -19}\special{fp}\special{pa 630 -58}\special{pa 630 -96}\special{fp}%
\special{pa 630 -134}\special{pa 630 -173}\special{fp}\special{pa 630 -211}\special{pa 630 -250}\special{fp}%
\special{pa 630 -288}\special{pa 630 -326}\special{fp}\special{pa 630 -365}\special{pa 630 -403}\special{fp}%
\special{pa 630 -442}\special{pa 630 -480}\special{fp}\special{pa 630 -519}\special{pa 630 -557}\special{fp}%
\special{pa 630 -595}\special{pa 630 -634}\special{fp}\special{pa 630 -672}\special{pa 630 -711}\special{fp}%
\special{pa 630 -749}\special{pa 630 -787}\special{fp}%
%
\special{pa 787 787}\special{pa 787 749}\special{fp}\special{pa 787 711}\special{pa 787 672}\special{fp}%
\special{pa 787 634}\special{pa 787 595}\special{fp}\special{pa 787 557}\special{pa 787 519}\special{fp}%
\special{pa 787 480}\special{pa 787 442}\special{fp}\special{pa 787 403}\special{pa 787 365}\special{fp}%
\special{pa 787 326}\special{pa 787 288}\special{fp}\special{pa 787 250}\special{pa 787 211}\special{fp}%
\special{pa 787 173}\special{pa 787 134}\special{fp}\special{pa 787 96}\special{pa 787 58}\special{fp}%
\special{pa 787 19}\special{pa 787 -19}\special{fp}\special{pa 787 -58}\special{pa 787 -96}\special{fp}%
\special{pa 787 -134}\special{pa 787 -173}\special{fp}\special{pa 787 -211}\special{pa 787 -250}\special{fp}%
\special{pa 787 -288}\special{pa 787 -326}\special{fp}\special{pa 787 -365}\special{pa 787 -403}\special{fp}%
\special{pa 787 -442}\special{pa 787 -480}\special{fp}\special{pa 787 -519}\special{pa 787 -557}\special{fp}%
\special{pa 787 -595}\special{pa 787 -634}\special{fp}\special{pa 787 -672}\special{pa 787 -711}\special{fp}%
\special{pa 787 -749}\special{pa 787 -787}\special{fp}%
%
\special{pa -157 787}\special{pa -120 787}\special{fp}\special{pa -82 787}\special{pa -44 787}\special{fp}%
\special{pa -6 787}\special{pa 31 787}\special{fp}\special{pa 69 787}\special{pa 107 787}\special{fp}%
\special{pa 145 787}\special{pa 183 787}\special{fp}\special{pa 220 787}\special{pa 258 787}\special{fp}%
\special{pa 296 787}\special{pa 334 787}\special{fp}\special{pa 372 787}\special{pa 409 787}\special{fp}%
\special{pa 447 787}\special{pa 485 787}\special{fp}\special{pa 523 787}\special{pa 561 787}\special{fp}%
\special{pa 598 787}\special{pa 636 787}\special{fp}\special{pa 674 787}\special{pa 712 787}\special{fp}%
\special{pa 750 787}\special{pa 787 787}\special{fp}%
%
\special{pa -157 630}\special{pa -120 630}\special{fp}\special{pa -82 630}\special{pa -44 630}\special{fp}%
\special{pa -6 630}\special{pa 31 630}\special{fp}\special{pa 69 630}\special{pa 107 630}\special{fp}%
\special{pa 145 630}\special{pa 183 630}\special{fp}\special{pa 220 630}\special{pa 258 630}\special{fp}%
\special{pa 296 630}\special{pa 334 630}\special{fp}\special{pa 372 630}\special{pa 409 630}\special{fp}%
\special{pa 447 630}\special{pa 485 630}\special{fp}\special{pa 523 630}\special{pa 561 630}\special{fp}%
\special{pa 598 630}\special{pa 636 630}\special{fp}\special{pa 674 630}\special{pa 712 630}\special{fp}%
\special{pa 750 630}\special{pa 787 630}\special{fp}%
%
\special{pa -157 472}\special{pa -120 472}\special{fp}\special{pa -82 472}\special{pa -44 472}\special{fp}%
\special{pa -6 472}\special{pa 31 472}\special{fp}\special{pa 69 472}\special{pa 107 472}\special{fp}%
\special{pa 145 472}\special{pa 183 472}\special{fp}\special{pa 220 472}\special{pa 258 472}\special{fp}%
\special{pa 296 472}\special{pa 334 472}\special{fp}\special{pa 372 472}\special{pa 409 472}\special{fp}%
\special{pa 447 472}\special{pa 485 472}\special{fp}\special{pa 523 472}\special{pa 561 472}\special{fp}%
\special{pa 598 472}\special{pa 636 472}\special{fp}\special{pa 674 472}\special{pa 712 472}\special{fp}%
\special{pa 750 472}\special{pa 787 472}\special{fp}%
%
\special{pa -157 315}\special{pa -120 315}\special{fp}\special{pa -82 315}\special{pa -44 315}\special{fp}%
\special{pa -6 315}\special{pa 31 315}\special{fp}\special{pa 69 315}\special{pa 107 315}\special{fp}%
\special{pa 145 315}\special{pa 183 315}\special{fp}\special{pa 220 315}\special{pa 258 315}\special{fp}%
\special{pa 296 315}\special{pa 334 315}\special{fp}\special{pa 372 315}\special{pa 409 315}\special{fp}%
\special{pa 447 315}\special{pa 485 315}\special{fp}\special{pa 523 315}\special{pa 561 315}\special{fp}%
\special{pa 598 315}\special{pa 636 315}\special{fp}\special{pa 674 315}\special{pa 712 315}\special{fp}%
\special{pa 750 315}\special{pa 787 315}\special{fp}%
%
\special{pa -157 157}\special{pa -120 157}\special{fp}\special{pa -82 157}\special{pa -44 157}\special{fp}%
\special{pa -6 157}\special{pa 31 157}\special{fp}\special{pa 69 157}\special{pa 107 157}\special{fp}%
\special{pa 145 157}\special{pa 183 157}\special{fp}\special{pa 220 157}\special{pa 258 157}\special{fp}%
\special{pa 296 157}\special{pa 334 157}\special{fp}\special{pa 372 157}\special{pa 409 157}\special{fp}%
\special{pa 447 157}\special{pa 485 157}\special{fp}\special{pa 523 157}\special{pa 561 157}\special{fp}%
\special{pa 598 157}\special{pa 636 157}\special{fp}\special{pa 674 157}\special{pa 712 157}\special{fp}%
\special{pa 750 157}\special{pa 787 157}\special{fp}%
%
\special{pa -787 -157}\special{pa -749 -157}\special{fp}\special{pa -711 -157}\special{pa -672 -157}\special{fp}%
\special{pa -634 -157}\special{pa -595 -157}\special{fp}\special{pa -557 -157}\special{pa -519 -157}\special{fp}%
\special{pa -480 -157}\special{pa -442 -157}\special{fp}\special{pa -403 -157}\special{pa -365 -157}\special{fp}%
\special{pa -326 -157}\special{pa -288 -157}\special{fp}\special{pa -250 -157}\special{pa -211 -157}\special{fp}%
\special{pa -173 -157}\special{pa -134 -157}\special{fp}\special{pa -96 -157}\special{pa -58 -157}\special{fp}%
\special{pa -19 -157}\special{pa 19 -157}\special{fp}\special{pa 58 -157}\special{pa 96 -157}\special{fp}%
\special{pa 134 -157}\special{pa 173 -157}\special{fp}\special{pa 211 -157}\special{pa 250 -157}\special{fp}%
\special{pa 288 -157}\special{pa 326 -157}\special{fp}\special{pa 365 -157}\special{pa 403 -157}\special{fp}%
\special{pa 442 -157}\special{pa 480 -157}\special{fp}\special{pa 519 -157}\special{pa 557 -157}\special{fp}%
\special{pa 595 -157}\special{pa 634 -157}\special{fp}\special{pa 672 -157}\special{pa 711 -157}\special{fp}%
\special{pa 749 -157}\special{pa 787 -157}\special{fp}%
%
\special{pa -787 -315}\special{pa -749 -315}\special{fp}\special{pa -711 -315}\special{pa -672 -315}\special{fp}%
\special{pa -634 -315}\special{pa -595 -315}\special{fp}\special{pa -557 -315}\special{pa -519 -315}\special{fp}%
\special{pa -480 -315}\special{pa -442 -315}\special{fp}\special{pa -403 -315}\special{pa -365 -315}\special{fp}%
\special{pa -326 -315}\special{pa -288 -315}\special{fp}\special{pa -250 -315}\special{pa -211 -315}\special{fp}%
\special{pa -173 -315}\special{pa -134 -315}\special{fp}\special{pa -96 -315}\special{pa -58 -315}\special{fp}%
\special{pa -19 -315}\special{pa 19 -315}\special{fp}\special{pa 58 -315}\special{pa 96 -315}\special{fp}%
\special{pa 134 -315}\special{pa 173 -315}\special{fp}\special{pa 211 -315}\special{pa 250 -315}\special{fp}%
\special{pa 288 -315}\special{pa 326 -315}\special{fp}\special{pa 365 -315}\special{pa 403 -315}\special{fp}%
\special{pa 442 -315}\special{pa 480 -315}\special{fp}\special{pa 519 -315}\special{pa 557 -315}\special{fp}%
\special{pa 595 -315}\special{pa 634 -315}\special{fp}\special{pa 672 -315}\special{pa 711 -315}\special{fp}%
\special{pa 749 -315}\special{pa 787 -315}\special{fp}%
%
\special{pa -787 -472}\special{pa -749 -472}\special{fp}\special{pa -711 -472}\special{pa -672 -472}\special{fp}%
\special{pa -634 -472}\special{pa -595 -472}\special{fp}\special{pa -557 -472}\special{pa -519 -472}\special{fp}%
\special{pa -480 -472}\special{pa -442 -472}\special{fp}\special{pa -403 -472}\special{pa -365 -472}\special{fp}%
\special{pa -326 -472}\special{pa -288 -472}\special{fp}\special{pa -250 -472}\special{pa -211 -472}\special{fp}%
\special{pa -173 -472}\special{pa -134 -472}\special{fp}\special{pa -96 -472}\special{pa -58 -472}\special{fp}%
\special{pa -19 -472}\special{pa 19 -472}\special{fp}\special{pa 58 -472}\special{pa 96 -472}\special{fp}%
\special{pa 134 -472}\special{pa 173 -472}\special{fp}\special{pa 211 -472}\special{pa 250 -472}\special{fp}%
\special{pa 288 -472}\special{pa 326 -472}\special{fp}\special{pa 365 -472}\special{pa 403 -472}\special{fp}%
\special{pa 442 -472}\special{pa 480 -472}\special{fp}\special{pa 519 -472}\special{pa 557 -472}\special{fp}%
\special{pa 595 -472}\special{pa 634 -472}\special{fp}\special{pa 672 -472}\special{pa 711 -472}\special{fp}%
\special{pa 749 -472}\special{pa 787 -472}\special{fp}%
%
\special{pa -787 -630}\special{pa -749 -630}\special{fp}\special{pa -711 -630}\special{pa -672 -630}\special{fp}%
\special{pa -634 -630}\special{pa -595 -630}\special{fp}\special{pa -557 -630}\special{pa -519 -630}\special{fp}%
\special{pa -480 -630}\special{pa -442 -630}\special{fp}\special{pa -403 -630}\special{pa -365 -630}\special{fp}%
\special{pa -326 -630}\special{pa -288 -630}\special{fp}\special{pa -250 -630}\special{pa -211 -630}\special{fp}%
\special{pa -173 -630}\special{pa -134 -630}\special{fp}\special{pa -96 -630}\special{pa -58 -630}\special{fp}%
\special{pa -19 -630}\special{pa 19 -630}\special{fp}\special{pa 58 -630}\special{pa 96 -630}\special{fp}%
\special{pa 134 -630}\special{pa 173 -630}\special{fp}\special{pa 211 -630}\special{pa 250 -630}\special{fp}%
\special{pa 288 -630}\special{pa 326 -630}\special{fp}\special{pa 365 -630}\special{pa 403 -630}\special{fp}%
\special{pa 442 -630}\special{pa 480 -630}\special{fp}\special{pa 519 -630}\special{pa 557 -630}\special{fp}%
\special{pa 595 -630}\special{pa 634 -630}\special{fp}\special{pa 672 -630}\special{pa 711 -630}\special{fp}%
\special{pa 749 -630}\special{pa 787 -630}\special{fp}%
%
\special{pa -787 -787}\special{pa -749 -787}\special{fp}\special{pa -711 -787}\special{pa -672 -787}\special{fp}%
\special{pa -634 -787}\special{pa -595 -787}\special{fp}\special{pa -557 -787}\special{pa -519 -787}\special{fp}%
\special{pa -480 -787}\special{pa -442 -787}\special{fp}\special{pa -403 -787}\special{pa -365 -787}\special{fp}%
\special{pa -326 -787}\special{pa -288 -787}\special{fp}\special{pa -250 -787}\special{pa -211 -787}\special{fp}%
\special{pa -173 -787}\special{pa -134 -787}\special{fp}\special{pa -96 -787}\special{pa -58 -787}\special{fp}%
\special{pa -19 -787}\special{pa 19 -787}\special{fp}\special{pa 58 -787}\special{pa 96 -787}\special{fp}%
\special{pa 134 -787}\special{pa 173 -787}\special{fp}\special{pa 211 -787}\special{pa 250 -787}\special{fp}%
\special{pa 288 -787}\special{pa 326 -787}\special{fp}\special{pa 365 -787}\special{pa 403 -787}\special{fp}%
\special{pa 442 -787}\special{pa 480 -787}\special{fp}\special{pa 519 -787}\special{pa 557 -787}\special{fp}%
\special{pa 595 -787}\special{pa 634 -787}\special{fp}\special{pa 672 -787}\special{pa 711 -787}\special{fp}%
\special{pa 749 -787}\special{pa 787 -787}\special{fp}%
%
\settowidth{\Width}{(1)}\setlength{\Width}{-0.5\Width}%
\settoheight{\Height}{(1)}\settodepth{\Depth}{(1)}\setlength{\Height}{\Depth}%
\put(1.5000000,5.2000000){\hspace*{\Width}\raisebox{\Height}{(1)}}%
%
\settowidth{\Width}{(2)}\setlength{\Width}{-0.5\Width}%
\settoheight{\Height}{(2)}\settodepth{\Depth}{(2)}\setlength{\Height}{\Depth}%
\put(2.3300000,5.2600000){\hspace*{\Width}\raisebox{\Height}{(2)}}%
%
\settowidth{\Width}{(3)}\setlength{\Width}{-0.5\Width}%
\settoheight{\Height}{(3)}\settodepth{\Depth}{(3)}\setlength{\Height}{\Depth}%
\put(4.9900000,2.7000000){\hspace*{\Width}\raisebox{\Height}{(3)}}%
%
\settowidth{\Width}{(4)}\setlength{\Width}{-0.5\Width}%
\settoheight{\Height}{(4)}\settodepth{\Depth}{(4)}\setlength{\Height}{\Depth}%
\put(5.0000000,0.2500000){\hspace*{\Width}\raisebox{\Height}{(4)}}%
%
\special{pn 8}%
\special{pa  -315   -20}\special{pa  -315    20}%
\special{fp}%
\settowidth{\Width}{$-2$}\setlength{\Width}{-0.5\Width}%
\settoheight{\Height}{$-2$}\settodepth{\Depth}{$-2$}\setlength{\Height}{-\Height}%
\put(-2.0000000,-0.2500000){\hspace*{\Width}\raisebox{\Height}{$-2$}}%
%
%
\special{pa  -157   -20}\special{pa  -157    20}%
\special{fp}%
\settowidth{\Width}{$-1$}\setlength{\Width}{-0.5\Width}%
\settoheight{\Height}{$-1$}\settodepth{\Depth}{$-1$}\setlength{\Height}{-\Height}%
\put(-1.0000000,-0.2500000){\hspace*{\Width}\raisebox{\Height}{$-1$}}%
%
%
\special{pa   157   -20}\special{pa   157    20}%
\special{fp}%
\settowidth{\Width}{$1$}\setlength{\Width}{-0.5\Width}%
\settoheight{\Height}{$1$}\settodepth{\Depth}{$1$}\setlength{\Height}{-\Height}%
\put(1.0000000,-0.2500000){\hspace*{\Width}\raisebox{\Height}{$1$}}%
%
%
\special{pa   315   -20}\special{pa   315    20}%
\special{fp}%
\settowidth{\Width}{$2$}\setlength{\Width}{-0.5\Width}%
\settoheight{\Height}{$2$}\settodepth{\Depth}{$2$}\setlength{\Height}{-\Height}%
\put(2.0000000,-0.2500000){\hspace*{\Width}\raisebox{\Height}{$2$}}%
%
%
\special{pa    20   315}\special{pa   -20   315}%
\special{fp}%
\settowidth{\Width}{$-2$}\setlength{\Width}{-1\Width}%
\settoheight{\Height}{$-2$}\settodepth{\Depth}{$-2$}\setlength{\Height}{-0.5\Height}\setlength{\Depth}{0.5\Depth}\addtolength{\Height}{\Depth}%
\put(-0.2500000,-2.0000000){\hspace*{\Width}\raisebox{\Height}{$-2$}}%
%
%
\special{pa    20   157}\special{pa   -20   157}%
\special{fp}%
\settowidth{\Width}{$-1$}\setlength{\Width}{-1\Width}%
\settoheight{\Height}{$-1$}\settodepth{\Depth}{$-1$}\setlength{\Height}{-0.5\Height}\setlength{\Depth}{0.5\Depth}\addtolength{\Height}{\Depth}%
\put(-0.2500000,-1.0000000){\hspace*{\Width}\raisebox{\Height}{$-1$}}%
%
%
\special{pa    20  -157}\special{pa   -20  -157}%
\special{fp}%
\settowidth{\Width}{$1$}\setlength{\Width}{-1\Width}%
\settoheight{\Height}{$1$}\settodepth{\Depth}{$1$}\setlength{\Height}{-0.5\Height}\setlength{\Depth}{0.5\Depth}\addtolength{\Height}{\Depth}%
\put(-0.2500000,1.0000000){\hspace*{\Width}\raisebox{\Height}{$1$}}%
%
%
\special{pa    20  -315}\special{pa   -20  -315}%
\special{fp}%
\settowidth{\Width}{$2$}\setlength{\Width}{-1\Width}%
\settoheight{\Height}{$2$}\settodepth{\Depth}{$2$}\setlength{\Height}{-0.5\Height}\setlength{\Depth}{0.5\Depth}\addtolength{\Height}{\Depth}%
\put(-0.2500000,2.0000000){\hspace*{\Width}\raisebox{\Height}{$2$}}%
%
%
\special{pa  -787    -0}\special{pa   787    -0}%
\special{fp}%
\special{pa     0   787}\special{pa     0  -787}%
\special{fp}%
\settowidth{\Width}{$x$}\setlength{\Width}{0\Width}%
\settoheight{\Height}{$x$}\settodepth{\Depth}{$x$}\setlength{\Height}{-0.5\Height}\setlength{\Depth}{0.5\Depth}\addtolength{\Height}{\Depth}%
\put(5.1250000,0.0000000){\hspace*{\Width}\raisebox{\Height}{$x$}}%
%
\settowidth{\Width}{$y$}\setlength{\Width}{-0.5\Width}%
\settoheight{\Height}{$y$}\settodepth{\Depth}{$y$}\setlength{\Height}{\Depth}%
\put(0.0000000,5.1250000){\hspace*{\Width}\raisebox{\Height}{$y$}}%
%
\settowidth{\Width}{O}\setlength{\Width}{-1\Width}%
\settoheight{\Height}{O}\settodepth{\Depth}{O}\setlength{\Height}{-\Height}%
\put(-0.1250000,-0.1250000){\hspace*{\Width}\raisebox{\Height}{O}}%
%
\end{picture}}%}
\end{layer}

\begin{enumerate}[(1)]
\item 右のグラフのうち\\
 $y=2^x$は\;\hako{6}{5}{}\\
 $y=\log_2 x$は\;\hako{6}{6}{}
\item $a^0=\;\hako{6}{5}{}$\;,$a^{-1}=\;\hako{6}{5}{}$
\item $\log_a 1=\;\hako{6}{5}{}$\;,$\log_a a^2=\;\hako{6}{5}{}$
\item $5^{1.4}\times 5^{0.6}=5^{\,\hako{2}{2}{}}=\;\hako{6}{5}{}$
\item $\log_2 3-\log_2 6=\log_2\hako{5}{5}{}=\;\hako{6}{5}{}$
\end{enumerate}

\newpage
%各4 36
\item 
\begin{enumerate}[(1)]
\item 数列$-1,\ 1,\ -1,\ 1,\ \cdots$の一般項(第$n$項)は\\
\hfill\hako{20}{6}{} %
\item 等差数列$1,\ 3,\ 5,\cdots$の一般項(第$n$項)は\\
\hfill\hako{20}{6}{} %
\item 等比数列$1,\ 3,\ 9,\ \cdots$の一般項(第$n$項)は\\ %
\hfill\hako{20}{6}{} %
\item 数列$\dfrac{3}{2},\ \dfrac{9}{3},\ \dfrac{27}{4},\ \cdots$の第$n$項までの和を%
$\displaystyle\sum$で表せ.\vspace{6zw}
\end{enumerate}

\item 
\begin{enumerate}[(1)]
\item $x^2+4x+1=0$の解は \hfill$x=$\;\hako{20}{6}{} \vspace{1zw}%
\item $x^2+4x+5=0$の解は \hfill$x=$\;\hako{20}{6}{} \vspace{1zw}%
\item $\sqrt{-8}$を$i$で表すと \hfill$x=$\hako{20}{6}{}\vspace{1zw} 
\item $(1+2i)(2+3i)$を計算せよ.\vspace{4zw}
\item $\dfrac{2+3i}{1+2i}$を計算せよ.
\end{enumerate}


\end{Enumerate}

\end{document}