%%% Title presen23107
\documentclass[landscape,10pt]{ujarticle}
\special{papersize=\the\paperwidth,\the\paperheight}
\usepackage{ketpic,ketlayer}
\usepackage{ketslide}
\usepackage{amsmath,amssymb}
\usepackage{bm,enumerate}
\usepackage[dvipdfmx]{graphicx}
\usepackage{color}
\definecolor{slidecolora}{cmyk}{0.98,0.13,0,0.43}
\definecolor{slidecolorb}{cmyk}{0.2,0,0,0}
\definecolor{slidecolorc}{cmyk}{0.2,0,0,0}
\definecolor{slidecolord}{cmyk}{0.2,0,0,0}
\definecolor{slidecolore}{cmyk}{0,0,0,0.5}
\definecolor{slidecolorf}{cmyk}{0,0,0,0.5}
\definecolor{slidecolori}{cmyk}{0.98,0.13,0,0.43}
\def\setthin#1{\def\thin{#1}}
\setthin{0}
\newcommand{\slidepage}[1][s]{%
\setcounter{ketpicctra}{18}%
\if#1m \setcounter{ketpicctra}{1}\fi
\hypersetup{linkcolor=black}%

\begin{layer}{118}{0}
\putnotee{122}{-\theketpicctra.05}{\small\thepage/\pageref{pageend}}
\end{layer}\hypersetup{linkcolor=blue}

}
\usepackage{emath}
\usepackage{emathEy}
\usepackage{emathMw}
\usepackage{pict2e}
\usepackage{ketlayermorewith2e}
\usepackage[dvipdfmx,colorlinks=true,linkcolor=blue,filecolor=blue]{hyperref}
\newcommand{\hiduke}{0605}
\newcommand{\hako}[2][1]{\fbox{\raisebox{#1mm}{\mbox{}}\raisebox{-#1mm}{\mbox{}}\,\phantom{#2}\,}}
\newcommand{\hakoa}[2][1]{\fbox{\raisebox{#1mm}{\mbox{}}\raisebox{-#1mm}{\mbox{}}\,#2\,}}
\newcommand{\hakom}[2][1]{\hako[#1]{$#2$}}
\newcommand{\hakoma}[2][1]{\hakoa[#1]{$#2$}}
\def\rad{\;\mathrm{rad}}
\def\deg#1{#1^{\circ}}
\newcommand{\sbunsuu}[2]{\scalebox{0.6}{$\bunsuu{#1}{#2}$}}
\def\pow{$\hspace{-1.5mm}^\hspace{-1mm}$}
\def\dlim{\displaystyle\lim}
\newcommand{\brd}[2][1]{\scalebox{#1}{\color{red}\fbox{\color{black}$#2$}}}
\newcommand\down[1][0.5zw]{\vspace{#1}\\}
\newcommand{\sfrac}[3][0.65]{\scalebox{#1}{$\frac{#2}{#3}$}}
\newcommand{\phn}[1]{\phantom{#1}}
\newcommand{\scb}[2][0.6]{\scalebox{#1}{#2}}
\newcommand{\dsum}{\displaystyle\sum}

\setmargin{25}{145}{15}{100}

\ketslideinit

\pagestyle{empty}

\begin{document}

\begin{layer}{120}{0}
\putnotese{0}{0}{{\Large\bf
\color[cmyk]{1,1,0,0}

\begin{layer}{120}{0}
{\Huge \putnotes{60}{20}{三角比と三角関数}}
\putnotes{60}{70}{2022.04.25}
\end{layer}

}
}
\end{layer}

\def\mainslidetitley{22}
\def\ketcletter{slidecolora}
\def\ketcbox{slidecolorb}
\def\ketdbox{slidecolorc}
\def\ketcframe{slidecolord}
\def\ketcshadow{slidecolore}
\def\ketdshadow{slidecolorf}
\def\slidetitlex{6}
\def\slidetitlesize{1.3}
\def\mketcletter{slidecolori}
\def\mketcbox{yellow}
\def\mketdbox{yellow}
\def\mketcframe{yellow}
\def\mslidetitlex{62}
\def\mslidetitlesize{2}

\color{black}
\Large\bf\boldmath
\addtocounter{page}{-1}

\def\MARU{}
\renewcommand{\MARU}[1]{{\ooalign{\hfil$#1$\/\hfil\crcr\raise.167ex\hbox{\mathhexbox20D}}}}
\renewcommand{\slidepage}[1][s]{%
\setcounter{ketpicctra}{18}%
\if#1m \setcounter{ketpicctra}{1}\fi
\hypersetup{linkcolor=black}%
\begin{layer}{118}{0}
\putnotee{115}{-\theketpicctra.05}{\small\hiduke-\thepage/\pageref{pageend}}
\end{layer}\hypersetup{linkcolor=blue}
}
\newcounter{ban}
\setcounter{ban}{1}
\newcommand{\monban}[1][\hiduke]{%
#1-\theban\ %
\addtocounter{ban}{1}%
}
\newcommand{\monbannoadd}[1][\hiduke]{%
#1-\theban\ %
}
\newcommand{\addban}{%
\addtocounter{ban}{1}%%210614
}
\newcounter{edawidth}
\newcounter{edactr}
\newcommand{\seteda}[1]{
\setcounter{edawidth}{#1}
\setcounter{edactr}{1}
}
\newcommand{\eda}[2][\theedawidth ]{%
\noindent\Ltab{#1 mm}{[\theedactr]\ #2}%
\addtocounter{edactr}{1}%
}
%%%%%%%%%%%%

%%%%%%%%%%%%%%%%%%%%

\mainslide{数列}


\slidepage[m]
%%%%%%%%%%%%
%%%%%%%%%%%%

%%%%%%%%%%%%%%%%%%%%

\newslide{等差数列と等比数列}

\vspace*{18mm}

\hypertarget{para1pg1}{}

\begin{layer}{120}{0}
\putnotew{96}{73}{\hyperlink{para0pg0}{\fbox{\Ctab{2.5mm}{\scalebox{1}{\scriptsize $\mathstrut||\!\lhd$}}}}}
\putnotew{101}{73}{\hyperlink{para1pg1}{\fbox{\Ctab{2.5mm}{\scalebox{1}{\scriptsize $\mathstrut|\!\lhd$}}}}}
\putnotew{108}{73}{\hyperlink{para0pg0}{\fbox{\Ctab{4.5mm}{\scalebox{1}{\scriptsize $\mathstrut\!\!\lhd\!\!$}}}}}
\putnotew{115}{73}{\hyperlink{para1pg2}{\fbox{\Ctab{4.5mm}{\scalebox{1}{\scriptsize $\mathstrut\!\rhd\!$}}}}}
\putnotew{120}{73}{\hyperlink{para1pg6}{\fbox{\Ctab{2.5mm}{\scalebox{1}{\scriptsize $\mathstrut \!\rhd\!\!|$}}}}}
\putnotew{125}{73}{\hyperlink{para2pg1}{\fbox{\Ctab{2.5mm}{\scalebox{1}{\scriptsize $\mathstrut \!\rhd\!\!||$}}}}}
\putnotee{126}{73}{\scriptsize\color{blue} 1/6}
\end{layer}

\slidepage
\begin{itemize}
\item
[]差が等しい数列を等差数列,等しい差を公差という
\item
初項を$a$,公差を$d$の等差数列の第$n$項を$a_n$とおくと\\
 $a_1=a,\ a_2=a+d,\ a_3=a+2d,\ \cdots$\\
\hspace*{3zw}一般項$a_n=\hakom{a+(n-1)d}$
\end{itemize}

%%%%%%%%%%%%%%%%%%%%


\sameslide

\vspace*{18mm}

\hypertarget{para1pg2}{}

\begin{layer}{120}{0}
\putnotew{96}{73}{\hyperlink{para0pg0}{\fbox{\Ctab{2.5mm}{\scalebox{1}{\scriptsize $\mathstrut||\!\lhd$}}}}}
\putnotew{101}{73}{\hyperlink{para1pg1}{\fbox{\Ctab{2.5mm}{\scalebox{1}{\scriptsize $\mathstrut|\!\lhd$}}}}}
\putnotew{108}{73}{\hyperlink{para1pg1}{\fbox{\Ctab{4.5mm}{\scalebox{1}{\scriptsize $\mathstrut\!\!\lhd\!\!$}}}}}
\putnotew{115}{73}{\hyperlink{para1pg3}{\fbox{\Ctab{4.5mm}{\scalebox{1}{\scriptsize $\mathstrut\!\rhd\!$}}}}}
\putnotew{120}{73}{\hyperlink{para1pg6}{\fbox{\Ctab{2.5mm}{\scalebox{1}{\scriptsize $\mathstrut \!\rhd\!\!|$}}}}}
\putnotew{125}{73}{\hyperlink{para2pg1}{\fbox{\Ctab{2.5mm}{\scalebox{1}{\scriptsize $\mathstrut \!\rhd\!\!||$}}}}}
\putnotee{126}{73}{\scriptsize\color{blue} 2/6}
\end{layer}

\slidepage
\begin{itemize}
\item
[]差が等しい数列を等差数列,等しい差を公差という
\item
初項を$a$,公差を$d$の等差数列の第$n$項を$a_n$とおくと\\
 $a_1=a,\ a_2=a+d,\ a_3=a+2d,\ \cdots$\\
\hspace*{3zw}一般項$a_n=\hakoma{a+(n-1)d}$
\end{itemize}

\sameslide

\vspace*{18mm}

\hypertarget{para1pg3}{}

\begin{layer}{120}{0}
\putnotew{96}{73}{\hyperlink{para0pg0}{\fbox{\Ctab{2.5mm}{\scalebox{1}{\scriptsize $\mathstrut||\!\lhd$}}}}}
\putnotew{101}{73}{\hyperlink{para1pg1}{\fbox{\Ctab{2.5mm}{\scalebox{1}{\scriptsize $\mathstrut|\!\lhd$}}}}}
\putnotew{108}{73}{\hyperlink{para1pg2}{\fbox{\Ctab{4.5mm}{\scalebox{1}{\scriptsize $\mathstrut\!\!\lhd\!\!$}}}}}
\putnotew{115}{73}{\hyperlink{para1pg4}{\fbox{\Ctab{4.5mm}{\scalebox{1}{\scriptsize $\mathstrut\!\rhd\!$}}}}}
\putnotew{120}{73}{\hyperlink{para1pg6}{\fbox{\Ctab{2.5mm}{\scalebox{1}{\scriptsize $\mathstrut \!\rhd\!\!|$}}}}}
\putnotew{125}{73}{\hyperlink{para2pg1}{\fbox{\Ctab{2.5mm}{\scalebox{1}{\scriptsize $\mathstrut \!\rhd\!\!||$}}}}}
\putnotee{126}{73}{\scriptsize\color{blue} 3/6}
\end{layer}

\slidepage
\begin{itemize}
\item
[]差が等しい数列を等差数列,等しい差を公差という
\item
初項を$a$,公差を$d$の等差数列の第$n$項を$a_n$とおくと\\
 $a_1=a,\ a_2=a+d,\ a_3=a+2d,\ \cdots$\\
\hspace*{3zw}一般項$a_n=\hakoma{a+(n-1)d}$
\item
初項を$a$,公比を$r$の等比数列の第$n$項を$a_n$とおくと\\
\end{itemize}

\sameslide

\vspace*{18mm}

\hypertarget{para1pg4}{}

\begin{layer}{120}{0}
\putnotew{96}{73}{\hyperlink{para0pg0}{\fbox{\Ctab{2.5mm}{\scalebox{1}{\scriptsize $\mathstrut||\!\lhd$}}}}}
\putnotew{101}{73}{\hyperlink{para1pg1}{\fbox{\Ctab{2.5mm}{\scalebox{1}{\scriptsize $\mathstrut|\!\lhd$}}}}}
\putnotew{108}{73}{\hyperlink{para1pg3}{\fbox{\Ctab{4.5mm}{\scalebox{1}{\scriptsize $\mathstrut\!\!\lhd\!\!$}}}}}
\putnotew{115}{73}{\hyperlink{para1pg5}{\fbox{\Ctab{4.5mm}{\scalebox{1}{\scriptsize $\mathstrut\!\rhd\!$}}}}}
\putnotew{120}{73}{\hyperlink{para1pg6}{\fbox{\Ctab{2.5mm}{\scalebox{1}{\scriptsize $\mathstrut \!\rhd\!\!|$}}}}}
\putnotew{125}{73}{\hyperlink{para2pg1}{\fbox{\Ctab{2.5mm}{\scalebox{1}{\scriptsize $\mathstrut \!\rhd\!\!||$}}}}}
\putnotee{126}{73}{\scriptsize\color{blue} 4/6}
\end{layer}

\slidepage
\begin{itemize}
\item
[]差が等しい数列を等差数列,等しい差を公差という
\item
初項を$a$,公差を$d$の等差数列の第$n$項を$a_n$とおくと\\
 $a_1=a,\ a_2=a+d,\ a_3=a+2d,\ \cdots$\\
\hspace*{3zw}一般項$a_n=\hakoma{a+(n-1)d}$
\item
初項を$a$,公比を$r$の等比数列の第$n$項を$a_n$とおくと\\
 $a_1=a,\ a_2=ar,\ a_3=ar^2,\ \cdots$\\
\end{itemize}

\sameslide

\vspace*{18mm}

\hypertarget{para1pg5}{}

\begin{layer}{120}{0}
\putnotew{96}{73}{\hyperlink{para0pg0}{\fbox{\Ctab{2.5mm}{\scalebox{1}{\scriptsize $\mathstrut||\!\lhd$}}}}}
\putnotew{101}{73}{\hyperlink{para1pg1}{\fbox{\Ctab{2.5mm}{\scalebox{1}{\scriptsize $\mathstrut|\!\lhd$}}}}}
\putnotew{108}{73}{\hyperlink{para1pg4}{\fbox{\Ctab{4.5mm}{\scalebox{1}{\scriptsize $\mathstrut\!\!\lhd\!\!$}}}}}
\putnotew{115}{73}{\hyperlink{para1pg6}{\fbox{\Ctab{4.5mm}{\scalebox{1}{\scriptsize $\mathstrut\!\rhd\!$}}}}}
\putnotew{120}{73}{\hyperlink{para1pg6}{\fbox{\Ctab{2.5mm}{\scalebox{1}{\scriptsize $\mathstrut \!\rhd\!\!|$}}}}}
\putnotew{125}{73}{\hyperlink{para2pg1}{\fbox{\Ctab{2.5mm}{\scalebox{1}{\scriptsize $\mathstrut \!\rhd\!\!||$}}}}}
\putnotee{126}{73}{\scriptsize\color{blue} 5/6}
\end{layer}

\slidepage
\begin{itemize}
\item
[]差が等しい数列を等差数列,等しい差を公差という
\item
初項を$a$,公差を$d$の等差数列の第$n$項を$a_n$とおくと\\
 $a_1=a,\ a_2=a+d,\ a_3=a+2d,\ \cdots$\\
\hspace*{3zw}一般項$a_n=\hakoma{a+(n-1)d}$
\item
初項を$a$,公比を$r$の等比数列の第$n$項を$a_n$とおくと\\
 $a_1=a,\ a_2=ar,\ a_3=ar^2,\ \cdots$\\
\hspace*{3zw}一般項$a_n=\hakom{a r^{n-1}}$
\end{itemize}

\sameslide

\vspace*{18mm}

\hypertarget{para1pg6}{}

\begin{layer}{120}{0}
\putnotew{96}{73}{\hyperlink{para0pg0}{\fbox{\Ctab{2.5mm}{\scalebox{1}{\scriptsize $\mathstrut||\!\lhd$}}}}}
\putnotew{101}{73}{\hyperlink{para1pg1}{\fbox{\Ctab{2.5mm}{\scalebox{1}{\scriptsize $\mathstrut|\!\lhd$}}}}}
\putnotew{108}{73}{\hyperlink{para1pg5}{\fbox{\Ctab{4.5mm}{\scalebox{1}{\scriptsize $\mathstrut\!\!\lhd\!\!$}}}}}
\putnotew{115}{73}{\hyperlink{para1pg6}{\fbox{\Ctab{4.5mm}{\scalebox{1}{\scriptsize $\mathstrut\!\rhd\!$}}}}}
\putnotew{120}{73}{\hyperlink{para1pg6}{\fbox{\Ctab{2.5mm}{\scalebox{1}{\scriptsize $\mathstrut \!\rhd\!\!|$}}}}}
\putnotew{125}{73}{\hyperlink{para2pg1}{\fbox{\Ctab{2.5mm}{\scalebox{1}{\scriptsize $\mathstrut \!\rhd\!\!||$}}}}}
\putnotee{126}{73}{\scriptsize\color{blue} 6/6}
\end{layer}

\slidepage
\begin{itemize}
\item
[]差が等しい数列を等差数列,等しい差を公差という
\item
初項を$a$,公差を$d$の等差数列の第$n$項を$a_n$とおくと\\
 $a_1=a,\ a_2=a+d,\ a_3=a+2d,\ \cdots$\\
\hspace*{3zw}一般項$a_n=\hakoma{a+(n-1)d}$
\item
初項を$a$,公比を$r$の等比数列の第$n$項を$a_n$とおくと\\
 $a_1=a,\ a_2=ar,\ a_3=ar^2,\ \cdots$\\
\hspace*{3zw}一般項$a_n=\hakoma{a r^{n-1}}$
\end{itemize}

\newslide{等差数列の和}

\vspace*{18mm}

\slidepage
\vspace*{5mm}

初項$a$,公差$d$,項数が4の場合で説明する
\begin{itemize}
\item
[]\hspace*{1zw}$S=\Ctab{20mm}{a}+\Ctab{17mm}{(a+d)}+\Ctab{17mm}{(a+2d)}+\Ctab{17mm}{(a+3d)}$
\end{itemize}
%%%%%%%%%%%%%

%%%%%%%%%%%%%%%%%%%%


\sameslide

\vspace*{18mm}

\slidepage
\vspace*{5mm}

初項$a$,公差$d$,項数が4の場合で説明する
\begin{itemize}
\item
[]\hspace*{1zw}$S=\Ctab{20mm}{a}+\Ctab{17mm}{(a+d)}+\Ctab{17mm}{(a+2d)}+\Ctab{17mm}{(a+3d)}$
\item
[]逆順にして
\item
[]\hspace*{1zw}$S=\Ctab{20mm}{(a+3d)}+\Ctab{17mm}{(a+2d)}+\Ctab{17mm}{(a+d)}+\Ctab{17mm}{a}$
\end{itemize}

\sameslide

\vspace*{18mm}

\slidepage
\vspace*{5mm}

初項$a$,公差$d$,項数が4の場合で説明する
\begin{itemize}
\item
[]\hspace*{1zw}$S=\Ctab{20mm}{a}+\Ctab{17mm}{(a+d)}+\Ctab{17mm}{(a+2d)}+\Ctab{17mm}{(a+3d)}$
\item
[]逆順にして
\item
[]\hspace*{1zw}$S=\Ctab{20mm}{(a+3d)}+\Ctab{17mm}{(a+2d)}+\Ctab{17mm}{(a+d)}+\Ctab{17mm}{a}$
\item
[]2つの式を加えると\\
\end{itemize}

\sameslide

\vspace*{18mm}

\slidepage
\vspace*{5mm}

初項$a$,公差$d$,項数が4の場合で説明する
\begin{itemize}
\item
[]\hspace*{1zw}$S=\Ctab{20mm}{a}+\Ctab{17mm}{(a+d)}+\Ctab{17mm}{(a+2d)}+\Ctab{17mm}{(a+3d)}$
\item
[]逆順にして
\item
[]\hspace*{1zw}$S=\Ctab{20mm}{(a+3d)}+\Ctab{17mm}{(a+2d)}+\Ctab{17mm}{(a+d)}+\Ctab{17mm}{a}$
\item
[]2つの式を加えると\\
 $2S=(2a+3d)\times 4$\\
\end{itemize}

\sameslide

\vspace*{18mm}

\slidepage
\vspace*{5mm}

初項$a$,公差$d$,項数が4の場合で説明する
\begin{itemize}
\item
[]\hspace*{1zw}$S=\Ctab{20mm}{a}+\Ctab{17mm}{(a+d)}+\Ctab{17mm}{(a+2d)}+\Ctab{17mm}{(a+3d)}$
\item
[]逆順にして
\item
[]\hspace*{1zw}$S=\Ctab{20mm}{(a+3d)}+\Ctab{17mm}{(a+2d)}+\Ctab{17mm}{(a+d)}+\Ctab{17mm}{a}$
\item
[]2つの式を加えると\\
 $2S=(2a+3d)\times 4$\\
 $2$で割ると $S=\bunsuu{4(2a+3d)}{2}=\hakom{\bunsuu{n(a+a+3d)}{2}}$
\end{itemize}

\sameslide

\vspace*{18mm}

\slidepage
\vspace*{5mm}

初項$a$,公差$d$,項数が4の場合で説明する
\begin{itemize}
\item
[]\hspace*{1zw}$S=\Ctab{20mm}{a}+\Ctab{17mm}{(a+d)}+\Ctab{17mm}{(a+2d)}+\Ctab{17mm}{(a+3d)}$
\item
[]逆順にして
\item
[]\hspace*{1zw}$S=\Ctab{20mm}{(a+3d)}+\Ctab{17mm}{(a+2d)}+\Ctab{17mm}{(a+d)}+\Ctab{17mm}{a}$
\item
[]2つの式を加えると\\
 $2S=(2a+3d)\times 4$\\
 $2$で割ると $S=\bunsuu{4(2a+3d)}{2}=\hakoma{\bunsuu{n(a+a+3d)}{2}}$
\end{itemize}

\newslide{等差数列の和の公式}

\vspace*{18mm}

\hypertarget{para2pg1}{}

\begin{layer}{120}{0}
\putnotew{96}{73}{\hyperlink{para1pg6}{\fbox{\Ctab{2.5mm}{\scalebox{1}{\scriptsize $\mathstrut||\!\lhd$}}}}}
\putnotew{101}{73}{\hyperlink{para2pg1}{\fbox{\Ctab{2.5mm}{\scalebox{1}{\scriptsize $\mathstrut|\!\lhd$}}}}}
\putnotew{108}{73}{\hyperlink{para1pg6}{\fbox{\Ctab{4.5mm}{\scalebox{1}{\scriptsize $\mathstrut\!\!\lhd\!\!$}}}}}
\putnotew{115}{73}{\hyperlink{para2pg2}{\fbox{\Ctab{4.5mm}{\scalebox{1}{\scriptsize $\mathstrut\!\rhd\!$}}}}}
\putnotew{120}{73}{\hyperlink{para2pg3}{\fbox{\Ctab{2.5mm}{\scalebox{1}{\scriptsize $\mathstrut \!\rhd\!\!|$}}}}}
\putnotew{125}{73}{\hyperlink{para3pg1}{\fbox{\Ctab{2.5mm}{\scalebox{1}{\scriptsize $\mathstrut \!\rhd\!\!||$}}}}}
\putnotee{126}{73}{\scriptsize\color{blue} 1/3}
\end{layer}

\slidepage
\begin{itemize}
\item
[]初項$a$,公差$d$,項数$n$の等差数列の和$S$は\vspace{2mm}\\
  \fbox{$S=\bunsuu{n(2a+(n-1)d)}{2}$}
\end{itemize}

%%%%%%%%%%%%%%%%%%%%


\sameslide

\vspace*{18mm}

\hypertarget{para2pg2}{}

\begin{layer}{120}{0}
\putnotew{96}{73}{\hyperlink{para1pg6}{\fbox{\Ctab{2.5mm}{\scalebox{1}{\scriptsize $\mathstrut||\!\lhd$}}}}}
\putnotew{101}{73}{\hyperlink{para2pg1}{\fbox{\Ctab{2.5mm}{\scalebox{1}{\scriptsize $\mathstrut|\!\lhd$}}}}}
\putnotew{108}{73}{\hyperlink{para2pg1}{\fbox{\Ctab{4.5mm}{\scalebox{1}{\scriptsize $\mathstrut\!\!\lhd\!\!$}}}}}
\putnotew{115}{73}{\hyperlink{para2pg3}{\fbox{\Ctab{4.5mm}{\scalebox{1}{\scriptsize $\mathstrut\!\rhd\!$}}}}}
\putnotew{120}{73}{\hyperlink{para2pg3}{\fbox{\Ctab{2.5mm}{\scalebox{1}{\scriptsize $\mathstrut \!\rhd\!\!|$}}}}}
\putnotew{125}{73}{\hyperlink{para3pg1}{\fbox{\Ctab{2.5mm}{\scalebox{1}{\scriptsize $\mathstrut \!\rhd\!\!||$}}}}}
\putnotee{126}{73}{\scriptsize\color{blue} 2/3}
\end{layer}

\slidepage
\begin{itemize}
\item
[]初項$a$,公差$d$,項数$n$の等差数列の和$S$は\vspace{2mm}\\
  \fbox{$S=\bunsuu{n(2a+(n-1)d)}{2}$}
\item
[]$2a+(n-1)d=a+(a+(n-1)d)=\mbox{初項}+\mbox{末項}$\\
\end{itemize}

\sameslide

\vspace*{18mm}

\hypertarget{para2pg3}{}

\begin{layer}{120}{0}
\putnotew{96}{73}{\hyperlink{para1pg6}{\fbox{\Ctab{2.5mm}{\scalebox{1}{\scriptsize $\mathstrut||\!\lhd$}}}}}
\putnotew{101}{73}{\hyperlink{para2pg1}{\fbox{\Ctab{2.5mm}{\scalebox{1}{\scriptsize $\mathstrut|\!\lhd$}}}}}
\putnotew{108}{73}{\hyperlink{para2pg2}{\fbox{\Ctab{4.5mm}{\scalebox{1}{\scriptsize $\mathstrut\!\!\lhd\!\!$}}}}}
\putnotew{115}{73}{\hyperlink{para2pg3}{\fbox{\Ctab{4.5mm}{\scalebox{1}{\scriptsize $\mathstrut\!\rhd\!$}}}}}
\putnotew{120}{73}{\hyperlink{para2pg3}{\fbox{\Ctab{2.5mm}{\scalebox{1}{\scriptsize $\mathstrut \!\rhd\!\!|$}}}}}
\putnotew{125}{73}{\hyperlink{para3pg1}{\fbox{\Ctab{2.5mm}{\scalebox{1}{\scriptsize $\mathstrut \!\rhd\!\!||$}}}}}
\putnotee{126}{73}{\scriptsize\color{blue} 3/3}
\end{layer}

\slidepage
\begin{itemize}
\item
[]初項$a$,公差$d$,項数$n$の等差数列の和$S$は\vspace{2mm}\\
  \fbox{$S=\bunsuu{n(2a+(n-1)d)}{2}$}
\item
[]$2a+(n-1)d=a+(a+(n-1)d)=\mbox{初項}+\mbox{末項}$\\
\item
[]  \raisebox{-5mm}{\fbox{$S=\bunsuu{\mbox{項数}\times (\mbox{初項}+\mbox{末項})}{2}$}}
\end{itemize}

\newslide{等差数列の和の例題}

\vspace*{18mm}

\hypertarget{para3pg1}{}

\begin{layer}{120}{0}
\putnotew{96}{73}{\hyperlink{para2pg3}{\fbox{\Ctab{2.5mm}{\scalebox{1}{\scriptsize $\mathstrut||\!\lhd$}}}}}
\putnotew{101}{73}{\hyperlink{para3pg1}{\fbox{\Ctab{2.5mm}{\scalebox{1}{\scriptsize $\mathstrut|\!\lhd$}}}}}
\putnotew{108}{73}{\hyperlink{para2pg3}{\fbox{\Ctab{4.5mm}{\scalebox{1}{\scriptsize $\mathstrut\!\!\lhd\!\!$}}}}}
\putnotew{115}{73}{\hyperlink{para3pg2}{\fbox{\Ctab{4.5mm}{\scalebox{1}{\scriptsize $\mathstrut\!\rhd\!$}}}}}
\putnotew{120}{73}{\hyperlink{para3pg8}{\fbox{\Ctab{2.5mm}{\scalebox{1}{\scriptsize $\mathstrut \!\rhd\!\!|$}}}}}
\putnotew{125}{73}{\hyperlink{para4pg1}{\fbox{\Ctab{2.5mm}{\scalebox{1}{\scriptsize $\mathstrut \!\rhd\!\!||$}}}}}
\putnotee{126}{73}{\scriptsize\color{blue} 1/8}
\end{layer}

\slidepage
\begin{itemize}
\item
[例題)]$S=1+3+5+7+\cdots+99$を求めよ.
\end{itemize}

%%%%%%%%%%%%%%%%%%%%


\sameslide

\vspace*{18mm}

\hypertarget{para3pg2}{}

\begin{layer}{120}{0}
\putnotew{96}{73}{\hyperlink{para2pg3}{\fbox{\Ctab{2.5mm}{\scalebox{1}{\scriptsize $\mathstrut||\!\lhd$}}}}}
\putnotew{101}{73}{\hyperlink{para3pg1}{\fbox{\Ctab{2.5mm}{\scalebox{1}{\scriptsize $\mathstrut|\!\lhd$}}}}}
\putnotew{108}{73}{\hyperlink{para3pg1}{\fbox{\Ctab{4.5mm}{\scalebox{1}{\scriptsize $\mathstrut\!\!\lhd\!\!$}}}}}
\putnotew{115}{73}{\hyperlink{para3pg3}{\fbox{\Ctab{4.5mm}{\scalebox{1}{\scriptsize $\mathstrut\!\rhd\!$}}}}}
\putnotew{120}{73}{\hyperlink{para3pg8}{\fbox{\Ctab{2.5mm}{\scalebox{1}{\scriptsize $\mathstrut \!\rhd\!\!|$}}}}}
\putnotew{125}{73}{\hyperlink{para4pg1}{\fbox{\Ctab{2.5mm}{\scalebox{1}{\scriptsize $\mathstrut \!\rhd\!\!||$}}}}}
\putnotee{126}{73}{\scriptsize\color{blue} 2/8}
\end{layer}

\slidepage
\begin{itemize}
\item
[例題)]$S=1+3+5+7+\cdots+99$を求めよ.
\item
[解)]項数$n$を求める.\\
\end{itemize}

\sameslide

\vspace*{18mm}

\hypertarget{para3pg3}{}

\begin{layer}{120}{0}
\putnotew{96}{73}{\hyperlink{para2pg3}{\fbox{\Ctab{2.5mm}{\scalebox{1}{\scriptsize $\mathstrut||\!\lhd$}}}}}
\putnotew{101}{73}{\hyperlink{para3pg1}{\fbox{\Ctab{2.5mm}{\scalebox{1}{\scriptsize $\mathstrut|\!\lhd$}}}}}
\putnotew{108}{73}{\hyperlink{para3pg2}{\fbox{\Ctab{4.5mm}{\scalebox{1}{\scriptsize $\mathstrut\!\!\lhd\!\!$}}}}}
\putnotew{115}{73}{\hyperlink{para3pg4}{\fbox{\Ctab{4.5mm}{\scalebox{1}{\scriptsize $\mathstrut\!\rhd\!$}}}}}
\putnotew{120}{73}{\hyperlink{para3pg8}{\fbox{\Ctab{2.5mm}{\scalebox{1}{\scriptsize $\mathstrut \!\rhd\!\!|$}}}}}
\putnotew{125}{73}{\hyperlink{para4pg1}{\fbox{\Ctab{2.5mm}{\scalebox{1}{\scriptsize $\mathstrut \!\rhd\!\!||$}}}}}
\putnotee{126}{73}{\scriptsize\color{blue} 3/8}
\end{layer}

\slidepage
\begin{itemize}
\item
[例題)]$S=1+3+5+7+\cdots+99$を求めよ.
\item
[解)]項数$n$を求める.\\
初項$1$,公差$2$より,末項(第$n$項)$a_n$は\\
\end{itemize}

\sameslide

\vspace*{18mm}

\hypertarget{para3pg4}{}

\begin{layer}{120}{0}
\putnotew{96}{73}{\hyperlink{para2pg3}{\fbox{\Ctab{2.5mm}{\scalebox{1}{\scriptsize $\mathstrut||\!\lhd$}}}}}
\putnotew{101}{73}{\hyperlink{para3pg1}{\fbox{\Ctab{2.5mm}{\scalebox{1}{\scriptsize $\mathstrut|\!\lhd$}}}}}
\putnotew{108}{73}{\hyperlink{para3pg3}{\fbox{\Ctab{4.5mm}{\scalebox{1}{\scriptsize $\mathstrut\!\!\lhd\!\!$}}}}}
\putnotew{115}{73}{\hyperlink{para3pg5}{\fbox{\Ctab{4.5mm}{\scalebox{1}{\scriptsize $\mathstrut\!\rhd\!$}}}}}
\putnotew{120}{73}{\hyperlink{para3pg8}{\fbox{\Ctab{2.5mm}{\scalebox{1}{\scriptsize $\mathstrut \!\rhd\!\!|$}}}}}
\putnotew{125}{73}{\hyperlink{para4pg1}{\fbox{\Ctab{2.5mm}{\scalebox{1}{\scriptsize $\mathstrut \!\rhd\!\!||$}}}}}
\putnotee{126}{73}{\scriptsize\color{blue} 4/8}
\end{layer}

\slidepage
\begin{itemize}
\item
[例題)]$S=1+3+5+7+\cdots+99$を求めよ.
\item
[解)]項数$n$を求める.\\
初項$1$,公差$2$より,末項(第$n$項)$a_n$は\\
   $a_n=1+2(n-1)=2n-1$\vspace{2mm}\\
\end{itemize}

\sameslide

\vspace*{18mm}

\hypertarget{para3pg5}{}

\begin{layer}{120}{0}
\putnotew{96}{73}{\hyperlink{para2pg3}{\fbox{\Ctab{2.5mm}{\scalebox{1}{\scriptsize $\mathstrut||\!\lhd$}}}}}
\putnotew{101}{73}{\hyperlink{para3pg1}{\fbox{\Ctab{2.5mm}{\scalebox{1}{\scriptsize $\mathstrut|\!\lhd$}}}}}
\putnotew{108}{73}{\hyperlink{para3pg4}{\fbox{\Ctab{4.5mm}{\scalebox{1}{\scriptsize $\mathstrut\!\!\lhd\!\!$}}}}}
\putnotew{115}{73}{\hyperlink{para3pg6}{\fbox{\Ctab{4.5mm}{\scalebox{1}{\scriptsize $\mathstrut\!\rhd\!$}}}}}
\putnotew{120}{73}{\hyperlink{para3pg8}{\fbox{\Ctab{2.5mm}{\scalebox{1}{\scriptsize $\mathstrut \!\rhd\!\!|$}}}}}
\putnotew{125}{73}{\hyperlink{para4pg1}{\fbox{\Ctab{2.5mm}{\scalebox{1}{\scriptsize $\mathstrut \!\rhd\!\!||$}}}}}
\putnotee{126}{73}{\scriptsize\color{blue} 5/8}
\end{layer}

\slidepage
\begin{itemize}
\item
[例題)]$S=1+3+5+7+\cdots+99$を求めよ.
\item
[解)]項数$n$を求める.\\
初項$1$,公差$2$より,末項(第$n$項)$a_n$は\\
   $a_n=1+2(n-1)=2n-1$\vspace{2mm}\\
$a_n=2n-1=99$より
\end{itemize}

\sameslide

\vspace*{18mm}

\hypertarget{para3pg6}{}

\begin{layer}{120}{0}
\putnotew{96}{73}{\hyperlink{para2pg3}{\fbox{\Ctab{2.5mm}{\scalebox{1}{\scriptsize $\mathstrut||\!\lhd$}}}}}
\putnotew{101}{73}{\hyperlink{para3pg1}{\fbox{\Ctab{2.5mm}{\scalebox{1}{\scriptsize $\mathstrut|\!\lhd$}}}}}
\putnotew{108}{73}{\hyperlink{para3pg5}{\fbox{\Ctab{4.5mm}{\scalebox{1}{\scriptsize $\mathstrut\!\!\lhd\!\!$}}}}}
\putnotew{115}{73}{\hyperlink{para3pg7}{\fbox{\Ctab{4.5mm}{\scalebox{1}{\scriptsize $\mathstrut\!\rhd\!$}}}}}
\putnotew{120}{73}{\hyperlink{para3pg8}{\fbox{\Ctab{2.5mm}{\scalebox{1}{\scriptsize $\mathstrut \!\rhd\!\!|$}}}}}
\putnotew{125}{73}{\hyperlink{para4pg1}{\fbox{\Ctab{2.5mm}{\scalebox{1}{\scriptsize $\mathstrut \!\rhd\!\!||$}}}}}
\putnotee{126}{73}{\scriptsize\color{blue} 6/8}
\end{layer}

\slidepage
\begin{itemize}
\item
[例題)]$S=1+3+5+7+\cdots+99$を求めよ.
\item
[解)]項数$n$を求める.\\
初項$1$,公差$2$より,末項(第$n$項)$a_n$は\\
   $a_n=1+2(n-1)=2n-1$\vspace{2mm}\\
$a_n=2n-1=99$より
 $n=\bunsuu{99+1}{2}=50$\\
\end{itemize}

\sameslide

\vspace*{18mm}

\hypertarget{para3pg7}{}

\begin{layer}{120}{0}
\putnotew{96}{73}{\hyperlink{para2pg3}{\fbox{\Ctab{2.5mm}{\scalebox{1}{\scriptsize $\mathstrut||\!\lhd$}}}}}
\putnotew{101}{73}{\hyperlink{para3pg1}{\fbox{\Ctab{2.5mm}{\scalebox{1}{\scriptsize $\mathstrut|\!\lhd$}}}}}
\putnotew{108}{73}{\hyperlink{para3pg6}{\fbox{\Ctab{4.5mm}{\scalebox{1}{\scriptsize $\mathstrut\!\!\lhd\!\!$}}}}}
\putnotew{115}{73}{\hyperlink{para3pg8}{\fbox{\Ctab{4.5mm}{\scalebox{1}{\scriptsize $\mathstrut\!\rhd\!$}}}}}
\putnotew{120}{73}{\hyperlink{para3pg8}{\fbox{\Ctab{2.5mm}{\scalebox{1}{\scriptsize $\mathstrut \!\rhd\!\!|$}}}}}
\putnotew{125}{73}{\hyperlink{para4pg1}{\fbox{\Ctab{2.5mm}{\scalebox{1}{\scriptsize $\mathstrut \!\rhd\!\!||$}}}}}
\putnotee{126}{73}{\scriptsize\color{blue} 7/8}
\end{layer}

\slidepage
\begin{itemize}
\item
[例題)]$S=1+3+5+7+\cdots+99$を求めよ.
\item
[解)]項数$n$を求める.\\
初項$1$,公差$2$より,末項(第$n$項)$a_n$は\\
   $a_n=1+2(n-1)=2n-1$\vspace{2mm}\\
$a_n=2n-1=99$より
 $n=\bunsuu{99+1}{2}=50$\\
したがって $S=\bunsuu{50(1+99)}{2}$
\end{itemize}

\sameslide

\vspace*{18mm}

\hypertarget{para3pg8}{}

\begin{layer}{120}{0}
\putnotew{96}{73}{\hyperlink{para2pg3}{\fbox{\Ctab{2.5mm}{\scalebox{1}{\scriptsize $\mathstrut||\!\lhd$}}}}}
\putnotew{101}{73}{\hyperlink{para3pg1}{\fbox{\Ctab{2.5mm}{\scalebox{1}{\scriptsize $\mathstrut|\!\lhd$}}}}}
\putnotew{108}{73}{\hyperlink{para3pg7}{\fbox{\Ctab{4.5mm}{\scalebox{1}{\scriptsize $\mathstrut\!\!\lhd\!\!$}}}}}
\putnotew{115}{73}{\hyperlink{para3pg8}{\fbox{\Ctab{4.5mm}{\scalebox{1}{\scriptsize $\mathstrut\!\rhd\!$}}}}}
\putnotew{120}{73}{\hyperlink{para3pg8}{\fbox{\Ctab{2.5mm}{\scalebox{1}{\scriptsize $\mathstrut \!\rhd\!\!|$}}}}}
\putnotew{125}{73}{\hyperlink{para4pg1}{\fbox{\Ctab{2.5mm}{\scalebox{1}{\scriptsize $\mathstrut \!\rhd\!\!||$}}}}}
\putnotee{126}{73}{\scriptsize\color{blue} 8/8}
\end{layer}

\slidepage
\begin{itemize}
\item
[例題)]$S=1+3+5+7+\cdots+99$を求めよ.
\item
[解)]項数$n$を求める.\\
初項$1$,公差$2$より,末項(第$n$項)$a_n$は\\
   $a_n=1+2(n-1)=2n-1$\vspace{2mm}\\
$a_n=2n-1=99$より
 $n=\bunsuu{99+1}{2}=50$\\
したがって $S=\bunsuu{50(1+99)}{2}$
$=2500$
\end{itemize}

\newslide{等比数列の和}

\vspace*{18mm}

\hypertarget{para4pg1}{}

\begin{layer}{120}{0}
\putnotew{96}{73}{\hyperlink{para3pg8}{\fbox{\Ctab{2.5mm}{\scalebox{1}{\scriptsize $\mathstrut||\!\lhd$}}}}}
\putnotew{101}{73}{\hyperlink{para4pg1}{\fbox{\Ctab{2.5mm}{\scalebox{1}{\scriptsize $\mathstrut|\!\lhd$}}}}}
\putnotew{108}{73}{\hyperlink{para3pg8}{\fbox{\Ctab{4.5mm}{\scalebox{1}{\scriptsize $\mathstrut\!\!\lhd\!\!$}}}}}
\putnotew{115}{73}{\hyperlink{para4pg2}{\fbox{\Ctab{4.5mm}{\scalebox{1}{\scriptsize $\mathstrut\!\rhd\!$}}}}}
\putnotew{120}{73}{\hyperlink{para4pg7}{\fbox{\Ctab{2.5mm}{\scalebox{1}{\scriptsize $\mathstrut \!\rhd\!\!|$}}}}}
\putnotew{125}{73}{\hyperlink{para5pg1}{\fbox{\Ctab{2.5mm}{\scalebox{1}{\scriptsize $\mathstrut \!\rhd\!\!||$}}}}}
\putnotee{126}{73}{\scriptsize\color{blue} 1/7}
\end{layer}

\slidepage
{\color{blue}

\begin{layer}{120}{0}
\end{layer}

}
初項$a$,公比$r$,項数が5の場合で説明する
\begin{itemize}
\item
[]$S=a+ar+ar^2+ar^3+ar^4$
\end{itemize}
%%%%%%%%%%%%%

%%%%%%%%%%%%%%%%%%%%


\sameslide

\vspace*{18mm}

\hypertarget{para4pg2}{}

\begin{layer}{120}{0}
\putnotew{96}{73}{\hyperlink{para3pg8}{\fbox{\Ctab{2.5mm}{\scalebox{1}{\scriptsize $\mathstrut||\!\lhd$}}}}}
\putnotew{101}{73}{\hyperlink{para4pg1}{\fbox{\Ctab{2.5mm}{\scalebox{1}{\scriptsize $\mathstrut|\!\lhd$}}}}}
\putnotew{108}{73}{\hyperlink{para4pg1}{\fbox{\Ctab{4.5mm}{\scalebox{1}{\scriptsize $\mathstrut\!\!\lhd\!\!$}}}}}
\putnotew{115}{73}{\hyperlink{para4pg3}{\fbox{\Ctab{4.5mm}{\scalebox{1}{\scriptsize $\mathstrut\!\rhd\!$}}}}}
\putnotew{120}{73}{\hyperlink{para4pg7}{\fbox{\Ctab{2.5mm}{\scalebox{1}{\scriptsize $\mathstrut \!\rhd\!\!|$}}}}}
\putnotew{125}{73}{\hyperlink{para5pg1}{\fbox{\Ctab{2.5mm}{\scalebox{1}{\scriptsize $\mathstrut \!\rhd\!\!||$}}}}}
\putnotee{126}{73}{\scriptsize\color{blue} 2/7}
\end{layer}

\slidepage
{\color{blue}

\begin{layer}{120}{0}
\end{layer}

}
初項$a$,公比$r$,項数が5の場合で説明する
\begin{itemize}
\item
[]$S=a+ar+ar^2+ar^3+ar^4$
\item
[]$rS=ar+ar^2+ar^3+ar^4+ar^5$
\end{itemize}

\sameslide

\vspace*{18mm}

\hypertarget{para4pg3}{}

\begin{layer}{120}{0}
\putnotew{96}{73}{\hyperlink{para3pg8}{\fbox{\Ctab{2.5mm}{\scalebox{1}{\scriptsize $\mathstrut||\!\lhd$}}}}}
\putnotew{101}{73}{\hyperlink{para4pg1}{\fbox{\Ctab{2.5mm}{\scalebox{1}{\scriptsize $\mathstrut|\!\lhd$}}}}}
\putnotew{108}{73}{\hyperlink{para4pg2}{\fbox{\Ctab{4.5mm}{\scalebox{1}{\scriptsize $\mathstrut\!\!\lhd\!\!$}}}}}
\putnotew{115}{73}{\hyperlink{para4pg4}{\fbox{\Ctab{4.5mm}{\scalebox{1}{\scriptsize $\mathstrut\!\rhd\!$}}}}}
\putnotew{120}{73}{\hyperlink{para4pg7}{\fbox{\Ctab{2.5mm}{\scalebox{1}{\scriptsize $\mathstrut \!\rhd\!\!|$}}}}}
\putnotew{125}{73}{\hyperlink{para5pg1}{\fbox{\Ctab{2.5mm}{\scalebox{1}{\scriptsize $\mathstrut \!\rhd\!\!||$}}}}}
\putnotee{126}{73}{\scriptsize\color{blue} 3/7}
\end{layer}

\slidepage
{\color{blue}

\begin{layer}{120}{0}
\end{layer}

}
初項$a$,公比$r$,項数が5の場合で説明する
\begin{itemize}
\item
[]$S=a+ar+ar^2+ar^3+ar^4$
\item
[]$rS=ar+ar^2+ar^3+ar^4+ar^5$
\item
[]2つの式を引くと\\
\end{itemize}

\sameslide

\vspace*{18mm}

\hypertarget{para4pg4}{}

\begin{layer}{120}{0}
\putnotew{96}{73}{\hyperlink{para3pg8}{\fbox{\Ctab{2.5mm}{\scalebox{1}{\scriptsize $\mathstrut||\!\lhd$}}}}}
\putnotew{101}{73}{\hyperlink{para4pg1}{\fbox{\Ctab{2.5mm}{\scalebox{1}{\scriptsize $\mathstrut|\!\lhd$}}}}}
\putnotew{108}{73}{\hyperlink{para4pg3}{\fbox{\Ctab{4.5mm}{\scalebox{1}{\scriptsize $\mathstrut\!\!\lhd\!\!$}}}}}
\putnotew{115}{73}{\hyperlink{para4pg5}{\fbox{\Ctab{4.5mm}{\scalebox{1}{\scriptsize $\mathstrut\!\rhd\!$}}}}}
\putnotew{120}{73}{\hyperlink{para4pg7}{\fbox{\Ctab{2.5mm}{\scalebox{1}{\scriptsize $\mathstrut \!\rhd\!\!|$}}}}}
\putnotew{125}{73}{\hyperlink{para5pg1}{\fbox{\Ctab{2.5mm}{\scalebox{1}{\scriptsize $\mathstrut \!\rhd\!\!||$}}}}}
\putnotee{126}{73}{\scriptsize\color{blue} 4/7}
\end{layer}

\slidepage
{\color{blue}

\begin{layer}{120}{0}
\putnotesw{32}{19}{\large /}
\putnotesw{45}{19}{\large /}
\putnotesw{61}{19}{\large /}
\putnotesw{76}{19}{\large /}
\end{layer}

}
初項$a$,公比$r$,項数が5の場合で説明する
\begin{itemize}
\item
[]$S=a+ar+ar^2+ar^3+ar^4$
\item
[]$rS=ar+ar^2+ar^3+ar^4+ar^5$
\item
[]2つの式を引くと\\
\end{itemize}

\sameslide

\vspace*{18mm}

\hypertarget{para4pg5}{}

\begin{layer}{120}{0}
\putnotew{96}{73}{\hyperlink{para3pg8}{\fbox{\Ctab{2.5mm}{\scalebox{1}{\scriptsize $\mathstrut||\!\lhd$}}}}}
\putnotew{101}{73}{\hyperlink{para4pg1}{\fbox{\Ctab{2.5mm}{\scalebox{1}{\scriptsize $\mathstrut|\!\lhd$}}}}}
\putnotew{108}{73}{\hyperlink{para4pg4}{\fbox{\Ctab{4.5mm}{\scalebox{1}{\scriptsize $\mathstrut\!\!\lhd\!\!$}}}}}
\putnotew{115}{73}{\hyperlink{para4pg6}{\fbox{\Ctab{4.5mm}{\scalebox{1}{\scriptsize $\mathstrut\!\rhd\!$}}}}}
\putnotew{120}{73}{\hyperlink{para4pg7}{\fbox{\Ctab{2.5mm}{\scalebox{1}{\scriptsize $\mathstrut \!\rhd\!\!|$}}}}}
\putnotew{125}{73}{\hyperlink{para5pg1}{\fbox{\Ctab{2.5mm}{\scalebox{1}{\scriptsize $\mathstrut \!\rhd\!\!||$}}}}}
\putnotee{126}{73}{\scriptsize\color{blue} 5/7}
\end{layer}

\slidepage
{\color{blue}

\begin{layer}{120}{0}
\putnotesw{32}{19}{\large /}
\putnotesw{45}{19}{\large /}
\putnotesw{61}{19}{\large /}
\putnotesw{76}{19}{\large /}
\end{layer}

}
初項$a$,公比$r$,項数が5の場合で説明する
\begin{itemize}
\item
[]$S=a+ar+ar^2+ar^3+ar^4$
\item
[]$rS=ar+ar^2+ar^3+ar^4+ar^5$
\item
[]2つの式を引くと\\
 $S-rS=a-ar^5$\hspace{1zw}\\
\end{itemize}

\sameslide

\vspace*{18mm}

\hypertarget{para4pg6}{}

\begin{layer}{120}{0}
\putnotew{96}{73}{\hyperlink{para3pg8}{\fbox{\Ctab{2.5mm}{\scalebox{1}{\scriptsize $\mathstrut||\!\lhd$}}}}}
\putnotew{101}{73}{\hyperlink{para4pg1}{\fbox{\Ctab{2.5mm}{\scalebox{1}{\scriptsize $\mathstrut|\!\lhd$}}}}}
\putnotew{108}{73}{\hyperlink{para4pg5}{\fbox{\Ctab{4.5mm}{\scalebox{1}{\scriptsize $\mathstrut\!\!\lhd\!\!$}}}}}
\putnotew{115}{73}{\hyperlink{para4pg7}{\fbox{\Ctab{4.5mm}{\scalebox{1}{\scriptsize $\mathstrut\!\rhd\!$}}}}}
\putnotew{120}{73}{\hyperlink{para4pg7}{\fbox{\Ctab{2.5mm}{\scalebox{1}{\scriptsize $\mathstrut \!\rhd\!\!|$}}}}}
\putnotew{125}{73}{\hyperlink{para5pg1}{\fbox{\Ctab{2.5mm}{\scalebox{1}{\scriptsize $\mathstrut \!\rhd\!\!||$}}}}}
\putnotee{126}{73}{\scriptsize\color{blue} 6/7}
\end{layer}

\slidepage
{\color{blue}

\begin{layer}{120}{0}
\putnotesw{32}{19}{\large /}
\putnotesw{45}{19}{\large /}
\putnotesw{61}{19}{\large /}
\putnotesw{76}{19}{\large /}
\end{layer}

}
初項$a$,公比$r$,項数が5の場合で説明する
\begin{itemize}
\item
[]$S=a+ar+ar^2+ar^3+ar^4$
\item
[]$rS=ar+ar^2+ar^3+ar^4+ar^5$
\item
[]2つの式を引くと\\
 $S-rS=a-ar^5$\hspace{1zw}\\
$\therefore\hspace{0.75zw}(1-r)S=a(1-r^5)$
\end{itemize}

\sameslide

\vspace*{18mm}

\hypertarget{para4pg7}{}

\begin{layer}{120}{0}
\putnotew{96}{73}{\hyperlink{para3pg8}{\fbox{\Ctab{2.5mm}{\scalebox{1}{\scriptsize $\mathstrut||\!\lhd$}}}}}
\putnotew{101}{73}{\hyperlink{para4pg1}{\fbox{\Ctab{2.5mm}{\scalebox{1}{\scriptsize $\mathstrut|\!\lhd$}}}}}
\putnotew{108}{73}{\hyperlink{para4pg6}{\fbox{\Ctab{4.5mm}{\scalebox{1}{\scriptsize $\mathstrut\!\!\lhd\!\!$}}}}}
\putnotew{115}{73}{\hyperlink{para4pg7}{\fbox{\Ctab{4.5mm}{\scalebox{1}{\scriptsize $\mathstrut\!\rhd\!$}}}}}
\putnotew{120}{73}{\hyperlink{para4pg7}{\fbox{\Ctab{2.5mm}{\scalebox{1}{\scriptsize $\mathstrut \!\rhd\!\!|$}}}}}
\putnotew{125}{73}{\hyperlink{para5pg1}{\fbox{\Ctab{2.5mm}{\scalebox{1}{\scriptsize $\mathstrut \!\rhd\!\!||$}}}}}
\putnotee{126}{73}{\scriptsize\color{blue} 7/7}
\end{layer}

\slidepage
{\color{blue}

\begin{layer}{120}{0}
\putnotesw{32}{19}{\large /}
\putnotesw{45}{19}{\large /}
\putnotesw{61}{19}{\large /}
\putnotesw{76}{19}{\large /}
\end{layer}

}
初項$a$,公比$r$,項数が5の場合で説明する
\begin{itemize}
\item
[]$S=a+ar+ar^2+ar^3+ar^4$
\item
[]$rS=ar+ar^2+ar^3+ar^4+ar^5$
\item
[]2つの式を引くと\\
 $S-rS=a-ar^5$\hspace{1zw}\\
$\therefore\hspace{0.75zw}(1-r)S=a(1-r^5)$
\item
[]したがって,$r\neqq 1$のとき $S=\bunsuu{a(1-r^5)}{1-r}$
\end{itemize}

\newslide{等比数列の和の公式}

\vspace*{18mm}

\hypertarget{para5pg1}{}

\begin{layer}{120}{0}
\putnotew{96}{73}{\hyperlink{para4pg7}{\fbox{\Ctab{2.5mm}{\scalebox{1}{\scriptsize $\mathstrut||\!\lhd$}}}}}
\putnotew{101}{73}{\hyperlink{para5pg1}{\fbox{\Ctab{2.5mm}{\scalebox{1}{\scriptsize $\mathstrut|\!\lhd$}}}}}
\putnotew{108}{73}{\hyperlink{para4pg7}{\fbox{\Ctab{4.5mm}{\scalebox{1}{\scriptsize $\mathstrut\!\!\lhd\!\!$}}}}}
\putnotew{115}{73}{\hyperlink{para5pg2}{\fbox{\Ctab{4.5mm}{\scalebox{1}{\scriptsize $\mathstrut\!\rhd\!$}}}}}
\putnotew{120}{73}{\hyperlink{para5pg3}{\fbox{\Ctab{2.5mm}{\scalebox{1}{\scriptsize $\mathstrut \!\rhd\!\!|$}}}}}
\putnotew{125}{73}{\hyperlink{para6pg1}{\fbox{\Ctab{2.5mm}{\scalebox{1}{\scriptsize $\mathstrut \!\rhd\!\!||$}}}}}
\putnotee{126}{73}{\scriptsize\color{blue} 1/3}
\end{layer}

\slidepage
\begin{itemize}
\item
[]初項$a$,公比$r$,項数$n$の等比数列の和$S$は\vspace{2mm}\\
    \fbox{$S=\bunsuu{a(1-r^n)}{1-r}$}
\item
[] ただし,$r\neqq 1$とする
\end{itemize}
%%itemize
%%item::等差$S=\bunsuu{\mbox{項数}\times (\mbox{初項}+\mbox{末項})}{2}$
%%item::等比$S=\bunsuu{\mbox{初項}(1-r^{\normalsize\mbox{項数}})}{1-r}$
%%end
%%%%%%%%%%%%%

%%%%%%%%%%%%%%%%%%%%


\sameslide

\vspace*{18mm}

\hypertarget{para5pg2}{}

\begin{layer}{120}{0}
\putnotew{96}{73}{\hyperlink{para4pg7}{\fbox{\Ctab{2.5mm}{\scalebox{1}{\scriptsize $\mathstrut||\!\lhd$}}}}}
\putnotew{101}{73}{\hyperlink{para5pg1}{\fbox{\Ctab{2.5mm}{\scalebox{1}{\scriptsize $\mathstrut|\!\lhd$}}}}}
\putnotew{108}{73}{\hyperlink{para5pg1}{\fbox{\Ctab{4.5mm}{\scalebox{1}{\scriptsize $\mathstrut\!\!\lhd\!\!$}}}}}
\putnotew{115}{73}{\hyperlink{para5pg3}{\fbox{\Ctab{4.5mm}{\scalebox{1}{\scriptsize $\mathstrut\!\rhd\!$}}}}}
\putnotew{120}{73}{\hyperlink{para5pg3}{\fbox{\Ctab{2.5mm}{\scalebox{1}{\scriptsize $\mathstrut \!\rhd\!\!|$}}}}}
\putnotew{125}{73}{\hyperlink{para6pg1}{\fbox{\Ctab{2.5mm}{\scalebox{1}{\scriptsize $\mathstrut \!\rhd\!\!||$}}}}}
\putnotee{126}{73}{\scriptsize\color{blue} 2/3}
\end{layer}

\slidepage
\begin{itemize}
\item
[]初項$a$,公比$r$,項数$n$の等比数列の和$S$は\vspace{2mm}\\
    \fbox{$S=\bunsuu{a(1-r^n)}{1-r}$}
\item
[] ただし,$r\neqq 1$とする
\item
[][覚え方]\vspace{-2mm}\
\hspace{2zw}$S=\bunsuu{\mbox{初項}(1-r^{\normalsize\mbox{項数}})}{1-r}$\vspace{-0mm}
\end{itemize}

\sameslide

\vspace*{18mm}

\hypertarget{para5pg3}{}

\begin{layer}{120}{0}
\putnotew{96}{73}{\hyperlink{para4pg7}{\fbox{\Ctab{2.5mm}{\scalebox{1}{\scriptsize $\mathstrut||\!\lhd$}}}}}
\putnotew{101}{73}{\hyperlink{para5pg1}{\fbox{\Ctab{2.5mm}{\scalebox{1}{\scriptsize $\mathstrut|\!\lhd$}}}}}
\putnotew{108}{73}{\hyperlink{para5pg2}{\fbox{\Ctab{4.5mm}{\scalebox{1}{\scriptsize $\mathstrut\!\!\lhd\!\!$}}}}}
\putnotew{115}{73}{\hyperlink{para5pg3}{\fbox{\Ctab{4.5mm}{\scalebox{1}{\scriptsize $\mathstrut\!\rhd\!$}}}}}
\putnotew{120}{73}{\hyperlink{para5pg3}{\fbox{\Ctab{2.5mm}{\scalebox{1}{\scriptsize $\mathstrut \!\rhd\!\!|$}}}}}
\putnotew{125}{73}{\hyperlink{para6pg1}{\fbox{\Ctab{2.5mm}{\scalebox{1}{\scriptsize $\mathstrut \!\rhd\!\!||$}}}}}
\putnotee{126}{73}{\scriptsize\color{blue} 3/3}
\end{layer}

\slidepage
\begin{itemize}
\item
[]初項$a$,公比$r$,項数$n$の等比数列の和$S$は\vspace{2mm}\\
    \fbox{$S=\bunsuu{a(1-r^n)}{1-r}$}
\item
[] ただし,$r\neqq 1$とする
\item
[][覚え方]\vspace{-2mm}\
\hspace{2zw}$S=\bunsuu{\mbox{初項}(1-r^{\normalsize\mbox{項数}})}{1-r}$\vspace{-0mm}
\item
[]\hspace{1zw}項数は第$n$項の式から求める\\
\hspace{2zw}$a_n=a r^{n-1}$
\end{itemize}

\newslide{等比数列の和の例題}

\vspace*{18mm}

\hypertarget{para6pg1}{}

\begin{layer}{120}{0}
\putnotew{96}{73}{\hyperlink{para5pg3}{\fbox{\Ctab{2.5mm}{\scalebox{1}{\scriptsize $\mathstrut||\!\lhd$}}}}}
\putnotew{101}{73}{\hyperlink{para6pg1}{\fbox{\Ctab{2.5mm}{\scalebox{1}{\scriptsize $\mathstrut|\!\lhd$}}}}}
\putnotew{108}{73}{\hyperlink{para5pg3}{\fbox{\Ctab{4.5mm}{\scalebox{1}{\scriptsize $\mathstrut\!\!\lhd\!\!$}}}}}
\putnotew{115}{73}{\hyperlink{para6pg2}{\fbox{\Ctab{4.5mm}{\scalebox{1}{\scriptsize $\mathstrut\!\rhd\!$}}}}}
\putnotew{120}{73}{\hyperlink{para6pg8}{\fbox{\Ctab{2.5mm}{\scalebox{1}{\scriptsize $\mathstrut \!\rhd\!\!|$}}}}}
\putnotew{125}{73}{\hyperlink{para7pg1}{\fbox{\Ctab{2.5mm}{\scalebox{1}{\scriptsize $\mathstrut \!\rhd\!\!||$}}}}}
\putnotee{126}{73}{\scriptsize\color{blue} 1/8}
\end{layer}

\slidepage
\begin{itemize}
\item
[例題)]$S=1+2+2^2+2^3+\cdots+2^{10}$を求めよ.
\end{itemize}
%%%%%%%%%%%%%

%%%%%%%%%%%%%%%%%%%%


\sameslide

\vspace*{18mm}

\hypertarget{para6pg2}{}

\begin{layer}{120}{0}
\putnotew{96}{73}{\hyperlink{para5pg3}{\fbox{\Ctab{2.5mm}{\scalebox{1}{\scriptsize $\mathstrut||\!\lhd$}}}}}
\putnotew{101}{73}{\hyperlink{para6pg1}{\fbox{\Ctab{2.5mm}{\scalebox{1}{\scriptsize $\mathstrut|\!\lhd$}}}}}
\putnotew{108}{73}{\hyperlink{para6pg1}{\fbox{\Ctab{4.5mm}{\scalebox{1}{\scriptsize $\mathstrut\!\!\lhd\!\!$}}}}}
\putnotew{115}{73}{\hyperlink{para6pg3}{\fbox{\Ctab{4.5mm}{\scalebox{1}{\scriptsize $\mathstrut\!\rhd\!$}}}}}
\putnotew{120}{73}{\hyperlink{para6pg8}{\fbox{\Ctab{2.5mm}{\scalebox{1}{\scriptsize $\mathstrut \!\rhd\!\!|$}}}}}
\putnotew{125}{73}{\hyperlink{para7pg1}{\fbox{\Ctab{2.5mm}{\scalebox{1}{\scriptsize $\mathstrut \!\rhd\!\!||$}}}}}
\putnotee{126}{73}{\scriptsize\color{blue} 2/8}
\end{layer}

\slidepage
\begin{itemize}
\item
[例題)]$S=1+2+2^2+2^3+\cdots+2^{10}$を求めよ.
\item
[解)]初項$1$,公比$2$より,末項(第$n$項)$a_n$は\\
\end{itemize}

\sameslide

\vspace*{18mm}

\hypertarget{para6pg3}{}

\begin{layer}{120}{0}
\putnotew{96}{73}{\hyperlink{para5pg3}{\fbox{\Ctab{2.5mm}{\scalebox{1}{\scriptsize $\mathstrut||\!\lhd$}}}}}
\putnotew{101}{73}{\hyperlink{para6pg1}{\fbox{\Ctab{2.5mm}{\scalebox{1}{\scriptsize $\mathstrut|\!\lhd$}}}}}
\putnotew{108}{73}{\hyperlink{para6pg2}{\fbox{\Ctab{4.5mm}{\scalebox{1}{\scriptsize $\mathstrut\!\!\lhd\!\!$}}}}}
\putnotew{115}{73}{\hyperlink{para6pg4}{\fbox{\Ctab{4.5mm}{\scalebox{1}{\scriptsize $\mathstrut\!\rhd\!$}}}}}
\putnotew{120}{73}{\hyperlink{para6pg8}{\fbox{\Ctab{2.5mm}{\scalebox{1}{\scriptsize $\mathstrut \!\rhd\!\!|$}}}}}
\putnotew{125}{73}{\hyperlink{para7pg1}{\fbox{\Ctab{2.5mm}{\scalebox{1}{\scriptsize $\mathstrut \!\rhd\!\!||$}}}}}
\putnotee{126}{73}{\scriptsize\color{blue} 3/8}
\end{layer}

\slidepage
\begin{itemize}
\item
[例題)]$S=1+2+2^2+2^3+\cdots+2^{10}$を求めよ.
\item
[解)]初項$1$,公比$2$より,末項(第$n$項)$a_n$は\\
  $a_n=a r^{n-1}=2^{n-1}$
\end{itemize}

\sameslide

\vspace*{18mm}

\hypertarget{para6pg4}{}

\begin{layer}{120}{0}
\putnotew{96}{73}{\hyperlink{para5pg3}{\fbox{\Ctab{2.5mm}{\scalebox{1}{\scriptsize $\mathstrut||\!\lhd$}}}}}
\putnotew{101}{73}{\hyperlink{para6pg1}{\fbox{\Ctab{2.5mm}{\scalebox{1}{\scriptsize $\mathstrut|\!\lhd$}}}}}
\putnotew{108}{73}{\hyperlink{para6pg3}{\fbox{\Ctab{4.5mm}{\scalebox{1}{\scriptsize $\mathstrut\!\!\lhd\!\!$}}}}}
\putnotew{115}{73}{\hyperlink{para6pg5}{\fbox{\Ctab{4.5mm}{\scalebox{1}{\scriptsize $\mathstrut\!\rhd\!$}}}}}
\putnotew{120}{73}{\hyperlink{para6pg8}{\fbox{\Ctab{2.5mm}{\scalebox{1}{\scriptsize $\mathstrut \!\rhd\!\!|$}}}}}
\putnotew{125}{73}{\hyperlink{para7pg1}{\fbox{\Ctab{2.5mm}{\scalebox{1}{\scriptsize $\mathstrut \!\rhd\!\!||$}}}}}
\putnotee{126}{73}{\scriptsize\color{blue} 4/8}
\end{layer}

\slidepage
\begin{itemize}
\item
[例題)]$S=1+2+2^2+2^3+\cdots+2^{10}$を求めよ.
\item
[解)]初項$1$,公比$2$より,末項(第$n$項)$a_n$は\\
  $a_n=a r^{n-1}=2^{n-1}$
\item
[]$a_n=2^{n-1}=2^{10}$より\hspace{0.75zw}$n-1=10$\\
\end{itemize}

\sameslide

\vspace*{18mm}

\hypertarget{para6pg5}{}

\begin{layer}{120}{0}
\putnotew{96}{73}{\hyperlink{para5pg3}{\fbox{\Ctab{2.5mm}{\scalebox{1}{\scriptsize $\mathstrut||\!\lhd$}}}}}
\putnotew{101}{73}{\hyperlink{para6pg1}{\fbox{\Ctab{2.5mm}{\scalebox{1}{\scriptsize $\mathstrut|\!\lhd$}}}}}
\putnotew{108}{73}{\hyperlink{para6pg4}{\fbox{\Ctab{4.5mm}{\scalebox{1}{\scriptsize $\mathstrut\!\!\lhd\!\!$}}}}}
\putnotew{115}{73}{\hyperlink{para6pg6}{\fbox{\Ctab{4.5mm}{\scalebox{1}{\scriptsize $\mathstrut\!\rhd\!$}}}}}
\putnotew{120}{73}{\hyperlink{para6pg8}{\fbox{\Ctab{2.5mm}{\scalebox{1}{\scriptsize $\mathstrut \!\rhd\!\!|$}}}}}
\putnotew{125}{73}{\hyperlink{para7pg1}{\fbox{\Ctab{2.5mm}{\scalebox{1}{\scriptsize $\mathstrut \!\rhd\!\!||$}}}}}
\putnotee{126}{73}{\scriptsize\color{blue} 5/8}
\end{layer}

\slidepage
\begin{itemize}
\item
[例題)]$S=1+2+2^2+2^3+\cdots+2^{10}$を求めよ.
\item
[解)]初項$1$,公比$2$より,末項(第$n$項)$a_n$は\\
  $a_n=a r^{n-1}=2^{n-1}$
\item
[]$a_n=2^{n-1}=2^{10}$より\hspace{0.75zw}$n-1=10$\\
  $\therefore\hspace{0.75zw}n=11$
\end{itemize}

\sameslide

\vspace*{18mm}

\hypertarget{para6pg6}{}

\begin{layer}{120}{0}
\putnotew{96}{73}{\hyperlink{para5pg3}{\fbox{\Ctab{2.5mm}{\scalebox{1}{\scriptsize $\mathstrut||\!\lhd$}}}}}
\putnotew{101}{73}{\hyperlink{para6pg1}{\fbox{\Ctab{2.5mm}{\scalebox{1}{\scriptsize $\mathstrut|\!\lhd$}}}}}
\putnotew{108}{73}{\hyperlink{para6pg5}{\fbox{\Ctab{4.5mm}{\scalebox{1}{\scriptsize $\mathstrut\!\!\lhd\!\!$}}}}}
\putnotew{115}{73}{\hyperlink{para6pg7}{\fbox{\Ctab{4.5mm}{\scalebox{1}{\scriptsize $\mathstrut\!\rhd\!$}}}}}
\putnotew{120}{73}{\hyperlink{para6pg8}{\fbox{\Ctab{2.5mm}{\scalebox{1}{\scriptsize $\mathstrut \!\rhd\!\!|$}}}}}
\putnotew{125}{73}{\hyperlink{para7pg1}{\fbox{\Ctab{2.5mm}{\scalebox{1}{\scriptsize $\mathstrut \!\rhd\!\!||$}}}}}
\putnotee{126}{73}{\scriptsize\color{blue} 6/8}
\end{layer}

\slidepage
\begin{itemize}
\item
[例題)]$S=1+2+2^2+2^3+\cdots+2^{10}$を求めよ.
\item
[解)]初項$1$,公比$2$より,末項(第$n$項)$a_n$は\\
  $a_n=a r^{n-1}=2^{n-1}$
\item
[]$a_n=2^{n-1}=2^{10}$より\hspace{0.75zw}$n-1=10$\\
  $\therefore\hspace{0.75zw}n=11$
\item
[]$S=\bunsuu{1\times(1-2^{11})}{1-2}$
\end{itemize}

\sameslide

\vspace*{18mm}

\hypertarget{para6pg7}{}

\begin{layer}{120}{0}
\putnotew{96}{73}{\hyperlink{para5pg3}{\fbox{\Ctab{2.5mm}{\scalebox{1}{\scriptsize $\mathstrut||\!\lhd$}}}}}
\putnotew{101}{73}{\hyperlink{para6pg1}{\fbox{\Ctab{2.5mm}{\scalebox{1}{\scriptsize $\mathstrut|\!\lhd$}}}}}
\putnotew{108}{73}{\hyperlink{para6pg6}{\fbox{\Ctab{4.5mm}{\scalebox{1}{\scriptsize $\mathstrut\!\!\lhd\!\!$}}}}}
\putnotew{115}{73}{\hyperlink{para6pg8}{\fbox{\Ctab{4.5mm}{\scalebox{1}{\scriptsize $\mathstrut\!\rhd\!$}}}}}
\putnotew{120}{73}{\hyperlink{para6pg8}{\fbox{\Ctab{2.5mm}{\scalebox{1}{\scriptsize $\mathstrut \!\rhd\!\!|$}}}}}
\putnotew{125}{73}{\hyperlink{para7pg1}{\fbox{\Ctab{2.5mm}{\scalebox{1}{\scriptsize $\mathstrut \!\rhd\!\!||$}}}}}
\putnotee{126}{73}{\scriptsize\color{blue} 7/8}
\end{layer}

\slidepage
\begin{itemize}
\item
[例題)]$S=1+2+2^2+2^3+\cdots+2^{10}$を求めよ.
\item
[解)]初項$1$,公比$2$より,末項(第$n$項)$a_n$は\\
  $a_n=a r^{n-1}=2^{n-1}$
\item
[]$a_n=2^{n-1}=2^{10}$より\hspace{0.75zw}$n-1=10$\\
  $\therefore\hspace{0.75zw}n=11$
\item
[]$S=\bunsuu{1\times(1-2^{11})}{1-2}$
$=\bunsuu{1-2^{11}}{-1}$
\end{itemize}

\sameslide

\vspace*{18mm}

\hypertarget{para6pg8}{}

\begin{layer}{120}{0}
\putnotew{96}{73}{\hyperlink{para5pg3}{\fbox{\Ctab{2.5mm}{\scalebox{1}{\scriptsize $\mathstrut||\!\lhd$}}}}}
\putnotew{101}{73}{\hyperlink{para6pg1}{\fbox{\Ctab{2.5mm}{\scalebox{1}{\scriptsize $\mathstrut|\!\lhd$}}}}}
\putnotew{108}{73}{\hyperlink{para6pg7}{\fbox{\Ctab{4.5mm}{\scalebox{1}{\scriptsize $\mathstrut\!\!\lhd\!\!$}}}}}
\putnotew{115}{73}{\hyperlink{para6pg8}{\fbox{\Ctab{4.5mm}{\scalebox{1}{\scriptsize $\mathstrut\!\rhd\!$}}}}}
\putnotew{120}{73}{\hyperlink{para6pg8}{\fbox{\Ctab{2.5mm}{\scalebox{1}{\scriptsize $\mathstrut \!\rhd\!\!|$}}}}}
\putnotew{125}{73}{\hyperlink{para7pg1}{\fbox{\Ctab{2.5mm}{\scalebox{1}{\scriptsize $\mathstrut \!\rhd\!\!||$}}}}}
\putnotee{126}{73}{\scriptsize\color{blue} 8/8}
\end{layer}

\slidepage
\begin{itemize}
\item
[例題)]$S=1+2+2^2+2^3+\cdots+2^{10}$を求めよ.
\item
[解)]初項$1$,公比$2$より,末項(第$n$項)$a_n$は\\
  $a_n=a r^{n-1}=2^{n-1}$
\item
[]$a_n=2^{n-1}=2^{10}$より\hspace{0.75zw}$n-1=10$\\
  $\therefore\hspace{0.75zw}n=11$
\item
[]$S=\bunsuu{1\times(1-2^{11})}{1-2}$
$=\bunsuu{1-2^{11}}{-1}$
$=2^{11}-1$
\end{itemize}

\newslide{課題(等差数列と等比数列の和)}

\vspace*{18mm}

\hypertarget{para7pg1}{}

\begin{layer}{120}{0}
\putnotew{96}{73}{\hyperlink{para6pg8}{\fbox{\Ctab{2.5mm}{\scalebox{1}{\scriptsize $\mathstrut||\!\lhd$}}}}}
\putnotew{125}{73}{\hyperlink{para8pg1}{\fbox{\Ctab{2.5mm}{\scalebox{1}{\scriptsize $\mathstrut \!\rhd\!\!||$}}}}}
\putnotee{126}{73}{\scriptsize\color{blue} 1/1}
\end{layer}

\slidepage
\seteda{70}
\begin{itemize}
\item
[課題]\monban 次の和を求めよ.(TextP202,204)\vspace{2mm}\\
\eda{初項3,\ 末項19,\ 項数15の等差数列の和}\vspace{2mm}\\
\eda{初項3,\ 公比2の等比数列の初項から第$n$項までの和}
\end{itemize}
%%%%%%%%%%%%%

%%%%%%%%%%%%%%%%%%%%


\newslide{和の記号$\dsum$(シグマ)}

\vspace*{18mm}

\hypertarget{para8pg1}{}

\begin{layer}{120}{0}
\putnotew{96}{73}{\hyperlink{para7pg1}{\fbox{\Ctab{2.5mm}{\scalebox{1}{\scriptsize $\mathstrut||\!\lhd$}}}}}
\putnotew{101}{73}{\hyperlink{para8pg1}{\fbox{\Ctab{2.5mm}{\scalebox{1}{\scriptsize $\mathstrut|\!\lhd$}}}}}
\putnotew{108}{73}{\hyperlink{para7pg1}{\fbox{\Ctab{4.5mm}{\scalebox{1}{\scriptsize $\mathstrut\!\!\lhd\!\!$}}}}}
\putnotew{115}{73}{\hyperlink{para8pg2}{\fbox{\Ctab{4.5mm}{\scalebox{1}{\scriptsize $\mathstrut\!\rhd\!$}}}}}
\putnotew{120}{73}{\hyperlink{para8pg8}{\fbox{\Ctab{2.5mm}{\scalebox{1}{\scriptsize $\mathstrut \!\rhd\!\!|$}}}}}
\putnotew{125}{73}{\hyperlink{para9pg1}{\fbox{\Ctab{2.5mm}{\scalebox{1}{\scriptsize $\mathstrut \!\rhd\!\!||$}}}}}
\putnotee{126}{73}{\scriptsize\color{blue} 1/8}
\end{layer}

\slidepage
{\color{blue}

\begin{layer}{120}{0}
\end{layer}

}
\begin{itemize}
\item
[例)]$1,\ 2,\ 3,\ \cdots,\ 10$の和を表す
\end{itemize}
%%%%%%%%%%%%%

%%%%%%%%%%%%%%%%%%%%


\sameslide

\vspace*{18mm}

\hypertarget{para8pg2}{}

\begin{layer}{120}{0}
\putnotew{96}{73}{\hyperlink{para7pg1}{\fbox{\Ctab{2.5mm}{\scalebox{1}{\scriptsize $\mathstrut||\!\lhd$}}}}}
\putnotew{101}{73}{\hyperlink{para8pg1}{\fbox{\Ctab{2.5mm}{\scalebox{1}{\scriptsize $\mathstrut|\!\lhd$}}}}}
\putnotew{108}{73}{\hyperlink{para8pg1}{\fbox{\Ctab{4.5mm}{\scalebox{1}{\scriptsize $\mathstrut\!\!\lhd\!\!$}}}}}
\putnotew{115}{73}{\hyperlink{para8pg3}{\fbox{\Ctab{4.5mm}{\scalebox{1}{\scriptsize $\mathstrut\!\rhd\!$}}}}}
\putnotew{120}{73}{\hyperlink{para8pg8}{\fbox{\Ctab{2.5mm}{\scalebox{1}{\scriptsize $\mathstrut \!\rhd\!\!|$}}}}}
\putnotew{125}{73}{\hyperlink{para9pg1}{\fbox{\Ctab{2.5mm}{\scalebox{1}{\scriptsize $\mathstrut \!\rhd\!\!||$}}}}}
\putnotee{126}{73}{\scriptsize\color{blue} 2/8}
\end{layer}

\slidepage
{\color{blue}

\begin{layer}{120}{0}
\end{layer}

}
\begin{itemize}
\item
[例)]$1,\ 2,\ 3,\ \cdots,\ 10$の和を表す
\item
[(1)]$S=1+2+3+4+5+6+7+8+9+10$
\end{itemize}

\sameslide

\vspace*{18mm}

\hypertarget{para8pg3}{}

\begin{layer}{120}{0}
\putnotew{96}{73}{\hyperlink{para7pg1}{\fbox{\Ctab{2.5mm}{\scalebox{1}{\scriptsize $\mathstrut||\!\lhd$}}}}}
\putnotew{101}{73}{\hyperlink{para8pg1}{\fbox{\Ctab{2.5mm}{\scalebox{1}{\scriptsize $\mathstrut|\!\lhd$}}}}}
\putnotew{108}{73}{\hyperlink{para8pg2}{\fbox{\Ctab{4.5mm}{\scalebox{1}{\scriptsize $\mathstrut\!\!\lhd\!\!$}}}}}
\putnotew{115}{73}{\hyperlink{para8pg4}{\fbox{\Ctab{4.5mm}{\scalebox{1}{\scriptsize $\mathstrut\!\rhd\!$}}}}}
\putnotew{120}{73}{\hyperlink{para8pg8}{\fbox{\Ctab{2.5mm}{\scalebox{1}{\scriptsize $\mathstrut \!\rhd\!\!|$}}}}}
\putnotew{125}{73}{\hyperlink{para9pg1}{\fbox{\Ctab{2.5mm}{\scalebox{1}{\scriptsize $\mathstrut \!\rhd\!\!||$}}}}}
\putnotee{126}{73}{\scriptsize\color{blue} 3/8}
\end{layer}

\slidepage
{\color{blue}

\begin{layer}{120}{0}
\end{layer}

}
\begin{itemize}
\item
[例)]$1,\ 2,\ 3,\ \cdots,\ 10$の和を表す
\item
[(1)]$S=1+2+3+4+5+6+7+8+9+10$
\item
[(2)]$S=1+2+3+\cdots+9+10$
\end{itemize}

\sameslide

\vspace*{18mm}

\hypertarget{para8pg4}{}

\begin{layer}{120}{0}
\putnotew{96}{73}{\hyperlink{para7pg1}{\fbox{\Ctab{2.5mm}{\scalebox{1}{\scriptsize $\mathstrut||\!\lhd$}}}}}
\putnotew{101}{73}{\hyperlink{para8pg1}{\fbox{\Ctab{2.5mm}{\scalebox{1}{\scriptsize $\mathstrut|\!\lhd$}}}}}
\putnotew{108}{73}{\hyperlink{para8pg3}{\fbox{\Ctab{4.5mm}{\scalebox{1}{\scriptsize $\mathstrut\!\!\lhd\!\!$}}}}}
\putnotew{115}{73}{\hyperlink{para8pg5}{\fbox{\Ctab{4.5mm}{\scalebox{1}{\scriptsize $\mathstrut\!\rhd\!$}}}}}
\putnotew{120}{73}{\hyperlink{para8pg8}{\fbox{\Ctab{2.5mm}{\scalebox{1}{\scriptsize $\mathstrut \!\rhd\!\!|$}}}}}
\putnotew{125}{73}{\hyperlink{para9pg1}{\fbox{\Ctab{2.5mm}{\scalebox{1}{\scriptsize $\mathstrut \!\rhd\!\!||$}}}}}
\putnotee{126}{73}{\scriptsize\color{blue} 4/8}
\end{layer}

\slidepage
{\color{blue}

\begin{layer}{120}{0}
\end{layer}

}
\begin{itemize}
\item
[例)]$1,\ 2,\ 3,\ \cdots,\ 10$の和を表す
\item
[(1)]$S=1+2+3+4+5+6+7+8+9+10$
\item
[(2)]$S=1+2+3+\cdots+9+10$
\item
[(3)]第$k$項は $k$\\
\end{itemize}

\sameslide

\vspace*{18mm}

\hypertarget{para8pg5}{}

\begin{layer}{120}{0}
\putnotew{96}{73}{\hyperlink{para7pg1}{\fbox{\Ctab{2.5mm}{\scalebox{1}{\scriptsize $\mathstrut||\!\lhd$}}}}}
\putnotew{101}{73}{\hyperlink{para8pg1}{\fbox{\Ctab{2.5mm}{\scalebox{1}{\scriptsize $\mathstrut|\!\lhd$}}}}}
\putnotew{108}{73}{\hyperlink{para8pg4}{\fbox{\Ctab{4.5mm}{\scalebox{1}{\scriptsize $\mathstrut\!\!\lhd\!\!$}}}}}
\putnotew{115}{73}{\hyperlink{para8pg6}{\fbox{\Ctab{4.5mm}{\scalebox{1}{\scriptsize $\mathstrut\!\rhd\!$}}}}}
\putnotew{120}{73}{\hyperlink{para8pg8}{\fbox{\Ctab{2.5mm}{\scalebox{1}{\scriptsize $\mathstrut \!\rhd\!\!|$}}}}}
\putnotew{125}{73}{\hyperlink{para9pg1}{\fbox{\Ctab{2.5mm}{\scalebox{1}{\scriptsize $\mathstrut \!\rhd\!\!||$}}}}}
\putnotee{126}{73}{\scriptsize\color{blue} 5/8}
\end{layer}

\slidepage
{\color{blue}

\begin{layer}{120}{0}
\end{layer}

}
\begin{itemize}
\item
[例)]$1,\ 2,\ 3,\ \cdots,\ 10$の和を表す
\item
[(1)]$S=1+2+3+4+5+6+7+8+9+10$
\item
[(2)]$S=1+2+3+\cdots+9+10$
\item
[(3)]第$k$項は $k$\\
 $k=1,\ 2,\ 3,\ \cdots, 10$\\
\end{itemize}

\sameslide

\vspace*{18mm}

\hypertarget{para8pg6}{}

\begin{layer}{120}{0}
\putnotew{96}{73}{\hyperlink{para7pg1}{\fbox{\Ctab{2.5mm}{\scalebox{1}{\scriptsize $\mathstrut||\!\lhd$}}}}}
\putnotew{101}{73}{\hyperlink{para8pg1}{\fbox{\Ctab{2.5mm}{\scalebox{1}{\scriptsize $\mathstrut|\!\lhd$}}}}}
\putnotew{108}{73}{\hyperlink{para8pg5}{\fbox{\Ctab{4.5mm}{\scalebox{1}{\scriptsize $\mathstrut\!\!\lhd\!\!$}}}}}
\putnotew{115}{73}{\hyperlink{para8pg7}{\fbox{\Ctab{4.5mm}{\scalebox{1}{\scriptsize $\mathstrut\!\rhd\!$}}}}}
\putnotew{120}{73}{\hyperlink{para8pg8}{\fbox{\Ctab{2.5mm}{\scalebox{1}{\scriptsize $\mathstrut \!\rhd\!\!|$}}}}}
\putnotew{125}{73}{\hyperlink{para9pg1}{\fbox{\Ctab{2.5mm}{\scalebox{1}{\scriptsize $\mathstrut \!\rhd\!\!||$}}}}}
\putnotee{126}{73}{\scriptsize\color{blue} 6/8}
\end{layer}

\slidepage
{\color{blue}

\begin{layer}{120}{0}
\end{layer}

}
\begin{itemize}
\item
[例)]$1,\ 2,\ 3,\ \cdots,\ 10$の和を表す
\item
[(1)]$S=1+2+3+4+5+6+7+8+9+10$
\item
[(2)]$S=1+2+3+\cdots+9+10$
\item
[(3)]第$k$項は $k$\\
 $k=1,\ 2,\ 3,\ \cdots, 10$\\
  $S=\dsum_{k=1}^{10}k$\\
\end{itemize}

\sameslide

\vspace*{18mm}

\hypertarget{para8pg7}{}

\begin{layer}{120}{0}
\putnotew{96}{73}{\hyperlink{para7pg1}{\fbox{\Ctab{2.5mm}{\scalebox{1}{\scriptsize $\mathstrut||\!\lhd$}}}}}
\putnotew{101}{73}{\hyperlink{para8pg1}{\fbox{\Ctab{2.5mm}{\scalebox{1}{\scriptsize $\mathstrut|\!\lhd$}}}}}
\putnotew{108}{73}{\hyperlink{para8pg6}{\fbox{\Ctab{4.5mm}{\scalebox{1}{\scriptsize $\mathstrut\!\!\lhd\!\!$}}}}}
\putnotew{115}{73}{\hyperlink{para8pg8}{\fbox{\Ctab{4.5mm}{\scalebox{1}{\scriptsize $\mathstrut\!\rhd\!$}}}}}
\putnotew{120}{73}{\hyperlink{para8pg8}{\fbox{\Ctab{2.5mm}{\scalebox{1}{\scriptsize $\mathstrut \!\rhd\!\!|$}}}}}
\putnotew{125}{73}{\hyperlink{para9pg1}{\fbox{\Ctab{2.5mm}{\scalebox{1}{\scriptsize $\mathstrut \!\rhd\!\!||$}}}}}
\putnotee{126}{73}{\scriptsize\color{blue} 7/8}
\end{layer}

\slidepage
{\color{blue}

\begin{layer}{120}{0}
\putnotee{5}{69}{\normalsize $k$に$1,2,\cdots,10$を順に入れて加えるという意味}
\end{layer}

}
\begin{itemize}
\item
[例)]$1,\ 2,\ 3,\ \cdots,\ 10$の和を表す
\item
[(1)]$S=1+2+3+4+5+6+7+8+9+10$
\item
[(2)]$S=1+2+3+\cdots+9+10$
\item
[(3)]第$k$項は $k$\\
 $k=1,\ 2,\ 3,\ \cdots, 10$\\
  $S=\dsum_{k=1}^{10}k$\\
\end{itemize}

\sameslide

\vspace*{18mm}

\hypertarget{para8pg8}{}

\begin{layer}{120}{0}
\putnotew{96}{73}{\hyperlink{para7pg1}{\fbox{\Ctab{2.5mm}{\scalebox{1}{\scriptsize $\mathstrut||\!\lhd$}}}}}
\putnotew{101}{73}{\hyperlink{para8pg1}{\fbox{\Ctab{2.5mm}{\scalebox{1}{\scriptsize $\mathstrut|\!\lhd$}}}}}
\putnotew{108}{73}{\hyperlink{para8pg7}{\fbox{\Ctab{4.5mm}{\scalebox{1}{\scriptsize $\mathstrut\!\!\lhd\!\!$}}}}}
\putnotew{115}{73}{\hyperlink{para8pg8}{\fbox{\Ctab{4.5mm}{\scalebox{1}{\scriptsize $\mathstrut\!\rhd\!$}}}}}
\putnotew{120}{73}{\hyperlink{para8pg8}{\fbox{\Ctab{2.5mm}{\scalebox{1}{\scriptsize $\mathstrut \!\rhd\!\!|$}}}}}
\putnotew{125}{73}{\hyperlink{para9pg1}{\fbox{\Ctab{2.5mm}{\scalebox{1}{\scriptsize $\mathstrut \!\rhd\!\!||$}}}}}
\putnotee{126}{73}{\scriptsize\color{blue} 8/8}
\end{layer}

\slidepage
{\color{blue}

\begin{layer}{120}{0}
\putnotee{5}{69}{\normalsize $k$に$1,2,\cdots,10$を順に入れて加えるという意味}
\putnotee{60}{60}{\normalsize KeTMath sum(k=1,10,k)}
\end{layer}

}
\begin{itemize}
\item
[例)]$1,\ 2,\ 3,\ \cdots,\ 10$の和を表す
\item
[(1)]$S=1+2+3+4+5+6+7+8+9+10$
\item
[(2)]$S=1+2+3+\cdots+9+10$
\item
[(3)]第$k$項は $k$\\
 $k=1,\ 2,\ 3,\ \cdots, 10$\\
  $S=\dsum_{k=1}^{10}k$\\
\end{itemize}

\newslide{$\dsum$の使い方}

\vspace*{18mm}

\slidepage
\vspace{2mm}

\begin{itemize}
\item
[例1)]$\dsum_{k=1}^{5}(2k+1)=\hakom{3+5+7+9+11}$\vspace{-2mm}
\end{itemize}
%%%%%%%%%%%%%

%%%%%%%%%%%%%%%%%%%%


\sameslide

\vspace*{18mm}

\slidepage
\vspace{2mm}

\begin{itemize}
\item
[例1)]$\dsum_{k=1}^{5}(2k+1)=\hakoma{3+5+7+9+11}$\vspace{-5mm}
\end{itemize}

\sameslide

\vspace*{18mm}

\slidepage
\vspace{2mm}

\begin{itemize}
\item
[例1)]$\dsum_{k=1}^{5}(2k+1)=\hakoma{3+5+7+9+11}$\vspace{-5mm}
\item
[例2)]$1^2+2^2+3^2+\cdots+20^2=\dsum_{k=\scb{\hakom{1}}}^{\scb{\hakom{20}}}\hakom{k^2}$\vspace{-1mm}
\end{itemize}

\sameslide

\vspace*{18mm}

\slidepage
\vspace{2mm}

\begin{itemize}
\item
[例1)]$\dsum_{k=1}^{5}(2k+1)=\hakoma{3+5+7+9+11}$\vspace{-5mm}
\item
[例2)]$1^2+2^2+3^2+\cdots+20^2=\dsum_{k=\scb{\hakom{1}}}^{\scb{\hakom{20}}}\hakom{k^2}$\vspace{-1mm}
\end{itemize}

\sameslide

\vspace*{18mm}

\slidepage
\vspace{2mm}

\begin{itemize}
\item
[例1)]$\dsum_{k=1}^{5}(2k+1)=\hakoma{3+5+7+9+11}$\vspace{-5mm}
\item
[例2)]$1^2+2^2+3^2+\cdots+20^2=\dsum_{k=\scb{\hakom{1}}}^{\scb{\hakom{20}}}\hakom{k^2}$\vspace{-1mm}
\\  第$k$項は \hakom{k^2},$k=\hakom{1,\ 2,\ \cdots 20}$
\end{itemize}

\sameslide

\vspace*{18mm}

\slidepage
\vspace{2mm}

\begin{itemize}
\item
[例1)]$\dsum_{k=1}^{5}(2k+1)=\hakoma{3+5+7+9+11}$\vspace{-5mm}
\item
[例2)]$1^2+2^2+3^2+\cdots+20^2=\dsum_{k=\scb{\hakom{1}}}^{\scb{\hakom{20}}}\hakom{k^2}$\vspace{-1mm}
\\  第$k$項は \hakoma{k^2},$k=\hakoma{1,\ 2,\ \cdots,\ 20}$\vspace{-2mm}
\end{itemize}

\sameslide

\vspace*{18mm}

\slidepage
\vspace{2mm}

\begin{itemize}
\item
[例1)]$\dsum_{k=1}^{5}(2k+1)=\hakoma{3+5+7+9+11}$\vspace{-5mm}
\item
[例2)]$1^2+2^2+3^2+\cdots+20^2=\dsum_{k=\scb{\hakoma{1}}}^{\scb{\hakoma{20}}}\hakoma{k^2}$
\\  第$k$項は \hakoma{k^2},$k=\hakoma{1,\ 2,\ \cdots 20}$\vspace{-1mm}
\end{itemize}

\sameslide

\vspace*{18mm}

\slidepage
\vspace{2mm}

\begin{itemize}
\item
[例1)]$\dsum_{k=1}^{5}(2k+1)=\hakoma{3+5+7+9+11}$\vspace{-5mm}
\item
[例2)]$1^2+2^2+3^2+\cdots+20^2=\dsum_{k=\scb{\hakoma{1}}}^{\scb{\hakoma{20}}}\hakoma{k^2}$
\\  第$k$項は \hakoma{k^2},$k=\hakoma{1,\ 2,\ \cdots 20}$\vspace{-1mm}
\item
[課題]\monban $S, n, a_k$を求めよ.\seteda{40}\\
\eda{$S=\dsum_{k=1}^{4}\bunsuu{1}{k}$}
\eda{$2+4+6+\cdots+10=\dsum_{k=1}^{n}a_k$}
\end{itemize}

\mainslide{複素数}


\slidepage[m]
%%%%%%%%%%%%%
%%%%%%%%%%%%%

%%%%%%%%%%%%%%%%%%%%

\newslide{虚数}

\vspace*{18mm}

\hypertarget{para9pg1}{}

\begin{layer}{120}{0}
\putnotew{96}{73}{\hyperlink{para8pg8}{\fbox{\Ctab{2.5mm}{\scalebox{1}{\scriptsize $\mathstrut||\!\lhd$}}}}}
\putnotew{101}{73}{\hyperlink{para9pg1}{\fbox{\Ctab{2.5mm}{\scalebox{1}{\scriptsize $\mathstrut|\!\lhd$}}}}}
\putnotew{108}{73}{\hyperlink{para8pg8}{\fbox{\Ctab{4.5mm}{\scalebox{1}{\scriptsize $\mathstrut\!\!\lhd\!\!$}}}}}
\putnotew{115}{73}{\hyperlink{para9pg2}{\fbox{\Ctab{4.5mm}{\scalebox{1}{\scriptsize $\mathstrut\!\rhd\!$}}}}}
\putnotew{120}{73}{\hyperlink{para9pg10}{\fbox{\Ctab{2.5mm}{\scalebox{1}{\scriptsize $\mathstrut \!\rhd\!\!|$}}}}}
\putnotew{125}{73}{\hyperlink{para10pg1}{\fbox{\Ctab{2.5mm}{\scalebox{1}{\scriptsize $\mathstrut \!\rhd\!\!||$}}}}}
\putnotee{126}{73}{\scriptsize\color{blue} 1/10}
\end{layer}

\slidepage

\begin{layer}{120}{0}
\end{layer}

\begin{itemize}
\item
$x^2+6x+10=0$\hfill(1)\\
\end{itemize}
%%%%%%%%%%%%%

%%%%%%%%%%%%%%%%%%%%


\sameslide

\vspace*{18mm}

\hypertarget{para9pg2}{}

\begin{layer}{120}{0}
\putnotew{96}{73}{\hyperlink{para8pg8}{\fbox{\Ctab{2.5mm}{\scalebox{1}{\scriptsize $\mathstrut||\!\lhd$}}}}}
\putnotew{101}{73}{\hyperlink{para9pg1}{\fbox{\Ctab{2.5mm}{\scalebox{1}{\scriptsize $\mathstrut|\!\lhd$}}}}}
\putnotew{108}{73}{\hyperlink{para9pg1}{\fbox{\Ctab{4.5mm}{\scalebox{1}{\scriptsize $\mathstrut\!\!\lhd\!\!$}}}}}
\putnotew{115}{73}{\hyperlink{para9pg3}{\fbox{\Ctab{4.5mm}{\scalebox{1}{\scriptsize $\mathstrut\!\rhd\!$}}}}}
\putnotew{120}{73}{\hyperlink{para9pg10}{\fbox{\Ctab{2.5mm}{\scalebox{1}{\scriptsize $\mathstrut \!\rhd\!\!|$}}}}}
\putnotew{125}{73}{\hyperlink{para10pg1}{\fbox{\Ctab{2.5mm}{\scalebox{1}{\scriptsize $\mathstrut \!\rhd\!\!||$}}}}}
\putnotee{126}{73}{\scriptsize\color{blue} 2/10}
\end{layer}

\slidepage

\begin{layer}{120}{0}
\end{layer}

\begin{itemize}
\item
$x^2+6x+10=0$\hfill(1)\\
$(x+3)^2-9+10=0$より
\end{itemize}

\sameslide

\vspace*{18mm}

\hypertarget{para9pg3}{}

\begin{layer}{120}{0}
\putnotew{96}{73}{\hyperlink{para8pg8}{\fbox{\Ctab{2.5mm}{\scalebox{1}{\scriptsize $\mathstrut||\!\lhd$}}}}}
\putnotew{101}{73}{\hyperlink{para9pg1}{\fbox{\Ctab{2.5mm}{\scalebox{1}{\scriptsize $\mathstrut|\!\lhd$}}}}}
\putnotew{108}{73}{\hyperlink{para9pg2}{\fbox{\Ctab{4.5mm}{\scalebox{1}{\scriptsize $\mathstrut\!\!\lhd\!\!$}}}}}
\putnotew{115}{73}{\hyperlink{para9pg4}{\fbox{\Ctab{4.5mm}{\scalebox{1}{\scriptsize $\mathstrut\!\rhd\!$}}}}}
\putnotew{120}{73}{\hyperlink{para9pg10}{\fbox{\Ctab{2.5mm}{\scalebox{1}{\scriptsize $\mathstrut \!\rhd\!\!|$}}}}}
\putnotew{125}{73}{\hyperlink{para10pg1}{\fbox{\Ctab{2.5mm}{\scalebox{1}{\scriptsize $\mathstrut \!\rhd\!\!||$}}}}}
\putnotee{126}{73}{\scriptsize\color{blue} 3/10}
\end{layer}

\slidepage

\begin{layer}{120}{0}
\end{layer}

\begin{itemize}
\item
$x^2+6x+10=0$\hfill(1)\\
$(x+3)^2-9+10=0$より
 $(x+3)^2=-1$\hfill(2)\vspace{-1mm}
\end{itemize}

\sameslide

\vspace*{18mm}

\hypertarget{para9pg4}{}

\begin{layer}{120}{0}
\putnotew{96}{73}{\hyperlink{para8pg8}{\fbox{\Ctab{2.5mm}{\scalebox{1}{\scriptsize $\mathstrut||\!\lhd$}}}}}
\putnotew{101}{73}{\hyperlink{para9pg1}{\fbox{\Ctab{2.5mm}{\scalebox{1}{\scriptsize $\mathstrut|\!\lhd$}}}}}
\putnotew{108}{73}{\hyperlink{para9pg3}{\fbox{\Ctab{4.5mm}{\scalebox{1}{\scriptsize $\mathstrut\!\!\lhd\!\!$}}}}}
\putnotew{115}{73}{\hyperlink{para9pg5}{\fbox{\Ctab{4.5mm}{\scalebox{1}{\scriptsize $\mathstrut\!\rhd\!$}}}}}
\putnotew{120}{73}{\hyperlink{para9pg10}{\fbox{\Ctab{2.5mm}{\scalebox{1}{\scriptsize $\mathstrut \!\rhd\!\!|$}}}}}
\putnotew{125}{73}{\hyperlink{para10pg1}{\fbox{\Ctab{2.5mm}{\scalebox{1}{\scriptsize $\mathstrut \!\rhd\!\!||$}}}}}
\putnotee{126}{73}{\scriptsize\color{blue} 4/10}
\end{layer}

\slidepage

\begin{layer}{120}{0}
\end{layer}

\begin{itemize}
\item
$x^2+6x+10=0$\hfill(1)\\
$(x+3)^2-9+10=0$より
 $(x+3)^2=-1$\hfill(2)\vspace{-1mm}
\item
実数では,2乗して$-1$になることはない\vspace{-1mm}
\end{itemize}

\sameslide

\vspace*{18mm}

\hypertarget{para9pg5}{}

\begin{layer}{120}{0}
\putnotew{96}{73}{\hyperlink{para8pg8}{\fbox{\Ctab{2.5mm}{\scalebox{1}{\scriptsize $\mathstrut||\!\lhd$}}}}}
\putnotew{101}{73}{\hyperlink{para9pg1}{\fbox{\Ctab{2.5mm}{\scalebox{1}{\scriptsize $\mathstrut|\!\lhd$}}}}}
\putnotew{108}{73}{\hyperlink{para9pg4}{\fbox{\Ctab{4.5mm}{\scalebox{1}{\scriptsize $\mathstrut\!\!\lhd\!\!$}}}}}
\putnotew{115}{73}{\hyperlink{para9pg6}{\fbox{\Ctab{4.5mm}{\scalebox{1}{\scriptsize $\mathstrut\!\rhd\!$}}}}}
\putnotew{120}{73}{\hyperlink{para9pg10}{\fbox{\Ctab{2.5mm}{\scalebox{1}{\scriptsize $\mathstrut \!\rhd\!\!|$}}}}}
\putnotew{125}{73}{\hyperlink{para10pg1}{\fbox{\Ctab{2.5mm}{\scalebox{1}{\scriptsize $\mathstrut \!\rhd\!\!||$}}}}}
\putnotee{126}{73}{\scriptsize\color{blue} 5/10}
\end{layer}

\slidepage

\begin{layer}{120}{0}
\end{layer}

\begin{itemize}
\item
$x^2+6x+10=0$\hfill(1)\\
$(x+3)^2-9+10=0$より
 $(x+3)^2=-1$\hfill(2)\vspace{-1mm}
\item
実数では,2乗して$-1$になることはない\vspace{-1mm}
\item
2乗して$-1$になるものも数と考え,{\color{red}$i$}とおく({\color{red}虚数単位})\vspace{-1mm}
\end{itemize}

\sameslide

\vspace*{18mm}

\hypertarget{para9pg6}{}

\begin{layer}{120}{0}
\putnotew{96}{73}{\hyperlink{para8pg8}{\fbox{\Ctab{2.5mm}{\scalebox{1}{\scriptsize $\mathstrut||\!\lhd$}}}}}
\putnotew{101}{73}{\hyperlink{para9pg1}{\fbox{\Ctab{2.5mm}{\scalebox{1}{\scriptsize $\mathstrut|\!\lhd$}}}}}
\putnotew{108}{73}{\hyperlink{para9pg5}{\fbox{\Ctab{4.5mm}{\scalebox{1}{\scriptsize $\mathstrut\!\!\lhd\!\!$}}}}}
\putnotew{115}{73}{\hyperlink{para9pg7}{\fbox{\Ctab{4.5mm}{\scalebox{1}{\scriptsize $\mathstrut\!\rhd\!$}}}}}
\putnotew{120}{73}{\hyperlink{para9pg10}{\fbox{\Ctab{2.5mm}{\scalebox{1}{\scriptsize $\mathstrut \!\rhd\!\!|$}}}}}
\putnotew{125}{73}{\hyperlink{para10pg1}{\fbox{\Ctab{2.5mm}{\scalebox{1}{\scriptsize $\mathstrut \!\rhd\!\!||$}}}}}
\putnotee{126}{73}{\scriptsize\color{blue} 6/10}
\end{layer}

\slidepage

\begin{layer}{120}{0}
\end{layer}

\begin{itemize}
\item
$x^2+6x+10=0$\hfill(1)\\
$(x+3)^2-9+10=0$より
 $(x+3)^2=-1$\hfill(2)\vspace{-1mm}
\item
実数では,2乗して$-1$になることはない\vspace{-1mm}
\item
2乗して$-1$になるものも数と考え,{\color{red}$i$}とおく({\color{red}虚数単位})\vspace{-1mm}
\item
$i^2=-1$
\end{itemize}

\sameslide

\vspace*{18mm}

\hypertarget{para9pg7}{}

\begin{layer}{120}{0}
\putnotew{96}{73}{\hyperlink{para8pg8}{\fbox{\Ctab{2.5mm}{\scalebox{1}{\scriptsize $\mathstrut||\!\lhd$}}}}}
\putnotew{101}{73}{\hyperlink{para9pg1}{\fbox{\Ctab{2.5mm}{\scalebox{1}{\scriptsize $\mathstrut|\!\lhd$}}}}}
\putnotew{108}{73}{\hyperlink{para9pg6}{\fbox{\Ctab{4.5mm}{\scalebox{1}{\scriptsize $\mathstrut\!\!\lhd\!\!$}}}}}
\putnotew{115}{73}{\hyperlink{para9pg8}{\fbox{\Ctab{4.5mm}{\scalebox{1}{\scriptsize $\mathstrut\!\rhd\!$}}}}}
\putnotew{120}{73}{\hyperlink{para9pg10}{\fbox{\Ctab{2.5mm}{\scalebox{1}{\scriptsize $\mathstrut \!\rhd\!\!|$}}}}}
\putnotew{125}{73}{\hyperlink{para10pg1}{\fbox{\Ctab{2.5mm}{\scalebox{1}{\scriptsize $\mathstrut \!\rhd\!\!||$}}}}}
\putnotee{126}{73}{\scriptsize\color{blue} 7/10}
\end{layer}

\slidepage

\begin{layer}{120}{0}
\end{layer}

\begin{itemize}
\item
$x^2+6x+10=0$\hfill(1)\\
$(x+3)^2-9+10=0$より
 $(x+3)^2=-1$\hfill(2)\vspace{-1mm}
\item
実数では,2乗して$-1$になることはない\vspace{-1mm}
\item
2乗して$-1$になるものも数と考え,{\color{red}$i$}とおく({\color{red}虚数単位})\vspace{-1mm}
\item
$i^2=-1$
 また $(-i)^2=-1$\vspace{-1mm}
\end{itemize}

\sameslide

\vspace*{18mm}

\hypertarget{para9pg8}{}

\begin{layer}{120}{0}
\putnotew{96}{73}{\hyperlink{para8pg8}{\fbox{\Ctab{2.5mm}{\scalebox{1}{\scriptsize $\mathstrut||\!\lhd$}}}}}
\putnotew{101}{73}{\hyperlink{para9pg1}{\fbox{\Ctab{2.5mm}{\scalebox{1}{\scriptsize $\mathstrut|\!\lhd$}}}}}
\putnotew{108}{73}{\hyperlink{para9pg7}{\fbox{\Ctab{4.5mm}{\scalebox{1}{\scriptsize $\mathstrut\!\!\lhd\!\!$}}}}}
\putnotew{115}{73}{\hyperlink{para9pg9}{\fbox{\Ctab{4.5mm}{\scalebox{1}{\scriptsize $\mathstrut\!\rhd\!$}}}}}
\putnotew{120}{73}{\hyperlink{para9pg10}{\fbox{\Ctab{2.5mm}{\scalebox{1}{\scriptsize $\mathstrut \!\rhd\!\!|$}}}}}
\putnotew{125}{73}{\hyperlink{para10pg1}{\fbox{\Ctab{2.5mm}{\scalebox{1}{\scriptsize $\mathstrut \!\rhd\!\!||$}}}}}
\putnotee{126}{73}{\scriptsize\color{blue} 8/10}
\end{layer}

\slidepage

\begin{layer}{120}{0}
\end{layer}

\begin{itemize}
\item
$x^2+6x+10=0$\hfill(1)\\
$(x+3)^2-9+10=0$より
 $(x+3)^2=-1$\hfill(2)\vspace{-1mm}
\item
実数では,2乗して$-1$になることはない\vspace{-1mm}
\item
2乗して$-1$になるものも数と考え,{\color{red}$i$}とおく({\color{red}虚数単位})\vspace{-1mm}
\item
$i^2=-1$
 また $(-i)^2=-1$\vspace{-1mm}
\item
2乗して$-1$になる数は$\pm i$があるが$\sqrt{-1}=i$とする\vspace{-1mm}
\end{itemize}

\sameslide

\vspace*{18mm}

\hypertarget{para9pg9}{}

\begin{layer}{120}{0}
\putnotew{96}{73}{\hyperlink{para8pg8}{\fbox{\Ctab{2.5mm}{\scalebox{1}{\scriptsize $\mathstrut||\!\lhd$}}}}}
\putnotew{101}{73}{\hyperlink{para9pg1}{\fbox{\Ctab{2.5mm}{\scalebox{1}{\scriptsize $\mathstrut|\!\lhd$}}}}}
\putnotew{108}{73}{\hyperlink{para9pg8}{\fbox{\Ctab{4.5mm}{\scalebox{1}{\scriptsize $\mathstrut\!\!\lhd\!\!$}}}}}
\putnotew{115}{73}{\hyperlink{para9pg10}{\fbox{\Ctab{4.5mm}{\scalebox{1}{\scriptsize $\mathstrut\!\rhd\!$}}}}}
\putnotew{120}{73}{\hyperlink{para9pg10}{\fbox{\Ctab{2.5mm}{\scalebox{1}{\scriptsize $\mathstrut \!\rhd\!\!|$}}}}}
\putnotew{125}{73}{\hyperlink{para10pg1}{\fbox{\Ctab{2.5mm}{\scalebox{1}{\scriptsize $\mathstrut \!\rhd\!\!||$}}}}}
\putnotee{126}{73}{\scriptsize\color{blue} 9/10}
\end{layer}

\slidepage

\begin{layer}{120}{0}
\end{layer}

\begin{itemize}
\item
$x^2+6x+10=0$\hfill(1)\\
$(x+3)^2-9+10=0$より
 $(x+3)^2=-1$\hfill(2)\vspace{-1mm}
\item
実数では,2乗して$-1$になることはない\vspace{-1mm}
\item
2乗して$-1$になるものも数と考え,{\color{red}$i$}とおく({\color{red}虚数単位})\vspace{-1mm}
\item
$i^2=-1$
 また $(-i)^2=-1$\vspace{-1mm}
\item
2乗して$-1$になる数は$\pm i$があるが$\sqrt{-1}=i$とする\vspace{-1mm}
\item
(2)より $x+3=\pm i$
\end{itemize}

\sameslide

\vspace*{18mm}

\hypertarget{para9pg10}{}

\begin{layer}{120}{0}
\putnotew{96}{73}{\hyperlink{para8pg8}{\fbox{\Ctab{2.5mm}{\scalebox{1}{\scriptsize $\mathstrut||\!\lhd$}}}}}
\putnotew{101}{73}{\hyperlink{para9pg1}{\fbox{\Ctab{2.5mm}{\scalebox{1}{\scriptsize $\mathstrut|\!\lhd$}}}}}
\putnotew{108}{73}{\hyperlink{para9pg9}{\fbox{\Ctab{4.5mm}{\scalebox{1}{\scriptsize $\mathstrut\!\!\lhd\!\!$}}}}}
\putnotew{115}{73}{\hyperlink{para9pg10}{\fbox{\Ctab{4.5mm}{\scalebox{1}{\scriptsize $\mathstrut\!\rhd\!$}}}}}
\putnotew{120}{73}{\hyperlink{para9pg10}{\fbox{\Ctab{2.5mm}{\scalebox{1}{\scriptsize $\mathstrut \!\rhd\!\!|$}}}}}
\putnotew{125}{73}{\hyperlink{para10pg1}{\fbox{\Ctab{2.5mm}{\scalebox{1}{\scriptsize $\mathstrut \!\rhd\!\!||$}}}}}
\putnotee{126}{73}{\scriptsize\color{blue} 10/10}
\end{layer}

\slidepage

\begin{layer}{120}{0}
\end{layer}

\begin{itemize}
\item
$x^2+6x+10=0$\hfill(1)\\
$(x+3)^2-9+10=0$より
 $(x+3)^2=-1$\hfill(2)\vspace{-1mm}
\item
実数では,2乗して$-1$になることはない\vspace{-1mm}
\item
2乗して$-1$になるものも数と考え,{\color{red}$i$}とおく({\color{red}虚数単位})\vspace{-1mm}
\item
$i^2=-1$
 また $(-i)^2=-1$\vspace{-1mm}
\item
2乗して$-1$になる数は$\pm i$があるが$\sqrt{-1}=i$とする\vspace{-1mm}
\item
(2)より $x+3=\pm i$
 $x=-3\pm i$
\end{itemize}

\newslide{負の数の平方根}

\vspace*{18mm}

\slidepage

\begin{layer}{120}{0}
\end{layer}

\begin{itemize}
\item
[例)]$\sqrt{-4}$
\item
2乗して$-4$になる数\\
\end{itemize}
%%%%%%%%%%%%%

%%%%%%%%%%%%%%%%%%%%


\sameslide

\vspace*{18mm}

\slidepage

\begin{layer}{120}{0}
\end{layer}

\begin{itemize}
\item
[例)]$\sqrt{-4}$
\item
2乗して$-4$になる数\\
 $(2i)^2=4i^2=-4,\ (-2i)^2=4i^2=-4$\\
\end{itemize}

\sameslide

\vspace*{18mm}

\slidepage

\begin{layer}{120}{0}
\end{layer}

\begin{itemize}
\item
[例)]$\sqrt{-4}$
\item
2乗して$-4$になる数\\
 $(2i)^2=4i^2=-4,\ (-2i)^2=4i^2=-4$\\
$\sqrt{-4}=2i$と定める
\end{itemize}

\sameslide

\vspace*{18mm}

\slidepage

\begin{layer}{120}{0}
\end{layer}

\begin{itemize}
\item
[例)]$\sqrt{-4}$
\item
2乗して$-4$になる数\\
 $(2i)^2=4i^2=-4,\ (-2i)^2=4i^2=-4$\\
$\sqrt{-4}=2i$と定める
\item
[例)]$\sqrt{-2}=\hakom{{\color{blue}\sqrt{2}\,i}}$
\end{itemize}

\sameslide

\vspace*{18mm}

\slidepage

\begin{layer}{120}{0}
\end{layer}

\begin{itemize}
\item
[例)]$\sqrt{-4}$
\item
2乗して$-4$になる数\\
 $(2i)^2=4i^2=-4,\ (-2i)^2=4i^2=-4$\\
$\sqrt{-4}=2i$と定める
\item
[例)]$\sqrt{-2}=\hakoma{{\color{blue}\sqrt{2}\,i}}$
\end{itemize}

\sameslide

\vspace*{18mm}

\slidepage

\begin{layer}{120}{0}
\putnotee{60}{43}{{\color{red}$\sqrt{-a}=\sqrt{a}\,i$}}
\end{layer}

\begin{itemize}
\item
[例)]$\sqrt{-4}$
\item
2乗して$-4$になる数\\
 $(2i)^2=4i^2=-4,\ (-2i)^2=4i^2=-4$\\
$\sqrt{-4}=2i$と定める
\item
[例)]$\sqrt{-2}=\hakoma{{\color{blue}\sqrt{2}\,i}}$
\end{itemize}

\sameslide

\vspace*{18mm}

\slidepage

\begin{layer}{120}{0}
\putnotee{60}{43}{{\color{red}$\sqrt{-a}=\sqrt{a}\,i$}}
\end{layer}

\begin{itemize}
\item
[例)]$\sqrt{-4}$
\item
2乗して$-4$になる数\\
 $(2i)^2=4i^2=-4,\ (-2i)^2=4i^2=-4$\\
$\sqrt{-4}=2i$と定める
\item
[例)]$\sqrt{-2}=\hakoma{{\color{blue}\sqrt{2}\,i}}$
\item
[課題]\monban 計算せよ.\hfill Text P133問1\seteda{55}\\
\eda{$\sqrt{-6}$}\eda{$\sqrt{-\bunsuu{9}{4}}$}\\
\eda{$\sqrt{-6}\sqrt{-2}$}\eda{$\sqrt{-3}\sqrt{2}$}
\end{itemize}

\newslide{複素数}

\vspace*{18mm}

\slidepage
\begin{itemize}
\item
$z=a+b\,i$の形の数を{\color{red}複素数}という ($a,b$は実数)\\
  $1+i,\ 2+3i,\ 4-2i,\ \cdots$
\end{itemize}
%%%%%%%%%%%%%

%%%%%%%%%%%%%%%%%%%%


\sameslide

\vspace*{18mm}

\slidepage
\begin{itemize}
\item
$z=a+b\,i$の形の数を{\color{red}複素数}という ($a,b$は実数)\\
  $1+i,\ 2+3i,\ 4-2i,\ \cdots$
\item
$a$を$z$の{\color{red}実部},$b$を$z$の{\color{red}虚部}という
\end{itemize}

\newslide{複素数の計算}

\vspace*{18mm}

\hypertarget{para10pg1}{}

\begin{layer}{120}{0}
\putnotew{96}{73}{\hyperlink{para9pg2}{\fbox{\Ctab{2.5mm}{\scalebox{1}{\scriptsize $\mathstrut||\!\lhd$}}}}}
\putnotew{101}{73}{\hyperlink{para10pg1}{\fbox{\Ctab{2.5mm}{\scalebox{1}{\scriptsize $\mathstrut|\!\lhd$}}}}}
\putnotew{108}{73}{\hyperlink{para9pg2}{\fbox{\Ctab{4.5mm}{\scalebox{1}{\scriptsize $\mathstrut\!\!\lhd\!\!$}}}}}
\putnotew{115}{73}{\hyperlink{para10pg2}{\fbox{\Ctab{4.5mm}{\scalebox{1}{\scriptsize $\mathstrut\!\rhd\!$}}}}}
\putnotew{120}{73}{\hyperlink{para10pg10}{\fbox{\Ctab{2.5mm}{\scalebox{1}{\scriptsize $\mathstrut \!\rhd\!\!|$}}}}}
\putnotew{125}{73}{\hyperlink{para11pg1}{\fbox{\Ctab{2.5mm}{\scalebox{1}{\scriptsize $\mathstrut \!\rhd\!\!||$}}}}}
\putnotee{126}{73}{\scriptsize\color{blue} 1/10}
\end{layer}

\slidepage

\begin{layer}{120}{0}
\end{layer}

\begin{itemize}
\item
ふつうの式のように計算し,$i^2$が出たら$-1$で置き換える
\item
和 $(2+3i)+(5-i)=\hakom{{\color{blue}7+2i}}$
\end{itemize}
%%%%%%%%%%%%%

%%%%%%%%%%%%%%%%%%%%


\sameslide

\vspace*{18mm}

\hypertarget{para10pg2}{}

\begin{layer}{120}{0}
\putnotew{96}{73}{\hyperlink{para9pg2}{\fbox{\Ctab{2.5mm}{\scalebox{1}{\scriptsize $\mathstrut||\!\lhd$}}}}}
\putnotew{101}{73}{\hyperlink{para10pg1}{\fbox{\Ctab{2.5mm}{\scalebox{1}{\scriptsize $\mathstrut|\!\lhd$}}}}}
\putnotew{108}{73}{\hyperlink{para10pg1}{\fbox{\Ctab{4.5mm}{\scalebox{1}{\scriptsize $\mathstrut\!\!\lhd\!\!$}}}}}
\putnotew{115}{73}{\hyperlink{para10pg3}{\fbox{\Ctab{4.5mm}{\scalebox{1}{\scriptsize $\mathstrut\!\rhd\!$}}}}}
\putnotew{120}{73}{\hyperlink{para10pg10}{\fbox{\Ctab{2.5mm}{\scalebox{1}{\scriptsize $\mathstrut \!\rhd\!\!|$}}}}}
\putnotew{125}{73}{\hyperlink{para11pg1}{\fbox{\Ctab{2.5mm}{\scalebox{1}{\scriptsize $\mathstrut \!\rhd\!\!||$}}}}}
\putnotee{126}{73}{\scriptsize\color{blue} 2/10}
\end{layer}

\slidepage

\begin{layer}{120}{0}
\end{layer}

\begin{itemize}
\item
ふつうの式のように計算し,$i^2$が出たら$-1$で置き換える
\item
和 $(2+3i)+(5-i)=\hakoma{{\color{blue}7+2i}}$
\end{itemize}

\sameslide

\vspace*{18mm}

\hypertarget{para10pg3}{}

\begin{layer}{120}{0}
\putnotew{96}{73}{\hyperlink{para9pg2}{\fbox{\Ctab{2.5mm}{\scalebox{1}{\scriptsize $\mathstrut||\!\lhd$}}}}}
\putnotew{101}{73}{\hyperlink{para10pg1}{\fbox{\Ctab{2.5mm}{\scalebox{1}{\scriptsize $\mathstrut|\!\lhd$}}}}}
\putnotew{108}{73}{\hyperlink{para10pg2}{\fbox{\Ctab{4.5mm}{\scalebox{1}{\scriptsize $\mathstrut\!\!\lhd\!\!$}}}}}
\putnotew{115}{73}{\hyperlink{para10pg4}{\fbox{\Ctab{4.5mm}{\scalebox{1}{\scriptsize $\mathstrut\!\rhd\!$}}}}}
\putnotew{120}{73}{\hyperlink{para10pg10}{\fbox{\Ctab{2.5mm}{\scalebox{1}{\scriptsize $\mathstrut \!\rhd\!\!|$}}}}}
\putnotew{125}{73}{\hyperlink{para11pg1}{\fbox{\Ctab{2.5mm}{\scalebox{1}{\scriptsize $\mathstrut \!\rhd\!\!||$}}}}}
\putnotee{126}{73}{\scriptsize\color{blue} 3/10}
\end{layer}

\slidepage

\begin{layer}{120}{0}
\end{layer}

\begin{itemize}
\item
ふつうの式のように計算し,$i^2$が出たら$-1$で置き換える
\item
和 $(2+3i)+(5-i)=\hakoma{{\color{blue}7+2i}}$
\item
積 $(2+3i)(4+i)$\\
\end{itemize}

\sameslide

\vspace*{18mm}

\hypertarget{para10pg4}{}

\begin{layer}{120}{0}
\putnotew{96}{73}{\hyperlink{para9pg2}{\fbox{\Ctab{2.5mm}{\scalebox{1}{\scriptsize $\mathstrut||\!\lhd$}}}}}
\putnotew{101}{73}{\hyperlink{para10pg1}{\fbox{\Ctab{2.5mm}{\scalebox{1}{\scriptsize $\mathstrut|\!\lhd$}}}}}
\putnotew{108}{73}{\hyperlink{para10pg3}{\fbox{\Ctab{4.5mm}{\scalebox{1}{\scriptsize $\mathstrut\!\!\lhd\!\!$}}}}}
\putnotew{115}{73}{\hyperlink{para10pg5}{\fbox{\Ctab{4.5mm}{\scalebox{1}{\scriptsize $\mathstrut\!\rhd\!$}}}}}
\putnotew{120}{73}{\hyperlink{para10pg10}{\fbox{\Ctab{2.5mm}{\scalebox{1}{\scriptsize $\mathstrut \!\rhd\!\!|$}}}}}
\putnotew{125}{73}{\hyperlink{para11pg1}{\fbox{\Ctab{2.5mm}{\scalebox{1}{\scriptsize $\mathstrut \!\rhd\!\!||$}}}}}
\putnotee{126}{73}{\scriptsize\color{blue} 4/10}
\end{layer}

\slidepage

\begin{layer}{120}{0}
\end{layer}

\begin{itemize}
\item
ふつうの式のように計算し,$i^2$が出たら$-1$で置き換える
\item
和 $(2+3i)+(5-i)=\hakoma{{\color{blue}7+2i}}$
\item
積 $(2+3i)(4+i)$\\
 $=8+2i+12i+3i^2$
\end{itemize}

\sameslide

\vspace*{18mm}

\hypertarget{para10pg5}{}

\begin{layer}{120}{0}
\putnotew{96}{73}{\hyperlink{para9pg2}{\fbox{\Ctab{2.5mm}{\scalebox{1}{\scriptsize $\mathstrut||\!\lhd$}}}}}
\putnotew{101}{73}{\hyperlink{para10pg1}{\fbox{\Ctab{2.5mm}{\scalebox{1}{\scriptsize $\mathstrut|\!\lhd$}}}}}
\putnotew{108}{73}{\hyperlink{para10pg4}{\fbox{\Ctab{4.5mm}{\scalebox{1}{\scriptsize $\mathstrut\!\!\lhd\!\!$}}}}}
\putnotew{115}{73}{\hyperlink{para10pg6}{\fbox{\Ctab{4.5mm}{\scalebox{1}{\scriptsize $\mathstrut\!\rhd\!$}}}}}
\putnotew{120}{73}{\hyperlink{para10pg10}{\fbox{\Ctab{2.5mm}{\scalebox{1}{\scriptsize $\mathstrut \!\rhd\!\!|$}}}}}
\putnotew{125}{73}{\hyperlink{para11pg1}{\fbox{\Ctab{2.5mm}{\scalebox{1}{\scriptsize $\mathstrut \!\rhd\!\!||$}}}}}
\putnotee{126}{73}{\scriptsize\color{blue} 5/10}
\end{layer}

\slidepage

\begin{layer}{120}{0}
\end{layer}

\begin{itemize}
\item
ふつうの式のように計算し,$i^2$が出たら$-1$で置き換える
\item
和 $(2+3i)+(5-i)=\hakoma{{\color{blue}7+2i}}$
\item
積 $(2+3i)(4+i)$\\
 $=8+2i+12i+3i^2$
$=8+14i-3$
\end{itemize}

\sameslide

\vspace*{18mm}

\hypertarget{para10pg6}{}

\begin{layer}{120}{0}
\putnotew{96}{73}{\hyperlink{para9pg2}{\fbox{\Ctab{2.5mm}{\scalebox{1}{\scriptsize $\mathstrut||\!\lhd$}}}}}
\putnotew{101}{73}{\hyperlink{para10pg1}{\fbox{\Ctab{2.5mm}{\scalebox{1}{\scriptsize $\mathstrut|\!\lhd$}}}}}
\putnotew{108}{73}{\hyperlink{para10pg5}{\fbox{\Ctab{4.5mm}{\scalebox{1}{\scriptsize $\mathstrut\!\!\lhd\!\!$}}}}}
\putnotew{115}{73}{\hyperlink{para10pg7}{\fbox{\Ctab{4.5mm}{\scalebox{1}{\scriptsize $\mathstrut\!\rhd\!$}}}}}
\putnotew{120}{73}{\hyperlink{para10pg10}{\fbox{\Ctab{2.5mm}{\scalebox{1}{\scriptsize $\mathstrut \!\rhd\!\!|$}}}}}
\putnotew{125}{73}{\hyperlink{para11pg1}{\fbox{\Ctab{2.5mm}{\scalebox{1}{\scriptsize $\mathstrut \!\rhd\!\!||$}}}}}
\putnotee{126}{73}{\scriptsize\color{blue} 6/10}
\end{layer}

\slidepage

\begin{layer}{120}{0}
\end{layer}

\begin{itemize}
\item
ふつうの式のように計算し,$i^2$が出たら$-1$で置き換える
\item
和 $(2+3i)+(5-i)=\hakoma{{\color{blue}7+2i}}$
\item
積 $(2+3i)(4+i)$\\
 $=8+2i+12i+3i^2$
$=8+14i-3$
$=5+14i$
\end{itemize}

\sameslide

\vspace*{18mm}

\hypertarget{para10pg7}{}

\begin{layer}{120}{0}
\putnotew{96}{73}{\hyperlink{para9pg2}{\fbox{\Ctab{2.5mm}{\scalebox{1}{\scriptsize $\mathstrut||\!\lhd$}}}}}
\putnotew{101}{73}{\hyperlink{para10pg1}{\fbox{\Ctab{2.5mm}{\scalebox{1}{\scriptsize $\mathstrut|\!\lhd$}}}}}
\putnotew{108}{73}{\hyperlink{para10pg6}{\fbox{\Ctab{4.5mm}{\scalebox{1}{\scriptsize $\mathstrut\!\!\lhd\!\!$}}}}}
\putnotew{115}{73}{\hyperlink{para10pg8}{\fbox{\Ctab{4.5mm}{\scalebox{1}{\scriptsize $\mathstrut\!\rhd\!$}}}}}
\putnotew{120}{73}{\hyperlink{para10pg10}{\fbox{\Ctab{2.5mm}{\scalebox{1}{\scriptsize $\mathstrut \!\rhd\!\!|$}}}}}
\putnotew{125}{73}{\hyperlink{para11pg1}{\fbox{\Ctab{2.5mm}{\scalebox{1}{\scriptsize $\mathstrut \!\rhd\!\!||$}}}}}
\putnotee{126}{73}{\scriptsize\color{blue} 7/10}
\end{layer}

\slidepage

\begin{layer}{120}{0}
\end{layer}

\begin{itemize}
\item
ふつうの式のように計算し,$i^2$が出たら$-1$で置き換える
\item
和 $(2+3i)+(5-i)=\hakoma{{\color{blue}7+2i}}$
\item
積 $(2+3i)(4+i)$\\
 $=8+2i+12i+3i^2$
$=8+14i-3$
$=5+14i$
\item
[] $(2+i)(2-i)$\\
\end{itemize}

\sameslide

\vspace*{18mm}

\hypertarget{para10pg8}{}

\begin{layer}{120}{0}
\putnotew{96}{73}{\hyperlink{para9pg2}{\fbox{\Ctab{2.5mm}{\scalebox{1}{\scriptsize $\mathstrut||\!\lhd$}}}}}
\putnotew{101}{73}{\hyperlink{para10pg1}{\fbox{\Ctab{2.5mm}{\scalebox{1}{\scriptsize $\mathstrut|\!\lhd$}}}}}
\putnotew{108}{73}{\hyperlink{para10pg7}{\fbox{\Ctab{4.5mm}{\scalebox{1}{\scriptsize $\mathstrut\!\!\lhd\!\!$}}}}}
\putnotew{115}{73}{\hyperlink{para10pg9}{\fbox{\Ctab{4.5mm}{\scalebox{1}{\scriptsize $\mathstrut\!\rhd\!$}}}}}
\putnotew{120}{73}{\hyperlink{para10pg10}{\fbox{\Ctab{2.5mm}{\scalebox{1}{\scriptsize $\mathstrut \!\rhd\!\!|$}}}}}
\putnotew{125}{73}{\hyperlink{para11pg1}{\fbox{\Ctab{2.5mm}{\scalebox{1}{\scriptsize $\mathstrut \!\rhd\!\!||$}}}}}
\putnotee{126}{73}{\scriptsize\color{blue} 8/10}
\end{layer}

\slidepage

\begin{layer}{120}{0}
\end{layer}

\begin{itemize}
\item
ふつうの式のように計算し,$i^2$が出たら$-1$で置き換える
\item
和 $(2+3i)+(5-i)=\hakoma{{\color{blue}7+2i}}$
\item
積 $(2+3i)(4+i)$\\
 $=8+2i+12i+3i^2$
$=8+14i-3$
$=5+14i$
\item
[] $(2+i)(2-i)$\\
 $=4-i^2$
\end{itemize}

\sameslide

\vspace*{18mm}

\hypertarget{para10pg9}{}

\begin{layer}{120}{0}
\putnotew{96}{73}{\hyperlink{para9pg2}{\fbox{\Ctab{2.5mm}{\scalebox{1}{\scriptsize $\mathstrut||\!\lhd$}}}}}
\putnotew{101}{73}{\hyperlink{para10pg1}{\fbox{\Ctab{2.5mm}{\scalebox{1}{\scriptsize $\mathstrut|\!\lhd$}}}}}
\putnotew{108}{73}{\hyperlink{para10pg8}{\fbox{\Ctab{4.5mm}{\scalebox{1}{\scriptsize $\mathstrut\!\!\lhd\!\!$}}}}}
\putnotew{115}{73}{\hyperlink{para10pg10}{\fbox{\Ctab{4.5mm}{\scalebox{1}{\scriptsize $\mathstrut\!\rhd\!$}}}}}
\putnotew{120}{73}{\hyperlink{para10pg10}{\fbox{\Ctab{2.5mm}{\scalebox{1}{\scriptsize $\mathstrut \!\rhd\!\!|$}}}}}
\putnotew{125}{73}{\hyperlink{para11pg1}{\fbox{\Ctab{2.5mm}{\scalebox{1}{\scriptsize $\mathstrut \!\rhd\!\!||$}}}}}
\putnotee{126}{73}{\scriptsize\color{blue} 9/10}
\end{layer}

\slidepage

\begin{layer}{120}{0}
\end{layer}

\begin{itemize}
\item
ふつうの式のように計算し,$i^2$が出たら$-1$で置き換える
\item
和 $(2+3i)+(5-i)=\hakoma{{\color{blue}7+2i}}$
\item
積 $(2+3i)(4+i)$\\
 $=8+2i+12i+3i^2$
$=8+14i-3$
$=5+14i$
\item
[] $(2+i)(2-i)$\\
 $=4-i^2$
$=4+1=5$
\end{itemize}

\sameslide

\vspace*{18mm}

\hypertarget{para10pg10}{}

\begin{layer}{120}{0}
\putnotew{96}{73}{\hyperlink{para9pg2}{\fbox{\Ctab{2.5mm}{\scalebox{1}{\scriptsize $\mathstrut||\!\lhd$}}}}}
\putnotew{101}{73}{\hyperlink{para10pg1}{\fbox{\Ctab{2.5mm}{\scalebox{1}{\scriptsize $\mathstrut|\!\lhd$}}}}}
\putnotew{108}{73}{\hyperlink{para10pg9}{\fbox{\Ctab{4.5mm}{\scalebox{1}{\scriptsize $\mathstrut\!\!\lhd\!\!$}}}}}
\putnotew{115}{73}{\hyperlink{para10pg10}{\fbox{\Ctab{4.5mm}{\scalebox{1}{\scriptsize $\mathstrut\!\rhd\!$}}}}}
\putnotew{120}{73}{\hyperlink{para10pg10}{\fbox{\Ctab{2.5mm}{\scalebox{1}{\scriptsize $\mathstrut \!\rhd\!\!|$}}}}}
\putnotew{125}{73}{\hyperlink{para11pg1}{\fbox{\Ctab{2.5mm}{\scalebox{1}{\scriptsize $\mathstrut \!\rhd\!\!||$}}}}}
\putnotee{126}{73}{\scriptsize\color{blue} 10/10}
\end{layer}

\slidepage

\begin{layer}{120}{0}
\putnotee{22}{72}{{\color{red}$(a+bi)(a-bi)=a^2+b^2$}}
\end{layer}

\begin{itemize}
\item
ふつうの式のように計算し,$i^2$が出たら$-1$で置き換える
\item
和 $(2+3i)+(5-i)=\hakoma{{\color{blue}7+2i}}$
\item
積 $(2+3i)(4+i)$\\
 $=8+2i+12i+3i^2$
$=8+14i-3$
$=5+14i$
\item
[] $(2+i)(2-i)$\\
 $=4-i^2$
$=4+1=5$
\end{itemize}

\newslide{複素数の計算(商)}

\vspace*{18mm}

\slidepage
\begin{itemize}
\item
$\bunsuu{1+3i}{2+i}$\\
\end{itemize}
%%%%%%%%%%%%%

%%%%%%%%%%%%%%%%%%%%


\sameslide

\vspace*{18mm}

\slidepage
\begin{itemize}
\item
$\bunsuu{1+3i}{2+i}$\\
  {\color{red}$(2+i)(2-i)=5$を用いる}\vspace{1mm}\\
\end{itemize}

\sameslide

\vspace*{18mm}

\slidepage
\begin{itemize}
\item
$\bunsuu{1+3i}{2+i}$\\
  {\color{red}$(2+i)(2-i)=5$を用いる}\vspace{1mm}\\
$=\bunsuu{(1+3i)(2-i)}{(2+i)(2-i)}$
\end{itemize}

\sameslide

\vspace*{18mm}

\slidepage
\begin{itemize}
\item
$\bunsuu{1+3i}{2+i}$\\
  {\color{red}$(2+i)(2-i)=5$を用いる}\vspace{1mm}\\
$=\bunsuu{(1+3i)(2-i)}{(2+i)(2-i)}$
$=\bunsuu{5+5i}{5}$
\end{itemize}

\sameslide

\vspace*{18mm}

\slidepage
\begin{itemize}
\item
$\bunsuu{1+3i}{2+i}$\\
  {\color{red}$(2+i)(2-i)=5$を用いる}\vspace{1mm}\\
$=\bunsuu{(1+3i)(2-i)}{(2+i)(2-i)}$
$=\bunsuu{5+5i}{5}$
$=1+i$
\end{itemize}

\newslide{複素数の計算問題}

\vspace*{18mm}

\slidepage
\begin{itemize}
\item
[課題]\monban 計算せよ.\hfill TextP132問1\seteda{55}\\
\eda{$(1-3i)+(2-5i)$}\eda{$(10-7i)-(3+9i)$}\vspace{2mm}\\
\eda{$(-4+7i)(3+2i)$}\eda{$(-2+6i)(2+6i)$}\vspace{2mm}\\
\eda{$\bunsuu{5+2i}{1-3i}$}\eda{$\bunsuu{1+i}{-2+5i}-\bunsuu{4-2i}{2+5i}$}
\end{itemize}
%%%%%%%%%%%%%

%%%%%%%%%%%%%%%%%%%%


\newslide{複素数平面}

\vspace*{18mm}

\slidepage

\begin{layer}{120}{0}
\putnotese{75}{15}{%%% /Users/takatoosetsuo/Dropbox/2018polytec/lecture/0611/presen/fig/plane1.tex 
%%% Generator=presen0611.cdy 
{\unitlength=5mm%
\begin{picture}%
(10,10)(-5,-5)%
\special{pn 8}%
%
\Large\bf\boldmath%
\small%
\special{pn 4}%
\special{pa -984 886}\special{pa -965 886}\special{fp}\special{pa -945 886}\special{pa -926 886}\special{fp}%
\special{pa -906 886}\special{pa -887 886}\special{fp}\special{pa -867 886}\special{pa -848 886}\special{fp}%
\special{pa -828 886}\special{pa -809 886}\special{fp}\special{pa -789 886}\special{pa -770 886}\special{fp}%
\special{pa -750 886}\special{pa -731 886}\special{fp}\special{pa -711 886}\special{pa -692 886}\special{fp}%
\special{pa -672 886}\special{pa -653 886}\special{fp}\special{pa -633 886}\special{pa -614 886}\special{fp}%
\special{pa -594 886}\special{pa -575 886}\special{fp}\special{pa -555 886}\special{pa -536 886}\special{fp}%
\special{pa -516 886}\special{pa -497 886}\special{fp}\special{pa -478 886}\special{pa -458 886}\special{fp}%
\special{pa -439 886}\special{pa -419 886}\special{fp}\special{pa -400 886}\special{pa -380 886}\special{fp}%
\special{pa -361 886}\special{pa -341 886}\special{fp}\special{pa -322 886}\special{pa -302 886}\special{fp}%
\special{pa -283 886}\special{pa -263 886}\special{fp}\special{pa -244 886}\special{pa -224 886}\special{fp}%
\special{pa -205 886}\special{pa -185 886}\special{fp}\special{pa -166 886}\special{pa -146 886}\special{fp}%
\special{pa -127 886}\special{pa -107 886}\special{fp}\special{pa -88 886}\special{pa -68 886}\special{fp}%
\special{pa -49 886}\special{pa -29 886}\special{fp}\special{pa -10 886}\special{pa 10 886}\special{fp}%
\special{pa 29 886}\special{pa 49 886}\special{fp}\special{pa 68 886}\special{pa 88 886}\special{fp}%
\special{pa 107 886}\special{pa 127 886}\special{fp}\special{pa 146 886}\special{pa 166 886}\special{fp}%
\special{pa 185 886}\special{pa 205 886}\special{fp}\special{pa 224 886}\special{pa 244 886}\special{fp}%
\special{pa 263 886}\special{pa 283 886}\special{fp}\special{pa 302 886}\special{pa 322 886}\special{fp}%
\special{pa 341 886}\special{pa 361 886}\special{fp}\special{pa 380 886}\special{pa 400 886}\special{fp}%
\special{pa 419 886}\special{pa 439 886}\special{fp}\special{pa 458 886}\special{pa 478 886}\special{fp}%
\special{pa 497 886}\special{pa 516 886}\special{fp}\special{pa 536 886}\special{pa 555 886}\special{fp}%
\special{pa 575 886}\special{pa 594 886}\special{fp}\special{pa 614 886}\special{pa 633 886}\special{fp}%
\special{pa 653 886}\special{pa 672 886}\special{fp}\special{pa 692 886}\special{pa 711 886}\special{fp}%
\special{pa 731 886}\special{pa 750 886}\special{fp}\special{pa 770 886}\special{pa 789 886}\special{fp}%
\special{pa 809 886}\special{pa 828 886}\special{fp}\special{pa 848 886}\special{pa 867 886}\special{fp}%
\special{pa 887 886}\special{pa 906 886}\special{fp}\special{pa 926 886}\special{pa 945 886}\special{fp}%
\special{pa 965 886}\special{pa 984 886}\special{fp}%
%
\special{pa -984 689}\special{pa -965 689}\special{fp}\special{pa -945 689}\special{pa -926 689}\special{fp}%
\special{pa -906 689}\special{pa -887 689}\special{fp}\special{pa -867 689}\special{pa -848 689}\special{fp}%
\special{pa -828 689}\special{pa -809 689}\special{fp}\special{pa -789 689}\special{pa -770 689}\special{fp}%
\special{pa -750 689}\special{pa -731 689}\special{fp}\special{pa -711 689}\special{pa -692 689}\special{fp}%
\special{pa -672 689}\special{pa -653 689}\special{fp}\special{pa -633 689}\special{pa -614 689}\special{fp}%
\special{pa -594 689}\special{pa -575 689}\special{fp}\special{pa -555 689}\special{pa -536 689}\special{fp}%
\special{pa -516 689}\special{pa -497 689}\special{fp}\special{pa -478 689}\special{pa -458 689}\special{fp}%
\special{pa -439 689}\special{pa -419 689}\special{fp}\special{pa -400 689}\special{pa -380 689}\special{fp}%
\special{pa -361 689}\special{pa -341 689}\special{fp}\special{pa -322 689}\special{pa -302 689}\special{fp}%
\special{pa -283 689}\special{pa -263 689}\special{fp}\special{pa -244 689}\special{pa -224 689}\special{fp}%
\special{pa -205 689}\special{pa -185 689}\special{fp}\special{pa -166 689}\special{pa -146 689}\special{fp}%
\special{pa -127 689}\special{pa -107 689}\special{fp}\special{pa -88 689}\special{pa -68 689}\special{fp}%
\special{pa -49 689}\special{pa -29 689}\special{fp}\special{pa -10 689}\special{pa 10 689}\special{fp}%
\special{pa 29 689}\special{pa 49 689}\special{fp}\special{pa 68 689}\special{pa 88 689}\special{fp}%
\special{pa 107 689}\special{pa 127 689}\special{fp}\special{pa 146 689}\special{pa 166 689}\special{fp}%
\special{pa 185 689}\special{pa 205 689}\special{fp}\special{pa 224 689}\special{pa 244 689}\special{fp}%
\special{pa 263 689}\special{pa 283 689}\special{fp}\special{pa 302 689}\special{pa 322 689}\special{fp}%
\special{pa 341 689}\special{pa 361 689}\special{fp}\special{pa 380 689}\special{pa 400 689}\special{fp}%
\special{pa 419 689}\special{pa 439 689}\special{fp}\special{pa 458 689}\special{pa 478 689}\special{fp}%
\special{pa 497 689}\special{pa 516 689}\special{fp}\special{pa 536 689}\special{pa 555 689}\special{fp}%
\special{pa 575 689}\special{pa 594 689}\special{fp}\special{pa 614 689}\special{pa 633 689}\special{fp}%
\special{pa 653 689}\special{pa 672 689}\special{fp}\special{pa 692 689}\special{pa 711 689}\special{fp}%
\special{pa 731 689}\special{pa 750 689}\special{fp}\special{pa 770 689}\special{pa 789 689}\special{fp}%
\special{pa 809 689}\special{pa 828 689}\special{fp}\special{pa 848 689}\special{pa 867 689}\special{fp}%
\special{pa 887 689}\special{pa 906 689}\special{fp}\special{pa 926 689}\special{pa 945 689}\special{fp}%
\special{pa 965 689}\special{pa 984 689}\special{fp}%
%
\special{pa -984 492}\special{pa -965 492}\special{fp}\special{pa -945 492}\special{pa -926 492}\special{fp}%
\special{pa -906 492}\special{pa -887 492}\special{fp}\special{pa -867 492}\special{pa -848 492}\special{fp}%
\special{pa -828 492}\special{pa -809 492}\special{fp}\special{pa -789 492}\special{pa -770 492}\special{fp}%
\special{pa -750 492}\special{pa -731 492}\special{fp}\special{pa -711 492}\special{pa -692 492}\special{fp}%
\special{pa -672 492}\special{pa -653 492}\special{fp}\special{pa -633 492}\special{pa -614 492}\special{fp}%
\special{pa -594 492}\special{pa -575 492}\special{fp}\special{pa -555 492}\special{pa -536 492}\special{fp}%
\special{pa -516 492}\special{pa -497 492}\special{fp}\special{pa -478 492}\special{pa -458 492}\special{fp}%
\special{pa -439 492}\special{pa -419 492}\special{fp}\special{pa -400 492}\special{pa -380 492}\special{fp}%
\special{pa -361 492}\special{pa -341 492}\special{fp}\special{pa -322 492}\special{pa -302 492}\special{fp}%
\special{pa -283 492}\special{pa -263 492}\special{fp}\special{pa -244 492}\special{pa -224 492}\special{fp}%
\special{pa -205 492}\special{pa -185 492}\special{fp}\special{pa -166 492}\special{pa -146 492}\special{fp}%
\special{pa -127 492}\special{pa -107 492}\special{fp}\special{pa -88 492}\special{pa -68 492}\special{fp}%
\special{pa -49 492}\special{pa -29 492}\special{fp}\special{pa -10 492}\special{pa 10 492}\special{fp}%
\special{pa 29 492}\special{pa 49 492}\special{fp}\special{pa 68 492}\special{pa 88 492}\special{fp}%
\special{pa 107 492}\special{pa 127 492}\special{fp}\special{pa 146 492}\special{pa 166 492}\special{fp}%
\special{pa 185 492}\special{pa 205 492}\special{fp}\special{pa 224 492}\special{pa 244 492}\special{fp}%
\special{pa 263 492}\special{pa 283 492}\special{fp}\special{pa 302 492}\special{pa 322 492}\special{fp}%
\special{pa 341 492}\special{pa 361 492}\special{fp}\special{pa 380 492}\special{pa 400 492}\special{fp}%
\special{pa 419 492}\special{pa 439 492}\special{fp}\special{pa 458 492}\special{pa 478 492}\special{fp}%
\special{pa 497 492}\special{pa 516 492}\special{fp}\special{pa 536 492}\special{pa 555 492}\special{fp}%
\special{pa 575 492}\special{pa 594 492}\special{fp}\special{pa 614 492}\special{pa 633 492}\special{fp}%
\special{pa 653 492}\special{pa 672 492}\special{fp}\special{pa 692 492}\special{pa 711 492}\special{fp}%
\special{pa 731 492}\special{pa 750 492}\special{fp}\special{pa 770 492}\special{pa 789 492}\special{fp}%
\special{pa 809 492}\special{pa 828 492}\special{fp}\special{pa 848 492}\special{pa 867 492}\special{fp}%
\special{pa 887 492}\special{pa 906 492}\special{fp}\special{pa 926 492}\special{pa 945 492}\special{fp}%
\special{pa 965 492}\special{pa 984 492}\special{fp}%
%
\special{pa -984 295}\special{pa -965 295}\special{fp}\special{pa -945 295}\special{pa -926 295}\special{fp}%
\special{pa -906 295}\special{pa -887 295}\special{fp}\special{pa -867 295}\special{pa -848 295}\special{fp}%
\special{pa -828 295}\special{pa -809 295}\special{fp}\special{pa -789 295}\special{pa -770 295}\special{fp}%
\special{pa -750 295}\special{pa -731 295}\special{fp}\special{pa -711 295}\special{pa -692 295}\special{fp}%
\special{pa -672 295}\special{pa -653 295}\special{fp}\special{pa -633 295}\special{pa -614 295}\special{fp}%
\special{pa -594 295}\special{pa -575 295}\special{fp}\special{pa -555 295}\special{pa -536 295}\special{fp}%
\special{pa -516 295}\special{pa -497 295}\special{fp}\special{pa -478 295}\special{pa -458 295}\special{fp}%
\special{pa -439 295}\special{pa -419 295}\special{fp}\special{pa -400 295}\special{pa -380 295}\special{fp}%
\special{pa -361 295}\special{pa -341 295}\special{fp}\special{pa -322 295}\special{pa -302 295}\special{fp}%
\special{pa -283 295}\special{pa -263 295}\special{fp}\special{pa -244 295}\special{pa -224 295}\special{fp}%
\special{pa -205 295}\special{pa -185 295}\special{fp}\special{pa -166 295}\special{pa -146 295}\special{fp}%
\special{pa -127 295}\special{pa -107 295}\special{fp}\special{pa -88 295}\special{pa -68 295}\special{fp}%
\special{pa -49 295}\special{pa -29 295}\special{fp}\special{pa -10 295}\special{pa 10 295}\special{fp}%
\special{pa 29 295}\special{pa 49 295}\special{fp}\special{pa 68 295}\special{pa 88 295}\special{fp}%
\special{pa 107 295}\special{pa 127 295}\special{fp}\special{pa 146 295}\special{pa 166 295}\special{fp}%
\special{pa 185 295}\special{pa 205 295}\special{fp}\special{pa 224 295}\special{pa 244 295}\special{fp}%
\special{pa 263 295}\special{pa 283 295}\special{fp}\special{pa 302 295}\special{pa 322 295}\special{fp}%
\special{pa 341 295}\special{pa 361 295}\special{fp}\special{pa 380 295}\special{pa 400 295}\special{fp}%
\special{pa 419 295}\special{pa 439 295}\special{fp}\special{pa 458 295}\special{pa 478 295}\special{fp}%
\special{pa 497 295}\special{pa 516 295}\special{fp}\special{pa 536 295}\special{pa 555 295}\special{fp}%
\special{pa 575 295}\special{pa 594 295}\special{fp}\special{pa 614 295}\special{pa 633 295}\special{fp}%
\special{pa 653 295}\special{pa 672 295}\special{fp}\special{pa 692 295}\special{pa 711 295}\special{fp}%
\special{pa 731 295}\special{pa 750 295}\special{fp}\special{pa 770 295}\special{pa 789 295}\special{fp}%
\special{pa 809 295}\special{pa 828 295}\special{fp}\special{pa 848 295}\special{pa 867 295}\special{fp}%
\special{pa 887 295}\special{pa 906 295}\special{fp}\special{pa 926 295}\special{pa 945 295}\special{fp}%
\special{pa 965 295}\special{pa 984 295}\special{fp}%
%
\special{pa -984 98}\special{pa -965 98}\special{fp}\special{pa -945 98}\special{pa -926 98}\special{fp}%
\special{pa -906 98}\special{pa -887 98}\special{fp}\special{pa -867 98}\special{pa -848 98}\special{fp}%
\special{pa -828 98}\special{pa -809 98}\special{fp}\special{pa -789 98}\special{pa -770 98}\special{fp}%
\special{pa -750 98}\special{pa -731 98}\special{fp}\special{pa -711 98}\special{pa -692 98}\special{fp}%
\special{pa -672 98}\special{pa -653 98}\special{fp}\special{pa -633 98}\special{pa -614 98}\special{fp}%
\special{pa -594 98}\special{pa -575 98}\special{fp}\special{pa -555 98}\special{pa -536 98}\special{fp}%
\special{pa -516 98}\special{pa -497 98}\special{fp}\special{pa -478 98}\special{pa -458 98}\special{fp}%
\special{pa -439 98}\special{pa -419 98}\special{fp}\special{pa -400 98}\special{pa -380 98}\special{fp}%
\special{pa -361 98}\special{pa -341 98}\special{fp}\special{pa -322 98}\special{pa -302 98}\special{fp}%
\special{pa -283 98}\special{pa -263 98}\special{fp}\special{pa -244 98}\special{pa -224 98}\special{fp}%
\special{pa -205 98}\special{pa -185 98}\special{fp}\special{pa -166 98}\special{pa -146 98}\special{fp}%
\special{pa -127 98}\special{pa -107 98}\special{fp}\special{pa -88 98}\special{pa -68 98}\special{fp}%
\special{pa -49 98}\special{pa -29 98}\special{fp}\special{pa -10 98}\special{pa 10 98}\special{fp}%
\special{pa 29 98}\special{pa 49 98}\special{fp}\special{pa 68 98}\special{pa 88 98}\special{fp}%
\special{pa 107 98}\special{pa 127 98}\special{fp}\special{pa 146 98}\special{pa 166 98}\special{fp}%
\special{pa 185 98}\special{pa 205 98}\special{fp}\special{pa 224 98}\special{pa 244 98}\special{fp}%
\special{pa 263 98}\special{pa 283 98}\special{fp}\special{pa 302 98}\special{pa 322 98}\special{fp}%
\special{pa 341 98}\special{pa 361 98}\special{fp}\special{pa 380 98}\special{pa 400 98}\special{fp}%
\special{pa 419 98}\special{pa 439 98}\special{fp}\special{pa 458 98}\special{pa 478 98}\special{fp}%
\special{pa 497 98}\special{pa 516 98}\special{fp}\special{pa 536 98}\special{pa 555 98}\special{fp}%
\special{pa 575 98}\special{pa 594 98}\special{fp}\special{pa 614 98}\special{pa 633 98}\special{fp}%
\special{pa 653 98}\special{pa 672 98}\special{fp}\special{pa 692 98}\special{pa 711 98}\special{fp}%
\special{pa 731 98}\special{pa 750 98}\special{fp}\special{pa 770 98}\special{pa 789 98}\special{fp}%
\special{pa 809 98}\special{pa 828 98}\special{fp}\special{pa 848 98}\special{pa 867 98}\special{fp}%
\special{pa 887 98}\special{pa 906 98}\special{fp}\special{pa 926 98}\special{pa 945 98}\special{fp}%
\special{pa 965 98}\special{pa 984 98}\special{fp}%
%
\special{pa -984 -98}\special{pa -965 -98}\special{fp}\special{pa -945 -98}\special{pa -926 -98}\special{fp}%
\special{pa -906 -98}\special{pa -887 -98}\special{fp}\special{pa -867 -98}\special{pa -848 -98}\special{fp}%
\special{pa -828 -98}\special{pa -809 -98}\special{fp}\special{pa -789 -98}\special{pa -770 -98}\special{fp}%
\special{pa -750 -98}\special{pa -731 -98}\special{fp}\special{pa -711 -98}\special{pa -692 -98}\special{fp}%
\special{pa -672 -98}\special{pa -653 -98}\special{fp}\special{pa -633 -98}\special{pa -614 -98}\special{fp}%
\special{pa -594 -98}\special{pa -575 -98}\special{fp}\special{pa -555 -98}\special{pa -536 -98}\special{fp}%
\special{pa -516 -98}\special{pa -497 -98}\special{fp}\special{pa -478 -98}\special{pa -458 -98}\special{fp}%
\special{pa -439 -98}\special{pa -419 -98}\special{fp}\special{pa -400 -98}\special{pa -380 -98}\special{fp}%
\special{pa -361 -98}\special{pa -341 -98}\special{fp}\special{pa -322 -98}\special{pa -302 -98}\special{fp}%
\special{pa -283 -98}\special{pa -263 -98}\special{fp}\special{pa -244 -98}\special{pa -224 -98}\special{fp}%
\special{pa -205 -98}\special{pa -185 -98}\special{fp}\special{pa -166 -98}\special{pa -146 -98}\special{fp}%
\special{pa -127 -98}\special{pa -107 -98}\special{fp}\special{pa -88 -98}\special{pa -68 -98}\special{fp}%
\special{pa -49 -98}\special{pa -29 -98}\special{fp}\special{pa -10 -98}\special{pa 10 -98}\special{fp}%
\special{pa 29 -98}\special{pa 49 -98}\special{fp}\special{pa 68 -98}\special{pa 88 -98}\special{fp}%
\special{pa 107 -98}\special{pa 127 -98}\special{fp}\special{pa 146 -98}\special{pa 166 -98}\special{fp}%
\special{pa 185 -98}\special{pa 205 -98}\special{fp}\special{pa 224 -98}\special{pa 244 -98}\special{fp}%
\special{pa 263 -98}\special{pa 283 -98}\special{fp}\special{pa 302 -98}\special{pa 322 -98}\special{fp}%
\special{pa 341 -98}\special{pa 361 -98}\special{fp}\special{pa 380 -98}\special{pa 400 -98}\special{fp}%
\special{pa 419 -98}\special{pa 439 -98}\special{fp}\special{pa 458 -98}\special{pa 478 -98}\special{fp}%
\special{pa 497 -98}\special{pa 516 -98}\special{fp}\special{pa 536 -98}\special{pa 555 -98}\special{fp}%
\special{pa 575 -98}\special{pa 594 -98}\special{fp}\special{pa 614 -98}\special{pa 633 -98}\special{fp}%
\special{pa 653 -98}\special{pa 672 -98}\special{fp}\special{pa 692 -98}\special{pa 711 -98}\special{fp}%
\special{pa 731 -98}\special{pa 750 -98}\special{fp}\special{pa 770 -98}\special{pa 789 -98}\special{fp}%
\special{pa 809 -98}\special{pa 828 -98}\special{fp}\special{pa 848 -98}\special{pa 867 -98}\special{fp}%
\special{pa 887 -98}\special{pa 906 -98}\special{fp}\special{pa 926 -98}\special{pa 945 -98}\special{fp}%
\special{pa 965 -98}\special{pa 984 -98}\special{fp}%
%
\special{pa -984 -295}\special{pa -965 -295}\special{fp}\special{pa -945 -295}\special{pa -926 -295}\special{fp}%
\special{pa -906 -295}\special{pa -887 -295}\special{fp}\special{pa -867 -295}\special{pa -848 -295}\special{fp}%
\special{pa -828 -295}\special{pa -809 -295}\special{fp}\special{pa -789 -295}\special{pa -770 -295}\special{fp}%
\special{pa -750 -295}\special{pa -731 -295}\special{fp}\special{pa -711 -295}\special{pa -692 -295}\special{fp}%
\special{pa -672 -295}\special{pa -653 -295}\special{fp}\special{pa -633 -295}\special{pa -614 -295}\special{fp}%
\special{pa -594 -295}\special{pa -575 -295}\special{fp}\special{pa -555 -295}\special{pa -536 -295}\special{fp}%
\special{pa -516 -295}\special{pa -497 -295}\special{fp}\special{pa -478 -295}\special{pa -458 -295}\special{fp}%
\special{pa -439 -295}\special{pa -419 -295}\special{fp}\special{pa -400 -295}\special{pa -380 -295}\special{fp}%
\special{pa -361 -295}\special{pa -341 -295}\special{fp}\special{pa -322 -295}\special{pa -302 -295}\special{fp}%
\special{pa -283 -295}\special{pa -263 -295}\special{fp}\special{pa -244 -295}\special{pa -224 -295}\special{fp}%
\special{pa -205 -295}\special{pa -185 -295}\special{fp}\special{pa -166 -295}\special{pa -146 -295}\special{fp}%
\special{pa -127 -295}\special{pa -107 -295}\special{fp}\special{pa -88 -295}\special{pa -68 -295}\special{fp}%
\special{pa -49 -295}\special{pa -29 -295}\special{fp}\special{pa -10 -295}\special{pa 10 -295}\special{fp}%
\special{pa 29 -295}\special{pa 49 -295}\special{fp}\special{pa 68 -295}\special{pa 88 -295}\special{fp}%
\special{pa 107 -295}\special{pa 127 -295}\special{fp}\special{pa 146 -295}\special{pa 166 -295}\special{fp}%
\special{pa 185 -295}\special{pa 205 -295}\special{fp}\special{pa 224 -295}\special{pa 244 -295}\special{fp}%
\special{pa 263 -295}\special{pa 283 -295}\special{fp}\special{pa 302 -295}\special{pa 322 -295}\special{fp}%
\special{pa 341 -295}\special{pa 361 -295}\special{fp}\special{pa 380 -295}\special{pa 400 -295}\special{fp}%
\special{pa 419 -295}\special{pa 439 -295}\special{fp}\special{pa 458 -295}\special{pa 478 -295}\special{fp}%
\special{pa 497 -295}\special{pa 516 -295}\special{fp}\special{pa 536 -295}\special{pa 555 -295}\special{fp}%
\special{pa 575 -295}\special{pa 594 -295}\special{fp}\special{pa 614 -295}\special{pa 633 -295}\special{fp}%
\special{pa 653 -295}\special{pa 672 -295}\special{fp}\special{pa 692 -295}\special{pa 711 -295}\special{fp}%
\special{pa 731 -295}\special{pa 750 -295}\special{fp}\special{pa 770 -295}\special{pa 789 -295}\special{fp}%
\special{pa 809 -295}\special{pa 828 -295}\special{fp}\special{pa 848 -295}\special{pa 867 -295}\special{fp}%
\special{pa 887 -295}\special{pa 906 -295}\special{fp}\special{pa 926 -295}\special{pa 945 -295}\special{fp}%
\special{pa 965 -295}\special{pa 984 -295}\special{fp}%
%
\special{pa -984 -492}\special{pa -965 -492}\special{fp}\special{pa -945 -492}\special{pa -926 -492}\special{fp}%
\special{pa -906 -492}\special{pa -887 -492}\special{fp}\special{pa -867 -492}\special{pa -848 -492}\special{fp}%
\special{pa -828 -492}\special{pa -809 -492}\special{fp}\special{pa -789 -492}\special{pa -770 -492}\special{fp}%
\special{pa -750 -492}\special{pa -731 -492}\special{fp}\special{pa -711 -492}\special{pa -692 -492}\special{fp}%
\special{pa -672 -492}\special{pa -653 -492}\special{fp}\special{pa -633 -492}\special{pa -614 -492}\special{fp}%
\special{pa -594 -492}\special{pa -575 -492}\special{fp}\special{pa -555 -492}\special{pa -536 -492}\special{fp}%
\special{pa -516 -492}\special{pa -497 -492}\special{fp}\special{pa -478 -492}\special{pa -458 -492}\special{fp}%
\special{pa -439 -492}\special{pa -419 -492}\special{fp}\special{pa -400 -492}\special{pa -380 -492}\special{fp}%
\special{pa -361 -492}\special{pa -341 -492}\special{fp}\special{pa -322 -492}\special{pa -302 -492}\special{fp}%
\special{pa -283 -492}\special{pa -263 -492}\special{fp}\special{pa -244 -492}\special{pa -224 -492}\special{fp}%
\special{pa -205 -492}\special{pa -185 -492}\special{fp}\special{pa -166 -492}\special{pa -146 -492}\special{fp}%
\special{pa -127 -492}\special{pa -107 -492}\special{fp}\special{pa -88 -492}\special{pa -68 -492}\special{fp}%
\special{pa -49 -492}\special{pa -29 -492}\special{fp}\special{pa -10 -492}\special{pa 10 -492}\special{fp}%
\special{pa 29 -492}\special{pa 49 -492}\special{fp}\special{pa 68 -492}\special{pa 88 -492}\special{fp}%
\special{pa 107 -492}\special{pa 127 -492}\special{fp}\special{pa 146 -492}\special{pa 166 -492}\special{fp}%
\special{pa 185 -492}\special{pa 205 -492}\special{fp}\special{pa 224 -492}\special{pa 244 -492}\special{fp}%
\special{pa 263 -492}\special{pa 283 -492}\special{fp}\special{pa 302 -492}\special{pa 322 -492}\special{fp}%
\special{pa 341 -492}\special{pa 361 -492}\special{fp}\special{pa 380 -492}\special{pa 400 -492}\special{fp}%
\special{pa 419 -492}\special{pa 439 -492}\special{fp}\special{pa 458 -492}\special{pa 478 -492}\special{fp}%
\special{pa 497 -492}\special{pa 516 -492}\special{fp}\special{pa 536 -492}\special{pa 555 -492}\special{fp}%
\special{pa 575 -492}\special{pa 594 -492}\special{fp}\special{pa 614 -492}\special{pa 633 -492}\special{fp}%
\special{pa 653 -492}\special{pa 672 -492}\special{fp}\special{pa 692 -492}\special{pa 711 -492}\special{fp}%
\special{pa 731 -492}\special{pa 750 -492}\special{fp}\special{pa 770 -492}\special{pa 789 -492}\special{fp}%
\special{pa 809 -492}\special{pa 828 -492}\special{fp}\special{pa 848 -492}\special{pa 867 -492}\special{fp}%
\special{pa 887 -492}\special{pa 906 -492}\special{fp}\special{pa 926 -492}\special{pa 945 -492}\special{fp}%
\special{pa 965 -492}\special{pa 984 -492}\special{fp}%
%
\special{pa -984 -689}\special{pa -965 -689}\special{fp}\special{pa -945 -689}\special{pa -926 -689}\special{fp}%
\special{pa -906 -689}\special{pa -887 -689}\special{fp}\special{pa -867 -689}\special{pa -848 -689}\special{fp}%
\special{pa -828 -689}\special{pa -809 -689}\special{fp}\special{pa -789 -689}\special{pa -770 -689}\special{fp}%
\special{pa -750 -689}\special{pa -731 -689}\special{fp}\special{pa -711 -689}\special{pa -692 -689}\special{fp}%
\special{pa -672 -689}\special{pa -653 -689}\special{fp}\special{pa -633 -689}\special{pa -614 -689}\special{fp}%
\special{pa -594 -689}\special{pa -575 -689}\special{fp}\special{pa -555 -689}\special{pa -536 -689}\special{fp}%
\special{pa -516 -689}\special{pa -497 -689}\special{fp}\special{pa -478 -689}\special{pa -458 -689}\special{fp}%
\special{pa -439 -689}\special{pa -419 -689}\special{fp}\special{pa -400 -689}\special{pa -380 -689}\special{fp}%
\special{pa -361 -689}\special{pa -341 -689}\special{fp}\special{pa -322 -689}\special{pa -302 -689}\special{fp}%
\special{pa -283 -689}\special{pa -263 -689}\special{fp}\special{pa -244 -689}\special{pa -224 -689}\special{fp}%
\special{pa -205 -689}\special{pa -185 -689}\special{fp}\special{pa -166 -689}\special{pa -146 -689}\special{fp}%
\special{pa -127 -689}\special{pa -107 -689}\special{fp}\special{pa -88 -689}\special{pa -68 -689}\special{fp}%
\special{pa -49 -689}\special{pa -29 -689}\special{fp}\special{pa -10 -689}\special{pa 10 -689}\special{fp}%
\special{pa 29 -689}\special{pa 49 -689}\special{fp}\special{pa 68 -689}\special{pa 88 -689}\special{fp}%
\special{pa 107 -689}\special{pa 127 -689}\special{fp}\special{pa 146 -689}\special{pa 166 -689}\special{fp}%
\special{pa 185 -689}\special{pa 205 -689}\special{fp}\special{pa 224 -689}\special{pa 244 -689}\special{fp}%
\special{pa 263 -689}\special{pa 283 -689}\special{fp}\special{pa 302 -689}\special{pa 322 -689}\special{fp}%
\special{pa 341 -689}\special{pa 361 -689}\special{fp}\special{pa 380 -689}\special{pa 400 -689}\special{fp}%
\special{pa 419 -689}\special{pa 439 -689}\special{fp}\special{pa 458 -689}\special{pa 478 -689}\special{fp}%
\special{pa 497 -689}\special{pa 516 -689}\special{fp}\special{pa 536 -689}\special{pa 555 -689}\special{fp}%
\special{pa 575 -689}\special{pa 594 -689}\special{fp}\special{pa 614 -689}\special{pa 633 -689}\special{fp}%
\special{pa 653 -689}\special{pa 672 -689}\special{fp}\special{pa 692 -689}\special{pa 711 -689}\special{fp}%
\special{pa 731 -689}\special{pa 750 -689}\special{fp}\special{pa 770 -689}\special{pa 789 -689}\special{fp}%
\special{pa 809 -689}\special{pa 828 -689}\special{fp}\special{pa 848 -689}\special{pa 867 -689}\special{fp}%
\special{pa 887 -689}\special{pa 906 -689}\special{fp}\special{pa 926 -689}\special{pa 945 -689}\special{fp}%
\special{pa 965 -689}\special{pa 984 -689}\special{fp}%
%
\special{pa -984 -886}\special{pa -965 -886}\special{fp}\special{pa -945 -886}\special{pa -926 -886}\special{fp}%
\special{pa -906 -886}\special{pa -887 -886}\special{fp}\special{pa -867 -886}\special{pa -848 -886}\special{fp}%
\special{pa -828 -886}\special{pa -809 -886}\special{fp}\special{pa -789 -886}\special{pa -770 -886}\special{fp}%
\special{pa -750 -886}\special{pa -731 -886}\special{fp}\special{pa -711 -886}\special{pa -692 -886}\special{fp}%
\special{pa -672 -886}\special{pa -653 -886}\special{fp}\special{pa -633 -886}\special{pa -614 -886}\special{fp}%
\special{pa -594 -886}\special{pa -575 -886}\special{fp}\special{pa -555 -886}\special{pa -536 -886}\special{fp}%
\special{pa -516 -886}\special{pa -497 -886}\special{fp}\special{pa -478 -886}\special{pa -458 -886}\special{fp}%
\special{pa -439 -886}\special{pa -419 -886}\special{fp}\special{pa -400 -886}\special{pa -380 -886}\special{fp}%
\special{pa -361 -886}\special{pa -341 -886}\special{fp}\special{pa -322 -886}\special{pa -302 -886}\special{fp}%
\special{pa -283 -886}\special{pa -263 -886}\special{fp}\special{pa -244 -886}\special{pa -224 -886}\special{fp}%
\special{pa -205 -886}\special{pa -185 -886}\special{fp}\special{pa -166 -886}\special{pa -146 -886}\special{fp}%
\special{pa -127 -886}\special{pa -107 -886}\special{fp}\special{pa -88 -886}\special{pa -68 -886}\special{fp}%
\special{pa -49 -886}\special{pa -29 -886}\special{fp}\special{pa -10 -886}\special{pa 10 -886}\special{fp}%
\special{pa 29 -886}\special{pa 49 -886}\special{fp}\special{pa 68 -886}\special{pa 88 -886}\special{fp}%
\special{pa 107 -886}\special{pa 127 -886}\special{fp}\special{pa 146 -886}\special{pa 166 -886}\special{fp}%
\special{pa 185 -886}\special{pa 205 -886}\special{fp}\special{pa 224 -886}\special{pa 244 -886}\special{fp}%
\special{pa 263 -886}\special{pa 283 -886}\special{fp}\special{pa 302 -886}\special{pa 322 -886}\special{fp}%
\special{pa 341 -886}\special{pa 361 -886}\special{fp}\special{pa 380 -886}\special{pa 400 -886}\special{fp}%
\special{pa 419 -886}\special{pa 439 -886}\special{fp}\special{pa 458 -886}\special{pa 478 -886}\special{fp}%
\special{pa 497 -886}\special{pa 516 -886}\special{fp}\special{pa 536 -886}\special{pa 555 -886}\special{fp}%
\special{pa 575 -886}\special{pa 594 -886}\special{fp}\special{pa 614 -886}\special{pa 633 -886}\special{fp}%
\special{pa 653 -886}\special{pa 672 -886}\special{fp}\special{pa 692 -886}\special{pa 711 -886}\special{fp}%
\special{pa 731 -886}\special{pa 750 -886}\special{fp}\special{pa 770 -886}\special{pa 789 -886}\special{fp}%
\special{pa 809 -886}\special{pa 828 -886}\special{fp}\special{pa 848 -886}\special{pa 867 -886}\special{fp}%
\special{pa 887 -886}\special{pa 906 -886}\special{fp}\special{pa 926 -886}\special{pa 945 -886}\special{fp}%
\special{pa 965 -886}\special{pa 984 -886}\special{fp}%
%
\special{pa  -984   984}\special{pa   984   984}%
\special{fp}%
\special{pa  -984   787}\special{pa   984   787}%
\special{fp}%
\special{pa  -984   591}\special{pa   984   591}%
\special{fp}%
\special{pa  -984   394}\special{pa   984   394}%
\special{fp}%
\special{pa  -984   197}\special{pa   984   197}%
\special{fp}%
\special{pa  -984    -0}\special{pa   984    -0}%
\special{fp}%
\special{pa  -984  -197}\special{pa   984  -197}%
\special{fp}%
\special{pa  -984  -394}\special{pa   984  -394}%
\special{fp}%
\special{pa  -984  -591}\special{pa   984  -591}%
\special{fp}%
\special{pa  -984  -787}\special{pa   984  -787}%
\special{fp}%
\special{pa  -984  -984}\special{pa   984  -984}%
\special{fp}%
\special{pa -886 984}\special{pa -886 965}\special{fp}\special{pa -886 945}\special{pa -886 926}\special{fp}%
\special{pa -886 906}\special{pa -886 887}\special{fp}\special{pa -886 867}\special{pa -886 848}\special{fp}%
\special{pa -886 828}\special{pa -886 809}\special{fp}\special{pa -886 789}\special{pa -886 770}\special{fp}%
\special{pa -886 750}\special{pa -886 731}\special{fp}\special{pa -886 711}\special{pa -886 692}\special{fp}%
\special{pa -886 672}\special{pa -886 653}\special{fp}\special{pa -886 633}\special{pa -886 614}\special{fp}%
\special{pa -886 594}\special{pa -886 575}\special{fp}\special{pa -886 555}\special{pa -886 536}\special{fp}%
\special{pa -886 516}\special{pa -886 497}\special{fp}\special{pa -886 478}\special{pa -886 458}\special{fp}%
\special{pa -886 439}\special{pa -886 419}\special{fp}\special{pa -886 400}\special{pa -886 380}\special{fp}%
\special{pa -886 361}\special{pa -886 341}\special{fp}\special{pa -886 322}\special{pa -886 302}\special{fp}%
\special{pa -886 283}\special{pa -886 263}\special{fp}\special{pa -886 244}\special{pa -886 224}\special{fp}%
\special{pa -886 205}\special{pa -886 185}\special{fp}\special{pa -886 166}\special{pa -886 146}\special{fp}%
\special{pa -886 127}\special{pa -886 107}\special{fp}\special{pa -886 88}\special{pa -886 68}\special{fp}%
\special{pa -886 49}\special{pa -886 29}\special{fp}\special{pa -886 10}\special{pa -886 -10}\special{fp}%
\special{pa -886 -29}\special{pa -886 -49}\special{fp}\special{pa -886 -68}\special{pa -886 -88}\special{fp}%
\special{pa -886 -107}\special{pa -886 -127}\special{fp}\special{pa -886 -146}\special{pa -886 -166}\special{fp}%
\special{pa -886 -185}\special{pa -886 -205}\special{fp}\special{pa -886 -224}\special{pa -886 -244}\special{fp}%
\special{pa -886 -263}\special{pa -886 -283}\special{fp}\special{pa -886 -302}\special{pa -886 -322}\special{fp}%
\special{pa -886 -341}\special{pa -886 -361}\special{fp}\special{pa -886 -380}\special{pa -886 -400}\special{fp}%
\special{pa -886 -419}\special{pa -886 -439}\special{fp}\special{pa -886 -458}\special{pa -886 -478}\special{fp}%
\special{pa -886 -497}\special{pa -886 -516}\special{fp}\special{pa -886 -536}\special{pa -886 -555}\special{fp}%
\special{pa -886 -575}\special{pa -886 -594}\special{fp}\special{pa -886 -614}\special{pa -886 -633}\special{fp}%
\special{pa -886 -653}\special{pa -886 -672}\special{fp}\special{pa -886 -692}\special{pa -886 -711}\special{fp}%
\special{pa -886 -731}\special{pa -886 -750}\special{fp}\special{pa -886 -770}\special{pa -886 -789}\special{fp}%
\special{pa -886 -809}\special{pa -886 -828}\special{fp}\special{pa -886 -848}\special{pa -886 -867}\special{fp}%
\special{pa -886 -887}\special{pa -886 -906}\special{fp}\special{pa -886 -926}\special{pa -886 -945}\special{fp}%
\special{pa -886 -965}\special{pa -886 -984}\special{fp}%
%
\special{pa -689 984}\special{pa -689 965}\special{fp}\special{pa -689 945}\special{pa -689 926}\special{fp}%
\special{pa -689 906}\special{pa -689 887}\special{fp}\special{pa -689 867}\special{pa -689 848}\special{fp}%
\special{pa -689 828}\special{pa -689 809}\special{fp}\special{pa -689 789}\special{pa -689 770}\special{fp}%
\special{pa -689 750}\special{pa -689 731}\special{fp}\special{pa -689 711}\special{pa -689 692}\special{fp}%
\special{pa -689 672}\special{pa -689 653}\special{fp}\special{pa -689 633}\special{pa -689 614}\special{fp}%
\special{pa -689 594}\special{pa -689 575}\special{fp}\special{pa -689 555}\special{pa -689 536}\special{fp}%
\special{pa -689 516}\special{pa -689 497}\special{fp}\special{pa -689 478}\special{pa -689 458}\special{fp}%
\special{pa -689 439}\special{pa -689 419}\special{fp}\special{pa -689 400}\special{pa -689 380}\special{fp}%
\special{pa -689 361}\special{pa -689 341}\special{fp}\special{pa -689 322}\special{pa -689 302}\special{fp}%
\special{pa -689 283}\special{pa -689 263}\special{fp}\special{pa -689 244}\special{pa -689 224}\special{fp}%
\special{pa -689 205}\special{pa -689 185}\special{fp}\special{pa -689 166}\special{pa -689 146}\special{fp}%
\special{pa -689 127}\special{pa -689 107}\special{fp}\special{pa -689 88}\special{pa -689 68}\special{fp}%
\special{pa -689 49}\special{pa -689 29}\special{fp}\special{pa -689 10}\special{pa -689 -10}\special{fp}%
\special{pa -689 -29}\special{pa -689 -49}\special{fp}\special{pa -689 -68}\special{pa -689 -88}\special{fp}%
\special{pa -689 -107}\special{pa -689 -127}\special{fp}\special{pa -689 -146}\special{pa -689 -166}\special{fp}%
\special{pa -689 -185}\special{pa -689 -205}\special{fp}\special{pa -689 -224}\special{pa -689 -244}\special{fp}%
\special{pa -689 -263}\special{pa -689 -283}\special{fp}\special{pa -689 -302}\special{pa -689 -322}\special{fp}%
\special{pa -689 -341}\special{pa -689 -361}\special{fp}\special{pa -689 -380}\special{pa -689 -400}\special{fp}%
\special{pa -689 -419}\special{pa -689 -439}\special{fp}\special{pa -689 -458}\special{pa -689 -478}\special{fp}%
\special{pa -689 -497}\special{pa -689 -516}\special{fp}\special{pa -689 -536}\special{pa -689 -555}\special{fp}%
\special{pa -689 -575}\special{pa -689 -594}\special{fp}\special{pa -689 -614}\special{pa -689 -633}\special{fp}%
\special{pa -689 -653}\special{pa -689 -672}\special{fp}\special{pa -689 -692}\special{pa -689 -711}\special{fp}%
\special{pa -689 -731}\special{pa -689 -750}\special{fp}\special{pa -689 -770}\special{pa -689 -789}\special{fp}%
\special{pa -689 -809}\special{pa -689 -828}\special{fp}\special{pa -689 -848}\special{pa -689 -867}\special{fp}%
\special{pa -689 -887}\special{pa -689 -906}\special{fp}\special{pa -689 -926}\special{pa -689 -945}\special{fp}%
\special{pa -689 -965}\special{pa -689 -984}\special{fp}%
%
\special{pa -492 984}\special{pa -492 965}\special{fp}\special{pa -492 945}\special{pa -492 926}\special{fp}%
\special{pa -492 906}\special{pa -492 887}\special{fp}\special{pa -492 867}\special{pa -492 848}\special{fp}%
\special{pa -492 828}\special{pa -492 809}\special{fp}\special{pa -492 789}\special{pa -492 770}\special{fp}%
\special{pa -492 750}\special{pa -492 731}\special{fp}\special{pa -492 711}\special{pa -492 692}\special{fp}%
\special{pa -492 672}\special{pa -492 653}\special{fp}\special{pa -492 633}\special{pa -492 614}\special{fp}%
\special{pa -492 594}\special{pa -492 575}\special{fp}\special{pa -492 555}\special{pa -492 536}\special{fp}%
\special{pa -492 516}\special{pa -492 497}\special{fp}\special{pa -492 478}\special{pa -492 458}\special{fp}%
\special{pa -492 439}\special{pa -492 419}\special{fp}\special{pa -492 400}\special{pa -492 380}\special{fp}%
\special{pa -492 361}\special{pa -492 341}\special{fp}\special{pa -492 322}\special{pa -492 302}\special{fp}%
\special{pa -492 283}\special{pa -492 263}\special{fp}\special{pa -492 244}\special{pa -492 224}\special{fp}%
\special{pa -492 205}\special{pa -492 185}\special{fp}\special{pa -492 166}\special{pa -492 146}\special{fp}%
\special{pa -492 127}\special{pa -492 107}\special{fp}\special{pa -492 88}\special{pa -492 68}\special{fp}%
\special{pa -492 49}\special{pa -492 29}\special{fp}\special{pa -492 10}\special{pa -492 -10}\special{fp}%
\special{pa -492 -29}\special{pa -492 -49}\special{fp}\special{pa -492 -68}\special{pa -492 -88}\special{fp}%
\special{pa -492 -107}\special{pa -492 -127}\special{fp}\special{pa -492 -146}\special{pa -492 -166}\special{fp}%
\special{pa -492 -185}\special{pa -492 -205}\special{fp}\special{pa -492 -224}\special{pa -492 -244}\special{fp}%
\special{pa -492 -263}\special{pa -492 -283}\special{fp}\special{pa -492 -302}\special{pa -492 -322}\special{fp}%
\special{pa -492 -341}\special{pa -492 -361}\special{fp}\special{pa -492 -380}\special{pa -492 -400}\special{fp}%
\special{pa -492 -419}\special{pa -492 -439}\special{fp}\special{pa -492 -458}\special{pa -492 -478}\special{fp}%
\special{pa -492 -497}\special{pa -492 -516}\special{fp}\special{pa -492 -536}\special{pa -492 -555}\special{fp}%
\special{pa -492 -575}\special{pa -492 -594}\special{fp}\special{pa -492 -614}\special{pa -492 -633}\special{fp}%
\special{pa -492 -653}\special{pa -492 -672}\special{fp}\special{pa -492 -692}\special{pa -492 -711}\special{fp}%
\special{pa -492 -731}\special{pa -492 -750}\special{fp}\special{pa -492 -770}\special{pa -492 -789}\special{fp}%
\special{pa -492 -809}\special{pa -492 -828}\special{fp}\special{pa -492 -848}\special{pa -492 -867}\special{fp}%
\special{pa -492 -887}\special{pa -492 -906}\special{fp}\special{pa -492 -926}\special{pa -492 -945}\special{fp}%
\special{pa -492 -965}\special{pa -492 -984}\special{fp}%
%
\special{pa -295 984}\special{pa -295 965}\special{fp}\special{pa -295 945}\special{pa -295 926}\special{fp}%
\special{pa -295 906}\special{pa -295 887}\special{fp}\special{pa -295 867}\special{pa -295 848}\special{fp}%
\special{pa -295 828}\special{pa -295 809}\special{fp}\special{pa -295 789}\special{pa -295 770}\special{fp}%
\special{pa -295 750}\special{pa -295 731}\special{fp}\special{pa -295 711}\special{pa -295 692}\special{fp}%
\special{pa -295 672}\special{pa -295 653}\special{fp}\special{pa -295 633}\special{pa -295 614}\special{fp}%
\special{pa -295 594}\special{pa -295 575}\special{fp}\special{pa -295 555}\special{pa -295 536}\special{fp}%
\special{pa -295 516}\special{pa -295 497}\special{fp}\special{pa -295 478}\special{pa -295 458}\special{fp}%
\special{pa -295 439}\special{pa -295 419}\special{fp}\special{pa -295 400}\special{pa -295 380}\special{fp}%
\special{pa -295 361}\special{pa -295 341}\special{fp}\special{pa -295 322}\special{pa -295 302}\special{fp}%
\special{pa -295 283}\special{pa -295 263}\special{fp}\special{pa -295 244}\special{pa -295 224}\special{fp}%
\special{pa -295 205}\special{pa -295 185}\special{fp}\special{pa -295 166}\special{pa -295 146}\special{fp}%
\special{pa -295 127}\special{pa -295 107}\special{fp}\special{pa -295 88}\special{pa -295 68}\special{fp}%
\special{pa -295 49}\special{pa -295 29}\special{fp}\special{pa -295 10}\special{pa -295 -10}\special{fp}%
\special{pa -295 -29}\special{pa -295 -49}\special{fp}\special{pa -295 -68}\special{pa -295 -88}\special{fp}%
\special{pa -295 -107}\special{pa -295 -127}\special{fp}\special{pa -295 -146}\special{pa -295 -166}\special{fp}%
\special{pa -295 -185}\special{pa -295 -205}\special{fp}\special{pa -295 -224}\special{pa -295 -244}\special{fp}%
\special{pa -295 -263}\special{pa -295 -283}\special{fp}\special{pa -295 -302}\special{pa -295 -322}\special{fp}%
\special{pa -295 -341}\special{pa -295 -361}\special{fp}\special{pa -295 -380}\special{pa -295 -400}\special{fp}%
\special{pa -295 -419}\special{pa -295 -439}\special{fp}\special{pa -295 -458}\special{pa -295 -478}\special{fp}%
\special{pa -295 -497}\special{pa -295 -516}\special{fp}\special{pa -295 -536}\special{pa -295 -555}\special{fp}%
\special{pa -295 -575}\special{pa -295 -594}\special{fp}\special{pa -295 -614}\special{pa -295 -633}\special{fp}%
\special{pa -295 -653}\special{pa -295 -672}\special{fp}\special{pa -295 -692}\special{pa -295 -711}\special{fp}%
\special{pa -295 -731}\special{pa -295 -750}\special{fp}\special{pa -295 -770}\special{pa -295 -789}\special{fp}%
\special{pa -295 -809}\special{pa -295 -828}\special{fp}\special{pa -295 -848}\special{pa -295 -867}\special{fp}%
\special{pa -295 -887}\special{pa -295 -906}\special{fp}\special{pa -295 -926}\special{pa -295 -945}\special{fp}%
\special{pa -295 -965}\special{pa -295 -984}\special{fp}%
%
\special{pa -98 984}\special{pa -98 965}\special{fp}\special{pa -98 945}\special{pa -98 926}\special{fp}%
\special{pa -98 906}\special{pa -98 887}\special{fp}\special{pa -98 867}\special{pa -98 848}\special{fp}%
\special{pa -98 828}\special{pa -98 809}\special{fp}\special{pa -98 789}\special{pa -98 770}\special{fp}%
\special{pa -98 750}\special{pa -98 731}\special{fp}\special{pa -98 711}\special{pa -98 692}\special{fp}%
\special{pa -98 672}\special{pa -98 653}\special{fp}\special{pa -98 633}\special{pa -98 614}\special{fp}%
\special{pa -98 594}\special{pa -98 575}\special{fp}\special{pa -98 555}\special{pa -98 536}\special{fp}%
\special{pa -98 516}\special{pa -98 497}\special{fp}\special{pa -98 478}\special{pa -98 458}\special{fp}%
\special{pa -98 439}\special{pa -98 419}\special{fp}\special{pa -98 400}\special{pa -98 380}\special{fp}%
\special{pa -98 361}\special{pa -98 341}\special{fp}\special{pa -98 322}\special{pa -98 302}\special{fp}%
\special{pa -98 283}\special{pa -98 263}\special{fp}\special{pa -98 244}\special{pa -98 224}\special{fp}%
\special{pa -98 205}\special{pa -98 185}\special{fp}\special{pa -98 166}\special{pa -98 146}\special{fp}%
\special{pa -98 127}\special{pa -98 107}\special{fp}\special{pa -98 88}\special{pa -98 68}\special{fp}%
\special{pa -98 49}\special{pa -98 29}\special{fp}\special{pa -98 10}\special{pa -98 -10}\special{fp}%
\special{pa -98 -29}\special{pa -98 -49}\special{fp}\special{pa -98 -68}\special{pa -98 -88}\special{fp}%
\special{pa -98 -107}\special{pa -98 -127}\special{fp}\special{pa -98 -146}\special{pa -98 -166}\special{fp}%
\special{pa -98 -185}\special{pa -98 -205}\special{fp}\special{pa -98 -224}\special{pa -98 -244}\special{fp}%
\special{pa -98 -263}\special{pa -98 -283}\special{fp}\special{pa -98 -302}\special{pa -98 -322}\special{fp}%
\special{pa -98 -341}\special{pa -98 -361}\special{fp}\special{pa -98 -380}\special{pa -98 -400}\special{fp}%
\special{pa -98 -419}\special{pa -98 -439}\special{fp}\special{pa -98 -458}\special{pa -98 -478}\special{fp}%
\special{pa -98 -497}\special{pa -98 -516}\special{fp}\special{pa -98 -536}\special{pa -98 -555}\special{fp}%
\special{pa -98 -575}\special{pa -98 -594}\special{fp}\special{pa -98 -614}\special{pa -98 -633}\special{fp}%
\special{pa -98 -653}\special{pa -98 -672}\special{fp}\special{pa -98 -692}\special{pa -98 -711}\special{fp}%
\special{pa -98 -731}\special{pa -98 -750}\special{fp}\special{pa -98 -770}\special{pa -98 -789}\special{fp}%
\special{pa -98 -809}\special{pa -98 -828}\special{fp}\special{pa -98 -848}\special{pa -98 -867}\special{fp}%
\special{pa -98 -887}\special{pa -98 -906}\special{fp}\special{pa -98 -926}\special{pa -98 -945}\special{fp}%
\special{pa -98 -965}\special{pa -98 -984}\special{fp}%
%
\special{pa 98 984}\special{pa 98 965}\special{fp}\special{pa 98 945}\special{pa 98 926}\special{fp}%
\special{pa 98 906}\special{pa 98 887}\special{fp}\special{pa 98 867}\special{pa 98 848}\special{fp}%
\special{pa 98 828}\special{pa 98 809}\special{fp}\special{pa 98 789}\special{pa 98 770}\special{fp}%
\special{pa 98 750}\special{pa 98 731}\special{fp}\special{pa 98 711}\special{pa 98 692}\special{fp}%
\special{pa 98 672}\special{pa 98 653}\special{fp}\special{pa 98 633}\special{pa 98 614}\special{fp}%
\special{pa 98 594}\special{pa 98 575}\special{fp}\special{pa 98 555}\special{pa 98 536}\special{fp}%
\special{pa 98 516}\special{pa 98 497}\special{fp}\special{pa 98 478}\special{pa 98 458}\special{fp}%
\special{pa 98 439}\special{pa 98 419}\special{fp}\special{pa 98 400}\special{pa 98 380}\special{fp}%
\special{pa 98 361}\special{pa 98 341}\special{fp}\special{pa 98 322}\special{pa 98 302}\special{fp}%
\special{pa 98 283}\special{pa 98 263}\special{fp}\special{pa 98 244}\special{pa 98 224}\special{fp}%
\special{pa 98 205}\special{pa 98 185}\special{fp}\special{pa 98 166}\special{pa 98 146}\special{fp}%
\special{pa 98 127}\special{pa 98 107}\special{fp}\special{pa 98 88}\special{pa 98 68}\special{fp}%
\special{pa 98 49}\special{pa 98 29}\special{fp}\special{pa 98 10}\special{pa 98 -10}\special{fp}%
\special{pa 98 -29}\special{pa 98 -49}\special{fp}\special{pa 98 -68}\special{pa 98 -88}\special{fp}%
\special{pa 98 -107}\special{pa 98 -127}\special{fp}\special{pa 98 -146}\special{pa 98 -166}\special{fp}%
\special{pa 98 -185}\special{pa 98 -205}\special{fp}\special{pa 98 -224}\special{pa 98 -244}\special{fp}%
\special{pa 98 -263}\special{pa 98 -283}\special{fp}\special{pa 98 -302}\special{pa 98 -322}\special{fp}%
\special{pa 98 -341}\special{pa 98 -361}\special{fp}\special{pa 98 -380}\special{pa 98 -400}\special{fp}%
\special{pa 98 -419}\special{pa 98 -439}\special{fp}\special{pa 98 -458}\special{pa 98 -478}\special{fp}%
\special{pa 98 -497}\special{pa 98 -516}\special{fp}\special{pa 98 -536}\special{pa 98 -555}\special{fp}%
\special{pa 98 -575}\special{pa 98 -594}\special{fp}\special{pa 98 -614}\special{pa 98 -633}\special{fp}%
\special{pa 98 -653}\special{pa 98 -672}\special{fp}\special{pa 98 -692}\special{pa 98 -711}\special{fp}%
\special{pa 98 -731}\special{pa 98 -750}\special{fp}\special{pa 98 -770}\special{pa 98 -789}\special{fp}%
\special{pa 98 -809}\special{pa 98 -828}\special{fp}\special{pa 98 -848}\special{pa 98 -867}\special{fp}%
\special{pa 98 -887}\special{pa 98 -906}\special{fp}\special{pa 98 -926}\special{pa 98 -945}\special{fp}%
\special{pa 98 -965}\special{pa 98 -984}\special{fp}%
%
\special{pa 295 984}\special{pa 295 965}\special{fp}\special{pa 295 945}\special{pa 295 926}\special{fp}%
\special{pa 295 906}\special{pa 295 887}\special{fp}\special{pa 295 867}\special{pa 295 848}\special{fp}%
\special{pa 295 828}\special{pa 295 809}\special{fp}\special{pa 295 789}\special{pa 295 770}\special{fp}%
\special{pa 295 750}\special{pa 295 731}\special{fp}\special{pa 295 711}\special{pa 295 692}\special{fp}%
\special{pa 295 672}\special{pa 295 653}\special{fp}\special{pa 295 633}\special{pa 295 614}\special{fp}%
\special{pa 295 594}\special{pa 295 575}\special{fp}\special{pa 295 555}\special{pa 295 536}\special{fp}%
\special{pa 295 516}\special{pa 295 497}\special{fp}\special{pa 295 478}\special{pa 295 458}\special{fp}%
\special{pa 295 439}\special{pa 295 419}\special{fp}\special{pa 295 400}\special{pa 295 380}\special{fp}%
\special{pa 295 361}\special{pa 295 341}\special{fp}\special{pa 295 322}\special{pa 295 302}\special{fp}%
\special{pa 295 283}\special{pa 295 263}\special{fp}\special{pa 295 244}\special{pa 295 224}\special{fp}%
\special{pa 295 205}\special{pa 295 185}\special{fp}\special{pa 295 166}\special{pa 295 146}\special{fp}%
\special{pa 295 127}\special{pa 295 107}\special{fp}\special{pa 295 88}\special{pa 295 68}\special{fp}%
\special{pa 295 49}\special{pa 295 29}\special{fp}\special{pa 295 10}\special{pa 295 -10}\special{fp}%
\special{pa 295 -29}\special{pa 295 -49}\special{fp}\special{pa 295 -68}\special{pa 295 -88}\special{fp}%
\special{pa 295 -107}\special{pa 295 -127}\special{fp}\special{pa 295 -146}\special{pa 295 -166}\special{fp}%
\special{pa 295 -185}\special{pa 295 -205}\special{fp}\special{pa 295 -224}\special{pa 295 -244}\special{fp}%
\special{pa 295 -263}\special{pa 295 -283}\special{fp}\special{pa 295 -302}\special{pa 295 -322}\special{fp}%
\special{pa 295 -341}\special{pa 295 -361}\special{fp}\special{pa 295 -380}\special{pa 295 -400}\special{fp}%
\special{pa 295 -419}\special{pa 295 -439}\special{fp}\special{pa 295 -458}\special{pa 295 -478}\special{fp}%
\special{pa 295 -497}\special{pa 295 -516}\special{fp}\special{pa 295 -536}\special{pa 295 -555}\special{fp}%
\special{pa 295 -575}\special{pa 295 -594}\special{fp}\special{pa 295 -614}\special{pa 295 -633}\special{fp}%
\special{pa 295 -653}\special{pa 295 -672}\special{fp}\special{pa 295 -692}\special{pa 295 -711}\special{fp}%
\special{pa 295 -731}\special{pa 295 -750}\special{fp}\special{pa 295 -770}\special{pa 295 -789}\special{fp}%
\special{pa 295 -809}\special{pa 295 -828}\special{fp}\special{pa 295 -848}\special{pa 295 -867}\special{fp}%
\special{pa 295 -887}\special{pa 295 -906}\special{fp}\special{pa 295 -926}\special{pa 295 -945}\special{fp}%
\special{pa 295 -965}\special{pa 295 -984}\special{fp}%
%
\special{pa 492 984}\special{pa 492 965}\special{fp}\special{pa 492 945}\special{pa 492 926}\special{fp}%
\special{pa 492 906}\special{pa 492 887}\special{fp}\special{pa 492 867}\special{pa 492 848}\special{fp}%
\special{pa 492 828}\special{pa 492 809}\special{fp}\special{pa 492 789}\special{pa 492 770}\special{fp}%
\special{pa 492 750}\special{pa 492 731}\special{fp}\special{pa 492 711}\special{pa 492 692}\special{fp}%
\special{pa 492 672}\special{pa 492 653}\special{fp}\special{pa 492 633}\special{pa 492 614}\special{fp}%
\special{pa 492 594}\special{pa 492 575}\special{fp}\special{pa 492 555}\special{pa 492 536}\special{fp}%
\special{pa 492 516}\special{pa 492 497}\special{fp}\special{pa 492 478}\special{pa 492 458}\special{fp}%
\special{pa 492 439}\special{pa 492 419}\special{fp}\special{pa 492 400}\special{pa 492 380}\special{fp}%
\special{pa 492 361}\special{pa 492 341}\special{fp}\special{pa 492 322}\special{pa 492 302}\special{fp}%
\special{pa 492 283}\special{pa 492 263}\special{fp}\special{pa 492 244}\special{pa 492 224}\special{fp}%
\special{pa 492 205}\special{pa 492 185}\special{fp}\special{pa 492 166}\special{pa 492 146}\special{fp}%
\special{pa 492 127}\special{pa 492 107}\special{fp}\special{pa 492 88}\special{pa 492 68}\special{fp}%
\special{pa 492 49}\special{pa 492 29}\special{fp}\special{pa 492 10}\special{pa 492 -10}\special{fp}%
\special{pa 492 -29}\special{pa 492 -49}\special{fp}\special{pa 492 -68}\special{pa 492 -88}\special{fp}%
\special{pa 492 -107}\special{pa 492 -127}\special{fp}\special{pa 492 -146}\special{pa 492 -166}\special{fp}%
\special{pa 492 -185}\special{pa 492 -205}\special{fp}\special{pa 492 -224}\special{pa 492 -244}\special{fp}%
\special{pa 492 -263}\special{pa 492 -283}\special{fp}\special{pa 492 -302}\special{pa 492 -322}\special{fp}%
\special{pa 492 -341}\special{pa 492 -361}\special{fp}\special{pa 492 -380}\special{pa 492 -400}\special{fp}%
\special{pa 492 -419}\special{pa 492 -439}\special{fp}\special{pa 492 -458}\special{pa 492 -478}\special{fp}%
\special{pa 492 -497}\special{pa 492 -516}\special{fp}\special{pa 492 -536}\special{pa 492 -555}\special{fp}%
\special{pa 492 -575}\special{pa 492 -594}\special{fp}\special{pa 492 -614}\special{pa 492 -633}\special{fp}%
\special{pa 492 -653}\special{pa 492 -672}\special{fp}\special{pa 492 -692}\special{pa 492 -711}\special{fp}%
\special{pa 492 -731}\special{pa 492 -750}\special{fp}\special{pa 492 -770}\special{pa 492 -789}\special{fp}%
\special{pa 492 -809}\special{pa 492 -828}\special{fp}\special{pa 492 -848}\special{pa 492 -867}\special{fp}%
\special{pa 492 -887}\special{pa 492 -906}\special{fp}\special{pa 492 -926}\special{pa 492 -945}\special{fp}%
\special{pa 492 -965}\special{pa 492 -984}\special{fp}%
%
\special{pa 689 984}\special{pa 689 965}\special{fp}\special{pa 689 945}\special{pa 689 926}\special{fp}%
\special{pa 689 906}\special{pa 689 887}\special{fp}\special{pa 689 867}\special{pa 689 848}\special{fp}%
\special{pa 689 828}\special{pa 689 809}\special{fp}\special{pa 689 789}\special{pa 689 770}\special{fp}%
\special{pa 689 750}\special{pa 689 731}\special{fp}\special{pa 689 711}\special{pa 689 692}\special{fp}%
\special{pa 689 672}\special{pa 689 653}\special{fp}\special{pa 689 633}\special{pa 689 614}\special{fp}%
\special{pa 689 594}\special{pa 689 575}\special{fp}\special{pa 689 555}\special{pa 689 536}\special{fp}%
\special{pa 689 516}\special{pa 689 497}\special{fp}\special{pa 689 478}\special{pa 689 458}\special{fp}%
\special{pa 689 439}\special{pa 689 419}\special{fp}\special{pa 689 400}\special{pa 689 380}\special{fp}%
\special{pa 689 361}\special{pa 689 341}\special{fp}\special{pa 689 322}\special{pa 689 302}\special{fp}%
\special{pa 689 283}\special{pa 689 263}\special{fp}\special{pa 689 244}\special{pa 689 224}\special{fp}%
\special{pa 689 205}\special{pa 689 185}\special{fp}\special{pa 689 166}\special{pa 689 146}\special{fp}%
\special{pa 689 127}\special{pa 689 107}\special{fp}\special{pa 689 88}\special{pa 689 68}\special{fp}%
\special{pa 689 49}\special{pa 689 29}\special{fp}\special{pa 689 10}\special{pa 689 -10}\special{fp}%
\special{pa 689 -29}\special{pa 689 -49}\special{fp}\special{pa 689 -68}\special{pa 689 -88}\special{fp}%
\special{pa 689 -107}\special{pa 689 -127}\special{fp}\special{pa 689 -146}\special{pa 689 -166}\special{fp}%
\special{pa 689 -185}\special{pa 689 -205}\special{fp}\special{pa 689 -224}\special{pa 689 -244}\special{fp}%
\special{pa 689 -263}\special{pa 689 -283}\special{fp}\special{pa 689 -302}\special{pa 689 -322}\special{fp}%
\special{pa 689 -341}\special{pa 689 -361}\special{fp}\special{pa 689 -380}\special{pa 689 -400}\special{fp}%
\special{pa 689 -419}\special{pa 689 -439}\special{fp}\special{pa 689 -458}\special{pa 689 -478}\special{fp}%
\special{pa 689 -497}\special{pa 689 -516}\special{fp}\special{pa 689 -536}\special{pa 689 -555}\special{fp}%
\special{pa 689 -575}\special{pa 689 -594}\special{fp}\special{pa 689 -614}\special{pa 689 -633}\special{fp}%
\special{pa 689 -653}\special{pa 689 -672}\special{fp}\special{pa 689 -692}\special{pa 689 -711}\special{fp}%
\special{pa 689 -731}\special{pa 689 -750}\special{fp}\special{pa 689 -770}\special{pa 689 -789}\special{fp}%
\special{pa 689 -809}\special{pa 689 -828}\special{fp}\special{pa 689 -848}\special{pa 689 -867}\special{fp}%
\special{pa 689 -887}\special{pa 689 -906}\special{fp}\special{pa 689 -926}\special{pa 689 -945}\special{fp}%
\special{pa 689 -965}\special{pa 689 -984}\special{fp}%
%
\special{pa 886 984}\special{pa 886 965}\special{fp}\special{pa 886 945}\special{pa 886 926}\special{fp}%
\special{pa 886 906}\special{pa 886 887}\special{fp}\special{pa 886 867}\special{pa 886 848}\special{fp}%
\special{pa 886 828}\special{pa 886 809}\special{fp}\special{pa 886 789}\special{pa 886 770}\special{fp}%
\special{pa 886 750}\special{pa 886 731}\special{fp}\special{pa 886 711}\special{pa 886 692}\special{fp}%
\special{pa 886 672}\special{pa 886 653}\special{fp}\special{pa 886 633}\special{pa 886 614}\special{fp}%
\special{pa 886 594}\special{pa 886 575}\special{fp}\special{pa 886 555}\special{pa 886 536}\special{fp}%
\special{pa 886 516}\special{pa 886 497}\special{fp}\special{pa 886 478}\special{pa 886 458}\special{fp}%
\special{pa 886 439}\special{pa 886 419}\special{fp}\special{pa 886 400}\special{pa 886 380}\special{fp}%
\special{pa 886 361}\special{pa 886 341}\special{fp}\special{pa 886 322}\special{pa 886 302}\special{fp}%
\special{pa 886 283}\special{pa 886 263}\special{fp}\special{pa 886 244}\special{pa 886 224}\special{fp}%
\special{pa 886 205}\special{pa 886 185}\special{fp}\special{pa 886 166}\special{pa 886 146}\special{fp}%
\special{pa 886 127}\special{pa 886 107}\special{fp}\special{pa 886 88}\special{pa 886 68}\special{fp}%
\special{pa 886 49}\special{pa 886 29}\special{fp}\special{pa 886 10}\special{pa 886 -10}\special{fp}%
\special{pa 886 -29}\special{pa 886 -49}\special{fp}\special{pa 886 -68}\special{pa 886 -88}\special{fp}%
\special{pa 886 -107}\special{pa 886 -127}\special{fp}\special{pa 886 -146}\special{pa 886 -166}\special{fp}%
\special{pa 886 -185}\special{pa 886 -205}\special{fp}\special{pa 886 -224}\special{pa 886 -244}\special{fp}%
\special{pa 886 -263}\special{pa 886 -283}\special{fp}\special{pa 886 -302}\special{pa 886 -322}\special{fp}%
\special{pa 886 -341}\special{pa 886 -361}\special{fp}\special{pa 886 -380}\special{pa 886 -400}\special{fp}%
\special{pa 886 -419}\special{pa 886 -439}\special{fp}\special{pa 886 -458}\special{pa 886 -478}\special{fp}%
\special{pa 886 -497}\special{pa 886 -516}\special{fp}\special{pa 886 -536}\special{pa 886 -555}\special{fp}%
\special{pa 886 -575}\special{pa 886 -594}\special{fp}\special{pa 886 -614}\special{pa 886 -633}\special{fp}%
\special{pa 886 -653}\special{pa 886 -672}\special{fp}\special{pa 886 -692}\special{pa 886 -711}\special{fp}%
\special{pa 886 -731}\special{pa 886 -750}\special{fp}\special{pa 886 -770}\special{pa 886 -789}\special{fp}%
\special{pa 886 -809}\special{pa 886 -828}\special{fp}\special{pa 886 -848}\special{pa 886 -867}\special{fp}%
\special{pa 886 -887}\special{pa 886 -906}\special{fp}\special{pa 886 -926}\special{pa 886 -945}\special{fp}%
\special{pa 886 -965}\special{pa 886 -984}\special{fp}%
%
\special{pa  -984   984}\special{pa  -984  -984}%
\special{fp}%
\special{pa  -787   984}\special{pa  -787  -984}%
\special{fp}%
\special{pa  -591   984}\special{pa  -591  -984}%
\special{fp}%
\special{pa  -394   984}\special{pa  -394  -984}%
\special{fp}%
\special{pa  -197   984}\special{pa  -197  -984}%
\special{fp}%
\special{pa     0   984}\special{pa     0  -984}%
\special{fp}%
\special{pa   197   984}\special{pa   197  -984}%
\special{fp}%
\special{pa   394   984}\special{pa   394  -984}%
\special{fp}%
\special{pa   591   984}\special{pa   591  -984}%
\special{fp}%
\special{pa   787   984}\special{pa   787  -984}%
\special{fp}%
\special{pa   984   984}\special{pa   984  -984}%
\special{fp}%
\special{pn 8}%
\scriptsize%
\special{pa  -197   -20}\special{pa  -197    20}%
\special{fp}%
\settowidth{\Width}{$-1$}\setlength{\Width}{-0.5\Width}%
\settoheight{\Height}{$-1$}\settodepth{\Depth}{$-1$}\setlength{\Height}{-\Height}%
\put(-1.0000000,-0.2000000){\hspace*{\Width}\raisebox{\Height}{$-1$}}%
%
%
\special{pa   197   -20}\special{pa   197    20}%
\special{fp}%
\settowidth{\Width}{$1$}\setlength{\Width}{-0.5\Width}%
\settoheight{\Height}{$1$}\settodepth{\Depth}{$1$}\setlength{\Height}{-\Height}%
\put(1.0000000,-0.2000000){\hspace*{\Width}\raisebox{\Height}{$1$}}%
%
%
\special{pa    20   197}\special{pa   -20   197}%
\special{fp}%
\settowidth{\Width}{$-1$}\setlength{\Width}{-1\Width}%
\settoheight{\Height}{$-1$}\settodepth{\Depth}{$-1$}\setlength{\Height}{-0.5\Height}\setlength{\Depth}{0.5\Depth}\addtolength{\Height}{\Depth}%
\put(-0.2000000,-1.0000000){\hspace*{\Width}\raisebox{\Height}{$-1$}}%
%
%
\special{pa    20  -197}\special{pa   -20  -197}%
\special{fp}%
\settowidth{\Width}{$1$}\setlength{\Width}{-1\Width}%
\settoheight{\Height}{$1$}\settodepth{\Depth}{$1$}\setlength{\Height}{-0.5\Height}\setlength{\Depth}{0.5\Depth}\addtolength{\Height}{\Depth}%
\put(-0.2000000,1.0000000){\hspace*{\Width}\raisebox{\Height}{$1$}}%
%
%
\special{pn 8}%
\special{pa  -984    -0}\special{pa   965    -0}%
\special{fp}%
\special{pn 8}%
\special{pa 909 24}\special{pa 984 0}\special{pa 909 -24}\special{pa 909 0}\special{pa 909 24}%
\special{sh 1}\special{ip}%
\special{pn 1}%
\special{pa   909    24}\special{pa   984    -0}\special{pa   909   -24}\special{pa   909    -0}%
\special{pa   909    24}\special{pa   984    -0}%
\special{fp}%
\special{pn 8}%
\special{pn 8}%
\special{pa     0   984}\special{pa     0  -965}%
\special{fp}%
\special{pn 8}%
\special{pa 24 -909}\special{pa 0 -984}\special{pa -24 -909}\special{pa 0 -909}\special{pa 24 -909}%
\special{sh 1}\special{ip}%
\special{pn 1}%
\special{pa    24  -909}\special{pa     0  -984}\special{pa   -24  -909}\special{pa     0  -909}%
\special{pa    24  -909}\special{pa     0  -984}%
\special{fp}%
\special{pn 8}%
\settowidth{\Width}{$x\mbox{実軸}$}\setlength{\Width}{0\Width}%
\settoheight{\Height}{$x\mbox{実軸}$}\settodepth{\Depth}{$x\mbox{実軸}$}\setlength{\Height}{-0.5\Height}\setlength{\Depth}{0.5\Depth}\addtolength{\Height}{\Depth}%
\put(5.1000000,0.0000000){\hspace*{\Width}\raisebox{\Height}{$x\mbox{実軸}$}}%
%
\settowidth{\Width}{$y\mbox{虚軸}$}\setlength{\Width}{-0.5\Width}%
\settoheight{\Height}{$y\mbox{虚軸}$}\settodepth{\Depth}{$y\mbox{虚軸}$}\setlength{\Height}{\Depth}%
\put(0.0000000,5.1000000){\hspace*{\Width}\raisebox{\Height}{$y\mbox{虚軸}$}}%
%
\settowidth{\Width}{ }\setlength{\Width}{-1\Width}%
\settoheight{\Height}{ }\settodepth{\Depth}{ }\setlength{\Height}{-\Height}%
\put(-0.1000000,-0.1000000){\hspace*{\Width}\raisebox{\Height}{ }}%
%
\end{picture}}%}
\end{layer}

{\color{red}

\begin{layer}{120}{0}
\end{layer}

}
\begin{itemize}
\item
$z=a+b\,i$を平面上の点$(a,\ b)$で表す\seteda{50}\\
\end{itemize}
%%%%%%%%%%%%%

%%%%%%%%%%%%%%%%%%%%


\sameslide

\vspace*{18mm}

\slidepage

\begin{layer}{120}{0}
\putnotese{75}{15}{%%% /Users/takatoosetsuo/Dropbox/2018polytec/lecture/0611/presen/fig/plane1.tex 
%%% Generator=presen0611.cdy 
{\unitlength=5mm%
\begin{picture}%
(10,10)(-5,-5)%
\special{pn 8}%
%
\Large\bf\boldmath%
\small%
\special{pn 4}%
\special{pa -984 886}\special{pa -965 886}\special{fp}\special{pa -945 886}\special{pa -926 886}\special{fp}%
\special{pa -906 886}\special{pa -887 886}\special{fp}\special{pa -867 886}\special{pa -848 886}\special{fp}%
\special{pa -828 886}\special{pa -809 886}\special{fp}\special{pa -789 886}\special{pa -770 886}\special{fp}%
\special{pa -750 886}\special{pa -731 886}\special{fp}\special{pa -711 886}\special{pa -692 886}\special{fp}%
\special{pa -672 886}\special{pa -653 886}\special{fp}\special{pa -633 886}\special{pa -614 886}\special{fp}%
\special{pa -594 886}\special{pa -575 886}\special{fp}\special{pa -555 886}\special{pa -536 886}\special{fp}%
\special{pa -516 886}\special{pa -497 886}\special{fp}\special{pa -478 886}\special{pa -458 886}\special{fp}%
\special{pa -439 886}\special{pa -419 886}\special{fp}\special{pa -400 886}\special{pa -380 886}\special{fp}%
\special{pa -361 886}\special{pa -341 886}\special{fp}\special{pa -322 886}\special{pa -302 886}\special{fp}%
\special{pa -283 886}\special{pa -263 886}\special{fp}\special{pa -244 886}\special{pa -224 886}\special{fp}%
\special{pa -205 886}\special{pa -185 886}\special{fp}\special{pa -166 886}\special{pa -146 886}\special{fp}%
\special{pa -127 886}\special{pa -107 886}\special{fp}\special{pa -88 886}\special{pa -68 886}\special{fp}%
\special{pa -49 886}\special{pa -29 886}\special{fp}\special{pa -10 886}\special{pa 10 886}\special{fp}%
\special{pa 29 886}\special{pa 49 886}\special{fp}\special{pa 68 886}\special{pa 88 886}\special{fp}%
\special{pa 107 886}\special{pa 127 886}\special{fp}\special{pa 146 886}\special{pa 166 886}\special{fp}%
\special{pa 185 886}\special{pa 205 886}\special{fp}\special{pa 224 886}\special{pa 244 886}\special{fp}%
\special{pa 263 886}\special{pa 283 886}\special{fp}\special{pa 302 886}\special{pa 322 886}\special{fp}%
\special{pa 341 886}\special{pa 361 886}\special{fp}\special{pa 380 886}\special{pa 400 886}\special{fp}%
\special{pa 419 886}\special{pa 439 886}\special{fp}\special{pa 458 886}\special{pa 478 886}\special{fp}%
\special{pa 497 886}\special{pa 516 886}\special{fp}\special{pa 536 886}\special{pa 555 886}\special{fp}%
\special{pa 575 886}\special{pa 594 886}\special{fp}\special{pa 614 886}\special{pa 633 886}\special{fp}%
\special{pa 653 886}\special{pa 672 886}\special{fp}\special{pa 692 886}\special{pa 711 886}\special{fp}%
\special{pa 731 886}\special{pa 750 886}\special{fp}\special{pa 770 886}\special{pa 789 886}\special{fp}%
\special{pa 809 886}\special{pa 828 886}\special{fp}\special{pa 848 886}\special{pa 867 886}\special{fp}%
\special{pa 887 886}\special{pa 906 886}\special{fp}\special{pa 926 886}\special{pa 945 886}\special{fp}%
\special{pa 965 886}\special{pa 984 886}\special{fp}%
%
\special{pa -984 689}\special{pa -965 689}\special{fp}\special{pa -945 689}\special{pa -926 689}\special{fp}%
\special{pa -906 689}\special{pa -887 689}\special{fp}\special{pa -867 689}\special{pa -848 689}\special{fp}%
\special{pa -828 689}\special{pa -809 689}\special{fp}\special{pa -789 689}\special{pa -770 689}\special{fp}%
\special{pa -750 689}\special{pa -731 689}\special{fp}\special{pa -711 689}\special{pa -692 689}\special{fp}%
\special{pa -672 689}\special{pa -653 689}\special{fp}\special{pa -633 689}\special{pa -614 689}\special{fp}%
\special{pa -594 689}\special{pa -575 689}\special{fp}\special{pa -555 689}\special{pa -536 689}\special{fp}%
\special{pa -516 689}\special{pa -497 689}\special{fp}\special{pa -478 689}\special{pa -458 689}\special{fp}%
\special{pa -439 689}\special{pa -419 689}\special{fp}\special{pa -400 689}\special{pa -380 689}\special{fp}%
\special{pa -361 689}\special{pa -341 689}\special{fp}\special{pa -322 689}\special{pa -302 689}\special{fp}%
\special{pa -283 689}\special{pa -263 689}\special{fp}\special{pa -244 689}\special{pa -224 689}\special{fp}%
\special{pa -205 689}\special{pa -185 689}\special{fp}\special{pa -166 689}\special{pa -146 689}\special{fp}%
\special{pa -127 689}\special{pa -107 689}\special{fp}\special{pa -88 689}\special{pa -68 689}\special{fp}%
\special{pa -49 689}\special{pa -29 689}\special{fp}\special{pa -10 689}\special{pa 10 689}\special{fp}%
\special{pa 29 689}\special{pa 49 689}\special{fp}\special{pa 68 689}\special{pa 88 689}\special{fp}%
\special{pa 107 689}\special{pa 127 689}\special{fp}\special{pa 146 689}\special{pa 166 689}\special{fp}%
\special{pa 185 689}\special{pa 205 689}\special{fp}\special{pa 224 689}\special{pa 244 689}\special{fp}%
\special{pa 263 689}\special{pa 283 689}\special{fp}\special{pa 302 689}\special{pa 322 689}\special{fp}%
\special{pa 341 689}\special{pa 361 689}\special{fp}\special{pa 380 689}\special{pa 400 689}\special{fp}%
\special{pa 419 689}\special{pa 439 689}\special{fp}\special{pa 458 689}\special{pa 478 689}\special{fp}%
\special{pa 497 689}\special{pa 516 689}\special{fp}\special{pa 536 689}\special{pa 555 689}\special{fp}%
\special{pa 575 689}\special{pa 594 689}\special{fp}\special{pa 614 689}\special{pa 633 689}\special{fp}%
\special{pa 653 689}\special{pa 672 689}\special{fp}\special{pa 692 689}\special{pa 711 689}\special{fp}%
\special{pa 731 689}\special{pa 750 689}\special{fp}\special{pa 770 689}\special{pa 789 689}\special{fp}%
\special{pa 809 689}\special{pa 828 689}\special{fp}\special{pa 848 689}\special{pa 867 689}\special{fp}%
\special{pa 887 689}\special{pa 906 689}\special{fp}\special{pa 926 689}\special{pa 945 689}\special{fp}%
\special{pa 965 689}\special{pa 984 689}\special{fp}%
%
\special{pa -984 492}\special{pa -965 492}\special{fp}\special{pa -945 492}\special{pa -926 492}\special{fp}%
\special{pa -906 492}\special{pa -887 492}\special{fp}\special{pa -867 492}\special{pa -848 492}\special{fp}%
\special{pa -828 492}\special{pa -809 492}\special{fp}\special{pa -789 492}\special{pa -770 492}\special{fp}%
\special{pa -750 492}\special{pa -731 492}\special{fp}\special{pa -711 492}\special{pa -692 492}\special{fp}%
\special{pa -672 492}\special{pa -653 492}\special{fp}\special{pa -633 492}\special{pa -614 492}\special{fp}%
\special{pa -594 492}\special{pa -575 492}\special{fp}\special{pa -555 492}\special{pa -536 492}\special{fp}%
\special{pa -516 492}\special{pa -497 492}\special{fp}\special{pa -478 492}\special{pa -458 492}\special{fp}%
\special{pa -439 492}\special{pa -419 492}\special{fp}\special{pa -400 492}\special{pa -380 492}\special{fp}%
\special{pa -361 492}\special{pa -341 492}\special{fp}\special{pa -322 492}\special{pa -302 492}\special{fp}%
\special{pa -283 492}\special{pa -263 492}\special{fp}\special{pa -244 492}\special{pa -224 492}\special{fp}%
\special{pa -205 492}\special{pa -185 492}\special{fp}\special{pa -166 492}\special{pa -146 492}\special{fp}%
\special{pa -127 492}\special{pa -107 492}\special{fp}\special{pa -88 492}\special{pa -68 492}\special{fp}%
\special{pa -49 492}\special{pa -29 492}\special{fp}\special{pa -10 492}\special{pa 10 492}\special{fp}%
\special{pa 29 492}\special{pa 49 492}\special{fp}\special{pa 68 492}\special{pa 88 492}\special{fp}%
\special{pa 107 492}\special{pa 127 492}\special{fp}\special{pa 146 492}\special{pa 166 492}\special{fp}%
\special{pa 185 492}\special{pa 205 492}\special{fp}\special{pa 224 492}\special{pa 244 492}\special{fp}%
\special{pa 263 492}\special{pa 283 492}\special{fp}\special{pa 302 492}\special{pa 322 492}\special{fp}%
\special{pa 341 492}\special{pa 361 492}\special{fp}\special{pa 380 492}\special{pa 400 492}\special{fp}%
\special{pa 419 492}\special{pa 439 492}\special{fp}\special{pa 458 492}\special{pa 478 492}\special{fp}%
\special{pa 497 492}\special{pa 516 492}\special{fp}\special{pa 536 492}\special{pa 555 492}\special{fp}%
\special{pa 575 492}\special{pa 594 492}\special{fp}\special{pa 614 492}\special{pa 633 492}\special{fp}%
\special{pa 653 492}\special{pa 672 492}\special{fp}\special{pa 692 492}\special{pa 711 492}\special{fp}%
\special{pa 731 492}\special{pa 750 492}\special{fp}\special{pa 770 492}\special{pa 789 492}\special{fp}%
\special{pa 809 492}\special{pa 828 492}\special{fp}\special{pa 848 492}\special{pa 867 492}\special{fp}%
\special{pa 887 492}\special{pa 906 492}\special{fp}\special{pa 926 492}\special{pa 945 492}\special{fp}%
\special{pa 965 492}\special{pa 984 492}\special{fp}%
%
\special{pa -984 295}\special{pa -965 295}\special{fp}\special{pa -945 295}\special{pa -926 295}\special{fp}%
\special{pa -906 295}\special{pa -887 295}\special{fp}\special{pa -867 295}\special{pa -848 295}\special{fp}%
\special{pa -828 295}\special{pa -809 295}\special{fp}\special{pa -789 295}\special{pa -770 295}\special{fp}%
\special{pa -750 295}\special{pa -731 295}\special{fp}\special{pa -711 295}\special{pa -692 295}\special{fp}%
\special{pa -672 295}\special{pa -653 295}\special{fp}\special{pa -633 295}\special{pa -614 295}\special{fp}%
\special{pa -594 295}\special{pa -575 295}\special{fp}\special{pa -555 295}\special{pa -536 295}\special{fp}%
\special{pa -516 295}\special{pa -497 295}\special{fp}\special{pa -478 295}\special{pa -458 295}\special{fp}%
\special{pa -439 295}\special{pa -419 295}\special{fp}\special{pa -400 295}\special{pa -380 295}\special{fp}%
\special{pa -361 295}\special{pa -341 295}\special{fp}\special{pa -322 295}\special{pa -302 295}\special{fp}%
\special{pa -283 295}\special{pa -263 295}\special{fp}\special{pa -244 295}\special{pa -224 295}\special{fp}%
\special{pa -205 295}\special{pa -185 295}\special{fp}\special{pa -166 295}\special{pa -146 295}\special{fp}%
\special{pa -127 295}\special{pa -107 295}\special{fp}\special{pa -88 295}\special{pa -68 295}\special{fp}%
\special{pa -49 295}\special{pa -29 295}\special{fp}\special{pa -10 295}\special{pa 10 295}\special{fp}%
\special{pa 29 295}\special{pa 49 295}\special{fp}\special{pa 68 295}\special{pa 88 295}\special{fp}%
\special{pa 107 295}\special{pa 127 295}\special{fp}\special{pa 146 295}\special{pa 166 295}\special{fp}%
\special{pa 185 295}\special{pa 205 295}\special{fp}\special{pa 224 295}\special{pa 244 295}\special{fp}%
\special{pa 263 295}\special{pa 283 295}\special{fp}\special{pa 302 295}\special{pa 322 295}\special{fp}%
\special{pa 341 295}\special{pa 361 295}\special{fp}\special{pa 380 295}\special{pa 400 295}\special{fp}%
\special{pa 419 295}\special{pa 439 295}\special{fp}\special{pa 458 295}\special{pa 478 295}\special{fp}%
\special{pa 497 295}\special{pa 516 295}\special{fp}\special{pa 536 295}\special{pa 555 295}\special{fp}%
\special{pa 575 295}\special{pa 594 295}\special{fp}\special{pa 614 295}\special{pa 633 295}\special{fp}%
\special{pa 653 295}\special{pa 672 295}\special{fp}\special{pa 692 295}\special{pa 711 295}\special{fp}%
\special{pa 731 295}\special{pa 750 295}\special{fp}\special{pa 770 295}\special{pa 789 295}\special{fp}%
\special{pa 809 295}\special{pa 828 295}\special{fp}\special{pa 848 295}\special{pa 867 295}\special{fp}%
\special{pa 887 295}\special{pa 906 295}\special{fp}\special{pa 926 295}\special{pa 945 295}\special{fp}%
\special{pa 965 295}\special{pa 984 295}\special{fp}%
%
\special{pa -984 98}\special{pa -965 98}\special{fp}\special{pa -945 98}\special{pa -926 98}\special{fp}%
\special{pa -906 98}\special{pa -887 98}\special{fp}\special{pa -867 98}\special{pa -848 98}\special{fp}%
\special{pa -828 98}\special{pa -809 98}\special{fp}\special{pa -789 98}\special{pa -770 98}\special{fp}%
\special{pa -750 98}\special{pa -731 98}\special{fp}\special{pa -711 98}\special{pa -692 98}\special{fp}%
\special{pa -672 98}\special{pa -653 98}\special{fp}\special{pa -633 98}\special{pa -614 98}\special{fp}%
\special{pa -594 98}\special{pa -575 98}\special{fp}\special{pa -555 98}\special{pa -536 98}\special{fp}%
\special{pa -516 98}\special{pa -497 98}\special{fp}\special{pa -478 98}\special{pa -458 98}\special{fp}%
\special{pa -439 98}\special{pa -419 98}\special{fp}\special{pa -400 98}\special{pa -380 98}\special{fp}%
\special{pa -361 98}\special{pa -341 98}\special{fp}\special{pa -322 98}\special{pa -302 98}\special{fp}%
\special{pa -283 98}\special{pa -263 98}\special{fp}\special{pa -244 98}\special{pa -224 98}\special{fp}%
\special{pa -205 98}\special{pa -185 98}\special{fp}\special{pa -166 98}\special{pa -146 98}\special{fp}%
\special{pa -127 98}\special{pa -107 98}\special{fp}\special{pa -88 98}\special{pa -68 98}\special{fp}%
\special{pa -49 98}\special{pa -29 98}\special{fp}\special{pa -10 98}\special{pa 10 98}\special{fp}%
\special{pa 29 98}\special{pa 49 98}\special{fp}\special{pa 68 98}\special{pa 88 98}\special{fp}%
\special{pa 107 98}\special{pa 127 98}\special{fp}\special{pa 146 98}\special{pa 166 98}\special{fp}%
\special{pa 185 98}\special{pa 205 98}\special{fp}\special{pa 224 98}\special{pa 244 98}\special{fp}%
\special{pa 263 98}\special{pa 283 98}\special{fp}\special{pa 302 98}\special{pa 322 98}\special{fp}%
\special{pa 341 98}\special{pa 361 98}\special{fp}\special{pa 380 98}\special{pa 400 98}\special{fp}%
\special{pa 419 98}\special{pa 439 98}\special{fp}\special{pa 458 98}\special{pa 478 98}\special{fp}%
\special{pa 497 98}\special{pa 516 98}\special{fp}\special{pa 536 98}\special{pa 555 98}\special{fp}%
\special{pa 575 98}\special{pa 594 98}\special{fp}\special{pa 614 98}\special{pa 633 98}\special{fp}%
\special{pa 653 98}\special{pa 672 98}\special{fp}\special{pa 692 98}\special{pa 711 98}\special{fp}%
\special{pa 731 98}\special{pa 750 98}\special{fp}\special{pa 770 98}\special{pa 789 98}\special{fp}%
\special{pa 809 98}\special{pa 828 98}\special{fp}\special{pa 848 98}\special{pa 867 98}\special{fp}%
\special{pa 887 98}\special{pa 906 98}\special{fp}\special{pa 926 98}\special{pa 945 98}\special{fp}%
\special{pa 965 98}\special{pa 984 98}\special{fp}%
%
\special{pa -984 -98}\special{pa -965 -98}\special{fp}\special{pa -945 -98}\special{pa -926 -98}\special{fp}%
\special{pa -906 -98}\special{pa -887 -98}\special{fp}\special{pa -867 -98}\special{pa -848 -98}\special{fp}%
\special{pa -828 -98}\special{pa -809 -98}\special{fp}\special{pa -789 -98}\special{pa -770 -98}\special{fp}%
\special{pa -750 -98}\special{pa -731 -98}\special{fp}\special{pa -711 -98}\special{pa -692 -98}\special{fp}%
\special{pa -672 -98}\special{pa -653 -98}\special{fp}\special{pa -633 -98}\special{pa -614 -98}\special{fp}%
\special{pa -594 -98}\special{pa -575 -98}\special{fp}\special{pa -555 -98}\special{pa -536 -98}\special{fp}%
\special{pa -516 -98}\special{pa -497 -98}\special{fp}\special{pa -478 -98}\special{pa -458 -98}\special{fp}%
\special{pa -439 -98}\special{pa -419 -98}\special{fp}\special{pa -400 -98}\special{pa -380 -98}\special{fp}%
\special{pa -361 -98}\special{pa -341 -98}\special{fp}\special{pa -322 -98}\special{pa -302 -98}\special{fp}%
\special{pa -283 -98}\special{pa -263 -98}\special{fp}\special{pa -244 -98}\special{pa -224 -98}\special{fp}%
\special{pa -205 -98}\special{pa -185 -98}\special{fp}\special{pa -166 -98}\special{pa -146 -98}\special{fp}%
\special{pa -127 -98}\special{pa -107 -98}\special{fp}\special{pa -88 -98}\special{pa -68 -98}\special{fp}%
\special{pa -49 -98}\special{pa -29 -98}\special{fp}\special{pa -10 -98}\special{pa 10 -98}\special{fp}%
\special{pa 29 -98}\special{pa 49 -98}\special{fp}\special{pa 68 -98}\special{pa 88 -98}\special{fp}%
\special{pa 107 -98}\special{pa 127 -98}\special{fp}\special{pa 146 -98}\special{pa 166 -98}\special{fp}%
\special{pa 185 -98}\special{pa 205 -98}\special{fp}\special{pa 224 -98}\special{pa 244 -98}\special{fp}%
\special{pa 263 -98}\special{pa 283 -98}\special{fp}\special{pa 302 -98}\special{pa 322 -98}\special{fp}%
\special{pa 341 -98}\special{pa 361 -98}\special{fp}\special{pa 380 -98}\special{pa 400 -98}\special{fp}%
\special{pa 419 -98}\special{pa 439 -98}\special{fp}\special{pa 458 -98}\special{pa 478 -98}\special{fp}%
\special{pa 497 -98}\special{pa 516 -98}\special{fp}\special{pa 536 -98}\special{pa 555 -98}\special{fp}%
\special{pa 575 -98}\special{pa 594 -98}\special{fp}\special{pa 614 -98}\special{pa 633 -98}\special{fp}%
\special{pa 653 -98}\special{pa 672 -98}\special{fp}\special{pa 692 -98}\special{pa 711 -98}\special{fp}%
\special{pa 731 -98}\special{pa 750 -98}\special{fp}\special{pa 770 -98}\special{pa 789 -98}\special{fp}%
\special{pa 809 -98}\special{pa 828 -98}\special{fp}\special{pa 848 -98}\special{pa 867 -98}\special{fp}%
\special{pa 887 -98}\special{pa 906 -98}\special{fp}\special{pa 926 -98}\special{pa 945 -98}\special{fp}%
\special{pa 965 -98}\special{pa 984 -98}\special{fp}%
%
\special{pa -984 -295}\special{pa -965 -295}\special{fp}\special{pa -945 -295}\special{pa -926 -295}\special{fp}%
\special{pa -906 -295}\special{pa -887 -295}\special{fp}\special{pa -867 -295}\special{pa -848 -295}\special{fp}%
\special{pa -828 -295}\special{pa -809 -295}\special{fp}\special{pa -789 -295}\special{pa -770 -295}\special{fp}%
\special{pa -750 -295}\special{pa -731 -295}\special{fp}\special{pa -711 -295}\special{pa -692 -295}\special{fp}%
\special{pa -672 -295}\special{pa -653 -295}\special{fp}\special{pa -633 -295}\special{pa -614 -295}\special{fp}%
\special{pa -594 -295}\special{pa -575 -295}\special{fp}\special{pa -555 -295}\special{pa -536 -295}\special{fp}%
\special{pa -516 -295}\special{pa -497 -295}\special{fp}\special{pa -478 -295}\special{pa -458 -295}\special{fp}%
\special{pa -439 -295}\special{pa -419 -295}\special{fp}\special{pa -400 -295}\special{pa -380 -295}\special{fp}%
\special{pa -361 -295}\special{pa -341 -295}\special{fp}\special{pa -322 -295}\special{pa -302 -295}\special{fp}%
\special{pa -283 -295}\special{pa -263 -295}\special{fp}\special{pa -244 -295}\special{pa -224 -295}\special{fp}%
\special{pa -205 -295}\special{pa -185 -295}\special{fp}\special{pa -166 -295}\special{pa -146 -295}\special{fp}%
\special{pa -127 -295}\special{pa -107 -295}\special{fp}\special{pa -88 -295}\special{pa -68 -295}\special{fp}%
\special{pa -49 -295}\special{pa -29 -295}\special{fp}\special{pa -10 -295}\special{pa 10 -295}\special{fp}%
\special{pa 29 -295}\special{pa 49 -295}\special{fp}\special{pa 68 -295}\special{pa 88 -295}\special{fp}%
\special{pa 107 -295}\special{pa 127 -295}\special{fp}\special{pa 146 -295}\special{pa 166 -295}\special{fp}%
\special{pa 185 -295}\special{pa 205 -295}\special{fp}\special{pa 224 -295}\special{pa 244 -295}\special{fp}%
\special{pa 263 -295}\special{pa 283 -295}\special{fp}\special{pa 302 -295}\special{pa 322 -295}\special{fp}%
\special{pa 341 -295}\special{pa 361 -295}\special{fp}\special{pa 380 -295}\special{pa 400 -295}\special{fp}%
\special{pa 419 -295}\special{pa 439 -295}\special{fp}\special{pa 458 -295}\special{pa 478 -295}\special{fp}%
\special{pa 497 -295}\special{pa 516 -295}\special{fp}\special{pa 536 -295}\special{pa 555 -295}\special{fp}%
\special{pa 575 -295}\special{pa 594 -295}\special{fp}\special{pa 614 -295}\special{pa 633 -295}\special{fp}%
\special{pa 653 -295}\special{pa 672 -295}\special{fp}\special{pa 692 -295}\special{pa 711 -295}\special{fp}%
\special{pa 731 -295}\special{pa 750 -295}\special{fp}\special{pa 770 -295}\special{pa 789 -295}\special{fp}%
\special{pa 809 -295}\special{pa 828 -295}\special{fp}\special{pa 848 -295}\special{pa 867 -295}\special{fp}%
\special{pa 887 -295}\special{pa 906 -295}\special{fp}\special{pa 926 -295}\special{pa 945 -295}\special{fp}%
\special{pa 965 -295}\special{pa 984 -295}\special{fp}%
%
\special{pa -984 -492}\special{pa -965 -492}\special{fp}\special{pa -945 -492}\special{pa -926 -492}\special{fp}%
\special{pa -906 -492}\special{pa -887 -492}\special{fp}\special{pa -867 -492}\special{pa -848 -492}\special{fp}%
\special{pa -828 -492}\special{pa -809 -492}\special{fp}\special{pa -789 -492}\special{pa -770 -492}\special{fp}%
\special{pa -750 -492}\special{pa -731 -492}\special{fp}\special{pa -711 -492}\special{pa -692 -492}\special{fp}%
\special{pa -672 -492}\special{pa -653 -492}\special{fp}\special{pa -633 -492}\special{pa -614 -492}\special{fp}%
\special{pa -594 -492}\special{pa -575 -492}\special{fp}\special{pa -555 -492}\special{pa -536 -492}\special{fp}%
\special{pa -516 -492}\special{pa -497 -492}\special{fp}\special{pa -478 -492}\special{pa -458 -492}\special{fp}%
\special{pa -439 -492}\special{pa -419 -492}\special{fp}\special{pa -400 -492}\special{pa -380 -492}\special{fp}%
\special{pa -361 -492}\special{pa -341 -492}\special{fp}\special{pa -322 -492}\special{pa -302 -492}\special{fp}%
\special{pa -283 -492}\special{pa -263 -492}\special{fp}\special{pa -244 -492}\special{pa -224 -492}\special{fp}%
\special{pa -205 -492}\special{pa -185 -492}\special{fp}\special{pa -166 -492}\special{pa -146 -492}\special{fp}%
\special{pa -127 -492}\special{pa -107 -492}\special{fp}\special{pa -88 -492}\special{pa -68 -492}\special{fp}%
\special{pa -49 -492}\special{pa -29 -492}\special{fp}\special{pa -10 -492}\special{pa 10 -492}\special{fp}%
\special{pa 29 -492}\special{pa 49 -492}\special{fp}\special{pa 68 -492}\special{pa 88 -492}\special{fp}%
\special{pa 107 -492}\special{pa 127 -492}\special{fp}\special{pa 146 -492}\special{pa 166 -492}\special{fp}%
\special{pa 185 -492}\special{pa 205 -492}\special{fp}\special{pa 224 -492}\special{pa 244 -492}\special{fp}%
\special{pa 263 -492}\special{pa 283 -492}\special{fp}\special{pa 302 -492}\special{pa 322 -492}\special{fp}%
\special{pa 341 -492}\special{pa 361 -492}\special{fp}\special{pa 380 -492}\special{pa 400 -492}\special{fp}%
\special{pa 419 -492}\special{pa 439 -492}\special{fp}\special{pa 458 -492}\special{pa 478 -492}\special{fp}%
\special{pa 497 -492}\special{pa 516 -492}\special{fp}\special{pa 536 -492}\special{pa 555 -492}\special{fp}%
\special{pa 575 -492}\special{pa 594 -492}\special{fp}\special{pa 614 -492}\special{pa 633 -492}\special{fp}%
\special{pa 653 -492}\special{pa 672 -492}\special{fp}\special{pa 692 -492}\special{pa 711 -492}\special{fp}%
\special{pa 731 -492}\special{pa 750 -492}\special{fp}\special{pa 770 -492}\special{pa 789 -492}\special{fp}%
\special{pa 809 -492}\special{pa 828 -492}\special{fp}\special{pa 848 -492}\special{pa 867 -492}\special{fp}%
\special{pa 887 -492}\special{pa 906 -492}\special{fp}\special{pa 926 -492}\special{pa 945 -492}\special{fp}%
\special{pa 965 -492}\special{pa 984 -492}\special{fp}%
%
\special{pa -984 -689}\special{pa -965 -689}\special{fp}\special{pa -945 -689}\special{pa -926 -689}\special{fp}%
\special{pa -906 -689}\special{pa -887 -689}\special{fp}\special{pa -867 -689}\special{pa -848 -689}\special{fp}%
\special{pa -828 -689}\special{pa -809 -689}\special{fp}\special{pa -789 -689}\special{pa -770 -689}\special{fp}%
\special{pa -750 -689}\special{pa -731 -689}\special{fp}\special{pa -711 -689}\special{pa -692 -689}\special{fp}%
\special{pa -672 -689}\special{pa -653 -689}\special{fp}\special{pa -633 -689}\special{pa -614 -689}\special{fp}%
\special{pa -594 -689}\special{pa -575 -689}\special{fp}\special{pa -555 -689}\special{pa -536 -689}\special{fp}%
\special{pa -516 -689}\special{pa -497 -689}\special{fp}\special{pa -478 -689}\special{pa -458 -689}\special{fp}%
\special{pa -439 -689}\special{pa -419 -689}\special{fp}\special{pa -400 -689}\special{pa -380 -689}\special{fp}%
\special{pa -361 -689}\special{pa -341 -689}\special{fp}\special{pa -322 -689}\special{pa -302 -689}\special{fp}%
\special{pa -283 -689}\special{pa -263 -689}\special{fp}\special{pa -244 -689}\special{pa -224 -689}\special{fp}%
\special{pa -205 -689}\special{pa -185 -689}\special{fp}\special{pa -166 -689}\special{pa -146 -689}\special{fp}%
\special{pa -127 -689}\special{pa -107 -689}\special{fp}\special{pa -88 -689}\special{pa -68 -689}\special{fp}%
\special{pa -49 -689}\special{pa -29 -689}\special{fp}\special{pa -10 -689}\special{pa 10 -689}\special{fp}%
\special{pa 29 -689}\special{pa 49 -689}\special{fp}\special{pa 68 -689}\special{pa 88 -689}\special{fp}%
\special{pa 107 -689}\special{pa 127 -689}\special{fp}\special{pa 146 -689}\special{pa 166 -689}\special{fp}%
\special{pa 185 -689}\special{pa 205 -689}\special{fp}\special{pa 224 -689}\special{pa 244 -689}\special{fp}%
\special{pa 263 -689}\special{pa 283 -689}\special{fp}\special{pa 302 -689}\special{pa 322 -689}\special{fp}%
\special{pa 341 -689}\special{pa 361 -689}\special{fp}\special{pa 380 -689}\special{pa 400 -689}\special{fp}%
\special{pa 419 -689}\special{pa 439 -689}\special{fp}\special{pa 458 -689}\special{pa 478 -689}\special{fp}%
\special{pa 497 -689}\special{pa 516 -689}\special{fp}\special{pa 536 -689}\special{pa 555 -689}\special{fp}%
\special{pa 575 -689}\special{pa 594 -689}\special{fp}\special{pa 614 -689}\special{pa 633 -689}\special{fp}%
\special{pa 653 -689}\special{pa 672 -689}\special{fp}\special{pa 692 -689}\special{pa 711 -689}\special{fp}%
\special{pa 731 -689}\special{pa 750 -689}\special{fp}\special{pa 770 -689}\special{pa 789 -689}\special{fp}%
\special{pa 809 -689}\special{pa 828 -689}\special{fp}\special{pa 848 -689}\special{pa 867 -689}\special{fp}%
\special{pa 887 -689}\special{pa 906 -689}\special{fp}\special{pa 926 -689}\special{pa 945 -689}\special{fp}%
\special{pa 965 -689}\special{pa 984 -689}\special{fp}%
%
\special{pa -984 -886}\special{pa -965 -886}\special{fp}\special{pa -945 -886}\special{pa -926 -886}\special{fp}%
\special{pa -906 -886}\special{pa -887 -886}\special{fp}\special{pa -867 -886}\special{pa -848 -886}\special{fp}%
\special{pa -828 -886}\special{pa -809 -886}\special{fp}\special{pa -789 -886}\special{pa -770 -886}\special{fp}%
\special{pa -750 -886}\special{pa -731 -886}\special{fp}\special{pa -711 -886}\special{pa -692 -886}\special{fp}%
\special{pa -672 -886}\special{pa -653 -886}\special{fp}\special{pa -633 -886}\special{pa -614 -886}\special{fp}%
\special{pa -594 -886}\special{pa -575 -886}\special{fp}\special{pa -555 -886}\special{pa -536 -886}\special{fp}%
\special{pa -516 -886}\special{pa -497 -886}\special{fp}\special{pa -478 -886}\special{pa -458 -886}\special{fp}%
\special{pa -439 -886}\special{pa -419 -886}\special{fp}\special{pa -400 -886}\special{pa -380 -886}\special{fp}%
\special{pa -361 -886}\special{pa -341 -886}\special{fp}\special{pa -322 -886}\special{pa -302 -886}\special{fp}%
\special{pa -283 -886}\special{pa -263 -886}\special{fp}\special{pa -244 -886}\special{pa -224 -886}\special{fp}%
\special{pa -205 -886}\special{pa -185 -886}\special{fp}\special{pa -166 -886}\special{pa -146 -886}\special{fp}%
\special{pa -127 -886}\special{pa -107 -886}\special{fp}\special{pa -88 -886}\special{pa -68 -886}\special{fp}%
\special{pa -49 -886}\special{pa -29 -886}\special{fp}\special{pa -10 -886}\special{pa 10 -886}\special{fp}%
\special{pa 29 -886}\special{pa 49 -886}\special{fp}\special{pa 68 -886}\special{pa 88 -886}\special{fp}%
\special{pa 107 -886}\special{pa 127 -886}\special{fp}\special{pa 146 -886}\special{pa 166 -886}\special{fp}%
\special{pa 185 -886}\special{pa 205 -886}\special{fp}\special{pa 224 -886}\special{pa 244 -886}\special{fp}%
\special{pa 263 -886}\special{pa 283 -886}\special{fp}\special{pa 302 -886}\special{pa 322 -886}\special{fp}%
\special{pa 341 -886}\special{pa 361 -886}\special{fp}\special{pa 380 -886}\special{pa 400 -886}\special{fp}%
\special{pa 419 -886}\special{pa 439 -886}\special{fp}\special{pa 458 -886}\special{pa 478 -886}\special{fp}%
\special{pa 497 -886}\special{pa 516 -886}\special{fp}\special{pa 536 -886}\special{pa 555 -886}\special{fp}%
\special{pa 575 -886}\special{pa 594 -886}\special{fp}\special{pa 614 -886}\special{pa 633 -886}\special{fp}%
\special{pa 653 -886}\special{pa 672 -886}\special{fp}\special{pa 692 -886}\special{pa 711 -886}\special{fp}%
\special{pa 731 -886}\special{pa 750 -886}\special{fp}\special{pa 770 -886}\special{pa 789 -886}\special{fp}%
\special{pa 809 -886}\special{pa 828 -886}\special{fp}\special{pa 848 -886}\special{pa 867 -886}\special{fp}%
\special{pa 887 -886}\special{pa 906 -886}\special{fp}\special{pa 926 -886}\special{pa 945 -886}\special{fp}%
\special{pa 965 -886}\special{pa 984 -886}\special{fp}%
%
\special{pa  -984   984}\special{pa   984   984}%
\special{fp}%
\special{pa  -984   787}\special{pa   984   787}%
\special{fp}%
\special{pa  -984   591}\special{pa   984   591}%
\special{fp}%
\special{pa  -984   394}\special{pa   984   394}%
\special{fp}%
\special{pa  -984   197}\special{pa   984   197}%
\special{fp}%
\special{pa  -984    -0}\special{pa   984    -0}%
\special{fp}%
\special{pa  -984  -197}\special{pa   984  -197}%
\special{fp}%
\special{pa  -984  -394}\special{pa   984  -394}%
\special{fp}%
\special{pa  -984  -591}\special{pa   984  -591}%
\special{fp}%
\special{pa  -984  -787}\special{pa   984  -787}%
\special{fp}%
\special{pa  -984  -984}\special{pa   984  -984}%
\special{fp}%
\special{pa -886 984}\special{pa -886 965}\special{fp}\special{pa -886 945}\special{pa -886 926}\special{fp}%
\special{pa -886 906}\special{pa -886 887}\special{fp}\special{pa -886 867}\special{pa -886 848}\special{fp}%
\special{pa -886 828}\special{pa -886 809}\special{fp}\special{pa -886 789}\special{pa -886 770}\special{fp}%
\special{pa -886 750}\special{pa -886 731}\special{fp}\special{pa -886 711}\special{pa -886 692}\special{fp}%
\special{pa -886 672}\special{pa -886 653}\special{fp}\special{pa -886 633}\special{pa -886 614}\special{fp}%
\special{pa -886 594}\special{pa -886 575}\special{fp}\special{pa -886 555}\special{pa -886 536}\special{fp}%
\special{pa -886 516}\special{pa -886 497}\special{fp}\special{pa -886 478}\special{pa -886 458}\special{fp}%
\special{pa -886 439}\special{pa -886 419}\special{fp}\special{pa -886 400}\special{pa -886 380}\special{fp}%
\special{pa -886 361}\special{pa -886 341}\special{fp}\special{pa -886 322}\special{pa -886 302}\special{fp}%
\special{pa -886 283}\special{pa -886 263}\special{fp}\special{pa -886 244}\special{pa -886 224}\special{fp}%
\special{pa -886 205}\special{pa -886 185}\special{fp}\special{pa -886 166}\special{pa -886 146}\special{fp}%
\special{pa -886 127}\special{pa -886 107}\special{fp}\special{pa -886 88}\special{pa -886 68}\special{fp}%
\special{pa -886 49}\special{pa -886 29}\special{fp}\special{pa -886 10}\special{pa -886 -10}\special{fp}%
\special{pa -886 -29}\special{pa -886 -49}\special{fp}\special{pa -886 -68}\special{pa -886 -88}\special{fp}%
\special{pa -886 -107}\special{pa -886 -127}\special{fp}\special{pa -886 -146}\special{pa -886 -166}\special{fp}%
\special{pa -886 -185}\special{pa -886 -205}\special{fp}\special{pa -886 -224}\special{pa -886 -244}\special{fp}%
\special{pa -886 -263}\special{pa -886 -283}\special{fp}\special{pa -886 -302}\special{pa -886 -322}\special{fp}%
\special{pa -886 -341}\special{pa -886 -361}\special{fp}\special{pa -886 -380}\special{pa -886 -400}\special{fp}%
\special{pa -886 -419}\special{pa -886 -439}\special{fp}\special{pa -886 -458}\special{pa -886 -478}\special{fp}%
\special{pa -886 -497}\special{pa -886 -516}\special{fp}\special{pa -886 -536}\special{pa -886 -555}\special{fp}%
\special{pa -886 -575}\special{pa -886 -594}\special{fp}\special{pa -886 -614}\special{pa -886 -633}\special{fp}%
\special{pa -886 -653}\special{pa -886 -672}\special{fp}\special{pa -886 -692}\special{pa -886 -711}\special{fp}%
\special{pa -886 -731}\special{pa -886 -750}\special{fp}\special{pa -886 -770}\special{pa -886 -789}\special{fp}%
\special{pa -886 -809}\special{pa -886 -828}\special{fp}\special{pa -886 -848}\special{pa -886 -867}\special{fp}%
\special{pa -886 -887}\special{pa -886 -906}\special{fp}\special{pa -886 -926}\special{pa -886 -945}\special{fp}%
\special{pa -886 -965}\special{pa -886 -984}\special{fp}%
%
\special{pa -689 984}\special{pa -689 965}\special{fp}\special{pa -689 945}\special{pa -689 926}\special{fp}%
\special{pa -689 906}\special{pa -689 887}\special{fp}\special{pa -689 867}\special{pa -689 848}\special{fp}%
\special{pa -689 828}\special{pa -689 809}\special{fp}\special{pa -689 789}\special{pa -689 770}\special{fp}%
\special{pa -689 750}\special{pa -689 731}\special{fp}\special{pa -689 711}\special{pa -689 692}\special{fp}%
\special{pa -689 672}\special{pa -689 653}\special{fp}\special{pa -689 633}\special{pa -689 614}\special{fp}%
\special{pa -689 594}\special{pa -689 575}\special{fp}\special{pa -689 555}\special{pa -689 536}\special{fp}%
\special{pa -689 516}\special{pa -689 497}\special{fp}\special{pa -689 478}\special{pa -689 458}\special{fp}%
\special{pa -689 439}\special{pa -689 419}\special{fp}\special{pa -689 400}\special{pa -689 380}\special{fp}%
\special{pa -689 361}\special{pa -689 341}\special{fp}\special{pa -689 322}\special{pa -689 302}\special{fp}%
\special{pa -689 283}\special{pa -689 263}\special{fp}\special{pa -689 244}\special{pa -689 224}\special{fp}%
\special{pa -689 205}\special{pa -689 185}\special{fp}\special{pa -689 166}\special{pa -689 146}\special{fp}%
\special{pa -689 127}\special{pa -689 107}\special{fp}\special{pa -689 88}\special{pa -689 68}\special{fp}%
\special{pa -689 49}\special{pa -689 29}\special{fp}\special{pa -689 10}\special{pa -689 -10}\special{fp}%
\special{pa -689 -29}\special{pa -689 -49}\special{fp}\special{pa -689 -68}\special{pa -689 -88}\special{fp}%
\special{pa -689 -107}\special{pa -689 -127}\special{fp}\special{pa -689 -146}\special{pa -689 -166}\special{fp}%
\special{pa -689 -185}\special{pa -689 -205}\special{fp}\special{pa -689 -224}\special{pa -689 -244}\special{fp}%
\special{pa -689 -263}\special{pa -689 -283}\special{fp}\special{pa -689 -302}\special{pa -689 -322}\special{fp}%
\special{pa -689 -341}\special{pa -689 -361}\special{fp}\special{pa -689 -380}\special{pa -689 -400}\special{fp}%
\special{pa -689 -419}\special{pa -689 -439}\special{fp}\special{pa -689 -458}\special{pa -689 -478}\special{fp}%
\special{pa -689 -497}\special{pa -689 -516}\special{fp}\special{pa -689 -536}\special{pa -689 -555}\special{fp}%
\special{pa -689 -575}\special{pa -689 -594}\special{fp}\special{pa -689 -614}\special{pa -689 -633}\special{fp}%
\special{pa -689 -653}\special{pa -689 -672}\special{fp}\special{pa -689 -692}\special{pa -689 -711}\special{fp}%
\special{pa -689 -731}\special{pa -689 -750}\special{fp}\special{pa -689 -770}\special{pa -689 -789}\special{fp}%
\special{pa -689 -809}\special{pa -689 -828}\special{fp}\special{pa -689 -848}\special{pa -689 -867}\special{fp}%
\special{pa -689 -887}\special{pa -689 -906}\special{fp}\special{pa -689 -926}\special{pa -689 -945}\special{fp}%
\special{pa -689 -965}\special{pa -689 -984}\special{fp}%
%
\special{pa -492 984}\special{pa -492 965}\special{fp}\special{pa -492 945}\special{pa -492 926}\special{fp}%
\special{pa -492 906}\special{pa -492 887}\special{fp}\special{pa -492 867}\special{pa -492 848}\special{fp}%
\special{pa -492 828}\special{pa -492 809}\special{fp}\special{pa -492 789}\special{pa -492 770}\special{fp}%
\special{pa -492 750}\special{pa -492 731}\special{fp}\special{pa -492 711}\special{pa -492 692}\special{fp}%
\special{pa -492 672}\special{pa -492 653}\special{fp}\special{pa -492 633}\special{pa -492 614}\special{fp}%
\special{pa -492 594}\special{pa -492 575}\special{fp}\special{pa -492 555}\special{pa -492 536}\special{fp}%
\special{pa -492 516}\special{pa -492 497}\special{fp}\special{pa -492 478}\special{pa -492 458}\special{fp}%
\special{pa -492 439}\special{pa -492 419}\special{fp}\special{pa -492 400}\special{pa -492 380}\special{fp}%
\special{pa -492 361}\special{pa -492 341}\special{fp}\special{pa -492 322}\special{pa -492 302}\special{fp}%
\special{pa -492 283}\special{pa -492 263}\special{fp}\special{pa -492 244}\special{pa -492 224}\special{fp}%
\special{pa -492 205}\special{pa -492 185}\special{fp}\special{pa -492 166}\special{pa -492 146}\special{fp}%
\special{pa -492 127}\special{pa -492 107}\special{fp}\special{pa -492 88}\special{pa -492 68}\special{fp}%
\special{pa -492 49}\special{pa -492 29}\special{fp}\special{pa -492 10}\special{pa -492 -10}\special{fp}%
\special{pa -492 -29}\special{pa -492 -49}\special{fp}\special{pa -492 -68}\special{pa -492 -88}\special{fp}%
\special{pa -492 -107}\special{pa -492 -127}\special{fp}\special{pa -492 -146}\special{pa -492 -166}\special{fp}%
\special{pa -492 -185}\special{pa -492 -205}\special{fp}\special{pa -492 -224}\special{pa -492 -244}\special{fp}%
\special{pa -492 -263}\special{pa -492 -283}\special{fp}\special{pa -492 -302}\special{pa -492 -322}\special{fp}%
\special{pa -492 -341}\special{pa -492 -361}\special{fp}\special{pa -492 -380}\special{pa -492 -400}\special{fp}%
\special{pa -492 -419}\special{pa -492 -439}\special{fp}\special{pa -492 -458}\special{pa -492 -478}\special{fp}%
\special{pa -492 -497}\special{pa -492 -516}\special{fp}\special{pa -492 -536}\special{pa -492 -555}\special{fp}%
\special{pa -492 -575}\special{pa -492 -594}\special{fp}\special{pa -492 -614}\special{pa -492 -633}\special{fp}%
\special{pa -492 -653}\special{pa -492 -672}\special{fp}\special{pa -492 -692}\special{pa -492 -711}\special{fp}%
\special{pa -492 -731}\special{pa -492 -750}\special{fp}\special{pa -492 -770}\special{pa -492 -789}\special{fp}%
\special{pa -492 -809}\special{pa -492 -828}\special{fp}\special{pa -492 -848}\special{pa -492 -867}\special{fp}%
\special{pa -492 -887}\special{pa -492 -906}\special{fp}\special{pa -492 -926}\special{pa -492 -945}\special{fp}%
\special{pa -492 -965}\special{pa -492 -984}\special{fp}%
%
\special{pa -295 984}\special{pa -295 965}\special{fp}\special{pa -295 945}\special{pa -295 926}\special{fp}%
\special{pa -295 906}\special{pa -295 887}\special{fp}\special{pa -295 867}\special{pa -295 848}\special{fp}%
\special{pa -295 828}\special{pa -295 809}\special{fp}\special{pa -295 789}\special{pa -295 770}\special{fp}%
\special{pa -295 750}\special{pa -295 731}\special{fp}\special{pa -295 711}\special{pa -295 692}\special{fp}%
\special{pa -295 672}\special{pa -295 653}\special{fp}\special{pa -295 633}\special{pa -295 614}\special{fp}%
\special{pa -295 594}\special{pa -295 575}\special{fp}\special{pa -295 555}\special{pa -295 536}\special{fp}%
\special{pa -295 516}\special{pa -295 497}\special{fp}\special{pa -295 478}\special{pa -295 458}\special{fp}%
\special{pa -295 439}\special{pa -295 419}\special{fp}\special{pa -295 400}\special{pa -295 380}\special{fp}%
\special{pa -295 361}\special{pa -295 341}\special{fp}\special{pa -295 322}\special{pa -295 302}\special{fp}%
\special{pa -295 283}\special{pa -295 263}\special{fp}\special{pa -295 244}\special{pa -295 224}\special{fp}%
\special{pa -295 205}\special{pa -295 185}\special{fp}\special{pa -295 166}\special{pa -295 146}\special{fp}%
\special{pa -295 127}\special{pa -295 107}\special{fp}\special{pa -295 88}\special{pa -295 68}\special{fp}%
\special{pa -295 49}\special{pa -295 29}\special{fp}\special{pa -295 10}\special{pa -295 -10}\special{fp}%
\special{pa -295 -29}\special{pa -295 -49}\special{fp}\special{pa -295 -68}\special{pa -295 -88}\special{fp}%
\special{pa -295 -107}\special{pa -295 -127}\special{fp}\special{pa -295 -146}\special{pa -295 -166}\special{fp}%
\special{pa -295 -185}\special{pa -295 -205}\special{fp}\special{pa -295 -224}\special{pa -295 -244}\special{fp}%
\special{pa -295 -263}\special{pa -295 -283}\special{fp}\special{pa -295 -302}\special{pa -295 -322}\special{fp}%
\special{pa -295 -341}\special{pa -295 -361}\special{fp}\special{pa -295 -380}\special{pa -295 -400}\special{fp}%
\special{pa -295 -419}\special{pa -295 -439}\special{fp}\special{pa -295 -458}\special{pa -295 -478}\special{fp}%
\special{pa -295 -497}\special{pa -295 -516}\special{fp}\special{pa -295 -536}\special{pa -295 -555}\special{fp}%
\special{pa -295 -575}\special{pa -295 -594}\special{fp}\special{pa -295 -614}\special{pa -295 -633}\special{fp}%
\special{pa -295 -653}\special{pa -295 -672}\special{fp}\special{pa -295 -692}\special{pa -295 -711}\special{fp}%
\special{pa -295 -731}\special{pa -295 -750}\special{fp}\special{pa -295 -770}\special{pa -295 -789}\special{fp}%
\special{pa -295 -809}\special{pa -295 -828}\special{fp}\special{pa -295 -848}\special{pa -295 -867}\special{fp}%
\special{pa -295 -887}\special{pa -295 -906}\special{fp}\special{pa -295 -926}\special{pa -295 -945}\special{fp}%
\special{pa -295 -965}\special{pa -295 -984}\special{fp}%
%
\special{pa -98 984}\special{pa -98 965}\special{fp}\special{pa -98 945}\special{pa -98 926}\special{fp}%
\special{pa -98 906}\special{pa -98 887}\special{fp}\special{pa -98 867}\special{pa -98 848}\special{fp}%
\special{pa -98 828}\special{pa -98 809}\special{fp}\special{pa -98 789}\special{pa -98 770}\special{fp}%
\special{pa -98 750}\special{pa -98 731}\special{fp}\special{pa -98 711}\special{pa -98 692}\special{fp}%
\special{pa -98 672}\special{pa -98 653}\special{fp}\special{pa -98 633}\special{pa -98 614}\special{fp}%
\special{pa -98 594}\special{pa -98 575}\special{fp}\special{pa -98 555}\special{pa -98 536}\special{fp}%
\special{pa -98 516}\special{pa -98 497}\special{fp}\special{pa -98 478}\special{pa -98 458}\special{fp}%
\special{pa -98 439}\special{pa -98 419}\special{fp}\special{pa -98 400}\special{pa -98 380}\special{fp}%
\special{pa -98 361}\special{pa -98 341}\special{fp}\special{pa -98 322}\special{pa -98 302}\special{fp}%
\special{pa -98 283}\special{pa -98 263}\special{fp}\special{pa -98 244}\special{pa -98 224}\special{fp}%
\special{pa -98 205}\special{pa -98 185}\special{fp}\special{pa -98 166}\special{pa -98 146}\special{fp}%
\special{pa -98 127}\special{pa -98 107}\special{fp}\special{pa -98 88}\special{pa -98 68}\special{fp}%
\special{pa -98 49}\special{pa -98 29}\special{fp}\special{pa -98 10}\special{pa -98 -10}\special{fp}%
\special{pa -98 -29}\special{pa -98 -49}\special{fp}\special{pa -98 -68}\special{pa -98 -88}\special{fp}%
\special{pa -98 -107}\special{pa -98 -127}\special{fp}\special{pa -98 -146}\special{pa -98 -166}\special{fp}%
\special{pa -98 -185}\special{pa -98 -205}\special{fp}\special{pa -98 -224}\special{pa -98 -244}\special{fp}%
\special{pa -98 -263}\special{pa -98 -283}\special{fp}\special{pa -98 -302}\special{pa -98 -322}\special{fp}%
\special{pa -98 -341}\special{pa -98 -361}\special{fp}\special{pa -98 -380}\special{pa -98 -400}\special{fp}%
\special{pa -98 -419}\special{pa -98 -439}\special{fp}\special{pa -98 -458}\special{pa -98 -478}\special{fp}%
\special{pa -98 -497}\special{pa -98 -516}\special{fp}\special{pa -98 -536}\special{pa -98 -555}\special{fp}%
\special{pa -98 -575}\special{pa -98 -594}\special{fp}\special{pa -98 -614}\special{pa -98 -633}\special{fp}%
\special{pa -98 -653}\special{pa -98 -672}\special{fp}\special{pa -98 -692}\special{pa -98 -711}\special{fp}%
\special{pa -98 -731}\special{pa -98 -750}\special{fp}\special{pa -98 -770}\special{pa -98 -789}\special{fp}%
\special{pa -98 -809}\special{pa -98 -828}\special{fp}\special{pa -98 -848}\special{pa -98 -867}\special{fp}%
\special{pa -98 -887}\special{pa -98 -906}\special{fp}\special{pa -98 -926}\special{pa -98 -945}\special{fp}%
\special{pa -98 -965}\special{pa -98 -984}\special{fp}%
%
\special{pa 98 984}\special{pa 98 965}\special{fp}\special{pa 98 945}\special{pa 98 926}\special{fp}%
\special{pa 98 906}\special{pa 98 887}\special{fp}\special{pa 98 867}\special{pa 98 848}\special{fp}%
\special{pa 98 828}\special{pa 98 809}\special{fp}\special{pa 98 789}\special{pa 98 770}\special{fp}%
\special{pa 98 750}\special{pa 98 731}\special{fp}\special{pa 98 711}\special{pa 98 692}\special{fp}%
\special{pa 98 672}\special{pa 98 653}\special{fp}\special{pa 98 633}\special{pa 98 614}\special{fp}%
\special{pa 98 594}\special{pa 98 575}\special{fp}\special{pa 98 555}\special{pa 98 536}\special{fp}%
\special{pa 98 516}\special{pa 98 497}\special{fp}\special{pa 98 478}\special{pa 98 458}\special{fp}%
\special{pa 98 439}\special{pa 98 419}\special{fp}\special{pa 98 400}\special{pa 98 380}\special{fp}%
\special{pa 98 361}\special{pa 98 341}\special{fp}\special{pa 98 322}\special{pa 98 302}\special{fp}%
\special{pa 98 283}\special{pa 98 263}\special{fp}\special{pa 98 244}\special{pa 98 224}\special{fp}%
\special{pa 98 205}\special{pa 98 185}\special{fp}\special{pa 98 166}\special{pa 98 146}\special{fp}%
\special{pa 98 127}\special{pa 98 107}\special{fp}\special{pa 98 88}\special{pa 98 68}\special{fp}%
\special{pa 98 49}\special{pa 98 29}\special{fp}\special{pa 98 10}\special{pa 98 -10}\special{fp}%
\special{pa 98 -29}\special{pa 98 -49}\special{fp}\special{pa 98 -68}\special{pa 98 -88}\special{fp}%
\special{pa 98 -107}\special{pa 98 -127}\special{fp}\special{pa 98 -146}\special{pa 98 -166}\special{fp}%
\special{pa 98 -185}\special{pa 98 -205}\special{fp}\special{pa 98 -224}\special{pa 98 -244}\special{fp}%
\special{pa 98 -263}\special{pa 98 -283}\special{fp}\special{pa 98 -302}\special{pa 98 -322}\special{fp}%
\special{pa 98 -341}\special{pa 98 -361}\special{fp}\special{pa 98 -380}\special{pa 98 -400}\special{fp}%
\special{pa 98 -419}\special{pa 98 -439}\special{fp}\special{pa 98 -458}\special{pa 98 -478}\special{fp}%
\special{pa 98 -497}\special{pa 98 -516}\special{fp}\special{pa 98 -536}\special{pa 98 -555}\special{fp}%
\special{pa 98 -575}\special{pa 98 -594}\special{fp}\special{pa 98 -614}\special{pa 98 -633}\special{fp}%
\special{pa 98 -653}\special{pa 98 -672}\special{fp}\special{pa 98 -692}\special{pa 98 -711}\special{fp}%
\special{pa 98 -731}\special{pa 98 -750}\special{fp}\special{pa 98 -770}\special{pa 98 -789}\special{fp}%
\special{pa 98 -809}\special{pa 98 -828}\special{fp}\special{pa 98 -848}\special{pa 98 -867}\special{fp}%
\special{pa 98 -887}\special{pa 98 -906}\special{fp}\special{pa 98 -926}\special{pa 98 -945}\special{fp}%
\special{pa 98 -965}\special{pa 98 -984}\special{fp}%
%
\special{pa 295 984}\special{pa 295 965}\special{fp}\special{pa 295 945}\special{pa 295 926}\special{fp}%
\special{pa 295 906}\special{pa 295 887}\special{fp}\special{pa 295 867}\special{pa 295 848}\special{fp}%
\special{pa 295 828}\special{pa 295 809}\special{fp}\special{pa 295 789}\special{pa 295 770}\special{fp}%
\special{pa 295 750}\special{pa 295 731}\special{fp}\special{pa 295 711}\special{pa 295 692}\special{fp}%
\special{pa 295 672}\special{pa 295 653}\special{fp}\special{pa 295 633}\special{pa 295 614}\special{fp}%
\special{pa 295 594}\special{pa 295 575}\special{fp}\special{pa 295 555}\special{pa 295 536}\special{fp}%
\special{pa 295 516}\special{pa 295 497}\special{fp}\special{pa 295 478}\special{pa 295 458}\special{fp}%
\special{pa 295 439}\special{pa 295 419}\special{fp}\special{pa 295 400}\special{pa 295 380}\special{fp}%
\special{pa 295 361}\special{pa 295 341}\special{fp}\special{pa 295 322}\special{pa 295 302}\special{fp}%
\special{pa 295 283}\special{pa 295 263}\special{fp}\special{pa 295 244}\special{pa 295 224}\special{fp}%
\special{pa 295 205}\special{pa 295 185}\special{fp}\special{pa 295 166}\special{pa 295 146}\special{fp}%
\special{pa 295 127}\special{pa 295 107}\special{fp}\special{pa 295 88}\special{pa 295 68}\special{fp}%
\special{pa 295 49}\special{pa 295 29}\special{fp}\special{pa 295 10}\special{pa 295 -10}\special{fp}%
\special{pa 295 -29}\special{pa 295 -49}\special{fp}\special{pa 295 -68}\special{pa 295 -88}\special{fp}%
\special{pa 295 -107}\special{pa 295 -127}\special{fp}\special{pa 295 -146}\special{pa 295 -166}\special{fp}%
\special{pa 295 -185}\special{pa 295 -205}\special{fp}\special{pa 295 -224}\special{pa 295 -244}\special{fp}%
\special{pa 295 -263}\special{pa 295 -283}\special{fp}\special{pa 295 -302}\special{pa 295 -322}\special{fp}%
\special{pa 295 -341}\special{pa 295 -361}\special{fp}\special{pa 295 -380}\special{pa 295 -400}\special{fp}%
\special{pa 295 -419}\special{pa 295 -439}\special{fp}\special{pa 295 -458}\special{pa 295 -478}\special{fp}%
\special{pa 295 -497}\special{pa 295 -516}\special{fp}\special{pa 295 -536}\special{pa 295 -555}\special{fp}%
\special{pa 295 -575}\special{pa 295 -594}\special{fp}\special{pa 295 -614}\special{pa 295 -633}\special{fp}%
\special{pa 295 -653}\special{pa 295 -672}\special{fp}\special{pa 295 -692}\special{pa 295 -711}\special{fp}%
\special{pa 295 -731}\special{pa 295 -750}\special{fp}\special{pa 295 -770}\special{pa 295 -789}\special{fp}%
\special{pa 295 -809}\special{pa 295 -828}\special{fp}\special{pa 295 -848}\special{pa 295 -867}\special{fp}%
\special{pa 295 -887}\special{pa 295 -906}\special{fp}\special{pa 295 -926}\special{pa 295 -945}\special{fp}%
\special{pa 295 -965}\special{pa 295 -984}\special{fp}%
%
\special{pa 492 984}\special{pa 492 965}\special{fp}\special{pa 492 945}\special{pa 492 926}\special{fp}%
\special{pa 492 906}\special{pa 492 887}\special{fp}\special{pa 492 867}\special{pa 492 848}\special{fp}%
\special{pa 492 828}\special{pa 492 809}\special{fp}\special{pa 492 789}\special{pa 492 770}\special{fp}%
\special{pa 492 750}\special{pa 492 731}\special{fp}\special{pa 492 711}\special{pa 492 692}\special{fp}%
\special{pa 492 672}\special{pa 492 653}\special{fp}\special{pa 492 633}\special{pa 492 614}\special{fp}%
\special{pa 492 594}\special{pa 492 575}\special{fp}\special{pa 492 555}\special{pa 492 536}\special{fp}%
\special{pa 492 516}\special{pa 492 497}\special{fp}\special{pa 492 478}\special{pa 492 458}\special{fp}%
\special{pa 492 439}\special{pa 492 419}\special{fp}\special{pa 492 400}\special{pa 492 380}\special{fp}%
\special{pa 492 361}\special{pa 492 341}\special{fp}\special{pa 492 322}\special{pa 492 302}\special{fp}%
\special{pa 492 283}\special{pa 492 263}\special{fp}\special{pa 492 244}\special{pa 492 224}\special{fp}%
\special{pa 492 205}\special{pa 492 185}\special{fp}\special{pa 492 166}\special{pa 492 146}\special{fp}%
\special{pa 492 127}\special{pa 492 107}\special{fp}\special{pa 492 88}\special{pa 492 68}\special{fp}%
\special{pa 492 49}\special{pa 492 29}\special{fp}\special{pa 492 10}\special{pa 492 -10}\special{fp}%
\special{pa 492 -29}\special{pa 492 -49}\special{fp}\special{pa 492 -68}\special{pa 492 -88}\special{fp}%
\special{pa 492 -107}\special{pa 492 -127}\special{fp}\special{pa 492 -146}\special{pa 492 -166}\special{fp}%
\special{pa 492 -185}\special{pa 492 -205}\special{fp}\special{pa 492 -224}\special{pa 492 -244}\special{fp}%
\special{pa 492 -263}\special{pa 492 -283}\special{fp}\special{pa 492 -302}\special{pa 492 -322}\special{fp}%
\special{pa 492 -341}\special{pa 492 -361}\special{fp}\special{pa 492 -380}\special{pa 492 -400}\special{fp}%
\special{pa 492 -419}\special{pa 492 -439}\special{fp}\special{pa 492 -458}\special{pa 492 -478}\special{fp}%
\special{pa 492 -497}\special{pa 492 -516}\special{fp}\special{pa 492 -536}\special{pa 492 -555}\special{fp}%
\special{pa 492 -575}\special{pa 492 -594}\special{fp}\special{pa 492 -614}\special{pa 492 -633}\special{fp}%
\special{pa 492 -653}\special{pa 492 -672}\special{fp}\special{pa 492 -692}\special{pa 492 -711}\special{fp}%
\special{pa 492 -731}\special{pa 492 -750}\special{fp}\special{pa 492 -770}\special{pa 492 -789}\special{fp}%
\special{pa 492 -809}\special{pa 492 -828}\special{fp}\special{pa 492 -848}\special{pa 492 -867}\special{fp}%
\special{pa 492 -887}\special{pa 492 -906}\special{fp}\special{pa 492 -926}\special{pa 492 -945}\special{fp}%
\special{pa 492 -965}\special{pa 492 -984}\special{fp}%
%
\special{pa 689 984}\special{pa 689 965}\special{fp}\special{pa 689 945}\special{pa 689 926}\special{fp}%
\special{pa 689 906}\special{pa 689 887}\special{fp}\special{pa 689 867}\special{pa 689 848}\special{fp}%
\special{pa 689 828}\special{pa 689 809}\special{fp}\special{pa 689 789}\special{pa 689 770}\special{fp}%
\special{pa 689 750}\special{pa 689 731}\special{fp}\special{pa 689 711}\special{pa 689 692}\special{fp}%
\special{pa 689 672}\special{pa 689 653}\special{fp}\special{pa 689 633}\special{pa 689 614}\special{fp}%
\special{pa 689 594}\special{pa 689 575}\special{fp}\special{pa 689 555}\special{pa 689 536}\special{fp}%
\special{pa 689 516}\special{pa 689 497}\special{fp}\special{pa 689 478}\special{pa 689 458}\special{fp}%
\special{pa 689 439}\special{pa 689 419}\special{fp}\special{pa 689 400}\special{pa 689 380}\special{fp}%
\special{pa 689 361}\special{pa 689 341}\special{fp}\special{pa 689 322}\special{pa 689 302}\special{fp}%
\special{pa 689 283}\special{pa 689 263}\special{fp}\special{pa 689 244}\special{pa 689 224}\special{fp}%
\special{pa 689 205}\special{pa 689 185}\special{fp}\special{pa 689 166}\special{pa 689 146}\special{fp}%
\special{pa 689 127}\special{pa 689 107}\special{fp}\special{pa 689 88}\special{pa 689 68}\special{fp}%
\special{pa 689 49}\special{pa 689 29}\special{fp}\special{pa 689 10}\special{pa 689 -10}\special{fp}%
\special{pa 689 -29}\special{pa 689 -49}\special{fp}\special{pa 689 -68}\special{pa 689 -88}\special{fp}%
\special{pa 689 -107}\special{pa 689 -127}\special{fp}\special{pa 689 -146}\special{pa 689 -166}\special{fp}%
\special{pa 689 -185}\special{pa 689 -205}\special{fp}\special{pa 689 -224}\special{pa 689 -244}\special{fp}%
\special{pa 689 -263}\special{pa 689 -283}\special{fp}\special{pa 689 -302}\special{pa 689 -322}\special{fp}%
\special{pa 689 -341}\special{pa 689 -361}\special{fp}\special{pa 689 -380}\special{pa 689 -400}\special{fp}%
\special{pa 689 -419}\special{pa 689 -439}\special{fp}\special{pa 689 -458}\special{pa 689 -478}\special{fp}%
\special{pa 689 -497}\special{pa 689 -516}\special{fp}\special{pa 689 -536}\special{pa 689 -555}\special{fp}%
\special{pa 689 -575}\special{pa 689 -594}\special{fp}\special{pa 689 -614}\special{pa 689 -633}\special{fp}%
\special{pa 689 -653}\special{pa 689 -672}\special{fp}\special{pa 689 -692}\special{pa 689 -711}\special{fp}%
\special{pa 689 -731}\special{pa 689 -750}\special{fp}\special{pa 689 -770}\special{pa 689 -789}\special{fp}%
\special{pa 689 -809}\special{pa 689 -828}\special{fp}\special{pa 689 -848}\special{pa 689 -867}\special{fp}%
\special{pa 689 -887}\special{pa 689 -906}\special{fp}\special{pa 689 -926}\special{pa 689 -945}\special{fp}%
\special{pa 689 -965}\special{pa 689 -984}\special{fp}%
%
\special{pa 886 984}\special{pa 886 965}\special{fp}\special{pa 886 945}\special{pa 886 926}\special{fp}%
\special{pa 886 906}\special{pa 886 887}\special{fp}\special{pa 886 867}\special{pa 886 848}\special{fp}%
\special{pa 886 828}\special{pa 886 809}\special{fp}\special{pa 886 789}\special{pa 886 770}\special{fp}%
\special{pa 886 750}\special{pa 886 731}\special{fp}\special{pa 886 711}\special{pa 886 692}\special{fp}%
\special{pa 886 672}\special{pa 886 653}\special{fp}\special{pa 886 633}\special{pa 886 614}\special{fp}%
\special{pa 886 594}\special{pa 886 575}\special{fp}\special{pa 886 555}\special{pa 886 536}\special{fp}%
\special{pa 886 516}\special{pa 886 497}\special{fp}\special{pa 886 478}\special{pa 886 458}\special{fp}%
\special{pa 886 439}\special{pa 886 419}\special{fp}\special{pa 886 400}\special{pa 886 380}\special{fp}%
\special{pa 886 361}\special{pa 886 341}\special{fp}\special{pa 886 322}\special{pa 886 302}\special{fp}%
\special{pa 886 283}\special{pa 886 263}\special{fp}\special{pa 886 244}\special{pa 886 224}\special{fp}%
\special{pa 886 205}\special{pa 886 185}\special{fp}\special{pa 886 166}\special{pa 886 146}\special{fp}%
\special{pa 886 127}\special{pa 886 107}\special{fp}\special{pa 886 88}\special{pa 886 68}\special{fp}%
\special{pa 886 49}\special{pa 886 29}\special{fp}\special{pa 886 10}\special{pa 886 -10}\special{fp}%
\special{pa 886 -29}\special{pa 886 -49}\special{fp}\special{pa 886 -68}\special{pa 886 -88}\special{fp}%
\special{pa 886 -107}\special{pa 886 -127}\special{fp}\special{pa 886 -146}\special{pa 886 -166}\special{fp}%
\special{pa 886 -185}\special{pa 886 -205}\special{fp}\special{pa 886 -224}\special{pa 886 -244}\special{fp}%
\special{pa 886 -263}\special{pa 886 -283}\special{fp}\special{pa 886 -302}\special{pa 886 -322}\special{fp}%
\special{pa 886 -341}\special{pa 886 -361}\special{fp}\special{pa 886 -380}\special{pa 886 -400}\special{fp}%
\special{pa 886 -419}\special{pa 886 -439}\special{fp}\special{pa 886 -458}\special{pa 886 -478}\special{fp}%
\special{pa 886 -497}\special{pa 886 -516}\special{fp}\special{pa 886 -536}\special{pa 886 -555}\special{fp}%
\special{pa 886 -575}\special{pa 886 -594}\special{fp}\special{pa 886 -614}\special{pa 886 -633}\special{fp}%
\special{pa 886 -653}\special{pa 886 -672}\special{fp}\special{pa 886 -692}\special{pa 886 -711}\special{fp}%
\special{pa 886 -731}\special{pa 886 -750}\special{fp}\special{pa 886 -770}\special{pa 886 -789}\special{fp}%
\special{pa 886 -809}\special{pa 886 -828}\special{fp}\special{pa 886 -848}\special{pa 886 -867}\special{fp}%
\special{pa 886 -887}\special{pa 886 -906}\special{fp}\special{pa 886 -926}\special{pa 886 -945}\special{fp}%
\special{pa 886 -965}\special{pa 886 -984}\special{fp}%
%
\special{pa  -984   984}\special{pa  -984  -984}%
\special{fp}%
\special{pa  -787   984}\special{pa  -787  -984}%
\special{fp}%
\special{pa  -591   984}\special{pa  -591  -984}%
\special{fp}%
\special{pa  -394   984}\special{pa  -394  -984}%
\special{fp}%
\special{pa  -197   984}\special{pa  -197  -984}%
\special{fp}%
\special{pa     0   984}\special{pa     0  -984}%
\special{fp}%
\special{pa   197   984}\special{pa   197  -984}%
\special{fp}%
\special{pa   394   984}\special{pa   394  -984}%
\special{fp}%
\special{pa   591   984}\special{pa   591  -984}%
\special{fp}%
\special{pa   787   984}\special{pa   787  -984}%
\special{fp}%
\special{pa   984   984}\special{pa   984  -984}%
\special{fp}%
\special{pn 8}%
\scriptsize%
\special{pa  -197   -20}\special{pa  -197    20}%
\special{fp}%
\settowidth{\Width}{$-1$}\setlength{\Width}{-0.5\Width}%
\settoheight{\Height}{$-1$}\settodepth{\Depth}{$-1$}\setlength{\Height}{-\Height}%
\put(-1.0000000,-0.2000000){\hspace*{\Width}\raisebox{\Height}{$-1$}}%
%
%
\special{pa   197   -20}\special{pa   197    20}%
\special{fp}%
\settowidth{\Width}{$1$}\setlength{\Width}{-0.5\Width}%
\settoheight{\Height}{$1$}\settodepth{\Depth}{$1$}\setlength{\Height}{-\Height}%
\put(1.0000000,-0.2000000){\hspace*{\Width}\raisebox{\Height}{$1$}}%
%
%
\special{pa    20   197}\special{pa   -20   197}%
\special{fp}%
\settowidth{\Width}{$-1$}\setlength{\Width}{-1\Width}%
\settoheight{\Height}{$-1$}\settodepth{\Depth}{$-1$}\setlength{\Height}{-0.5\Height}\setlength{\Depth}{0.5\Depth}\addtolength{\Height}{\Depth}%
\put(-0.2000000,-1.0000000){\hspace*{\Width}\raisebox{\Height}{$-1$}}%
%
%
\special{pa    20  -197}\special{pa   -20  -197}%
\special{fp}%
\settowidth{\Width}{$1$}\setlength{\Width}{-1\Width}%
\settoheight{\Height}{$1$}\settodepth{\Depth}{$1$}\setlength{\Height}{-0.5\Height}\setlength{\Depth}{0.5\Depth}\addtolength{\Height}{\Depth}%
\put(-0.2000000,1.0000000){\hspace*{\Width}\raisebox{\Height}{$1$}}%
%
%
\special{pn 8}%
\special{pa  -984    -0}\special{pa   965    -0}%
\special{fp}%
\special{pn 8}%
\special{pa 909 24}\special{pa 984 0}\special{pa 909 -24}\special{pa 909 0}\special{pa 909 24}%
\special{sh 1}\special{ip}%
\special{pn 1}%
\special{pa   909    24}\special{pa   984    -0}\special{pa   909   -24}\special{pa   909    -0}%
\special{pa   909    24}\special{pa   984    -0}%
\special{fp}%
\special{pn 8}%
\special{pn 8}%
\special{pa     0   984}\special{pa     0  -965}%
\special{fp}%
\special{pn 8}%
\special{pa 24 -909}\special{pa 0 -984}\special{pa -24 -909}\special{pa 0 -909}\special{pa 24 -909}%
\special{sh 1}\special{ip}%
\special{pn 1}%
\special{pa    24  -909}\special{pa     0  -984}\special{pa   -24  -909}\special{pa     0  -909}%
\special{pa    24  -909}\special{pa     0  -984}%
\special{fp}%
\special{pn 8}%
\settowidth{\Width}{$x\mbox{実軸}$}\setlength{\Width}{0\Width}%
\settoheight{\Height}{$x\mbox{実軸}$}\settodepth{\Depth}{$x\mbox{実軸}$}\setlength{\Height}{-0.5\Height}\setlength{\Depth}{0.5\Depth}\addtolength{\Height}{\Depth}%
\put(5.1000000,0.0000000){\hspace*{\Width}\raisebox{\Height}{$x\mbox{実軸}$}}%
%
\settowidth{\Width}{$y\mbox{虚軸}$}\setlength{\Width}{-0.5\Width}%
\settoheight{\Height}{$y\mbox{虚軸}$}\settodepth{\Depth}{$y\mbox{虚軸}$}\setlength{\Height}{\Depth}%
\put(0.0000000,5.1000000){\hspace*{\Width}\raisebox{\Height}{$y\mbox{虚軸}$}}%
%
\settowidth{\Width}{ }\setlength{\Width}{-1\Width}%
\settoheight{\Height}{ }\settodepth{\Depth}{ }\setlength{\Height}{-\Height}%
\put(-0.1000000,-0.1000000){\hspace*{\Width}\raisebox{\Height}{ }}%
%
\end{picture}}%}
\end{layer}

{\color{red}

\begin{layer}{120}{0}
\putnotec{110}{25}{\small$\bullet$}
\end{layer}

}
\begin{itemize}
\item
$z=a+b\,i$を平面上の点$(a,\ b)$で表す\seteda{50}\\
\eda{$2+3i \leftrightarrow\ \mbox{点}(2,3)$}\\
\end{itemize}

\sameslide

\vspace*{18mm}

\slidepage

\begin{layer}{120}{0}
\putnotese{75}{15}{%%% /Users/takatoosetsuo/Dropbox/2018polytec/lecture/0611/presen/fig/plane1.tex 
%%% Generator=presen0611.cdy 
{\unitlength=5mm%
\begin{picture}%
(10,10)(-5,-5)%
\special{pn 8}%
%
\Large\bf\boldmath%
\small%
\special{pn 4}%
\special{pa -984 886}\special{pa -965 886}\special{fp}\special{pa -945 886}\special{pa -926 886}\special{fp}%
\special{pa -906 886}\special{pa -887 886}\special{fp}\special{pa -867 886}\special{pa -848 886}\special{fp}%
\special{pa -828 886}\special{pa -809 886}\special{fp}\special{pa -789 886}\special{pa -770 886}\special{fp}%
\special{pa -750 886}\special{pa -731 886}\special{fp}\special{pa -711 886}\special{pa -692 886}\special{fp}%
\special{pa -672 886}\special{pa -653 886}\special{fp}\special{pa -633 886}\special{pa -614 886}\special{fp}%
\special{pa -594 886}\special{pa -575 886}\special{fp}\special{pa -555 886}\special{pa -536 886}\special{fp}%
\special{pa -516 886}\special{pa -497 886}\special{fp}\special{pa -478 886}\special{pa -458 886}\special{fp}%
\special{pa -439 886}\special{pa -419 886}\special{fp}\special{pa -400 886}\special{pa -380 886}\special{fp}%
\special{pa -361 886}\special{pa -341 886}\special{fp}\special{pa -322 886}\special{pa -302 886}\special{fp}%
\special{pa -283 886}\special{pa -263 886}\special{fp}\special{pa -244 886}\special{pa -224 886}\special{fp}%
\special{pa -205 886}\special{pa -185 886}\special{fp}\special{pa -166 886}\special{pa -146 886}\special{fp}%
\special{pa -127 886}\special{pa -107 886}\special{fp}\special{pa -88 886}\special{pa -68 886}\special{fp}%
\special{pa -49 886}\special{pa -29 886}\special{fp}\special{pa -10 886}\special{pa 10 886}\special{fp}%
\special{pa 29 886}\special{pa 49 886}\special{fp}\special{pa 68 886}\special{pa 88 886}\special{fp}%
\special{pa 107 886}\special{pa 127 886}\special{fp}\special{pa 146 886}\special{pa 166 886}\special{fp}%
\special{pa 185 886}\special{pa 205 886}\special{fp}\special{pa 224 886}\special{pa 244 886}\special{fp}%
\special{pa 263 886}\special{pa 283 886}\special{fp}\special{pa 302 886}\special{pa 322 886}\special{fp}%
\special{pa 341 886}\special{pa 361 886}\special{fp}\special{pa 380 886}\special{pa 400 886}\special{fp}%
\special{pa 419 886}\special{pa 439 886}\special{fp}\special{pa 458 886}\special{pa 478 886}\special{fp}%
\special{pa 497 886}\special{pa 516 886}\special{fp}\special{pa 536 886}\special{pa 555 886}\special{fp}%
\special{pa 575 886}\special{pa 594 886}\special{fp}\special{pa 614 886}\special{pa 633 886}\special{fp}%
\special{pa 653 886}\special{pa 672 886}\special{fp}\special{pa 692 886}\special{pa 711 886}\special{fp}%
\special{pa 731 886}\special{pa 750 886}\special{fp}\special{pa 770 886}\special{pa 789 886}\special{fp}%
\special{pa 809 886}\special{pa 828 886}\special{fp}\special{pa 848 886}\special{pa 867 886}\special{fp}%
\special{pa 887 886}\special{pa 906 886}\special{fp}\special{pa 926 886}\special{pa 945 886}\special{fp}%
\special{pa 965 886}\special{pa 984 886}\special{fp}%
%
\special{pa -984 689}\special{pa -965 689}\special{fp}\special{pa -945 689}\special{pa -926 689}\special{fp}%
\special{pa -906 689}\special{pa -887 689}\special{fp}\special{pa -867 689}\special{pa -848 689}\special{fp}%
\special{pa -828 689}\special{pa -809 689}\special{fp}\special{pa -789 689}\special{pa -770 689}\special{fp}%
\special{pa -750 689}\special{pa -731 689}\special{fp}\special{pa -711 689}\special{pa -692 689}\special{fp}%
\special{pa -672 689}\special{pa -653 689}\special{fp}\special{pa -633 689}\special{pa -614 689}\special{fp}%
\special{pa -594 689}\special{pa -575 689}\special{fp}\special{pa -555 689}\special{pa -536 689}\special{fp}%
\special{pa -516 689}\special{pa -497 689}\special{fp}\special{pa -478 689}\special{pa -458 689}\special{fp}%
\special{pa -439 689}\special{pa -419 689}\special{fp}\special{pa -400 689}\special{pa -380 689}\special{fp}%
\special{pa -361 689}\special{pa -341 689}\special{fp}\special{pa -322 689}\special{pa -302 689}\special{fp}%
\special{pa -283 689}\special{pa -263 689}\special{fp}\special{pa -244 689}\special{pa -224 689}\special{fp}%
\special{pa -205 689}\special{pa -185 689}\special{fp}\special{pa -166 689}\special{pa -146 689}\special{fp}%
\special{pa -127 689}\special{pa -107 689}\special{fp}\special{pa -88 689}\special{pa -68 689}\special{fp}%
\special{pa -49 689}\special{pa -29 689}\special{fp}\special{pa -10 689}\special{pa 10 689}\special{fp}%
\special{pa 29 689}\special{pa 49 689}\special{fp}\special{pa 68 689}\special{pa 88 689}\special{fp}%
\special{pa 107 689}\special{pa 127 689}\special{fp}\special{pa 146 689}\special{pa 166 689}\special{fp}%
\special{pa 185 689}\special{pa 205 689}\special{fp}\special{pa 224 689}\special{pa 244 689}\special{fp}%
\special{pa 263 689}\special{pa 283 689}\special{fp}\special{pa 302 689}\special{pa 322 689}\special{fp}%
\special{pa 341 689}\special{pa 361 689}\special{fp}\special{pa 380 689}\special{pa 400 689}\special{fp}%
\special{pa 419 689}\special{pa 439 689}\special{fp}\special{pa 458 689}\special{pa 478 689}\special{fp}%
\special{pa 497 689}\special{pa 516 689}\special{fp}\special{pa 536 689}\special{pa 555 689}\special{fp}%
\special{pa 575 689}\special{pa 594 689}\special{fp}\special{pa 614 689}\special{pa 633 689}\special{fp}%
\special{pa 653 689}\special{pa 672 689}\special{fp}\special{pa 692 689}\special{pa 711 689}\special{fp}%
\special{pa 731 689}\special{pa 750 689}\special{fp}\special{pa 770 689}\special{pa 789 689}\special{fp}%
\special{pa 809 689}\special{pa 828 689}\special{fp}\special{pa 848 689}\special{pa 867 689}\special{fp}%
\special{pa 887 689}\special{pa 906 689}\special{fp}\special{pa 926 689}\special{pa 945 689}\special{fp}%
\special{pa 965 689}\special{pa 984 689}\special{fp}%
%
\special{pa -984 492}\special{pa -965 492}\special{fp}\special{pa -945 492}\special{pa -926 492}\special{fp}%
\special{pa -906 492}\special{pa -887 492}\special{fp}\special{pa -867 492}\special{pa -848 492}\special{fp}%
\special{pa -828 492}\special{pa -809 492}\special{fp}\special{pa -789 492}\special{pa -770 492}\special{fp}%
\special{pa -750 492}\special{pa -731 492}\special{fp}\special{pa -711 492}\special{pa -692 492}\special{fp}%
\special{pa -672 492}\special{pa -653 492}\special{fp}\special{pa -633 492}\special{pa -614 492}\special{fp}%
\special{pa -594 492}\special{pa -575 492}\special{fp}\special{pa -555 492}\special{pa -536 492}\special{fp}%
\special{pa -516 492}\special{pa -497 492}\special{fp}\special{pa -478 492}\special{pa -458 492}\special{fp}%
\special{pa -439 492}\special{pa -419 492}\special{fp}\special{pa -400 492}\special{pa -380 492}\special{fp}%
\special{pa -361 492}\special{pa -341 492}\special{fp}\special{pa -322 492}\special{pa -302 492}\special{fp}%
\special{pa -283 492}\special{pa -263 492}\special{fp}\special{pa -244 492}\special{pa -224 492}\special{fp}%
\special{pa -205 492}\special{pa -185 492}\special{fp}\special{pa -166 492}\special{pa -146 492}\special{fp}%
\special{pa -127 492}\special{pa -107 492}\special{fp}\special{pa -88 492}\special{pa -68 492}\special{fp}%
\special{pa -49 492}\special{pa -29 492}\special{fp}\special{pa -10 492}\special{pa 10 492}\special{fp}%
\special{pa 29 492}\special{pa 49 492}\special{fp}\special{pa 68 492}\special{pa 88 492}\special{fp}%
\special{pa 107 492}\special{pa 127 492}\special{fp}\special{pa 146 492}\special{pa 166 492}\special{fp}%
\special{pa 185 492}\special{pa 205 492}\special{fp}\special{pa 224 492}\special{pa 244 492}\special{fp}%
\special{pa 263 492}\special{pa 283 492}\special{fp}\special{pa 302 492}\special{pa 322 492}\special{fp}%
\special{pa 341 492}\special{pa 361 492}\special{fp}\special{pa 380 492}\special{pa 400 492}\special{fp}%
\special{pa 419 492}\special{pa 439 492}\special{fp}\special{pa 458 492}\special{pa 478 492}\special{fp}%
\special{pa 497 492}\special{pa 516 492}\special{fp}\special{pa 536 492}\special{pa 555 492}\special{fp}%
\special{pa 575 492}\special{pa 594 492}\special{fp}\special{pa 614 492}\special{pa 633 492}\special{fp}%
\special{pa 653 492}\special{pa 672 492}\special{fp}\special{pa 692 492}\special{pa 711 492}\special{fp}%
\special{pa 731 492}\special{pa 750 492}\special{fp}\special{pa 770 492}\special{pa 789 492}\special{fp}%
\special{pa 809 492}\special{pa 828 492}\special{fp}\special{pa 848 492}\special{pa 867 492}\special{fp}%
\special{pa 887 492}\special{pa 906 492}\special{fp}\special{pa 926 492}\special{pa 945 492}\special{fp}%
\special{pa 965 492}\special{pa 984 492}\special{fp}%
%
\special{pa -984 295}\special{pa -965 295}\special{fp}\special{pa -945 295}\special{pa -926 295}\special{fp}%
\special{pa -906 295}\special{pa -887 295}\special{fp}\special{pa -867 295}\special{pa -848 295}\special{fp}%
\special{pa -828 295}\special{pa -809 295}\special{fp}\special{pa -789 295}\special{pa -770 295}\special{fp}%
\special{pa -750 295}\special{pa -731 295}\special{fp}\special{pa -711 295}\special{pa -692 295}\special{fp}%
\special{pa -672 295}\special{pa -653 295}\special{fp}\special{pa -633 295}\special{pa -614 295}\special{fp}%
\special{pa -594 295}\special{pa -575 295}\special{fp}\special{pa -555 295}\special{pa -536 295}\special{fp}%
\special{pa -516 295}\special{pa -497 295}\special{fp}\special{pa -478 295}\special{pa -458 295}\special{fp}%
\special{pa -439 295}\special{pa -419 295}\special{fp}\special{pa -400 295}\special{pa -380 295}\special{fp}%
\special{pa -361 295}\special{pa -341 295}\special{fp}\special{pa -322 295}\special{pa -302 295}\special{fp}%
\special{pa -283 295}\special{pa -263 295}\special{fp}\special{pa -244 295}\special{pa -224 295}\special{fp}%
\special{pa -205 295}\special{pa -185 295}\special{fp}\special{pa -166 295}\special{pa -146 295}\special{fp}%
\special{pa -127 295}\special{pa -107 295}\special{fp}\special{pa -88 295}\special{pa -68 295}\special{fp}%
\special{pa -49 295}\special{pa -29 295}\special{fp}\special{pa -10 295}\special{pa 10 295}\special{fp}%
\special{pa 29 295}\special{pa 49 295}\special{fp}\special{pa 68 295}\special{pa 88 295}\special{fp}%
\special{pa 107 295}\special{pa 127 295}\special{fp}\special{pa 146 295}\special{pa 166 295}\special{fp}%
\special{pa 185 295}\special{pa 205 295}\special{fp}\special{pa 224 295}\special{pa 244 295}\special{fp}%
\special{pa 263 295}\special{pa 283 295}\special{fp}\special{pa 302 295}\special{pa 322 295}\special{fp}%
\special{pa 341 295}\special{pa 361 295}\special{fp}\special{pa 380 295}\special{pa 400 295}\special{fp}%
\special{pa 419 295}\special{pa 439 295}\special{fp}\special{pa 458 295}\special{pa 478 295}\special{fp}%
\special{pa 497 295}\special{pa 516 295}\special{fp}\special{pa 536 295}\special{pa 555 295}\special{fp}%
\special{pa 575 295}\special{pa 594 295}\special{fp}\special{pa 614 295}\special{pa 633 295}\special{fp}%
\special{pa 653 295}\special{pa 672 295}\special{fp}\special{pa 692 295}\special{pa 711 295}\special{fp}%
\special{pa 731 295}\special{pa 750 295}\special{fp}\special{pa 770 295}\special{pa 789 295}\special{fp}%
\special{pa 809 295}\special{pa 828 295}\special{fp}\special{pa 848 295}\special{pa 867 295}\special{fp}%
\special{pa 887 295}\special{pa 906 295}\special{fp}\special{pa 926 295}\special{pa 945 295}\special{fp}%
\special{pa 965 295}\special{pa 984 295}\special{fp}%
%
\special{pa -984 98}\special{pa -965 98}\special{fp}\special{pa -945 98}\special{pa -926 98}\special{fp}%
\special{pa -906 98}\special{pa -887 98}\special{fp}\special{pa -867 98}\special{pa -848 98}\special{fp}%
\special{pa -828 98}\special{pa -809 98}\special{fp}\special{pa -789 98}\special{pa -770 98}\special{fp}%
\special{pa -750 98}\special{pa -731 98}\special{fp}\special{pa -711 98}\special{pa -692 98}\special{fp}%
\special{pa -672 98}\special{pa -653 98}\special{fp}\special{pa -633 98}\special{pa -614 98}\special{fp}%
\special{pa -594 98}\special{pa -575 98}\special{fp}\special{pa -555 98}\special{pa -536 98}\special{fp}%
\special{pa -516 98}\special{pa -497 98}\special{fp}\special{pa -478 98}\special{pa -458 98}\special{fp}%
\special{pa -439 98}\special{pa -419 98}\special{fp}\special{pa -400 98}\special{pa -380 98}\special{fp}%
\special{pa -361 98}\special{pa -341 98}\special{fp}\special{pa -322 98}\special{pa -302 98}\special{fp}%
\special{pa -283 98}\special{pa -263 98}\special{fp}\special{pa -244 98}\special{pa -224 98}\special{fp}%
\special{pa -205 98}\special{pa -185 98}\special{fp}\special{pa -166 98}\special{pa -146 98}\special{fp}%
\special{pa -127 98}\special{pa -107 98}\special{fp}\special{pa -88 98}\special{pa -68 98}\special{fp}%
\special{pa -49 98}\special{pa -29 98}\special{fp}\special{pa -10 98}\special{pa 10 98}\special{fp}%
\special{pa 29 98}\special{pa 49 98}\special{fp}\special{pa 68 98}\special{pa 88 98}\special{fp}%
\special{pa 107 98}\special{pa 127 98}\special{fp}\special{pa 146 98}\special{pa 166 98}\special{fp}%
\special{pa 185 98}\special{pa 205 98}\special{fp}\special{pa 224 98}\special{pa 244 98}\special{fp}%
\special{pa 263 98}\special{pa 283 98}\special{fp}\special{pa 302 98}\special{pa 322 98}\special{fp}%
\special{pa 341 98}\special{pa 361 98}\special{fp}\special{pa 380 98}\special{pa 400 98}\special{fp}%
\special{pa 419 98}\special{pa 439 98}\special{fp}\special{pa 458 98}\special{pa 478 98}\special{fp}%
\special{pa 497 98}\special{pa 516 98}\special{fp}\special{pa 536 98}\special{pa 555 98}\special{fp}%
\special{pa 575 98}\special{pa 594 98}\special{fp}\special{pa 614 98}\special{pa 633 98}\special{fp}%
\special{pa 653 98}\special{pa 672 98}\special{fp}\special{pa 692 98}\special{pa 711 98}\special{fp}%
\special{pa 731 98}\special{pa 750 98}\special{fp}\special{pa 770 98}\special{pa 789 98}\special{fp}%
\special{pa 809 98}\special{pa 828 98}\special{fp}\special{pa 848 98}\special{pa 867 98}\special{fp}%
\special{pa 887 98}\special{pa 906 98}\special{fp}\special{pa 926 98}\special{pa 945 98}\special{fp}%
\special{pa 965 98}\special{pa 984 98}\special{fp}%
%
\special{pa -984 -98}\special{pa -965 -98}\special{fp}\special{pa -945 -98}\special{pa -926 -98}\special{fp}%
\special{pa -906 -98}\special{pa -887 -98}\special{fp}\special{pa -867 -98}\special{pa -848 -98}\special{fp}%
\special{pa -828 -98}\special{pa -809 -98}\special{fp}\special{pa -789 -98}\special{pa -770 -98}\special{fp}%
\special{pa -750 -98}\special{pa -731 -98}\special{fp}\special{pa -711 -98}\special{pa -692 -98}\special{fp}%
\special{pa -672 -98}\special{pa -653 -98}\special{fp}\special{pa -633 -98}\special{pa -614 -98}\special{fp}%
\special{pa -594 -98}\special{pa -575 -98}\special{fp}\special{pa -555 -98}\special{pa -536 -98}\special{fp}%
\special{pa -516 -98}\special{pa -497 -98}\special{fp}\special{pa -478 -98}\special{pa -458 -98}\special{fp}%
\special{pa -439 -98}\special{pa -419 -98}\special{fp}\special{pa -400 -98}\special{pa -380 -98}\special{fp}%
\special{pa -361 -98}\special{pa -341 -98}\special{fp}\special{pa -322 -98}\special{pa -302 -98}\special{fp}%
\special{pa -283 -98}\special{pa -263 -98}\special{fp}\special{pa -244 -98}\special{pa -224 -98}\special{fp}%
\special{pa -205 -98}\special{pa -185 -98}\special{fp}\special{pa -166 -98}\special{pa -146 -98}\special{fp}%
\special{pa -127 -98}\special{pa -107 -98}\special{fp}\special{pa -88 -98}\special{pa -68 -98}\special{fp}%
\special{pa -49 -98}\special{pa -29 -98}\special{fp}\special{pa -10 -98}\special{pa 10 -98}\special{fp}%
\special{pa 29 -98}\special{pa 49 -98}\special{fp}\special{pa 68 -98}\special{pa 88 -98}\special{fp}%
\special{pa 107 -98}\special{pa 127 -98}\special{fp}\special{pa 146 -98}\special{pa 166 -98}\special{fp}%
\special{pa 185 -98}\special{pa 205 -98}\special{fp}\special{pa 224 -98}\special{pa 244 -98}\special{fp}%
\special{pa 263 -98}\special{pa 283 -98}\special{fp}\special{pa 302 -98}\special{pa 322 -98}\special{fp}%
\special{pa 341 -98}\special{pa 361 -98}\special{fp}\special{pa 380 -98}\special{pa 400 -98}\special{fp}%
\special{pa 419 -98}\special{pa 439 -98}\special{fp}\special{pa 458 -98}\special{pa 478 -98}\special{fp}%
\special{pa 497 -98}\special{pa 516 -98}\special{fp}\special{pa 536 -98}\special{pa 555 -98}\special{fp}%
\special{pa 575 -98}\special{pa 594 -98}\special{fp}\special{pa 614 -98}\special{pa 633 -98}\special{fp}%
\special{pa 653 -98}\special{pa 672 -98}\special{fp}\special{pa 692 -98}\special{pa 711 -98}\special{fp}%
\special{pa 731 -98}\special{pa 750 -98}\special{fp}\special{pa 770 -98}\special{pa 789 -98}\special{fp}%
\special{pa 809 -98}\special{pa 828 -98}\special{fp}\special{pa 848 -98}\special{pa 867 -98}\special{fp}%
\special{pa 887 -98}\special{pa 906 -98}\special{fp}\special{pa 926 -98}\special{pa 945 -98}\special{fp}%
\special{pa 965 -98}\special{pa 984 -98}\special{fp}%
%
\special{pa -984 -295}\special{pa -965 -295}\special{fp}\special{pa -945 -295}\special{pa -926 -295}\special{fp}%
\special{pa -906 -295}\special{pa -887 -295}\special{fp}\special{pa -867 -295}\special{pa -848 -295}\special{fp}%
\special{pa -828 -295}\special{pa -809 -295}\special{fp}\special{pa -789 -295}\special{pa -770 -295}\special{fp}%
\special{pa -750 -295}\special{pa -731 -295}\special{fp}\special{pa -711 -295}\special{pa -692 -295}\special{fp}%
\special{pa -672 -295}\special{pa -653 -295}\special{fp}\special{pa -633 -295}\special{pa -614 -295}\special{fp}%
\special{pa -594 -295}\special{pa -575 -295}\special{fp}\special{pa -555 -295}\special{pa -536 -295}\special{fp}%
\special{pa -516 -295}\special{pa -497 -295}\special{fp}\special{pa -478 -295}\special{pa -458 -295}\special{fp}%
\special{pa -439 -295}\special{pa -419 -295}\special{fp}\special{pa -400 -295}\special{pa -380 -295}\special{fp}%
\special{pa -361 -295}\special{pa -341 -295}\special{fp}\special{pa -322 -295}\special{pa -302 -295}\special{fp}%
\special{pa -283 -295}\special{pa -263 -295}\special{fp}\special{pa -244 -295}\special{pa -224 -295}\special{fp}%
\special{pa -205 -295}\special{pa -185 -295}\special{fp}\special{pa -166 -295}\special{pa -146 -295}\special{fp}%
\special{pa -127 -295}\special{pa -107 -295}\special{fp}\special{pa -88 -295}\special{pa -68 -295}\special{fp}%
\special{pa -49 -295}\special{pa -29 -295}\special{fp}\special{pa -10 -295}\special{pa 10 -295}\special{fp}%
\special{pa 29 -295}\special{pa 49 -295}\special{fp}\special{pa 68 -295}\special{pa 88 -295}\special{fp}%
\special{pa 107 -295}\special{pa 127 -295}\special{fp}\special{pa 146 -295}\special{pa 166 -295}\special{fp}%
\special{pa 185 -295}\special{pa 205 -295}\special{fp}\special{pa 224 -295}\special{pa 244 -295}\special{fp}%
\special{pa 263 -295}\special{pa 283 -295}\special{fp}\special{pa 302 -295}\special{pa 322 -295}\special{fp}%
\special{pa 341 -295}\special{pa 361 -295}\special{fp}\special{pa 380 -295}\special{pa 400 -295}\special{fp}%
\special{pa 419 -295}\special{pa 439 -295}\special{fp}\special{pa 458 -295}\special{pa 478 -295}\special{fp}%
\special{pa 497 -295}\special{pa 516 -295}\special{fp}\special{pa 536 -295}\special{pa 555 -295}\special{fp}%
\special{pa 575 -295}\special{pa 594 -295}\special{fp}\special{pa 614 -295}\special{pa 633 -295}\special{fp}%
\special{pa 653 -295}\special{pa 672 -295}\special{fp}\special{pa 692 -295}\special{pa 711 -295}\special{fp}%
\special{pa 731 -295}\special{pa 750 -295}\special{fp}\special{pa 770 -295}\special{pa 789 -295}\special{fp}%
\special{pa 809 -295}\special{pa 828 -295}\special{fp}\special{pa 848 -295}\special{pa 867 -295}\special{fp}%
\special{pa 887 -295}\special{pa 906 -295}\special{fp}\special{pa 926 -295}\special{pa 945 -295}\special{fp}%
\special{pa 965 -295}\special{pa 984 -295}\special{fp}%
%
\special{pa -984 -492}\special{pa -965 -492}\special{fp}\special{pa -945 -492}\special{pa -926 -492}\special{fp}%
\special{pa -906 -492}\special{pa -887 -492}\special{fp}\special{pa -867 -492}\special{pa -848 -492}\special{fp}%
\special{pa -828 -492}\special{pa -809 -492}\special{fp}\special{pa -789 -492}\special{pa -770 -492}\special{fp}%
\special{pa -750 -492}\special{pa -731 -492}\special{fp}\special{pa -711 -492}\special{pa -692 -492}\special{fp}%
\special{pa -672 -492}\special{pa -653 -492}\special{fp}\special{pa -633 -492}\special{pa -614 -492}\special{fp}%
\special{pa -594 -492}\special{pa -575 -492}\special{fp}\special{pa -555 -492}\special{pa -536 -492}\special{fp}%
\special{pa -516 -492}\special{pa -497 -492}\special{fp}\special{pa -478 -492}\special{pa -458 -492}\special{fp}%
\special{pa -439 -492}\special{pa -419 -492}\special{fp}\special{pa -400 -492}\special{pa -380 -492}\special{fp}%
\special{pa -361 -492}\special{pa -341 -492}\special{fp}\special{pa -322 -492}\special{pa -302 -492}\special{fp}%
\special{pa -283 -492}\special{pa -263 -492}\special{fp}\special{pa -244 -492}\special{pa -224 -492}\special{fp}%
\special{pa -205 -492}\special{pa -185 -492}\special{fp}\special{pa -166 -492}\special{pa -146 -492}\special{fp}%
\special{pa -127 -492}\special{pa -107 -492}\special{fp}\special{pa -88 -492}\special{pa -68 -492}\special{fp}%
\special{pa -49 -492}\special{pa -29 -492}\special{fp}\special{pa -10 -492}\special{pa 10 -492}\special{fp}%
\special{pa 29 -492}\special{pa 49 -492}\special{fp}\special{pa 68 -492}\special{pa 88 -492}\special{fp}%
\special{pa 107 -492}\special{pa 127 -492}\special{fp}\special{pa 146 -492}\special{pa 166 -492}\special{fp}%
\special{pa 185 -492}\special{pa 205 -492}\special{fp}\special{pa 224 -492}\special{pa 244 -492}\special{fp}%
\special{pa 263 -492}\special{pa 283 -492}\special{fp}\special{pa 302 -492}\special{pa 322 -492}\special{fp}%
\special{pa 341 -492}\special{pa 361 -492}\special{fp}\special{pa 380 -492}\special{pa 400 -492}\special{fp}%
\special{pa 419 -492}\special{pa 439 -492}\special{fp}\special{pa 458 -492}\special{pa 478 -492}\special{fp}%
\special{pa 497 -492}\special{pa 516 -492}\special{fp}\special{pa 536 -492}\special{pa 555 -492}\special{fp}%
\special{pa 575 -492}\special{pa 594 -492}\special{fp}\special{pa 614 -492}\special{pa 633 -492}\special{fp}%
\special{pa 653 -492}\special{pa 672 -492}\special{fp}\special{pa 692 -492}\special{pa 711 -492}\special{fp}%
\special{pa 731 -492}\special{pa 750 -492}\special{fp}\special{pa 770 -492}\special{pa 789 -492}\special{fp}%
\special{pa 809 -492}\special{pa 828 -492}\special{fp}\special{pa 848 -492}\special{pa 867 -492}\special{fp}%
\special{pa 887 -492}\special{pa 906 -492}\special{fp}\special{pa 926 -492}\special{pa 945 -492}\special{fp}%
\special{pa 965 -492}\special{pa 984 -492}\special{fp}%
%
\special{pa -984 -689}\special{pa -965 -689}\special{fp}\special{pa -945 -689}\special{pa -926 -689}\special{fp}%
\special{pa -906 -689}\special{pa -887 -689}\special{fp}\special{pa -867 -689}\special{pa -848 -689}\special{fp}%
\special{pa -828 -689}\special{pa -809 -689}\special{fp}\special{pa -789 -689}\special{pa -770 -689}\special{fp}%
\special{pa -750 -689}\special{pa -731 -689}\special{fp}\special{pa -711 -689}\special{pa -692 -689}\special{fp}%
\special{pa -672 -689}\special{pa -653 -689}\special{fp}\special{pa -633 -689}\special{pa -614 -689}\special{fp}%
\special{pa -594 -689}\special{pa -575 -689}\special{fp}\special{pa -555 -689}\special{pa -536 -689}\special{fp}%
\special{pa -516 -689}\special{pa -497 -689}\special{fp}\special{pa -478 -689}\special{pa -458 -689}\special{fp}%
\special{pa -439 -689}\special{pa -419 -689}\special{fp}\special{pa -400 -689}\special{pa -380 -689}\special{fp}%
\special{pa -361 -689}\special{pa -341 -689}\special{fp}\special{pa -322 -689}\special{pa -302 -689}\special{fp}%
\special{pa -283 -689}\special{pa -263 -689}\special{fp}\special{pa -244 -689}\special{pa -224 -689}\special{fp}%
\special{pa -205 -689}\special{pa -185 -689}\special{fp}\special{pa -166 -689}\special{pa -146 -689}\special{fp}%
\special{pa -127 -689}\special{pa -107 -689}\special{fp}\special{pa -88 -689}\special{pa -68 -689}\special{fp}%
\special{pa -49 -689}\special{pa -29 -689}\special{fp}\special{pa -10 -689}\special{pa 10 -689}\special{fp}%
\special{pa 29 -689}\special{pa 49 -689}\special{fp}\special{pa 68 -689}\special{pa 88 -689}\special{fp}%
\special{pa 107 -689}\special{pa 127 -689}\special{fp}\special{pa 146 -689}\special{pa 166 -689}\special{fp}%
\special{pa 185 -689}\special{pa 205 -689}\special{fp}\special{pa 224 -689}\special{pa 244 -689}\special{fp}%
\special{pa 263 -689}\special{pa 283 -689}\special{fp}\special{pa 302 -689}\special{pa 322 -689}\special{fp}%
\special{pa 341 -689}\special{pa 361 -689}\special{fp}\special{pa 380 -689}\special{pa 400 -689}\special{fp}%
\special{pa 419 -689}\special{pa 439 -689}\special{fp}\special{pa 458 -689}\special{pa 478 -689}\special{fp}%
\special{pa 497 -689}\special{pa 516 -689}\special{fp}\special{pa 536 -689}\special{pa 555 -689}\special{fp}%
\special{pa 575 -689}\special{pa 594 -689}\special{fp}\special{pa 614 -689}\special{pa 633 -689}\special{fp}%
\special{pa 653 -689}\special{pa 672 -689}\special{fp}\special{pa 692 -689}\special{pa 711 -689}\special{fp}%
\special{pa 731 -689}\special{pa 750 -689}\special{fp}\special{pa 770 -689}\special{pa 789 -689}\special{fp}%
\special{pa 809 -689}\special{pa 828 -689}\special{fp}\special{pa 848 -689}\special{pa 867 -689}\special{fp}%
\special{pa 887 -689}\special{pa 906 -689}\special{fp}\special{pa 926 -689}\special{pa 945 -689}\special{fp}%
\special{pa 965 -689}\special{pa 984 -689}\special{fp}%
%
\special{pa -984 -886}\special{pa -965 -886}\special{fp}\special{pa -945 -886}\special{pa -926 -886}\special{fp}%
\special{pa -906 -886}\special{pa -887 -886}\special{fp}\special{pa -867 -886}\special{pa -848 -886}\special{fp}%
\special{pa -828 -886}\special{pa -809 -886}\special{fp}\special{pa -789 -886}\special{pa -770 -886}\special{fp}%
\special{pa -750 -886}\special{pa -731 -886}\special{fp}\special{pa -711 -886}\special{pa -692 -886}\special{fp}%
\special{pa -672 -886}\special{pa -653 -886}\special{fp}\special{pa -633 -886}\special{pa -614 -886}\special{fp}%
\special{pa -594 -886}\special{pa -575 -886}\special{fp}\special{pa -555 -886}\special{pa -536 -886}\special{fp}%
\special{pa -516 -886}\special{pa -497 -886}\special{fp}\special{pa -478 -886}\special{pa -458 -886}\special{fp}%
\special{pa -439 -886}\special{pa -419 -886}\special{fp}\special{pa -400 -886}\special{pa -380 -886}\special{fp}%
\special{pa -361 -886}\special{pa -341 -886}\special{fp}\special{pa -322 -886}\special{pa -302 -886}\special{fp}%
\special{pa -283 -886}\special{pa -263 -886}\special{fp}\special{pa -244 -886}\special{pa -224 -886}\special{fp}%
\special{pa -205 -886}\special{pa -185 -886}\special{fp}\special{pa -166 -886}\special{pa -146 -886}\special{fp}%
\special{pa -127 -886}\special{pa -107 -886}\special{fp}\special{pa -88 -886}\special{pa -68 -886}\special{fp}%
\special{pa -49 -886}\special{pa -29 -886}\special{fp}\special{pa -10 -886}\special{pa 10 -886}\special{fp}%
\special{pa 29 -886}\special{pa 49 -886}\special{fp}\special{pa 68 -886}\special{pa 88 -886}\special{fp}%
\special{pa 107 -886}\special{pa 127 -886}\special{fp}\special{pa 146 -886}\special{pa 166 -886}\special{fp}%
\special{pa 185 -886}\special{pa 205 -886}\special{fp}\special{pa 224 -886}\special{pa 244 -886}\special{fp}%
\special{pa 263 -886}\special{pa 283 -886}\special{fp}\special{pa 302 -886}\special{pa 322 -886}\special{fp}%
\special{pa 341 -886}\special{pa 361 -886}\special{fp}\special{pa 380 -886}\special{pa 400 -886}\special{fp}%
\special{pa 419 -886}\special{pa 439 -886}\special{fp}\special{pa 458 -886}\special{pa 478 -886}\special{fp}%
\special{pa 497 -886}\special{pa 516 -886}\special{fp}\special{pa 536 -886}\special{pa 555 -886}\special{fp}%
\special{pa 575 -886}\special{pa 594 -886}\special{fp}\special{pa 614 -886}\special{pa 633 -886}\special{fp}%
\special{pa 653 -886}\special{pa 672 -886}\special{fp}\special{pa 692 -886}\special{pa 711 -886}\special{fp}%
\special{pa 731 -886}\special{pa 750 -886}\special{fp}\special{pa 770 -886}\special{pa 789 -886}\special{fp}%
\special{pa 809 -886}\special{pa 828 -886}\special{fp}\special{pa 848 -886}\special{pa 867 -886}\special{fp}%
\special{pa 887 -886}\special{pa 906 -886}\special{fp}\special{pa 926 -886}\special{pa 945 -886}\special{fp}%
\special{pa 965 -886}\special{pa 984 -886}\special{fp}%
%
\special{pa  -984   984}\special{pa   984   984}%
\special{fp}%
\special{pa  -984   787}\special{pa   984   787}%
\special{fp}%
\special{pa  -984   591}\special{pa   984   591}%
\special{fp}%
\special{pa  -984   394}\special{pa   984   394}%
\special{fp}%
\special{pa  -984   197}\special{pa   984   197}%
\special{fp}%
\special{pa  -984    -0}\special{pa   984    -0}%
\special{fp}%
\special{pa  -984  -197}\special{pa   984  -197}%
\special{fp}%
\special{pa  -984  -394}\special{pa   984  -394}%
\special{fp}%
\special{pa  -984  -591}\special{pa   984  -591}%
\special{fp}%
\special{pa  -984  -787}\special{pa   984  -787}%
\special{fp}%
\special{pa  -984  -984}\special{pa   984  -984}%
\special{fp}%
\special{pa -886 984}\special{pa -886 965}\special{fp}\special{pa -886 945}\special{pa -886 926}\special{fp}%
\special{pa -886 906}\special{pa -886 887}\special{fp}\special{pa -886 867}\special{pa -886 848}\special{fp}%
\special{pa -886 828}\special{pa -886 809}\special{fp}\special{pa -886 789}\special{pa -886 770}\special{fp}%
\special{pa -886 750}\special{pa -886 731}\special{fp}\special{pa -886 711}\special{pa -886 692}\special{fp}%
\special{pa -886 672}\special{pa -886 653}\special{fp}\special{pa -886 633}\special{pa -886 614}\special{fp}%
\special{pa -886 594}\special{pa -886 575}\special{fp}\special{pa -886 555}\special{pa -886 536}\special{fp}%
\special{pa -886 516}\special{pa -886 497}\special{fp}\special{pa -886 478}\special{pa -886 458}\special{fp}%
\special{pa -886 439}\special{pa -886 419}\special{fp}\special{pa -886 400}\special{pa -886 380}\special{fp}%
\special{pa -886 361}\special{pa -886 341}\special{fp}\special{pa -886 322}\special{pa -886 302}\special{fp}%
\special{pa -886 283}\special{pa -886 263}\special{fp}\special{pa -886 244}\special{pa -886 224}\special{fp}%
\special{pa -886 205}\special{pa -886 185}\special{fp}\special{pa -886 166}\special{pa -886 146}\special{fp}%
\special{pa -886 127}\special{pa -886 107}\special{fp}\special{pa -886 88}\special{pa -886 68}\special{fp}%
\special{pa -886 49}\special{pa -886 29}\special{fp}\special{pa -886 10}\special{pa -886 -10}\special{fp}%
\special{pa -886 -29}\special{pa -886 -49}\special{fp}\special{pa -886 -68}\special{pa -886 -88}\special{fp}%
\special{pa -886 -107}\special{pa -886 -127}\special{fp}\special{pa -886 -146}\special{pa -886 -166}\special{fp}%
\special{pa -886 -185}\special{pa -886 -205}\special{fp}\special{pa -886 -224}\special{pa -886 -244}\special{fp}%
\special{pa -886 -263}\special{pa -886 -283}\special{fp}\special{pa -886 -302}\special{pa -886 -322}\special{fp}%
\special{pa -886 -341}\special{pa -886 -361}\special{fp}\special{pa -886 -380}\special{pa -886 -400}\special{fp}%
\special{pa -886 -419}\special{pa -886 -439}\special{fp}\special{pa -886 -458}\special{pa -886 -478}\special{fp}%
\special{pa -886 -497}\special{pa -886 -516}\special{fp}\special{pa -886 -536}\special{pa -886 -555}\special{fp}%
\special{pa -886 -575}\special{pa -886 -594}\special{fp}\special{pa -886 -614}\special{pa -886 -633}\special{fp}%
\special{pa -886 -653}\special{pa -886 -672}\special{fp}\special{pa -886 -692}\special{pa -886 -711}\special{fp}%
\special{pa -886 -731}\special{pa -886 -750}\special{fp}\special{pa -886 -770}\special{pa -886 -789}\special{fp}%
\special{pa -886 -809}\special{pa -886 -828}\special{fp}\special{pa -886 -848}\special{pa -886 -867}\special{fp}%
\special{pa -886 -887}\special{pa -886 -906}\special{fp}\special{pa -886 -926}\special{pa -886 -945}\special{fp}%
\special{pa -886 -965}\special{pa -886 -984}\special{fp}%
%
\special{pa -689 984}\special{pa -689 965}\special{fp}\special{pa -689 945}\special{pa -689 926}\special{fp}%
\special{pa -689 906}\special{pa -689 887}\special{fp}\special{pa -689 867}\special{pa -689 848}\special{fp}%
\special{pa -689 828}\special{pa -689 809}\special{fp}\special{pa -689 789}\special{pa -689 770}\special{fp}%
\special{pa -689 750}\special{pa -689 731}\special{fp}\special{pa -689 711}\special{pa -689 692}\special{fp}%
\special{pa -689 672}\special{pa -689 653}\special{fp}\special{pa -689 633}\special{pa -689 614}\special{fp}%
\special{pa -689 594}\special{pa -689 575}\special{fp}\special{pa -689 555}\special{pa -689 536}\special{fp}%
\special{pa -689 516}\special{pa -689 497}\special{fp}\special{pa -689 478}\special{pa -689 458}\special{fp}%
\special{pa -689 439}\special{pa -689 419}\special{fp}\special{pa -689 400}\special{pa -689 380}\special{fp}%
\special{pa -689 361}\special{pa -689 341}\special{fp}\special{pa -689 322}\special{pa -689 302}\special{fp}%
\special{pa -689 283}\special{pa -689 263}\special{fp}\special{pa -689 244}\special{pa -689 224}\special{fp}%
\special{pa -689 205}\special{pa -689 185}\special{fp}\special{pa -689 166}\special{pa -689 146}\special{fp}%
\special{pa -689 127}\special{pa -689 107}\special{fp}\special{pa -689 88}\special{pa -689 68}\special{fp}%
\special{pa -689 49}\special{pa -689 29}\special{fp}\special{pa -689 10}\special{pa -689 -10}\special{fp}%
\special{pa -689 -29}\special{pa -689 -49}\special{fp}\special{pa -689 -68}\special{pa -689 -88}\special{fp}%
\special{pa -689 -107}\special{pa -689 -127}\special{fp}\special{pa -689 -146}\special{pa -689 -166}\special{fp}%
\special{pa -689 -185}\special{pa -689 -205}\special{fp}\special{pa -689 -224}\special{pa -689 -244}\special{fp}%
\special{pa -689 -263}\special{pa -689 -283}\special{fp}\special{pa -689 -302}\special{pa -689 -322}\special{fp}%
\special{pa -689 -341}\special{pa -689 -361}\special{fp}\special{pa -689 -380}\special{pa -689 -400}\special{fp}%
\special{pa -689 -419}\special{pa -689 -439}\special{fp}\special{pa -689 -458}\special{pa -689 -478}\special{fp}%
\special{pa -689 -497}\special{pa -689 -516}\special{fp}\special{pa -689 -536}\special{pa -689 -555}\special{fp}%
\special{pa -689 -575}\special{pa -689 -594}\special{fp}\special{pa -689 -614}\special{pa -689 -633}\special{fp}%
\special{pa -689 -653}\special{pa -689 -672}\special{fp}\special{pa -689 -692}\special{pa -689 -711}\special{fp}%
\special{pa -689 -731}\special{pa -689 -750}\special{fp}\special{pa -689 -770}\special{pa -689 -789}\special{fp}%
\special{pa -689 -809}\special{pa -689 -828}\special{fp}\special{pa -689 -848}\special{pa -689 -867}\special{fp}%
\special{pa -689 -887}\special{pa -689 -906}\special{fp}\special{pa -689 -926}\special{pa -689 -945}\special{fp}%
\special{pa -689 -965}\special{pa -689 -984}\special{fp}%
%
\special{pa -492 984}\special{pa -492 965}\special{fp}\special{pa -492 945}\special{pa -492 926}\special{fp}%
\special{pa -492 906}\special{pa -492 887}\special{fp}\special{pa -492 867}\special{pa -492 848}\special{fp}%
\special{pa -492 828}\special{pa -492 809}\special{fp}\special{pa -492 789}\special{pa -492 770}\special{fp}%
\special{pa -492 750}\special{pa -492 731}\special{fp}\special{pa -492 711}\special{pa -492 692}\special{fp}%
\special{pa -492 672}\special{pa -492 653}\special{fp}\special{pa -492 633}\special{pa -492 614}\special{fp}%
\special{pa -492 594}\special{pa -492 575}\special{fp}\special{pa -492 555}\special{pa -492 536}\special{fp}%
\special{pa -492 516}\special{pa -492 497}\special{fp}\special{pa -492 478}\special{pa -492 458}\special{fp}%
\special{pa -492 439}\special{pa -492 419}\special{fp}\special{pa -492 400}\special{pa -492 380}\special{fp}%
\special{pa -492 361}\special{pa -492 341}\special{fp}\special{pa -492 322}\special{pa -492 302}\special{fp}%
\special{pa -492 283}\special{pa -492 263}\special{fp}\special{pa -492 244}\special{pa -492 224}\special{fp}%
\special{pa -492 205}\special{pa -492 185}\special{fp}\special{pa -492 166}\special{pa -492 146}\special{fp}%
\special{pa -492 127}\special{pa -492 107}\special{fp}\special{pa -492 88}\special{pa -492 68}\special{fp}%
\special{pa -492 49}\special{pa -492 29}\special{fp}\special{pa -492 10}\special{pa -492 -10}\special{fp}%
\special{pa -492 -29}\special{pa -492 -49}\special{fp}\special{pa -492 -68}\special{pa -492 -88}\special{fp}%
\special{pa -492 -107}\special{pa -492 -127}\special{fp}\special{pa -492 -146}\special{pa -492 -166}\special{fp}%
\special{pa -492 -185}\special{pa -492 -205}\special{fp}\special{pa -492 -224}\special{pa -492 -244}\special{fp}%
\special{pa -492 -263}\special{pa -492 -283}\special{fp}\special{pa -492 -302}\special{pa -492 -322}\special{fp}%
\special{pa -492 -341}\special{pa -492 -361}\special{fp}\special{pa -492 -380}\special{pa -492 -400}\special{fp}%
\special{pa -492 -419}\special{pa -492 -439}\special{fp}\special{pa -492 -458}\special{pa -492 -478}\special{fp}%
\special{pa -492 -497}\special{pa -492 -516}\special{fp}\special{pa -492 -536}\special{pa -492 -555}\special{fp}%
\special{pa -492 -575}\special{pa -492 -594}\special{fp}\special{pa -492 -614}\special{pa -492 -633}\special{fp}%
\special{pa -492 -653}\special{pa -492 -672}\special{fp}\special{pa -492 -692}\special{pa -492 -711}\special{fp}%
\special{pa -492 -731}\special{pa -492 -750}\special{fp}\special{pa -492 -770}\special{pa -492 -789}\special{fp}%
\special{pa -492 -809}\special{pa -492 -828}\special{fp}\special{pa -492 -848}\special{pa -492 -867}\special{fp}%
\special{pa -492 -887}\special{pa -492 -906}\special{fp}\special{pa -492 -926}\special{pa -492 -945}\special{fp}%
\special{pa -492 -965}\special{pa -492 -984}\special{fp}%
%
\special{pa -295 984}\special{pa -295 965}\special{fp}\special{pa -295 945}\special{pa -295 926}\special{fp}%
\special{pa -295 906}\special{pa -295 887}\special{fp}\special{pa -295 867}\special{pa -295 848}\special{fp}%
\special{pa -295 828}\special{pa -295 809}\special{fp}\special{pa -295 789}\special{pa -295 770}\special{fp}%
\special{pa -295 750}\special{pa -295 731}\special{fp}\special{pa -295 711}\special{pa -295 692}\special{fp}%
\special{pa -295 672}\special{pa -295 653}\special{fp}\special{pa -295 633}\special{pa -295 614}\special{fp}%
\special{pa -295 594}\special{pa -295 575}\special{fp}\special{pa -295 555}\special{pa -295 536}\special{fp}%
\special{pa -295 516}\special{pa -295 497}\special{fp}\special{pa -295 478}\special{pa -295 458}\special{fp}%
\special{pa -295 439}\special{pa -295 419}\special{fp}\special{pa -295 400}\special{pa -295 380}\special{fp}%
\special{pa -295 361}\special{pa -295 341}\special{fp}\special{pa -295 322}\special{pa -295 302}\special{fp}%
\special{pa -295 283}\special{pa -295 263}\special{fp}\special{pa -295 244}\special{pa -295 224}\special{fp}%
\special{pa -295 205}\special{pa -295 185}\special{fp}\special{pa -295 166}\special{pa -295 146}\special{fp}%
\special{pa -295 127}\special{pa -295 107}\special{fp}\special{pa -295 88}\special{pa -295 68}\special{fp}%
\special{pa -295 49}\special{pa -295 29}\special{fp}\special{pa -295 10}\special{pa -295 -10}\special{fp}%
\special{pa -295 -29}\special{pa -295 -49}\special{fp}\special{pa -295 -68}\special{pa -295 -88}\special{fp}%
\special{pa -295 -107}\special{pa -295 -127}\special{fp}\special{pa -295 -146}\special{pa -295 -166}\special{fp}%
\special{pa -295 -185}\special{pa -295 -205}\special{fp}\special{pa -295 -224}\special{pa -295 -244}\special{fp}%
\special{pa -295 -263}\special{pa -295 -283}\special{fp}\special{pa -295 -302}\special{pa -295 -322}\special{fp}%
\special{pa -295 -341}\special{pa -295 -361}\special{fp}\special{pa -295 -380}\special{pa -295 -400}\special{fp}%
\special{pa -295 -419}\special{pa -295 -439}\special{fp}\special{pa -295 -458}\special{pa -295 -478}\special{fp}%
\special{pa -295 -497}\special{pa -295 -516}\special{fp}\special{pa -295 -536}\special{pa -295 -555}\special{fp}%
\special{pa -295 -575}\special{pa -295 -594}\special{fp}\special{pa -295 -614}\special{pa -295 -633}\special{fp}%
\special{pa -295 -653}\special{pa -295 -672}\special{fp}\special{pa -295 -692}\special{pa -295 -711}\special{fp}%
\special{pa -295 -731}\special{pa -295 -750}\special{fp}\special{pa -295 -770}\special{pa -295 -789}\special{fp}%
\special{pa -295 -809}\special{pa -295 -828}\special{fp}\special{pa -295 -848}\special{pa -295 -867}\special{fp}%
\special{pa -295 -887}\special{pa -295 -906}\special{fp}\special{pa -295 -926}\special{pa -295 -945}\special{fp}%
\special{pa -295 -965}\special{pa -295 -984}\special{fp}%
%
\special{pa -98 984}\special{pa -98 965}\special{fp}\special{pa -98 945}\special{pa -98 926}\special{fp}%
\special{pa -98 906}\special{pa -98 887}\special{fp}\special{pa -98 867}\special{pa -98 848}\special{fp}%
\special{pa -98 828}\special{pa -98 809}\special{fp}\special{pa -98 789}\special{pa -98 770}\special{fp}%
\special{pa -98 750}\special{pa -98 731}\special{fp}\special{pa -98 711}\special{pa -98 692}\special{fp}%
\special{pa -98 672}\special{pa -98 653}\special{fp}\special{pa -98 633}\special{pa -98 614}\special{fp}%
\special{pa -98 594}\special{pa -98 575}\special{fp}\special{pa -98 555}\special{pa -98 536}\special{fp}%
\special{pa -98 516}\special{pa -98 497}\special{fp}\special{pa -98 478}\special{pa -98 458}\special{fp}%
\special{pa -98 439}\special{pa -98 419}\special{fp}\special{pa -98 400}\special{pa -98 380}\special{fp}%
\special{pa -98 361}\special{pa -98 341}\special{fp}\special{pa -98 322}\special{pa -98 302}\special{fp}%
\special{pa -98 283}\special{pa -98 263}\special{fp}\special{pa -98 244}\special{pa -98 224}\special{fp}%
\special{pa -98 205}\special{pa -98 185}\special{fp}\special{pa -98 166}\special{pa -98 146}\special{fp}%
\special{pa -98 127}\special{pa -98 107}\special{fp}\special{pa -98 88}\special{pa -98 68}\special{fp}%
\special{pa -98 49}\special{pa -98 29}\special{fp}\special{pa -98 10}\special{pa -98 -10}\special{fp}%
\special{pa -98 -29}\special{pa -98 -49}\special{fp}\special{pa -98 -68}\special{pa -98 -88}\special{fp}%
\special{pa -98 -107}\special{pa -98 -127}\special{fp}\special{pa -98 -146}\special{pa -98 -166}\special{fp}%
\special{pa -98 -185}\special{pa -98 -205}\special{fp}\special{pa -98 -224}\special{pa -98 -244}\special{fp}%
\special{pa -98 -263}\special{pa -98 -283}\special{fp}\special{pa -98 -302}\special{pa -98 -322}\special{fp}%
\special{pa -98 -341}\special{pa -98 -361}\special{fp}\special{pa -98 -380}\special{pa -98 -400}\special{fp}%
\special{pa -98 -419}\special{pa -98 -439}\special{fp}\special{pa -98 -458}\special{pa -98 -478}\special{fp}%
\special{pa -98 -497}\special{pa -98 -516}\special{fp}\special{pa -98 -536}\special{pa -98 -555}\special{fp}%
\special{pa -98 -575}\special{pa -98 -594}\special{fp}\special{pa -98 -614}\special{pa -98 -633}\special{fp}%
\special{pa -98 -653}\special{pa -98 -672}\special{fp}\special{pa -98 -692}\special{pa -98 -711}\special{fp}%
\special{pa -98 -731}\special{pa -98 -750}\special{fp}\special{pa -98 -770}\special{pa -98 -789}\special{fp}%
\special{pa -98 -809}\special{pa -98 -828}\special{fp}\special{pa -98 -848}\special{pa -98 -867}\special{fp}%
\special{pa -98 -887}\special{pa -98 -906}\special{fp}\special{pa -98 -926}\special{pa -98 -945}\special{fp}%
\special{pa -98 -965}\special{pa -98 -984}\special{fp}%
%
\special{pa 98 984}\special{pa 98 965}\special{fp}\special{pa 98 945}\special{pa 98 926}\special{fp}%
\special{pa 98 906}\special{pa 98 887}\special{fp}\special{pa 98 867}\special{pa 98 848}\special{fp}%
\special{pa 98 828}\special{pa 98 809}\special{fp}\special{pa 98 789}\special{pa 98 770}\special{fp}%
\special{pa 98 750}\special{pa 98 731}\special{fp}\special{pa 98 711}\special{pa 98 692}\special{fp}%
\special{pa 98 672}\special{pa 98 653}\special{fp}\special{pa 98 633}\special{pa 98 614}\special{fp}%
\special{pa 98 594}\special{pa 98 575}\special{fp}\special{pa 98 555}\special{pa 98 536}\special{fp}%
\special{pa 98 516}\special{pa 98 497}\special{fp}\special{pa 98 478}\special{pa 98 458}\special{fp}%
\special{pa 98 439}\special{pa 98 419}\special{fp}\special{pa 98 400}\special{pa 98 380}\special{fp}%
\special{pa 98 361}\special{pa 98 341}\special{fp}\special{pa 98 322}\special{pa 98 302}\special{fp}%
\special{pa 98 283}\special{pa 98 263}\special{fp}\special{pa 98 244}\special{pa 98 224}\special{fp}%
\special{pa 98 205}\special{pa 98 185}\special{fp}\special{pa 98 166}\special{pa 98 146}\special{fp}%
\special{pa 98 127}\special{pa 98 107}\special{fp}\special{pa 98 88}\special{pa 98 68}\special{fp}%
\special{pa 98 49}\special{pa 98 29}\special{fp}\special{pa 98 10}\special{pa 98 -10}\special{fp}%
\special{pa 98 -29}\special{pa 98 -49}\special{fp}\special{pa 98 -68}\special{pa 98 -88}\special{fp}%
\special{pa 98 -107}\special{pa 98 -127}\special{fp}\special{pa 98 -146}\special{pa 98 -166}\special{fp}%
\special{pa 98 -185}\special{pa 98 -205}\special{fp}\special{pa 98 -224}\special{pa 98 -244}\special{fp}%
\special{pa 98 -263}\special{pa 98 -283}\special{fp}\special{pa 98 -302}\special{pa 98 -322}\special{fp}%
\special{pa 98 -341}\special{pa 98 -361}\special{fp}\special{pa 98 -380}\special{pa 98 -400}\special{fp}%
\special{pa 98 -419}\special{pa 98 -439}\special{fp}\special{pa 98 -458}\special{pa 98 -478}\special{fp}%
\special{pa 98 -497}\special{pa 98 -516}\special{fp}\special{pa 98 -536}\special{pa 98 -555}\special{fp}%
\special{pa 98 -575}\special{pa 98 -594}\special{fp}\special{pa 98 -614}\special{pa 98 -633}\special{fp}%
\special{pa 98 -653}\special{pa 98 -672}\special{fp}\special{pa 98 -692}\special{pa 98 -711}\special{fp}%
\special{pa 98 -731}\special{pa 98 -750}\special{fp}\special{pa 98 -770}\special{pa 98 -789}\special{fp}%
\special{pa 98 -809}\special{pa 98 -828}\special{fp}\special{pa 98 -848}\special{pa 98 -867}\special{fp}%
\special{pa 98 -887}\special{pa 98 -906}\special{fp}\special{pa 98 -926}\special{pa 98 -945}\special{fp}%
\special{pa 98 -965}\special{pa 98 -984}\special{fp}%
%
\special{pa 295 984}\special{pa 295 965}\special{fp}\special{pa 295 945}\special{pa 295 926}\special{fp}%
\special{pa 295 906}\special{pa 295 887}\special{fp}\special{pa 295 867}\special{pa 295 848}\special{fp}%
\special{pa 295 828}\special{pa 295 809}\special{fp}\special{pa 295 789}\special{pa 295 770}\special{fp}%
\special{pa 295 750}\special{pa 295 731}\special{fp}\special{pa 295 711}\special{pa 295 692}\special{fp}%
\special{pa 295 672}\special{pa 295 653}\special{fp}\special{pa 295 633}\special{pa 295 614}\special{fp}%
\special{pa 295 594}\special{pa 295 575}\special{fp}\special{pa 295 555}\special{pa 295 536}\special{fp}%
\special{pa 295 516}\special{pa 295 497}\special{fp}\special{pa 295 478}\special{pa 295 458}\special{fp}%
\special{pa 295 439}\special{pa 295 419}\special{fp}\special{pa 295 400}\special{pa 295 380}\special{fp}%
\special{pa 295 361}\special{pa 295 341}\special{fp}\special{pa 295 322}\special{pa 295 302}\special{fp}%
\special{pa 295 283}\special{pa 295 263}\special{fp}\special{pa 295 244}\special{pa 295 224}\special{fp}%
\special{pa 295 205}\special{pa 295 185}\special{fp}\special{pa 295 166}\special{pa 295 146}\special{fp}%
\special{pa 295 127}\special{pa 295 107}\special{fp}\special{pa 295 88}\special{pa 295 68}\special{fp}%
\special{pa 295 49}\special{pa 295 29}\special{fp}\special{pa 295 10}\special{pa 295 -10}\special{fp}%
\special{pa 295 -29}\special{pa 295 -49}\special{fp}\special{pa 295 -68}\special{pa 295 -88}\special{fp}%
\special{pa 295 -107}\special{pa 295 -127}\special{fp}\special{pa 295 -146}\special{pa 295 -166}\special{fp}%
\special{pa 295 -185}\special{pa 295 -205}\special{fp}\special{pa 295 -224}\special{pa 295 -244}\special{fp}%
\special{pa 295 -263}\special{pa 295 -283}\special{fp}\special{pa 295 -302}\special{pa 295 -322}\special{fp}%
\special{pa 295 -341}\special{pa 295 -361}\special{fp}\special{pa 295 -380}\special{pa 295 -400}\special{fp}%
\special{pa 295 -419}\special{pa 295 -439}\special{fp}\special{pa 295 -458}\special{pa 295 -478}\special{fp}%
\special{pa 295 -497}\special{pa 295 -516}\special{fp}\special{pa 295 -536}\special{pa 295 -555}\special{fp}%
\special{pa 295 -575}\special{pa 295 -594}\special{fp}\special{pa 295 -614}\special{pa 295 -633}\special{fp}%
\special{pa 295 -653}\special{pa 295 -672}\special{fp}\special{pa 295 -692}\special{pa 295 -711}\special{fp}%
\special{pa 295 -731}\special{pa 295 -750}\special{fp}\special{pa 295 -770}\special{pa 295 -789}\special{fp}%
\special{pa 295 -809}\special{pa 295 -828}\special{fp}\special{pa 295 -848}\special{pa 295 -867}\special{fp}%
\special{pa 295 -887}\special{pa 295 -906}\special{fp}\special{pa 295 -926}\special{pa 295 -945}\special{fp}%
\special{pa 295 -965}\special{pa 295 -984}\special{fp}%
%
\special{pa 492 984}\special{pa 492 965}\special{fp}\special{pa 492 945}\special{pa 492 926}\special{fp}%
\special{pa 492 906}\special{pa 492 887}\special{fp}\special{pa 492 867}\special{pa 492 848}\special{fp}%
\special{pa 492 828}\special{pa 492 809}\special{fp}\special{pa 492 789}\special{pa 492 770}\special{fp}%
\special{pa 492 750}\special{pa 492 731}\special{fp}\special{pa 492 711}\special{pa 492 692}\special{fp}%
\special{pa 492 672}\special{pa 492 653}\special{fp}\special{pa 492 633}\special{pa 492 614}\special{fp}%
\special{pa 492 594}\special{pa 492 575}\special{fp}\special{pa 492 555}\special{pa 492 536}\special{fp}%
\special{pa 492 516}\special{pa 492 497}\special{fp}\special{pa 492 478}\special{pa 492 458}\special{fp}%
\special{pa 492 439}\special{pa 492 419}\special{fp}\special{pa 492 400}\special{pa 492 380}\special{fp}%
\special{pa 492 361}\special{pa 492 341}\special{fp}\special{pa 492 322}\special{pa 492 302}\special{fp}%
\special{pa 492 283}\special{pa 492 263}\special{fp}\special{pa 492 244}\special{pa 492 224}\special{fp}%
\special{pa 492 205}\special{pa 492 185}\special{fp}\special{pa 492 166}\special{pa 492 146}\special{fp}%
\special{pa 492 127}\special{pa 492 107}\special{fp}\special{pa 492 88}\special{pa 492 68}\special{fp}%
\special{pa 492 49}\special{pa 492 29}\special{fp}\special{pa 492 10}\special{pa 492 -10}\special{fp}%
\special{pa 492 -29}\special{pa 492 -49}\special{fp}\special{pa 492 -68}\special{pa 492 -88}\special{fp}%
\special{pa 492 -107}\special{pa 492 -127}\special{fp}\special{pa 492 -146}\special{pa 492 -166}\special{fp}%
\special{pa 492 -185}\special{pa 492 -205}\special{fp}\special{pa 492 -224}\special{pa 492 -244}\special{fp}%
\special{pa 492 -263}\special{pa 492 -283}\special{fp}\special{pa 492 -302}\special{pa 492 -322}\special{fp}%
\special{pa 492 -341}\special{pa 492 -361}\special{fp}\special{pa 492 -380}\special{pa 492 -400}\special{fp}%
\special{pa 492 -419}\special{pa 492 -439}\special{fp}\special{pa 492 -458}\special{pa 492 -478}\special{fp}%
\special{pa 492 -497}\special{pa 492 -516}\special{fp}\special{pa 492 -536}\special{pa 492 -555}\special{fp}%
\special{pa 492 -575}\special{pa 492 -594}\special{fp}\special{pa 492 -614}\special{pa 492 -633}\special{fp}%
\special{pa 492 -653}\special{pa 492 -672}\special{fp}\special{pa 492 -692}\special{pa 492 -711}\special{fp}%
\special{pa 492 -731}\special{pa 492 -750}\special{fp}\special{pa 492 -770}\special{pa 492 -789}\special{fp}%
\special{pa 492 -809}\special{pa 492 -828}\special{fp}\special{pa 492 -848}\special{pa 492 -867}\special{fp}%
\special{pa 492 -887}\special{pa 492 -906}\special{fp}\special{pa 492 -926}\special{pa 492 -945}\special{fp}%
\special{pa 492 -965}\special{pa 492 -984}\special{fp}%
%
\special{pa 689 984}\special{pa 689 965}\special{fp}\special{pa 689 945}\special{pa 689 926}\special{fp}%
\special{pa 689 906}\special{pa 689 887}\special{fp}\special{pa 689 867}\special{pa 689 848}\special{fp}%
\special{pa 689 828}\special{pa 689 809}\special{fp}\special{pa 689 789}\special{pa 689 770}\special{fp}%
\special{pa 689 750}\special{pa 689 731}\special{fp}\special{pa 689 711}\special{pa 689 692}\special{fp}%
\special{pa 689 672}\special{pa 689 653}\special{fp}\special{pa 689 633}\special{pa 689 614}\special{fp}%
\special{pa 689 594}\special{pa 689 575}\special{fp}\special{pa 689 555}\special{pa 689 536}\special{fp}%
\special{pa 689 516}\special{pa 689 497}\special{fp}\special{pa 689 478}\special{pa 689 458}\special{fp}%
\special{pa 689 439}\special{pa 689 419}\special{fp}\special{pa 689 400}\special{pa 689 380}\special{fp}%
\special{pa 689 361}\special{pa 689 341}\special{fp}\special{pa 689 322}\special{pa 689 302}\special{fp}%
\special{pa 689 283}\special{pa 689 263}\special{fp}\special{pa 689 244}\special{pa 689 224}\special{fp}%
\special{pa 689 205}\special{pa 689 185}\special{fp}\special{pa 689 166}\special{pa 689 146}\special{fp}%
\special{pa 689 127}\special{pa 689 107}\special{fp}\special{pa 689 88}\special{pa 689 68}\special{fp}%
\special{pa 689 49}\special{pa 689 29}\special{fp}\special{pa 689 10}\special{pa 689 -10}\special{fp}%
\special{pa 689 -29}\special{pa 689 -49}\special{fp}\special{pa 689 -68}\special{pa 689 -88}\special{fp}%
\special{pa 689 -107}\special{pa 689 -127}\special{fp}\special{pa 689 -146}\special{pa 689 -166}\special{fp}%
\special{pa 689 -185}\special{pa 689 -205}\special{fp}\special{pa 689 -224}\special{pa 689 -244}\special{fp}%
\special{pa 689 -263}\special{pa 689 -283}\special{fp}\special{pa 689 -302}\special{pa 689 -322}\special{fp}%
\special{pa 689 -341}\special{pa 689 -361}\special{fp}\special{pa 689 -380}\special{pa 689 -400}\special{fp}%
\special{pa 689 -419}\special{pa 689 -439}\special{fp}\special{pa 689 -458}\special{pa 689 -478}\special{fp}%
\special{pa 689 -497}\special{pa 689 -516}\special{fp}\special{pa 689 -536}\special{pa 689 -555}\special{fp}%
\special{pa 689 -575}\special{pa 689 -594}\special{fp}\special{pa 689 -614}\special{pa 689 -633}\special{fp}%
\special{pa 689 -653}\special{pa 689 -672}\special{fp}\special{pa 689 -692}\special{pa 689 -711}\special{fp}%
\special{pa 689 -731}\special{pa 689 -750}\special{fp}\special{pa 689 -770}\special{pa 689 -789}\special{fp}%
\special{pa 689 -809}\special{pa 689 -828}\special{fp}\special{pa 689 -848}\special{pa 689 -867}\special{fp}%
\special{pa 689 -887}\special{pa 689 -906}\special{fp}\special{pa 689 -926}\special{pa 689 -945}\special{fp}%
\special{pa 689 -965}\special{pa 689 -984}\special{fp}%
%
\special{pa 886 984}\special{pa 886 965}\special{fp}\special{pa 886 945}\special{pa 886 926}\special{fp}%
\special{pa 886 906}\special{pa 886 887}\special{fp}\special{pa 886 867}\special{pa 886 848}\special{fp}%
\special{pa 886 828}\special{pa 886 809}\special{fp}\special{pa 886 789}\special{pa 886 770}\special{fp}%
\special{pa 886 750}\special{pa 886 731}\special{fp}\special{pa 886 711}\special{pa 886 692}\special{fp}%
\special{pa 886 672}\special{pa 886 653}\special{fp}\special{pa 886 633}\special{pa 886 614}\special{fp}%
\special{pa 886 594}\special{pa 886 575}\special{fp}\special{pa 886 555}\special{pa 886 536}\special{fp}%
\special{pa 886 516}\special{pa 886 497}\special{fp}\special{pa 886 478}\special{pa 886 458}\special{fp}%
\special{pa 886 439}\special{pa 886 419}\special{fp}\special{pa 886 400}\special{pa 886 380}\special{fp}%
\special{pa 886 361}\special{pa 886 341}\special{fp}\special{pa 886 322}\special{pa 886 302}\special{fp}%
\special{pa 886 283}\special{pa 886 263}\special{fp}\special{pa 886 244}\special{pa 886 224}\special{fp}%
\special{pa 886 205}\special{pa 886 185}\special{fp}\special{pa 886 166}\special{pa 886 146}\special{fp}%
\special{pa 886 127}\special{pa 886 107}\special{fp}\special{pa 886 88}\special{pa 886 68}\special{fp}%
\special{pa 886 49}\special{pa 886 29}\special{fp}\special{pa 886 10}\special{pa 886 -10}\special{fp}%
\special{pa 886 -29}\special{pa 886 -49}\special{fp}\special{pa 886 -68}\special{pa 886 -88}\special{fp}%
\special{pa 886 -107}\special{pa 886 -127}\special{fp}\special{pa 886 -146}\special{pa 886 -166}\special{fp}%
\special{pa 886 -185}\special{pa 886 -205}\special{fp}\special{pa 886 -224}\special{pa 886 -244}\special{fp}%
\special{pa 886 -263}\special{pa 886 -283}\special{fp}\special{pa 886 -302}\special{pa 886 -322}\special{fp}%
\special{pa 886 -341}\special{pa 886 -361}\special{fp}\special{pa 886 -380}\special{pa 886 -400}\special{fp}%
\special{pa 886 -419}\special{pa 886 -439}\special{fp}\special{pa 886 -458}\special{pa 886 -478}\special{fp}%
\special{pa 886 -497}\special{pa 886 -516}\special{fp}\special{pa 886 -536}\special{pa 886 -555}\special{fp}%
\special{pa 886 -575}\special{pa 886 -594}\special{fp}\special{pa 886 -614}\special{pa 886 -633}\special{fp}%
\special{pa 886 -653}\special{pa 886 -672}\special{fp}\special{pa 886 -692}\special{pa 886 -711}\special{fp}%
\special{pa 886 -731}\special{pa 886 -750}\special{fp}\special{pa 886 -770}\special{pa 886 -789}\special{fp}%
\special{pa 886 -809}\special{pa 886 -828}\special{fp}\special{pa 886 -848}\special{pa 886 -867}\special{fp}%
\special{pa 886 -887}\special{pa 886 -906}\special{fp}\special{pa 886 -926}\special{pa 886 -945}\special{fp}%
\special{pa 886 -965}\special{pa 886 -984}\special{fp}%
%
\special{pa  -984   984}\special{pa  -984  -984}%
\special{fp}%
\special{pa  -787   984}\special{pa  -787  -984}%
\special{fp}%
\special{pa  -591   984}\special{pa  -591  -984}%
\special{fp}%
\special{pa  -394   984}\special{pa  -394  -984}%
\special{fp}%
\special{pa  -197   984}\special{pa  -197  -984}%
\special{fp}%
\special{pa     0   984}\special{pa     0  -984}%
\special{fp}%
\special{pa   197   984}\special{pa   197  -984}%
\special{fp}%
\special{pa   394   984}\special{pa   394  -984}%
\special{fp}%
\special{pa   591   984}\special{pa   591  -984}%
\special{fp}%
\special{pa   787   984}\special{pa   787  -984}%
\special{fp}%
\special{pa   984   984}\special{pa   984  -984}%
\special{fp}%
\special{pn 8}%
\scriptsize%
\special{pa  -197   -20}\special{pa  -197    20}%
\special{fp}%
\settowidth{\Width}{$-1$}\setlength{\Width}{-0.5\Width}%
\settoheight{\Height}{$-1$}\settodepth{\Depth}{$-1$}\setlength{\Height}{-\Height}%
\put(-1.0000000,-0.2000000){\hspace*{\Width}\raisebox{\Height}{$-1$}}%
%
%
\special{pa   197   -20}\special{pa   197    20}%
\special{fp}%
\settowidth{\Width}{$1$}\setlength{\Width}{-0.5\Width}%
\settoheight{\Height}{$1$}\settodepth{\Depth}{$1$}\setlength{\Height}{-\Height}%
\put(1.0000000,-0.2000000){\hspace*{\Width}\raisebox{\Height}{$1$}}%
%
%
\special{pa    20   197}\special{pa   -20   197}%
\special{fp}%
\settowidth{\Width}{$-1$}\setlength{\Width}{-1\Width}%
\settoheight{\Height}{$-1$}\settodepth{\Depth}{$-1$}\setlength{\Height}{-0.5\Height}\setlength{\Depth}{0.5\Depth}\addtolength{\Height}{\Depth}%
\put(-0.2000000,-1.0000000){\hspace*{\Width}\raisebox{\Height}{$-1$}}%
%
%
\special{pa    20  -197}\special{pa   -20  -197}%
\special{fp}%
\settowidth{\Width}{$1$}\setlength{\Width}{-1\Width}%
\settoheight{\Height}{$1$}\settodepth{\Depth}{$1$}\setlength{\Height}{-0.5\Height}\setlength{\Depth}{0.5\Depth}\addtolength{\Height}{\Depth}%
\put(-0.2000000,1.0000000){\hspace*{\Width}\raisebox{\Height}{$1$}}%
%
%
\special{pn 8}%
\special{pa  -984    -0}\special{pa   965    -0}%
\special{fp}%
\special{pn 8}%
\special{pa 909 24}\special{pa 984 0}\special{pa 909 -24}\special{pa 909 0}\special{pa 909 24}%
\special{sh 1}\special{ip}%
\special{pn 1}%
\special{pa   909    24}\special{pa   984    -0}\special{pa   909   -24}\special{pa   909    -0}%
\special{pa   909    24}\special{pa   984    -0}%
\special{fp}%
\special{pn 8}%
\special{pn 8}%
\special{pa     0   984}\special{pa     0  -965}%
\special{fp}%
\special{pn 8}%
\special{pa 24 -909}\special{pa 0 -984}\special{pa -24 -909}\special{pa 0 -909}\special{pa 24 -909}%
\special{sh 1}\special{ip}%
\special{pn 1}%
\special{pa    24  -909}\special{pa     0  -984}\special{pa   -24  -909}\special{pa     0  -909}%
\special{pa    24  -909}\special{pa     0  -984}%
\special{fp}%
\special{pn 8}%
\settowidth{\Width}{$x\mbox{実軸}$}\setlength{\Width}{0\Width}%
\settoheight{\Height}{$x\mbox{実軸}$}\settodepth{\Depth}{$x\mbox{実軸}$}\setlength{\Height}{-0.5\Height}\setlength{\Depth}{0.5\Depth}\addtolength{\Height}{\Depth}%
\put(5.1000000,0.0000000){\hspace*{\Width}\raisebox{\Height}{$x\mbox{実軸}$}}%
%
\settowidth{\Width}{$y\mbox{虚軸}$}\setlength{\Width}{-0.5\Width}%
\settoheight{\Height}{$y\mbox{虚軸}$}\settodepth{\Depth}{$y\mbox{虚軸}$}\setlength{\Height}{\Depth}%
\put(0.0000000,5.1000000){\hspace*{\Width}\raisebox{\Height}{$y\mbox{虚軸}$}}%
%
\settowidth{\Width}{ }\setlength{\Width}{-1\Width}%
\settoheight{\Height}{ }\settodepth{\Depth}{ }\setlength{\Height}{-\Height}%
\put(-0.1000000,-0.1000000){\hspace*{\Width}\raisebox{\Height}{ }}%
%
\end{picture}}%}
\end{layer}

{\color{red}

\begin{layer}{120}{0}
\putnotec{110}{25}{\small$\bullet$}
\end{layer}

}
\begin{itemize}
\item
$z=a+b\,i$を平面上の点$(a,\ b)$で表す\seteda{50}\\
\eda{$2+3i \leftrightarrow\ \mbox{点}(2,3)$}\\
\eda{$-2+i \leftrightarrow\ \mbox{点}$}\\
\eda{$3 \leftrightarrow\ \mbox{点}$}\\
\eda{$-3 \leftrightarrow\ \mbox{点}$}\\
\eda{$4i \leftrightarrow\ \mbox{点}$}
\end{itemize}

\sameslide

\vspace*{18mm}

\slidepage

\begin{layer}{120}{0}
\putnotese{75}{15}{%%% /Users/takatoosetsuo/Dropbox/2018polytec/lecture/0611/presen/fig/plane1.tex 
%%% Generator=presen0611.cdy 
{\unitlength=5mm%
\begin{picture}%
(10,10)(-5,-5)%
\special{pn 8}%
%
\Large\bf\boldmath%
\small%
\special{pn 4}%
\special{pa -984 886}\special{pa -965 886}\special{fp}\special{pa -945 886}\special{pa -926 886}\special{fp}%
\special{pa -906 886}\special{pa -887 886}\special{fp}\special{pa -867 886}\special{pa -848 886}\special{fp}%
\special{pa -828 886}\special{pa -809 886}\special{fp}\special{pa -789 886}\special{pa -770 886}\special{fp}%
\special{pa -750 886}\special{pa -731 886}\special{fp}\special{pa -711 886}\special{pa -692 886}\special{fp}%
\special{pa -672 886}\special{pa -653 886}\special{fp}\special{pa -633 886}\special{pa -614 886}\special{fp}%
\special{pa -594 886}\special{pa -575 886}\special{fp}\special{pa -555 886}\special{pa -536 886}\special{fp}%
\special{pa -516 886}\special{pa -497 886}\special{fp}\special{pa -478 886}\special{pa -458 886}\special{fp}%
\special{pa -439 886}\special{pa -419 886}\special{fp}\special{pa -400 886}\special{pa -380 886}\special{fp}%
\special{pa -361 886}\special{pa -341 886}\special{fp}\special{pa -322 886}\special{pa -302 886}\special{fp}%
\special{pa -283 886}\special{pa -263 886}\special{fp}\special{pa -244 886}\special{pa -224 886}\special{fp}%
\special{pa -205 886}\special{pa -185 886}\special{fp}\special{pa -166 886}\special{pa -146 886}\special{fp}%
\special{pa -127 886}\special{pa -107 886}\special{fp}\special{pa -88 886}\special{pa -68 886}\special{fp}%
\special{pa -49 886}\special{pa -29 886}\special{fp}\special{pa -10 886}\special{pa 10 886}\special{fp}%
\special{pa 29 886}\special{pa 49 886}\special{fp}\special{pa 68 886}\special{pa 88 886}\special{fp}%
\special{pa 107 886}\special{pa 127 886}\special{fp}\special{pa 146 886}\special{pa 166 886}\special{fp}%
\special{pa 185 886}\special{pa 205 886}\special{fp}\special{pa 224 886}\special{pa 244 886}\special{fp}%
\special{pa 263 886}\special{pa 283 886}\special{fp}\special{pa 302 886}\special{pa 322 886}\special{fp}%
\special{pa 341 886}\special{pa 361 886}\special{fp}\special{pa 380 886}\special{pa 400 886}\special{fp}%
\special{pa 419 886}\special{pa 439 886}\special{fp}\special{pa 458 886}\special{pa 478 886}\special{fp}%
\special{pa 497 886}\special{pa 516 886}\special{fp}\special{pa 536 886}\special{pa 555 886}\special{fp}%
\special{pa 575 886}\special{pa 594 886}\special{fp}\special{pa 614 886}\special{pa 633 886}\special{fp}%
\special{pa 653 886}\special{pa 672 886}\special{fp}\special{pa 692 886}\special{pa 711 886}\special{fp}%
\special{pa 731 886}\special{pa 750 886}\special{fp}\special{pa 770 886}\special{pa 789 886}\special{fp}%
\special{pa 809 886}\special{pa 828 886}\special{fp}\special{pa 848 886}\special{pa 867 886}\special{fp}%
\special{pa 887 886}\special{pa 906 886}\special{fp}\special{pa 926 886}\special{pa 945 886}\special{fp}%
\special{pa 965 886}\special{pa 984 886}\special{fp}%
%
\special{pa -984 689}\special{pa -965 689}\special{fp}\special{pa -945 689}\special{pa -926 689}\special{fp}%
\special{pa -906 689}\special{pa -887 689}\special{fp}\special{pa -867 689}\special{pa -848 689}\special{fp}%
\special{pa -828 689}\special{pa -809 689}\special{fp}\special{pa -789 689}\special{pa -770 689}\special{fp}%
\special{pa -750 689}\special{pa -731 689}\special{fp}\special{pa -711 689}\special{pa -692 689}\special{fp}%
\special{pa -672 689}\special{pa -653 689}\special{fp}\special{pa -633 689}\special{pa -614 689}\special{fp}%
\special{pa -594 689}\special{pa -575 689}\special{fp}\special{pa -555 689}\special{pa -536 689}\special{fp}%
\special{pa -516 689}\special{pa -497 689}\special{fp}\special{pa -478 689}\special{pa -458 689}\special{fp}%
\special{pa -439 689}\special{pa -419 689}\special{fp}\special{pa -400 689}\special{pa -380 689}\special{fp}%
\special{pa -361 689}\special{pa -341 689}\special{fp}\special{pa -322 689}\special{pa -302 689}\special{fp}%
\special{pa -283 689}\special{pa -263 689}\special{fp}\special{pa -244 689}\special{pa -224 689}\special{fp}%
\special{pa -205 689}\special{pa -185 689}\special{fp}\special{pa -166 689}\special{pa -146 689}\special{fp}%
\special{pa -127 689}\special{pa -107 689}\special{fp}\special{pa -88 689}\special{pa -68 689}\special{fp}%
\special{pa -49 689}\special{pa -29 689}\special{fp}\special{pa -10 689}\special{pa 10 689}\special{fp}%
\special{pa 29 689}\special{pa 49 689}\special{fp}\special{pa 68 689}\special{pa 88 689}\special{fp}%
\special{pa 107 689}\special{pa 127 689}\special{fp}\special{pa 146 689}\special{pa 166 689}\special{fp}%
\special{pa 185 689}\special{pa 205 689}\special{fp}\special{pa 224 689}\special{pa 244 689}\special{fp}%
\special{pa 263 689}\special{pa 283 689}\special{fp}\special{pa 302 689}\special{pa 322 689}\special{fp}%
\special{pa 341 689}\special{pa 361 689}\special{fp}\special{pa 380 689}\special{pa 400 689}\special{fp}%
\special{pa 419 689}\special{pa 439 689}\special{fp}\special{pa 458 689}\special{pa 478 689}\special{fp}%
\special{pa 497 689}\special{pa 516 689}\special{fp}\special{pa 536 689}\special{pa 555 689}\special{fp}%
\special{pa 575 689}\special{pa 594 689}\special{fp}\special{pa 614 689}\special{pa 633 689}\special{fp}%
\special{pa 653 689}\special{pa 672 689}\special{fp}\special{pa 692 689}\special{pa 711 689}\special{fp}%
\special{pa 731 689}\special{pa 750 689}\special{fp}\special{pa 770 689}\special{pa 789 689}\special{fp}%
\special{pa 809 689}\special{pa 828 689}\special{fp}\special{pa 848 689}\special{pa 867 689}\special{fp}%
\special{pa 887 689}\special{pa 906 689}\special{fp}\special{pa 926 689}\special{pa 945 689}\special{fp}%
\special{pa 965 689}\special{pa 984 689}\special{fp}%
%
\special{pa -984 492}\special{pa -965 492}\special{fp}\special{pa -945 492}\special{pa -926 492}\special{fp}%
\special{pa -906 492}\special{pa -887 492}\special{fp}\special{pa -867 492}\special{pa -848 492}\special{fp}%
\special{pa -828 492}\special{pa -809 492}\special{fp}\special{pa -789 492}\special{pa -770 492}\special{fp}%
\special{pa -750 492}\special{pa -731 492}\special{fp}\special{pa -711 492}\special{pa -692 492}\special{fp}%
\special{pa -672 492}\special{pa -653 492}\special{fp}\special{pa -633 492}\special{pa -614 492}\special{fp}%
\special{pa -594 492}\special{pa -575 492}\special{fp}\special{pa -555 492}\special{pa -536 492}\special{fp}%
\special{pa -516 492}\special{pa -497 492}\special{fp}\special{pa -478 492}\special{pa -458 492}\special{fp}%
\special{pa -439 492}\special{pa -419 492}\special{fp}\special{pa -400 492}\special{pa -380 492}\special{fp}%
\special{pa -361 492}\special{pa -341 492}\special{fp}\special{pa -322 492}\special{pa -302 492}\special{fp}%
\special{pa -283 492}\special{pa -263 492}\special{fp}\special{pa -244 492}\special{pa -224 492}\special{fp}%
\special{pa -205 492}\special{pa -185 492}\special{fp}\special{pa -166 492}\special{pa -146 492}\special{fp}%
\special{pa -127 492}\special{pa -107 492}\special{fp}\special{pa -88 492}\special{pa -68 492}\special{fp}%
\special{pa -49 492}\special{pa -29 492}\special{fp}\special{pa -10 492}\special{pa 10 492}\special{fp}%
\special{pa 29 492}\special{pa 49 492}\special{fp}\special{pa 68 492}\special{pa 88 492}\special{fp}%
\special{pa 107 492}\special{pa 127 492}\special{fp}\special{pa 146 492}\special{pa 166 492}\special{fp}%
\special{pa 185 492}\special{pa 205 492}\special{fp}\special{pa 224 492}\special{pa 244 492}\special{fp}%
\special{pa 263 492}\special{pa 283 492}\special{fp}\special{pa 302 492}\special{pa 322 492}\special{fp}%
\special{pa 341 492}\special{pa 361 492}\special{fp}\special{pa 380 492}\special{pa 400 492}\special{fp}%
\special{pa 419 492}\special{pa 439 492}\special{fp}\special{pa 458 492}\special{pa 478 492}\special{fp}%
\special{pa 497 492}\special{pa 516 492}\special{fp}\special{pa 536 492}\special{pa 555 492}\special{fp}%
\special{pa 575 492}\special{pa 594 492}\special{fp}\special{pa 614 492}\special{pa 633 492}\special{fp}%
\special{pa 653 492}\special{pa 672 492}\special{fp}\special{pa 692 492}\special{pa 711 492}\special{fp}%
\special{pa 731 492}\special{pa 750 492}\special{fp}\special{pa 770 492}\special{pa 789 492}\special{fp}%
\special{pa 809 492}\special{pa 828 492}\special{fp}\special{pa 848 492}\special{pa 867 492}\special{fp}%
\special{pa 887 492}\special{pa 906 492}\special{fp}\special{pa 926 492}\special{pa 945 492}\special{fp}%
\special{pa 965 492}\special{pa 984 492}\special{fp}%
%
\special{pa -984 295}\special{pa -965 295}\special{fp}\special{pa -945 295}\special{pa -926 295}\special{fp}%
\special{pa -906 295}\special{pa -887 295}\special{fp}\special{pa -867 295}\special{pa -848 295}\special{fp}%
\special{pa -828 295}\special{pa -809 295}\special{fp}\special{pa -789 295}\special{pa -770 295}\special{fp}%
\special{pa -750 295}\special{pa -731 295}\special{fp}\special{pa -711 295}\special{pa -692 295}\special{fp}%
\special{pa -672 295}\special{pa -653 295}\special{fp}\special{pa -633 295}\special{pa -614 295}\special{fp}%
\special{pa -594 295}\special{pa -575 295}\special{fp}\special{pa -555 295}\special{pa -536 295}\special{fp}%
\special{pa -516 295}\special{pa -497 295}\special{fp}\special{pa -478 295}\special{pa -458 295}\special{fp}%
\special{pa -439 295}\special{pa -419 295}\special{fp}\special{pa -400 295}\special{pa -380 295}\special{fp}%
\special{pa -361 295}\special{pa -341 295}\special{fp}\special{pa -322 295}\special{pa -302 295}\special{fp}%
\special{pa -283 295}\special{pa -263 295}\special{fp}\special{pa -244 295}\special{pa -224 295}\special{fp}%
\special{pa -205 295}\special{pa -185 295}\special{fp}\special{pa -166 295}\special{pa -146 295}\special{fp}%
\special{pa -127 295}\special{pa -107 295}\special{fp}\special{pa -88 295}\special{pa -68 295}\special{fp}%
\special{pa -49 295}\special{pa -29 295}\special{fp}\special{pa -10 295}\special{pa 10 295}\special{fp}%
\special{pa 29 295}\special{pa 49 295}\special{fp}\special{pa 68 295}\special{pa 88 295}\special{fp}%
\special{pa 107 295}\special{pa 127 295}\special{fp}\special{pa 146 295}\special{pa 166 295}\special{fp}%
\special{pa 185 295}\special{pa 205 295}\special{fp}\special{pa 224 295}\special{pa 244 295}\special{fp}%
\special{pa 263 295}\special{pa 283 295}\special{fp}\special{pa 302 295}\special{pa 322 295}\special{fp}%
\special{pa 341 295}\special{pa 361 295}\special{fp}\special{pa 380 295}\special{pa 400 295}\special{fp}%
\special{pa 419 295}\special{pa 439 295}\special{fp}\special{pa 458 295}\special{pa 478 295}\special{fp}%
\special{pa 497 295}\special{pa 516 295}\special{fp}\special{pa 536 295}\special{pa 555 295}\special{fp}%
\special{pa 575 295}\special{pa 594 295}\special{fp}\special{pa 614 295}\special{pa 633 295}\special{fp}%
\special{pa 653 295}\special{pa 672 295}\special{fp}\special{pa 692 295}\special{pa 711 295}\special{fp}%
\special{pa 731 295}\special{pa 750 295}\special{fp}\special{pa 770 295}\special{pa 789 295}\special{fp}%
\special{pa 809 295}\special{pa 828 295}\special{fp}\special{pa 848 295}\special{pa 867 295}\special{fp}%
\special{pa 887 295}\special{pa 906 295}\special{fp}\special{pa 926 295}\special{pa 945 295}\special{fp}%
\special{pa 965 295}\special{pa 984 295}\special{fp}%
%
\special{pa -984 98}\special{pa -965 98}\special{fp}\special{pa -945 98}\special{pa -926 98}\special{fp}%
\special{pa -906 98}\special{pa -887 98}\special{fp}\special{pa -867 98}\special{pa -848 98}\special{fp}%
\special{pa -828 98}\special{pa -809 98}\special{fp}\special{pa -789 98}\special{pa -770 98}\special{fp}%
\special{pa -750 98}\special{pa -731 98}\special{fp}\special{pa -711 98}\special{pa -692 98}\special{fp}%
\special{pa -672 98}\special{pa -653 98}\special{fp}\special{pa -633 98}\special{pa -614 98}\special{fp}%
\special{pa -594 98}\special{pa -575 98}\special{fp}\special{pa -555 98}\special{pa -536 98}\special{fp}%
\special{pa -516 98}\special{pa -497 98}\special{fp}\special{pa -478 98}\special{pa -458 98}\special{fp}%
\special{pa -439 98}\special{pa -419 98}\special{fp}\special{pa -400 98}\special{pa -380 98}\special{fp}%
\special{pa -361 98}\special{pa -341 98}\special{fp}\special{pa -322 98}\special{pa -302 98}\special{fp}%
\special{pa -283 98}\special{pa -263 98}\special{fp}\special{pa -244 98}\special{pa -224 98}\special{fp}%
\special{pa -205 98}\special{pa -185 98}\special{fp}\special{pa -166 98}\special{pa -146 98}\special{fp}%
\special{pa -127 98}\special{pa -107 98}\special{fp}\special{pa -88 98}\special{pa -68 98}\special{fp}%
\special{pa -49 98}\special{pa -29 98}\special{fp}\special{pa -10 98}\special{pa 10 98}\special{fp}%
\special{pa 29 98}\special{pa 49 98}\special{fp}\special{pa 68 98}\special{pa 88 98}\special{fp}%
\special{pa 107 98}\special{pa 127 98}\special{fp}\special{pa 146 98}\special{pa 166 98}\special{fp}%
\special{pa 185 98}\special{pa 205 98}\special{fp}\special{pa 224 98}\special{pa 244 98}\special{fp}%
\special{pa 263 98}\special{pa 283 98}\special{fp}\special{pa 302 98}\special{pa 322 98}\special{fp}%
\special{pa 341 98}\special{pa 361 98}\special{fp}\special{pa 380 98}\special{pa 400 98}\special{fp}%
\special{pa 419 98}\special{pa 439 98}\special{fp}\special{pa 458 98}\special{pa 478 98}\special{fp}%
\special{pa 497 98}\special{pa 516 98}\special{fp}\special{pa 536 98}\special{pa 555 98}\special{fp}%
\special{pa 575 98}\special{pa 594 98}\special{fp}\special{pa 614 98}\special{pa 633 98}\special{fp}%
\special{pa 653 98}\special{pa 672 98}\special{fp}\special{pa 692 98}\special{pa 711 98}\special{fp}%
\special{pa 731 98}\special{pa 750 98}\special{fp}\special{pa 770 98}\special{pa 789 98}\special{fp}%
\special{pa 809 98}\special{pa 828 98}\special{fp}\special{pa 848 98}\special{pa 867 98}\special{fp}%
\special{pa 887 98}\special{pa 906 98}\special{fp}\special{pa 926 98}\special{pa 945 98}\special{fp}%
\special{pa 965 98}\special{pa 984 98}\special{fp}%
%
\special{pa -984 -98}\special{pa -965 -98}\special{fp}\special{pa -945 -98}\special{pa -926 -98}\special{fp}%
\special{pa -906 -98}\special{pa -887 -98}\special{fp}\special{pa -867 -98}\special{pa -848 -98}\special{fp}%
\special{pa -828 -98}\special{pa -809 -98}\special{fp}\special{pa -789 -98}\special{pa -770 -98}\special{fp}%
\special{pa -750 -98}\special{pa -731 -98}\special{fp}\special{pa -711 -98}\special{pa -692 -98}\special{fp}%
\special{pa -672 -98}\special{pa -653 -98}\special{fp}\special{pa -633 -98}\special{pa -614 -98}\special{fp}%
\special{pa -594 -98}\special{pa -575 -98}\special{fp}\special{pa -555 -98}\special{pa -536 -98}\special{fp}%
\special{pa -516 -98}\special{pa -497 -98}\special{fp}\special{pa -478 -98}\special{pa -458 -98}\special{fp}%
\special{pa -439 -98}\special{pa -419 -98}\special{fp}\special{pa -400 -98}\special{pa -380 -98}\special{fp}%
\special{pa -361 -98}\special{pa -341 -98}\special{fp}\special{pa -322 -98}\special{pa -302 -98}\special{fp}%
\special{pa -283 -98}\special{pa -263 -98}\special{fp}\special{pa -244 -98}\special{pa -224 -98}\special{fp}%
\special{pa -205 -98}\special{pa -185 -98}\special{fp}\special{pa -166 -98}\special{pa -146 -98}\special{fp}%
\special{pa -127 -98}\special{pa -107 -98}\special{fp}\special{pa -88 -98}\special{pa -68 -98}\special{fp}%
\special{pa -49 -98}\special{pa -29 -98}\special{fp}\special{pa -10 -98}\special{pa 10 -98}\special{fp}%
\special{pa 29 -98}\special{pa 49 -98}\special{fp}\special{pa 68 -98}\special{pa 88 -98}\special{fp}%
\special{pa 107 -98}\special{pa 127 -98}\special{fp}\special{pa 146 -98}\special{pa 166 -98}\special{fp}%
\special{pa 185 -98}\special{pa 205 -98}\special{fp}\special{pa 224 -98}\special{pa 244 -98}\special{fp}%
\special{pa 263 -98}\special{pa 283 -98}\special{fp}\special{pa 302 -98}\special{pa 322 -98}\special{fp}%
\special{pa 341 -98}\special{pa 361 -98}\special{fp}\special{pa 380 -98}\special{pa 400 -98}\special{fp}%
\special{pa 419 -98}\special{pa 439 -98}\special{fp}\special{pa 458 -98}\special{pa 478 -98}\special{fp}%
\special{pa 497 -98}\special{pa 516 -98}\special{fp}\special{pa 536 -98}\special{pa 555 -98}\special{fp}%
\special{pa 575 -98}\special{pa 594 -98}\special{fp}\special{pa 614 -98}\special{pa 633 -98}\special{fp}%
\special{pa 653 -98}\special{pa 672 -98}\special{fp}\special{pa 692 -98}\special{pa 711 -98}\special{fp}%
\special{pa 731 -98}\special{pa 750 -98}\special{fp}\special{pa 770 -98}\special{pa 789 -98}\special{fp}%
\special{pa 809 -98}\special{pa 828 -98}\special{fp}\special{pa 848 -98}\special{pa 867 -98}\special{fp}%
\special{pa 887 -98}\special{pa 906 -98}\special{fp}\special{pa 926 -98}\special{pa 945 -98}\special{fp}%
\special{pa 965 -98}\special{pa 984 -98}\special{fp}%
%
\special{pa -984 -295}\special{pa -965 -295}\special{fp}\special{pa -945 -295}\special{pa -926 -295}\special{fp}%
\special{pa -906 -295}\special{pa -887 -295}\special{fp}\special{pa -867 -295}\special{pa -848 -295}\special{fp}%
\special{pa -828 -295}\special{pa -809 -295}\special{fp}\special{pa -789 -295}\special{pa -770 -295}\special{fp}%
\special{pa -750 -295}\special{pa -731 -295}\special{fp}\special{pa -711 -295}\special{pa -692 -295}\special{fp}%
\special{pa -672 -295}\special{pa -653 -295}\special{fp}\special{pa -633 -295}\special{pa -614 -295}\special{fp}%
\special{pa -594 -295}\special{pa -575 -295}\special{fp}\special{pa -555 -295}\special{pa -536 -295}\special{fp}%
\special{pa -516 -295}\special{pa -497 -295}\special{fp}\special{pa -478 -295}\special{pa -458 -295}\special{fp}%
\special{pa -439 -295}\special{pa -419 -295}\special{fp}\special{pa -400 -295}\special{pa -380 -295}\special{fp}%
\special{pa -361 -295}\special{pa -341 -295}\special{fp}\special{pa -322 -295}\special{pa -302 -295}\special{fp}%
\special{pa -283 -295}\special{pa -263 -295}\special{fp}\special{pa -244 -295}\special{pa -224 -295}\special{fp}%
\special{pa -205 -295}\special{pa -185 -295}\special{fp}\special{pa -166 -295}\special{pa -146 -295}\special{fp}%
\special{pa -127 -295}\special{pa -107 -295}\special{fp}\special{pa -88 -295}\special{pa -68 -295}\special{fp}%
\special{pa -49 -295}\special{pa -29 -295}\special{fp}\special{pa -10 -295}\special{pa 10 -295}\special{fp}%
\special{pa 29 -295}\special{pa 49 -295}\special{fp}\special{pa 68 -295}\special{pa 88 -295}\special{fp}%
\special{pa 107 -295}\special{pa 127 -295}\special{fp}\special{pa 146 -295}\special{pa 166 -295}\special{fp}%
\special{pa 185 -295}\special{pa 205 -295}\special{fp}\special{pa 224 -295}\special{pa 244 -295}\special{fp}%
\special{pa 263 -295}\special{pa 283 -295}\special{fp}\special{pa 302 -295}\special{pa 322 -295}\special{fp}%
\special{pa 341 -295}\special{pa 361 -295}\special{fp}\special{pa 380 -295}\special{pa 400 -295}\special{fp}%
\special{pa 419 -295}\special{pa 439 -295}\special{fp}\special{pa 458 -295}\special{pa 478 -295}\special{fp}%
\special{pa 497 -295}\special{pa 516 -295}\special{fp}\special{pa 536 -295}\special{pa 555 -295}\special{fp}%
\special{pa 575 -295}\special{pa 594 -295}\special{fp}\special{pa 614 -295}\special{pa 633 -295}\special{fp}%
\special{pa 653 -295}\special{pa 672 -295}\special{fp}\special{pa 692 -295}\special{pa 711 -295}\special{fp}%
\special{pa 731 -295}\special{pa 750 -295}\special{fp}\special{pa 770 -295}\special{pa 789 -295}\special{fp}%
\special{pa 809 -295}\special{pa 828 -295}\special{fp}\special{pa 848 -295}\special{pa 867 -295}\special{fp}%
\special{pa 887 -295}\special{pa 906 -295}\special{fp}\special{pa 926 -295}\special{pa 945 -295}\special{fp}%
\special{pa 965 -295}\special{pa 984 -295}\special{fp}%
%
\special{pa -984 -492}\special{pa -965 -492}\special{fp}\special{pa -945 -492}\special{pa -926 -492}\special{fp}%
\special{pa -906 -492}\special{pa -887 -492}\special{fp}\special{pa -867 -492}\special{pa -848 -492}\special{fp}%
\special{pa -828 -492}\special{pa -809 -492}\special{fp}\special{pa -789 -492}\special{pa -770 -492}\special{fp}%
\special{pa -750 -492}\special{pa -731 -492}\special{fp}\special{pa -711 -492}\special{pa -692 -492}\special{fp}%
\special{pa -672 -492}\special{pa -653 -492}\special{fp}\special{pa -633 -492}\special{pa -614 -492}\special{fp}%
\special{pa -594 -492}\special{pa -575 -492}\special{fp}\special{pa -555 -492}\special{pa -536 -492}\special{fp}%
\special{pa -516 -492}\special{pa -497 -492}\special{fp}\special{pa -478 -492}\special{pa -458 -492}\special{fp}%
\special{pa -439 -492}\special{pa -419 -492}\special{fp}\special{pa -400 -492}\special{pa -380 -492}\special{fp}%
\special{pa -361 -492}\special{pa -341 -492}\special{fp}\special{pa -322 -492}\special{pa -302 -492}\special{fp}%
\special{pa -283 -492}\special{pa -263 -492}\special{fp}\special{pa -244 -492}\special{pa -224 -492}\special{fp}%
\special{pa -205 -492}\special{pa -185 -492}\special{fp}\special{pa -166 -492}\special{pa -146 -492}\special{fp}%
\special{pa -127 -492}\special{pa -107 -492}\special{fp}\special{pa -88 -492}\special{pa -68 -492}\special{fp}%
\special{pa -49 -492}\special{pa -29 -492}\special{fp}\special{pa -10 -492}\special{pa 10 -492}\special{fp}%
\special{pa 29 -492}\special{pa 49 -492}\special{fp}\special{pa 68 -492}\special{pa 88 -492}\special{fp}%
\special{pa 107 -492}\special{pa 127 -492}\special{fp}\special{pa 146 -492}\special{pa 166 -492}\special{fp}%
\special{pa 185 -492}\special{pa 205 -492}\special{fp}\special{pa 224 -492}\special{pa 244 -492}\special{fp}%
\special{pa 263 -492}\special{pa 283 -492}\special{fp}\special{pa 302 -492}\special{pa 322 -492}\special{fp}%
\special{pa 341 -492}\special{pa 361 -492}\special{fp}\special{pa 380 -492}\special{pa 400 -492}\special{fp}%
\special{pa 419 -492}\special{pa 439 -492}\special{fp}\special{pa 458 -492}\special{pa 478 -492}\special{fp}%
\special{pa 497 -492}\special{pa 516 -492}\special{fp}\special{pa 536 -492}\special{pa 555 -492}\special{fp}%
\special{pa 575 -492}\special{pa 594 -492}\special{fp}\special{pa 614 -492}\special{pa 633 -492}\special{fp}%
\special{pa 653 -492}\special{pa 672 -492}\special{fp}\special{pa 692 -492}\special{pa 711 -492}\special{fp}%
\special{pa 731 -492}\special{pa 750 -492}\special{fp}\special{pa 770 -492}\special{pa 789 -492}\special{fp}%
\special{pa 809 -492}\special{pa 828 -492}\special{fp}\special{pa 848 -492}\special{pa 867 -492}\special{fp}%
\special{pa 887 -492}\special{pa 906 -492}\special{fp}\special{pa 926 -492}\special{pa 945 -492}\special{fp}%
\special{pa 965 -492}\special{pa 984 -492}\special{fp}%
%
\special{pa -984 -689}\special{pa -965 -689}\special{fp}\special{pa -945 -689}\special{pa -926 -689}\special{fp}%
\special{pa -906 -689}\special{pa -887 -689}\special{fp}\special{pa -867 -689}\special{pa -848 -689}\special{fp}%
\special{pa -828 -689}\special{pa -809 -689}\special{fp}\special{pa -789 -689}\special{pa -770 -689}\special{fp}%
\special{pa -750 -689}\special{pa -731 -689}\special{fp}\special{pa -711 -689}\special{pa -692 -689}\special{fp}%
\special{pa -672 -689}\special{pa -653 -689}\special{fp}\special{pa -633 -689}\special{pa -614 -689}\special{fp}%
\special{pa -594 -689}\special{pa -575 -689}\special{fp}\special{pa -555 -689}\special{pa -536 -689}\special{fp}%
\special{pa -516 -689}\special{pa -497 -689}\special{fp}\special{pa -478 -689}\special{pa -458 -689}\special{fp}%
\special{pa -439 -689}\special{pa -419 -689}\special{fp}\special{pa -400 -689}\special{pa -380 -689}\special{fp}%
\special{pa -361 -689}\special{pa -341 -689}\special{fp}\special{pa -322 -689}\special{pa -302 -689}\special{fp}%
\special{pa -283 -689}\special{pa -263 -689}\special{fp}\special{pa -244 -689}\special{pa -224 -689}\special{fp}%
\special{pa -205 -689}\special{pa -185 -689}\special{fp}\special{pa -166 -689}\special{pa -146 -689}\special{fp}%
\special{pa -127 -689}\special{pa -107 -689}\special{fp}\special{pa -88 -689}\special{pa -68 -689}\special{fp}%
\special{pa -49 -689}\special{pa -29 -689}\special{fp}\special{pa -10 -689}\special{pa 10 -689}\special{fp}%
\special{pa 29 -689}\special{pa 49 -689}\special{fp}\special{pa 68 -689}\special{pa 88 -689}\special{fp}%
\special{pa 107 -689}\special{pa 127 -689}\special{fp}\special{pa 146 -689}\special{pa 166 -689}\special{fp}%
\special{pa 185 -689}\special{pa 205 -689}\special{fp}\special{pa 224 -689}\special{pa 244 -689}\special{fp}%
\special{pa 263 -689}\special{pa 283 -689}\special{fp}\special{pa 302 -689}\special{pa 322 -689}\special{fp}%
\special{pa 341 -689}\special{pa 361 -689}\special{fp}\special{pa 380 -689}\special{pa 400 -689}\special{fp}%
\special{pa 419 -689}\special{pa 439 -689}\special{fp}\special{pa 458 -689}\special{pa 478 -689}\special{fp}%
\special{pa 497 -689}\special{pa 516 -689}\special{fp}\special{pa 536 -689}\special{pa 555 -689}\special{fp}%
\special{pa 575 -689}\special{pa 594 -689}\special{fp}\special{pa 614 -689}\special{pa 633 -689}\special{fp}%
\special{pa 653 -689}\special{pa 672 -689}\special{fp}\special{pa 692 -689}\special{pa 711 -689}\special{fp}%
\special{pa 731 -689}\special{pa 750 -689}\special{fp}\special{pa 770 -689}\special{pa 789 -689}\special{fp}%
\special{pa 809 -689}\special{pa 828 -689}\special{fp}\special{pa 848 -689}\special{pa 867 -689}\special{fp}%
\special{pa 887 -689}\special{pa 906 -689}\special{fp}\special{pa 926 -689}\special{pa 945 -689}\special{fp}%
\special{pa 965 -689}\special{pa 984 -689}\special{fp}%
%
\special{pa -984 -886}\special{pa -965 -886}\special{fp}\special{pa -945 -886}\special{pa -926 -886}\special{fp}%
\special{pa -906 -886}\special{pa -887 -886}\special{fp}\special{pa -867 -886}\special{pa -848 -886}\special{fp}%
\special{pa -828 -886}\special{pa -809 -886}\special{fp}\special{pa -789 -886}\special{pa -770 -886}\special{fp}%
\special{pa -750 -886}\special{pa -731 -886}\special{fp}\special{pa -711 -886}\special{pa -692 -886}\special{fp}%
\special{pa -672 -886}\special{pa -653 -886}\special{fp}\special{pa -633 -886}\special{pa -614 -886}\special{fp}%
\special{pa -594 -886}\special{pa -575 -886}\special{fp}\special{pa -555 -886}\special{pa -536 -886}\special{fp}%
\special{pa -516 -886}\special{pa -497 -886}\special{fp}\special{pa -478 -886}\special{pa -458 -886}\special{fp}%
\special{pa -439 -886}\special{pa -419 -886}\special{fp}\special{pa -400 -886}\special{pa -380 -886}\special{fp}%
\special{pa -361 -886}\special{pa -341 -886}\special{fp}\special{pa -322 -886}\special{pa -302 -886}\special{fp}%
\special{pa -283 -886}\special{pa -263 -886}\special{fp}\special{pa -244 -886}\special{pa -224 -886}\special{fp}%
\special{pa -205 -886}\special{pa -185 -886}\special{fp}\special{pa -166 -886}\special{pa -146 -886}\special{fp}%
\special{pa -127 -886}\special{pa -107 -886}\special{fp}\special{pa -88 -886}\special{pa -68 -886}\special{fp}%
\special{pa -49 -886}\special{pa -29 -886}\special{fp}\special{pa -10 -886}\special{pa 10 -886}\special{fp}%
\special{pa 29 -886}\special{pa 49 -886}\special{fp}\special{pa 68 -886}\special{pa 88 -886}\special{fp}%
\special{pa 107 -886}\special{pa 127 -886}\special{fp}\special{pa 146 -886}\special{pa 166 -886}\special{fp}%
\special{pa 185 -886}\special{pa 205 -886}\special{fp}\special{pa 224 -886}\special{pa 244 -886}\special{fp}%
\special{pa 263 -886}\special{pa 283 -886}\special{fp}\special{pa 302 -886}\special{pa 322 -886}\special{fp}%
\special{pa 341 -886}\special{pa 361 -886}\special{fp}\special{pa 380 -886}\special{pa 400 -886}\special{fp}%
\special{pa 419 -886}\special{pa 439 -886}\special{fp}\special{pa 458 -886}\special{pa 478 -886}\special{fp}%
\special{pa 497 -886}\special{pa 516 -886}\special{fp}\special{pa 536 -886}\special{pa 555 -886}\special{fp}%
\special{pa 575 -886}\special{pa 594 -886}\special{fp}\special{pa 614 -886}\special{pa 633 -886}\special{fp}%
\special{pa 653 -886}\special{pa 672 -886}\special{fp}\special{pa 692 -886}\special{pa 711 -886}\special{fp}%
\special{pa 731 -886}\special{pa 750 -886}\special{fp}\special{pa 770 -886}\special{pa 789 -886}\special{fp}%
\special{pa 809 -886}\special{pa 828 -886}\special{fp}\special{pa 848 -886}\special{pa 867 -886}\special{fp}%
\special{pa 887 -886}\special{pa 906 -886}\special{fp}\special{pa 926 -886}\special{pa 945 -886}\special{fp}%
\special{pa 965 -886}\special{pa 984 -886}\special{fp}%
%
\special{pa  -984   984}\special{pa   984   984}%
\special{fp}%
\special{pa  -984   787}\special{pa   984   787}%
\special{fp}%
\special{pa  -984   591}\special{pa   984   591}%
\special{fp}%
\special{pa  -984   394}\special{pa   984   394}%
\special{fp}%
\special{pa  -984   197}\special{pa   984   197}%
\special{fp}%
\special{pa  -984    -0}\special{pa   984    -0}%
\special{fp}%
\special{pa  -984  -197}\special{pa   984  -197}%
\special{fp}%
\special{pa  -984  -394}\special{pa   984  -394}%
\special{fp}%
\special{pa  -984  -591}\special{pa   984  -591}%
\special{fp}%
\special{pa  -984  -787}\special{pa   984  -787}%
\special{fp}%
\special{pa  -984  -984}\special{pa   984  -984}%
\special{fp}%
\special{pa -886 984}\special{pa -886 965}\special{fp}\special{pa -886 945}\special{pa -886 926}\special{fp}%
\special{pa -886 906}\special{pa -886 887}\special{fp}\special{pa -886 867}\special{pa -886 848}\special{fp}%
\special{pa -886 828}\special{pa -886 809}\special{fp}\special{pa -886 789}\special{pa -886 770}\special{fp}%
\special{pa -886 750}\special{pa -886 731}\special{fp}\special{pa -886 711}\special{pa -886 692}\special{fp}%
\special{pa -886 672}\special{pa -886 653}\special{fp}\special{pa -886 633}\special{pa -886 614}\special{fp}%
\special{pa -886 594}\special{pa -886 575}\special{fp}\special{pa -886 555}\special{pa -886 536}\special{fp}%
\special{pa -886 516}\special{pa -886 497}\special{fp}\special{pa -886 478}\special{pa -886 458}\special{fp}%
\special{pa -886 439}\special{pa -886 419}\special{fp}\special{pa -886 400}\special{pa -886 380}\special{fp}%
\special{pa -886 361}\special{pa -886 341}\special{fp}\special{pa -886 322}\special{pa -886 302}\special{fp}%
\special{pa -886 283}\special{pa -886 263}\special{fp}\special{pa -886 244}\special{pa -886 224}\special{fp}%
\special{pa -886 205}\special{pa -886 185}\special{fp}\special{pa -886 166}\special{pa -886 146}\special{fp}%
\special{pa -886 127}\special{pa -886 107}\special{fp}\special{pa -886 88}\special{pa -886 68}\special{fp}%
\special{pa -886 49}\special{pa -886 29}\special{fp}\special{pa -886 10}\special{pa -886 -10}\special{fp}%
\special{pa -886 -29}\special{pa -886 -49}\special{fp}\special{pa -886 -68}\special{pa -886 -88}\special{fp}%
\special{pa -886 -107}\special{pa -886 -127}\special{fp}\special{pa -886 -146}\special{pa -886 -166}\special{fp}%
\special{pa -886 -185}\special{pa -886 -205}\special{fp}\special{pa -886 -224}\special{pa -886 -244}\special{fp}%
\special{pa -886 -263}\special{pa -886 -283}\special{fp}\special{pa -886 -302}\special{pa -886 -322}\special{fp}%
\special{pa -886 -341}\special{pa -886 -361}\special{fp}\special{pa -886 -380}\special{pa -886 -400}\special{fp}%
\special{pa -886 -419}\special{pa -886 -439}\special{fp}\special{pa -886 -458}\special{pa -886 -478}\special{fp}%
\special{pa -886 -497}\special{pa -886 -516}\special{fp}\special{pa -886 -536}\special{pa -886 -555}\special{fp}%
\special{pa -886 -575}\special{pa -886 -594}\special{fp}\special{pa -886 -614}\special{pa -886 -633}\special{fp}%
\special{pa -886 -653}\special{pa -886 -672}\special{fp}\special{pa -886 -692}\special{pa -886 -711}\special{fp}%
\special{pa -886 -731}\special{pa -886 -750}\special{fp}\special{pa -886 -770}\special{pa -886 -789}\special{fp}%
\special{pa -886 -809}\special{pa -886 -828}\special{fp}\special{pa -886 -848}\special{pa -886 -867}\special{fp}%
\special{pa -886 -887}\special{pa -886 -906}\special{fp}\special{pa -886 -926}\special{pa -886 -945}\special{fp}%
\special{pa -886 -965}\special{pa -886 -984}\special{fp}%
%
\special{pa -689 984}\special{pa -689 965}\special{fp}\special{pa -689 945}\special{pa -689 926}\special{fp}%
\special{pa -689 906}\special{pa -689 887}\special{fp}\special{pa -689 867}\special{pa -689 848}\special{fp}%
\special{pa -689 828}\special{pa -689 809}\special{fp}\special{pa -689 789}\special{pa -689 770}\special{fp}%
\special{pa -689 750}\special{pa -689 731}\special{fp}\special{pa -689 711}\special{pa -689 692}\special{fp}%
\special{pa -689 672}\special{pa -689 653}\special{fp}\special{pa -689 633}\special{pa -689 614}\special{fp}%
\special{pa -689 594}\special{pa -689 575}\special{fp}\special{pa -689 555}\special{pa -689 536}\special{fp}%
\special{pa -689 516}\special{pa -689 497}\special{fp}\special{pa -689 478}\special{pa -689 458}\special{fp}%
\special{pa -689 439}\special{pa -689 419}\special{fp}\special{pa -689 400}\special{pa -689 380}\special{fp}%
\special{pa -689 361}\special{pa -689 341}\special{fp}\special{pa -689 322}\special{pa -689 302}\special{fp}%
\special{pa -689 283}\special{pa -689 263}\special{fp}\special{pa -689 244}\special{pa -689 224}\special{fp}%
\special{pa -689 205}\special{pa -689 185}\special{fp}\special{pa -689 166}\special{pa -689 146}\special{fp}%
\special{pa -689 127}\special{pa -689 107}\special{fp}\special{pa -689 88}\special{pa -689 68}\special{fp}%
\special{pa -689 49}\special{pa -689 29}\special{fp}\special{pa -689 10}\special{pa -689 -10}\special{fp}%
\special{pa -689 -29}\special{pa -689 -49}\special{fp}\special{pa -689 -68}\special{pa -689 -88}\special{fp}%
\special{pa -689 -107}\special{pa -689 -127}\special{fp}\special{pa -689 -146}\special{pa -689 -166}\special{fp}%
\special{pa -689 -185}\special{pa -689 -205}\special{fp}\special{pa -689 -224}\special{pa -689 -244}\special{fp}%
\special{pa -689 -263}\special{pa -689 -283}\special{fp}\special{pa -689 -302}\special{pa -689 -322}\special{fp}%
\special{pa -689 -341}\special{pa -689 -361}\special{fp}\special{pa -689 -380}\special{pa -689 -400}\special{fp}%
\special{pa -689 -419}\special{pa -689 -439}\special{fp}\special{pa -689 -458}\special{pa -689 -478}\special{fp}%
\special{pa -689 -497}\special{pa -689 -516}\special{fp}\special{pa -689 -536}\special{pa -689 -555}\special{fp}%
\special{pa -689 -575}\special{pa -689 -594}\special{fp}\special{pa -689 -614}\special{pa -689 -633}\special{fp}%
\special{pa -689 -653}\special{pa -689 -672}\special{fp}\special{pa -689 -692}\special{pa -689 -711}\special{fp}%
\special{pa -689 -731}\special{pa -689 -750}\special{fp}\special{pa -689 -770}\special{pa -689 -789}\special{fp}%
\special{pa -689 -809}\special{pa -689 -828}\special{fp}\special{pa -689 -848}\special{pa -689 -867}\special{fp}%
\special{pa -689 -887}\special{pa -689 -906}\special{fp}\special{pa -689 -926}\special{pa -689 -945}\special{fp}%
\special{pa -689 -965}\special{pa -689 -984}\special{fp}%
%
\special{pa -492 984}\special{pa -492 965}\special{fp}\special{pa -492 945}\special{pa -492 926}\special{fp}%
\special{pa -492 906}\special{pa -492 887}\special{fp}\special{pa -492 867}\special{pa -492 848}\special{fp}%
\special{pa -492 828}\special{pa -492 809}\special{fp}\special{pa -492 789}\special{pa -492 770}\special{fp}%
\special{pa -492 750}\special{pa -492 731}\special{fp}\special{pa -492 711}\special{pa -492 692}\special{fp}%
\special{pa -492 672}\special{pa -492 653}\special{fp}\special{pa -492 633}\special{pa -492 614}\special{fp}%
\special{pa -492 594}\special{pa -492 575}\special{fp}\special{pa -492 555}\special{pa -492 536}\special{fp}%
\special{pa -492 516}\special{pa -492 497}\special{fp}\special{pa -492 478}\special{pa -492 458}\special{fp}%
\special{pa -492 439}\special{pa -492 419}\special{fp}\special{pa -492 400}\special{pa -492 380}\special{fp}%
\special{pa -492 361}\special{pa -492 341}\special{fp}\special{pa -492 322}\special{pa -492 302}\special{fp}%
\special{pa -492 283}\special{pa -492 263}\special{fp}\special{pa -492 244}\special{pa -492 224}\special{fp}%
\special{pa -492 205}\special{pa -492 185}\special{fp}\special{pa -492 166}\special{pa -492 146}\special{fp}%
\special{pa -492 127}\special{pa -492 107}\special{fp}\special{pa -492 88}\special{pa -492 68}\special{fp}%
\special{pa -492 49}\special{pa -492 29}\special{fp}\special{pa -492 10}\special{pa -492 -10}\special{fp}%
\special{pa -492 -29}\special{pa -492 -49}\special{fp}\special{pa -492 -68}\special{pa -492 -88}\special{fp}%
\special{pa -492 -107}\special{pa -492 -127}\special{fp}\special{pa -492 -146}\special{pa -492 -166}\special{fp}%
\special{pa -492 -185}\special{pa -492 -205}\special{fp}\special{pa -492 -224}\special{pa -492 -244}\special{fp}%
\special{pa -492 -263}\special{pa -492 -283}\special{fp}\special{pa -492 -302}\special{pa -492 -322}\special{fp}%
\special{pa -492 -341}\special{pa -492 -361}\special{fp}\special{pa -492 -380}\special{pa -492 -400}\special{fp}%
\special{pa -492 -419}\special{pa -492 -439}\special{fp}\special{pa -492 -458}\special{pa -492 -478}\special{fp}%
\special{pa -492 -497}\special{pa -492 -516}\special{fp}\special{pa -492 -536}\special{pa -492 -555}\special{fp}%
\special{pa -492 -575}\special{pa -492 -594}\special{fp}\special{pa -492 -614}\special{pa -492 -633}\special{fp}%
\special{pa -492 -653}\special{pa -492 -672}\special{fp}\special{pa -492 -692}\special{pa -492 -711}\special{fp}%
\special{pa -492 -731}\special{pa -492 -750}\special{fp}\special{pa -492 -770}\special{pa -492 -789}\special{fp}%
\special{pa -492 -809}\special{pa -492 -828}\special{fp}\special{pa -492 -848}\special{pa -492 -867}\special{fp}%
\special{pa -492 -887}\special{pa -492 -906}\special{fp}\special{pa -492 -926}\special{pa -492 -945}\special{fp}%
\special{pa -492 -965}\special{pa -492 -984}\special{fp}%
%
\special{pa -295 984}\special{pa -295 965}\special{fp}\special{pa -295 945}\special{pa -295 926}\special{fp}%
\special{pa -295 906}\special{pa -295 887}\special{fp}\special{pa -295 867}\special{pa -295 848}\special{fp}%
\special{pa -295 828}\special{pa -295 809}\special{fp}\special{pa -295 789}\special{pa -295 770}\special{fp}%
\special{pa -295 750}\special{pa -295 731}\special{fp}\special{pa -295 711}\special{pa -295 692}\special{fp}%
\special{pa -295 672}\special{pa -295 653}\special{fp}\special{pa -295 633}\special{pa -295 614}\special{fp}%
\special{pa -295 594}\special{pa -295 575}\special{fp}\special{pa -295 555}\special{pa -295 536}\special{fp}%
\special{pa -295 516}\special{pa -295 497}\special{fp}\special{pa -295 478}\special{pa -295 458}\special{fp}%
\special{pa -295 439}\special{pa -295 419}\special{fp}\special{pa -295 400}\special{pa -295 380}\special{fp}%
\special{pa -295 361}\special{pa -295 341}\special{fp}\special{pa -295 322}\special{pa -295 302}\special{fp}%
\special{pa -295 283}\special{pa -295 263}\special{fp}\special{pa -295 244}\special{pa -295 224}\special{fp}%
\special{pa -295 205}\special{pa -295 185}\special{fp}\special{pa -295 166}\special{pa -295 146}\special{fp}%
\special{pa -295 127}\special{pa -295 107}\special{fp}\special{pa -295 88}\special{pa -295 68}\special{fp}%
\special{pa -295 49}\special{pa -295 29}\special{fp}\special{pa -295 10}\special{pa -295 -10}\special{fp}%
\special{pa -295 -29}\special{pa -295 -49}\special{fp}\special{pa -295 -68}\special{pa -295 -88}\special{fp}%
\special{pa -295 -107}\special{pa -295 -127}\special{fp}\special{pa -295 -146}\special{pa -295 -166}\special{fp}%
\special{pa -295 -185}\special{pa -295 -205}\special{fp}\special{pa -295 -224}\special{pa -295 -244}\special{fp}%
\special{pa -295 -263}\special{pa -295 -283}\special{fp}\special{pa -295 -302}\special{pa -295 -322}\special{fp}%
\special{pa -295 -341}\special{pa -295 -361}\special{fp}\special{pa -295 -380}\special{pa -295 -400}\special{fp}%
\special{pa -295 -419}\special{pa -295 -439}\special{fp}\special{pa -295 -458}\special{pa -295 -478}\special{fp}%
\special{pa -295 -497}\special{pa -295 -516}\special{fp}\special{pa -295 -536}\special{pa -295 -555}\special{fp}%
\special{pa -295 -575}\special{pa -295 -594}\special{fp}\special{pa -295 -614}\special{pa -295 -633}\special{fp}%
\special{pa -295 -653}\special{pa -295 -672}\special{fp}\special{pa -295 -692}\special{pa -295 -711}\special{fp}%
\special{pa -295 -731}\special{pa -295 -750}\special{fp}\special{pa -295 -770}\special{pa -295 -789}\special{fp}%
\special{pa -295 -809}\special{pa -295 -828}\special{fp}\special{pa -295 -848}\special{pa -295 -867}\special{fp}%
\special{pa -295 -887}\special{pa -295 -906}\special{fp}\special{pa -295 -926}\special{pa -295 -945}\special{fp}%
\special{pa -295 -965}\special{pa -295 -984}\special{fp}%
%
\special{pa -98 984}\special{pa -98 965}\special{fp}\special{pa -98 945}\special{pa -98 926}\special{fp}%
\special{pa -98 906}\special{pa -98 887}\special{fp}\special{pa -98 867}\special{pa -98 848}\special{fp}%
\special{pa -98 828}\special{pa -98 809}\special{fp}\special{pa -98 789}\special{pa -98 770}\special{fp}%
\special{pa -98 750}\special{pa -98 731}\special{fp}\special{pa -98 711}\special{pa -98 692}\special{fp}%
\special{pa -98 672}\special{pa -98 653}\special{fp}\special{pa -98 633}\special{pa -98 614}\special{fp}%
\special{pa -98 594}\special{pa -98 575}\special{fp}\special{pa -98 555}\special{pa -98 536}\special{fp}%
\special{pa -98 516}\special{pa -98 497}\special{fp}\special{pa -98 478}\special{pa -98 458}\special{fp}%
\special{pa -98 439}\special{pa -98 419}\special{fp}\special{pa -98 400}\special{pa -98 380}\special{fp}%
\special{pa -98 361}\special{pa -98 341}\special{fp}\special{pa -98 322}\special{pa -98 302}\special{fp}%
\special{pa -98 283}\special{pa -98 263}\special{fp}\special{pa -98 244}\special{pa -98 224}\special{fp}%
\special{pa -98 205}\special{pa -98 185}\special{fp}\special{pa -98 166}\special{pa -98 146}\special{fp}%
\special{pa -98 127}\special{pa -98 107}\special{fp}\special{pa -98 88}\special{pa -98 68}\special{fp}%
\special{pa -98 49}\special{pa -98 29}\special{fp}\special{pa -98 10}\special{pa -98 -10}\special{fp}%
\special{pa -98 -29}\special{pa -98 -49}\special{fp}\special{pa -98 -68}\special{pa -98 -88}\special{fp}%
\special{pa -98 -107}\special{pa -98 -127}\special{fp}\special{pa -98 -146}\special{pa -98 -166}\special{fp}%
\special{pa -98 -185}\special{pa -98 -205}\special{fp}\special{pa -98 -224}\special{pa -98 -244}\special{fp}%
\special{pa -98 -263}\special{pa -98 -283}\special{fp}\special{pa -98 -302}\special{pa -98 -322}\special{fp}%
\special{pa -98 -341}\special{pa -98 -361}\special{fp}\special{pa -98 -380}\special{pa -98 -400}\special{fp}%
\special{pa -98 -419}\special{pa -98 -439}\special{fp}\special{pa -98 -458}\special{pa -98 -478}\special{fp}%
\special{pa -98 -497}\special{pa -98 -516}\special{fp}\special{pa -98 -536}\special{pa -98 -555}\special{fp}%
\special{pa -98 -575}\special{pa -98 -594}\special{fp}\special{pa -98 -614}\special{pa -98 -633}\special{fp}%
\special{pa -98 -653}\special{pa -98 -672}\special{fp}\special{pa -98 -692}\special{pa -98 -711}\special{fp}%
\special{pa -98 -731}\special{pa -98 -750}\special{fp}\special{pa -98 -770}\special{pa -98 -789}\special{fp}%
\special{pa -98 -809}\special{pa -98 -828}\special{fp}\special{pa -98 -848}\special{pa -98 -867}\special{fp}%
\special{pa -98 -887}\special{pa -98 -906}\special{fp}\special{pa -98 -926}\special{pa -98 -945}\special{fp}%
\special{pa -98 -965}\special{pa -98 -984}\special{fp}%
%
\special{pa 98 984}\special{pa 98 965}\special{fp}\special{pa 98 945}\special{pa 98 926}\special{fp}%
\special{pa 98 906}\special{pa 98 887}\special{fp}\special{pa 98 867}\special{pa 98 848}\special{fp}%
\special{pa 98 828}\special{pa 98 809}\special{fp}\special{pa 98 789}\special{pa 98 770}\special{fp}%
\special{pa 98 750}\special{pa 98 731}\special{fp}\special{pa 98 711}\special{pa 98 692}\special{fp}%
\special{pa 98 672}\special{pa 98 653}\special{fp}\special{pa 98 633}\special{pa 98 614}\special{fp}%
\special{pa 98 594}\special{pa 98 575}\special{fp}\special{pa 98 555}\special{pa 98 536}\special{fp}%
\special{pa 98 516}\special{pa 98 497}\special{fp}\special{pa 98 478}\special{pa 98 458}\special{fp}%
\special{pa 98 439}\special{pa 98 419}\special{fp}\special{pa 98 400}\special{pa 98 380}\special{fp}%
\special{pa 98 361}\special{pa 98 341}\special{fp}\special{pa 98 322}\special{pa 98 302}\special{fp}%
\special{pa 98 283}\special{pa 98 263}\special{fp}\special{pa 98 244}\special{pa 98 224}\special{fp}%
\special{pa 98 205}\special{pa 98 185}\special{fp}\special{pa 98 166}\special{pa 98 146}\special{fp}%
\special{pa 98 127}\special{pa 98 107}\special{fp}\special{pa 98 88}\special{pa 98 68}\special{fp}%
\special{pa 98 49}\special{pa 98 29}\special{fp}\special{pa 98 10}\special{pa 98 -10}\special{fp}%
\special{pa 98 -29}\special{pa 98 -49}\special{fp}\special{pa 98 -68}\special{pa 98 -88}\special{fp}%
\special{pa 98 -107}\special{pa 98 -127}\special{fp}\special{pa 98 -146}\special{pa 98 -166}\special{fp}%
\special{pa 98 -185}\special{pa 98 -205}\special{fp}\special{pa 98 -224}\special{pa 98 -244}\special{fp}%
\special{pa 98 -263}\special{pa 98 -283}\special{fp}\special{pa 98 -302}\special{pa 98 -322}\special{fp}%
\special{pa 98 -341}\special{pa 98 -361}\special{fp}\special{pa 98 -380}\special{pa 98 -400}\special{fp}%
\special{pa 98 -419}\special{pa 98 -439}\special{fp}\special{pa 98 -458}\special{pa 98 -478}\special{fp}%
\special{pa 98 -497}\special{pa 98 -516}\special{fp}\special{pa 98 -536}\special{pa 98 -555}\special{fp}%
\special{pa 98 -575}\special{pa 98 -594}\special{fp}\special{pa 98 -614}\special{pa 98 -633}\special{fp}%
\special{pa 98 -653}\special{pa 98 -672}\special{fp}\special{pa 98 -692}\special{pa 98 -711}\special{fp}%
\special{pa 98 -731}\special{pa 98 -750}\special{fp}\special{pa 98 -770}\special{pa 98 -789}\special{fp}%
\special{pa 98 -809}\special{pa 98 -828}\special{fp}\special{pa 98 -848}\special{pa 98 -867}\special{fp}%
\special{pa 98 -887}\special{pa 98 -906}\special{fp}\special{pa 98 -926}\special{pa 98 -945}\special{fp}%
\special{pa 98 -965}\special{pa 98 -984}\special{fp}%
%
\special{pa 295 984}\special{pa 295 965}\special{fp}\special{pa 295 945}\special{pa 295 926}\special{fp}%
\special{pa 295 906}\special{pa 295 887}\special{fp}\special{pa 295 867}\special{pa 295 848}\special{fp}%
\special{pa 295 828}\special{pa 295 809}\special{fp}\special{pa 295 789}\special{pa 295 770}\special{fp}%
\special{pa 295 750}\special{pa 295 731}\special{fp}\special{pa 295 711}\special{pa 295 692}\special{fp}%
\special{pa 295 672}\special{pa 295 653}\special{fp}\special{pa 295 633}\special{pa 295 614}\special{fp}%
\special{pa 295 594}\special{pa 295 575}\special{fp}\special{pa 295 555}\special{pa 295 536}\special{fp}%
\special{pa 295 516}\special{pa 295 497}\special{fp}\special{pa 295 478}\special{pa 295 458}\special{fp}%
\special{pa 295 439}\special{pa 295 419}\special{fp}\special{pa 295 400}\special{pa 295 380}\special{fp}%
\special{pa 295 361}\special{pa 295 341}\special{fp}\special{pa 295 322}\special{pa 295 302}\special{fp}%
\special{pa 295 283}\special{pa 295 263}\special{fp}\special{pa 295 244}\special{pa 295 224}\special{fp}%
\special{pa 295 205}\special{pa 295 185}\special{fp}\special{pa 295 166}\special{pa 295 146}\special{fp}%
\special{pa 295 127}\special{pa 295 107}\special{fp}\special{pa 295 88}\special{pa 295 68}\special{fp}%
\special{pa 295 49}\special{pa 295 29}\special{fp}\special{pa 295 10}\special{pa 295 -10}\special{fp}%
\special{pa 295 -29}\special{pa 295 -49}\special{fp}\special{pa 295 -68}\special{pa 295 -88}\special{fp}%
\special{pa 295 -107}\special{pa 295 -127}\special{fp}\special{pa 295 -146}\special{pa 295 -166}\special{fp}%
\special{pa 295 -185}\special{pa 295 -205}\special{fp}\special{pa 295 -224}\special{pa 295 -244}\special{fp}%
\special{pa 295 -263}\special{pa 295 -283}\special{fp}\special{pa 295 -302}\special{pa 295 -322}\special{fp}%
\special{pa 295 -341}\special{pa 295 -361}\special{fp}\special{pa 295 -380}\special{pa 295 -400}\special{fp}%
\special{pa 295 -419}\special{pa 295 -439}\special{fp}\special{pa 295 -458}\special{pa 295 -478}\special{fp}%
\special{pa 295 -497}\special{pa 295 -516}\special{fp}\special{pa 295 -536}\special{pa 295 -555}\special{fp}%
\special{pa 295 -575}\special{pa 295 -594}\special{fp}\special{pa 295 -614}\special{pa 295 -633}\special{fp}%
\special{pa 295 -653}\special{pa 295 -672}\special{fp}\special{pa 295 -692}\special{pa 295 -711}\special{fp}%
\special{pa 295 -731}\special{pa 295 -750}\special{fp}\special{pa 295 -770}\special{pa 295 -789}\special{fp}%
\special{pa 295 -809}\special{pa 295 -828}\special{fp}\special{pa 295 -848}\special{pa 295 -867}\special{fp}%
\special{pa 295 -887}\special{pa 295 -906}\special{fp}\special{pa 295 -926}\special{pa 295 -945}\special{fp}%
\special{pa 295 -965}\special{pa 295 -984}\special{fp}%
%
\special{pa 492 984}\special{pa 492 965}\special{fp}\special{pa 492 945}\special{pa 492 926}\special{fp}%
\special{pa 492 906}\special{pa 492 887}\special{fp}\special{pa 492 867}\special{pa 492 848}\special{fp}%
\special{pa 492 828}\special{pa 492 809}\special{fp}\special{pa 492 789}\special{pa 492 770}\special{fp}%
\special{pa 492 750}\special{pa 492 731}\special{fp}\special{pa 492 711}\special{pa 492 692}\special{fp}%
\special{pa 492 672}\special{pa 492 653}\special{fp}\special{pa 492 633}\special{pa 492 614}\special{fp}%
\special{pa 492 594}\special{pa 492 575}\special{fp}\special{pa 492 555}\special{pa 492 536}\special{fp}%
\special{pa 492 516}\special{pa 492 497}\special{fp}\special{pa 492 478}\special{pa 492 458}\special{fp}%
\special{pa 492 439}\special{pa 492 419}\special{fp}\special{pa 492 400}\special{pa 492 380}\special{fp}%
\special{pa 492 361}\special{pa 492 341}\special{fp}\special{pa 492 322}\special{pa 492 302}\special{fp}%
\special{pa 492 283}\special{pa 492 263}\special{fp}\special{pa 492 244}\special{pa 492 224}\special{fp}%
\special{pa 492 205}\special{pa 492 185}\special{fp}\special{pa 492 166}\special{pa 492 146}\special{fp}%
\special{pa 492 127}\special{pa 492 107}\special{fp}\special{pa 492 88}\special{pa 492 68}\special{fp}%
\special{pa 492 49}\special{pa 492 29}\special{fp}\special{pa 492 10}\special{pa 492 -10}\special{fp}%
\special{pa 492 -29}\special{pa 492 -49}\special{fp}\special{pa 492 -68}\special{pa 492 -88}\special{fp}%
\special{pa 492 -107}\special{pa 492 -127}\special{fp}\special{pa 492 -146}\special{pa 492 -166}\special{fp}%
\special{pa 492 -185}\special{pa 492 -205}\special{fp}\special{pa 492 -224}\special{pa 492 -244}\special{fp}%
\special{pa 492 -263}\special{pa 492 -283}\special{fp}\special{pa 492 -302}\special{pa 492 -322}\special{fp}%
\special{pa 492 -341}\special{pa 492 -361}\special{fp}\special{pa 492 -380}\special{pa 492 -400}\special{fp}%
\special{pa 492 -419}\special{pa 492 -439}\special{fp}\special{pa 492 -458}\special{pa 492 -478}\special{fp}%
\special{pa 492 -497}\special{pa 492 -516}\special{fp}\special{pa 492 -536}\special{pa 492 -555}\special{fp}%
\special{pa 492 -575}\special{pa 492 -594}\special{fp}\special{pa 492 -614}\special{pa 492 -633}\special{fp}%
\special{pa 492 -653}\special{pa 492 -672}\special{fp}\special{pa 492 -692}\special{pa 492 -711}\special{fp}%
\special{pa 492 -731}\special{pa 492 -750}\special{fp}\special{pa 492 -770}\special{pa 492 -789}\special{fp}%
\special{pa 492 -809}\special{pa 492 -828}\special{fp}\special{pa 492 -848}\special{pa 492 -867}\special{fp}%
\special{pa 492 -887}\special{pa 492 -906}\special{fp}\special{pa 492 -926}\special{pa 492 -945}\special{fp}%
\special{pa 492 -965}\special{pa 492 -984}\special{fp}%
%
\special{pa 689 984}\special{pa 689 965}\special{fp}\special{pa 689 945}\special{pa 689 926}\special{fp}%
\special{pa 689 906}\special{pa 689 887}\special{fp}\special{pa 689 867}\special{pa 689 848}\special{fp}%
\special{pa 689 828}\special{pa 689 809}\special{fp}\special{pa 689 789}\special{pa 689 770}\special{fp}%
\special{pa 689 750}\special{pa 689 731}\special{fp}\special{pa 689 711}\special{pa 689 692}\special{fp}%
\special{pa 689 672}\special{pa 689 653}\special{fp}\special{pa 689 633}\special{pa 689 614}\special{fp}%
\special{pa 689 594}\special{pa 689 575}\special{fp}\special{pa 689 555}\special{pa 689 536}\special{fp}%
\special{pa 689 516}\special{pa 689 497}\special{fp}\special{pa 689 478}\special{pa 689 458}\special{fp}%
\special{pa 689 439}\special{pa 689 419}\special{fp}\special{pa 689 400}\special{pa 689 380}\special{fp}%
\special{pa 689 361}\special{pa 689 341}\special{fp}\special{pa 689 322}\special{pa 689 302}\special{fp}%
\special{pa 689 283}\special{pa 689 263}\special{fp}\special{pa 689 244}\special{pa 689 224}\special{fp}%
\special{pa 689 205}\special{pa 689 185}\special{fp}\special{pa 689 166}\special{pa 689 146}\special{fp}%
\special{pa 689 127}\special{pa 689 107}\special{fp}\special{pa 689 88}\special{pa 689 68}\special{fp}%
\special{pa 689 49}\special{pa 689 29}\special{fp}\special{pa 689 10}\special{pa 689 -10}\special{fp}%
\special{pa 689 -29}\special{pa 689 -49}\special{fp}\special{pa 689 -68}\special{pa 689 -88}\special{fp}%
\special{pa 689 -107}\special{pa 689 -127}\special{fp}\special{pa 689 -146}\special{pa 689 -166}\special{fp}%
\special{pa 689 -185}\special{pa 689 -205}\special{fp}\special{pa 689 -224}\special{pa 689 -244}\special{fp}%
\special{pa 689 -263}\special{pa 689 -283}\special{fp}\special{pa 689 -302}\special{pa 689 -322}\special{fp}%
\special{pa 689 -341}\special{pa 689 -361}\special{fp}\special{pa 689 -380}\special{pa 689 -400}\special{fp}%
\special{pa 689 -419}\special{pa 689 -439}\special{fp}\special{pa 689 -458}\special{pa 689 -478}\special{fp}%
\special{pa 689 -497}\special{pa 689 -516}\special{fp}\special{pa 689 -536}\special{pa 689 -555}\special{fp}%
\special{pa 689 -575}\special{pa 689 -594}\special{fp}\special{pa 689 -614}\special{pa 689 -633}\special{fp}%
\special{pa 689 -653}\special{pa 689 -672}\special{fp}\special{pa 689 -692}\special{pa 689 -711}\special{fp}%
\special{pa 689 -731}\special{pa 689 -750}\special{fp}\special{pa 689 -770}\special{pa 689 -789}\special{fp}%
\special{pa 689 -809}\special{pa 689 -828}\special{fp}\special{pa 689 -848}\special{pa 689 -867}\special{fp}%
\special{pa 689 -887}\special{pa 689 -906}\special{fp}\special{pa 689 -926}\special{pa 689 -945}\special{fp}%
\special{pa 689 -965}\special{pa 689 -984}\special{fp}%
%
\special{pa 886 984}\special{pa 886 965}\special{fp}\special{pa 886 945}\special{pa 886 926}\special{fp}%
\special{pa 886 906}\special{pa 886 887}\special{fp}\special{pa 886 867}\special{pa 886 848}\special{fp}%
\special{pa 886 828}\special{pa 886 809}\special{fp}\special{pa 886 789}\special{pa 886 770}\special{fp}%
\special{pa 886 750}\special{pa 886 731}\special{fp}\special{pa 886 711}\special{pa 886 692}\special{fp}%
\special{pa 886 672}\special{pa 886 653}\special{fp}\special{pa 886 633}\special{pa 886 614}\special{fp}%
\special{pa 886 594}\special{pa 886 575}\special{fp}\special{pa 886 555}\special{pa 886 536}\special{fp}%
\special{pa 886 516}\special{pa 886 497}\special{fp}\special{pa 886 478}\special{pa 886 458}\special{fp}%
\special{pa 886 439}\special{pa 886 419}\special{fp}\special{pa 886 400}\special{pa 886 380}\special{fp}%
\special{pa 886 361}\special{pa 886 341}\special{fp}\special{pa 886 322}\special{pa 886 302}\special{fp}%
\special{pa 886 283}\special{pa 886 263}\special{fp}\special{pa 886 244}\special{pa 886 224}\special{fp}%
\special{pa 886 205}\special{pa 886 185}\special{fp}\special{pa 886 166}\special{pa 886 146}\special{fp}%
\special{pa 886 127}\special{pa 886 107}\special{fp}\special{pa 886 88}\special{pa 886 68}\special{fp}%
\special{pa 886 49}\special{pa 886 29}\special{fp}\special{pa 886 10}\special{pa 886 -10}\special{fp}%
\special{pa 886 -29}\special{pa 886 -49}\special{fp}\special{pa 886 -68}\special{pa 886 -88}\special{fp}%
\special{pa 886 -107}\special{pa 886 -127}\special{fp}\special{pa 886 -146}\special{pa 886 -166}\special{fp}%
\special{pa 886 -185}\special{pa 886 -205}\special{fp}\special{pa 886 -224}\special{pa 886 -244}\special{fp}%
\special{pa 886 -263}\special{pa 886 -283}\special{fp}\special{pa 886 -302}\special{pa 886 -322}\special{fp}%
\special{pa 886 -341}\special{pa 886 -361}\special{fp}\special{pa 886 -380}\special{pa 886 -400}\special{fp}%
\special{pa 886 -419}\special{pa 886 -439}\special{fp}\special{pa 886 -458}\special{pa 886 -478}\special{fp}%
\special{pa 886 -497}\special{pa 886 -516}\special{fp}\special{pa 886 -536}\special{pa 886 -555}\special{fp}%
\special{pa 886 -575}\special{pa 886 -594}\special{fp}\special{pa 886 -614}\special{pa 886 -633}\special{fp}%
\special{pa 886 -653}\special{pa 886 -672}\special{fp}\special{pa 886 -692}\special{pa 886 -711}\special{fp}%
\special{pa 886 -731}\special{pa 886 -750}\special{fp}\special{pa 886 -770}\special{pa 886 -789}\special{fp}%
\special{pa 886 -809}\special{pa 886 -828}\special{fp}\special{pa 886 -848}\special{pa 886 -867}\special{fp}%
\special{pa 886 -887}\special{pa 886 -906}\special{fp}\special{pa 886 -926}\special{pa 886 -945}\special{fp}%
\special{pa 886 -965}\special{pa 886 -984}\special{fp}%
%
\special{pa  -984   984}\special{pa  -984  -984}%
\special{fp}%
\special{pa  -787   984}\special{pa  -787  -984}%
\special{fp}%
\special{pa  -591   984}\special{pa  -591  -984}%
\special{fp}%
\special{pa  -394   984}\special{pa  -394  -984}%
\special{fp}%
\special{pa  -197   984}\special{pa  -197  -984}%
\special{fp}%
\special{pa     0   984}\special{pa     0  -984}%
\special{fp}%
\special{pa   197   984}\special{pa   197  -984}%
\special{fp}%
\special{pa   394   984}\special{pa   394  -984}%
\special{fp}%
\special{pa   591   984}\special{pa   591  -984}%
\special{fp}%
\special{pa   787   984}\special{pa   787  -984}%
\special{fp}%
\special{pa   984   984}\special{pa   984  -984}%
\special{fp}%
\special{pn 8}%
\scriptsize%
\special{pa  -197   -20}\special{pa  -197    20}%
\special{fp}%
\settowidth{\Width}{$-1$}\setlength{\Width}{-0.5\Width}%
\settoheight{\Height}{$-1$}\settodepth{\Depth}{$-1$}\setlength{\Height}{-\Height}%
\put(-1.0000000,-0.2000000){\hspace*{\Width}\raisebox{\Height}{$-1$}}%
%
%
\special{pa   197   -20}\special{pa   197    20}%
\special{fp}%
\settowidth{\Width}{$1$}\setlength{\Width}{-0.5\Width}%
\settoheight{\Height}{$1$}\settodepth{\Depth}{$1$}\setlength{\Height}{-\Height}%
\put(1.0000000,-0.2000000){\hspace*{\Width}\raisebox{\Height}{$1$}}%
%
%
\special{pa    20   197}\special{pa   -20   197}%
\special{fp}%
\settowidth{\Width}{$-1$}\setlength{\Width}{-1\Width}%
\settoheight{\Height}{$-1$}\settodepth{\Depth}{$-1$}\setlength{\Height}{-0.5\Height}\setlength{\Depth}{0.5\Depth}\addtolength{\Height}{\Depth}%
\put(-0.2000000,-1.0000000){\hspace*{\Width}\raisebox{\Height}{$-1$}}%
%
%
\special{pa    20  -197}\special{pa   -20  -197}%
\special{fp}%
\settowidth{\Width}{$1$}\setlength{\Width}{-1\Width}%
\settoheight{\Height}{$1$}\settodepth{\Depth}{$1$}\setlength{\Height}{-0.5\Height}\setlength{\Depth}{0.5\Depth}\addtolength{\Height}{\Depth}%
\put(-0.2000000,1.0000000){\hspace*{\Width}\raisebox{\Height}{$1$}}%
%
%
\special{pn 8}%
\special{pa  -984    -0}\special{pa   965    -0}%
\special{fp}%
\special{pn 8}%
\special{pa 909 24}\special{pa 984 0}\special{pa 909 -24}\special{pa 909 0}\special{pa 909 24}%
\special{sh 1}\special{ip}%
\special{pn 1}%
\special{pa   909    24}\special{pa   984    -0}\special{pa   909   -24}\special{pa   909    -0}%
\special{pa   909    24}\special{pa   984    -0}%
\special{fp}%
\special{pn 8}%
\special{pn 8}%
\special{pa     0   984}\special{pa     0  -965}%
\special{fp}%
\special{pn 8}%
\special{pa 24 -909}\special{pa 0 -984}\special{pa -24 -909}\special{pa 0 -909}\special{pa 24 -909}%
\special{sh 1}\special{ip}%
\special{pn 1}%
\special{pa    24  -909}\special{pa     0  -984}\special{pa   -24  -909}\special{pa     0  -909}%
\special{pa    24  -909}\special{pa     0  -984}%
\special{fp}%
\special{pn 8}%
\settowidth{\Width}{$x\mbox{実軸}$}\setlength{\Width}{0\Width}%
\settoheight{\Height}{$x\mbox{実軸}$}\settodepth{\Depth}{$x\mbox{実軸}$}\setlength{\Height}{-0.5\Height}\setlength{\Depth}{0.5\Depth}\addtolength{\Height}{\Depth}%
\put(5.1000000,0.0000000){\hspace*{\Width}\raisebox{\Height}{$x\mbox{実軸}$}}%
%
\settowidth{\Width}{$y\mbox{虚軸}$}\setlength{\Width}{-0.5\Width}%
\settoheight{\Height}{$y\mbox{虚軸}$}\settodepth{\Depth}{$y\mbox{虚軸}$}\setlength{\Height}{\Depth}%
\put(0.0000000,5.1000000){\hspace*{\Width}\raisebox{\Height}{$y\mbox{虚軸}$}}%
%
\settowidth{\Width}{ }\setlength{\Width}{-1\Width}%
\settoheight{\Height}{ }\settodepth{\Depth}{ }\setlength{\Height}{-\Height}%
\put(-0.1000000,-0.1000000){\hspace*{\Width}\raisebox{\Height}{ }}%
%
\end{picture}}%}
\end{layer}

{\color{red}

\begin{layer}{120}{0}
\putnotec{110}{25}{\small$\bullet$}
\end{layer}

}
\begin{itemize}
\item
$z=a+b\,i$を平面上の点$(a,\ b)$で表す\seteda{50}\\
\eda{$2+3i \leftrightarrow\ \mbox{点}(2,3)$}\\
\eda{$-2+i \leftrightarrow\ \mbox{点}$}\\
\eda{$3 \leftrightarrow\ \mbox{点}$}\\
\eda{$-3 \leftrightarrow\ \mbox{点}$}\\
\eda{$4i \leftrightarrow\ \mbox{点}$}
\item
[課題]\monban\ [2]-[5]の点を求めよ
\end{itemize}

\newslide{複素数の和$z+w$と図形}

\vspace*{18mm}

\slidepage

\begin{layer}{120}{0}
%%[-,3]::putnote::se{72}{27}::complexcalcsum,0.8
%%[4]::putnote::se{72}{27}::complexcalcsum2,0.8
\end{layer}

\begin{itemize}
\item
$z=a+bi,\ w=c+di$($a,b,c,d$は実数)
\end{itemize}
%%%%%%%%%%%%%

%%%%%%%%%%%%%%%%%%%%


\sameslide

\vspace*{18mm}

\slidepage

\begin{layer}{120}{0}
\end{layer}

\begin{itemize}
\item
$z=a+bi,\ w=c+di$($a,b,c,d$は実数)
\item
[]$z+w=a+bi+c+di=(a+c)+(b+d)i$
\end{itemize}

\sameslide

\vspace*{18mm}

\slidepage

\begin{layer}{120}{0}
\end{layer}

\begin{itemize}
\item
$z=a+bi,\ w=c+di$($a,b,c,d$は実数)
\item
[]$z+w=a+bi+c+di=(a+c)+(b+d)i$
\item
\href{https://s-takato.github.io/polytech22/offlineapp/fukusowa_ttttjsoffL.html}{複素数の和}を動かそう
\begin{enumerate}[(1)]
\item
IDに学生番号を入れて「確認」「出題」を押す.
\item
赤い点を$z+w(=\alpha+\beta)$の位置に動かす.
\item
点が決まったら,「採点」を押す.
\item
以上を4回ほど繰り返す.
\end{enumerate}
\end{itemize}

\sameslide

\vspace*{18mm}

\slidepage

\begin{layer}{120}{0}
\end{layer}

\begin{itemize}
\item
$z=a+bi,\ w=c+di$($a,b,c,d$は実数)
\item
[]$z+w=a+bi+c+di=(a+c)+(b+d)i$
\item
\href{https://s-takato.github.io/polytech22/offlineapp/fukusowa_ttttjsoffL.html}{複素数の和}を動かそう
\begin{enumerate}[(1)]
\item
IDに学生番号を入れて「確認」「出題」を押す.
\item
赤い点を$z+w(=\alpha+\beta)$の位置に動かす.
\item
点が決まったら,「採点」を押す.
\item
以上を4回ほど繰り返す.
\end{enumerate}
\item
[課題]\monban $\mathrm{O},z,z+w,w$でできる四辺形は何か.
\end{itemize}

\newslide{絶対値と偏角}

\vspace*{18mm}

\slidepage

\begin{layer}{120}{0}
\putnotese{70}{17}{%%% /Users/takatoosetsuo/Dropbox/2018polytec/lecture/0611/presen/fig/absangle.tex 
%%% Generator=presen0611.cdy 
{\unitlength=1cm%
\begin{picture}%
(5,5)(-1,-1)%
\special{pn 8}%
%
\Large\bf\boldmath%
\small%
\special{pn 12}%
\special{pa     0    -0}\special{pa  1270 -1110}%
\special{fp}%
\special{pn 8}%
\special{pa 1270 0}\special{pa 1270 -39}\special{fp}\special{pa 1270 -78}\special{pa 1270 -117}\special{fp}%
\special{pa 1270 -156}\special{pa 1270 -195}\special{fp}\special{pa 1270 -234}\special{pa 1270 -273}\special{fp}%
\special{pa 1270 -312}\special{pa 1270 -351}\special{fp}\special{pa 1270 -390}\special{pa 1270 -429}\special{fp}%
\special{pa 1270 -468}\special{pa 1270 -507}\special{fp}\special{pa 1270 -546}\special{pa 1270 -585}\special{fp}%
\special{pa 1270 -624}\special{pa 1270 -663}\special{fp}\special{pa 1270 -702}\special{pa 1270 -741}\special{fp}%
\special{pa 1270 -780}\special{pa 1270 -819}\special{fp}\special{pa 1270 -858}\special{pa 1270 -897}\special{fp}%
\special{pa 1270 -936}\special{pa 1270 -975}\special{fp}\special{pa 1270 -1014}\special{pa 1270 -1053}\special{fp}%
\special{pa 1270 -1092}\special{pa 1270 -1110}\special{pa 1248 -1110}\special{fp}%
\special{pa 1209 -1110}\special{pa 1170 -1110}\special{fp}\special{pa 1131 -1110}\special{pa 1092 -1110}\special{fp}%
\special{pa 1053 -1110}\special{pa 1014 -1110}\special{fp}\special{pa 975 -1110}\special{pa 936 -1110}\special{fp}%
\special{pa 897 -1110}\special{pa 858 -1110}\special{fp}\special{pa 819 -1110}\special{pa 780 -1110}\special{fp}%
\special{pa 741 -1110}\special{pa 702 -1110}\special{fp}\special{pa 663 -1110}\special{pa 624 -1110}\special{fp}%
\special{pa 585 -1110}\special{pa 546 -1110}\special{fp}\special{pa 507 -1110}\special{pa 468 -1110}\special{fp}%
\special{pa 429 -1110}\special{pa 390 -1110}\special{fp}\special{pa 351 -1110}\special{pa 312 -1110}\special{fp}%
\special{pa 273 -1110}\special{pa 234 -1110}\special{fp}\special{pa 195 -1110}\special{pa 156 -1110}\special{fp}%
\special{pa 117 -1110}\special{pa 78 -1110}\special{fp}\special{pa 39 -1110}\special{pa 0 -1110}\special{fp}%
%
%
\settowidth{\Width}{$\theta$}\setlength{\Width}{-0.5\Width}%
\settoheight{\Height}{$\theta$}\settodepth{\Depth}{$\theta$}\setlength{\Height}{-0.5\Height}\setlength{\Depth}{0.5\Depth}\addtolength{\Height}{\Depth}%
\put(0.6100000,0.2300000){\hspace*{\Width}\raisebox{\Height}{$\theta$}}%
%
\special{pa   197    -0}\special{pa   195   -25}\special{pa   191   -49}\special{pa   183   -72}%
\special{pa   173   -95}\special{pa   159  -116}\special{pa   148  -129}%
\special{fp}%
\settowidth{\Width}{$|z|$}\setlength{\Width}{-0.5\Width}%
\settoheight{\Height}{$|z|$}\settodepth{\Depth}{$|z|$}\setlength{\Height}{-0.5\Height}\setlength{\Depth}{0.5\Depth}\addtolength{\Height}{\Depth}%
\put(1.3300000,1.7300000){\hspace*{\Width}\raisebox{\Height}{$|z|$}}%
%
\special{pa  1270 -1110}\special{pa  1255 -1105}\special{pa  1241 -1100}\special{pa  1226 -1095}%
\special{pa  1212 -1090}\special{pa  1197 -1084}\special{pa  1183 -1079}\special{pa  1169 -1073}%
\special{pa  1154 -1068}\special{pa  1140 -1062}\special{pa  1126 -1056}\special{pa  1112 -1050}%
\special{pa  1098 -1044}\special{pa  1084 -1038}\special{pa  1070 -1032}\special{pa  1056 -1025}%
\special{pa  1042 -1019}\special{pa  1028 -1012}\special{pa  1014 -1006}\special{pa  1000  -999}%
\special{pa   986  -992}\special{pa   973  -985}\special{pa   959  -978}\special{pa   946  -971}%
\special{pa   932  -964}\special{pa   919  -957}\special{pa   905  -949}\special{pa   892  -942}%
\special{pa   878  -934}\special{pa   865  -927}\special{pa   852  -919}\special{pa   839  -911}%
\special{pa   825  -903}\special{pa   812  -895}\special{pa   799  -887}\special{pa   786  -879}%
\special{pa   774  -871}\special{pa   761  -862}\special{pa   748  -854}\special{pa   735  -845}%
\special{pa   723  -836}\special{pa   710  -828}\special{pa   697  -819}\special{pa   685  -810}%
\special{pa   672  -801}\special{pa   660  -792}\special{pa   648  -783}\special{pa   636  -773}%
\special{pa   623  -764}\special{pa   611  -755}\special{pa   599  -745}%
\special{fp}%
\special{pa   451  -615}\special{pa   440  -605}\special{pa   429  -594}\special{pa   418  -583}%
\special{pa   407  -572}\special{pa   397  -562}\special{pa   386  -550}\special{pa   375  -539}%
\special{pa   365  -528}\special{pa   355  -517}\special{pa   344  -506}\special{pa   334  -494}%
\special{pa   324  -483}\special{pa   314  -471}\special{pa   304  -460}\special{pa   294  -448}%
\special{pa   284  -436}\special{pa   274  -424}\special{pa   264  -413}\special{pa   255  -401}%
\special{pa   245  -389}\special{pa   236  -377}\special{pa   226  -364}\special{pa   217  -352}%
\special{pa   208  -340}\special{pa   199  -328}\special{pa   190  -315}\special{pa   181  -303}%
\special{pa   172  -290}\special{pa   163  -278}\special{pa   154  -265}\special{pa   146  -252}%
\special{pa   137  -239}\special{pa   129  -227}\special{pa   121  -214}\special{pa   112  -201}%
\special{pa   104  -188}\special{pa    96  -175}\special{pa    88  -162}\special{pa    80  -148}%
\special{pa    73  -135}\special{pa    65  -122}\special{pa    57  -109}\special{pa    50   -95}%
\special{pa    42   -82}\special{pa    35   -68}\special{pa    28   -55}\special{pa    21   -41}%
\special{pa    14   -27}\special{pa     7   -14}\special{pa    -0     0}%
\special{fp}%
\settowidth{\Width}{$z=a+bi$}\setlength{\Width}{0\Width}%
\settoheight{\Height}{$z=a+bi$}\settodepth{\Depth}{$z=a+bi$}\setlength{\Height}{\Depth}%
\put(3.2700000,2.8700000){\hspace*{\Width}\raisebox{\Height}{$z=a+bi$}}%
%
\special{pa  1270   -20}\special{pa  1270    20}%
\special{fp}%
\settowidth{\Width}{$a$}\setlength{\Width}{-0.5\Width}%
\settoheight{\Height}{$a$}\settodepth{\Depth}{$a$}\setlength{\Height}{-\Height}%
\put(3.2248800,-0.1000000){\hspace*{\Width}\raisebox{\Height}{$a$}}%
%
%
\special{pa    20 -1110}\special{pa   -20 -1110}%
\special{fp}%
\settowidth{\Width}{$b$}\setlength{\Width}{-1\Width}%
\settoheight{\Height}{$b$}\settodepth{\Depth}{$b$}\setlength{\Height}{-0.5\Height}\setlength{\Depth}{0.5\Depth}\addtolength{\Height}{\Depth}%
\put(-0.1000000,2.8191000){\hspace*{\Width}\raisebox{\Height}{$b$}}%
%
%
\special{pa  -394    -0}\special{pa  1575    -0}%
\special{fp}%
\special{pa     0   394}\special{pa     0 -1575}%
\special{fp}%
\settowidth{\Width}{$x$}\setlength{\Width}{0\Width}%
\settoheight{\Height}{$x$}\settodepth{\Depth}{$x$}\setlength{\Height}{-0.5\Height}\setlength{\Depth}{0.5\Depth}\addtolength{\Height}{\Depth}%
\put(4.0500000,0.0000000){\hspace*{\Width}\raisebox{\Height}{$x$}}%
%
\settowidth{\Width}{$y$}\setlength{\Width}{-0.5\Width}%
\settoheight{\Height}{$y$}\settodepth{\Depth}{$y$}\setlength{\Height}{\Depth}%
\put(0.0000000,4.0500000){\hspace*{\Width}\raisebox{\Height}{$y$}}%
%
\settowidth{\Width}{O}\setlength{\Width}{-1\Width}%
\settoheight{\Height}{O}\settodepth{\Depth}{O}\setlength{\Height}{-\Height}%
\put(-0.0500000,-0.0500000){\hspace*{\Width}\raisebox{\Height}{O}}%
%
\end{picture}}%}
\end{layer}

\begin{itemize}
\item
$z=a+b\,i$を平面上の点$(a,\ b)$で表したとき\vspace{-2mm}
\item
[]{\color{red}絶対値}$|z|$\\
 原点Oと$z$の距離\\
  $|z|=\sqrt{a^2+b^2}$\vspace{-2mm}
\end{itemize}
%%%%%%%%%%%%%

%%%%%%%%%%%%%%%%%%%%


\sameslide

\vspace*{18mm}

\slidepage

\begin{layer}{120}{0}
\putnotese{70}{17}{%%% /Users/takatoosetsuo/Dropbox/2018polytec/lecture/0611/presen/fig/absangle.tex 
%%% Generator=presen0611.cdy 
{\unitlength=1cm%
\begin{picture}%
(5,5)(-1,-1)%
\special{pn 8}%
%
\Large\bf\boldmath%
\small%
\special{pn 12}%
\special{pa     0    -0}\special{pa  1270 -1110}%
\special{fp}%
\special{pn 8}%
\special{pa 1270 0}\special{pa 1270 -39}\special{fp}\special{pa 1270 -78}\special{pa 1270 -117}\special{fp}%
\special{pa 1270 -156}\special{pa 1270 -195}\special{fp}\special{pa 1270 -234}\special{pa 1270 -273}\special{fp}%
\special{pa 1270 -312}\special{pa 1270 -351}\special{fp}\special{pa 1270 -390}\special{pa 1270 -429}\special{fp}%
\special{pa 1270 -468}\special{pa 1270 -507}\special{fp}\special{pa 1270 -546}\special{pa 1270 -585}\special{fp}%
\special{pa 1270 -624}\special{pa 1270 -663}\special{fp}\special{pa 1270 -702}\special{pa 1270 -741}\special{fp}%
\special{pa 1270 -780}\special{pa 1270 -819}\special{fp}\special{pa 1270 -858}\special{pa 1270 -897}\special{fp}%
\special{pa 1270 -936}\special{pa 1270 -975}\special{fp}\special{pa 1270 -1014}\special{pa 1270 -1053}\special{fp}%
\special{pa 1270 -1092}\special{pa 1270 -1110}\special{pa 1248 -1110}\special{fp}%
\special{pa 1209 -1110}\special{pa 1170 -1110}\special{fp}\special{pa 1131 -1110}\special{pa 1092 -1110}\special{fp}%
\special{pa 1053 -1110}\special{pa 1014 -1110}\special{fp}\special{pa 975 -1110}\special{pa 936 -1110}\special{fp}%
\special{pa 897 -1110}\special{pa 858 -1110}\special{fp}\special{pa 819 -1110}\special{pa 780 -1110}\special{fp}%
\special{pa 741 -1110}\special{pa 702 -1110}\special{fp}\special{pa 663 -1110}\special{pa 624 -1110}\special{fp}%
\special{pa 585 -1110}\special{pa 546 -1110}\special{fp}\special{pa 507 -1110}\special{pa 468 -1110}\special{fp}%
\special{pa 429 -1110}\special{pa 390 -1110}\special{fp}\special{pa 351 -1110}\special{pa 312 -1110}\special{fp}%
\special{pa 273 -1110}\special{pa 234 -1110}\special{fp}\special{pa 195 -1110}\special{pa 156 -1110}\special{fp}%
\special{pa 117 -1110}\special{pa 78 -1110}\special{fp}\special{pa 39 -1110}\special{pa 0 -1110}\special{fp}%
%
%
\settowidth{\Width}{$\theta$}\setlength{\Width}{-0.5\Width}%
\settoheight{\Height}{$\theta$}\settodepth{\Depth}{$\theta$}\setlength{\Height}{-0.5\Height}\setlength{\Depth}{0.5\Depth}\addtolength{\Height}{\Depth}%
\put(0.6100000,0.2300000){\hspace*{\Width}\raisebox{\Height}{$\theta$}}%
%
\special{pa   197    -0}\special{pa   195   -25}\special{pa   191   -49}\special{pa   183   -72}%
\special{pa   173   -95}\special{pa   159  -116}\special{pa   148  -129}%
\special{fp}%
\settowidth{\Width}{$|z|$}\setlength{\Width}{-0.5\Width}%
\settoheight{\Height}{$|z|$}\settodepth{\Depth}{$|z|$}\setlength{\Height}{-0.5\Height}\setlength{\Depth}{0.5\Depth}\addtolength{\Height}{\Depth}%
\put(1.3300000,1.7300000){\hspace*{\Width}\raisebox{\Height}{$|z|$}}%
%
\special{pa  1270 -1110}\special{pa  1255 -1105}\special{pa  1241 -1100}\special{pa  1226 -1095}%
\special{pa  1212 -1090}\special{pa  1197 -1084}\special{pa  1183 -1079}\special{pa  1169 -1073}%
\special{pa  1154 -1068}\special{pa  1140 -1062}\special{pa  1126 -1056}\special{pa  1112 -1050}%
\special{pa  1098 -1044}\special{pa  1084 -1038}\special{pa  1070 -1032}\special{pa  1056 -1025}%
\special{pa  1042 -1019}\special{pa  1028 -1012}\special{pa  1014 -1006}\special{pa  1000  -999}%
\special{pa   986  -992}\special{pa   973  -985}\special{pa   959  -978}\special{pa   946  -971}%
\special{pa   932  -964}\special{pa   919  -957}\special{pa   905  -949}\special{pa   892  -942}%
\special{pa   878  -934}\special{pa   865  -927}\special{pa   852  -919}\special{pa   839  -911}%
\special{pa   825  -903}\special{pa   812  -895}\special{pa   799  -887}\special{pa   786  -879}%
\special{pa   774  -871}\special{pa   761  -862}\special{pa   748  -854}\special{pa   735  -845}%
\special{pa   723  -836}\special{pa   710  -828}\special{pa   697  -819}\special{pa   685  -810}%
\special{pa   672  -801}\special{pa   660  -792}\special{pa   648  -783}\special{pa   636  -773}%
\special{pa   623  -764}\special{pa   611  -755}\special{pa   599  -745}%
\special{fp}%
\special{pa   451  -615}\special{pa   440  -605}\special{pa   429  -594}\special{pa   418  -583}%
\special{pa   407  -572}\special{pa   397  -562}\special{pa   386  -550}\special{pa   375  -539}%
\special{pa   365  -528}\special{pa   355  -517}\special{pa   344  -506}\special{pa   334  -494}%
\special{pa   324  -483}\special{pa   314  -471}\special{pa   304  -460}\special{pa   294  -448}%
\special{pa   284  -436}\special{pa   274  -424}\special{pa   264  -413}\special{pa   255  -401}%
\special{pa   245  -389}\special{pa   236  -377}\special{pa   226  -364}\special{pa   217  -352}%
\special{pa   208  -340}\special{pa   199  -328}\special{pa   190  -315}\special{pa   181  -303}%
\special{pa   172  -290}\special{pa   163  -278}\special{pa   154  -265}\special{pa   146  -252}%
\special{pa   137  -239}\special{pa   129  -227}\special{pa   121  -214}\special{pa   112  -201}%
\special{pa   104  -188}\special{pa    96  -175}\special{pa    88  -162}\special{pa    80  -148}%
\special{pa    73  -135}\special{pa    65  -122}\special{pa    57  -109}\special{pa    50   -95}%
\special{pa    42   -82}\special{pa    35   -68}\special{pa    28   -55}\special{pa    21   -41}%
\special{pa    14   -27}\special{pa     7   -14}\special{pa    -0     0}%
\special{fp}%
\settowidth{\Width}{$z=a+bi$}\setlength{\Width}{0\Width}%
\settoheight{\Height}{$z=a+bi$}\settodepth{\Depth}{$z=a+bi$}\setlength{\Height}{\Depth}%
\put(3.2700000,2.8700000){\hspace*{\Width}\raisebox{\Height}{$z=a+bi$}}%
%
\special{pa  1270   -20}\special{pa  1270    20}%
\special{fp}%
\settowidth{\Width}{$a$}\setlength{\Width}{-0.5\Width}%
\settoheight{\Height}{$a$}\settodepth{\Depth}{$a$}\setlength{\Height}{-\Height}%
\put(3.2248800,-0.1000000){\hspace*{\Width}\raisebox{\Height}{$a$}}%
%
%
\special{pa    20 -1110}\special{pa   -20 -1110}%
\special{fp}%
\settowidth{\Width}{$b$}\setlength{\Width}{-1\Width}%
\settoheight{\Height}{$b$}\settodepth{\Depth}{$b$}\setlength{\Height}{-0.5\Height}\setlength{\Depth}{0.5\Depth}\addtolength{\Height}{\Depth}%
\put(-0.1000000,2.8191000){\hspace*{\Width}\raisebox{\Height}{$b$}}%
%
%
\special{pa  -394    -0}\special{pa  1575    -0}%
\special{fp}%
\special{pa     0   394}\special{pa     0 -1575}%
\special{fp}%
\settowidth{\Width}{$x$}\setlength{\Width}{0\Width}%
\settoheight{\Height}{$x$}\settodepth{\Depth}{$x$}\setlength{\Height}{-0.5\Height}\setlength{\Depth}{0.5\Depth}\addtolength{\Height}{\Depth}%
\put(4.0500000,0.0000000){\hspace*{\Width}\raisebox{\Height}{$x$}}%
%
\settowidth{\Width}{$y$}\setlength{\Width}{-0.5\Width}%
\settoheight{\Height}{$y$}\settodepth{\Depth}{$y$}\setlength{\Height}{\Depth}%
\put(0.0000000,4.0500000){\hspace*{\Width}\raisebox{\Height}{$y$}}%
%
\settowidth{\Width}{O}\setlength{\Width}{-1\Width}%
\settoheight{\Height}{O}\settodepth{\Depth}{O}\setlength{\Height}{-\Height}%
\put(-0.0500000,-0.0500000){\hspace*{\Width}\raisebox{\Height}{O}}%
%
\end{picture}}%}
\end{layer}

\begin{itemize}
\item
$z=a+b\,i$を平面上の点$(a,\ b)$で表したとき\vspace{-2mm}
\item
[]{\color{red}絶対値}$|z|$\\
 原点Oと$z$の距離\\
  $|z|=\sqrt{a^2+b^2}$\vspace{-2mm}
\item
[]{\color{red}偏角}$\mathrm{arg}z$\\
 $\mathrm{O}z$と$x$軸(正)の角$\theta$\vspace{-2mm}
\end{itemize}

\sameslide

\vspace*{18mm}

\slidepage

\begin{layer}{120}{0}
\putnotese{70}{17}{%%% /polytech22.git/107-0606/presen/fig/absangle2.tex 
%%% Generator=fig22108.cdy 
{\unitlength=1cm%
\begin{picture}%
(5,5)(-1,-1)%
\linethickness{0.008in}%%
\linethickness{0.012in}%%
\polyline(0.00000,0.00000)(2.50000,2.50000)%
%
\linethickness{0.008in}%%
\polyline(2.50000,0.00000)(2.50000,0.09804)\polyline(2.50000,0.19608)(2.50000,0.29412)%
\polyline(2.50000,0.39216)(2.50000,0.49020)\polyline(2.50000,0.58824)(2.50000,0.68627)%
\polyline(2.50000,0.78431)(2.50000,0.88235)\polyline(2.50000,0.98039)(2.50000,1.07843)%
\polyline(2.50000,1.17647)(2.50000,1.27451)\polyline(2.50000,1.37255)(2.50000,1.47059)%
\polyline(2.50000,1.56863)(2.50000,1.66667)\polyline(2.50000,1.76471)(2.50000,1.86275)%
\polyline(2.50000,1.96078)(2.50000,2.05882)\polyline(2.50000,2.15686)(2.50000,2.25490)%
\polyline(2.50000,2.35294)(2.50000,2.45098)\polyline(2.45098,2.50000)(2.35294,2.50000)%
\polyline(2.25490,2.50000)(2.15686,2.50000)\polyline(2.05882,2.50000)(1.96078,2.50000)%
\polyline(1.86275,2.50000)(1.76471,2.50000)\polyline(1.66667,2.50000)(1.56863,2.50000)%
\polyline(1.47059,2.50000)(1.37255,2.50000)\polyline(1.27451,2.50000)(1.17647,2.50000)%
\polyline(1.07843,2.50000)(0.98039,2.50000)\polyline(0.88235,2.50000)(0.78431,2.50000)%
\polyline(0.68627,2.50000)(0.58824,2.50000)\polyline(0.49020,2.50000)(0.39216,2.50000)%
\polyline(0.29412,2.50000)(0.19608,2.50000)\polyline(0.09804,2.50000)(0.00000,2.50000)%
%
%
\polyline(0.50000,0.00000)(0.49606,0.06267)(0.48429,0.12434)(0.46489,0.18406)(0.43815,0.24088)%
(0.40451,0.29389)(0.36448,0.34227)(0.35303,0.35303)%
%
{\large%
\settowidth{\Width}{$\dfrac{\pi}{4}$}\setlength{\Width}{-0.5\Width}%
\settoheight{\Height}{$\dfrac{\pi}{4}$}\settodepth{\Depth}{$\dfrac{\pi}{4}$}\setlength{\Height}{-0.5\Height}\setlength{\Depth}{0.5\Depth}\addtolength{\Height}{\Depth}%
\put(0.8300000,0.3400000){\hspace*{\Width}\raisebox{\Height}{$\dfrac{\pi}{4}$}}%
}%
%
\polyline(2.50000,2.50000)(2.47295,2.48876)(2.44597,2.47735)(2.41907,2.46577)(2.39224,2.45401)%
(2.36549,2.44209)(2.33881,2.43000)(2.31221,2.41773)(2.28569,2.40530)(2.25925,2.39269)%
(2.23289,2.37992)(2.20661,2.36699)(2.18042,2.35388)(2.15431,2.34061)(2.12828,2.32717)%
(2.10234,2.31356)(2.07649,2.29980)(2.05072,2.28586)(2.02505,2.27176)(1.99946,2.25750)%
(1.97397,2.24308)(1.94857,2.22849)(1.92326,2.21375)(1.89805,2.19884)(1.87293,2.18377)%
(1.84791,2.16854)(1.82299,2.15315)(1.79816,2.13761)(1.77344,2.12190)(1.74881,2.10604)%
(1.72429,2.09003)(1.69987,2.07385)(1.67555,2.05752)(1.65134,2.04104)(1.62724,2.02440)%
(1.60324,2.00761)(1.57934,1.99067)(1.55556,1.97357)(1.53188,1.95632)(1.50832,1.93893)%
(1.48487,1.92138)(1.46153,1.90368)(1.43830,1.88584)(1.41518,1.86785)(1.39219,1.84971)%
(1.36930,1.83142)(1.34654,1.81299)(1.32389,1.79442)(1.30136,1.77570)(1.27895,1.75683)%
(1.25667,1.73783)%
%
\polyline(0.76217,1.24333)(0.74317,1.22105)(0.72430,1.19864)(0.70558,1.17611)(0.68701,1.15346)%
(0.66858,1.13070)(0.65029,1.10781)(0.63215,1.08482)(0.61416,1.06170)(0.59632,1.03847)%
(0.57862,1.01513)(0.56107,0.99168)(0.54368,0.96812)(0.52643,0.94444)(0.50933,0.92066)%
(0.49239,0.89676)(0.47560,0.87276)(0.45896,0.84866)(0.44248,0.82445)(0.42615,0.80013)%
(0.40997,0.77571)(0.39396,0.75119)(0.37810,0.72656)(0.36239,0.70184)(0.34685,0.67701)%
(0.33146,0.65209)(0.31623,0.62707)(0.30116,0.60195)(0.28625,0.57674)(0.27151,0.55143)%
(0.25692,0.52603)(0.24250,0.50054)(0.22824,0.47495)(0.21414,0.44928)(0.20020,0.42351)%
(0.18644,0.39766)(0.17283,0.37172)(0.15939,0.34569)(0.14612,0.31958)(0.13301,0.29339)%
(0.12008,0.26711)(0.10731,0.24075)(0.09470,0.21431)(0.08227,0.18779)(0.07000,0.16119)%
(0.05791,0.13451)(0.04599,0.10776)(0.03423,0.08093)(0.02265,0.05403)(0.01124,0.02705)%
(-0.00000,-0.00000)%
%
{\large%
\settowidth{\Width}{$\sqrt{2}$}\setlength{\Width}{-0.5\Width}%
\settoheight{\Height}{$\sqrt{2}$}\settodepth{\Depth}{$\sqrt{2}$}\setlength{\Height}{-0.5\Height}\setlength{\Depth}{0.5\Depth}\addtolength{\Height}{\Depth}%
\put(1.0000000,1.5000000){\hspace*{\Width}\raisebox{\Height}{$\sqrt{2}$}}%
%
\settowidth{\Width}{$z=1+i$}\setlength{\Width}{0\Width}%
\settoheight{\Height}{$z=1+i$}\settodepth{\Depth}{$z=1+i$}\setlength{\Height}{\Depth}%
\put(2.5500000,2.5500000){\hspace*{\Width}\raisebox{\Height}{$z=1+i$}}%
}%
%
\polyline(2.50000,0.05000)(2.50000,-0.05000)%
%
\settowidth{\Width}{$1$}\setlength{\Width}{-0.5\Width}%
\settoheight{\Height}{$1$}\settodepth{\Depth}{$1$}\setlength{\Height}{-\Height}%
\put(2.5000000,-0.1000000){\hspace*{\Width}\raisebox{\Height}{$1$}}%
%
\polyline(0.05000,2.50000)(-0.05000,2.50000)%
%
\settowidth{\Width}{$1$}\setlength{\Width}{-1\Width}%
\settoheight{\Height}{$1$}\settodepth{\Depth}{$1$}\setlength{\Height}{-0.5\Height}\setlength{\Depth}{0.5\Depth}\addtolength{\Height}{\Depth}%
\put(-0.1000000,2.5000000){\hspace*{\Width}\raisebox{\Height}{$1$}}%
%
\polyline(-1.00000,0.00000)(4.00000,0.00000)%
%
\polyline(0.00000,-1.00000)(0.00000,4.00000)%
%
\settowidth{\Width}{$x$}\setlength{\Width}{0\Width}%
\settoheight{\Height}{$x$}\settodepth{\Depth}{$x$}\setlength{\Height}{-0.5\Height}\setlength{\Depth}{0.5\Depth}\addtolength{\Height}{\Depth}%
\put(4.0500000,0.0000000){\hspace*{\Width}\raisebox{\Height}{$x$}}%
%
\settowidth{\Width}{$y$}\setlength{\Width}{-0.5\Width}%
\settoheight{\Height}{$y$}\settodepth{\Depth}{$y$}\setlength{\Height}{\Depth}%
\put(0.0000000,4.0500000){\hspace*{\Width}\raisebox{\Height}{$y$}}%
%
\settowidth{\Width}{O}\setlength{\Width}{-1\Width}%
\settoheight{\Height}{O}\settodepth{\Depth}{O}\setlength{\Height}{-\Height}%
\put(-0.0500000,-0.0500000){\hspace*{\Width}\raisebox{\Height}{O}}%
%
\end{picture}}%}
\end{layer}

\begin{itemize}
\item
$z=a+b\,i$を平面上の点$(a,\ b)$で表したとき\vspace{-2mm}
\item
[]{\color{red}絶対値}$|z|$\\
 原点Oと$z$の距離\\
  $|z|=\sqrt{a^2+b^2}$\vspace{-2mm}
\item
[]{\color{red}偏角}$\mathrm{arg}z$\\
 $\mathrm{O}z$と$x$軸(正)の角$\theta$\vspace{-2mm}
\item
[例)]$z=1+i$\\
 絶対値\ \hakom{{\color{blue}\sqrt{2}}},偏角\ \hakom{{\color{blue}\bunsuu{\pi}{4}}}
\end{itemize}

\sameslide

\vspace*{18mm}

\slidepage

\begin{layer}{120}{0}
\putnotese{70}{17}{%%% /polytech22.git/107-0606/presen/fig/absangle2.tex 
%%% Generator=fig22108.cdy 
{\unitlength=1cm%
\begin{picture}%
(5,5)(-1,-1)%
\linethickness{0.008in}%%
\linethickness{0.012in}%%
\polyline(0.00000,0.00000)(2.50000,2.50000)%
%
\linethickness{0.008in}%%
\polyline(2.50000,0.00000)(2.50000,0.09804)\polyline(2.50000,0.19608)(2.50000,0.29412)%
\polyline(2.50000,0.39216)(2.50000,0.49020)\polyline(2.50000,0.58824)(2.50000,0.68627)%
\polyline(2.50000,0.78431)(2.50000,0.88235)\polyline(2.50000,0.98039)(2.50000,1.07843)%
\polyline(2.50000,1.17647)(2.50000,1.27451)\polyline(2.50000,1.37255)(2.50000,1.47059)%
\polyline(2.50000,1.56863)(2.50000,1.66667)\polyline(2.50000,1.76471)(2.50000,1.86275)%
\polyline(2.50000,1.96078)(2.50000,2.05882)\polyline(2.50000,2.15686)(2.50000,2.25490)%
\polyline(2.50000,2.35294)(2.50000,2.45098)\polyline(2.45098,2.50000)(2.35294,2.50000)%
\polyline(2.25490,2.50000)(2.15686,2.50000)\polyline(2.05882,2.50000)(1.96078,2.50000)%
\polyline(1.86275,2.50000)(1.76471,2.50000)\polyline(1.66667,2.50000)(1.56863,2.50000)%
\polyline(1.47059,2.50000)(1.37255,2.50000)\polyline(1.27451,2.50000)(1.17647,2.50000)%
\polyline(1.07843,2.50000)(0.98039,2.50000)\polyline(0.88235,2.50000)(0.78431,2.50000)%
\polyline(0.68627,2.50000)(0.58824,2.50000)\polyline(0.49020,2.50000)(0.39216,2.50000)%
\polyline(0.29412,2.50000)(0.19608,2.50000)\polyline(0.09804,2.50000)(0.00000,2.50000)%
%
%
\polyline(0.50000,0.00000)(0.49606,0.06267)(0.48429,0.12434)(0.46489,0.18406)(0.43815,0.24088)%
(0.40451,0.29389)(0.36448,0.34227)(0.35303,0.35303)%
%
{\large%
\settowidth{\Width}{$\dfrac{\pi}{4}$}\setlength{\Width}{-0.5\Width}%
\settoheight{\Height}{$\dfrac{\pi}{4}$}\settodepth{\Depth}{$\dfrac{\pi}{4}$}\setlength{\Height}{-0.5\Height}\setlength{\Depth}{0.5\Depth}\addtolength{\Height}{\Depth}%
\put(0.8300000,0.3400000){\hspace*{\Width}\raisebox{\Height}{$\dfrac{\pi}{4}$}}%
}%
%
\polyline(2.50000,2.50000)(2.47295,2.48876)(2.44597,2.47735)(2.41907,2.46577)(2.39224,2.45401)%
(2.36549,2.44209)(2.33881,2.43000)(2.31221,2.41773)(2.28569,2.40530)(2.25925,2.39269)%
(2.23289,2.37992)(2.20661,2.36699)(2.18042,2.35388)(2.15431,2.34061)(2.12828,2.32717)%
(2.10234,2.31356)(2.07649,2.29980)(2.05072,2.28586)(2.02505,2.27176)(1.99946,2.25750)%
(1.97397,2.24308)(1.94857,2.22849)(1.92326,2.21375)(1.89805,2.19884)(1.87293,2.18377)%
(1.84791,2.16854)(1.82299,2.15315)(1.79816,2.13761)(1.77344,2.12190)(1.74881,2.10604)%
(1.72429,2.09003)(1.69987,2.07385)(1.67555,2.05752)(1.65134,2.04104)(1.62724,2.02440)%
(1.60324,2.00761)(1.57934,1.99067)(1.55556,1.97357)(1.53188,1.95632)(1.50832,1.93893)%
(1.48487,1.92138)(1.46153,1.90368)(1.43830,1.88584)(1.41518,1.86785)(1.39219,1.84971)%
(1.36930,1.83142)(1.34654,1.81299)(1.32389,1.79442)(1.30136,1.77570)(1.27895,1.75683)%
(1.25667,1.73783)%
%
\polyline(0.76217,1.24333)(0.74317,1.22105)(0.72430,1.19864)(0.70558,1.17611)(0.68701,1.15346)%
(0.66858,1.13070)(0.65029,1.10781)(0.63215,1.08482)(0.61416,1.06170)(0.59632,1.03847)%
(0.57862,1.01513)(0.56107,0.99168)(0.54368,0.96812)(0.52643,0.94444)(0.50933,0.92066)%
(0.49239,0.89676)(0.47560,0.87276)(0.45896,0.84866)(0.44248,0.82445)(0.42615,0.80013)%
(0.40997,0.77571)(0.39396,0.75119)(0.37810,0.72656)(0.36239,0.70184)(0.34685,0.67701)%
(0.33146,0.65209)(0.31623,0.62707)(0.30116,0.60195)(0.28625,0.57674)(0.27151,0.55143)%
(0.25692,0.52603)(0.24250,0.50054)(0.22824,0.47495)(0.21414,0.44928)(0.20020,0.42351)%
(0.18644,0.39766)(0.17283,0.37172)(0.15939,0.34569)(0.14612,0.31958)(0.13301,0.29339)%
(0.12008,0.26711)(0.10731,0.24075)(0.09470,0.21431)(0.08227,0.18779)(0.07000,0.16119)%
(0.05791,0.13451)(0.04599,0.10776)(0.03423,0.08093)(0.02265,0.05403)(0.01124,0.02705)%
(-0.00000,-0.00000)%
%
{\large%
\settowidth{\Width}{$\sqrt{2}$}\setlength{\Width}{-0.5\Width}%
\settoheight{\Height}{$\sqrt{2}$}\settodepth{\Depth}{$\sqrt{2}$}\setlength{\Height}{-0.5\Height}\setlength{\Depth}{0.5\Depth}\addtolength{\Height}{\Depth}%
\put(1.0000000,1.5000000){\hspace*{\Width}\raisebox{\Height}{$\sqrt{2}$}}%
%
\settowidth{\Width}{$z=1+i$}\setlength{\Width}{0\Width}%
\settoheight{\Height}{$z=1+i$}\settodepth{\Depth}{$z=1+i$}\setlength{\Height}{\Depth}%
\put(2.5500000,2.5500000){\hspace*{\Width}\raisebox{\Height}{$z=1+i$}}%
}%
%
\polyline(2.50000,0.05000)(2.50000,-0.05000)%
%
\settowidth{\Width}{$1$}\setlength{\Width}{-0.5\Width}%
\settoheight{\Height}{$1$}\settodepth{\Depth}{$1$}\setlength{\Height}{-\Height}%
\put(2.5000000,-0.1000000){\hspace*{\Width}\raisebox{\Height}{$1$}}%
%
\polyline(0.05000,2.50000)(-0.05000,2.50000)%
%
\settowidth{\Width}{$1$}\setlength{\Width}{-1\Width}%
\settoheight{\Height}{$1$}\settodepth{\Depth}{$1$}\setlength{\Height}{-0.5\Height}\setlength{\Depth}{0.5\Depth}\addtolength{\Height}{\Depth}%
\put(-0.1000000,2.5000000){\hspace*{\Width}\raisebox{\Height}{$1$}}%
%
\polyline(-1.00000,0.00000)(4.00000,0.00000)%
%
\polyline(0.00000,-1.00000)(0.00000,4.00000)%
%
\settowidth{\Width}{$x$}\setlength{\Width}{0\Width}%
\settoheight{\Height}{$x$}\settodepth{\Depth}{$x$}\setlength{\Height}{-0.5\Height}\setlength{\Depth}{0.5\Depth}\addtolength{\Height}{\Depth}%
\put(4.0500000,0.0000000){\hspace*{\Width}\raisebox{\Height}{$x$}}%
%
\settowidth{\Width}{$y$}\setlength{\Width}{-0.5\Width}%
\settoheight{\Height}{$y$}\settodepth{\Depth}{$y$}\setlength{\Height}{\Depth}%
\put(0.0000000,4.0500000){\hspace*{\Width}\raisebox{\Height}{$y$}}%
%
\settowidth{\Width}{O}\setlength{\Width}{-1\Width}%
\settoheight{\Height}{O}\settodepth{\Depth}{O}\setlength{\Height}{-\Height}%
\put(-0.0500000,-0.0500000){\hspace*{\Width}\raisebox{\Height}{O}}%
%
\end{picture}}%}
\end{layer}

\begin{itemize}
\item
$z=a+b\,i$を平面上の点$(a,\ b)$で表したとき\vspace{-2mm}
\item
[]{\color{red}絶対値}$|z|$\\
 原点Oと$z$の距離\\
  $|z|=\sqrt{a^2+b^2}$\vspace{-2mm}
\item
[]{\color{red}偏角}$\mathrm{arg}z$\\
 $\mathrm{O}z$と$x$軸(正)の角$\theta$\vspace{-2mm}
\item
[例)]$z=1+i$\\
 絶対値\ \hakoma{{\color{blue}\sqrt{2}}},偏角\ \hakoma{{\color{blue}\bunsuu{\pi}{4}}}
\end{itemize}

\newslide{絶対値と偏角の問題}

\vspace*{18mm}

\slidepage
\begin{itemize}
\item
[課題]\monban 次の複素数の絶対値と偏角を求めよ.\seteda{50}\\
\eda{$z_1=\sqrt{3}+i$}\eda{$z_2=i$}\\\eda{$z_3=-2$}\eda{$z_4=-3i$}
\end{itemize}
%%%%%%%%%%%%%

%%%%%%%%%%%%%%%%%%%%


\newslide{絶対値と偏角の応用問題}

\vspace*{18mm}

\slidepage
\begin{itemize}
\item
[課題]\monban $z=a+bi,\ w=c+di$とする.\seteda{800}\\
\eda{$|z|^2$を$a,b$で表せ}\\
\eda{$|w|^2$を$c,d$で表せ}\\
\eda{$zw$の実部と虚部を$a,b,c,d$で表せ}\\
\eda{$|zw|^2$を計算せよ}\\
\eda{$|zw|^2=|z|^2|w|^2$を証明せよ}
\end{itemize}
\ifnum 1=0
%%%%%%%%%%%%%

%%%%%%%%%%%%%%%%%%%%


\newslide{極形式}

\vspace*{18mm}

\slidepage

\begin{layer}{120}{0}
\putnotese{75}{17}{%%% /Users/takatoosetsuo/Dropbox/2018polytec/lecture/0611/presen/fig/absangle.tex 
%%% Generator=presen0611.cdy 
{\unitlength=1cm%
\begin{picture}%
(5,5)(-1,-1)%
\special{pn 8}%
%
\Large\bf\boldmath%
\small%
\special{pn 12}%
\special{pa     0    -0}\special{pa  1270 -1110}%
\special{fp}%
\special{pn 8}%
\special{pa 1270 0}\special{pa 1270 -39}\special{fp}\special{pa 1270 -78}\special{pa 1270 -117}\special{fp}%
\special{pa 1270 -156}\special{pa 1270 -195}\special{fp}\special{pa 1270 -234}\special{pa 1270 -273}\special{fp}%
\special{pa 1270 -312}\special{pa 1270 -351}\special{fp}\special{pa 1270 -390}\special{pa 1270 -429}\special{fp}%
\special{pa 1270 -468}\special{pa 1270 -507}\special{fp}\special{pa 1270 -546}\special{pa 1270 -585}\special{fp}%
\special{pa 1270 -624}\special{pa 1270 -663}\special{fp}\special{pa 1270 -702}\special{pa 1270 -741}\special{fp}%
\special{pa 1270 -780}\special{pa 1270 -819}\special{fp}\special{pa 1270 -858}\special{pa 1270 -897}\special{fp}%
\special{pa 1270 -936}\special{pa 1270 -975}\special{fp}\special{pa 1270 -1014}\special{pa 1270 -1053}\special{fp}%
\special{pa 1270 -1092}\special{pa 1270 -1110}\special{pa 1248 -1110}\special{fp}%
\special{pa 1209 -1110}\special{pa 1170 -1110}\special{fp}\special{pa 1131 -1110}\special{pa 1092 -1110}\special{fp}%
\special{pa 1053 -1110}\special{pa 1014 -1110}\special{fp}\special{pa 975 -1110}\special{pa 936 -1110}\special{fp}%
\special{pa 897 -1110}\special{pa 858 -1110}\special{fp}\special{pa 819 -1110}\special{pa 780 -1110}\special{fp}%
\special{pa 741 -1110}\special{pa 702 -1110}\special{fp}\special{pa 663 -1110}\special{pa 624 -1110}\special{fp}%
\special{pa 585 -1110}\special{pa 546 -1110}\special{fp}\special{pa 507 -1110}\special{pa 468 -1110}\special{fp}%
\special{pa 429 -1110}\special{pa 390 -1110}\special{fp}\special{pa 351 -1110}\special{pa 312 -1110}\special{fp}%
\special{pa 273 -1110}\special{pa 234 -1110}\special{fp}\special{pa 195 -1110}\special{pa 156 -1110}\special{fp}%
\special{pa 117 -1110}\special{pa 78 -1110}\special{fp}\special{pa 39 -1110}\special{pa 0 -1110}\special{fp}%
%
%
\settowidth{\Width}{$\theta$}\setlength{\Width}{-0.5\Width}%
\settoheight{\Height}{$\theta$}\settodepth{\Depth}{$\theta$}\setlength{\Height}{-0.5\Height}\setlength{\Depth}{0.5\Depth}\addtolength{\Height}{\Depth}%
\put(0.6100000,0.2300000){\hspace*{\Width}\raisebox{\Height}{$\theta$}}%
%
\special{pa   197    -0}\special{pa   195   -25}\special{pa   191   -49}\special{pa   183   -72}%
\special{pa   173   -95}\special{pa   159  -116}\special{pa   148  -129}%
\special{fp}%
\settowidth{\Width}{$|z|$}\setlength{\Width}{-0.5\Width}%
\settoheight{\Height}{$|z|$}\settodepth{\Depth}{$|z|$}\setlength{\Height}{-0.5\Height}\setlength{\Depth}{0.5\Depth}\addtolength{\Height}{\Depth}%
\put(1.3300000,1.7300000){\hspace*{\Width}\raisebox{\Height}{$|z|$}}%
%
\special{pa  1270 -1110}\special{pa  1255 -1105}\special{pa  1241 -1100}\special{pa  1226 -1095}%
\special{pa  1212 -1090}\special{pa  1197 -1084}\special{pa  1183 -1079}\special{pa  1169 -1073}%
\special{pa  1154 -1068}\special{pa  1140 -1062}\special{pa  1126 -1056}\special{pa  1112 -1050}%
\special{pa  1098 -1044}\special{pa  1084 -1038}\special{pa  1070 -1032}\special{pa  1056 -1025}%
\special{pa  1042 -1019}\special{pa  1028 -1012}\special{pa  1014 -1006}\special{pa  1000  -999}%
\special{pa   986  -992}\special{pa   973  -985}\special{pa   959  -978}\special{pa   946  -971}%
\special{pa   932  -964}\special{pa   919  -957}\special{pa   905  -949}\special{pa   892  -942}%
\special{pa   878  -934}\special{pa   865  -927}\special{pa   852  -919}\special{pa   839  -911}%
\special{pa   825  -903}\special{pa   812  -895}\special{pa   799  -887}\special{pa   786  -879}%
\special{pa   774  -871}\special{pa   761  -862}\special{pa   748  -854}\special{pa   735  -845}%
\special{pa   723  -836}\special{pa   710  -828}\special{pa   697  -819}\special{pa   685  -810}%
\special{pa   672  -801}\special{pa   660  -792}\special{pa   648  -783}\special{pa   636  -773}%
\special{pa   623  -764}\special{pa   611  -755}\special{pa   599  -745}%
\special{fp}%
\special{pa   451  -615}\special{pa   440  -605}\special{pa   429  -594}\special{pa   418  -583}%
\special{pa   407  -572}\special{pa   397  -562}\special{pa   386  -550}\special{pa   375  -539}%
\special{pa   365  -528}\special{pa   355  -517}\special{pa   344  -506}\special{pa   334  -494}%
\special{pa   324  -483}\special{pa   314  -471}\special{pa   304  -460}\special{pa   294  -448}%
\special{pa   284  -436}\special{pa   274  -424}\special{pa   264  -413}\special{pa   255  -401}%
\special{pa   245  -389}\special{pa   236  -377}\special{pa   226  -364}\special{pa   217  -352}%
\special{pa   208  -340}\special{pa   199  -328}\special{pa   190  -315}\special{pa   181  -303}%
\special{pa   172  -290}\special{pa   163  -278}\special{pa   154  -265}\special{pa   146  -252}%
\special{pa   137  -239}\special{pa   129  -227}\special{pa   121  -214}\special{pa   112  -201}%
\special{pa   104  -188}\special{pa    96  -175}\special{pa    88  -162}\special{pa    80  -148}%
\special{pa    73  -135}\special{pa    65  -122}\special{pa    57  -109}\special{pa    50   -95}%
\special{pa    42   -82}\special{pa    35   -68}\special{pa    28   -55}\special{pa    21   -41}%
\special{pa    14   -27}\special{pa     7   -14}\special{pa    -0     0}%
\special{fp}%
\settowidth{\Width}{$z=a+bi$}\setlength{\Width}{0\Width}%
\settoheight{\Height}{$z=a+bi$}\settodepth{\Depth}{$z=a+bi$}\setlength{\Height}{\Depth}%
\put(3.2700000,2.8700000){\hspace*{\Width}\raisebox{\Height}{$z=a+bi$}}%
%
\special{pa  1270   -20}\special{pa  1270    20}%
\special{fp}%
\settowidth{\Width}{$a$}\setlength{\Width}{-0.5\Width}%
\settoheight{\Height}{$a$}\settodepth{\Depth}{$a$}\setlength{\Height}{-\Height}%
\put(3.2248800,-0.1000000){\hspace*{\Width}\raisebox{\Height}{$a$}}%
%
%
\special{pa    20 -1110}\special{pa   -20 -1110}%
\special{fp}%
\settowidth{\Width}{$b$}\setlength{\Width}{-1\Width}%
\settoheight{\Height}{$b$}\settodepth{\Depth}{$b$}\setlength{\Height}{-0.5\Height}\setlength{\Depth}{0.5\Depth}\addtolength{\Height}{\Depth}%
\put(-0.1000000,2.8191000){\hspace*{\Width}\raisebox{\Height}{$b$}}%
%
%
\special{pa  -394    -0}\special{pa  1575    -0}%
\special{fp}%
\special{pa     0   394}\special{pa     0 -1575}%
\special{fp}%
\settowidth{\Width}{$x$}\setlength{\Width}{0\Width}%
\settoheight{\Height}{$x$}\settodepth{\Depth}{$x$}\setlength{\Height}{-0.5\Height}\setlength{\Depth}{0.5\Depth}\addtolength{\Height}{\Depth}%
\put(4.0500000,0.0000000){\hspace*{\Width}\raisebox{\Height}{$x$}}%
%
\settowidth{\Width}{$y$}\setlength{\Width}{-0.5\Width}%
\settoheight{\Height}{$y$}\settodepth{\Depth}{$y$}\setlength{\Height}{\Depth}%
\put(0.0000000,4.0500000){\hspace*{\Width}\raisebox{\Height}{$y$}}%
%
\settowidth{\Width}{O}\setlength{\Width}{-1\Width}%
\settoheight{\Height}{O}\settodepth{\Depth}{O}\setlength{\Height}{-\Height}%
\put(-0.0500000,-0.0500000){\hspace*{\Width}\raisebox{\Height}{O}}%
%
\end{picture}}%}
\end{layer}

\begin{itemize}
\item
$z=a+bi$の絶対値\ $r$,偏角\ $\theta$
\end{itemize}
%%%%%%%%%%%%%

%%%%%%%%%%%%%%%%%%%%


\sameslide

\vspace*{18mm}

\slidepage

\begin{layer}{120}{0}
\putnotese{75}{17}{%%% /Users/takatoosetsuo/Dropbox/2018polytec/lecture/0611/presen/fig/absangle.tex 
%%% Generator=presen0611.cdy 
{\unitlength=1cm%
\begin{picture}%
(5,5)(-1,-1)%
\special{pn 8}%
%
\Large\bf\boldmath%
\small%
\special{pn 12}%
\special{pa     0    -0}\special{pa  1270 -1110}%
\special{fp}%
\special{pn 8}%
\special{pa 1270 0}\special{pa 1270 -39}\special{fp}\special{pa 1270 -78}\special{pa 1270 -117}\special{fp}%
\special{pa 1270 -156}\special{pa 1270 -195}\special{fp}\special{pa 1270 -234}\special{pa 1270 -273}\special{fp}%
\special{pa 1270 -312}\special{pa 1270 -351}\special{fp}\special{pa 1270 -390}\special{pa 1270 -429}\special{fp}%
\special{pa 1270 -468}\special{pa 1270 -507}\special{fp}\special{pa 1270 -546}\special{pa 1270 -585}\special{fp}%
\special{pa 1270 -624}\special{pa 1270 -663}\special{fp}\special{pa 1270 -702}\special{pa 1270 -741}\special{fp}%
\special{pa 1270 -780}\special{pa 1270 -819}\special{fp}\special{pa 1270 -858}\special{pa 1270 -897}\special{fp}%
\special{pa 1270 -936}\special{pa 1270 -975}\special{fp}\special{pa 1270 -1014}\special{pa 1270 -1053}\special{fp}%
\special{pa 1270 -1092}\special{pa 1270 -1110}\special{pa 1248 -1110}\special{fp}%
\special{pa 1209 -1110}\special{pa 1170 -1110}\special{fp}\special{pa 1131 -1110}\special{pa 1092 -1110}\special{fp}%
\special{pa 1053 -1110}\special{pa 1014 -1110}\special{fp}\special{pa 975 -1110}\special{pa 936 -1110}\special{fp}%
\special{pa 897 -1110}\special{pa 858 -1110}\special{fp}\special{pa 819 -1110}\special{pa 780 -1110}\special{fp}%
\special{pa 741 -1110}\special{pa 702 -1110}\special{fp}\special{pa 663 -1110}\special{pa 624 -1110}\special{fp}%
\special{pa 585 -1110}\special{pa 546 -1110}\special{fp}\special{pa 507 -1110}\special{pa 468 -1110}\special{fp}%
\special{pa 429 -1110}\special{pa 390 -1110}\special{fp}\special{pa 351 -1110}\special{pa 312 -1110}\special{fp}%
\special{pa 273 -1110}\special{pa 234 -1110}\special{fp}\special{pa 195 -1110}\special{pa 156 -1110}\special{fp}%
\special{pa 117 -1110}\special{pa 78 -1110}\special{fp}\special{pa 39 -1110}\special{pa 0 -1110}\special{fp}%
%
%
\settowidth{\Width}{$\theta$}\setlength{\Width}{-0.5\Width}%
\settoheight{\Height}{$\theta$}\settodepth{\Depth}{$\theta$}\setlength{\Height}{-0.5\Height}\setlength{\Depth}{0.5\Depth}\addtolength{\Height}{\Depth}%
\put(0.6100000,0.2300000){\hspace*{\Width}\raisebox{\Height}{$\theta$}}%
%
\special{pa   197    -0}\special{pa   195   -25}\special{pa   191   -49}\special{pa   183   -72}%
\special{pa   173   -95}\special{pa   159  -116}\special{pa   148  -129}%
\special{fp}%
\settowidth{\Width}{$|z|$}\setlength{\Width}{-0.5\Width}%
\settoheight{\Height}{$|z|$}\settodepth{\Depth}{$|z|$}\setlength{\Height}{-0.5\Height}\setlength{\Depth}{0.5\Depth}\addtolength{\Height}{\Depth}%
\put(1.3300000,1.7300000){\hspace*{\Width}\raisebox{\Height}{$|z|$}}%
%
\special{pa  1270 -1110}\special{pa  1255 -1105}\special{pa  1241 -1100}\special{pa  1226 -1095}%
\special{pa  1212 -1090}\special{pa  1197 -1084}\special{pa  1183 -1079}\special{pa  1169 -1073}%
\special{pa  1154 -1068}\special{pa  1140 -1062}\special{pa  1126 -1056}\special{pa  1112 -1050}%
\special{pa  1098 -1044}\special{pa  1084 -1038}\special{pa  1070 -1032}\special{pa  1056 -1025}%
\special{pa  1042 -1019}\special{pa  1028 -1012}\special{pa  1014 -1006}\special{pa  1000  -999}%
\special{pa   986  -992}\special{pa   973  -985}\special{pa   959  -978}\special{pa   946  -971}%
\special{pa   932  -964}\special{pa   919  -957}\special{pa   905  -949}\special{pa   892  -942}%
\special{pa   878  -934}\special{pa   865  -927}\special{pa   852  -919}\special{pa   839  -911}%
\special{pa   825  -903}\special{pa   812  -895}\special{pa   799  -887}\special{pa   786  -879}%
\special{pa   774  -871}\special{pa   761  -862}\special{pa   748  -854}\special{pa   735  -845}%
\special{pa   723  -836}\special{pa   710  -828}\special{pa   697  -819}\special{pa   685  -810}%
\special{pa   672  -801}\special{pa   660  -792}\special{pa   648  -783}\special{pa   636  -773}%
\special{pa   623  -764}\special{pa   611  -755}\special{pa   599  -745}%
\special{fp}%
\special{pa   451  -615}\special{pa   440  -605}\special{pa   429  -594}\special{pa   418  -583}%
\special{pa   407  -572}\special{pa   397  -562}\special{pa   386  -550}\special{pa   375  -539}%
\special{pa   365  -528}\special{pa   355  -517}\special{pa   344  -506}\special{pa   334  -494}%
\special{pa   324  -483}\special{pa   314  -471}\special{pa   304  -460}\special{pa   294  -448}%
\special{pa   284  -436}\special{pa   274  -424}\special{pa   264  -413}\special{pa   255  -401}%
\special{pa   245  -389}\special{pa   236  -377}\special{pa   226  -364}\special{pa   217  -352}%
\special{pa   208  -340}\special{pa   199  -328}\special{pa   190  -315}\special{pa   181  -303}%
\special{pa   172  -290}\special{pa   163  -278}\special{pa   154  -265}\special{pa   146  -252}%
\special{pa   137  -239}\special{pa   129  -227}\special{pa   121  -214}\special{pa   112  -201}%
\special{pa   104  -188}\special{pa    96  -175}\special{pa    88  -162}\special{pa    80  -148}%
\special{pa    73  -135}\special{pa    65  -122}\special{pa    57  -109}\special{pa    50   -95}%
\special{pa    42   -82}\special{pa    35   -68}\special{pa    28   -55}\special{pa    21   -41}%
\special{pa    14   -27}\special{pa     7   -14}\special{pa    -0     0}%
\special{fp}%
\settowidth{\Width}{$z=a+bi$}\setlength{\Width}{0\Width}%
\settoheight{\Height}{$z=a+bi$}\settodepth{\Depth}{$z=a+bi$}\setlength{\Height}{\Depth}%
\put(3.2700000,2.8700000){\hspace*{\Width}\raisebox{\Height}{$z=a+bi$}}%
%
\special{pa  1270   -20}\special{pa  1270    20}%
\special{fp}%
\settowidth{\Width}{$a$}\setlength{\Width}{-0.5\Width}%
\settoheight{\Height}{$a$}\settodepth{\Depth}{$a$}\setlength{\Height}{-\Height}%
\put(3.2248800,-0.1000000){\hspace*{\Width}\raisebox{\Height}{$a$}}%
%
%
\special{pa    20 -1110}\special{pa   -20 -1110}%
\special{fp}%
\settowidth{\Width}{$b$}\setlength{\Width}{-1\Width}%
\settoheight{\Height}{$b$}\settodepth{\Depth}{$b$}\setlength{\Height}{-0.5\Height}\setlength{\Depth}{0.5\Depth}\addtolength{\Height}{\Depth}%
\put(-0.1000000,2.8191000){\hspace*{\Width}\raisebox{\Height}{$b$}}%
%
%
\special{pa  -394    -0}\special{pa  1575    -0}%
\special{fp}%
\special{pa     0   394}\special{pa     0 -1575}%
\special{fp}%
\settowidth{\Width}{$x$}\setlength{\Width}{0\Width}%
\settoheight{\Height}{$x$}\settodepth{\Depth}{$x$}\setlength{\Height}{-0.5\Height}\setlength{\Depth}{0.5\Depth}\addtolength{\Height}{\Depth}%
\put(4.0500000,0.0000000){\hspace*{\Width}\raisebox{\Height}{$x$}}%
%
\settowidth{\Width}{$y$}\setlength{\Width}{-0.5\Width}%
\settoheight{\Height}{$y$}\settodepth{\Depth}{$y$}\setlength{\Height}{\Depth}%
\put(0.0000000,4.0500000){\hspace*{\Width}\raisebox{\Height}{$y$}}%
%
\settowidth{\Width}{O}\setlength{\Width}{-1\Width}%
\settoheight{\Height}{O}\settodepth{\Depth}{O}\setlength{\Height}{-\Height}%
\put(-0.0500000,-0.0500000){\hspace*{\Width}\raisebox{\Height}{O}}%
%
\end{picture}}%}
\end{layer}

\begin{itemize}
\item
$z=a+bi$の絶対値\ $r$,偏角\ $\theta$
\item
$\cos\theta=\bunsuu{a}{r}$より $a=r\cos\theta$
\item
$\sin\theta=\bunsuu{b}{r}$より $b=r\sin\theta$
\end{itemize}

\sameslide

\vspace*{18mm}

\slidepage

\begin{layer}{120}{0}
\putnotese{75}{17}{%%% /Users/takatoosetsuo/Dropbox/2018polytec/lecture/0611/presen/fig/absangle.tex 
%%% Generator=presen0611.cdy 
{\unitlength=1cm%
\begin{picture}%
(5,5)(-1,-1)%
\special{pn 8}%
%
\Large\bf\boldmath%
\small%
\special{pn 12}%
\special{pa     0    -0}\special{pa  1270 -1110}%
\special{fp}%
\special{pn 8}%
\special{pa 1270 0}\special{pa 1270 -39}\special{fp}\special{pa 1270 -78}\special{pa 1270 -117}\special{fp}%
\special{pa 1270 -156}\special{pa 1270 -195}\special{fp}\special{pa 1270 -234}\special{pa 1270 -273}\special{fp}%
\special{pa 1270 -312}\special{pa 1270 -351}\special{fp}\special{pa 1270 -390}\special{pa 1270 -429}\special{fp}%
\special{pa 1270 -468}\special{pa 1270 -507}\special{fp}\special{pa 1270 -546}\special{pa 1270 -585}\special{fp}%
\special{pa 1270 -624}\special{pa 1270 -663}\special{fp}\special{pa 1270 -702}\special{pa 1270 -741}\special{fp}%
\special{pa 1270 -780}\special{pa 1270 -819}\special{fp}\special{pa 1270 -858}\special{pa 1270 -897}\special{fp}%
\special{pa 1270 -936}\special{pa 1270 -975}\special{fp}\special{pa 1270 -1014}\special{pa 1270 -1053}\special{fp}%
\special{pa 1270 -1092}\special{pa 1270 -1110}\special{pa 1248 -1110}\special{fp}%
\special{pa 1209 -1110}\special{pa 1170 -1110}\special{fp}\special{pa 1131 -1110}\special{pa 1092 -1110}\special{fp}%
\special{pa 1053 -1110}\special{pa 1014 -1110}\special{fp}\special{pa 975 -1110}\special{pa 936 -1110}\special{fp}%
\special{pa 897 -1110}\special{pa 858 -1110}\special{fp}\special{pa 819 -1110}\special{pa 780 -1110}\special{fp}%
\special{pa 741 -1110}\special{pa 702 -1110}\special{fp}\special{pa 663 -1110}\special{pa 624 -1110}\special{fp}%
\special{pa 585 -1110}\special{pa 546 -1110}\special{fp}\special{pa 507 -1110}\special{pa 468 -1110}\special{fp}%
\special{pa 429 -1110}\special{pa 390 -1110}\special{fp}\special{pa 351 -1110}\special{pa 312 -1110}\special{fp}%
\special{pa 273 -1110}\special{pa 234 -1110}\special{fp}\special{pa 195 -1110}\special{pa 156 -1110}\special{fp}%
\special{pa 117 -1110}\special{pa 78 -1110}\special{fp}\special{pa 39 -1110}\special{pa 0 -1110}\special{fp}%
%
%
\settowidth{\Width}{$\theta$}\setlength{\Width}{-0.5\Width}%
\settoheight{\Height}{$\theta$}\settodepth{\Depth}{$\theta$}\setlength{\Height}{-0.5\Height}\setlength{\Depth}{0.5\Depth}\addtolength{\Height}{\Depth}%
\put(0.6100000,0.2300000){\hspace*{\Width}\raisebox{\Height}{$\theta$}}%
%
\special{pa   197    -0}\special{pa   195   -25}\special{pa   191   -49}\special{pa   183   -72}%
\special{pa   173   -95}\special{pa   159  -116}\special{pa   148  -129}%
\special{fp}%
\settowidth{\Width}{$|z|$}\setlength{\Width}{-0.5\Width}%
\settoheight{\Height}{$|z|$}\settodepth{\Depth}{$|z|$}\setlength{\Height}{-0.5\Height}\setlength{\Depth}{0.5\Depth}\addtolength{\Height}{\Depth}%
\put(1.3300000,1.7300000){\hspace*{\Width}\raisebox{\Height}{$|z|$}}%
%
\special{pa  1270 -1110}\special{pa  1255 -1105}\special{pa  1241 -1100}\special{pa  1226 -1095}%
\special{pa  1212 -1090}\special{pa  1197 -1084}\special{pa  1183 -1079}\special{pa  1169 -1073}%
\special{pa  1154 -1068}\special{pa  1140 -1062}\special{pa  1126 -1056}\special{pa  1112 -1050}%
\special{pa  1098 -1044}\special{pa  1084 -1038}\special{pa  1070 -1032}\special{pa  1056 -1025}%
\special{pa  1042 -1019}\special{pa  1028 -1012}\special{pa  1014 -1006}\special{pa  1000  -999}%
\special{pa   986  -992}\special{pa   973  -985}\special{pa   959  -978}\special{pa   946  -971}%
\special{pa   932  -964}\special{pa   919  -957}\special{pa   905  -949}\special{pa   892  -942}%
\special{pa   878  -934}\special{pa   865  -927}\special{pa   852  -919}\special{pa   839  -911}%
\special{pa   825  -903}\special{pa   812  -895}\special{pa   799  -887}\special{pa   786  -879}%
\special{pa   774  -871}\special{pa   761  -862}\special{pa   748  -854}\special{pa   735  -845}%
\special{pa   723  -836}\special{pa   710  -828}\special{pa   697  -819}\special{pa   685  -810}%
\special{pa   672  -801}\special{pa   660  -792}\special{pa   648  -783}\special{pa   636  -773}%
\special{pa   623  -764}\special{pa   611  -755}\special{pa   599  -745}%
\special{fp}%
\special{pa   451  -615}\special{pa   440  -605}\special{pa   429  -594}\special{pa   418  -583}%
\special{pa   407  -572}\special{pa   397  -562}\special{pa   386  -550}\special{pa   375  -539}%
\special{pa   365  -528}\special{pa   355  -517}\special{pa   344  -506}\special{pa   334  -494}%
\special{pa   324  -483}\special{pa   314  -471}\special{pa   304  -460}\special{pa   294  -448}%
\special{pa   284  -436}\special{pa   274  -424}\special{pa   264  -413}\special{pa   255  -401}%
\special{pa   245  -389}\special{pa   236  -377}\special{pa   226  -364}\special{pa   217  -352}%
\special{pa   208  -340}\special{pa   199  -328}\special{pa   190  -315}\special{pa   181  -303}%
\special{pa   172  -290}\special{pa   163  -278}\special{pa   154  -265}\special{pa   146  -252}%
\special{pa   137  -239}\special{pa   129  -227}\special{pa   121  -214}\special{pa   112  -201}%
\special{pa   104  -188}\special{pa    96  -175}\special{pa    88  -162}\special{pa    80  -148}%
\special{pa    73  -135}\special{pa    65  -122}\special{pa    57  -109}\special{pa    50   -95}%
\special{pa    42   -82}\special{pa    35   -68}\special{pa    28   -55}\special{pa    21   -41}%
\special{pa    14   -27}\special{pa     7   -14}\special{pa    -0     0}%
\special{fp}%
\settowidth{\Width}{$z=a+bi$}\setlength{\Width}{0\Width}%
\settoheight{\Height}{$z=a+bi$}\settodepth{\Depth}{$z=a+bi$}\setlength{\Height}{\Depth}%
\put(3.2700000,2.8700000){\hspace*{\Width}\raisebox{\Height}{$z=a+bi$}}%
%
\special{pa  1270   -20}\special{pa  1270    20}%
\special{fp}%
\settowidth{\Width}{$a$}\setlength{\Width}{-0.5\Width}%
\settoheight{\Height}{$a$}\settodepth{\Depth}{$a$}\setlength{\Height}{-\Height}%
\put(3.2248800,-0.1000000){\hspace*{\Width}\raisebox{\Height}{$a$}}%
%
%
\special{pa    20 -1110}\special{pa   -20 -1110}%
\special{fp}%
\settowidth{\Width}{$b$}\setlength{\Width}{-1\Width}%
\settoheight{\Height}{$b$}\settodepth{\Depth}{$b$}\setlength{\Height}{-0.5\Height}\setlength{\Depth}{0.5\Depth}\addtolength{\Height}{\Depth}%
\put(-0.1000000,2.8191000){\hspace*{\Width}\raisebox{\Height}{$b$}}%
%
%
\special{pa  -394    -0}\special{pa  1575    -0}%
\special{fp}%
\special{pa     0   394}\special{pa     0 -1575}%
\special{fp}%
\settowidth{\Width}{$x$}\setlength{\Width}{0\Width}%
\settoheight{\Height}{$x$}\settodepth{\Depth}{$x$}\setlength{\Height}{-0.5\Height}\setlength{\Depth}{0.5\Depth}\addtolength{\Height}{\Depth}%
\put(4.0500000,0.0000000){\hspace*{\Width}\raisebox{\Height}{$x$}}%
%
\settowidth{\Width}{$y$}\setlength{\Width}{-0.5\Width}%
\settoheight{\Height}{$y$}\settodepth{\Depth}{$y$}\setlength{\Height}{\Depth}%
\put(0.0000000,4.0500000){\hspace*{\Width}\raisebox{\Height}{$y$}}%
%
\settowidth{\Width}{O}\setlength{\Width}{-1\Width}%
\settoheight{\Height}{O}\settodepth{\Depth}{O}\setlength{\Height}{-\Height}%
\put(-0.0500000,-0.0500000){\hspace*{\Width}\raisebox{\Height}{O}}%
%
\end{picture}}%}
\end{layer}

\begin{itemize}
\item
$z=a+bi$の絶対値\ $r$,偏角\ $\theta$
\item
$\cos\theta=\bunsuu{a}{r}$より $a=r\cos\theta$
\item
$\sin\theta=\bunsuu{b}{r}$より $b=r\sin\theta$
\item
{\color{red}$z=r(\cos\theta+\sin\theta)$(極形式)}
\end{itemize}

\sameslide

\vspace*{18mm}

\slidepage

\begin{layer}{120}{0}
\putnotese{75}{17}{%%% /Users/takatoosetsuo/Dropbox/2018polytec/lecture/0611/presen/fig/absangle.tex 
%%% Generator=presen0611.cdy 
{\unitlength=1cm%
\begin{picture}%
(5,5)(-1,-1)%
\special{pn 8}%
%
\Large\bf\boldmath%
\small%
\special{pn 12}%
\special{pa     0    -0}\special{pa  1270 -1110}%
\special{fp}%
\special{pn 8}%
\special{pa 1270 0}\special{pa 1270 -39}\special{fp}\special{pa 1270 -78}\special{pa 1270 -117}\special{fp}%
\special{pa 1270 -156}\special{pa 1270 -195}\special{fp}\special{pa 1270 -234}\special{pa 1270 -273}\special{fp}%
\special{pa 1270 -312}\special{pa 1270 -351}\special{fp}\special{pa 1270 -390}\special{pa 1270 -429}\special{fp}%
\special{pa 1270 -468}\special{pa 1270 -507}\special{fp}\special{pa 1270 -546}\special{pa 1270 -585}\special{fp}%
\special{pa 1270 -624}\special{pa 1270 -663}\special{fp}\special{pa 1270 -702}\special{pa 1270 -741}\special{fp}%
\special{pa 1270 -780}\special{pa 1270 -819}\special{fp}\special{pa 1270 -858}\special{pa 1270 -897}\special{fp}%
\special{pa 1270 -936}\special{pa 1270 -975}\special{fp}\special{pa 1270 -1014}\special{pa 1270 -1053}\special{fp}%
\special{pa 1270 -1092}\special{pa 1270 -1110}\special{pa 1248 -1110}\special{fp}%
\special{pa 1209 -1110}\special{pa 1170 -1110}\special{fp}\special{pa 1131 -1110}\special{pa 1092 -1110}\special{fp}%
\special{pa 1053 -1110}\special{pa 1014 -1110}\special{fp}\special{pa 975 -1110}\special{pa 936 -1110}\special{fp}%
\special{pa 897 -1110}\special{pa 858 -1110}\special{fp}\special{pa 819 -1110}\special{pa 780 -1110}\special{fp}%
\special{pa 741 -1110}\special{pa 702 -1110}\special{fp}\special{pa 663 -1110}\special{pa 624 -1110}\special{fp}%
\special{pa 585 -1110}\special{pa 546 -1110}\special{fp}\special{pa 507 -1110}\special{pa 468 -1110}\special{fp}%
\special{pa 429 -1110}\special{pa 390 -1110}\special{fp}\special{pa 351 -1110}\special{pa 312 -1110}\special{fp}%
\special{pa 273 -1110}\special{pa 234 -1110}\special{fp}\special{pa 195 -1110}\special{pa 156 -1110}\special{fp}%
\special{pa 117 -1110}\special{pa 78 -1110}\special{fp}\special{pa 39 -1110}\special{pa 0 -1110}\special{fp}%
%
%
\settowidth{\Width}{$\theta$}\setlength{\Width}{-0.5\Width}%
\settoheight{\Height}{$\theta$}\settodepth{\Depth}{$\theta$}\setlength{\Height}{-0.5\Height}\setlength{\Depth}{0.5\Depth}\addtolength{\Height}{\Depth}%
\put(0.6100000,0.2300000){\hspace*{\Width}\raisebox{\Height}{$\theta$}}%
%
\special{pa   197    -0}\special{pa   195   -25}\special{pa   191   -49}\special{pa   183   -72}%
\special{pa   173   -95}\special{pa   159  -116}\special{pa   148  -129}%
\special{fp}%
\settowidth{\Width}{$|z|$}\setlength{\Width}{-0.5\Width}%
\settoheight{\Height}{$|z|$}\settodepth{\Depth}{$|z|$}\setlength{\Height}{-0.5\Height}\setlength{\Depth}{0.5\Depth}\addtolength{\Height}{\Depth}%
\put(1.3300000,1.7300000){\hspace*{\Width}\raisebox{\Height}{$|z|$}}%
%
\special{pa  1270 -1110}\special{pa  1255 -1105}\special{pa  1241 -1100}\special{pa  1226 -1095}%
\special{pa  1212 -1090}\special{pa  1197 -1084}\special{pa  1183 -1079}\special{pa  1169 -1073}%
\special{pa  1154 -1068}\special{pa  1140 -1062}\special{pa  1126 -1056}\special{pa  1112 -1050}%
\special{pa  1098 -1044}\special{pa  1084 -1038}\special{pa  1070 -1032}\special{pa  1056 -1025}%
\special{pa  1042 -1019}\special{pa  1028 -1012}\special{pa  1014 -1006}\special{pa  1000  -999}%
\special{pa   986  -992}\special{pa   973  -985}\special{pa   959  -978}\special{pa   946  -971}%
\special{pa   932  -964}\special{pa   919  -957}\special{pa   905  -949}\special{pa   892  -942}%
\special{pa   878  -934}\special{pa   865  -927}\special{pa   852  -919}\special{pa   839  -911}%
\special{pa   825  -903}\special{pa   812  -895}\special{pa   799  -887}\special{pa   786  -879}%
\special{pa   774  -871}\special{pa   761  -862}\special{pa   748  -854}\special{pa   735  -845}%
\special{pa   723  -836}\special{pa   710  -828}\special{pa   697  -819}\special{pa   685  -810}%
\special{pa   672  -801}\special{pa   660  -792}\special{pa   648  -783}\special{pa   636  -773}%
\special{pa   623  -764}\special{pa   611  -755}\special{pa   599  -745}%
\special{fp}%
\special{pa   451  -615}\special{pa   440  -605}\special{pa   429  -594}\special{pa   418  -583}%
\special{pa   407  -572}\special{pa   397  -562}\special{pa   386  -550}\special{pa   375  -539}%
\special{pa   365  -528}\special{pa   355  -517}\special{pa   344  -506}\special{pa   334  -494}%
\special{pa   324  -483}\special{pa   314  -471}\special{pa   304  -460}\special{pa   294  -448}%
\special{pa   284  -436}\special{pa   274  -424}\special{pa   264  -413}\special{pa   255  -401}%
\special{pa   245  -389}\special{pa   236  -377}\special{pa   226  -364}\special{pa   217  -352}%
\special{pa   208  -340}\special{pa   199  -328}\special{pa   190  -315}\special{pa   181  -303}%
\special{pa   172  -290}\special{pa   163  -278}\special{pa   154  -265}\special{pa   146  -252}%
\special{pa   137  -239}\special{pa   129  -227}\special{pa   121  -214}\special{pa   112  -201}%
\special{pa   104  -188}\special{pa    96  -175}\special{pa    88  -162}\special{pa    80  -148}%
\special{pa    73  -135}\special{pa    65  -122}\special{pa    57  -109}\special{pa    50   -95}%
\special{pa    42   -82}\special{pa    35   -68}\special{pa    28   -55}\special{pa    21   -41}%
\special{pa    14   -27}\special{pa     7   -14}\special{pa    -0     0}%
\special{fp}%
\settowidth{\Width}{$z=a+bi$}\setlength{\Width}{0\Width}%
\settoheight{\Height}{$z=a+bi$}\settodepth{\Depth}{$z=a+bi$}\setlength{\Height}{\Depth}%
\put(3.2700000,2.8700000){\hspace*{\Width}\raisebox{\Height}{$z=a+bi$}}%
%
\special{pa  1270   -20}\special{pa  1270    20}%
\special{fp}%
\settowidth{\Width}{$a$}\setlength{\Width}{-0.5\Width}%
\settoheight{\Height}{$a$}\settodepth{\Depth}{$a$}\setlength{\Height}{-\Height}%
\put(3.2248800,-0.1000000){\hspace*{\Width}\raisebox{\Height}{$a$}}%
%
%
\special{pa    20 -1110}\special{pa   -20 -1110}%
\special{fp}%
\settowidth{\Width}{$b$}\setlength{\Width}{-1\Width}%
\settoheight{\Height}{$b$}\settodepth{\Depth}{$b$}\setlength{\Height}{-0.5\Height}\setlength{\Depth}{0.5\Depth}\addtolength{\Height}{\Depth}%
\put(-0.1000000,2.8191000){\hspace*{\Width}\raisebox{\Height}{$b$}}%
%
%
\special{pa  -394    -0}\special{pa  1575    -0}%
\special{fp}%
\special{pa     0   394}\special{pa     0 -1575}%
\special{fp}%
\settowidth{\Width}{$x$}\setlength{\Width}{0\Width}%
\settoheight{\Height}{$x$}\settodepth{\Depth}{$x$}\setlength{\Height}{-0.5\Height}\setlength{\Depth}{0.5\Depth}\addtolength{\Height}{\Depth}%
\put(4.0500000,0.0000000){\hspace*{\Width}\raisebox{\Height}{$x$}}%
%
\settowidth{\Width}{$y$}\setlength{\Width}{-0.5\Width}%
\settoheight{\Height}{$y$}\settodepth{\Depth}{$y$}\setlength{\Height}{\Depth}%
\put(0.0000000,4.0500000){\hspace*{\Width}\raisebox{\Height}{$y$}}%
%
\settowidth{\Width}{O}\setlength{\Width}{-1\Width}%
\settoheight{\Height}{O}\settodepth{\Depth}{O}\setlength{\Height}{-\Height}%
\put(-0.0500000,-0.0500000){\hspace*{\Width}\raisebox{\Height}{O}}%
%
\end{picture}}%}
\end{layer}

\begin{itemize}
\item
$z=a+bi$の絶対値\ $r$,偏角\ $\theta$
\item
$\cos\theta=\bunsuu{a}{r}$より $a=r\cos\theta$
\item
$\sin\theta=\bunsuu{b}{r}$より $b=r\sin\theta$
\item
{\color{red}$z=r(\cos\theta+\sin\theta)$(極形式)}
\item
[例)]$\sqrt{3}+i=2(\cos\bunsuu{\pi}{6}+\sin\bunsuu{\pi}{6})$
\end{itemize}

\newslide{複素数の積$zw$と図形}

\vspace*{18mm}

\slidepage

\begin{layer}{120}{0}
\putnotese{72}{27}{\scalebox{0.8}{%%% /Users/takatoosetsuo/Dropbox/2019polytec/lectures/0624/presen/fig/complexproduct.tex 
%%% Generator=complexproduct.cdy 
{\unitlength=6mm%
\begin{picture}%
(10,10)(-2,-2)%
\special{pn 8}%
%
\special{pa     0    -0}\special{pa   706  -146}%
\special{fp}%
\special{pa     0    -0}\special{pa   293  -356}%
\special{fp}%
\special{pa     0    -0}\special{pa   654 -1245}%
\special{fp}%
{%
\color[rgb]{1,0,0}%
{\special{pn 0}\color[rgb]{1,0,0}%
\special{pa 723 -163}\special{pa 715 -168}\special{pa 707 -170}\special{pa 698 -169}\special{pa 691 -164}\special{pa 685 -158}\special{pa 683 -150}\special{pa 683 -141}\special{pa 686 -133}\special{pa 693 -127}\special{pa 701 -123}\special{pa 709 -123}\special{pa 717 -126}\special{pa 724 -131}\special{pa 728 -139}\special{pa 730 -147}\special{pa 728 -156}\special{pa 723 -163}\special{sh}\special{fp}}%
\special{pn 4}\special{pa 723 -163}\special{pa 715 -168}\special{pa 707 -170}\special{pa 698 -169}\special{pa 691 -164}\special{pa 685 -158}\special{pa 683 -150}\special{pa 683 -141}\special{pa 686 -133}\special{pa 693 -127}\special{pa 701 -123}\special{pa 709 -123}\special{pa 717 -126}\special{pa 724 -131}\special{pa 728 -139}\special{pa 730 -147}\special{pa 728 -156}\special{pa 723 -163}\special{fp}%
{\special{pn 0}\color[rgb]{1,0,0}%
\special{pa 309 -373}\special{pa 302 -378}\special{pa 294 -380}\special{pa 285 -379}\special{pa 277 -374}\special{pa 272 -368}\special{pa 269 -359}\special{pa 270 -351}\special{pa 273 -343}\special{pa 279 -337}\special{pa 287 -333}\special{pa 296 -333}\special{pa 304 -336}\special{pa 311 -341}\special{pa 315 -349}\special{pa 316 -357}\special{pa 314 -366}\special{pa 309 -373}\special{sh}\special{fp}}%
\special{pn 4}\special{pa 309 -373}\special{pa 302 -378}\special{pa 294 -380}\special{pa 285 -379}\special{pa 277 -374}\special{pa 272 -368}\special{pa 269 -359}\special{pa 270 -351}\special{pa 273 -343}\special{pa 279 -337}\special{pa 287 -333}\special{pa 296 -333}\special{pa 304 -336}\special{pa 311 -341}\special{pa 315 -349}\special{pa 316 -357}\special{pa 314 -366}\special{pa 309 -373}\special{fp}%
{\special{pn 0}\color[rgb]{1,0,0}%
\special{pa 670 -1262}\special{pa 663 -1267}\special{pa 655 -1269}\special{pa 646 -1268}\special{pa 639 -1264}\special{pa 633 -1257}\special{pa 630 -1249}\special{pa 631 -1240}\special{pa 634 -1232}\special{pa 640 -1226}\special{pa 648 -1222}\special{pa 657 -1222}\special{pa 665 -1225}\special{pa 672 -1230}\special{pa 676 -1238}\special{pa 677 -1247}\special{pa 675 -1255}\special{pa 670 -1262}\special{sh}\special{fp}}%
\special{pn 4}\special{pa 670 -1262}\special{pa 663 -1267}\special{pa 655 -1269}\special{pa 646 -1268}\special{pa 639 -1264}\special{pa 633 -1257}\special{pa 630 -1249}\special{pa 631 -1240}\special{pa 634 -1232}\special{pa 640 -1226}\special{pa 648 -1222}\special{pa 657 -1222}\special{pa 665 -1225}\special{pa 672 -1230}\special{pa 676 -1238}\special{pa 677 -1247}\special{pa 675 -1255}\special{pa 670 -1262}\special{fp}%
}%
\settowidth{\Width}{$z$}\setlength{\Width}{0\Width}%
\settoheight{\Height}{$z$}\settodepth{\Depth}{$z$}\setlength{\Height}{\Depth}%
\put(3.1566667,0.7033333){\hspace*{\Width}\raisebox{\Height}{$z$}}%
%
\settowidth{\Width}{$w$}\setlength{\Width}{0\Width}%
\settoheight{\Height}{$w$}\settodepth{\Depth}{$w$}\setlength{\Height}{\Depth}%
\put(1.4066667,1.5933333){\hspace*{\Width}\raisebox{\Height}{$w$}}%
%
\settowidth{\Width}{$zw$}\setlength{\Width}{0\Width}%
\settoheight{\Height}{$zw$}\settodepth{\Depth}{$zw$}\setlength{\Height}{\Depth}%
\put(2.9366667,5.4366667){\hspace*{\Width}\raisebox{\Height}{$zw$}}%
%
\special{pa  -472    -0}\special{pa  1890    -0}%
\special{fp}%
\special{pa     0   472}\special{pa     0 -1890}%
\special{fp}%
\settowidth{\Width}{$x$}\setlength{\Width}{0\Width}%
\settoheight{\Height}{$x$}\settodepth{\Depth}{$x$}\setlength{\Height}{-0.5\Height}\setlength{\Depth}{0.5\Depth}\addtolength{\Height}{\Depth}%
\put(8.0833333,0.0000000){\hspace*{\Width}\raisebox{\Height}{$x$}}%
%
\settowidth{\Width}{$y$}\setlength{\Width}{-0.5\Width}%
\settoheight{\Height}{$y$}\settodepth{\Depth}{$y$}\setlength{\Height}{\Depth}%
\put(0.0000000,8.0833333){\hspace*{\Width}\raisebox{\Height}{$y$}}%
%
\settowidth{\Width}{O}\setlength{\Width}{-1\Width}%
\settoheight{\Height}{O}\settodepth{\Depth}{O}\setlength{\Height}{-\Height}%
\put(-0.0833333,-0.0833333){\hspace*{\Width}\raisebox{\Height}{O}}%
%
\end{picture}}%}}
\end{layer}

\begin{itemize}
\item
$z=a+bi,\ w=c+di$($a,b,c,d$は実数)
\end{itemize}
%%%%%%%%%%%%%

%%%%%%%%%%%%%%%%%%%%


\sameslide

\vspace*{18mm}

\slidepage

\begin{layer}{120}{0}
\putnotese{72}{27}{\scalebox{0.8}{%%% /Users/takatoosetsuo/Dropbox/2019polytec/lectures/0624/presen/fig/complexproduct.tex 
%%% Generator=complexproduct.cdy 
{\unitlength=6mm%
\begin{picture}%
(10,10)(-2,-2)%
\special{pn 8}%
%
\special{pa     0    -0}\special{pa   706  -146}%
\special{fp}%
\special{pa     0    -0}\special{pa   293  -356}%
\special{fp}%
\special{pa     0    -0}\special{pa   654 -1245}%
\special{fp}%
{%
\color[rgb]{1,0,0}%
{\special{pn 0}\color[rgb]{1,0,0}%
\special{pa 723 -163}\special{pa 715 -168}\special{pa 707 -170}\special{pa 698 -169}\special{pa 691 -164}\special{pa 685 -158}\special{pa 683 -150}\special{pa 683 -141}\special{pa 686 -133}\special{pa 693 -127}\special{pa 701 -123}\special{pa 709 -123}\special{pa 717 -126}\special{pa 724 -131}\special{pa 728 -139}\special{pa 730 -147}\special{pa 728 -156}\special{pa 723 -163}\special{sh}\special{fp}}%
\special{pn 4}\special{pa 723 -163}\special{pa 715 -168}\special{pa 707 -170}\special{pa 698 -169}\special{pa 691 -164}\special{pa 685 -158}\special{pa 683 -150}\special{pa 683 -141}\special{pa 686 -133}\special{pa 693 -127}\special{pa 701 -123}\special{pa 709 -123}\special{pa 717 -126}\special{pa 724 -131}\special{pa 728 -139}\special{pa 730 -147}\special{pa 728 -156}\special{pa 723 -163}\special{fp}%
{\special{pn 0}\color[rgb]{1,0,0}%
\special{pa 309 -373}\special{pa 302 -378}\special{pa 294 -380}\special{pa 285 -379}\special{pa 277 -374}\special{pa 272 -368}\special{pa 269 -359}\special{pa 270 -351}\special{pa 273 -343}\special{pa 279 -337}\special{pa 287 -333}\special{pa 296 -333}\special{pa 304 -336}\special{pa 311 -341}\special{pa 315 -349}\special{pa 316 -357}\special{pa 314 -366}\special{pa 309 -373}\special{sh}\special{fp}}%
\special{pn 4}\special{pa 309 -373}\special{pa 302 -378}\special{pa 294 -380}\special{pa 285 -379}\special{pa 277 -374}\special{pa 272 -368}\special{pa 269 -359}\special{pa 270 -351}\special{pa 273 -343}\special{pa 279 -337}\special{pa 287 -333}\special{pa 296 -333}\special{pa 304 -336}\special{pa 311 -341}\special{pa 315 -349}\special{pa 316 -357}\special{pa 314 -366}\special{pa 309 -373}\special{fp}%
{\special{pn 0}\color[rgb]{1,0,0}%
\special{pa 670 -1262}\special{pa 663 -1267}\special{pa 655 -1269}\special{pa 646 -1268}\special{pa 639 -1264}\special{pa 633 -1257}\special{pa 630 -1249}\special{pa 631 -1240}\special{pa 634 -1232}\special{pa 640 -1226}\special{pa 648 -1222}\special{pa 657 -1222}\special{pa 665 -1225}\special{pa 672 -1230}\special{pa 676 -1238}\special{pa 677 -1247}\special{pa 675 -1255}\special{pa 670 -1262}\special{sh}\special{fp}}%
\special{pn 4}\special{pa 670 -1262}\special{pa 663 -1267}\special{pa 655 -1269}\special{pa 646 -1268}\special{pa 639 -1264}\special{pa 633 -1257}\special{pa 630 -1249}\special{pa 631 -1240}\special{pa 634 -1232}\special{pa 640 -1226}\special{pa 648 -1222}\special{pa 657 -1222}\special{pa 665 -1225}\special{pa 672 -1230}\special{pa 676 -1238}\special{pa 677 -1247}\special{pa 675 -1255}\special{pa 670 -1262}\special{fp}%
}%
\settowidth{\Width}{$z$}\setlength{\Width}{0\Width}%
\settoheight{\Height}{$z$}\settodepth{\Depth}{$z$}\setlength{\Height}{\Depth}%
\put(3.1566667,0.7033333){\hspace*{\Width}\raisebox{\Height}{$z$}}%
%
\settowidth{\Width}{$w$}\setlength{\Width}{0\Width}%
\settoheight{\Height}{$w$}\settodepth{\Depth}{$w$}\setlength{\Height}{\Depth}%
\put(1.4066667,1.5933333){\hspace*{\Width}\raisebox{\Height}{$w$}}%
%
\settowidth{\Width}{$zw$}\setlength{\Width}{0\Width}%
\settoheight{\Height}{$zw$}\settodepth{\Depth}{$zw$}\setlength{\Height}{\Depth}%
\put(2.9366667,5.4366667){\hspace*{\Width}\raisebox{\Height}{$zw$}}%
%
\special{pa  -472    -0}\special{pa  1890    -0}%
\special{fp}%
\special{pa     0   472}\special{pa     0 -1890}%
\special{fp}%
\settowidth{\Width}{$x$}\setlength{\Width}{0\Width}%
\settoheight{\Height}{$x$}\settodepth{\Depth}{$x$}\setlength{\Height}{-0.5\Height}\setlength{\Depth}{0.5\Depth}\addtolength{\Height}{\Depth}%
\put(8.0833333,0.0000000){\hspace*{\Width}\raisebox{\Height}{$x$}}%
%
\settowidth{\Width}{$y$}\setlength{\Width}{-0.5\Width}%
\settoheight{\Height}{$y$}\settodepth{\Depth}{$y$}\setlength{\Height}{\Depth}%
\put(0.0000000,8.0833333){\hspace*{\Width}\raisebox{\Height}{$y$}}%
%
\settowidth{\Width}{O}\setlength{\Width}{-1\Width}%
\settoheight{\Height}{O}\settodepth{\Depth}{O}\setlength{\Height}{-\Height}%
\put(-0.0833333,-0.0833333){\hspace*{\Width}\raisebox{\Height}{O}}%
%
\end{picture}}%}}
\end{layer}

\begin{itemize}
\item
$z=a+bi,\ w=c+di$($a,b,c,d$は実数)
\item
$zw\!=\!(a+bi)(c+di)\!=\!(ac-bd)+(ad+bc)i$
\end{itemize}

\sameslide

\vspace*{18mm}

\slidepage

\begin{layer}{120}{0}
\putnotese{72}{27}{\scalebox{0.8}{%%% /Users/takatoosetsuo/Dropbox/2019polytec/lectures/0624/presen/fig/complexproduct.tex 
%%% Generator=complexproduct.cdy 
{\unitlength=6mm%
\begin{picture}%
(10,10)(-2,-2)%
\special{pn 8}%
%
\special{pa     0    -0}\special{pa   706  -146}%
\special{fp}%
\special{pa     0    -0}\special{pa   293  -356}%
\special{fp}%
\special{pa     0    -0}\special{pa   654 -1245}%
\special{fp}%
{%
\color[rgb]{1,0,0}%
{\special{pn 0}\color[rgb]{1,0,0}%
\special{pa 723 -163}\special{pa 715 -168}\special{pa 707 -170}\special{pa 698 -169}\special{pa 691 -164}\special{pa 685 -158}\special{pa 683 -150}\special{pa 683 -141}\special{pa 686 -133}\special{pa 693 -127}\special{pa 701 -123}\special{pa 709 -123}\special{pa 717 -126}\special{pa 724 -131}\special{pa 728 -139}\special{pa 730 -147}\special{pa 728 -156}\special{pa 723 -163}\special{sh}\special{fp}}%
\special{pn 4}\special{pa 723 -163}\special{pa 715 -168}\special{pa 707 -170}\special{pa 698 -169}\special{pa 691 -164}\special{pa 685 -158}\special{pa 683 -150}\special{pa 683 -141}\special{pa 686 -133}\special{pa 693 -127}\special{pa 701 -123}\special{pa 709 -123}\special{pa 717 -126}\special{pa 724 -131}\special{pa 728 -139}\special{pa 730 -147}\special{pa 728 -156}\special{pa 723 -163}\special{fp}%
{\special{pn 0}\color[rgb]{1,0,0}%
\special{pa 309 -373}\special{pa 302 -378}\special{pa 294 -380}\special{pa 285 -379}\special{pa 277 -374}\special{pa 272 -368}\special{pa 269 -359}\special{pa 270 -351}\special{pa 273 -343}\special{pa 279 -337}\special{pa 287 -333}\special{pa 296 -333}\special{pa 304 -336}\special{pa 311 -341}\special{pa 315 -349}\special{pa 316 -357}\special{pa 314 -366}\special{pa 309 -373}\special{sh}\special{fp}}%
\special{pn 4}\special{pa 309 -373}\special{pa 302 -378}\special{pa 294 -380}\special{pa 285 -379}\special{pa 277 -374}\special{pa 272 -368}\special{pa 269 -359}\special{pa 270 -351}\special{pa 273 -343}\special{pa 279 -337}\special{pa 287 -333}\special{pa 296 -333}\special{pa 304 -336}\special{pa 311 -341}\special{pa 315 -349}\special{pa 316 -357}\special{pa 314 -366}\special{pa 309 -373}\special{fp}%
{\special{pn 0}\color[rgb]{1,0,0}%
\special{pa 670 -1262}\special{pa 663 -1267}\special{pa 655 -1269}\special{pa 646 -1268}\special{pa 639 -1264}\special{pa 633 -1257}\special{pa 630 -1249}\special{pa 631 -1240}\special{pa 634 -1232}\special{pa 640 -1226}\special{pa 648 -1222}\special{pa 657 -1222}\special{pa 665 -1225}\special{pa 672 -1230}\special{pa 676 -1238}\special{pa 677 -1247}\special{pa 675 -1255}\special{pa 670 -1262}\special{sh}\special{fp}}%
\special{pn 4}\special{pa 670 -1262}\special{pa 663 -1267}\special{pa 655 -1269}\special{pa 646 -1268}\special{pa 639 -1264}\special{pa 633 -1257}\special{pa 630 -1249}\special{pa 631 -1240}\special{pa 634 -1232}\special{pa 640 -1226}\special{pa 648 -1222}\special{pa 657 -1222}\special{pa 665 -1225}\special{pa 672 -1230}\special{pa 676 -1238}\special{pa 677 -1247}\special{pa 675 -1255}\special{pa 670 -1262}\special{fp}%
}%
\settowidth{\Width}{$z$}\setlength{\Width}{0\Width}%
\settoheight{\Height}{$z$}\settodepth{\Depth}{$z$}\setlength{\Height}{\Depth}%
\put(3.1566667,0.7033333){\hspace*{\Width}\raisebox{\Height}{$z$}}%
%
\settowidth{\Width}{$w$}\setlength{\Width}{0\Width}%
\settoheight{\Height}{$w$}\settodepth{\Depth}{$w$}\setlength{\Height}{\Depth}%
\put(1.4066667,1.5933333){\hspace*{\Width}\raisebox{\Height}{$w$}}%
%
\settowidth{\Width}{$zw$}\setlength{\Width}{0\Width}%
\settoheight{\Height}{$zw$}\settodepth{\Depth}{$zw$}\setlength{\Height}{\Depth}%
\put(2.9366667,5.4366667){\hspace*{\Width}\raisebox{\Height}{$zw$}}%
%
\special{pa  -472    -0}\special{pa  1890    -0}%
\special{fp}%
\special{pa     0   472}\special{pa     0 -1890}%
\special{fp}%
\settowidth{\Width}{$x$}\setlength{\Width}{0\Width}%
\settoheight{\Height}{$x$}\settodepth{\Depth}{$x$}\setlength{\Height}{-0.5\Height}\setlength{\Depth}{0.5\Depth}\addtolength{\Height}{\Depth}%
\put(8.0833333,0.0000000){\hspace*{\Width}\raisebox{\Height}{$x$}}%
%
\settowidth{\Width}{$y$}\setlength{\Width}{-0.5\Width}%
\settoheight{\Height}{$y$}\settodepth{\Depth}{$y$}\setlength{\Height}{\Depth}%
\put(0.0000000,8.0833333){\hspace*{\Width}\raisebox{\Height}{$y$}}%
%
\settowidth{\Width}{O}\setlength{\Width}{-1\Width}%
\settoheight{\Height}{O}\settodepth{\Depth}{O}\setlength{\Height}{-\Height}%
\put(-0.0833333,-0.0833333){\hspace*{\Width}\raisebox{\Height}{O}}%
%
\end{picture}}%}}
\end{layer}

\begin{itemize}
\item
$z=a+bi,\ w=c+di$($a,b,c,d$は実数)
\item
$zw\!=\!(a+bi)(c+di)\!=\!(ac-bd)+(ad+bc)i$
\item
$z,w,zw$の位置関係は?
\end{itemize}
\hspace*{3zw}\href{run:complexproductjsoff.html}{複素数の積}

\sameslide

\vspace*{18mm}

\slidepage

\begin{layer}{120}{0}
\putnotese{72}{27}{\scalebox{0.8}{%%% /Users/takatoosetsuo/Dropbox/2019polytec/lectures/0624/presen/fig/complexproduct2.tex 
%%% Generator=complexproduct.cdy 
{\unitlength=6mm%
\begin{picture}%
(10,10)(-2,-2)%
\special{pn 8}%
%
\special{pa     0    -0}\special{pa   706  -146}%
\special{fp}%
\special{pa     0    -0}\special{pa   293  -356}%
\special{fp}%
\special{pa     0    -0}\special{pa   654 -1245}%
\special{fp}%
{%
\color[rgb]{1,0,0}%
{\special{pn 0}\color[rgb]{1,0,0}%
\special{pa 723 -163}\special{pa 715 -168}\special{pa 707 -170}\special{pa 698 -169}\special{pa 691 -164}\special{pa 685 -158}\special{pa 683 -150}\special{pa 683 -141}\special{pa 686 -133}\special{pa 693 -127}\special{pa 701 -123}\special{pa 709 -123}\special{pa 717 -126}\special{pa 724 -131}\special{pa 728 -139}\special{pa 730 -147}\special{pa 728 -156}\special{pa 723 -163}\special{sh}\special{fp}}%
\special{pn 4}\special{pa 723 -163}\special{pa 715 -168}\special{pa 707 -170}\special{pa 698 -169}\special{pa 691 -164}\special{pa 685 -158}\special{pa 683 -150}\special{pa 683 -141}\special{pa 686 -133}\special{pa 693 -127}\special{pa 701 -123}\special{pa 709 -123}\special{pa 717 -126}\special{pa 724 -131}\special{pa 728 -139}\special{pa 730 -147}\special{pa 728 -156}\special{pa 723 -163}\special{fp}%
{\special{pn 0}\color[rgb]{1,0,0}%
\special{pa 309 -373}\special{pa 302 -378}\special{pa 294 -380}\special{pa 285 -379}\special{pa 277 -374}\special{pa 272 -368}\special{pa 269 -359}\special{pa 270 -351}\special{pa 273 -343}\special{pa 279 -337}\special{pa 287 -333}\special{pa 296 -333}\special{pa 304 -336}\special{pa 311 -341}\special{pa 315 -349}\special{pa 316 -357}\special{pa 314 -366}\special{pa 309 -373}\special{sh}\special{fp}}%
\special{pn 4}\special{pa 309 -373}\special{pa 302 -378}\special{pa 294 -380}\special{pa 285 -379}\special{pa 277 -374}\special{pa 272 -368}\special{pa 269 -359}\special{pa 270 -351}\special{pa 273 -343}\special{pa 279 -337}\special{pa 287 -333}\special{pa 296 -333}\special{pa 304 -336}\special{pa 311 -341}\special{pa 315 -349}\special{pa 316 -357}\special{pa 314 -366}\special{pa 309 -373}\special{fp}%
{\special{pn 0}\color[rgb]{1,0,0}%
\special{pa 670 -1262}\special{pa 663 -1267}\special{pa 655 -1269}\special{pa 646 -1268}\special{pa 639 -1264}\special{pa 633 -1257}\special{pa 630 -1249}\special{pa 631 -1240}\special{pa 634 -1232}\special{pa 640 -1226}\special{pa 648 -1222}\special{pa 657 -1222}\special{pa 665 -1225}\special{pa 672 -1230}\special{pa 676 -1238}\special{pa 677 -1247}\special{pa 675 -1255}\special{pa 670 -1262}\special{sh}\special{fp}}%
\special{pn 4}\special{pa 670 -1262}\special{pa 663 -1267}\special{pa 655 -1269}\special{pa 646 -1268}\special{pa 639 -1264}\special{pa 633 -1257}\special{pa 630 -1249}\special{pa 631 -1240}\special{pa 634 -1232}\special{pa 640 -1226}\special{pa 648 -1222}\special{pa 657 -1222}\special{pa 665 -1225}\special{pa 672 -1230}\special{pa 676 -1238}\special{pa 677 -1247}\special{pa 675 -1255}\special{pa 670 -1262}\special{fp}%
}%
\settowidth{\Width}{$z$}\setlength{\Width}{0\Width}%
\settoheight{\Height}{$z$}\settodepth{\Depth}{$z$}\setlength{\Height}{\Depth}%
\put(3.1566667,0.7033333){\hspace*{\Width}\raisebox{\Height}{$z$}}%
%
\settowidth{\Width}{$w$}\setlength{\Width}{0\Width}%
\settoheight{\Height}{$w$}\settodepth{\Depth}{$w$}\setlength{\Height}{\Depth}%
\put(1.4066667,1.5933333){\hspace*{\Width}\raisebox{\Height}{$w$}}%
%
\settowidth{\Width}{$zw$}\setlength{\Width}{0\Width}%
\settoheight{\Height}{$zw$}\settodepth{\Depth}{$zw$}\setlength{\Height}{\Depth}%
\put(2.9366667,5.4366667){\hspace*{\Width}\raisebox{\Height}{$zw$}}%
%
{%
\color[rgb]{0,0,1}%
{\special{pn 0}\color[rgb]{0,0,1}%
\special{pa 248 -65}\special{pa 241 -70}\special{pa 232 -72}\special{pa 224 -70}\special{pa 216 -66}\special{pa 211 -59}\special{pa 208 -51}\special{pa 208 -43}\special{pa 212 -35}\special{pa 218 -28}\special{pa 226 -25}\special{pa 235 -25}\special{pa 243 -27}\special{pa 249 -33}\special{pa 254 -40}\special{pa 255 -49}\special{pa 253 -57}\special{pa 248 -65}\special{sh}\special{fp}}%
\special{pn 4}\special{pa 248 -65}\special{pa 241 -70}\special{pa 232 -72}\special{pa 224 -70}\special{pa 216 -66}\special{pa 211 -59}\special{pa 208 -51}\special{pa 208 -43}\special{pa 212 -35}\special{pa 218 -28}\special{pa 226 -25}\special{pa 235 -25}\special{pa 243 -27}\special{pa 249 -33}\special{pa 254 -40}\special{pa 255 -49}\special{pa 253 -57}\special{pa 248 -65}\special{fp}%
{\special{pn 0}\color[rgb]{0,0,1}%
\special{pa 479 -113}\special{pa 472 -117}\special{pa 464 -119}\special{pa 455 -118}\special{pa 448 -114}\special{pa 442 -107}\special{pa 439 -99}\special{pa 440 -90}\special{pa 443 -83}\special{pa 449 -76}\special{pa 457 -73}\special{pa 466 -72}\special{pa 474 -75}\special{pa 481 -81}\special{pa 485 -88}\special{pa 486 -97}\special{pa 484 -105}\special{pa 479 -113}\special{sh}\special{fp}}%
\special{pn 4}\special{pa 479 -113}\special{pa 472 -117}\special{pa 464 -119}\special{pa 455 -118}\special{pa 448 -114}\special{pa 442 -107}\special{pa 439 -99}\special{pa 440 -90}\special{pa 443 -83}\special{pa 449 -76}\special{pa 457 -73}\special{pa 466 -72}\special{pa 474 -75}\special{pa 481 -81}\special{pa 485 -88}\special{pa 486 -97}\special{pa 484 -105}\special{pa 479 -113}\special{fp}%
{\special{pn 0}\color[rgb]{0,0,1}%
\special{pa 711 -160}\special{pa 703 -165}\special{pa 695 -167}\special{pa 686 -166}\special{pa 679 -162}\special{pa 673 -155}\special{pa 671 -147}\special{pa 671 -138}\special{pa 674 -130}\special{pa 681 -124}\special{pa 689 -121}\special{pa 697 -120}\special{pa 705 -123}\special{pa 712 -129}\special{pa 716 -136}\special{pa 718 -145}\special{pa 716 -153}\special{pa 711 -160}\special{sh}\special{fp}}%
\special{pn 4}\special{pa 711 -160}\special{pa 703 -165}\special{pa 695 -167}\special{pa 686 -166}\special{pa 679 -162}\special{pa 673 -155}\special{pa 671 -147}\special{pa 671 -138}\special{pa 674 -130}\special{pa 681 -124}\special{pa 689 -121}\special{pa 697 -120}\special{pa 705 -123}\special{pa 712 -129}\special{pa 716 -136}\special{pa 718 -145}\special{pa 716 -153}\special{pa 711 -160}\special{fp}%
}%
{%
\color[rgb]{0,1,0}%
{\special{pn 0}\color[rgb]{0,1,0}%
\special{pa 127 -35}\special{pa 121 -39}\special{pa 114 -40}\special{pa 107 -37}\special{pa 102 -32}\special{pa 100 -26}\special{pa 101 -19}\special{pa 105 -13}\special{pa 110 -9}\special{pa 117 -8}\special{pa 124 -11}\special{pa 129 -16}\special{pa 131 -22}\special{pa 131 -29}\special{pa 127 -35}\special{sh}\special{fp}}%
\special{pn 4}\special{pa 127 -35}\special{pa 121 -39}\special{pa 114 -40}\special{pa 107 -37}\special{pa 102 -32}\special{pa 100 -26}\special{pa 101 -19}\special{pa 105 -13}\special{pa 110 -9}\special{pa 117 -8}\special{pa 124 -11}\special{pa 129 -16}\special{pa 131 -22}\special{pa 131 -29}\special{pa 127 -35}\special{fp}%
{\special{pn 0}\color[rgb]{0,1,0}%
\special{pa 358 -83}\special{pa 352 -87}\special{pa 345 -88}\special{pa 339 -85}\special{pa 334 -80}\special{pa 331 -74}\special{pa 332 -67}\special{pa 336 -61}\special{pa 342 -57}\special{pa 349 -56}\special{pa 355 -59}\special{pa 360 -64}\special{pa 363 -70}\special{pa 362 -77}\special{pa 358 -83}\special{sh}\special{fp}}%
\special{pn 4}\special{pa 358 -83}\special{pa 352 -87}\special{pa 345 -88}\special{pa 339 -85}\special{pa 334 -80}\special{pa 331 -74}\special{pa 332 -67}\special{pa 336 -61}\special{pa 342 -57}\special{pa 349 -56}\special{pa 355 -59}\special{pa 360 -64}\special{pa 363 -70}\special{pa 362 -77}\special{pa 358 -83}\special{fp}%
{\special{pn 0}\color[rgb]{0,1,0}%
\special{pa 589 -131}\special{pa 583 -135}\special{pa 577 -135}\special{pa 570 -133}\special{pa 565 -128}\special{pa 563 -122}\special{pa 563 -115}\special{pa 567 -109}\special{pa 573 -105}\special{pa 580 -104}\special{pa 587 -106}\special{pa 592 -111}\special{pa 594 -118}\special{pa 593 -125}\special{pa 589 -131}\special{sh}\special{fp}}%
\special{pn 4}\special{pa 589 -131}\special{pa 583 -135}\special{pa 577 -135}\special{pa 570 -133}\special{pa 565 -128}\special{pa 563 -122}\special{pa 563 -115}\special{pa 567 -109}\special{pa 573 -105}\special{pa 580 -104}\special{pa 587 -106}\special{pa 592 -111}\special{pa 594 -118}\special{pa 593 -125}\special{pa 589 -131}\special{fp}%
}%
{%
\color[rgb]{0,0,1}%
{\special{pn 0}\color[rgb]{0,0,1}%
\special{pa 167 -199}\special{pa 159 -204}\special{pa 151 -206}\special{pa 142 -205}\special{pa 135 -201}\special{pa 129 -194}\special{pa 127 -186}\special{pa 127 -177}\special{pa 130 -169}\special{pa 137 -163}\special{pa 145 -160}\special{pa 153 -159}\special{pa 161 -162}\special{pa 168 -167}\special{pa 172 -175}\special{pa 174 -184}\special{pa 172 -192}\special{pa 167 -199}\special{sh}\special{fp}}%
\special{pn 4}\special{pa 167 -199}\special{pa 159 -204}\special{pa 151 -206}\special{pa 142 -205}\special{pa 135 -201}\special{pa 129 -194}\special{pa 127 -186}\special{pa 127 -177}\special{pa 130 -169}\special{pa 137 -163}\special{pa 145 -160}\special{pa 153 -159}\special{pa 161 -162}\special{pa 168 -167}\special{pa 172 -175}\special{pa 174 -184}\special{pa 172 -192}\special{pa 167 -199}\special{fp}%
}%
{%
\color[rgb]{0,1,0}%
{\special{pn 0}\color[rgb]{0,1,0}%
\special{pa 86 -102}\special{pa 80 -106}\special{pa 73 -107}\special{pa 67 -105}\special{pa 62 -100}\special{pa 59 -93}\special{pa 60 -86}\special{pa 64 -80}\special{pa 70 -76}\special{pa 77 -76}\special{pa 83 -78}\special{pa 88 -83}\special{pa 91 -90}\special{pa 90 -96}\special{pa 86 -102}\special{sh}\special{fp}}%
\special{pn 4}\special{pa 86 -102}\special{pa 80 -106}\special{pa 73 -107}\special{pa 67 -105}\special{pa 62 -100}\special{pa 59 -93}\special{pa 60 -86}\special{pa 64 -80}\special{pa 70 -76}\special{pa 77 -76}\special{pa 83 -78}\special{pa 88 -83}\special{pa 91 -90}\special{pa 90 -96}\special{pa 86 -102}\special{fp}%
{\special{pn 0}\color[rgb]{0,1,0}%
\special{pa 236 -285}\special{pa 230 -289}\special{pa 223 -289}\special{pa 217 -287}\special{pa 212 -282}\special{pa 209 -276}\special{pa 210 -269}\special{pa 214 -263}\special{pa 220 -259}\special{pa 227 -258}\special{pa 233 -260}\special{pa 238 -265}\special{pa 241 -272}\special{pa 240 -279}\special{pa 236 -285}\special{sh}\special{fp}}%
\special{pn 4}\special{pa 236 -285}\special{pa 230 -289}\special{pa 223 -289}\special{pa 217 -287}\special{pa 212 -282}\special{pa 209 -276}\special{pa 210 -269}\special{pa 214 -263}\special{pa 220 -259}\special{pa 227 -258}\special{pa 233 -260}\special{pa 238 -265}\special{pa 241 -272}\special{pa 240 -279}\special{pa 236 -285}\special{fp}%
}%
{%
\color[rgb]{0,0,1}%
{\special{pn 0}\color[rgb]{0,0,1}%
\special{pa 126 -226}\special{pa 119 -231}\special{pa 111 -233}\special{pa 102 -232}\special{pa 95 -227}\special{pa 89 -221}\special{pa 86 -212}\special{pa 87 -204}\special{pa 90 -196}\special{pa 96 -190}\special{pa 104 -186}\special{pa 113 -186}\special{pa 121 -189}\special{pa 128 -194}\special{pa 132 -202}\special{pa 133 -210}\special{pa 131 -219}\special{pa 126 -226}\special{sh}\special{fp}}%
\special{pn 4}\special{pa 126 -226}\special{pa 119 -231}\special{pa 111 -233}\special{pa 102 -232}\special{pa 95 -227}\special{pa 89 -221}\special{pa 86 -212}\special{pa 87 -204}\special{pa 90 -196}\special{pa 96 -190}\special{pa 104 -186}\special{pa 113 -186}\special{pa 121 -189}\special{pa 128 -194}\special{pa 132 -202}\special{pa 133 -210}\special{pa 131 -219}\special{pa 126 -226}\special{fp}%
{\special{pn 0}\color[rgb]{0,0,1}%
\special{pa 236 -435}\special{pa 229 -440}\special{pa 221 -442}\special{pa 212 -441}\special{pa 204 -436}\special{pa 199 -430}\special{pa 196 -422}\special{pa 197 -413}\special{pa 200 -405}\special{pa 206 -399}\special{pa 214 -395}\special{pa 223 -395}\special{pa 231 -398}\special{pa 238 -403}\special{pa 242 -411}\special{pa 243 -419}\special{pa 241 -428}\special{pa 236 -435}\special{sh}\special{fp}}%
\special{pn 4}\special{pa 236 -435}\special{pa 229 -440}\special{pa 221 -442}\special{pa 212 -441}\special{pa 204 -436}\special{pa 199 -430}\special{pa 196 -422}\special{pa 197 -413}\special{pa 200 -405}\special{pa 206 -399}\special{pa 214 -395}\special{pa 223 -395}\special{pa 231 -398}\special{pa 238 -403}\special{pa 242 -411}\special{pa 243 -419}\special{pa 241 -428}\special{pa 236 -435}\special{fp}%
{\special{pn 0}\color[rgb]{0,0,1}%
\special{pa 346 -644}\special{pa 339 -649}\special{pa 330 -651}\special{pa 322 -650}\special{pa 314 -646}\special{pa 309 -639}\special{pa 306 -631}\special{pa 306 -622}\special{pa 310 -614}\special{pa 316 -608}\special{pa 324 -604}\special{pa 333 -604}\special{pa 341 -607}\special{pa 348 -612}\special{pa 352 -620}\special{pa 353 -629}\special{pa 351 -637}\special{pa 346 -644}\special{sh}\special{fp}}%
\special{pn 4}\special{pa 346 -644}\special{pa 339 -649}\special{pa 330 -651}\special{pa 322 -650}\special{pa 314 -646}\special{pa 309 -639}\special{pa 306 -631}\special{pa 306 -622}\special{pa 310 -614}\special{pa 316 -608}\special{pa 324 -604}\special{pa 333 -604}\special{pa 341 -607}\special{pa 348 -612}\special{pa 352 -620}\special{pa 353 -629}\special{pa 351 -637}\special{pa 346 -644}\special{fp}%
{\special{pn 0}\color[rgb]{0,0,1}%
\special{pa 456 -853}\special{pa 449 -858}\special{pa 440 -860}\special{pa 432 -859}\special{pa 424 -855}\special{pa 418 -848}\special{pa 416 -840}\special{pa 416 -831}\special{pa 420 -823}\special{pa 426 -817}\special{pa 434 -814}\special{pa 442 -813}\special{pa 451 -816}\special{pa 457 -822}\special{pa 462 -829}\special{pa 463 -838}\special{pa 461 -846}\special{pa 456 -853}\special{sh}\special{fp}}%
\special{pn 4}\special{pa 456 -853}\special{pa 449 -858}\special{pa 440 -860}\special{pa 432 -859}\special{pa 424 -855}\special{pa 418 -848}\special{pa 416 -840}\special{pa 416 -831}\special{pa 420 -823}\special{pa 426 -817}\special{pa 434 -814}\special{pa 442 -813}\special{pa 451 -816}\special{pa 457 -822}\special{pa 462 -829}\special{pa 463 -838}\special{pa 461 -846}\special{pa 456 -853}\special{fp}%
{\special{pn 0}\color[rgb]{0,0,1}%
\special{pa 566 -1062}\special{pa 558 -1067}\special{pa 550 -1069}\special{pa 541 -1068}\special{pa 534 -1064}\special{pa 528 -1057}\special{pa 526 -1049}\special{pa 526 -1040}\special{pa 529 -1032}\special{pa 536 -1026}\special{pa 544 -1023}\special{pa 552 -1022}\special{pa 560 -1025}\special{pa 567 -1031}\special{pa 571 -1038}\special{pa 573 -1047}\special{pa 571 -1055}\special{pa 566 -1062}\special{sh}\special{fp}}%
\special{pn 4}\special{pa 566 -1062}\special{pa 558 -1067}\special{pa 550 -1069}\special{pa 541 -1068}\special{pa 534 -1064}\special{pa 528 -1057}\special{pa 526 -1049}\special{pa 526 -1040}\special{pa 529 -1032}\special{pa 536 -1026}\special{pa 544 -1023}\special{pa 552 -1022}\special{pa 560 -1025}\special{pa 567 -1031}\special{pa 571 -1038}\special{pa 573 -1047}\special{pa 571 -1055}\special{pa 566 -1062}\special{fp}%
}%
{%
\color[rgb]{0,1,0}%
{\special{pn 0}\color[rgb]{0,1,0}%
\special{pa 66 -116}\special{pa 60 -119}\special{pa 53 -120}\special{pa 47 -118}\special{pa 42 -113}\special{pa 39 -106}\special{pa 40 -99}\special{pa 44 -93}\special{pa 50 -90}\special{pa 57 -89}\special{pa 63 -91}\special{pa 68 -96}\special{pa 71 -103}\special{pa 70 -110}\special{pa 66 -116}\special{sh}\special{fp}}%
\special{pn 4}\special{pa 66 -116}\special{pa 60 -119}\special{pa 53 -120}\special{pa 47 -118}\special{pa 42 -113}\special{pa 39 -106}\special{pa 40 -99}\special{pa 44 -93}\special{pa 50 -90}\special{pa 57 -89}\special{pa 63 -91}\special{pa 68 -96}\special{pa 71 -103}\special{pa 70 -110}\special{pa 66 -116}\special{fp}%
{\special{pn 0}\color[rgb]{0,1,0}%
\special{pa 176 -325}\special{pa 170 -329}\special{pa 163 -329}\special{pa 156 -327}\special{pa 151 -322}\special{pa 149 -316}\special{pa 150 -309}\special{pa 154 -303}\special{pa 159 -299}\special{pa 166 -298}\special{pa 173 -300}\special{pa 178 -305}\special{pa 180 -312}\special{pa 180 -319}\special{pa 176 -325}\special{sh}\special{fp}}%
\special{pn 4}\special{pa 176 -325}\special{pa 170 -329}\special{pa 163 -329}\special{pa 156 -327}\special{pa 151 -322}\special{pa 149 -316}\special{pa 150 -309}\special{pa 154 -303}\special{pa 159 -299}\special{pa 166 -298}\special{pa 173 -300}\special{pa 178 -305}\special{pa 180 -312}\special{pa 180 -319}\special{pa 176 -325}\special{fp}%
{\special{pn 0}\color[rgb]{0,1,0}%
\special{pa 286 -534}\special{pa 280 -538}\special{pa 273 -539}\special{pa 266 -536}\special{pa 261 -531}\special{pa 259 -525}\special{pa 260 -518}\special{pa 263 -512}\special{pa 269 -508}\special{pa 276 -507}\special{pa 283 -510}\special{pa 288 -515}\special{pa 290 -521}\special{pa 289 -528}\special{pa 286 -534}\special{sh}\special{fp}}%
\special{pn 4}\special{pa 286 -534}\special{pa 280 -538}\special{pa 273 -539}\special{pa 266 -536}\special{pa 261 -531}\special{pa 259 -525}\special{pa 260 -518}\special{pa 263 -512}\special{pa 269 -508}\special{pa 276 -507}\special{pa 283 -510}\special{pa 288 -515}\special{pa 290 -521}\special{pa 289 -528}\special{pa 286 -534}\special{fp}%
{\special{pn 0}\color[rgb]{0,1,0}%
\special{pa 395 -743}\special{pa 389 -747}\special{pa 382 -748}\special{pa 376 -745}\special{pa 371 -740}\special{pa 369 -734}\special{pa 369 -727}\special{pa 373 -721}\special{pa 379 -717}\special{pa 386 -716}\special{pa 393 -719}\special{pa 398 -724}\special{pa 400 -730}\special{pa 399 -737}\special{pa 395 -743}\special{sh}\special{fp}}%
\special{pn 4}\special{pa 395 -743}\special{pa 389 -747}\special{pa 382 -748}\special{pa 376 -745}\special{pa 371 -740}\special{pa 369 -734}\special{pa 369 -727}\special{pa 373 -721}\special{pa 379 -717}\special{pa 386 -716}\special{pa 393 -719}\special{pa 398 -724}\special{pa 400 -730}\special{pa 399 -737}\special{pa 395 -743}\special{fp}%
{\special{pn 0}\color[rgb]{0,1,0}%
\special{pa 505 -952}\special{pa 499 -956}\special{pa 492 -957}\special{pa 486 -955}\special{pa 481 -950}\special{pa 478 -943}\special{pa 479 -936}\special{pa 483 -930}\special{pa 489 -926}\special{pa 496 -926}\special{pa 502 -928}\special{pa 507 -933}\special{pa 510 -939}\special{pa 509 -946}\special{pa 505 -952}\special{sh}\special{fp}}%
\special{pn 4}\special{pa 505 -952}\special{pa 499 -956}\special{pa 492 -957}\special{pa 486 -955}\special{pa 481 -950}\special{pa 478 -943}\special{pa 479 -936}\special{pa 483 -930}\special{pa 489 -926}\special{pa 496 -926}\special{pa 502 -928}\special{pa 507 -933}\special{pa 510 -939}\special{pa 509 -946}\special{pa 505 -952}\special{fp}%
{\special{pn 0}\color[rgb]{0,1,0}%
\special{pa 615 -1162}\special{pa 609 -1165}\special{pa 602 -1166}\special{pa 595 -1164}\special{pa 590 -1159}\special{pa 588 -1152}\special{pa 589 -1145}\special{pa 593 -1139}\special{pa 599 -1136}\special{pa 606 -1135}\special{pa 612 -1137}\special{pa 617 -1142}\special{pa 619 -1149}\special{pa 619 -1156}\special{pa 615 -1162}\special{sh}\special{fp}}%
\special{pn 4}\special{pa 615 -1162}\special{pa 609 -1165}\special{pa 602 -1166}\special{pa 595 -1164}\special{pa 590 -1159}\special{pa 588 -1152}\special{pa 589 -1145}\special{pa 593 -1139}\special{pa 599 -1136}\special{pa 606 -1135}\special{pa 612 -1137}\special{pa 617 -1142}\special{pa 619 -1149}\special{pa 619 -1156}\special{pa 615 -1162}\special{fp}%
}%
\special{pa  -472    -0}\special{pa  1890    -0}%
\special{fp}%
\special{pa     0   472}\special{pa     0 -1890}%
\special{fp}%
\settowidth{\Width}{$x$}\setlength{\Width}{0\Width}%
\settoheight{\Height}{$x$}\settodepth{\Depth}{$x$}\setlength{\Height}{-0.5\Height}\setlength{\Depth}{0.5\Depth}\addtolength{\Height}{\Depth}%
\put(8.0833333,0.0000000){\hspace*{\Width}\raisebox{\Height}{$x$}}%
%
\settowidth{\Width}{$y$}\setlength{\Width}{-0.5\Width}%
\settoheight{\Height}{$y$}\settodepth{\Depth}{$y$}\setlength{\Height}{\Depth}%
\put(0.0000000,8.0833333){\hspace*{\Width}\raisebox{\Height}{$y$}}%
%
\settowidth{\Width}{O}\setlength{\Width}{-1\Width}%
\settoheight{\Height}{O}\settodepth{\Depth}{O}\setlength{\Height}{-\Height}%
\put(-0.0833333,-0.0833333){\hspace*{\Width}\raisebox{\Height}{O}}%
%
\end{picture}}%}}
\end{layer}

\begin{itemize}
\item
$z=a+bi,\ w=c+di$($a,b,c,d$は実数)
\item
$zw\!=\!(a+bi)(c+di)\!=\!(ac-bd)+(ad+bc)i$
\item
$z,w,zw$の位置関係は?
\item
[]{\color{red}絶対値は 積\\偏角は 和}
\end{itemize}

\newslide{複素数の積と図形}

\vspace*{18mm}

\slidepage

\begin{layer}{120}{0}
%%putnote::se{70}{17}::absangle3
\end{layer}

\begin{itemize}
\item
$z=r(\cos\alpha+i\sin\alpha),\ w=s(cos\beta+i\sin\beta)$
{\color[cmyk]{\thin,\thin,\thin,\thin}%
\item
[]$zw=rs(\cos\alpha+i\sin\alpha)(\cos\beta+i\sin\beta)$
}%
{\color[cmyk]{\thin,\thin,\thin,\thin}%
\item
[]$\phantom{zw}=rs\bigl\{(\cos\alpha\cos\beta-\sin\alpha\sin\beta)$\\
}%
{\color[cmyk]{\thin,\thin,\thin,\thin}%
\hfill$+i(\sin\alpha\cos\beta+\cos\alpha\sin\beta)\bigr\}$\\
}%
{\color[cmyk]{\thin,\thin,\thin,\thin}%
$\phantom{zw}=rs\bigl(\cos(\alpha+\beta)+i\sin(\alpha+\beta)\bigr)$
}%
{\color[cmyk]{\thin,\thin,\thin,\thin}%
\item
[]したがって,$|zw|=rs=|z||w|$で,偏角は$\alpha+\beta$\\
}%
\end{itemize}

\sameslide

\vspace*{18mm}

\slidepage

\begin{layer}{120}{0}
\end{layer}

\begin{itemize}
\item
$z=r(\cos\alpha+i\sin\alpha),\ w=s(cos\beta+i\sin\beta)$
\item
[]$zw=rs(\cos\alpha+i\sin\alpha)(\cos\beta+i\sin\beta)$
{\color[cmyk]{\thin,\thin,\thin,\thin}%
\item
[]$\phantom{zw}=rs\bigl\{(\cos\alpha\cos\beta-\sin\alpha\sin\beta)$\\
}%
{\color[cmyk]{\thin,\thin,\thin,\thin}%
\hfill$+i(\sin\alpha\cos\beta+\cos\alpha\sin\beta)\bigr\}$\\
}%
{\color[cmyk]{\thin,\thin,\thin,\thin}%
$\phantom{zw}=rs\bigl(\cos(\alpha+\beta)+i\sin(\alpha+\beta)\bigr)$
}%
{\color[cmyk]{\thin,\thin,\thin,\thin}%
\item
[]したがって,$|zw|=rs=|z||w|$で,偏角は$\alpha+\beta$\\
}%
\end{itemize}

\sameslide

\vspace*{18mm}

\slidepage

\begin{layer}{120}{0}
\end{layer}

\begin{itemize}
\item
$z=r(\cos\alpha+i\sin\alpha),\ w=s(cos\beta+i\sin\beta)$
\item
[]$zw=rs(\cos\alpha+i\sin\alpha)(\cos\beta+i\sin\beta)$
\item
[]$\phantom{zw}=rs\bigl\{(\cos\alpha\cos\beta-\sin\alpha\sin\beta)$\\
\hfill$+i(\sin\alpha\cos\beta+\cos\alpha\sin\beta)\bigr\}$\\
{\color[cmyk]{\thin,\thin,\thin,\thin}%
$\phantom{zw}=rs\bigl(\cos(\alpha+\beta)+i\sin(\alpha+\beta)\bigr)$
}%
{\color[cmyk]{\thin,\thin,\thin,\thin}%
\item
[]したがって,$|zw|=rs=|z||w|$で,偏角は$\alpha+\beta$\\
}%
\end{itemize}

\sameslide

\vspace*{18mm}

\slidepage

\begin{layer}{120}{0}
\end{layer}

\begin{itemize}
\item
$z=r(\cos\alpha+i\sin\alpha),\ w=s(cos\beta+i\sin\beta)$
\item
[]$zw=rs(\cos\alpha+i\sin\alpha)(\cos\beta+i\sin\beta)$
\item
[]$\phantom{zw}=rs\bigl\{(\cos\alpha\cos\beta-\sin\alpha\sin\beta)$\\
\hfill$+i(\sin\alpha\cos\beta+\cos\alpha\sin\beta)\bigr\}$\\
$\phantom{zw}=rs\bigl(\cos(\alpha+\beta)+i\sin(\alpha+\beta)\bigr)$
{\color[cmyk]{\thin,\thin,\thin,\thin}%
\item
[]したがって,$|zw|=rs=|z||w|$で,偏角は$\alpha+\beta$\\
}%
\end{itemize}

\sameslide

\vspace*{18mm}

\slidepage

\begin{layer}{120}{0}
\end{layer}

\begin{itemize}
\item
$z=r(\cos\alpha+i\sin\alpha),\ w=s(cos\beta+i\sin\beta)$
\item
[]$zw=rs(\cos\alpha+i\sin\alpha)(\cos\beta+i\sin\beta)$
\item
[]$\phantom{zw}=rs\bigl\{(\cos\alpha\cos\beta-\sin\alpha\sin\beta)$\\
\hfill$+i(\sin\alpha\cos\beta+\cos\alpha\sin\beta)\bigr\}$\\
$\phantom{zw}=rs\bigl(\cos(\alpha+\beta)+i\sin(\alpha+\beta)\bigr)$
\item
[]したがって,$|zw|=rs=|z||w|$で,偏角は$\alpha+\beta$\\
\end{itemize}
\fi
\label{pageend}\mbox{}

\end{document}
