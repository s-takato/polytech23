%%% Title presen23104
\documentclass[landscape,10pt]{ujarticle}
\special{papersize=\the\paperwidth,\the\paperheight}
\usepackage{ketpic,ketlayer}
\usepackage{ketslide}
\usepackage{amsmath,amssymb}
\usepackage{bm,enumerate}
\usepackage[dvipdfmx]{graphicx}
\usepackage{color}
\definecolor{slidecolora}{cmyk}{0.98,0.13,0,0.43}
\definecolor{slidecolorb}{cmyk}{0.2,0,0,0}
\definecolor{slidecolorc}{cmyk}{0.2,0,0,0}
\definecolor{slidecolord}{cmyk}{0.2,0,0,0}
\definecolor{slidecolore}{cmyk}{0,0,0,0.5}
\definecolor{slidecolorf}{cmyk}{0,0,0,0.5}
\definecolor{slidecolori}{cmyk}{0.98,0.13,0,0.43}
\def\setthin#1{\def\thin{#1}}
\setthin{0}
\newcommand{\slidepage}[1][s]{%
\setcounter{ketpicctra}{18}%
\if#1m \setcounter{ketpicctra}{1}\fi
\hypersetup{linkcolor=black}%

\begin{layer}{118}{0}
\putnotee{122}{-\theketpicctra.05}{\small\thepage/\pageref{pageend}}
\end{layer}\hypersetup{linkcolor=blue}

}
\usepackage{emath}
\usepackage{emathEy}
\usepackage{emathMw}
\usepackage{pict2e}
\usepackage{ketlayermorewith2e}
\usepackage[dvipdfmx,colorlinks=true,linkcolor=blue,filecolor=blue]{hyperref}
\newcommand{\hiduke}{0515}
\newcommand{\hako}[2][1]{\fbox{\raisebox{#1mm}{\mbox{}}\raisebox{-#1mm}{\mbox{}}\,\phantom{#2}\,}}
\newcommand{\hakoa}[2][1]{\fbox{\raisebox{#1mm}{\mbox{}}\raisebox{-#1mm}{\mbox{}}\,#2\,}}
\newcommand{\hakom}[2][1]{\hako[#1]{$#2$}}
\newcommand{\hakoma}[2][1]{\hakoa[#1]{$#2$}}
\def\rad{\;\mathrm{rad}}
\def\deg#1{#1^{\circ}}
\newcommand{\sbunsuu}[2]{\scalebox{0.6}{$\bunsuu{#1}{#2}$}}
\def\pow{$\hspace{-1.5mm}^\hspace{-1mm}$}
\def\dlim{\displaystyle\lim}
\newcommand{\brd}[2][1]{\scalebox{#1}{\color{red}\fbox{\color{black}$#2$}}}
\newcommand\down[1][0.5zw]{\vspace{#1}\\}
\newcommand{\sfrac}[3][0.65]{\scalebox{#1}{$\frac{#2}{#3}$}}
\newcommand{\phn}[1]{\phantom{#1}}
\newcommand{\scb}[2][0.6]{\scalebox{#1}{#2}}
\newcommand{\dsum}{\displaystyle\sum}
\def\pow{$\hspace{-1.5mm}^\hspace{-1mm}$}
\def\dlim{\displaystyle\lim}
\def\dint{\displaystyle\int}

\setmargin{25}{145}{15}{100}

\ketslideinit

\pagestyle{empty}

\begin{document}

\begin{layer}{120}{0}
\putnotese{0}{0}{{\Large\bf
\color[cmyk]{1,1,0,0}

\begin{layer}{120}{0}
{\Huge \putnotes{60}{20}{三角比と三角関数}}
\putnotes{60}{70}{2022.04.25}
\end{layer}

}
}
\end{layer}

\def\mainslidetitley{22}
\def\ketcletter{slidecolora}
\def\ketcbox{slidecolorb}
\def\ketdbox{slidecolorc}
\def\ketcframe{slidecolord}
\def\ketcshadow{slidecolore}
\def\ketdshadow{slidecolorf}
\def\slidetitlex{6}
\def\slidetitlesize{1.3}
\def\mketcletter{slidecolori}
\def\mketcbox{yellow}
\def\mketdbox{yellow}
\def\mketcframe{yellow}
\def\mslidetitlex{62}
\def\mslidetitlesize{2}

\color{black}
\Large\bf\boldmath
\addtocounter{page}{-1}

\def\MARU{}
\renewcommand{\MARU}[1]{{\ooalign{\hfil$#1$\/\hfil\crcr\raise.167ex\hbox{\mathhexbox20D}}}}
\renewcommand{\slidepage}[1][s]{%
\setcounter{ketpicctra}{18}%
\if#1m \setcounter{ketpicctra}{1}\fi
\hypersetup{linkcolor=black}%
\begin{layer}{118}{0}
\putnotee{115}{-\theketpicctra.05}{\small\hiduke-\thepage/\pageref{pageend}}
\end{layer}\hypersetup{linkcolor=blue}
}
\newcounter{ban}
\setcounter{ban}{1}
\newcommand{\monban}[1][\hiduke]{%
#1-\theban\ %
\addtocounter{ban}{1}%
}
\newcommand{\monbannoadd}[1][\hiduke]{%
#1-\theban\ %
}
\newcommand{\addban}{%
\addtocounter{ban}{1}%
}
\newcounter{edawidth}
\newcounter{edactr}
\newcommand{\seteda}[1]{% 20220708 modified
\setcounter{edawidth}{#1}
\setcounter{edactr}{1}
}
\newcommand{\eda}[2][\theedawidth]{%
\Ltab{#1 mm}{[\theedactr]\ #2}%
\addtocounter{edactr}{1}%
}
%%%%%%%%%%%%

%%%%%%%%%%%%%%%%%%%%

\mainslide{三角比から三角関数へ}


\slidepage[m]
%%%%%%%%%%%%

%%%%%%%%%%%%%%%%%%%%

\newslide{三角比(復習)}

\vspace*{18mm}

\slidepage

\begin{layer}{120}{0}
\putnotese{60}{-5}{\scalebox{0.7}{%%% /polytech22.git/104-0509/presen/fig/fig22104_1.tex 
%%% Generator=fig22104.cdy 
{\unitlength=15mm%
\begin{picture}%
(5.5,5.5)(-0.5,-0.5)%
\linethickness{0.008in}%%
\polyline(0.50000,0.50000)(4.50000,0.50000)(4.43761,3.58099)(0.50000,0.50000)%
%
\settowidth{\Width}{A}\setlength{\Width}{-1\Width}%
\settoheight{\Height}{A}\settodepth{\Depth}{A}\setlength{\Height}{-0.5\Height}\setlength{\Depth}{0.5\Depth}\addtolength{\Height}{\Depth}%
\put(0.4666667,0.5000000){\hspace*{\Width}\raisebox{\Height}{A}}%
%
\settowidth{\Width}{B}\setlength{\Width}{0\Width}%
\settoheight{\Height}{B}\settodepth{\Depth}{B}\setlength{\Height}{-0.5\Height}\setlength{\Depth}{0.5\Depth}\addtolength{\Height}{\Depth}%
\put(4.5333333,0.5000000){\hspace*{\Width}\raisebox{\Height}{B}}%
%
\settowidth{\Width}{C}\setlength{\Width}{-0.5\Width}%
\settoheight{\Height}{C}\settodepth{\Depth}{C}\setlength{\Height}{\Depth}%
\put(4.4400000,3.6133333){\hspace*{\Width}\raisebox{\Height}{C}}%
%
\polyline(1.10000,0.50000)(1.09527,0.57520)(1.08115,0.64921)(1.05787,0.72087)(1.02578,0.78905)%
(0.98541,0.85267)(0.97178,0.86915)%
%
\settowidth{\Width}{$A$}\setlength{\Width}{-0.5\Width}%
\settoheight{\Height}{$A$}\settodepth{\Depth}{$A$}\setlength{\Height}{-0.5\Height}\setlength{\Depth}{0.5\Depth}\addtolength{\Height}{\Depth}%
\put(1.3200000,0.7800000){\hspace*{\Width}\raisebox{\Height}{$A$}}%
%
\polyline(4.49291,0.84993)(4.14291,0.84993)(4.15000,0.50000)%
%
\end{picture}}%}}
\end{layer}

\begin{itemize}
\item
[]$\cos A=\bunsuu{\mbox{AB}}{\mbox{AC}}=\bunsuu{\mbox{底辺}}{\mbox{斜辺}}$
\item
[]$\sin A=\bunsuu{\mbox{CB}}{\mbox{AC}}=\bunsuu{\mbox{高さ}}{\mbox{斜辺}}$
\item
[]$\tan A=\bunsuu{\mbox{BC}}{\mbox{AB}}=\bunsuu{\mbox{高さ}}{\mbox{底辺}}$
\item
辺の比だから,三角形の大きさによらない.
\end{itemize}
%%%%%%%%%%%%

%%%%%%%%%%%%%%%%%%%%


\newslide{角が$90\degree$より小さい場合(鋭角)}

\vspace*{18mm}

\slidepage

\begin{layer}{120}{0}
\putnotese{65}{8}{\scalebox{0.65}{%%% /polytech22.git/104-0509/presen/fig/fig2210431.tex 
%%% Generator=fig22104.cdy 
{\unitlength=1cm%
\begin{picture}%
(10,10)(-5,-5)%
\linethickness{0.008in}%%
\polyline(4.00000,0.00000)(3.96846,0.50133)(3.87433,0.99476)(3.71911,1.47250)(3.50523,1.92701)%
(3.23607,2.35114)(2.91587,2.73819)(2.54970,3.08205)(2.14331,3.37731)(1.70312,3.61931)%
(1.23607,3.80423)(0.74953,3.92915)(0.25116,3.99211)(-0.25116,3.99211)(-0.74953,3.92915)%
(-1.23607,3.80423)(-1.70312,3.61931)(-2.14331,3.37731)(-2.54970,3.08205)(-2.91587,2.73819)%
(-3.23607,2.35114)(-3.50523,1.92701)(-3.71911,1.47250)(-3.87433,0.99476)(-3.96846,0.50133)%
(-4.00000,-0.00000)(-3.96846,-0.50133)(-3.87433,-0.99476)(-3.71911,-1.47250)(-3.50523,-1.92701)%
(-3.23607,-2.35114)(-2.91587,-2.73819)(-2.54970,-3.08205)(-2.14331,-3.37731)(-1.70312,-3.61931)%
(-1.23607,-3.80423)(-0.74953,-3.92915)(-0.25116,-3.99211)(0.25116,-3.99211)(0.74953,-3.92915)%
(1.23607,-3.80423)(1.70312,-3.61931)(2.14331,-3.37731)(2.54970,-3.08205)(2.91587,-2.73819)%
(3.23607,-2.35114)(3.50523,-1.92701)(3.71911,-1.47250)(3.87433,-0.99476)(3.96846,-0.50133)%
(4.00000,-0.00000)%
%
\settowidth{\Width}{$r$}\setlength{\Width}{0\Width}%
\settoheight{\Height}{$r$}\settodepth{\Depth}{$r$}\setlength{\Height}{-\Height}%
\put(4.0500000,-0.1000000){\hspace*{\Width}\raisebox{\Height}{$r$}}%
%
\polyline(0.00000,0.00000)(1.69047,3.62523)(1.69047,0.00000)%
%
\polyline(0.50000,0.00000)(0.49606,0.06267)(0.48429,0.12434)(0.46489,0.18406)(0.43815,0.24088)%
(0.40451,0.29389)(0.36448,0.34227)(0.31871,0.38526)(0.26791,0.42216)(0.21289,0.45241)%
(0.21126,0.45306)%
%
\settowidth{\Width}{$\theta$}\setlength{\Width}{-0.5\Width}%
\settoheight{\Height}{$\theta$}\settodepth{\Depth}{$\theta$}\setlength{\Height}{-0.5\Height}\setlength{\Depth}{0.5\Depth}\addtolength{\Height}{\Depth}%
\put(0.6100000,0.3900000){\hspace*{\Width}\raisebox{\Height}{$\theta$}}%
%
\polyline(1.69047,3.62523)(1.59103,3.62523)\polyline(1.49159,3.62523)(1.39215,3.62523)%
\polyline(1.29271,3.62523)(1.19327,3.62523)\polyline(1.09383,3.62523)(0.99439,3.62523)%
\polyline(0.89495,3.62523)(0.79552,3.62523)\polyline(0.69608,3.62523)(0.59664,3.62523)%
\polyline(0.49720,3.62523)(0.39776,3.62523)\polyline(0.29832,3.62523)(0.19888,3.62523)%
\polyline(0.09944,3.62523)(0.00000,3.62523)%
%
\settowidth{\Width}{$x$}\setlength{\Width}{-0.5\Width}%
\settoheight{\Height}{$x$}\settodepth{\Depth}{$x$}\setlength{\Height}{-\Height}%
\put(1.6900000,-0.1000000){\hspace*{\Width}\raisebox{\Height}{$x$}}%
%
\settowidth{\Width}{$y$}\setlength{\Width}{-1\Width}%
\settoheight{\Height}{$y$}\settodepth{\Depth}{$y$}\setlength{\Height}{-0.5\Height}\setlength{\Depth}{0.5\Depth}\addtolength{\Height}{\Depth}%
\put(-0.1000000,3.6300000){\hspace*{\Width}\raisebox{\Height}{$y$}}%
%
\polyline(-5.00000,0.00000)(5.00000,0.00000)%
%
\polyline(0.00000,-5.00000)(0.00000,5.00000)%
%
\settowidth{\Width}{$x$}\setlength{\Width}{0\Width}%
\settoheight{\Height}{$x$}\settodepth{\Depth}{$x$}\setlength{\Height}{-0.5\Height}\setlength{\Depth}{0.5\Depth}\addtolength{\Height}{\Depth}%
\put(5.0500000,0.0000000){\hspace*{\Width}\raisebox{\Height}{$x$}}%
%
\settowidth{\Width}{$y$}\setlength{\Width}{-0.5\Width}%
\settoheight{\Height}{$y$}\settodepth{\Depth}{$y$}\setlength{\Height}{\Depth}%
\put(0.0000000,5.0500000){\hspace*{\Width}\raisebox{\Height}{$y$}}%
%
\settowidth{\Width}{O}\setlength{\Width}{-1\Width}%
\settoheight{\Height}{O}\settodepth{\Depth}{O}\setlength{\Height}{-\Height}%
\put(-0.0500000,-0.0500000){\hspace*{\Width}\raisebox{\Height}{O}}%
%
\end{picture}}%}}
\end{layer}

\begin{itemize}
\item
角を$\theta$とおく
\item
左の角が$\theta$の直角三角形がかける
\item
斜辺$r$,底辺$x$,高さ$y$とすると\\
 $\cos \theta=\bunsuu{x}{r}$\\
 $\sin \theta=\bunsuu{y}{r}$\\
 $\tan \theta=\bunsuu{y}{x}$\\
\end{itemize}

\newslide{角が$90\degree$より大きい場合}

\vspace*{18mm}

\slidepage

\begin{layer}{120}{0}
\putnotese{65}{8}{\scalebox{0.65}{%%% /polytech22.git/104-0509/presen/fig/fig2210434.tex 
%%% Generator=fig22104.cdy 
{\unitlength=1cm%
\begin{picture}%
(10,10)(-5,-5)%
\linethickness{0.008in}%%
\polyline(4.00000,0.00000)(3.96846,0.50133)(3.87433,0.99476)(3.71911,1.47250)(3.50523,1.92701)%
(3.23607,2.35114)(2.91587,2.73819)(2.54970,3.08205)(2.14331,3.37731)(1.70312,3.61931)%
(1.23607,3.80423)(0.74953,3.92915)(0.25116,3.99211)(-0.25116,3.99211)(-0.74953,3.92915)%
(-1.23607,3.80423)(-1.70312,3.61931)(-2.14331,3.37731)(-2.54970,3.08205)(-2.91587,2.73819)%
(-3.23607,2.35114)(-3.50523,1.92701)(-3.71911,1.47250)(-3.87433,0.99476)(-3.96846,0.50133)%
(-4.00000,-0.00000)(-3.96846,-0.50133)(-3.87433,-0.99476)(-3.71911,-1.47250)(-3.50523,-1.92701)%
(-3.23607,-2.35114)(-2.91587,-2.73819)(-2.54970,-3.08205)(-2.14331,-3.37731)(-1.70312,-3.61931)%
(-1.23607,-3.80423)(-0.74953,-3.92915)(-0.25116,-3.99211)(0.25116,-3.99211)(0.74953,-3.92915)%
(1.23607,-3.80423)(1.70312,-3.61931)(2.14331,-3.37731)(2.54970,-3.08205)(2.91587,-2.73819)%
(3.23607,-2.35114)(3.50523,-1.92701)(3.71911,-1.47250)(3.87433,-0.99476)(3.96846,-0.50133)%
(4.00000,-0.00000)%
%
\settowidth{\Width}{$r$}\setlength{\Width}{0\Width}%
\settoheight{\Height}{$r$}\settodepth{\Depth}{$r$}\setlength{\Height}{-\Height}%
\put(4.0500000,-0.1000000){\hspace*{\Width}\raisebox{\Height}{$r$}}%
%
\polyline(0.00000,0.00000)(3.46410,-2.00000)(3.46410,0.00000)%
%
\polyline(0.50000,0.00000)(0.49606,0.06267)(0.48429,0.12434)(0.46489,0.18406)(0.43815,0.24088)%
(0.40451,0.29389)(0.36448,0.34227)(0.31871,0.38526)(0.26791,0.42216)(0.21289,0.45241)%
(0.15451,0.47553)(0.09369,0.49114)(0.03140,0.49901)(-0.03140,0.49901)(-0.09369,0.49114)%
(-0.15451,0.47553)(-0.21289,0.45241)(-0.26791,0.42216)(-0.31871,0.38526)(-0.36448,0.34227)%
(-0.40451,0.29389)(-0.43815,0.24088)(-0.46489,0.18406)(-0.48429,0.12434)(-0.49606,0.06267)%
(-0.50000,-0.00000)(-0.49606,-0.06267)(-0.48429,-0.12434)(-0.46489,-0.18406)(-0.43815,-0.24088)%
(-0.40451,-0.29389)(-0.36448,-0.34227)(-0.31871,-0.38526)(-0.26791,-0.42216)(-0.21289,-0.45241)%
(-0.15451,-0.47553)(-0.09369,-0.49114)(-0.03140,-0.49901)(0.03140,-0.49901)(0.09369,-0.49114)%
(0.15451,-0.47553)(0.21289,-0.45241)(0.26791,-0.42216)(0.31871,-0.38526)(0.36448,-0.34227)%
(0.40451,-0.29389)(0.43254,-0.24973)%
%
\polyline(3.46410,-2.00000)(3.36513,-2.00000)\polyline(3.26615,-2.00000)(3.16718,-2.00000)%
\polyline(3.06820,-2.00000)(2.96923,-2.00000)\polyline(2.87025,-2.00000)(2.77128,-2.00000)%
\polyline(2.67231,-2.00000)(2.57333,-2.00000)\polyline(2.47436,-2.00000)(2.37538,-2.00000)%
\polyline(2.27641,-2.00000)(2.17743,-2.00000)\polyline(2.07846,-2.00000)(1.97949,-2.00000)%
\polyline(1.88051,-2.00000)(1.78154,-2.00000)\polyline(1.68256,-2.00000)(1.58359,-2.00000)%
\polyline(1.48461,-2.00000)(1.38564,-2.00000)\polyline(1.28667,-2.00000)(1.18769,-2.00000)%
\polyline(1.08872,-2.00000)(0.98974,-2.00000)\polyline(0.89077,-2.00000)(0.79179,-2.00000)%
\polyline(0.69282,-2.00000)(0.59385,-2.00000)\polyline(0.49487,-2.00000)(0.39590,-2.00000)%
\polyline(0.29692,-2.00000)(0.19795,-2.00000)\polyline(0.09897,-2.00000)(0.00000,-2.00000)%
%
%
\settowidth{\Width}{$x$}\setlength{\Width}{-0.5\Width}%
\settoheight{\Height}{$x$}\settodepth{\Depth}{$x$}\setlength{\Height}{\Depth}%
\put(3.4600000,0.1000000){\hspace*{\Width}\raisebox{\Height}{$x$}}%
%
\settowidth{\Width}{$y$}\setlength{\Width}{-1\Width}%
\settoheight{\Height}{$y$}\settodepth{\Depth}{$y$}\setlength{\Height}{-0.5\Height}\setlength{\Depth}{0.5\Depth}\addtolength{\Height}{\Depth}%
\put(-0.1000000,-2.0000000){\hspace*{\Width}\raisebox{\Height}{$y$}}%
%
\settowidth{\Width}{P}\setlength{\Width}{0\Width}%
\settoheight{\Height}{P}\settodepth{\Depth}{P}\setlength{\Height}{-\Height}%
\put(3.5100000,-2.0500000){\hspace*{\Width}\raisebox{\Height}{P}}%
%
\polyline(-5.00000,0.00000)(5.00000,0.00000)%
%
\polyline(0.00000,-5.00000)(0.00000,5.00000)%
%
\settowidth{\Width}{$x$}\setlength{\Width}{0\Width}%
\settoheight{\Height}{$x$}\settodepth{\Depth}{$x$}\setlength{\Height}{-0.5\Height}\setlength{\Depth}{0.5\Depth}\addtolength{\Height}{\Depth}%
\put(5.0500000,0.0000000){\hspace*{\Width}\raisebox{\Height}{$x$}}%
%
\settowidth{\Width}{$y$}\setlength{\Width}{-0.5\Width}%
\settoheight{\Height}{$y$}\settodepth{\Depth}{$y$}\setlength{\Height}{\Depth}%
\put(0.0000000,5.0500000){\hspace*{\Width}\raisebox{\Height}{$y$}}%
%
\settowidth{\Width}{O}\setlength{\Width}{-1\Width}%
\settoheight{\Height}{O}\settodepth{\Depth}{O}\setlength{\Height}{-\Height}%
\put(-0.0500000,-0.0500000){\hspace*{\Width}\raisebox{\Height}{O}}%
%
\end{picture}}%}}
\end{layer}

\begin{itemize}
\item
角$\theta$の直角三角形がかけない
\item
半径$r$の円上に$x$軸との角が$\theta$\\である点Pはとれる
\item
Pの$x$座標は底辺\\
  $y$座標は高さに対応\\
 $\cos \theta=\bunsuu{x}{r}$\\
 $\sin \theta=\bunsuu{y}{r}$\\
 $\tan \theta=\bunsuu{y}{x}$
\end{itemize}

\newslide{一般角の三角関数の値}

\vspace*{18mm}

\slidepage

\begin{layer}{120}{0}
\putnotese{80}{15}{%%% /Users/takatoosetsuo/polytech23.git/103-0515/presen/fig/ippansankaku.tex 
%%% Generator=presen23104.cdy 
{\unitlength=1cm%
\begin{picture}%
(4.8,4.8)(-2.4,-2.4)%
\linethickness{0.008in}%%
\Large\bf\boldmath%
\small%
\linethickness{0.012in}%%
\polyline(2,0)(1.984,0.251)(1.937,0.497)(1.86,0.736)(1.753,0.964)(1.618,1.176)(1.458,1.369)%
(1.275,1.541)(1.072,1.689)(0.852,1.81)(0.618,1.902)(0.375,1.965)(0.126,1.996)(-0.126,1.996)%
(-0.375,1.965)(-0.618,1.902)(-0.852,1.81)(-1.072,1.689)(-1.275,1.541)(-1.458,1.369)%
(-1.618,1.176)(-1.753,0.964)(-1.86,0.736)(-1.937,0.497)(-1.984,0.251)(-2,0)(-1.984,-0.251)%
(-1.937,-0.497)(-1.86,-0.736)(-1.753,-0.964)(-1.618,-1.176)(-1.458,-1.369)(-1.275,-1.541)%
(-1.072,-1.689)(-0.852,-1.81)(-0.618,-1.902)(-0.375,-1.965)(-0.126,-1.996)(0.126,-1.996)%
(0.375,-1.965)(0.618,-1.902)(0.852,-1.81)(1.072,-1.689)(1.275,-1.541)(1.458,-1.369)%
(1.618,-1.176)(1.753,-0.964)(1.86,-0.736)(1.937,-0.497)(1.984,-0.251)(2,0)%
%
\linethickness{0.008in}%%
\polyline(0.15,0)(0.151,0.032)(0.145,0.065)(0.131,0.096)(0.111,0.124)(0.085,0.148)%
(0.054,0.167)(0.018,0.178)(-0.02,0.183)(-0.059,0.178)(-0.097,0.166)(-0.132,0.145)%
(-0.163,0.117)(-0.188,0.082)(-0.205,0.042)(-0.213,-0.002)(-0.212,-0.047)(-0.201,-0.092)%
(-0.181,-0.134)(-0.152,-0.172)(-0.115,-0.204)(-0.071,-0.227)(-0.022,-0.241)(0.029,-0.245)%
(0.081,-0.237)(0.131,-0.219)(0.176,-0.19)(0.215,-0.152)(0.246,-0.105)(0.267,-0.052)%
(0.276,0.005)(0.273,0.063)(0.258,0.12)(0.23,0.174)(0.192,0.221)(0.143,0.26)(0.087,0.288)%
(0.026,0.304)(-0.039,0.307)(-0.103,0.296)(-0.165,0.272)(-0.221,0.234)(-0.269,0.186)%
(-0.305,0.127)(-0.329,0.062)(-0.339,-0.009)(-0.334,-0.08)(-0.314,-0.15)(-0.279,-0.214)%
(-0.277,-0.216)%
%
\polyline(0.15,0)(0.151,0.032)(0.145,0.065)(0.131,0.096)(0.111,0.124)(0.085,0.148)%
(0.054,0.167)(0.018,0.178)(-0.02,0.183)(-0.059,0.178)(-0.097,0.166)(-0.132,0.145)%
(-0.163,0.117)(-0.188,0.082)(-0.205,0.042)(-0.213,-0.002)(-0.212,-0.047)(-0.201,-0.092)%
(-0.181,-0.134)(-0.152,-0.172)(-0.115,-0.204)(-0.071,-0.227)(-0.022,-0.241)(0.029,-0.245)%
(0.081,-0.237)(0.131,-0.219)(0.176,-0.19)(0.215,-0.152)(0.246,-0.105)(0.267,-0.052)%
(0.276,0.005)(0.273,0.063)(0.258,0.12)(0.23,0.174)(0.192,0.221)(0.143,0.26)(0.087,0.288)%
(0.026,0.304)(-0.039,0.307)(-0.103,0.296)(-0.165,0.272)(-0.221,0.234)(-0.269,0.186)%
(-0.305,0.127)(-0.329,0.062)(-0.339,-0.009)(-0.334,-0.08)(-0.314,-0.15)(-0.279,-0.214)%
(-0.277,-0.216)%
%
\polygon*(-0.346,-0.22)(-0.171,-0.317)(-0.256,-0.135)(-0.271,-0.21)(-0.346,-0.22)%
(-0.346,-0.22)\linethickness{0.001in}%%
\polyline(-0.346,-0.22)(-0.171,-0.317)(-0.256,-0.135)(-0.271,-0.21)(-0.346,-0.22)%
%
\linethickness{0.008in}%%
\linethickness{0.012in}%%
\polyline(0,0)(-0.951,-1.759)%
%
\linethickness{0.008in}%%
\put(-0.951,0){\circle*{ 0.030}}\put(-0.951,-0.1){\circle*{ 0.030}}\put(-0.951,-0.201){\circle*{ 0.030}}
\put(-0.951,-0.301){\circle*{ 0.030}}\put(-0.951,-0.402){\circle*{ 0.030}}\put(-0.951,-0.502){\circle*{ 0.030}}
\put(-0.951,-0.602){\circle*{ 0.030}}\put(-0.951,-0.703){\circle*{ 0.030}}\put(-0.951,-0.803){\circle*{ 0.030}}
\put(-0.951,-0.903){\circle*{ 0.030}}\put(-0.951,-1.004){\circle*{ 0.030}}\put(-0.951,-1.104){\circle*{ 0.030}}
\put(-0.951,-1.205){\circle*{ 0.030}}\put(-0.951,-1.305){\circle*{ 0.030}}\put(-0.951,-1.405){\circle*{ 0.030}}
\put(-0.951,-1.506){\circle*{ 0.030}}\put(-0.951,-1.606){\circle*{ 0.030}}\put(-0.951,-1.707){\circle*{ 0.030}}
\put(-0.903,-1.759){\circle*{ 0.030}}\put(-0.803,-1.759){\circle*{ 0.030}}\put(-0.703,-1.759){\circle*{ 0.030}}
\put(-0.602,-1.759){\circle*{ 0.030}}\put(-0.502,-1.759){\circle*{ 0.030}}\put(-0.402,-1.759){\circle*{ 0.030}}
\put(-0.301,-1.759){\circle*{ 0.030}}\put(-0.201,-1.759){\circle*{ 0.030}}\put(-0.1,-1.759){\circle*{ 0.030}}
\put(0,-1.759){\circle*{ 0.030}}
\linethickness{0.008in}%%
\settowidth{\Width}{$\theta$}\setlength{\Width}{0\Width}%
\settoheight{\Height}{$\theta$}\settodepth{\Depth}{$\theta$}\setlength{\Height}{\Depth}%
\put(  0.330,  0.270){\hspace*{\Width}\raisebox{\Height}{$\theta$}}%
%
\settowidth{\Width}{$x$}\setlength{\Width}{-0.5\Width}%
\settoheight{\Height}{$x$}\settodepth{\Depth}{$x$}\setlength{\Height}{\Depth}%
\put( -0.950,  0.100){\hspace*{\Width}\raisebox{\Height}{$x$}}%
%
\settowidth{\Width}{$y$}\setlength{\Width}{0\Width}%
\settoheight{\Height}{$y$}\settodepth{\Depth}{$y$}\setlength{\Height}{-0.5\Height}\setlength{\Depth}{0.5\Depth}\addtolength{\Height}{\Depth}%
\put(  0.100, -1.760){\hspace*{\Width}\raisebox{\Height}{$y$}}%
%
\settowidth{\Width}{$r$}\setlength{\Width}{0\Width}%
\settoheight{\Height}{$r$}\settodepth{\Depth}{$r$}\setlength{\Height}{-\Height}%
\put(  2.100, -0.100){\hspace*{\Width}\raisebox{\Height}{$r$}}%
%
\polyline(-2.4,0)(2.4,0)%
%
\polyline(0,-2.4)(0,2.4)%
%
\settowidth{\Width}{$x$}\setlength{\Width}{0\Width}%
\settoheight{\Height}{$x$}\settodepth{\Depth}{$x$}\setlength{\Height}{-0.5\Height}\setlength{\Depth}{0.5\Depth}\addtolength{\Height}{\Depth}%
\put(  2.450,  0.000){\hspace*{\Width}\raisebox{\Height}{$x$}}%
%
\settowidth{\Width}{$y$}\setlength{\Width}{-0.5\Width}%
\settoheight{\Height}{$y$}\settodepth{\Depth}{$y$}\setlength{\Height}{\Depth}%
\put(  0.000,  2.450){\hspace*{\Width}\raisebox{\Height}{$y$}}%
%
\settowidth{\Width}{ }\setlength{\Width}{-1\Width}%
\settoheight{\Height}{ }\settodepth{\Depth}{ }\setlength{\Height}{-\Height}%
\put( -0.050, -0.050){\hspace*{\Width}\raisebox{\Height}{ }}%
%
\end{picture}}%}
\end{layer}

\begin{itemize}
\item
半径$r$の円上に一般角$\theta$の点Pをとる
\item
Pの座標を$(x,\ y)$とすると\\
 $\cos \theta=\bunsuu{x}{r}$\\
 $\sin \theta=\bunsuu{y}{r}$\\
 $\tan \theta=\bunsuu{y}{x}$
\item
[課題]\monban 図の$\theta$について求めよ\seteda{25}\\
\eda{$\cos\theta$}\eda{$\sin\theta$}\eda{$\tan\theta$}
\end{itemize}

\newslide{三角関数の値の符号}

\vspace*{18mm}

\slidepage

\begin{layer}{120}{0}
\putnotese{75}{8}{\scalebox{0.5}{%%% /polytech22.git/104-0509/presen/fig/fig2210434.tex 
%%% Generator=fig22104.cdy 
{\unitlength=1cm%
\begin{picture}%
(10,10)(-5,-5)%
\linethickness{0.008in}%%
\polyline(4.00000,0.00000)(3.96846,0.50133)(3.87433,0.99476)(3.71911,1.47250)(3.50523,1.92701)%
(3.23607,2.35114)(2.91587,2.73819)(2.54970,3.08205)(2.14331,3.37731)(1.70312,3.61931)%
(1.23607,3.80423)(0.74953,3.92915)(0.25116,3.99211)(-0.25116,3.99211)(-0.74953,3.92915)%
(-1.23607,3.80423)(-1.70312,3.61931)(-2.14331,3.37731)(-2.54970,3.08205)(-2.91587,2.73819)%
(-3.23607,2.35114)(-3.50523,1.92701)(-3.71911,1.47250)(-3.87433,0.99476)(-3.96846,0.50133)%
(-4.00000,-0.00000)(-3.96846,-0.50133)(-3.87433,-0.99476)(-3.71911,-1.47250)(-3.50523,-1.92701)%
(-3.23607,-2.35114)(-2.91587,-2.73819)(-2.54970,-3.08205)(-2.14331,-3.37731)(-1.70312,-3.61931)%
(-1.23607,-3.80423)(-0.74953,-3.92915)(-0.25116,-3.99211)(0.25116,-3.99211)(0.74953,-3.92915)%
(1.23607,-3.80423)(1.70312,-3.61931)(2.14331,-3.37731)(2.54970,-3.08205)(2.91587,-2.73819)%
(3.23607,-2.35114)(3.50523,-1.92701)(3.71911,-1.47250)(3.87433,-0.99476)(3.96846,-0.50133)%
(4.00000,-0.00000)%
%
\settowidth{\Width}{$r$}\setlength{\Width}{0\Width}%
\settoheight{\Height}{$r$}\settodepth{\Depth}{$r$}\setlength{\Height}{-\Height}%
\put(4.0500000,-0.1000000){\hspace*{\Width}\raisebox{\Height}{$r$}}%
%
\polyline(0.00000,0.00000)(3.46410,-2.00000)(3.46410,0.00000)%
%
\polyline(0.50000,0.00000)(0.49606,0.06267)(0.48429,0.12434)(0.46489,0.18406)(0.43815,0.24088)%
(0.40451,0.29389)(0.36448,0.34227)(0.31871,0.38526)(0.26791,0.42216)(0.21289,0.45241)%
(0.15451,0.47553)(0.09369,0.49114)(0.03140,0.49901)(-0.03140,0.49901)(-0.09369,0.49114)%
(-0.15451,0.47553)(-0.21289,0.45241)(-0.26791,0.42216)(-0.31871,0.38526)(-0.36448,0.34227)%
(-0.40451,0.29389)(-0.43815,0.24088)(-0.46489,0.18406)(-0.48429,0.12434)(-0.49606,0.06267)%
(-0.50000,-0.00000)(-0.49606,-0.06267)(-0.48429,-0.12434)(-0.46489,-0.18406)(-0.43815,-0.24088)%
(-0.40451,-0.29389)(-0.36448,-0.34227)(-0.31871,-0.38526)(-0.26791,-0.42216)(-0.21289,-0.45241)%
(-0.15451,-0.47553)(-0.09369,-0.49114)(-0.03140,-0.49901)(0.03140,-0.49901)(0.09369,-0.49114)%
(0.15451,-0.47553)(0.21289,-0.45241)(0.26791,-0.42216)(0.31871,-0.38526)(0.36448,-0.34227)%
(0.40451,-0.29389)(0.43254,-0.24973)%
%
\polyline(3.46410,-2.00000)(3.36513,-2.00000)\polyline(3.26615,-2.00000)(3.16718,-2.00000)%
\polyline(3.06820,-2.00000)(2.96923,-2.00000)\polyline(2.87025,-2.00000)(2.77128,-2.00000)%
\polyline(2.67231,-2.00000)(2.57333,-2.00000)\polyline(2.47436,-2.00000)(2.37538,-2.00000)%
\polyline(2.27641,-2.00000)(2.17743,-2.00000)\polyline(2.07846,-2.00000)(1.97949,-2.00000)%
\polyline(1.88051,-2.00000)(1.78154,-2.00000)\polyline(1.68256,-2.00000)(1.58359,-2.00000)%
\polyline(1.48461,-2.00000)(1.38564,-2.00000)\polyline(1.28667,-2.00000)(1.18769,-2.00000)%
\polyline(1.08872,-2.00000)(0.98974,-2.00000)\polyline(0.89077,-2.00000)(0.79179,-2.00000)%
\polyline(0.69282,-2.00000)(0.59385,-2.00000)\polyline(0.49487,-2.00000)(0.39590,-2.00000)%
\polyline(0.29692,-2.00000)(0.19795,-2.00000)\polyline(0.09897,-2.00000)(0.00000,-2.00000)%
%
%
\settowidth{\Width}{$x$}\setlength{\Width}{-0.5\Width}%
\settoheight{\Height}{$x$}\settodepth{\Depth}{$x$}\setlength{\Height}{\Depth}%
\put(3.4600000,0.1000000){\hspace*{\Width}\raisebox{\Height}{$x$}}%
%
\settowidth{\Width}{$y$}\setlength{\Width}{-1\Width}%
\settoheight{\Height}{$y$}\settodepth{\Depth}{$y$}\setlength{\Height}{-0.5\Height}\setlength{\Depth}{0.5\Depth}\addtolength{\Height}{\Depth}%
\put(-0.1000000,-2.0000000){\hspace*{\Width}\raisebox{\Height}{$y$}}%
%
\settowidth{\Width}{P}\setlength{\Width}{0\Width}%
\settoheight{\Height}{P}\settodepth{\Depth}{P}\setlength{\Height}{-\Height}%
\put(3.5100000,-2.0500000){\hspace*{\Width}\raisebox{\Height}{P}}%
%
\polyline(-5.00000,0.00000)(5.00000,0.00000)%
%
\polyline(0.00000,-5.00000)(0.00000,5.00000)%
%
\settowidth{\Width}{$x$}\setlength{\Width}{0\Width}%
\settoheight{\Height}{$x$}\settodepth{\Depth}{$x$}\setlength{\Height}{-0.5\Height}\setlength{\Depth}{0.5\Depth}\addtolength{\Height}{\Depth}%
\put(5.0500000,0.0000000){\hspace*{\Width}\raisebox{\Height}{$x$}}%
%
\settowidth{\Width}{$y$}\setlength{\Width}{-0.5\Width}%
\settoheight{\Height}{$y$}\settodepth{\Depth}{$y$}\setlength{\Height}{\Depth}%
\put(0.0000000,5.0500000){\hspace*{\Width}\raisebox{\Height}{$y$}}%
%
\settowidth{\Width}{O}\setlength{\Width}{-1\Width}%
\settoheight{\Height}{O}\settodepth{\Depth}{O}\setlength{\Height}{-\Height}%
\put(-0.0500000,-0.0500000){\hspace*{\Width}\raisebox{\Height}{O}}%
%
\end{picture}}%}}
\end{layer}

\begin{itemize}
\item
[]\Ltab{20mm}{}\Ctab{15mm}{$\cos\theta$}\Ctab{15mm}{$\sin\theta$}\Ctab{15mm}{$\tan\theta$}
\item
第1象限\Ctab{15mm}{$+$}\Ctab{15mm}{$+$}\Ctab{15mm}{$+$}
\item
第2象限\Ctab{15mm}{$-$}\Ctab{15mm}{$+$}\Ctab{15mm}{$-$}
\item
第3象限\Ctab{15mm}{$-$}\Ctab{15mm}{$-$}\Ctab{15mm}{$+$}
\item
第4象限
\item
[課題]\monban 第4象限での符号を答えよ\seteda{40}\\
\eda{$\cos\theta$の符号}\eda{$\sin\theta$の符号}\eda{$\tan\theta$の符号}
\end{itemize}

\newslide{三角関数の相互関係}

\vspace*{18mm}

\slidepage

\begin{layer}{120}{0}
\putnotee{60}{46}{\color{red}$\bigl(\cos(\theta)\bigr)^2$を$\cos^2 \theta$と書く}
\putnotese{90}{6}{%%% /Users/takatoosetsuo/Dropbox/2021polytech/103/fig/presen10310305.tex 
%%% Generator=presen103.cdy 
{\unitlength=7mm%
\begin{picture}%
(4.5,3.5)(0,-0.5)%
\special{pn 8}%
%
\Large\bf\boldmath%
\small%
\normalsize%
\special{pn 16}%
\special{pa  1102  -551}\special{pa     0    -0}\special{pa  1102    -0}\special{pa  1102  -551}%
\special{fp}%
\special{pn 8}%
\special{pn 4}%
\special{pa    -0    -0}\special{pa    10     4}\special{pa    19     8}\special{pa    29    12}%
\special{pa    39    16}\special{pa    49    19}\special{pa    59    23}\special{pa    68    26}%
\special{pa    78    30}\special{pa    88    33}\special{pa    98    37}\special{pa   108    40}%
\special{pa   118    43}\special{pa   128    46}\special{pa   138    49}\special{pa   148    52}%
\special{pa   158    55}\special{pa   168    58}\special{pa   179    61}\special{pa   189    64}%
\special{pa   199    66}\special{pa   209    69}\special{pa   219    71}\special{pa   229    74}%
\special{pa   240    76}\special{pa   250    78}\special{pa   260    80}\special{pa   270    82}%
\special{pa   281    84}\special{pa   291    86}\special{pa   301    88}\special{pa   312    90}%
\special{pa   322    92}\special{pa   332    93}\special{pa   343    95}\special{pa   353    96}%
\special{pa   364    98}\special{pa   374    99}\special{pa   384   100}\special{pa   395   102}%
\special{pa   405   103}\special{pa   416   104}\special{pa   426   105}\special{pa   437   106}%
\special{pa   447   106}\special{pa   457   107}\special{pa   468   108}\special{pa   478   108}%
\special{pa   489   109}\special{pa   499   109}\special{pa   510   110}%
\special{fp}%
\special{pa   593   110}\special{pa   603   109}\special{pa   613   109}\special{pa   624   108}%
\special{pa   634   108}\special{pa   645   107}\special{pa   655   106}\special{pa   666   106}%
\special{pa   676   105}\special{pa   687   104}\special{pa   697   103}\special{pa   708   102}%
\special{pa   718   100}\special{pa   728    99}\special{pa   739    98}\special{pa   749    96}%
\special{pa   760    95}\special{pa   770    93}\special{pa   780    92}\special{pa   791    90}%
\special{pa   801    88}\special{pa   811    86}\special{pa   822    84}\special{pa   832    82}%
\special{pa   842    80}\special{pa   852    78}\special{pa   863    76}\special{pa   873    74}%
\special{pa   883    71}\special{pa   893    69}\special{pa   904    66}\special{pa   914    64}%
\special{pa   924    61}\special{pa   934    58}\special{pa   944    55}\special{pa   954    52}%
\special{pa   964    49}\special{pa   974    46}\special{pa   984    43}\special{pa   994    40}%
\special{pa  1004    37}\special{pa  1014    33}\special{pa  1024    30}\special{pa  1034    26}%
\special{pa  1044    23}\special{pa  1054    19}\special{pa  1063    16}\special{pa  1073    12}%
\special{pa  1083     8}\special{pa  1093     4}\special{pa  1102     0}%
\special{fp}%
\special{pn 8}%
\settowidth{\Width}{$x$}\setlength{\Width}{-0.5\Width}%
\settoheight{\Height}{$x$}\settodepth{\Depth}{$x$}\setlength{\Height}{-0.5\Height}\setlength{\Depth}{0.5\Depth}\addtolength{\Height}{\Depth}%
\put(2.0000000,-0.4000000){\hspace*{\Width}\raisebox{\Height}{$x$}}%
%
\special{pn 4}%
\special{pa  1102    -0}\special{pa  1104    -4}\special{pa  1106    -9}\special{pa  1108   -13}%
\special{pa  1110   -18}\special{pa  1111   -22}\special{pa  1113   -27}\special{pa  1115   -32}%
\special{pa  1116   -36}\special{pa  1118   -41}\special{pa  1119   -45}\special{pa  1121   -50}%
\special{pa  1122   -54}\special{pa  1124   -59}\special{pa  1125   -64}\special{pa  1127   -68}%
\special{pa  1128   -73}\special{pa  1130   -77}\special{pa  1131   -82}\special{pa  1132   -87}%
\special{pa  1133   -91}\special{pa  1135   -96}\special{pa  1136  -101}\special{pa  1137  -105}%
\special{pa  1138  -110}\special{pa  1139  -115}\special{pa  1140  -120}\special{pa  1141  -124}%
\special{pa  1142  -129}\special{pa  1143  -134}\special{pa  1144  -138}\special{pa  1145  -143}%
\special{pa  1146  -148}\special{pa  1147  -153}\special{pa  1148  -157}\special{pa  1148  -162}%
\special{pa  1149  -167}\special{pa  1150  -172}\special{pa  1151  -177}\special{pa  1151  -181}%
\special{pa  1152  -186}\special{pa  1152  -191}\special{pa  1153  -196}\special{pa  1154  -201}%
\special{pa  1154  -205}\special{pa  1154  -210}\special{pa  1155  -215}\special{pa  1155  -220}%
\special{pa  1156  -225}\special{pa  1156  -229}\special{pa  1156  -234}%
\special{fp}%
\special{pa  1156  -317}\special{pa  1156  -322}\special{pa  1156  -327}\special{pa  1155  -331}%
\special{pa  1155  -336}\special{pa  1154  -341}\special{pa  1154  -346}\special{pa  1154  -351}%
\special{pa  1153  -355}\special{pa  1152  -360}\special{pa  1152  -365}\special{pa  1151  -370}%
\special{pa  1151  -375}\special{pa  1150  -379}\special{pa  1149  -384}\special{pa  1148  -389}%
\special{pa  1148  -394}\special{pa  1147  -398}\special{pa  1146  -403}\special{pa  1145  -408}%
\special{pa  1144  -413}\special{pa  1143  -417}\special{pa  1142  -422}\special{pa  1141  -427}%
\special{pa  1140  -432}\special{pa  1139  -436}\special{pa  1138  -441}\special{pa  1137  -446}%
\special{pa  1136  -450}\special{pa  1135  -455}\special{pa  1133  -460}\special{pa  1132  -464}%
\special{pa  1131  -469}\special{pa  1130  -474}\special{pa  1128  -478}\special{pa  1127  -483}%
\special{pa  1125  -488}\special{pa  1124  -492}\special{pa  1122  -497}\special{pa  1121  -501}%
\special{pa  1119  -506}\special{pa  1118  -511}\special{pa  1116  -515}\special{pa  1115  -520}%
\special{pa  1113  -524}\special{pa  1111  -529}\special{pa  1110  -533}\special{pa  1108  -538}%
\special{pa  1106  -542}\special{pa  1104  -547}\special{pa  1102  -551}%
\special{fp}%
\special{pn 8}%
\settowidth{\Width}{$y$}\setlength{\Width}{-0.5\Width}%
\settoheight{\Height}{$y$}\settodepth{\Depth}{$y$}\setlength{\Height}{-0.5\Height}\setlength{\Depth}{0.5\Depth}\addtolength{\Height}{\Depth}%
\put(4.2000000,1.0000000){\hspace*{\Width}\raisebox{\Height}{$y$}}%
%
\special{pa  1102   -69}\special{pa  1033   -69}\special{pa  1033    -0}%
\special{fp}%
\special{pa   138    -0}\special{pa   137   -17}\special{pa   133   -34}\special{pa   128   -51}%
\special{pa   123   -62}%
\special{fp}%
\settowidth{\Width}{$\theta$}\setlength{\Width}{-0.5\Width}%
\settoheight{\Height}{$\theta$}\settodepth{\Depth}{$\theta$}\setlength{\Height}{-0.5\Height}\setlength{\Depth}{0.5\Depth}\addtolength{\Height}{\Depth}%
\put(0.9300000,0.2200000){\hspace*{\Width}\raisebox{\Height}{$\theta$}}%
%
\settowidth{\Width}{P}\setlength{\Width}{-0.5\Width}%
\settoheight{\Height}{P}\settodepth{\Depth}{P}\setlength{\Height}{\Depth}%
\put(4.0000000,2.1428571){\hspace*{\Width}\raisebox{\Height}{P}}%
%
\special{pa  1102   -20}\special{pa  1102    20}%
\special{fp}%
\settowidth{\Width}{$x$}\setlength{\Width}{-0.5\Width}%
\settoheight{\Height}{$x$}\settodepth{\Depth}{$x$}\setlength{\Height}{-\Height}%
\put(4.0000000,-0.1428571){\hspace*{\Width}\raisebox{\Height}{$x$}}%
%
\special{pa    20  -551}\special{pa     0  -551}%
\special{fp}%
\settowidth{\Width}{$y$}\setlength{\Width}{-1\Width}%
\settoheight{\Height}{$y$}\settodepth{\Depth}{$y$}\setlength{\Height}{-0.5\Height}\setlength{\Depth}{0.5\Depth}\addtolength{\Height}{\Depth}%
\put(-0.1428571,2.0000000){\hspace*{\Width}\raisebox{\Height}{$y$}}%
%
\special{pa     0    -0}\special{pa  1240    -0}%
\special{fp}%
\special{pa     0   138}\special{pa     0  -827}%
\special{fp}%
\settowidth{\Width}{$x$}\setlength{\Width}{0\Width}%
\settoheight{\Height}{$x$}\settodepth{\Depth}{$x$}\setlength{\Height}{-0.5\Height}\setlength{\Depth}{0.5\Depth}\addtolength{\Height}{\Depth}%
\put(4.5714286,0.0000000){\hspace*{\Width}\raisebox{\Height}{$x$}}%
%
\settowidth{\Width}{$y$}\setlength{\Width}{-0.5\Width}%
\settoheight{\Height}{$y$}\settodepth{\Depth}{$y$}\setlength{\Height}{\Depth}%
\put(0.0000000,3.0714286){\hspace*{\Width}\raisebox{\Height}{$y$}}%
%
\settowidth{\Width}{O}\setlength{\Width}{-1\Width}%
\settoheight{\Height}{O}\settodepth{\Depth}{O}\setlength{\Height}{-\Height}%
\put(-0.0714286,-0.0714286){\hspace*{\Width}\raisebox{\Height}{O}}%
%
\end{picture}}%}
\end{layer}

\begin{itemize}
\item
[(1)\ ]$\tan\theta=\bunsuu{\sin\theta}{\cos\theta}$
\item
[]以下の証明では$\mathrm{OP}=r$とおく
\item
[\color{blue}証)]{\color{blue}$\tan\theta=\bunsuu{y}{x}=\bunsuu{\frac{y}{r}}{\frac{x}{r}}$}
{\color{blue}$=\bunsuu{\sin\theta}{\cos\theta}$}
\item
[(2)\ ]$\cos^2\theta+\sin^2\theta=1$
\item
[\color{blue}証)]{\color{blue}$\cos^2\theta+\sin^2\theta=\bunsuu{x^2}{r^2}+\bunsuu{y^2}{r^2}$}
{\color{blue}$=\bunsuu{x^2+y^2}{r^2}=1$}
\end{itemize}

\newslide{弧度法}

\vspace*{18mm}


\begin{layer}{120}{0}
\putnotew{96}{73}{\hyperlink{para0pg7}{\fbox{\Ctab{2.5mm}{\scalebox{1}{\scriptsize $\mathstrut||\!\lhd$}}}}}
\putnotew{101}{73}{\hyperlink{para1pg1}{\fbox{\Ctab{2.5mm}{\scalebox{1}{\scriptsize $\mathstrut|\!\lhd$}}}}}
\putnotew{108}{73}{\hyperlink{para1pg8}{\fbox{\Ctab{4.5mm}{\scalebox{1}{\scriptsize $\mathstrut\!\!\lhd\!\!$}}}}}
\putnotew{115}{73}{\hyperlink{para1pg9}{\fbox{\Ctab{4.5mm}{\scalebox{1}{\scriptsize $\mathstrut\!\rhd\!$}}}}}
\putnotew{120}{73}{\hyperlink{para1pg9}{\fbox{\Ctab{2.5mm}{\scalebox{1}{\scriptsize $\mathstrut \!\rhd\!\!|$}}}}}
\putnotew{125}{73}{\hyperlink{para2pg1}{\fbox{\Ctab{2.5mm}{\scalebox{1}{\scriptsize $\mathstrut \!\rhd\!\!||$}}}}}
\putnotee{126}{73}{\scriptsize\color{blue} 9/9}
\end{layer}

\slidepage

\begin{layer}{120}{0}
\putnotese{89}{10}{%%% /Users/takatoosetsuo/Dropbox/2018polytec/lecture/0514/presen/fig/radian.tex 
%%% Generator=presen0514.cdy 
{\unitlength=1cm%
\begin{picture}%
(4.35,4.12)(-2.06,-2.06)%
\special{pn 8}%
%
\Large\bf\boldmath%
\small%
\special{pn 12}%
\special{pa   787    -0}\special{pa     0    -0}\special{pa   475  -628}%
\special{fp}%
\special{pn 8}%
\special{pn 8}%
\special{pa 788 4}\special{pa 787 -4}\special{fp}\special{pa 786 -35}\special{pa 786 -43}\special{fp}%
\special{pa 784 -74}\special{pa 783 -82}\special{fp}\special{pa 779 -113}\special{pa 778 -121}\special{fp}%
\special{pa 772 -152}\special{pa 771 -160}\special{fp}\special{pa 764 -190}\special{pa 762 -198}\special{fp}%
\special{pa 753 -228}\special{pa 751 -236}\special{fp}\special{pa 741 -265}\special{pa 738 -273}\special{fp}%
\special{pa 727 -302}\special{pa 724 -309}\special{fp}\special{pa 711 -338}\special{pa 708 -345}\special{fp}%
\special{pa 693 -373}\special{pa 690 -380}\special{fp}\special{pa 674 -407}\special{pa 670 -414}\special{fp}%
\special{pa 653 -440}\special{pa 648 -447}\special{fp}\special{pa 630 -472}\special{pa 625 -478}\special{fp}%
\special{pa 606 -503}\special{pa 601 -509}\special{fp}\special{pa 580 -533}\special{pa 574 -538}\special{fp}%
\special{pa 552 -561}\special{pa 547 -566}\special{fp}\special{pa 524 -587}\special{pa 518 -593}\special{fp}%
\special{pa 494 -613}\special{pa 488 -618}\special{fp}\special{pa 463 -637}\special{pa 456 -642}\special{fp}%
\special{pa 430 -659}\special{pa 424 -663}\special{fp}\special{pa 397 -680}\special{pa 390 -684}\special{fp}%
\special{pa 363 -699}\special{pa 355 -702}\special{fp}\special{pa 327 -716}\special{pa 320 -719}\special{fp}%
\special{pa 291 -731}\special{pa 284 -734}\special{fp}\special{pa 255 -745}\special{pa 247 -747}\special{fp}%
\special{pa 217 -757}\special{pa 209 -759}\special{fp}\special{pa 179 -766}\special{pa 171 -768}\special{fp}%
\special{pa 141 -774}\special{pa 133 -776}\special{fp}\special{pa 102 -781}\special{pa 94 -782}\special{fp}%
\special{pa 63 -785}\special{pa 55 -785}\special{fp}\special{pa 24 -787}\special{pa 16 -787}\special{fp}%
\special{pa -16 -787}\special{pa -24 -787}\special{fp}\special{pa -55 -785}\special{pa -63 -785}\special{fp}%
\special{pa -94 -782}\special{pa -102 -781}\special{fp}\special{pa -133 -776}\special{pa -141 -774}\special{fp}%
\special{pa -171 -768}\special{pa -179 -766}\special{fp}\special{pa -209 -759}\special{pa -217 -757}\special{fp}%
\special{pa -247 -747}\special{pa -255 -745}\special{fp}\special{pa -284 -734}\special{pa -291 -731}\special{fp}%
\special{pa -320 -719}\special{pa -327 -716}\special{fp}\special{pa -355 -702}\special{pa -363 -699}\special{fp}%
\special{pa -390 -684}\special{pa -397 -680}\special{fp}\special{pa -424 -663}\special{pa -430 -659}\special{fp}%
\special{pa -456 -642}\special{pa -463 -637}\special{fp}\special{pa -488 -618}\special{pa -494 -613}\special{fp}%
\special{pa -518 -593}\special{pa -524 -587}\special{fp}\special{pa -547 -566}\special{pa -552 -561}\special{fp}%
\special{pa -574 -538}\special{pa -580 -533}\special{fp}\special{pa -601 -509}\special{pa -606 -503}\special{fp}%
\special{pa -625 -478}\special{pa -630 -472}\special{fp}\special{pa -648 -447}\special{pa -653 -440}\special{fp}%
\special{pa -670 -414}\special{pa -674 -407}\special{fp}\special{pa -690 -380}\special{pa -693 -373}\special{fp}%
\special{pa -708 -345}\special{pa -711 -338}\special{fp}\special{pa -724 -309}\special{pa -727 -302}\special{fp}%
\special{pa -738 -273}\special{pa -741 -265}\special{fp}\special{pa -751 -236}\special{pa -753 -228}\special{fp}%
\special{pa -762 -198}\special{pa -764 -190}\special{fp}\special{pa -771 -160}\special{pa -772 -152}\special{fp}%
\special{pa -778 -121}\special{pa -779 -113}\special{fp}\special{pa -783 -82}\special{pa -784 -74}\special{fp}%
\special{pa -786 -43}\special{pa -786 -35}\special{fp}\special{pa -787 -4}\special{pa -787 4}\special{fp}%
\special{pa -786 35}\special{pa -786 43}\special{fp}\special{pa -784 74}\special{pa -783 82}\special{fp}%
\special{pa -779 113}\special{pa -778 121}\special{fp}\special{pa -772 152}\special{pa -771 160}\special{fp}%
\special{pa -764 190}\special{pa -762 198}\special{fp}\special{pa -753 228}\special{pa -751 236}\special{fp}%
\special{pa -741 265}\special{pa -738 273}\special{fp}\special{pa -727 302}\special{pa -724 309}\special{fp}%
\special{pa -711 338}\special{pa -708 345}\special{fp}\special{pa -693 373}\special{pa -690 380}\special{fp}%
\special{pa -674 407}\special{pa -670 414}\special{fp}\special{pa -653 440}\special{pa -648 447}\special{fp}%
\special{pa -630 472}\special{pa -625 478}\special{fp}\special{pa -606 503}\special{pa -601 509}\special{fp}%
\special{pa -580 533}\special{pa -574 538}\special{fp}\special{pa -552 561}\special{pa -547 566}\special{fp}%
\special{pa -524 587}\special{pa -518 593}\special{fp}\special{pa -494 613}\special{pa -488 618}\special{fp}%
\special{pa -463 637}\special{pa -456 642}\special{fp}\special{pa -430 659}\special{pa -424 663}\special{fp}%
\special{pa -397 680}\special{pa -390 684}\special{fp}\special{pa -363 699}\special{pa -355 702}\special{fp}%
\special{pa -327 716}\special{pa -320 719}\special{fp}\special{pa -291 731}\special{pa -284 734}\special{fp}%
\special{pa -255 745}\special{pa -247 747}\special{fp}\special{pa -217 757}\special{pa -209 759}\special{fp}%
\special{pa -179 766}\special{pa -171 768}\special{fp}\special{pa -141 774}\special{pa -133 776}\special{fp}%
\special{pa -102 781}\special{pa -94 782}\special{fp}\special{pa -63 785}\special{pa -55 785}\special{fp}%
\special{pa -24 787}\special{pa -16 787}\special{fp}\special{pa 16 787}\special{pa 24 787}\special{fp}%
\special{pa 55 785}\special{pa 63 785}\special{fp}\special{pa 94 782}\special{pa 102 781}\special{fp}%
\special{pa 133 776}\special{pa 141 774}\special{fp}\special{pa 171 768}\special{pa 179 766}\special{fp}%
\special{pa 209 759}\special{pa 217 757}\special{fp}\special{pa 247 747}\special{pa 255 745}\special{fp}%
\special{pa 284 734}\special{pa 291 731}\special{fp}\special{pa 320 719}\special{pa 327 716}\special{fp}%
\special{pa 355 702}\special{pa 363 699}\special{fp}\special{pa 390 684}\special{pa 397 680}\special{fp}%
\special{pa 424 663}\special{pa 430 659}\special{fp}\special{pa 456 642}\special{pa 463 637}\special{fp}%
\special{pa 488 618}\special{pa 494 613}\special{fp}\special{pa 518 593}\special{pa 524 587}\special{fp}%
\special{pa 547 566}\special{pa 552 561}\special{fp}\special{pa 574 538}\special{pa 580 533}\special{fp}%
\special{pa 601 509}\special{pa 606 503}\special{fp}\special{pa 625 478}\special{pa 630 472}\special{fp}%
\special{pa 648 447}\special{pa 653 440}\special{fp}\special{pa 670 414}\special{pa 674 407}\special{fp}%
\special{pa 690 380}\special{pa 693 373}\special{fp}\special{pa 708 345}\special{pa 711 338}\special{fp}%
\special{pa 724 309}\special{pa 727 302}\special{fp}\special{pa 738 273}\special{pa 741 265}\special{fp}%
\special{pa 751 236}\special{pa 753 228}\special{fp}\special{pa 762 198}\special{pa 764 190}\special{fp}%
\special{pa 771 160}\special{pa 772 152}\special{fp}\special{pa 778 121}\special{pa 779 113}\special{fp}%
\special{pa 783 82}\special{pa 784 74}\special{fp}\special{pa 786 43}\special{pa 786 35}\special{fp}%
\special{pa 787 4}\special{pa 788 -4}\special{fp}\special{pn 8}%
\special{pn 12}%
\special{pa   787    -0}\special{pa   787    -7}\special{pa   787   -15}\special{pa   787   -22}%
\special{pa   787   -29}\special{pa   787   -36}\special{pa   786   -44}\special{pa   786   -51}%
\special{pa   785   -58}\special{pa   785   -65}\special{pa   784   -73}\special{pa   783   -80}%
\special{pa   783   -87}\special{pa   782   -94}\special{pa   781  -101}\special{pa   780  -109}%
\special{pa   779  -116}\special{pa   778  -123}\special{pa   777  -130}\special{pa   775  -137}%
\special{pa   774  -144}\special{pa   773  -152}\special{pa   771  -159}\special{pa   770  -166}%
\special{pa   768  -173}\special{pa   767  -180}\special{pa   765  -187}\special{pa   763  -194}%
\special{pa   761  -201}\special{pa   759  -208}\special{pa   757  -215}\special{pa   755  -222}%
\special{pa   753  -229}\special{pa   751  -236}\special{pa   749  -243}\special{pa   747  -250}%
\special{pa   744  -257}\special{pa   742  -264}\special{pa   739  -270}\special{pa   737  -277}%
\special{pa   734  -284}\special{pa   732  -291}\special{pa   729  -298}\special{pa   726  -304}%
\special{pa   723  -311}\special{pa   720  -318}\special{pa   718  -324}\special{pa   715  -331}%
\special{pa   711  -337}\special{pa   708  -344}\special{pa   705  -351}\special{pa   702  -357}%
\special{pa   698  -363}\special{pa   695  -370}\special{pa   692  -376}\special{pa   688  -383}%
\special{pa   685  -389}\special{pa   681  -395}\special{pa   677  -402}\special{pa   674  -408}%
\special{pa   670  -414}\special{pa   666  -420}\special{pa   662  -426}\special{pa   658  -432}%
\special{pa   654  -438}\special{pa   650  -444}\special{pa   646  -450}\special{pa   642  -456}%
\special{pa   637  -462}\special{pa   633  -468}\special{pa   629  -474}\special{pa   624  -480}%
\special{pa   620  -485}\special{pa   615  -491}\special{pa   611  -497}\special{pa   606  -502}%
\special{pa   602  -508}\special{pa   597  -514}\special{pa   592  -519}\special{pa   587  -524}%
\special{pa   582  -530}\special{pa   578  -535}\special{pa   573  -541}\special{pa   568  -546}%
\special{pa   563  -551}\special{pa   557  -556}\special{pa   552  -561}\special{pa   547  -566}%
\special{pa   542  -571}\special{pa   536  -576}\special{pa   531  -581}\special{pa   526  -586}%
\special{pa   520  -591}\special{pa   515  -596}\special{pa   509  -600}\special{pa   504  -605}%
\special{pa   498  -610}\special{pa   493  -614}\special{pa   487  -619}\special{pa   481  -623}%
\special{pa   475  -628}%
\special{fp}%
\special{pn 8}%
\settowidth{\Width}{$\theta$}\setlength{\Width}{-0.5\Width}%
\settoheight{\Height}{$\theta$}\settodepth{\Depth}{$\theta$}\setlength{\Height}{-0.5\Height}\setlength{\Depth}{0.5\Depth}\addtolength{\Height}{\Depth}%
\put(0.6700000,0.3300000){\hspace*{\Width}\raisebox{\Height}{$\theta$}}%
%
\special{pa   197    -0}\special{pa   195   -25}\special{pa   191   -49}\special{pa   183   -72}%
\special{pa   173   -95}\special{pa   159  -116}\special{pa   143  -135}\special{pa   125  -152}%
\special{pa   119  -157}%
\special{fp}%
\settowidth{\Width}{$r$}\setlength{\Width}{-0.5\Width}%
\settoheight{\Height}{$r$}\settodepth{\Depth}{$r$}\setlength{\Height}{-0.5\Height}\setlength{\Depth}{0.5\Depth}\addtolength{\Height}{\Depth}%
\put(1.0000000,-0.2000000){\hspace*{\Width}\raisebox{\Height}{$r$}}%
%
\special{pn 5}%
\special{pa    -0    -0}\special{pa     6     2}\special{pa    11     5}\special{pa    17     7}%
\special{pa    23     9}\special{pa    28    11}\special{pa    34    13}\special{pa    40    16}%
\special{pa    46    18}\special{pa    51    20}\special{pa    57    22}\special{pa    63    24}%
\special{pa    69    26}\special{pa    74    28}\special{pa    80    30}\special{pa    86    31}%
\special{pa    92    33}\special{pa    98    35}\special{pa   104    37}\special{pa   110    39}%
\special{pa   115    40}\special{pa   121    42}\special{pa   127    43}\special{pa   133    45}%
\special{pa   139    47}\special{pa   145    48}\special{pa   151    50}\special{pa   157    51}%
\special{pa   163    52}\special{pa   169    54}\special{pa   175    55}\special{pa   181    56}%
\special{pa   187    58}\special{pa   193    59}\special{pa   199    60}\special{pa   205    61}%
\special{pa   211    62}\special{pa   217    63}\special{pa   223    64}\special{pa   229    65}%
\special{pa   235    66}\special{pa   241    67}\special{pa   247    68}\special{pa   253    69}%
\special{pa   259    70}\special{pa   265    71}\special{pa   271    71}\special{pa   277    72}%
\special{pa   283    73}\special{pa   289    73}\special{pa   295    74}%
\special{fp}%
\special{pa   492    74}\special{pa   498    73}\special{pa   504    73}\special{pa   510    72}%
\special{pa   516    71}\special{pa   522    71}\special{pa   528    70}\special{pa   534    69}%
\special{pa   541    68}\special{pa   547    67}\special{pa   553    66}\special{pa   559    65}%
\special{pa   565    64}\special{pa   571    63}\special{pa   577    62}\special{pa   583    61}%
\special{pa   589    60}\special{pa   595    59}\special{pa   601    58}\special{pa   607    56}%
\special{pa   613    55}\special{pa   619    54}\special{pa   625    52}\special{pa   631    51}%
\special{pa   637    50}\special{pa   642    48}\special{pa   648    47}\special{pa   654    45}%
\special{pa   660    43}\special{pa   666    42}\special{pa   672    40}\special{pa   678    39}%
\special{pa   684    37}\special{pa   690    35}\special{pa   695    33}\special{pa   701    31}%
\special{pa   707    30}\special{pa   713    28}\special{pa   719    26}\special{pa   725    24}%
\special{pa   730    22}\special{pa   736    20}\special{pa   742    18}\special{pa   748    16}%
\special{pa   753    13}\special{pa   759    11}\special{pa   765     9}\special{pa   770     7}%
\special{pa   776     5}\special{pa   782     2}\special{pa   787     0}%
\special{fp}%
\special{pn 8}%
\settowidth{\Width}{$\ell$}\setlength{\Width}{-0.5\Width}%
\settoheight{\Height}{$\ell$}\settodepth{\Depth}{$\ell$}\setlength{\Height}{-0.5\Height}\setlength{\Depth}{0.5\Depth}\addtolength{\Height}{\Depth}%
\put(1.9200000,0.9600000){\hspace*{\Width}\raisebox{\Height}{$\ell$}}%
%
\special{pn 5}%
\special{pa   787     0}\special{pa   789    -6}\special{pa   791   -11}\special{pa   792   -17}%
\special{pa   794   -22}\special{pa   795   -28}\special{pa   796   -33}\special{pa   798   -39}%
\special{pa   799   -45}\special{pa   800   -50}\special{pa   801   -56}\special{pa   802   -62}%
\special{pa   803   -67}\special{pa   804   -73}\special{pa   805   -79}\special{pa   806   -84}%
\special{pa   807   -90}\special{pa   807   -96}\special{pa   808  -102}\special{pa   808  -107}%
\special{pa   809  -113}\special{pa   809  -119}\special{pa   809  -125}\special{pa   810  -130}%
\special{pa   810  -136}\special{pa   810  -142}\special{pa   810  -148}\special{pa   810  -153}%
\special{pa   810  -159}\special{pa   810  -165}\special{pa   810  -171}\special{pa   809  -176}%
\special{pa   809  -182}\special{pa   809  -188}\special{pa   808  -194}\special{pa   808  -199}%
\special{pa   807  -205}\special{pa   806  -211}\special{pa   806  -217}\special{pa   805  -222}%
\special{pa   804  -228}\special{pa   803  -234}\special{pa   802  -239}\special{pa   801  -245}%
\special{pa   800  -251}\special{pa   799  -256}\special{pa   798  -262}\special{pa   796  -268}%
\special{pa   795  -273}\special{pa   793  -279}\special{pa   792  -284}%
\special{fp}%
\special{pa   705  -460}\special{pa   701  -464}\special{pa   698  -469}\special{pa   694  -473}%
\special{pa   690  -478}\special{pa   687  -482}\special{pa   683  -486}\special{pa   679  -491}%
\special{pa   675  -495}\special{pa   671  -499}\special{pa   667  -503}\special{pa   663  -507}%
\special{pa   659  -511}\special{pa   655  -516}\special{pa   651  -519}\special{pa   647  -523}%
\special{pa   642  -527}\special{pa   638  -531}\special{pa   634  -535}\special{pa   629  -539}%
\special{pa   625  -542}\special{pa   620  -546}\special{pa   616  -550}\special{pa   611  -553}%
\special{pa   607  -557}\special{pa   602  -560}\special{pa   597  -563}\special{pa   593  -567}%
\special{pa   588  -570}\special{pa   583  -573}\special{pa   578  -576}\special{pa   574  -580}%
\special{pa   569  -583}\special{pa   564  -586}\special{pa   559  -589}\special{pa   554  -591}%
\special{pa   549  -594}\special{pa   544  -597}\special{pa   539  -600}\special{pa   533  -602}%
\special{pa   528  -605}\special{pa   523  -608}\special{pa   518  -610}\special{pa   513  -612}%
\special{pa   507  -615}\special{pa   502  -617}\special{pa   497  -619}\special{pa   492  -622}%
\special{pa   486  -624}\special{pa   481  -626}\special{pa   475  -628}%
\special{fp}%
\special{pn 8}%
\settowidth{\Width}{P}\setlength{\Width}{-0.5\Width}%
\settoheight{\Height}{P}\settodepth{\Depth}{P}\setlength{\Height}{-0.5\Height}\setlength{\Depth}{0.5\Depth}\addtolength{\Height}{\Depth}%
\put(1.3300000,1.7500000){\hspace*{\Width}\raisebox{\Height}{P}}%
%
\end{picture}}%}
\end{layer}

\begin{itemize}
\item
弧の長さ$\ell$と半径$r$の比\ $\theta(ラジアン)=\bunsuu{\ell}{r}$\vspace{-2mm}
\item
半径$r$の円周は$2\pi r$だから\vspace{2mm}\\
 $\mbox{1周の角}(360^{\circ})=\bunsuu{2\pi r}{r}=2\pi$
\item
$\mbox{半周の角}(180^{\circ})=\pi$
\item
比なので単位はない($\sin$などと同じ)\\
 度と区別するときは,ラジアン(rad)を付ける
\end{itemize}

\newslide{度とラジアンの換算}

\vspace*{18mm}

\slidepage
\begin{itemize}
\item
[]1つの角について,$x\text{度}=y$(ラジアン)とする
\item
[]$1$度は$\bunsuu{\pi}{180}$
  $x$度は$\bunsuu{\pi}{180}\times x$
\item
[]$1$は$\bunsuu{180}{\pi}$度
  $y$は$\bunsuu{180}{\pi}\times y$度
\item
[課題]\monban 次の角を変換せよ(整数か$\pi$を含む分数で)\seteda{30}\\
\eda{$3.1416$}\eda{$10\degree$}\eda{$1$}\eda{$60\degree$}
\end{itemize}

\newslide{正弦関数と正弦曲線}

\vspace*{18mm}

\slidepage

\begin{layer}{120}{0}
\putnotes{65}{70}{%%% /polytech22.git/104-0509/presen/fig/fig221045.tex 
%%% Generator=fig22104.cdy 
{\unitlength=1mm%
\begin{picture}%
(92,2.2)(-1,-1.1)%
\linethickness{0.008in}%%
\polyline(-1.00000,-0.01745)(0.84000,0.01466)(2.68000,0.04676)(4.52000,0.07881)(6.36000,0.11078)%
(8.20000,0.14263)(10.04000,0.17434)(11.88000,0.20586)(13.72000,0.23718)(15.56000,0.26825)%
(17.40000,0.29904)(19.24000,0.32953)(21.08000,0.35967)(22.92000,0.38945)(24.76000,0.41882)%
(26.60000,0.44776)(28.44000,0.47624)(30.28000,0.50423)(32.12000,0.53169)(33.96000,0.55861)%
(35.80000,0.58496)(37.64000,0.61070)(39.48000,0.63581)(41.32000,0.66026)(43.16000,0.68404)%
(45.00000,0.70711)(46.84000,0.72945)(48.68000,0.75103)(50.52000,0.77185)(52.36000,0.79186)%
(54.20000,0.81106)(56.04000,0.82943)(57.88000,0.84694)(59.72000,0.86357)(61.56000,0.87932)%
(63.40000,0.89415)(65.24000,0.90807)(67.08000,0.92105)(68.92000,0.93308)(70.76000,0.94415)%
(72.60000,0.95424)(74.44000,0.96335)(76.28000,0.97147)(78.12000,0.97858)(79.96000,0.98469)%
(81.80000,0.98978)(83.64000,0.99385)(85.48000,0.99689)(87.32000,0.99891)(89.16000,0.99989)%
(91.00000,0.99985)%
%
\polyline(45.00000,0.50000)(45.00000,-0.50000)%
%
\settowidth{\Width}{$45$}\setlength{\Width}{-0.5\Width}%
\settoheight{\Height}{$45$}\settodepth{\Depth}{$45$}\setlength{\Height}{-\Height}%
\put(45.0000000,-1.0000000){\hspace*{\Width}\raisebox{\Height}{$45$}}%
%
\polyline(90.00000,0.50000)(90.00000,-0.50000)%
%
\settowidth{\Width}{$90$}\setlength{\Width}{-0.5\Width}%
\settoheight{\Height}{$90$}\settodepth{\Depth}{$90$}\setlength{\Height}{-\Height}%
\put(90.0000000,-1.0000000){\hspace*{\Width}\raisebox{\Height}{$90$}}%
%
\polyline(0.50000,-1.00000)(-0.50000,-1.00000)%
%
\settowidth{\Width}{$$}\setlength{\Width}{-1\Width}%
\settoheight{\Height}{$$}\settodepth{\Depth}{$$}\setlength{\Height}{-0.5\Height}\setlength{\Depth}{0.5\Depth}\addtolength{\Height}{\Depth}%
\put(-1.0000000,-1.0000000){\hspace*{\Width}\raisebox{\Height}{$$}}%
%
\polyline(0.50000,1.00000)(-0.50000,1.00000)%
%
\settowidth{\Width}{$$}\setlength{\Width}{-1\Width}%
\settoheight{\Height}{$$}\settodepth{\Depth}{$$}\setlength{\Height}{-0.5\Height}\setlength{\Depth}{0.5\Depth}\addtolength{\Height}{\Depth}%
\put(-1.0000000,1.0000000){\hspace*{\Width}\raisebox{\Height}{$$}}%
%
\polyline(-1.00000,0.00000)(91.00000,0.00000)%
%
\polyline(0.00000,-1.10000)(0.00000,1.10000)%
%
\settowidth{\Width}{$x$}\setlength{\Width}{0\Width}%
\settoheight{\Height}{$x$}\settodepth{\Depth}{$x$}\setlength{\Height}{-0.5\Height}\setlength{\Depth}{0.5\Depth}\addtolength{\Height}{\Depth}%
\put(91.5000000,0.0000000){\hspace*{\Width}\raisebox{\Height}{$x$}}%
%
\settowidth{\Width}{$y$}\setlength{\Width}{-0.5\Width}%
\settoheight{\Height}{$y$}\settodepth{\Depth}{$y$}\setlength{\Height}{\Depth}%
\put(0.0000000,1.6000000){\hspace*{\Width}\raisebox{\Height}{$y$}}%
%
\settowidth{\Width}{O}\setlength{\Width}{-1\Width}%
\settoheight{\Height}{O}\settodepth{\Depth}{O}\setlength{\Height}{-\Height}%
\put(-0.5000000,-0.5000000){\hspace*{\Width}\raisebox{\Height}{O}}%
%
\end{picture}}%}
\end{layer}

\begin{itemize}
\item
一般角を$x$とおく.
\item
任意の$x$に対して,$y=\sin x$の値が定まる.
\item
これを正弦関数という(三角関数の1つ).
\item
$y=\sin x$のグラフを正弦曲線という.
\item
{\color{red}$x$はラジアンとする.}\vspace{-2mm}
\item
[] 横軸を度とすると下図になってしまう
\end{itemize}

\newslide{$y=\sin x$のグラフ}

\vspace*{18mm}

\slidepage

\begin{layer}{120}{0}
\putnotese{80}{3}{\scalebox{0.8}{%%% /polytech22.git/104-0509/presen/fig/fig221046b.tex 
%%% Generator=fig22104.cdy 
{\unitlength=24mm%
\begin{picture}%
(2.4,2.4)(-1.2,-1.2)%
\linethickness{0.008in}%%
\polyline(1.00000,0.00000)(0.99211,0.12533)(0.96858,0.24869)(0.92978,0.36812)(0.87631,0.48175)%
(0.80902,0.58779)(0.72897,0.68455)(0.63742,0.77051)(0.53583,0.84433)(0.42578,0.90483)%
(0.30902,0.95106)(0.18738,0.98229)(0.06279,0.99803)(-0.06279,0.99803)(-0.18738,0.98229)%
(-0.30902,0.95106)(-0.42578,0.90483)(-0.53583,0.84433)(-0.63742,0.77051)(-0.72897,0.68455)%
(-0.80902,0.58779)(-0.87631,0.48175)(-0.92978,0.36812)(-0.96858,0.24869)(-0.99211,0.12533)%
(-1.00000,-0.00000)(-0.99211,-0.12533)(-0.96858,-0.24869)(-0.92978,-0.36812)(-0.87631,-0.48175)%
(-0.80902,-0.58779)(-0.72897,-0.68455)(-0.63742,-0.77051)(-0.53583,-0.84433)(-0.42578,-0.90483)%
(-0.30902,-0.95106)(-0.18738,-0.98229)(-0.06279,-0.99803)(0.06279,-0.99803)(0.18738,-0.98229)%
(0.30902,-0.95106)(0.42578,-0.90483)(0.53583,-0.84433)(0.63742,-0.77051)(0.72897,-0.68455)%
(0.80902,-0.58779)(0.87631,-0.48175)(0.92978,-0.36812)(0.96858,-0.24869)(0.99211,-0.12533)%
(1.00000,-0.00000)%
%
\polyline(0.00000,0.00000)(0.58779,0.80902)%
%
\settowidth{\Width}{$\mbox{P}(X,Y)$}\setlength{\Width}{0\Width}%
\settoheight{\Height}{$\mbox{P}(X,Y)$}\settodepth{\Depth}{$\mbox{P}(X,Y)$}\setlength{\Height}{\Depth}%
\put(0.6108333,0.8308333){\hspace*{\Width}\raisebox{\Height}{$\mbox{P}(X,Y)$}}%
%
\polyline(0.25000,0.00000)(0.24803,0.03133)(0.24215,0.06217)(0.23244,0.09203)(0.21908,0.12044)%
(0.20225,0.14695)(0.18224,0.17114)(0.15936,0.19263)(0.14666,0.20186)%
%
\settowidth{\Width}{$x$}\setlength{\Width}{-0.5\Width}%
\settoheight{\Height}{$x$}\settodepth{\Depth}{$x$}\setlength{\Height}{-0.5\Height}\setlength{\Depth}{0.5\Depth}\addtolength{\Height}{\Depth}%
\put(0.3200000,0.1600000){\hspace*{\Width}\raisebox{\Height}{$x$}}%
%
\polyline(-1.00000,0.02083)(-1.00000,-0.02083)%
%
\settowidth{\Width}{$-1$}\setlength{\Width}{-1\Width}%
\settoheight{\Height}{$-1$}\settodepth{\Depth}{$-1$}\setlength{\Height}{-\Height}%
\put(-1.0208333,-0.0208333){\hspace*{\Width}\raisebox{\Height}{$-1$}}%
%
\polyline(1.00000,0.02083)(1.00000,-0.02083)%
%
\settowidth{\Width}{$1$}\setlength{\Width}{0\Width}%
\settoheight{\Height}{$1$}\settodepth{\Depth}{$1$}\setlength{\Height}{-\Height}%
\put(1.0208333,-0.0208333){\hspace*{\Width}\raisebox{\Height}{$1$}}%
%
\polyline(0.02083,-1.00000)(-0.02083,-1.00000)%
%
\settowidth{\Width}{$-1$}\setlength{\Width}{-1\Width}%
\settoheight{\Height}{$-1$}\settodepth{\Depth}{$-1$}\setlength{\Height}{-\Height}%
\put(-0.0208333,-1.0208333){\hspace*{\Width}\raisebox{\Height}{$-1$}}%
%
\polyline(0.02083,1.00000)(-0.02083,1.00000)%
%
\settowidth{\Width}{$1$}\setlength{\Width}{-1\Width}%
\settoheight{\Height}{$1$}\settodepth{\Depth}{$1$}\setlength{\Height}{\Depth}%
\put(-0.0208333,1.0208333){\hspace*{\Width}\raisebox{\Height}{$1$}}%
%
{%
\color[cmyk]{0,1,1,0}%
\linethickness{0.016in}%%
\polyline(0.00000,0.00000)(0.58779,0.00000)%
%
\linethickness{0.008in}%%
}%
\settowidth{\Width}{(1)}\setlength{\Width}{-0.5\Width}%
\settoheight{\Height}{(1)}\settodepth{\Depth}{(1)}\setlength{\Height}{-\Height}%
\put(0.2900000,-0.0208333){\hspace*{\Width}\raisebox{\Height}{(1)}}%
%
{%
\color[cmyk]{1,0,0,0}%
\linethickness{0.016in}%%
\polyline(0.58779,0.80902)(0.58779,0.00000)%
%
\linethickness{0.008in}%%
}%
\settowidth{\Width}{(2)}\setlength{\Width}{-1\Width}%
\settoheight{\Height}{(2)}\settodepth{\Depth}{(2)}\setlength{\Height}{-0.5\Height}\setlength{\Depth}{0.5\Depth}\addtolength{\Height}{\Depth}%
\put(0.5900000,0.4000000){\hspace*{\Width}\raisebox{\Height}{(2)}}%
%
{%
\color[cmyk]{0,1,1,0}%
\linethickness{0.016in}%%
\polyline(0.58779,0.80902)(1.00000,0.00000)%
%
\linethickness{0.008in}%%
}%
\settowidth{\Width}{(3)}\setlength{\Width}{0\Width}%
\settoheight{\Height}{(3)}\settodepth{\Depth}{(3)}\setlength{\Height}{-\Height}%
\put(0.6525000,0.3791667){\hspace*{\Width}\raisebox{\Height}{(3)}}%
%
{%
\color[cmyk]{1,0,0,0}%
\linethickness{0.016in}%%
\polyline(1.00000,0.00000)(0.99982,0.01885)(0.99929,0.03769)(0.99840,0.05652)(0.99716,0.07533)%
(0.99556,0.09411)(0.99361,0.11286)(0.99131,0.13156)(0.98865,0.15023)(0.98564,0.16883)%
(0.98229,0.18738)(0.97858,0.20586)(0.97453,0.22427)(0.97013,0.24260)(0.96538,0.26084)%
(0.96029,0.27899)(0.95486,0.29704)(0.94910,0.31499)(0.94299,0.33282)(0.93655,0.35053)%
(0.92978,0.36812)(0.92267,0.38558)(0.91524,0.40291)(0.90748,0.42009)(0.89941,0.43712)%
(0.89101,0.45399)(0.88229,0.47070)(0.87326,0.48725)(0.86392,0.50362)(0.85428,0.51982)%
(0.84433,0.53583)(0.83408,0.55165)(0.82353,0.56727)(0.81269,0.58269)(0.80157,0.59790)%
(0.79016,0.61291)(0.77846,0.62769)(0.76649,0.64225)(0.75425,0.65659)(0.74174,0.67069)%
(0.72897,0.68455)(0.71594,0.69817)(0.70265,0.71154)(0.68911,0.72465)(0.67533,0.73751)%
(0.66131,0.75011)(0.64706,0.76244)(0.63257,0.77450)(0.61786,0.78629)(0.60293,0.79779)%
(0.58779,0.80902)%
%
\linethickness{0.008in}%%
}%
\settowidth{\Width}{(4)}\setlength{\Width}{0\Width}%
\settoheight{\Height}{(4)}\settodepth{\Depth}{(4)}\setlength{\Height}{\Depth}%
\put(0.8900000,0.4500000){\hspace*{\Width}\raisebox{\Height}{(4)}}%
%
\polyline(-1.20000,0.00000)(1.20000,0.00000)%
%
\polyline(0.00000,-1.20000)(0.00000,1.20000)%
%
\settowidth{\Width}{$ $}\setlength{\Width}{0\Width}%
\settoheight{\Height}{$ $}\settodepth{\Depth}{$ $}\setlength{\Height}{-0.5\Height}\setlength{\Depth}{0.5\Depth}\addtolength{\Height}{\Depth}%
\put(1.2208333,0.0000000){\hspace*{\Width}\raisebox{\Height}{$ $}}%
%
\settowidth{\Width}{$ $}\setlength{\Width}{-0.5\Width}%
\settoheight{\Height}{$ $}\settodepth{\Depth}{$ $}\setlength{\Height}{\Depth}%
\put(0.0000000,1.2208333){\hspace*{\Width}\raisebox{\Height}{$ $}}%
%
\settowidth{\Width}{O}\setlength{\Width}{-1\Width}%
\settoheight{\Height}{O}\settodepth{\Depth}{O}\setlength{\Height}{-\Height}%
\put(-0.0208333,-0.0208333){\hspace*{\Width}\raisebox{\Height}{O}}%
%
\end{picture}}%}}
\end{layer}

{\color{red}

\begin{layer}{120}{0}
\qarrowline[8]{43}{26}{33}{150}{40}
\circleline{46}{25}{1}
\qarrowline[8]{35}{43}{38}{122}{40}
\circleline{37}{44}{1}
\end{layer}

}
\begin{itemize}
\item
{\color{red}半径$1$}の円に点$\mathrm{P}(X,Y)$をとる
\item
[]\hspace*{3zw}$\sin x=\bunsuu{Y}{r}=Y$
\item
また弧の長さを$\ell$とすると\\
\hspace*{3zw}$x=\bunsuu{\ell}{r}=\ell$
\item
[課題]\monban $x,\ \sin x$は\\(1)-(4)のどの長さで表されるか.\seteda{40}\\
\eda{$x$は}\eda{$\sin x$は}
\end{itemize}

\newslide{正弦曲線を描く}

\vspace*{18mm}

\slidepage
\begin{itemize}
\item
アプリ「$y=\sin x$のグラフ」を動かしてみよう\vspace{-2mm}
\item
使い方\vspace{-2mm}
\begin{enumerate}[(1)]
\item
学生番号を入れる\vspace{-2mm}
\item
赤い点を動かして$x$を決め,「点を打つ」\\
 長さが$x$の弧を表示して$(x,\sin x)$に点を打つ.\vspace{-2mm}
\item
いくつかの点を打って「点を結ぶ」\\
 正弦曲線との違いが表示される\\
 さらに「点を打つ」,「点を結ぶ」を繰り返す.\vspace{-2mm}
\end{enumerate}
\item
[課題]\monban 「REC」を押して表示されるデータを提出せよ.
\end{itemize}
%%%%%%%%%%%%

%%%%%%%%%%%%%%%%%%%%


\newslide{正弦曲線の特徴}

\vspace*{18mm}

\slidepage

\begin{layer}{120}{0}
\putnotes{60}{5}{%%% /polytech22.git/104-0509/presen/fig/graphsin.tex 
%%% Generator=graphsincos.cdy 
{\unitlength=1cm%
\begin{picture}%
(13,2.4)(-6.5,-1.2)%
\linethickness{0.008in}%%
\polyline(-6.50000,1.00000)(-6.40076,1.00000)\polyline(-6.30153,1.00000)(-6.20229,1.00000)%
\polyline(-6.10305,1.00000)(-6.00382,1.00000)\polyline(-5.90458,1.00000)(-5.80534,1.00000)%
\polyline(-5.70611,1.00000)(-5.60687,1.00000)\polyline(-5.50763,1.00000)(-5.40840,1.00000)%
\polyline(-5.30916,1.00000)(-5.20992,1.00000)\polyline(-5.11069,1.00000)(-5.01145,1.00000)%
\polyline(-4.91221,1.00000)(-4.81298,1.00000)\polyline(-4.71374,1.00000)(-4.61450,1.00000)%
\polyline(-4.51527,1.00000)(-4.41603,1.00000)\polyline(-4.31679,1.00000)(-4.21756,1.00000)%
\polyline(-4.11832,1.00000)(-4.01908,1.00000)\polyline(-3.91985,1.00000)(-3.82061,1.00000)%
\polyline(-3.72137,1.00000)(-3.62214,1.00000)\polyline(-3.52290,1.00000)(-3.42366,1.00000)%
\polyline(-3.32443,1.00000)(-3.22519,1.00000)\polyline(-3.12595,1.00000)(-3.02672,1.00000)%
\polyline(-2.92748,1.00000)(-2.82824,1.00000)\polyline(-2.72901,1.00000)(-2.62977,1.00000)%
\polyline(-2.53053,1.00000)(-2.43130,1.00000)\polyline(-2.33206,1.00000)(-2.23282,1.00000)%
\polyline(-2.13359,1.00000)(-2.03435,1.00000)\polyline(-1.93511,1.00000)(-1.83588,1.00000)%
\polyline(-1.73664,1.00000)(-1.63740,1.00000)\polyline(-1.53817,1.00000)(-1.43893,1.00000)%
\polyline(-1.33969,1.00000)(-1.24046,1.00000)\polyline(-1.14122,1.00000)(-1.04198,1.00000)%
\polyline(-0.94275,1.00000)(-0.84351,1.00000)\polyline(-0.74427,1.00000)(-0.64504,1.00000)%
\polyline(-0.54580,1.00000)(-0.44656,1.00000)\polyline(-0.34733,1.00000)(-0.24809,1.00000)%
\polyline(-0.14885,1.00000)(-0.04962,1.00000)\polyline(0.04962,1.00000)(0.14885,1.00000)%
\polyline(0.24809,1.00000)(0.34733,1.00000)\polyline(0.44656,1.00000)(0.54580,1.00000)%
\polyline(0.64504,1.00000)(0.74427,1.00000)\polyline(0.84351,1.00000)(0.94275,1.00000)%
\polyline(1.04198,1.00000)(1.14122,1.00000)\polyline(1.24046,1.00000)(1.33969,1.00000)%
\polyline(1.43893,1.00000)(1.53817,1.00000)\polyline(1.63740,1.00000)(1.73664,1.00000)%
\polyline(1.83588,1.00000)(1.93511,1.00000)\polyline(2.03435,1.00000)(2.13359,1.00000)%
\polyline(2.23282,1.00000)(2.33206,1.00000)\polyline(2.43130,1.00000)(2.53053,1.00000)%
\polyline(2.62977,1.00000)(2.72901,1.00000)\polyline(2.82824,1.00000)(2.92748,1.00000)%
\polyline(3.02672,1.00000)(3.12595,1.00000)\polyline(3.22519,1.00000)(3.32443,1.00000)%
\polyline(3.42366,1.00000)(3.52290,1.00000)\polyline(3.62214,1.00000)(3.72137,1.00000)%
\polyline(3.82061,1.00000)(3.91985,1.00000)\polyline(4.01908,1.00000)(4.11832,1.00000)%
\polyline(4.21756,1.00000)(4.31679,1.00000)\polyline(4.41603,1.00000)(4.51527,1.00000)%
\polyline(4.61450,1.00000)(4.71374,1.00000)\polyline(4.81298,1.00000)(4.91221,1.00000)%
\polyline(5.01145,1.00000)(5.11069,1.00000)\polyline(5.20992,1.00000)(5.30916,1.00000)%
\polyline(5.40840,1.00000)(5.50763,1.00000)\polyline(5.60687,1.00000)(5.70611,1.00000)%
\polyline(5.80534,1.00000)(5.90458,1.00000)\polyline(6.00382,1.00000)(6.10305,1.00000)%
\polyline(6.20229,1.00000)(6.30153,1.00000)\polyline(6.40076,1.00000)(6.50000,1.00000)%
%
%
\polyline(-6.50000,-1.00000)(-6.40076,-1.00000)\polyline(-6.30153,-1.00000)(-6.20229,-1.00000)%
\polyline(-6.10305,-1.00000)(-6.00382,-1.00000)\polyline(-5.90458,-1.00000)(-5.80534,-1.00000)%
\polyline(-5.70611,-1.00000)(-5.60687,-1.00000)\polyline(-5.50763,-1.00000)(-5.40840,-1.00000)%
\polyline(-5.30916,-1.00000)(-5.20992,-1.00000)\polyline(-5.11069,-1.00000)(-5.01145,-1.00000)%
\polyline(-4.91221,-1.00000)(-4.81298,-1.00000)\polyline(-4.71374,-1.00000)(-4.61450,-1.00000)%
\polyline(-4.51527,-1.00000)(-4.41603,-1.00000)\polyline(-4.31679,-1.00000)(-4.21756,-1.00000)%
\polyline(-4.11832,-1.00000)(-4.01908,-1.00000)\polyline(-3.91985,-1.00000)(-3.82061,-1.00000)%
\polyline(-3.72137,-1.00000)(-3.62214,-1.00000)\polyline(-3.52290,-1.00000)(-3.42366,-1.00000)%
\polyline(-3.32443,-1.00000)(-3.22519,-1.00000)\polyline(-3.12595,-1.00000)(-3.02672,-1.00000)%
\polyline(-2.92748,-1.00000)(-2.82824,-1.00000)\polyline(-2.72901,-1.00000)(-2.62977,-1.00000)%
\polyline(-2.53053,-1.00000)(-2.43130,-1.00000)\polyline(-2.33206,-1.00000)(-2.23282,-1.00000)%
\polyline(-2.13359,-1.00000)(-2.03435,-1.00000)\polyline(-1.93511,-1.00000)(-1.83588,-1.00000)%
\polyline(-1.73664,-1.00000)(-1.63740,-1.00000)\polyline(-1.53817,-1.00000)(-1.43893,-1.00000)%
\polyline(-1.33969,-1.00000)(-1.24046,-1.00000)\polyline(-1.14122,-1.00000)(-1.04198,-1.00000)%
\polyline(-0.94275,-1.00000)(-0.84351,-1.00000)\polyline(-0.74427,-1.00000)(-0.64504,-1.00000)%
\polyline(-0.54580,-1.00000)(-0.44656,-1.00000)\polyline(-0.34733,-1.00000)(-0.24809,-1.00000)%
\polyline(-0.14885,-1.00000)(-0.04962,-1.00000)\polyline(0.04962,-1.00000)(0.14885,-1.00000)%
\polyline(0.24809,-1.00000)(0.34733,-1.00000)\polyline(0.44656,-1.00000)(0.54580,-1.00000)%
\polyline(0.64504,-1.00000)(0.74427,-1.00000)\polyline(0.84351,-1.00000)(0.94275,-1.00000)%
\polyline(1.04198,-1.00000)(1.14122,-1.00000)\polyline(1.24046,-1.00000)(1.33969,-1.00000)%
\polyline(1.43893,-1.00000)(1.53817,-1.00000)\polyline(1.63740,-1.00000)(1.73664,-1.00000)%
\polyline(1.83588,-1.00000)(1.93511,-1.00000)\polyline(2.03435,-1.00000)(2.13359,-1.00000)%
\polyline(2.23282,-1.00000)(2.33206,-1.00000)\polyline(2.43130,-1.00000)(2.53053,-1.00000)%
\polyline(2.62977,-1.00000)(2.72901,-1.00000)\polyline(2.82824,-1.00000)(2.92748,-1.00000)%
\polyline(3.02672,-1.00000)(3.12595,-1.00000)\polyline(3.22519,-1.00000)(3.32443,-1.00000)%
\polyline(3.42366,-1.00000)(3.52290,-1.00000)\polyline(3.62214,-1.00000)(3.72137,-1.00000)%
\polyline(3.82061,-1.00000)(3.91985,-1.00000)\polyline(4.01908,-1.00000)(4.11832,-1.00000)%
\polyline(4.21756,-1.00000)(4.31679,-1.00000)\polyline(4.41603,-1.00000)(4.51527,-1.00000)%
\polyline(4.61450,-1.00000)(4.71374,-1.00000)\polyline(4.81298,-1.00000)(4.91221,-1.00000)%
\polyline(5.01145,-1.00000)(5.11069,-1.00000)\polyline(5.20992,-1.00000)(5.30916,-1.00000)%
\polyline(5.40840,-1.00000)(5.50763,-1.00000)\polyline(5.60687,-1.00000)(5.70611,-1.00000)%
\polyline(5.80534,-1.00000)(5.90458,-1.00000)\polyline(6.00382,-1.00000)(6.10305,-1.00000)%
\polyline(6.20229,-1.00000)(6.30153,-1.00000)\polyline(6.40076,-1.00000)(6.50000,-1.00000)%
%
%
\polyline(1.57080,0.05000)(1.57080,-0.05000)%
%
\settowidth{\Width}{$\tfrac{\pi}{2}$}\setlength{\Width}{-0.5\Width}%
\settoheight{\Height}{$\tfrac{\pi}{2}$}\settodepth{\Depth}{$\tfrac{\pi}{2}$}\setlength{\Height}{-\Height}%
\put(1.5700000,-0.1000000){\hspace*{\Width}\raisebox{\Height}{$\tfrac{\pi}{2}$}}%
%
\polyline(3.14159,0.05000)(3.14159,-0.05000)%
%
\settowidth{\Width}{$\pi$}\setlength{\Width}{-0.5\Width}%
\settoheight{\Height}{$\pi$}\settodepth{\Depth}{$\pi$}\setlength{\Height}{-\Height}%
\put(3.1400000,-0.1000000){\hspace*{\Width}\raisebox{\Height}{$\pi$}}%
%
\polyline(6.28319,0.05000)(6.28319,-0.05000)%
%
\settowidth{\Width}{$2\pi$}\setlength{\Width}{-0.5\Width}%
\settoheight{\Height}{$2\pi$}\settodepth{\Depth}{$2\pi$}\setlength{\Height}{-\Height}%
\put(6.2800000,-0.1000000){\hspace*{\Width}\raisebox{\Height}{$2\pi$}}%
%
\polyline(-1.57080,0.05000)(-1.57080,-0.05000)%
%
\settowidth{\Width}{$-\tfrac{\pi}{2}$}\setlength{\Width}{-0.5\Width}%
\settoheight{\Height}{$-\tfrac{\pi}{2}$}\settodepth{\Depth}{$-\tfrac{\pi}{2}$}\setlength{\Height}{-\Height}%
\put(-1.5700000,-0.1000000){\hspace*{\Width}\raisebox{\Height}{$-\tfrac{\pi}{2}$}}%
%
\polyline(-3.14159,0.05000)(-3.14159,-0.05000)%
%
\settowidth{\Width}{$-\pi$}\setlength{\Width}{-0.5\Width}%
\settoheight{\Height}{$-\pi$}\settodepth{\Depth}{$-\pi$}\setlength{\Height}{-\Height}%
\put(-3.1400000,-0.1000000){\hspace*{\Width}\raisebox{\Height}{$-\pi$}}%
%
\polyline(-6.28319,0.05000)(-6.28319,-0.05000)%
%
\settowidth{\Width}{$-2\pi$}\setlength{\Width}{-0.5\Width}%
\settoheight{\Height}{$-2\pi$}\settodepth{\Depth}{$-2\pi$}\setlength{\Height}{-\Height}%
\put(-6.2800000,-0.1000000){\hspace*{\Width}\raisebox{\Height}{$-2\pi$}}%
%
\polyline(0.05000,-1.00000)(-0.05000,-1.00000)%
%
\settowidth{\Width}{$-1$}\setlength{\Width}{-1\Width}%
\settoheight{\Height}{$-1$}\settodepth{\Depth}{$-1$}\setlength{\Height}{-0.5\Height}\setlength{\Depth}{0.5\Depth}\addtolength{\Height}{\Depth}%
\put(-0.1000000,-1.0000000){\hspace*{\Width}\raisebox{\Height}{$-1$}}%
%
\polyline(0.05000,1.00000)(-0.05000,1.00000)%
%
\settowidth{\Width}{$1$}\setlength{\Width}{-1\Width}%
\settoheight{\Height}{$1$}\settodepth{\Depth}{$1$}\setlength{\Height}{-0.5\Height}\setlength{\Depth}{0.5\Depth}\addtolength{\Height}{\Depth}%
\put(-0.1000000,1.0000000){\hspace*{\Width}\raisebox{\Height}{$1$}}%
%
\polyline(-6.50000,-0.21512)(-6.43500,-0.15123)(-6.37000,-0.08671)(-6.30500,-0.02181)%
(-6.24000,0.04317)(-6.17500,0.10797)(-6.11000,0.17232)(-6.04500,0.23594)(-5.98000,0.29856)%
(-5.91500,0.35992)(-5.85000,0.41976)(-5.78500,0.47783)(-5.72000,0.53388)(-5.65500,0.58768)%
(-5.59000,0.63899)(-5.52500,0.68760)(-5.46000,0.73332)(-5.39500,0.77593)(-5.33000,0.81526)%
(-5.26500,0.85116)(-5.20000,0.88345)(-5.13500,0.91202)(-5.07000,0.93674)(-5.00500,0.95749)%
(-4.94000,0.97421)(-4.87500,0.98681)(-4.81000,0.99524)(-4.74500,0.99947)(-4.68000,0.99948)%
(-4.61500,0.99526)(-4.55000,0.98684)(-4.48500,0.97426)(-4.42000,0.95756)(-4.35500,0.93681)%
(-4.29000,0.91211)(-4.22500,0.88356)(-4.16000,0.85127)(-4.09500,0.81539)(-4.03000,0.77607)%
(-3.96500,0.73347)(-3.90000,0.68777)(-3.83500,0.63916)(-3.77000,0.58786)(-3.70500,0.53407)%
(-3.64000,0.47803)(-3.57500,0.41997)(-3.51000,0.36013)(-3.44500,0.29877)(-3.38000,0.23616)%
(-3.31500,0.17254)(-3.25000,0.10820)(-3.18500,0.04339)(-3.12000,-0.02159)(-3.05500,-0.08648)%
(-2.99000,-0.15101)(-2.92500,-0.21490)(-2.86000,-0.27789)(-2.79500,-0.33970)(-2.73000,-0.40007)%
(-2.66500,-0.45875)(-2.60000,-0.51550)(-2.53500,-0.57007)(-2.47000,-0.62223)(-2.40500,-0.67177)%
(-2.34000,-0.71846)(-2.27500,-0.76213)(-2.21000,-0.80257)(-2.14500,-0.83963)(-2.08000,-0.87313)%
(-2.01500,-0.90295)(-1.95000,-0.92896)(-1.88500,-0.95104)(-1.82000,-0.96911)(-1.75500,-0.98308)%
(-1.69000,-0.99290)(-1.62500,-0.99853)(-1.56000,-0.99994)(-1.49500,-0.99713)(-1.43000,-0.99010)%
(-1.36500,-0.97890)(-1.30000,-0.96356)(-1.23500,-0.94415)(-1.17000,-0.92075)(-1.10500,-0.89346)%
(-1.04000,-0.86240)(-0.97500,-0.82770)(-0.91000,-0.78950)(-0.84500,-0.74797)(-0.78000,-0.70328)%
(-0.71500,-0.65562)(-0.65000,-0.60519)(-0.58500,-0.55220)(-0.52000,-0.49688)(-0.45500,-0.43946)%
(-0.39000,-0.38019)(-0.32500,-0.31931)(-0.26000,-0.25708)(-0.19500,-0.19377)(-0.13000,-0.12963)%
(-0.06500,-0.06495)(0.00000,0.00000)(0.06500,0.06495)(0.13000,0.12963)(0.19500,0.19377)%
(0.26000,0.25708)(0.32500,0.31931)(0.39000,0.38019)(0.45500,0.43946)(0.52000,0.49688)%
(0.58500,0.55220)(0.65000,0.60519)(0.71500,0.65562)(0.78000,0.70328)(0.84500,0.74797)%
(0.91000,0.78950)(0.97500,0.82770)(1.04000,0.86240)(1.10500,0.89346)(1.17000,0.92075)%
(1.23500,0.94415)(1.30000,0.96356)(1.36500,0.97890)(1.43000,0.99010)(1.49500,0.99713)%
(1.56000,0.99994)(1.62500,0.99853)(1.69000,0.99290)(1.75500,0.98308)(1.82000,0.96911)%
(1.88500,0.95104)(1.95000,0.92896)(2.01500,0.90295)(2.08000,0.87313)(2.14500,0.83963)%
(2.21000,0.80257)(2.27500,0.76213)(2.34000,0.71846)(2.40500,0.67177)(2.47000,0.62223)%
(2.53500,0.57007)(2.60000,0.51550)(2.66500,0.45875)(2.73000,0.40007)(2.79500,0.33970)%
(2.86000,0.27789)(2.92500,0.21490)(2.99000,0.15101)(3.05500,0.08648)(3.12000,0.02159)%
(3.18500,-0.04339)(3.25000,-0.10820)(3.31500,-0.17254)(3.38000,-0.23616)(3.44500,-0.29877)%
(3.51000,-0.36013)(3.57500,-0.41997)(3.64000,-0.47803)(3.70500,-0.53407)(3.77000,-0.58786)%
(3.83500,-0.63916)(3.90000,-0.68777)(3.96500,-0.73347)(4.03000,-0.77607)(4.09500,-0.81539)%
(4.16000,-0.85127)(4.22500,-0.88356)(4.29000,-0.91211)(4.35500,-0.93681)(4.42000,-0.95756)%
(4.48500,-0.97426)(4.55000,-0.98684)(4.61500,-0.99526)(4.68000,-0.99948)(4.74500,-0.99947)%
(4.81000,-0.99524)(4.87500,-0.98681)(4.94000,-0.97421)(5.00500,-0.95749)(5.07000,-0.93674)%
(5.13500,-0.91202)(5.20000,-0.88345)(5.26500,-0.85116)(5.33000,-0.81526)(5.39500,-0.77593)%
(5.46000,-0.73332)(5.52500,-0.68760)(5.59000,-0.63899)(5.65500,-0.58768)(5.72000,-0.53388)%
(5.78500,-0.47783)(5.85000,-0.41976)(5.91500,-0.35992)(5.98000,-0.29856)(6.04500,-0.23594)%
(6.11000,-0.17232)(6.17500,-0.10797)(6.24000,-0.04317)(6.30500,0.02181)(6.37000,0.08671)%
(6.43500,0.15123)(6.50000,0.21512)%
%
\polyline(-6.50000,0.00000)(6.50000,0.00000)%
%
\polyline(0.00000,-1.20000)(0.00000,1.20000)%
%
\settowidth{\Width}{$x$}\setlength{\Width}{0\Width}%
\settoheight{\Height}{$x$}\settodepth{\Depth}{$x$}\setlength{\Height}{-0.5\Height}\setlength{\Depth}{0.5\Depth}\addtolength{\Height}{\Depth}%
\put(6.5500000,0.0000000){\hspace*{\Width}\raisebox{\Height}{$x$}}%
%
\settowidth{\Width}{$y$}\setlength{\Width}{-0.5\Width}%
\settoheight{\Height}{$y$}\settodepth{\Depth}{$y$}\setlength{\Height}{\Depth}%
\put(0.0000000,1.2500000){\hspace*{\Width}\raisebox{\Height}{$y$}}%
%
\settowidth{\Width}{O}\setlength{\Width}{0\Width}%
\settoheight{\Height}{O}\settodepth{\Depth}{O}\setlength{\Height}{-\Height}%
\put(0.0500000,-0.0500000){\hspace*{\Width}\raisebox{\Height}{O}}%
%
\end{picture}}%}
\end{layer}

\vspace{30mm}
\begin{itemize}
\item
{\color{red}振幅}は$1$(値の範囲は$-1$から$1$)
\item
{\color{red}周期}は$2\pi$($2\pi$で元に戻る)
\item
原点対称
\end{itemize}

\newslide{正弦曲線(課題)}

\vspace*{18mm}

\slidepage
\down
「関数のグラフ」でグラフをかいてみよう.
\begin{itemize}
\item
[課題]\monban 次の関数の振幅と周期を答えよ\seteda{50}\vspace{2mm}\\
\eda{$y=2\sin x$}\eda{$y=\bunsuu{1}{3}\sin x$}\\
\eda{$y=\sin 2x$}\eda{$y=4\sin\bunsuu{x}{2}$}
\item
[課題]\monban 次の関数の振幅と周期を答えよ\seteda{50}\vspace{2mm}\\
\eda{$y=A\sin x$}\eda{$y=\sin bx$}
\end{itemize}
%%%%%%%%%%%%

%%%%%%%%%%%%%%%%%%%%


\newslide{振幅・周期}

\vspace*{18mm}

\slidepage
\begin{itemize}
\item
$y=\sin x$の振幅は$1$,周期は$2\pi$
\item
$y=A \sin x$の振幅は$A$,周期は$2\pi$
\item
$y=\sin(bx)$の振幅は$1$,周期は$\bunsuu{2\pi}{b}$
\end{itemize}

\newslide{位相}

\vspace*{18mm}

\slidepage
\begin{itemize}
\item
「関数のグラフ」でグラフをかいてみよう.
\item
[課題]\monbannoadd $y=\sin x$のグラフとの関係を答えよ.\seteda{60}\\
\eda{$y=\sin(x-1)$}\eda{$y=\sin(x-2)$}\\
\eda{$y=\sin(x+1)$}\eda{$y=\sin(x+\bunsuu{\pi}{2})$}
\item
$y=\sin(x-c)$は$y=\sin x$を\\
\hspace*{3zw}右方向に$c$だけ平行移動 {\color{red}位相が$c$だけ遅れる}
\item
$y=\sin(x+c)$は$y=\sin x$を\\
\hspace*{3zw}左方向に$c$だけ平行移動 {\color{red}位相が$-c$だけ進む}
\end{itemize}

\newslide{$y=\cos x$のグラフ(余弦曲線)}

\vspace*{18mm}

\slidepage

\begin{layer}{120}{0}
\putnotes{62}{6}{%%% /polytech.git/n103/fig/graphcos.tex 
%%% Generator=graphsincos.cdy 
{\unitlength=1cm%
\begin{picture}%
(13,2.4)(-6.5,-1.2)%
\special{pn 8}%
%
\small%
{%
\color[rgb]{0,0,0}%
\special{pa -2559  -384}\special{pa -2533  -389}\special{pa -2508  -392}\special{pa -2482  -394}%
\special{pa -2457  -393}\special{pa -2431  -391}\special{pa -2406  -388}\special{pa -2380  -383}%
\special{pa -2354  -376}\special{pa -2329  -367}\special{pa -2303  -357}\special{pa -2278  -346}%
\special{pa -2252  -333}\special{pa -2226  -319}\special{pa -2201  -303}\special{pa -2175  -286}%
\special{pa -2150  -268}\special{pa -2124  -248}\special{pa -2098  -228}\special{pa -2073  -207}%
\special{pa -2047  -184}\special{pa -2022  -161}\special{pa -1996  -138}\special{pa -1970  -114}%
\special{pa -1945   -89}\special{pa -1919   -64}\special{pa -1894   -38}\special{pa -1868   -13}%
\special{pa -1843    13}\special{pa -1817    38}\special{pa -1791    64}\special{pa -1766    89}%
\special{pa -1740   113}\special{pa -1715   138}\special{pa -1689   161}\special{pa -1663   184}%
\special{pa -1638   207}\special{pa -1612   228}\special{pa -1587   248}\special{pa -1561   268}%
\special{pa -1535   286}\special{pa -1510   303}\special{pa -1484   318}\special{pa -1459   333}%
\special{pa -1433   346}\special{pa -1407   357}\special{pa -1382   367}\special{pa -1356   376}%
\special{pa -1331   383}\special{pa -1305   388}\special{pa -1280   391}\special{pa -1254   393}%
\special{pa -1228   394}\special{pa -1203   392}\special{pa -1177   389}\special{pa -1152   385}%
\special{pa -1126   378}\special{pa -1100   370}\special{pa -1075   361}\special{pa -1049   350}%
\special{pa -1024   337}\special{pa  -998   323}\special{pa  -972   308}\special{pa  -947   292}%
\special{pa  -921   274}\special{pa  -896   255}\special{pa  -870   235}\special{pa  -844   214}%
\special{pa  -819   192}\special{pa  -793   169}\special{pa  -768   146}\special{pa  -742   122}%
\special{pa  -717    97}\special{pa  -691    72}\special{pa  -665    47}\special{pa  -640    21}%
\special{pa  -614    -4}\special{pa  -589   -30}\special{pa  -563   -55}\special{pa  -537   -80}%
\special{pa  -512  -105}\special{pa  -486  -130}\special{pa  -461  -154}\special{pa  -435  -177}%
\special{pa  -409  -199}\special{pa  -384  -221}\special{pa  -358  -242}\special{pa  -333  -261}%
\special{pa  -307  -280}\special{pa  -281  -297}\special{pa  -256  -313}\special{pa  -230  -328}%
\special{pa  -205  -342}\special{pa  -179  -354}\special{pa  -154  -364}\special{pa  -128  -373}%
\special{pa  -102  -380}\special{pa   -77  -386}\special{pa   -51  -390}\special{pa   -26  -393}%
\special{pa     0  -394}\special{pa    26  -393}\special{pa    51  -390}\special{pa    77  -386}%
\special{pa   102  -380}\special{pa   128  -373}\special{pa   154  -364}\special{pa   179  -354}%
\special{pa   205  -342}\special{pa   230  -328}\special{pa   256  -313}\special{pa   281  -297}%
\special{pa   307  -280}\special{pa   333  -261}\special{pa   358  -242}\special{pa   384  -221}%
\special{pa   409  -199}\special{pa   435  -177}\special{pa   461  -154}\special{pa   486  -130}%
\special{pa   512  -105}\special{pa   537   -80}\special{pa   563   -55}\special{pa   589   -30}%
\special{pa   614    -4}\special{pa   640    21}\special{pa   665    47}\special{pa   691    72}%
\special{pa   717    97}\special{pa   742   122}\special{pa   768   146}\special{pa   793   169}%
\special{pa   819   192}\special{pa   844   214}\special{pa   870   235}\special{pa   896   255}%
\special{pa   921   274}\special{pa   947   292}\special{pa   972   308}\special{pa   998   323}%
\special{pa  1024   337}\special{pa  1049   350}\special{pa  1075   361}\special{pa  1100   370}%
\special{pa  1126   378}\special{pa  1152   385}\special{pa  1177   389}\special{pa  1203   392}%
\special{pa  1228   394}\special{pa  1254   393}\special{pa  1280   391}\special{pa  1305   388}%
\special{pa  1331   383}\special{pa  1356   376}\special{pa  1382   367}\special{pa  1407   357}%
\special{pa  1433   346}\special{pa  1459   333}\special{pa  1484   318}\special{pa  1510   303}%
\special{pa  1535   286}\special{pa  1561   268}\special{pa  1587   248}\special{pa  1612   228}%
\special{pa  1638   207}\special{pa  1663   184}\special{pa  1689   161}\special{pa  1715   138}%
\special{pa  1740   113}\special{pa  1766    89}\special{pa  1791    64}\special{pa  1817    38}%
\special{pa  1843    13}\special{pa  1868   -13}\special{pa  1894   -38}\special{pa  1919   -64}%
\special{pa  1945   -89}\special{pa  1970  -114}\special{pa  1996  -138}\special{pa  2022  -161}%
\special{pa  2047  -184}\special{pa  2073  -207}\special{pa  2098  -228}\special{pa  2124  -248}%
\special{pa  2150  -268}\special{pa  2175  -286}\special{pa  2201  -303}\special{pa  2226  -319}%
\special{pa  2252  -333}\special{pa  2278  -346}\special{pa  2303  -357}\special{pa  2329  -367}%
\special{pa  2354  -376}\special{pa  2380  -383}\special{pa  2406  -388}\special{pa  2431  -391}%
\special{pa  2457  -393}\special{pa  2482  -394}\special{pa  2508  -392}\special{pa  2533  -389}%
\special{pa  2559  -384}%
\special{fp}%
}%
{%
\color[rgb]{0,0,0}%
\special{pa -2559 -394}\special{pa -2520 -394}\special{fp}\special{pa -2481 -394}\special{pa -2442 -394}\special{fp}%
\special{pa -2403 -394}\special{pa -2364 -394}\special{fp}\special{pa -2325 -394}\special{pa -2286 -394}\special{fp}%
\special{pa -2246 -394}\special{pa -2207 -394}\special{fp}\special{pa -2168 -394}\special{pa -2129 -394}\special{fp}%
\special{pa -2090 -394}\special{pa -2051 -394}\special{fp}\special{pa -2012 -394}\special{pa -1973 -394}\special{fp}%
\special{pa -1934 -394}\special{pa -1895 -394}\special{fp}\special{pa -1856 -394}\special{pa -1817 -394}\special{fp}%
\special{pa -1778 -394}\special{pa -1739 -394}\special{fp}\special{pa -1700 -394}\special{pa -1660 -394}\special{fp}%
\special{pa -1621 -394}\special{pa -1582 -394}\special{fp}\special{pa -1543 -394}\special{pa -1504 -394}\special{fp}%
\special{pa -1465 -394}\special{pa -1426 -394}\special{fp}\special{pa -1387 -394}\special{pa -1348 -394}\special{fp}%
\special{pa -1309 -394}\special{pa -1270 -394}\special{fp}\special{pa -1231 -394}\special{pa -1192 -394}\special{fp}%
\special{pa -1153 -394}\special{pa -1113 -394}\special{fp}\special{pa -1074 -394}\special{pa -1035 -394}\special{fp}%
\special{pa -996 -394}\special{pa -957 -394}\special{fp}\special{pa -918 -394}\special{pa -879 -394}\special{fp}%
\special{pa -840 -394}\special{pa -801 -394}\special{fp}\special{pa -762 -394}\special{pa -723 -394}\special{fp}%
\special{pa -684 -394}\special{pa -645 -394}\special{fp}\special{pa -606 -394}\special{pa -567 -394}\special{fp}%
\special{pa -527 -394}\special{pa -488 -394}\special{fp}\special{pa -449 -394}\special{pa -410 -394}\special{fp}%
\special{pa -371 -394}\special{pa -332 -394}\special{fp}\special{pa -293 -394}\special{pa -254 -394}\special{fp}%
\special{pa -215 -394}\special{pa -176 -394}\special{fp}\special{pa -137 -394}\special{pa -98 -394}\special{fp}%
\special{pa -59 -394}\special{pa -20 -394}\special{fp}\special{pa 20 -394}\special{pa 59 -394}\special{fp}%
\special{pa 98 -394}\special{pa 137 -394}\special{fp}\special{pa 176 -394}\special{pa 215 -394}\special{fp}%
\special{pa 254 -394}\special{pa 293 -394}\special{fp}\special{pa 332 -394}\special{pa 371 -394}\special{fp}%
\special{pa 410 -394}\special{pa 449 -394}\special{fp}\special{pa 488 -394}\special{pa 527 -394}\special{fp}%
\special{pa 567 -394}\special{pa 606 -394}\special{fp}\special{pa 645 -394}\special{pa 684 -394}\special{fp}%
\special{pa 723 -394}\special{pa 762 -394}\special{fp}\special{pa 801 -394}\special{pa 840 -394}\special{fp}%
\special{pa 879 -394}\special{pa 918 -394}\special{fp}\special{pa 957 -394}\special{pa 996 -394}\special{fp}%
\special{pa 1035 -394}\special{pa 1074 -394}\special{fp}\special{pa 1113 -394}\special{pa 1153 -394}\special{fp}%
\special{pa 1192 -394}\special{pa 1231 -394}\special{fp}\special{pa 1270 -394}\special{pa 1309 -394}\special{fp}%
\special{pa 1348 -394}\special{pa 1387 -394}\special{fp}\special{pa 1426 -394}\special{pa 1465 -394}\special{fp}%
\special{pa 1504 -394}\special{pa 1543 -394}\special{fp}\special{pa 1582 -394}\special{pa 1621 -394}\special{fp}%
\special{pa 1660 -394}\special{pa 1700 -394}\special{fp}\special{pa 1739 -394}\special{pa 1778 -394}\special{fp}%
\special{pa 1817 -394}\special{pa 1856 -394}\special{fp}\special{pa 1895 -394}\special{pa 1934 -394}\special{fp}%
\special{pa 1973 -394}\special{pa 2012 -394}\special{fp}\special{pa 2051 -394}\special{pa 2090 -394}\special{fp}%
\special{pa 2129 -394}\special{pa 2168 -394}\special{fp}\special{pa 2207 -394}\special{pa 2246 -394}\special{fp}%
\special{pa 2286 -394}\special{pa 2325 -394}\special{fp}\special{pa 2364 -394}\special{pa 2403 -394}\special{fp}%
\special{pa 2442 -394}\special{pa 2481 -394}\special{fp}\special{pa 2520 -394}\special{pa 2559 -394}\special{fp}%
%
%
}%
{%
\color[rgb]{0,0,0}%
\special{pa -2559 394}\special{pa -2520 394}\special{fp}\special{pa -2481 394}\special{pa -2442 394}\special{fp}%
\special{pa -2403 394}\special{pa -2364 394}\special{fp}\special{pa -2325 394}\special{pa -2286 394}\special{fp}%
\special{pa -2246 394}\special{pa -2207 394}\special{fp}\special{pa -2168 394}\special{pa -2129 394}\special{fp}%
\special{pa -2090 394}\special{pa -2051 394}\special{fp}\special{pa -2012 394}\special{pa -1973 394}\special{fp}%
\special{pa -1934 394}\special{pa -1895 394}\special{fp}\special{pa -1856 394}\special{pa -1817 394}\special{fp}%
\special{pa -1778 394}\special{pa -1739 394}\special{fp}\special{pa -1700 394}\special{pa -1660 394}\special{fp}%
\special{pa -1621 394}\special{pa -1582 394}\special{fp}\special{pa -1543 394}\special{pa -1504 394}\special{fp}%
\special{pa -1465 394}\special{pa -1426 394}\special{fp}\special{pa -1387 394}\special{pa -1348 394}\special{fp}%
\special{pa -1309 394}\special{pa -1270 394}\special{fp}\special{pa -1231 394}\special{pa -1192 394}\special{fp}%
\special{pa -1153 394}\special{pa -1113 394}\special{fp}\special{pa -1074 394}\special{pa -1035 394}\special{fp}%
\special{pa -996 394}\special{pa -957 394}\special{fp}\special{pa -918 394}\special{pa -879 394}\special{fp}%
\special{pa -840 394}\special{pa -801 394}\special{fp}\special{pa -762 394}\special{pa -723 394}\special{fp}%
\special{pa -684 394}\special{pa -645 394}\special{fp}\special{pa -606 394}\special{pa -567 394}\special{fp}%
\special{pa -527 394}\special{pa -488 394}\special{fp}\special{pa -449 394}\special{pa -410 394}\special{fp}%
\special{pa -371 394}\special{pa -332 394}\special{fp}\special{pa -293 394}\special{pa -254 394}\special{fp}%
\special{pa -215 394}\special{pa -176 394}\special{fp}\special{pa -137 394}\special{pa -98 394}\special{fp}%
\special{pa -59 394}\special{pa -20 394}\special{fp}\special{pa 20 394}\special{pa 59 394}\special{fp}%
\special{pa 98 394}\special{pa 137 394}\special{fp}\special{pa 176 394}\special{pa 215 394}\special{fp}%
\special{pa 254 394}\special{pa 293 394}\special{fp}\special{pa 332 394}\special{pa 371 394}\special{fp}%
\special{pa 410 394}\special{pa 449 394}\special{fp}\special{pa 488 394}\special{pa 527 394}\special{fp}%
\special{pa 567 394}\special{pa 606 394}\special{fp}\special{pa 645 394}\special{pa 684 394}\special{fp}%
\special{pa 723 394}\special{pa 762 394}\special{fp}\special{pa 801 394}\special{pa 840 394}\special{fp}%
\special{pa 879 394}\special{pa 918 394}\special{fp}\special{pa 957 394}\special{pa 996 394}\special{fp}%
\special{pa 1035 394}\special{pa 1074 394}\special{fp}\special{pa 1113 394}\special{pa 1153 394}\special{fp}%
\special{pa 1192 394}\special{pa 1231 394}\special{fp}\special{pa 1270 394}\special{pa 1309 394}\special{fp}%
\special{pa 1348 394}\special{pa 1387 394}\special{fp}\special{pa 1426 394}\special{pa 1465 394}\special{fp}%
\special{pa 1504 394}\special{pa 1543 394}\special{fp}\special{pa 1582 394}\special{pa 1621 394}\special{fp}%
\special{pa 1660 394}\special{pa 1700 394}\special{fp}\special{pa 1739 394}\special{pa 1778 394}\special{fp}%
\special{pa 1817 394}\special{pa 1856 394}\special{fp}\special{pa 1895 394}\special{pa 1934 394}\special{fp}%
\special{pa 1973 394}\special{pa 2012 394}\special{fp}\special{pa 2051 394}\special{pa 2090 394}\special{fp}%
\special{pa 2129 394}\special{pa 2168 394}\special{fp}\special{pa 2207 394}\special{pa 2246 394}\special{fp}%
\special{pa 2286 394}\special{pa 2325 394}\special{fp}\special{pa 2364 394}\special{pa 2403 394}\special{fp}%
\special{pa 2442 394}\special{pa 2481 394}\special{fp}\special{pa 2520 394}\special{pa 2559 394}\special{fp}%
%
%
}%
{%
\color[rgb]{0,0,0}%
\special{pa   618   -20}\special{pa   618    20}%
\special{fp}%
}%
{%
\color[rgb]{0,0,0}%
\settowidth{\Width}{$\tfrac{\pi}{2}$}\setlength{\Width}{-0.5\Width}%
\settoheight{\Height}{$\tfrac{\pi}{2}$}\settodepth{\Depth}{$\tfrac{\pi}{2}$}\setlength{\Height}{-\Height}%
\put(1.5700000,-0.1000000){\hspace*{\Width}\raisebox{\Height}{$\tfrac{\pi}{2}$}}%
%
}%
{%
\color[rgb]{0,0,0}%
\special{pa  1237   -20}\special{pa  1237    20}%
\special{fp}%
}%
{%
\color[rgb]{0,0,0}%
\settowidth{\Width}{$\pi$}\setlength{\Width}{-0.5\Width}%
\settoheight{\Height}{$\pi$}\settodepth{\Depth}{$\pi$}\setlength{\Height}{-\Height}%
\put(3.1400000,-0.1000000){\hspace*{\Width}\raisebox{\Height}{$\pi$}}%
%
}%
{%
\color[rgb]{0,0,0}%
\special{pa  2474   -20}\special{pa  2474    20}%
\special{fp}%
}%
{%
\color[rgb]{0,0,0}%
\settowidth{\Width}{$2\pi$}\setlength{\Width}{-0.5\Width}%
\settoheight{\Height}{$2\pi$}\settodepth{\Depth}{$2\pi$}\setlength{\Height}{-\Height}%
\put(6.2800000,-0.1000000){\hspace*{\Width}\raisebox{\Height}{$2\pi$}}%
%
}%
{%
\color[rgb]{0,0,0}%
\special{pa  -618   -20}\special{pa  -618    20}%
\special{fp}%
}%
{%
\color[rgb]{0,0,0}%
\settowidth{\Width}{$-\tfrac{\pi}{2}$}\setlength{\Width}{-0.5\Width}%
\settoheight{\Height}{$-\tfrac{\pi}{2}$}\settodepth{\Depth}{$-\tfrac{\pi}{2}$}\setlength{\Height}{-\Height}%
\put(-1.5700000,-0.1000000){\hspace*{\Width}\raisebox{\Height}{$-\tfrac{\pi}{2}$}}%
%
}%
{%
\color[rgb]{0,0,0}%
\special{pa -1237   -20}\special{pa -1237    20}%
\special{fp}%
}%
{%
\color[rgb]{0,0,0}%
\settowidth{\Width}{$-\pi$}\setlength{\Width}{-0.5\Width}%
\settoheight{\Height}{$-\pi$}\settodepth{\Depth}{$-\pi$}\setlength{\Height}{-\Height}%
\put(-3.1400000,-0.1000000){\hspace*{\Width}\raisebox{\Height}{$-\pi$}}%
%
}%
{%
\color[rgb]{0,0,0}%
\special{pa -2474   -20}\special{pa -2474    20}%
\special{fp}%
}%
{%
\color[rgb]{0,0,0}%
\settowidth{\Width}{$-2\pi$}\setlength{\Width}{-0.5\Width}%
\settoheight{\Height}{$-2\pi$}\settodepth{\Depth}{$-2\pi$}\setlength{\Height}{-\Height}%
\put(-6.2800000,-0.1000000){\hspace*{\Width}\raisebox{\Height}{$-2\pi$}}%
%
}%
{%
\color[rgb]{0,0,0}%
\special{pa    20   394}\special{pa   -20   394}%
\special{fp}%
}%
{%
\color[rgb]{0,0,0}%
\settowidth{\Width}{$-1$}\setlength{\Width}{-1\Width}%
\settoheight{\Height}{$-1$}\settodepth{\Depth}{$-1$}\setlength{\Height}{-0.5\Height}\setlength{\Depth}{0.5\Depth}\addtolength{\Height}{\Depth}%
\put(-0.1000000,-1.0000000){\hspace*{\Width}\raisebox{\Height}{$-1$}}%
%
}%
{%
\color[rgb]{0,0,0}%
\special{pa    20  -394}\special{pa   -20  -394}%
\special{fp}%
}%
{%
\color[rgb]{0,0,0}%
\settowidth{\Width}{$1$}\setlength{\Width}{-1\Width}%
\settoheight{\Height}{$1$}\settodepth{\Depth}{$1$}\setlength{\Height}{-0.5\Height}\setlength{\Depth}{0.5\Depth}\addtolength{\Height}{\Depth}%
\put(-0.1000000,1.0000000){\hspace*{\Width}\raisebox{\Height}{$1$}}%
%
}%
\special{pa -2559    -0}\special{pa  2559    -0}%
\special{fp}%
\special{pa     0   472}\special{pa     0  -472}%
\special{fp}%
\settowidth{\Width}{$x$}\setlength{\Width}{0\Width}%
\settoheight{\Height}{$x$}\settodepth{\Depth}{$x$}\setlength{\Height}{-0.5\Height}\setlength{\Depth}{0.5\Depth}\addtolength{\Height}{\Depth}%
\put(6.5500000,0.0000000){\hspace*{\Width}\raisebox{\Height}{$x$}}%
%
\settowidth{\Width}{$y$}\setlength{\Width}{-0.5\Width}%
\settoheight{\Height}{$y$}\settodepth{\Depth}{$y$}\setlength{\Height}{\Depth}%
\put(0.0000000,1.2500000){\hspace*{\Width}\raisebox{\Height}{$y$}}%
%
\settowidth{\Width}{O}\setlength{\Width}{0\Width}%
\settoheight{\Height}{O}\settodepth{\Depth}{O}\setlength{\Height}{-\Height}%
\put(0.0500000,-0.0500000){\hspace*{\Width}\raisebox{\Height}{O}}%
%
\end{picture}}%}
\putnoten{118}{6}{\small $y=\cos x$}
\putnotes{62}{6}{%%% /polytech.git/n103/fig/graphsinadd.tex 
%%% Generator=graphsincos.cdy 
{\unitlength=1cm%
\begin{picture}%
(13,2.4)(-6.5,-1.2)%
\special{pn 8}%
%
\small%
\color[cmyk]{0,1,1,0}%
{%
\color[rgb]{1,0,0}%
\special{pa -2559    85}\special{pa -2533    60}\special{pa -2508    34}\special{pa -2482     9}%
\special{pa -2457   -17}\special{pa -2431   -43}\special{pa -2406   -68}\special{pa -2380   -93}%
\special{pa -2354  -118}\special{pa -2329  -142}\special{pa -2303  -165}\special{pa -2278  -188}%
\special{pa -2252  -210}\special{pa -2226  -231}\special{pa -2201  -252}\special{pa -2175  -271}%
\special{pa -2150  -289}\special{pa -2124  -305}\special{pa -2098  -321}\special{pa -2073  -335}%
\special{pa -2047  -348}\special{pa -2022  -359}\special{pa -1996  -369}\special{pa -1970  -377}%
\special{pa -1945  -384}\special{pa -1919  -389}\special{pa -1894  -392}\special{pa -1868  -393}%
\special{pa -1843  -393}\special{pa -1817  -392}\special{pa -1791  -389}\special{pa -1766  -384}%
\special{pa -1740  -377}\special{pa -1715  -369}\special{pa -1689  -359}\special{pa -1663  -348}%
\special{pa -1638  -335}\special{pa -1612  -321}\special{pa -1587  -306}\special{pa -1561  -289}%
\special{pa -1535  -271}\special{pa -1510  -252}\special{pa -1484  -231}\special{pa -1459  -210}%
\special{pa -1433  -188}\special{pa -1407  -165}\special{pa -1382  -142}\special{pa -1356  -118}%
\special{pa -1331   -93}\special{pa -1305   -68}\special{pa -1280   -43}\special{pa -1254   -17}%
\special{pa -1228     9}\special{pa -1203    34}\special{pa -1177    59}\special{pa -1152    85}%
\special{pa -1126   109}\special{pa -1100   134}\special{pa -1075   158}\special{pa -1049   181}%
\special{pa -1024   203}\special{pa  -998   224}\special{pa  -972   245}\special{pa  -947   264}%
\special{pa  -921   283}\special{pa  -896   300}\special{pa  -870   316}\special{pa  -844   331}%
\special{pa  -819   344}\special{pa  -793   355}\special{pa  -768   366}\special{pa  -742   374}%
\special{pa  -717   382}\special{pa  -691   387}\special{pa  -665   391}\special{pa  -640   393}%
\special{pa  -614   394}\special{pa  -589   393}\special{pa  -563   390}\special{pa  -537   385}%
\special{pa  -512   379}\special{pa  -486   372}\special{pa  -461   363}\special{pa  -435   352}%
\special{pa  -409   340}\special{pa  -384   326}\special{pa  -358   311}\special{pa  -333   294}%
\special{pa  -307   277}\special{pa  -281   258}\special{pa  -256   238}\special{pa  -230   217}%
\special{pa  -205   196}\special{pa  -179   173}\special{pa  -154   150}\special{pa  -128   126}%
\special{pa  -102   101}\special{pa   -77    76}\special{pa   -51    51}\special{pa   -26    26}%
\special{pa     0    -0}\special{pa    26   -26}\special{pa    51   -51}\special{pa    77   -76}%
\special{pa   102  -101}\special{pa   128  -126}\special{pa   154  -150}\special{pa   179  -173}%
\special{pa   205  -196}\special{pa   230  -217}\special{pa   256  -238}\special{pa   281  -258}%
\special{pa   307  -277}\special{pa   333  -294}\special{pa   358  -311}\special{pa   384  -326}%
\special{pa   409  -340}\special{pa   435  -352}\special{pa   461  -363}\special{pa   486  -372}%
\special{pa   512  -379}\special{pa   537  -385}\special{pa   563  -390}\special{pa   589  -393}%
\special{pa   614  -394}\special{pa   640  -393}\special{pa   665  -391}\special{pa   691  -387}%
\special{pa   717  -382}\special{pa   742  -374}\special{pa   768  -366}\special{pa   793  -355}%
\special{pa   819  -344}\special{pa   844  -331}\special{pa   870  -316}\special{pa   896  -300}%
\special{pa   921  -283}\special{pa   947  -264}\special{pa   972  -245}\special{pa   998  -224}%
\special{pa  1024  -203}\special{pa  1049  -181}\special{pa  1075  -158}\special{pa  1100  -134}%
\special{pa  1126  -109}\special{pa  1152   -85}\special{pa  1177   -59}\special{pa  1203   -34}%
\special{pa  1228    -9}\special{pa  1254    17}\special{pa  1280    43}\special{pa  1305    68}%
\special{pa  1331    93}\special{pa  1356   118}\special{pa  1382   142}\special{pa  1407   165}%
\special{pa  1433   188}\special{pa  1459   210}\special{pa  1484   231}\special{pa  1510   252}%
\special{pa  1535   271}\special{pa  1561   289}\special{pa  1587   306}\special{pa  1612   321}%
\special{pa  1638   335}\special{pa  1663   348}\special{pa  1689   359}\special{pa  1715   369}%
\special{pa  1740   377}\special{pa  1766   384}\special{pa  1791   389}\special{pa  1817   392}%
\special{pa  1843   393}\special{pa  1868   393}\special{pa  1894   392}\special{pa  1919   389}%
\special{pa  1945   384}\special{pa  1970   377}\special{pa  1996   369}\special{pa  2022   359}%
\special{pa  2047   348}\special{pa  2073   335}\special{pa  2098   321}\special{pa  2124   305}%
\special{pa  2150   289}\special{pa  2175   271}\special{pa  2201   252}\special{pa  2226   231}%
\special{pa  2252   210}\special{pa  2278   188}\special{pa  2303   165}\special{pa  2329   142}%
\special{pa  2354   118}\special{pa  2380    93}\special{pa  2406    68}\special{pa  2431    43}%
\special{pa  2457    17}\special{pa  2482    -9}\special{pa  2508   -34}\special{pa  2533   -60}%
\special{pa  2559   -85}%
\special{fp}%
}%
\end{picture}}%}
\putnoten{80}{6}{\color{red}\small $y=\sin x$}
\end{layer}

\vspace*{28mm}
\begin{itemize}
\item
{\color{red}振幅}は$1$(値の範囲は$-1$から$1$)\vspace{-1mm}
\item
{\color{red}周期}は$2\pi$($2\pi$で元に戻る)\vspace{-1mm}
\item
$\cos x$は$y$軸対称\vspace{-1mm}
\item
$\cos x$は$\sin x$を左に$\frac{\pi}{2}$平行移動({\color{red}位相}が$\frac{\pi}{2}$進む)
\end{itemize}

\newslide{角度の和の三角関数}

\vspace*{18mm}

\slidepage
\begin{itemize}
\item
2つの角を$A,\ B$とする(通常はギリシャ文字 $\alpha,\  \beta$)
\item
$\sin(A+B)=\sin A+\sin B$が成り立つかを考えよう
\item
$\sin\deg{30}+\sin\deg{60}=\sin(\deg{30}+\deg{60})$になるかを調べる
\item
{\large $\sin \deg{90}=1,\ \sin \deg{30}=\bunsuu{1}{2},\ \sin \deg{60}=\bunsuu{\sqrt{3}}{2}$}
\item
[課題]\monban $\sqrt{3}=1.732$を用いて答えよ.\seteda{90}\\
\eda{$\sin\deg{30}+\sin\deg{60}$を計算せよ}\\
\eda{$\sin(A+B)=\sin A+\sin B$は成り立つと言えるか}
\end{itemize}
%%%%%%%%%%%%

%%%%%%%%%%%%%%%%%%%%


\newslide{加法定理}

\vspace*{18mm}

\slidepage
\begin{itemize}
\item
[]$\sin(A+B)=\sin A \cos B+\cos A\sin B$
\item
[]$\sin( A- B)=\sin A\cos B-\cos A\sin B$
\item
[]$\cos( A+ B)=\cos A\cos B-\sin A\sin B$
\item
[]$\cos( A- B)=\cos A\cos B+\sin A\sin B$
\end{itemize}

\newslide{具体例(テキストP181)}

\vspace*{18mm}


\begin{layer}{120}{0}
\putnotew{96}{73}{\hyperlink{para1pg2}{\fbox{\Ctab{2.5mm}{\scalebox{1}{\scriptsize $\mathstrut||\!\lhd$}}}}}
\putnotew{101}{73}{\hyperlink{para2pg1}{\fbox{\Ctab{2.5mm}{\scalebox{1}{\scriptsize $\mathstrut|\!\lhd$}}}}}
\putnotew{108}{73}{\hyperlink{para2pg10}{\fbox{\Ctab{4.5mm}{\scalebox{1}{\scriptsize $\mathstrut\!\!\lhd\!\!$}}}}}
\putnotew{115}{73}{\hyperlink{para2pg11}{\fbox{\Ctab{4.5mm}{\scalebox{1}{\scriptsize $\mathstrut\!\rhd\!$}}}}}
\putnotew{120}{73}{\hyperlink{para2pg11}{\fbox{\Ctab{2.5mm}{\scalebox{1}{\scriptsize $\mathstrut \!\rhd\!\!|$}}}}}
\putnotew{125}{73}{\hyperlink{para3pg1}{\fbox{\Ctab{2.5mm}{\scalebox{1}{\scriptsize $\mathstrut \!\rhd\!\!||$}}}}}
\putnotee{126}{73}{\scriptsize\color{blue} 11/11}
\end{layer}

\slidepage
\begin{itemize}
\item
{\color{blue}\normalsize $\sin 30\degree=\hakoa{$\bunsuu{1}{2}$},\ \sin 45\degree=\hakoa{$\bunsuu{1}{\sqrt{2}}$},\ \sin 60\degree=\hakoa{$\bunsuu{\sqrt{3}}{2}$}$}\\
{\color{blue}\normalsize $\cos 30\degree=\hakoa{$\bunsuu{\sqrt{3}}{2}$},\ \cos 45\degree=\hakoa{$\bunsuu{1}{\sqrt{2}}$},\ \cos 60\degree=\hakoa{$\bunsuu{1}{2}$}$}
\item
$\sin 75\degree$\\
$=\sin(45\degree+30\degree)$
$=\sin 45\degree \cos 30\degree+\cos 45\degree \sin 30\degree$
$=\bunsuu{1}{\sqrt{2}}\bunsuu{\sqrt{3}}{2}+\bunsuu{1}{\sqrt{2}}\bunsuu{1}{2}=$
$\bunsuu{\sqrt{3}+1}{2\sqrt{2}}=$
$\bunsuu{\sqrt{6}+\sqrt{2}}{4}$
\item
[課題]\monban 次を求めよ\seteda{50}\\
\eda{$\sin 15\degree$}\eda{$\cos 75\degree$}
\end{itemize}

\newslide{グラフの対称性}

\vspace*{18mm}

\slidepage

\begin{layer}{120}{0}
\putnotes{62}{1}{\scalebox{0.9}{%%% /polytech22.git/104-0509/presen/fig/graphsin.tex 
%%% Generator=graphsincos.cdy 
{\unitlength=1cm%
\begin{picture}%
(13,2.4)(-6.5,-1.2)%
\linethickness{0.008in}%%
\polyline(-6.50000,1.00000)(-6.40076,1.00000)\polyline(-6.30153,1.00000)(-6.20229,1.00000)%
\polyline(-6.10305,1.00000)(-6.00382,1.00000)\polyline(-5.90458,1.00000)(-5.80534,1.00000)%
\polyline(-5.70611,1.00000)(-5.60687,1.00000)\polyline(-5.50763,1.00000)(-5.40840,1.00000)%
\polyline(-5.30916,1.00000)(-5.20992,1.00000)\polyline(-5.11069,1.00000)(-5.01145,1.00000)%
\polyline(-4.91221,1.00000)(-4.81298,1.00000)\polyline(-4.71374,1.00000)(-4.61450,1.00000)%
\polyline(-4.51527,1.00000)(-4.41603,1.00000)\polyline(-4.31679,1.00000)(-4.21756,1.00000)%
\polyline(-4.11832,1.00000)(-4.01908,1.00000)\polyline(-3.91985,1.00000)(-3.82061,1.00000)%
\polyline(-3.72137,1.00000)(-3.62214,1.00000)\polyline(-3.52290,1.00000)(-3.42366,1.00000)%
\polyline(-3.32443,1.00000)(-3.22519,1.00000)\polyline(-3.12595,1.00000)(-3.02672,1.00000)%
\polyline(-2.92748,1.00000)(-2.82824,1.00000)\polyline(-2.72901,1.00000)(-2.62977,1.00000)%
\polyline(-2.53053,1.00000)(-2.43130,1.00000)\polyline(-2.33206,1.00000)(-2.23282,1.00000)%
\polyline(-2.13359,1.00000)(-2.03435,1.00000)\polyline(-1.93511,1.00000)(-1.83588,1.00000)%
\polyline(-1.73664,1.00000)(-1.63740,1.00000)\polyline(-1.53817,1.00000)(-1.43893,1.00000)%
\polyline(-1.33969,1.00000)(-1.24046,1.00000)\polyline(-1.14122,1.00000)(-1.04198,1.00000)%
\polyline(-0.94275,1.00000)(-0.84351,1.00000)\polyline(-0.74427,1.00000)(-0.64504,1.00000)%
\polyline(-0.54580,1.00000)(-0.44656,1.00000)\polyline(-0.34733,1.00000)(-0.24809,1.00000)%
\polyline(-0.14885,1.00000)(-0.04962,1.00000)\polyline(0.04962,1.00000)(0.14885,1.00000)%
\polyline(0.24809,1.00000)(0.34733,1.00000)\polyline(0.44656,1.00000)(0.54580,1.00000)%
\polyline(0.64504,1.00000)(0.74427,1.00000)\polyline(0.84351,1.00000)(0.94275,1.00000)%
\polyline(1.04198,1.00000)(1.14122,1.00000)\polyline(1.24046,1.00000)(1.33969,1.00000)%
\polyline(1.43893,1.00000)(1.53817,1.00000)\polyline(1.63740,1.00000)(1.73664,1.00000)%
\polyline(1.83588,1.00000)(1.93511,1.00000)\polyline(2.03435,1.00000)(2.13359,1.00000)%
\polyline(2.23282,1.00000)(2.33206,1.00000)\polyline(2.43130,1.00000)(2.53053,1.00000)%
\polyline(2.62977,1.00000)(2.72901,1.00000)\polyline(2.82824,1.00000)(2.92748,1.00000)%
\polyline(3.02672,1.00000)(3.12595,1.00000)\polyline(3.22519,1.00000)(3.32443,1.00000)%
\polyline(3.42366,1.00000)(3.52290,1.00000)\polyline(3.62214,1.00000)(3.72137,1.00000)%
\polyline(3.82061,1.00000)(3.91985,1.00000)\polyline(4.01908,1.00000)(4.11832,1.00000)%
\polyline(4.21756,1.00000)(4.31679,1.00000)\polyline(4.41603,1.00000)(4.51527,1.00000)%
\polyline(4.61450,1.00000)(4.71374,1.00000)\polyline(4.81298,1.00000)(4.91221,1.00000)%
\polyline(5.01145,1.00000)(5.11069,1.00000)\polyline(5.20992,1.00000)(5.30916,1.00000)%
\polyline(5.40840,1.00000)(5.50763,1.00000)\polyline(5.60687,1.00000)(5.70611,1.00000)%
\polyline(5.80534,1.00000)(5.90458,1.00000)\polyline(6.00382,1.00000)(6.10305,1.00000)%
\polyline(6.20229,1.00000)(6.30153,1.00000)\polyline(6.40076,1.00000)(6.50000,1.00000)%
%
%
\polyline(-6.50000,-1.00000)(-6.40076,-1.00000)\polyline(-6.30153,-1.00000)(-6.20229,-1.00000)%
\polyline(-6.10305,-1.00000)(-6.00382,-1.00000)\polyline(-5.90458,-1.00000)(-5.80534,-1.00000)%
\polyline(-5.70611,-1.00000)(-5.60687,-1.00000)\polyline(-5.50763,-1.00000)(-5.40840,-1.00000)%
\polyline(-5.30916,-1.00000)(-5.20992,-1.00000)\polyline(-5.11069,-1.00000)(-5.01145,-1.00000)%
\polyline(-4.91221,-1.00000)(-4.81298,-1.00000)\polyline(-4.71374,-1.00000)(-4.61450,-1.00000)%
\polyline(-4.51527,-1.00000)(-4.41603,-1.00000)\polyline(-4.31679,-1.00000)(-4.21756,-1.00000)%
\polyline(-4.11832,-1.00000)(-4.01908,-1.00000)\polyline(-3.91985,-1.00000)(-3.82061,-1.00000)%
\polyline(-3.72137,-1.00000)(-3.62214,-1.00000)\polyline(-3.52290,-1.00000)(-3.42366,-1.00000)%
\polyline(-3.32443,-1.00000)(-3.22519,-1.00000)\polyline(-3.12595,-1.00000)(-3.02672,-1.00000)%
\polyline(-2.92748,-1.00000)(-2.82824,-1.00000)\polyline(-2.72901,-1.00000)(-2.62977,-1.00000)%
\polyline(-2.53053,-1.00000)(-2.43130,-1.00000)\polyline(-2.33206,-1.00000)(-2.23282,-1.00000)%
\polyline(-2.13359,-1.00000)(-2.03435,-1.00000)\polyline(-1.93511,-1.00000)(-1.83588,-1.00000)%
\polyline(-1.73664,-1.00000)(-1.63740,-1.00000)\polyline(-1.53817,-1.00000)(-1.43893,-1.00000)%
\polyline(-1.33969,-1.00000)(-1.24046,-1.00000)\polyline(-1.14122,-1.00000)(-1.04198,-1.00000)%
\polyline(-0.94275,-1.00000)(-0.84351,-1.00000)\polyline(-0.74427,-1.00000)(-0.64504,-1.00000)%
\polyline(-0.54580,-1.00000)(-0.44656,-1.00000)\polyline(-0.34733,-1.00000)(-0.24809,-1.00000)%
\polyline(-0.14885,-1.00000)(-0.04962,-1.00000)\polyline(0.04962,-1.00000)(0.14885,-1.00000)%
\polyline(0.24809,-1.00000)(0.34733,-1.00000)\polyline(0.44656,-1.00000)(0.54580,-1.00000)%
\polyline(0.64504,-1.00000)(0.74427,-1.00000)\polyline(0.84351,-1.00000)(0.94275,-1.00000)%
\polyline(1.04198,-1.00000)(1.14122,-1.00000)\polyline(1.24046,-1.00000)(1.33969,-1.00000)%
\polyline(1.43893,-1.00000)(1.53817,-1.00000)\polyline(1.63740,-1.00000)(1.73664,-1.00000)%
\polyline(1.83588,-1.00000)(1.93511,-1.00000)\polyline(2.03435,-1.00000)(2.13359,-1.00000)%
\polyline(2.23282,-1.00000)(2.33206,-1.00000)\polyline(2.43130,-1.00000)(2.53053,-1.00000)%
\polyline(2.62977,-1.00000)(2.72901,-1.00000)\polyline(2.82824,-1.00000)(2.92748,-1.00000)%
\polyline(3.02672,-1.00000)(3.12595,-1.00000)\polyline(3.22519,-1.00000)(3.32443,-1.00000)%
\polyline(3.42366,-1.00000)(3.52290,-1.00000)\polyline(3.62214,-1.00000)(3.72137,-1.00000)%
\polyline(3.82061,-1.00000)(3.91985,-1.00000)\polyline(4.01908,-1.00000)(4.11832,-1.00000)%
\polyline(4.21756,-1.00000)(4.31679,-1.00000)\polyline(4.41603,-1.00000)(4.51527,-1.00000)%
\polyline(4.61450,-1.00000)(4.71374,-1.00000)\polyline(4.81298,-1.00000)(4.91221,-1.00000)%
\polyline(5.01145,-1.00000)(5.11069,-1.00000)\polyline(5.20992,-1.00000)(5.30916,-1.00000)%
\polyline(5.40840,-1.00000)(5.50763,-1.00000)\polyline(5.60687,-1.00000)(5.70611,-1.00000)%
\polyline(5.80534,-1.00000)(5.90458,-1.00000)\polyline(6.00382,-1.00000)(6.10305,-1.00000)%
\polyline(6.20229,-1.00000)(6.30153,-1.00000)\polyline(6.40076,-1.00000)(6.50000,-1.00000)%
%
%
\polyline(1.57080,0.05000)(1.57080,-0.05000)%
%
\settowidth{\Width}{$\tfrac{\pi}{2}$}\setlength{\Width}{-0.5\Width}%
\settoheight{\Height}{$\tfrac{\pi}{2}$}\settodepth{\Depth}{$\tfrac{\pi}{2}$}\setlength{\Height}{-\Height}%
\put(1.5700000,-0.1000000){\hspace*{\Width}\raisebox{\Height}{$\tfrac{\pi}{2}$}}%
%
\polyline(3.14159,0.05000)(3.14159,-0.05000)%
%
\settowidth{\Width}{$\pi$}\setlength{\Width}{-0.5\Width}%
\settoheight{\Height}{$\pi$}\settodepth{\Depth}{$\pi$}\setlength{\Height}{-\Height}%
\put(3.1400000,-0.1000000){\hspace*{\Width}\raisebox{\Height}{$\pi$}}%
%
\polyline(6.28319,0.05000)(6.28319,-0.05000)%
%
\settowidth{\Width}{$2\pi$}\setlength{\Width}{-0.5\Width}%
\settoheight{\Height}{$2\pi$}\settodepth{\Depth}{$2\pi$}\setlength{\Height}{-\Height}%
\put(6.2800000,-0.1000000){\hspace*{\Width}\raisebox{\Height}{$2\pi$}}%
%
\polyline(-1.57080,0.05000)(-1.57080,-0.05000)%
%
\settowidth{\Width}{$-\tfrac{\pi}{2}$}\setlength{\Width}{-0.5\Width}%
\settoheight{\Height}{$-\tfrac{\pi}{2}$}\settodepth{\Depth}{$-\tfrac{\pi}{2}$}\setlength{\Height}{-\Height}%
\put(-1.5700000,-0.1000000){\hspace*{\Width}\raisebox{\Height}{$-\tfrac{\pi}{2}$}}%
%
\polyline(-3.14159,0.05000)(-3.14159,-0.05000)%
%
\settowidth{\Width}{$-\pi$}\setlength{\Width}{-0.5\Width}%
\settoheight{\Height}{$-\pi$}\settodepth{\Depth}{$-\pi$}\setlength{\Height}{-\Height}%
\put(-3.1400000,-0.1000000){\hspace*{\Width}\raisebox{\Height}{$-\pi$}}%
%
\polyline(-6.28319,0.05000)(-6.28319,-0.05000)%
%
\settowidth{\Width}{$-2\pi$}\setlength{\Width}{-0.5\Width}%
\settoheight{\Height}{$-2\pi$}\settodepth{\Depth}{$-2\pi$}\setlength{\Height}{-\Height}%
\put(-6.2800000,-0.1000000){\hspace*{\Width}\raisebox{\Height}{$-2\pi$}}%
%
\polyline(0.05000,-1.00000)(-0.05000,-1.00000)%
%
\settowidth{\Width}{$-1$}\setlength{\Width}{-1\Width}%
\settoheight{\Height}{$-1$}\settodepth{\Depth}{$-1$}\setlength{\Height}{-0.5\Height}\setlength{\Depth}{0.5\Depth}\addtolength{\Height}{\Depth}%
\put(-0.1000000,-1.0000000){\hspace*{\Width}\raisebox{\Height}{$-1$}}%
%
\polyline(0.05000,1.00000)(-0.05000,1.00000)%
%
\settowidth{\Width}{$1$}\setlength{\Width}{-1\Width}%
\settoheight{\Height}{$1$}\settodepth{\Depth}{$1$}\setlength{\Height}{-0.5\Height}\setlength{\Depth}{0.5\Depth}\addtolength{\Height}{\Depth}%
\put(-0.1000000,1.0000000){\hspace*{\Width}\raisebox{\Height}{$1$}}%
%
\polyline(-6.50000,-0.21512)(-6.43500,-0.15123)(-6.37000,-0.08671)(-6.30500,-0.02181)%
(-6.24000,0.04317)(-6.17500,0.10797)(-6.11000,0.17232)(-6.04500,0.23594)(-5.98000,0.29856)%
(-5.91500,0.35992)(-5.85000,0.41976)(-5.78500,0.47783)(-5.72000,0.53388)(-5.65500,0.58768)%
(-5.59000,0.63899)(-5.52500,0.68760)(-5.46000,0.73332)(-5.39500,0.77593)(-5.33000,0.81526)%
(-5.26500,0.85116)(-5.20000,0.88345)(-5.13500,0.91202)(-5.07000,0.93674)(-5.00500,0.95749)%
(-4.94000,0.97421)(-4.87500,0.98681)(-4.81000,0.99524)(-4.74500,0.99947)(-4.68000,0.99948)%
(-4.61500,0.99526)(-4.55000,0.98684)(-4.48500,0.97426)(-4.42000,0.95756)(-4.35500,0.93681)%
(-4.29000,0.91211)(-4.22500,0.88356)(-4.16000,0.85127)(-4.09500,0.81539)(-4.03000,0.77607)%
(-3.96500,0.73347)(-3.90000,0.68777)(-3.83500,0.63916)(-3.77000,0.58786)(-3.70500,0.53407)%
(-3.64000,0.47803)(-3.57500,0.41997)(-3.51000,0.36013)(-3.44500,0.29877)(-3.38000,0.23616)%
(-3.31500,0.17254)(-3.25000,0.10820)(-3.18500,0.04339)(-3.12000,-0.02159)(-3.05500,-0.08648)%
(-2.99000,-0.15101)(-2.92500,-0.21490)(-2.86000,-0.27789)(-2.79500,-0.33970)(-2.73000,-0.40007)%
(-2.66500,-0.45875)(-2.60000,-0.51550)(-2.53500,-0.57007)(-2.47000,-0.62223)(-2.40500,-0.67177)%
(-2.34000,-0.71846)(-2.27500,-0.76213)(-2.21000,-0.80257)(-2.14500,-0.83963)(-2.08000,-0.87313)%
(-2.01500,-0.90295)(-1.95000,-0.92896)(-1.88500,-0.95104)(-1.82000,-0.96911)(-1.75500,-0.98308)%
(-1.69000,-0.99290)(-1.62500,-0.99853)(-1.56000,-0.99994)(-1.49500,-0.99713)(-1.43000,-0.99010)%
(-1.36500,-0.97890)(-1.30000,-0.96356)(-1.23500,-0.94415)(-1.17000,-0.92075)(-1.10500,-0.89346)%
(-1.04000,-0.86240)(-0.97500,-0.82770)(-0.91000,-0.78950)(-0.84500,-0.74797)(-0.78000,-0.70328)%
(-0.71500,-0.65562)(-0.65000,-0.60519)(-0.58500,-0.55220)(-0.52000,-0.49688)(-0.45500,-0.43946)%
(-0.39000,-0.38019)(-0.32500,-0.31931)(-0.26000,-0.25708)(-0.19500,-0.19377)(-0.13000,-0.12963)%
(-0.06500,-0.06495)(0.00000,0.00000)(0.06500,0.06495)(0.13000,0.12963)(0.19500,0.19377)%
(0.26000,0.25708)(0.32500,0.31931)(0.39000,0.38019)(0.45500,0.43946)(0.52000,0.49688)%
(0.58500,0.55220)(0.65000,0.60519)(0.71500,0.65562)(0.78000,0.70328)(0.84500,0.74797)%
(0.91000,0.78950)(0.97500,0.82770)(1.04000,0.86240)(1.10500,0.89346)(1.17000,0.92075)%
(1.23500,0.94415)(1.30000,0.96356)(1.36500,0.97890)(1.43000,0.99010)(1.49500,0.99713)%
(1.56000,0.99994)(1.62500,0.99853)(1.69000,0.99290)(1.75500,0.98308)(1.82000,0.96911)%
(1.88500,0.95104)(1.95000,0.92896)(2.01500,0.90295)(2.08000,0.87313)(2.14500,0.83963)%
(2.21000,0.80257)(2.27500,0.76213)(2.34000,0.71846)(2.40500,0.67177)(2.47000,0.62223)%
(2.53500,0.57007)(2.60000,0.51550)(2.66500,0.45875)(2.73000,0.40007)(2.79500,0.33970)%
(2.86000,0.27789)(2.92500,0.21490)(2.99000,0.15101)(3.05500,0.08648)(3.12000,0.02159)%
(3.18500,-0.04339)(3.25000,-0.10820)(3.31500,-0.17254)(3.38000,-0.23616)(3.44500,-0.29877)%
(3.51000,-0.36013)(3.57500,-0.41997)(3.64000,-0.47803)(3.70500,-0.53407)(3.77000,-0.58786)%
(3.83500,-0.63916)(3.90000,-0.68777)(3.96500,-0.73347)(4.03000,-0.77607)(4.09500,-0.81539)%
(4.16000,-0.85127)(4.22500,-0.88356)(4.29000,-0.91211)(4.35500,-0.93681)(4.42000,-0.95756)%
(4.48500,-0.97426)(4.55000,-0.98684)(4.61500,-0.99526)(4.68000,-0.99948)(4.74500,-0.99947)%
(4.81000,-0.99524)(4.87500,-0.98681)(4.94000,-0.97421)(5.00500,-0.95749)(5.07000,-0.93674)%
(5.13500,-0.91202)(5.20000,-0.88345)(5.26500,-0.85116)(5.33000,-0.81526)(5.39500,-0.77593)%
(5.46000,-0.73332)(5.52500,-0.68760)(5.59000,-0.63899)(5.65500,-0.58768)(5.72000,-0.53388)%
(5.78500,-0.47783)(5.85000,-0.41976)(5.91500,-0.35992)(5.98000,-0.29856)(6.04500,-0.23594)%
(6.11000,-0.17232)(6.17500,-0.10797)(6.24000,-0.04317)(6.30500,0.02181)(6.37000,0.08671)%
(6.43500,0.15123)(6.50000,0.21512)%
%
\polyline(-6.50000,0.00000)(6.50000,0.00000)%
%
\polyline(0.00000,-1.20000)(0.00000,1.20000)%
%
\settowidth{\Width}{$x$}\setlength{\Width}{0\Width}%
\settoheight{\Height}{$x$}\settodepth{\Depth}{$x$}\setlength{\Height}{-0.5\Height}\setlength{\Depth}{0.5\Depth}\addtolength{\Height}{\Depth}%
\put(6.5500000,0.0000000){\hspace*{\Width}\raisebox{\Height}{$x$}}%
%
\settowidth{\Width}{$y$}\setlength{\Width}{-0.5\Width}%
\settoheight{\Height}{$y$}\settodepth{\Depth}{$y$}\setlength{\Height}{\Depth}%
\put(0.0000000,1.2500000){\hspace*{\Width}\raisebox{\Height}{$y$}}%
%
\settowidth{\Width}{O}\setlength{\Width}{0\Width}%
\settoheight{\Height}{O}\settodepth{\Depth}{O}\setlength{\Height}{-\Height}%
\put(0.0500000,-0.0500000){\hspace*{\Width}\raisebox{\Height}{O}}%
%
\end{picture}}%}}
\putnotes{62}{28}{\scalebox{0.9}{%%% /polytech.git/n103/fig/graphcos.tex 
%%% Generator=graphsincos.cdy 
{\unitlength=1cm%
\begin{picture}%
(13,2.4)(-6.5,-1.2)%
\special{pn 8}%
%
\small%
{%
\color[rgb]{0,0,0}%
\special{pa -2559  -384}\special{pa -2533  -389}\special{pa -2508  -392}\special{pa -2482  -394}%
\special{pa -2457  -393}\special{pa -2431  -391}\special{pa -2406  -388}\special{pa -2380  -383}%
\special{pa -2354  -376}\special{pa -2329  -367}\special{pa -2303  -357}\special{pa -2278  -346}%
\special{pa -2252  -333}\special{pa -2226  -319}\special{pa -2201  -303}\special{pa -2175  -286}%
\special{pa -2150  -268}\special{pa -2124  -248}\special{pa -2098  -228}\special{pa -2073  -207}%
\special{pa -2047  -184}\special{pa -2022  -161}\special{pa -1996  -138}\special{pa -1970  -114}%
\special{pa -1945   -89}\special{pa -1919   -64}\special{pa -1894   -38}\special{pa -1868   -13}%
\special{pa -1843    13}\special{pa -1817    38}\special{pa -1791    64}\special{pa -1766    89}%
\special{pa -1740   113}\special{pa -1715   138}\special{pa -1689   161}\special{pa -1663   184}%
\special{pa -1638   207}\special{pa -1612   228}\special{pa -1587   248}\special{pa -1561   268}%
\special{pa -1535   286}\special{pa -1510   303}\special{pa -1484   318}\special{pa -1459   333}%
\special{pa -1433   346}\special{pa -1407   357}\special{pa -1382   367}\special{pa -1356   376}%
\special{pa -1331   383}\special{pa -1305   388}\special{pa -1280   391}\special{pa -1254   393}%
\special{pa -1228   394}\special{pa -1203   392}\special{pa -1177   389}\special{pa -1152   385}%
\special{pa -1126   378}\special{pa -1100   370}\special{pa -1075   361}\special{pa -1049   350}%
\special{pa -1024   337}\special{pa  -998   323}\special{pa  -972   308}\special{pa  -947   292}%
\special{pa  -921   274}\special{pa  -896   255}\special{pa  -870   235}\special{pa  -844   214}%
\special{pa  -819   192}\special{pa  -793   169}\special{pa  -768   146}\special{pa  -742   122}%
\special{pa  -717    97}\special{pa  -691    72}\special{pa  -665    47}\special{pa  -640    21}%
\special{pa  -614    -4}\special{pa  -589   -30}\special{pa  -563   -55}\special{pa  -537   -80}%
\special{pa  -512  -105}\special{pa  -486  -130}\special{pa  -461  -154}\special{pa  -435  -177}%
\special{pa  -409  -199}\special{pa  -384  -221}\special{pa  -358  -242}\special{pa  -333  -261}%
\special{pa  -307  -280}\special{pa  -281  -297}\special{pa  -256  -313}\special{pa  -230  -328}%
\special{pa  -205  -342}\special{pa  -179  -354}\special{pa  -154  -364}\special{pa  -128  -373}%
\special{pa  -102  -380}\special{pa   -77  -386}\special{pa   -51  -390}\special{pa   -26  -393}%
\special{pa     0  -394}\special{pa    26  -393}\special{pa    51  -390}\special{pa    77  -386}%
\special{pa   102  -380}\special{pa   128  -373}\special{pa   154  -364}\special{pa   179  -354}%
\special{pa   205  -342}\special{pa   230  -328}\special{pa   256  -313}\special{pa   281  -297}%
\special{pa   307  -280}\special{pa   333  -261}\special{pa   358  -242}\special{pa   384  -221}%
\special{pa   409  -199}\special{pa   435  -177}\special{pa   461  -154}\special{pa   486  -130}%
\special{pa   512  -105}\special{pa   537   -80}\special{pa   563   -55}\special{pa   589   -30}%
\special{pa   614    -4}\special{pa   640    21}\special{pa   665    47}\special{pa   691    72}%
\special{pa   717    97}\special{pa   742   122}\special{pa   768   146}\special{pa   793   169}%
\special{pa   819   192}\special{pa   844   214}\special{pa   870   235}\special{pa   896   255}%
\special{pa   921   274}\special{pa   947   292}\special{pa   972   308}\special{pa   998   323}%
\special{pa  1024   337}\special{pa  1049   350}\special{pa  1075   361}\special{pa  1100   370}%
\special{pa  1126   378}\special{pa  1152   385}\special{pa  1177   389}\special{pa  1203   392}%
\special{pa  1228   394}\special{pa  1254   393}\special{pa  1280   391}\special{pa  1305   388}%
\special{pa  1331   383}\special{pa  1356   376}\special{pa  1382   367}\special{pa  1407   357}%
\special{pa  1433   346}\special{pa  1459   333}\special{pa  1484   318}\special{pa  1510   303}%
\special{pa  1535   286}\special{pa  1561   268}\special{pa  1587   248}\special{pa  1612   228}%
\special{pa  1638   207}\special{pa  1663   184}\special{pa  1689   161}\special{pa  1715   138}%
\special{pa  1740   113}\special{pa  1766    89}\special{pa  1791    64}\special{pa  1817    38}%
\special{pa  1843    13}\special{pa  1868   -13}\special{pa  1894   -38}\special{pa  1919   -64}%
\special{pa  1945   -89}\special{pa  1970  -114}\special{pa  1996  -138}\special{pa  2022  -161}%
\special{pa  2047  -184}\special{pa  2073  -207}\special{pa  2098  -228}\special{pa  2124  -248}%
\special{pa  2150  -268}\special{pa  2175  -286}\special{pa  2201  -303}\special{pa  2226  -319}%
\special{pa  2252  -333}\special{pa  2278  -346}\special{pa  2303  -357}\special{pa  2329  -367}%
\special{pa  2354  -376}\special{pa  2380  -383}\special{pa  2406  -388}\special{pa  2431  -391}%
\special{pa  2457  -393}\special{pa  2482  -394}\special{pa  2508  -392}\special{pa  2533  -389}%
\special{pa  2559  -384}%
\special{fp}%
}%
{%
\color[rgb]{0,0,0}%
\special{pa -2559 -394}\special{pa -2520 -394}\special{fp}\special{pa -2481 -394}\special{pa -2442 -394}\special{fp}%
\special{pa -2403 -394}\special{pa -2364 -394}\special{fp}\special{pa -2325 -394}\special{pa -2286 -394}\special{fp}%
\special{pa -2246 -394}\special{pa -2207 -394}\special{fp}\special{pa -2168 -394}\special{pa -2129 -394}\special{fp}%
\special{pa -2090 -394}\special{pa -2051 -394}\special{fp}\special{pa -2012 -394}\special{pa -1973 -394}\special{fp}%
\special{pa -1934 -394}\special{pa -1895 -394}\special{fp}\special{pa -1856 -394}\special{pa -1817 -394}\special{fp}%
\special{pa -1778 -394}\special{pa -1739 -394}\special{fp}\special{pa -1700 -394}\special{pa -1660 -394}\special{fp}%
\special{pa -1621 -394}\special{pa -1582 -394}\special{fp}\special{pa -1543 -394}\special{pa -1504 -394}\special{fp}%
\special{pa -1465 -394}\special{pa -1426 -394}\special{fp}\special{pa -1387 -394}\special{pa -1348 -394}\special{fp}%
\special{pa -1309 -394}\special{pa -1270 -394}\special{fp}\special{pa -1231 -394}\special{pa -1192 -394}\special{fp}%
\special{pa -1153 -394}\special{pa -1113 -394}\special{fp}\special{pa -1074 -394}\special{pa -1035 -394}\special{fp}%
\special{pa -996 -394}\special{pa -957 -394}\special{fp}\special{pa -918 -394}\special{pa -879 -394}\special{fp}%
\special{pa -840 -394}\special{pa -801 -394}\special{fp}\special{pa -762 -394}\special{pa -723 -394}\special{fp}%
\special{pa -684 -394}\special{pa -645 -394}\special{fp}\special{pa -606 -394}\special{pa -567 -394}\special{fp}%
\special{pa -527 -394}\special{pa -488 -394}\special{fp}\special{pa -449 -394}\special{pa -410 -394}\special{fp}%
\special{pa -371 -394}\special{pa -332 -394}\special{fp}\special{pa -293 -394}\special{pa -254 -394}\special{fp}%
\special{pa -215 -394}\special{pa -176 -394}\special{fp}\special{pa -137 -394}\special{pa -98 -394}\special{fp}%
\special{pa -59 -394}\special{pa -20 -394}\special{fp}\special{pa 20 -394}\special{pa 59 -394}\special{fp}%
\special{pa 98 -394}\special{pa 137 -394}\special{fp}\special{pa 176 -394}\special{pa 215 -394}\special{fp}%
\special{pa 254 -394}\special{pa 293 -394}\special{fp}\special{pa 332 -394}\special{pa 371 -394}\special{fp}%
\special{pa 410 -394}\special{pa 449 -394}\special{fp}\special{pa 488 -394}\special{pa 527 -394}\special{fp}%
\special{pa 567 -394}\special{pa 606 -394}\special{fp}\special{pa 645 -394}\special{pa 684 -394}\special{fp}%
\special{pa 723 -394}\special{pa 762 -394}\special{fp}\special{pa 801 -394}\special{pa 840 -394}\special{fp}%
\special{pa 879 -394}\special{pa 918 -394}\special{fp}\special{pa 957 -394}\special{pa 996 -394}\special{fp}%
\special{pa 1035 -394}\special{pa 1074 -394}\special{fp}\special{pa 1113 -394}\special{pa 1153 -394}\special{fp}%
\special{pa 1192 -394}\special{pa 1231 -394}\special{fp}\special{pa 1270 -394}\special{pa 1309 -394}\special{fp}%
\special{pa 1348 -394}\special{pa 1387 -394}\special{fp}\special{pa 1426 -394}\special{pa 1465 -394}\special{fp}%
\special{pa 1504 -394}\special{pa 1543 -394}\special{fp}\special{pa 1582 -394}\special{pa 1621 -394}\special{fp}%
\special{pa 1660 -394}\special{pa 1700 -394}\special{fp}\special{pa 1739 -394}\special{pa 1778 -394}\special{fp}%
\special{pa 1817 -394}\special{pa 1856 -394}\special{fp}\special{pa 1895 -394}\special{pa 1934 -394}\special{fp}%
\special{pa 1973 -394}\special{pa 2012 -394}\special{fp}\special{pa 2051 -394}\special{pa 2090 -394}\special{fp}%
\special{pa 2129 -394}\special{pa 2168 -394}\special{fp}\special{pa 2207 -394}\special{pa 2246 -394}\special{fp}%
\special{pa 2286 -394}\special{pa 2325 -394}\special{fp}\special{pa 2364 -394}\special{pa 2403 -394}\special{fp}%
\special{pa 2442 -394}\special{pa 2481 -394}\special{fp}\special{pa 2520 -394}\special{pa 2559 -394}\special{fp}%
%
%
}%
{%
\color[rgb]{0,0,0}%
\special{pa -2559 394}\special{pa -2520 394}\special{fp}\special{pa -2481 394}\special{pa -2442 394}\special{fp}%
\special{pa -2403 394}\special{pa -2364 394}\special{fp}\special{pa -2325 394}\special{pa -2286 394}\special{fp}%
\special{pa -2246 394}\special{pa -2207 394}\special{fp}\special{pa -2168 394}\special{pa -2129 394}\special{fp}%
\special{pa -2090 394}\special{pa -2051 394}\special{fp}\special{pa -2012 394}\special{pa -1973 394}\special{fp}%
\special{pa -1934 394}\special{pa -1895 394}\special{fp}\special{pa -1856 394}\special{pa -1817 394}\special{fp}%
\special{pa -1778 394}\special{pa -1739 394}\special{fp}\special{pa -1700 394}\special{pa -1660 394}\special{fp}%
\special{pa -1621 394}\special{pa -1582 394}\special{fp}\special{pa -1543 394}\special{pa -1504 394}\special{fp}%
\special{pa -1465 394}\special{pa -1426 394}\special{fp}\special{pa -1387 394}\special{pa -1348 394}\special{fp}%
\special{pa -1309 394}\special{pa -1270 394}\special{fp}\special{pa -1231 394}\special{pa -1192 394}\special{fp}%
\special{pa -1153 394}\special{pa -1113 394}\special{fp}\special{pa -1074 394}\special{pa -1035 394}\special{fp}%
\special{pa -996 394}\special{pa -957 394}\special{fp}\special{pa -918 394}\special{pa -879 394}\special{fp}%
\special{pa -840 394}\special{pa -801 394}\special{fp}\special{pa -762 394}\special{pa -723 394}\special{fp}%
\special{pa -684 394}\special{pa -645 394}\special{fp}\special{pa -606 394}\special{pa -567 394}\special{fp}%
\special{pa -527 394}\special{pa -488 394}\special{fp}\special{pa -449 394}\special{pa -410 394}\special{fp}%
\special{pa -371 394}\special{pa -332 394}\special{fp}\special{pa -293 394}\special{pa -254 394}\special{fp}%
\special{pa -215 394}\special{pa -176 394}\special{fp}\special{pa -137 394}\special{pa -98 394}\special{fp}%
\special{pa -59 394}\special{pa -20 394}\special{fp}\special{pa 20 394}\special{pa 59 394}\special{fp}%
\special{pa 98 394}\special{pa 137 394}\special{fp}\special{pa 176 394}\special{pa 215 394}\special{fp}%
\special{pa 254 394}\special{pa 293 394}\special{fp}\special{pa 332 394}\special{pa 371 394}\special{fp}%
\special{pa 410 394}\special{pa 449 394}\special{fp}\special{pa 488 394}\special{pa 527 394}\special{fp}%
\special{pa 567 394}\special{pa 606 394}\special{fp}\special{pa 645 394}\special{pa 684 394}\special{fp}%
\special{pa 723 394}\special{pa 762 394}\special{fp}\special{pa 801 394}\special{pa 840 394}\special{fp}%
\special{pa 879 394}\special{pa 918 394}\special{fp}\special{pa 957 394}\special{pa 996 394}\special{fp}%
\special{pa 1035 394}\special{pa 1074 394}\special{fp}\special{pa 1113 394}\special{pa 1153 394}\special{fp}%
\special{pa 1192 394}\special{pa 1231 394}\special{fp}\special{pa 1270 394}\special{pa 1309 394}\special{fp}%
\special{pa 1348 394}\special{pa 1387 394}\special{fp}\special{pa 1426 394}\special{pa 1465 394}\special{fp}%
\special{pa 1504 394}\special{pa 1543 394}\special{fp}\special{pa 1582 394}\special{pa 1621 394}\special{fp}%
\special{pa 1660 394}\special{pa 1700 394}\special{fp}\special{pa 1739 394}\special{pa 1778 394}\special{fp}%
\special{pa 1817 394}\special{pa 1856 394}\special{fp}\special{pa 1895 394}\special{pa 1934 394}\special{fp}%
\special{pa 1973 394}\special{pa 2012 394}\special{fp}\special{pa 2051 394}\special{pa 2090 394}\special{fp}%
\special{pa 2129 394}\special{pa 2168 394}\special{fp}\special{pa 2207 394}\special{pa 2246 394}\special{fp}%
\special{pa 2286 394}\special{pa 2325 394}\special{fp}\special{pa 2364 394}\special{pa 2403 394}\special{fp}%
\special{pa 2442 394}\special{pa 2481 394}\special{fp}\special{pa 2520 394}\special{pa 2559 394}\special{fp}%
%
%
}%
{%
\color[rgb]{0,0,0}%
\special{pa   618   -20}\special{pa   618    20}%
\special{fp}%
}%
{%
\color[rgb]{0,0,0}%
\settowidth{\Width}{$\tfrac{\pi}{2}$}\setlength{\Width}{-0.5\Width}%
\settoheight{\Height}{$\tfrac{\pi}{2}$}\settodepth{\Depth}{$\tfrac{\pi}{2}$}\setlength{\Height}{-\Height}%
\put(1.5700000,-0.1000000){\hspace*{\Width}\raisebox{\Height}{$\tfrac{\pi}{2}$}}%
%
}%
{%
\color[rgb]{0,0,0}%
\special{pa  1237   -20}\special{pa  1237    20}%
\special{fp}%
}%
{%
\color[rgb]{0,0,0}%
\settowidth{\Width}{$\pi$}\setlength{\Width}{-0.5\Width}%
\settoheight{\Height}{$\pi$}\settodepth{\Depth}{$\pi$}\setlength{\Height}{-\Height}%
\put(3.1400000,-0.1000000){\hspace*{\Width}\raisebox{\Height}{$\pi$}}%
%
}%
{%
\color[rgb]{0,0,0}%
\special{pa  2474   -20}\special{pa  2474    20}%
\special{fp}%
}%
{%
\color[rgb]{0,0,0}%
\settowidth{\Width}{$2\pi$}\setlength{\Width}{-0.5\Width}%
\settoheight{\Height}{$2\pi$}\settodepth{\Depth}{$2\pi$}\setlength{\Height}{-\Height}%
\put(6.2800000,-0.1000000){\hspace*{\Width}\raisebox{\Height}{$2\pi$}}%
%
}%
{%
\color[rgb]{0,0,0}%
\special{pa  -618   -20}\special{pa  -618    20}%
\special{fp}%
}%
{%
\color[rgb]{0,0,0}%
\settowidth{\Width}{$-\tfrac{\pi}{2}$}\setlength{\Width}{-0.5\Width}%
\settoheight{\Height}{$-\tfrac{\pi}{2}$}\settodepth{\Depth}{$-\tfrac{\pi}{2}$}\setlength{\Height}{-\Height}%
\put(-1.5700000,-0.1000000){\hspace*{\Width}\raisebox{\Height}{$-\tfrac{\pi}{2}$}}%
%
}%
{%
\color[rgb]{0,0,0}%
\special{pa -1237   -20}\special{pa -1237    20}%
\special{fp}%
}%
{%
\color[rgb]{0,0,0}%
\settowidth{\Width}{$-\pi$}\setlength{\Width}{-0.5\Width}%
\settoheight{\Height}{$-\pi$}\settodepth{\Depth}{$-\pi$}\setlength{\Height}{-\Height}%
\put(-3.1400000,-0.1000000){\hspace*{\Width}\raisebox{\Height}{$-\pi$}}%
%
}%
{%
\color[rgb]{0,0,0}%
\special{pa -2474   -20}\special{pa -2474    20}%
\special{fp}%
}%
{%
\color[rgb]{0,0,0}%
\settowidth{\Width}{$-2\pi$}\setlength{\Width}{-0.5\Width}%
\settoheight{\Height}{$-2\pi$}\settodepth{\Depth}{$-2\pi$}\setlength{\Height}{-\Height}%
\put(-6.2800000,-0.1000000){\hspace*{\Width}\raisebox{\Height}{$-2\pi$}}%
%
}%
{%
\color[rgb]{0,0,0}%
\special{pa    20   394}\special{pa   -20   394}%
\special{fp}%
}%
{%
\color[rgb]{0,0,0}%
\settowidth{\Width}{$-1$}\setlength{\Width}{-1\Width}%
\settoheight{\Height}{$-1$}\settodepth{\Depth}{$-1$}\setlength{\Height}{-0.5\Height}\setlength{\Depth}{0.5\Depth}\addtolength{\Height}{\Depth}%
\put(-0.1000000,-1.0000000){\hspace*{\Width}\raisebox{\Height}{$-1$}}%
%
}%
{%
\color[rgb]{0,0,0}%
\special{pa    20  -394}\special{pa   -20  -394}%
\special{fp}%
}%
{%
\color[rgb]{0,0,0}%
\settowidth{\Width}{$1$}\setlength{\Width}{-1\Width}%
\settoheight{\Height}{$1$}\settodepth{\Depth}{$1$}\setlength{\Height}{-0.5\Height}\setlength{\Depth}{0.5\Depth}\addtolength{\Height}{\Depth}%
\put(-0.1000000,1.0000000){\hspace*{\Width}\raisebox{\Height}{$1$}}%
%
}%
\special{pa -2559    -0}\special{pa  2559    -0}%
\special{fp}%
\special{pa     0   472}\special{pa     0  -472}%
\special{fp}%
\settowidth{\Width}{$x$}\setlength{\Width}{0\Width}%
\settoheight{\Height}{$x$}\settodepth{\Depth}{$x$}\setlength{\Height}{-0.5\Height}\setlength{\Depth}{0.5\Depth}\addtolength{\Height}{\Depth}%
\put(6.5500000,0.0000000){\hspace*{\Width}\raisebox{\Height}{$x$}}%
%
\settowidth{\Width}{$y$}\setlength{\Width}{-0.5\Width}%
\settoheight{\Height}{$y$}\settodepth{\Depth}{$y$}\setlength{\Height}{\Depth}%
\put(0.0000000,1.2500000){\hspace*{\Width}\raisebox{\Height}{$y$}}%
%
\settowidth{\Width}{O}\setlength{\Width}{0\Width}%
\settoheight{\Height}{O}\settodepth{\Depth}{O}\setlength{\Height}{-\Height}%
\put(0.0500000,-0.0500000){\hspace*{\Width}\raisebox{\Height}{O}}%
%
\end{picture}}%}}
\putnotese{10}{60}{[1] $\sin(-x)$を$\sin x$または$\cos x$で表せ}
\putnotese{10}{67}{[2] $\cos(-x)$を$\sin x$または$\cos x$で表せ}
\putnotese{70}{-2}{{\color{red}原点対称(奇関数)}}
\putnotese{70}{25}{{\color{red}$y$軸対称(偶関数)}}
\end{layer}

\vspace{46mm}

\begin{itemize}
\item
[課題]\monbannoadd 曲線上の点を動かしてみて答えよ
\end{itemize}

\newslide{加法定理による等式証明($-x$)}

\vspace*{18mm}

\slidepage
\begin{itemize}
\item
{\normalsize\color{blue} $\sin 0=0,\ \cos 0=1,\ \sin\pi=0,\ \cos\pi=-1$}
\item
$\sin(-x)$\\
$=\sin(0-x)=$
$\sin 0 \cos x-\cos 0\sin x=$
$-\sin x$
\item
$\cos(-x)$\\
$=\cos(0-x)=$
$\cos 0 \cos x+\sin 0\sin x=$
$\cos x$
\end{itemize}
\label{pageend}\mbox{}

\end{document}
