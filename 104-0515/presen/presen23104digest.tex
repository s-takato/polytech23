%%% Title presen23104
\documentclass[landscape,10pt]{ujarticle}
\special{papersize=\the\paperwidth,\the\paperheight}
\usepackage{ketpic,ketlayer}
\usepackage{ketslide}
\usepackage{amsmath,amssymb}
\usepackage{bm,enumerate}
\usepackage[dvipdfmx]{graphicx}
\usepackage{color}
\definecolor{slidecolora}{cmyk}{0.98,0.13,0,0.43}
\definecolor{slidecolorb}{cmyk}{0.2,0,0,0}
\definecolor{slidecolorc}{cmyk}{0.2,0,0,0}
\definecolor{slidecolord}{cmyk}{0.2,0,0,0}
\definecolor{slidecolore}{cmyk}{0,0,0,0.5}
\definecolor{slidecolorf}{cmyk}{0,0,0,0.5}
\definecolor{slidecolori}{cmyk}{0.98,0.13,0,0.43}
\def\setthin#1{\def\thin{#1}}
\setthin{0}
\newcommand{\slidepage}[1][s]{%
\setcounter{ketpicctra}{18}%
\if#1m \setcounter{ketpicctra}{1}\fi
\hypersetup{linkcolor=black}%

\begin{layer}{118}{0}
\putnotee{122}{-\theketpicctra.05}{\small\thepage/\pageref{pageend}}
\end{layer}\hypersetup{linkcolor=blue}

}
\usepackage{emath}
\usepackage{emathEy}
\usepackage{emathMw}
\usepackage{pict2e}
\usepackage{ketlayermorewith2e}
\usepackage[dvipdfmx,colorlinks=true,linkcolor=blue,filecolor=blue]{hyperref}
\newcommand{\hiduke}{0515}
\newcommand{\hako}[2][1]{\fbox{\raisebox{#1mm}{\mbox{}}\raisebox{-#1mm}{\mbox{}}\,\phantom{#2}\,}}
\newcommand{\hakoa}[2][1]{\fbox{\raisebox{#1mm}{\mbox{}}\raisebox{-#1mm}{\mbox{}}\,#2\,}}
\newcommand{\hakom}[2][1]{\hako[#1]{$#2$}}
\newcommand{\hakoma}[2][1]{\hakoa[#1]{$#2$}}
\def\rad{\;\mathrm{rad}}
\def\deg#1{#1^{\circ}}
\newcommand{\sbunsuu}[2]{\scalebox{0.6}{$\bunsuu{#1}{#2}$}}
\def\pow{$\hspace{-1.5mm}^\hspace{-1mm}$}
\def\dlim{\displaystyle\lim}
\newcommand{\brd}[2][1]{\scalebox{#1}{\color{red}\fbox{\color{black}$#2$}}}
\newcommand\down[1][0.5zw]{\vspace{#1}\\}
\newcommand{\sfrac}[3][0.65]{\scalebox{#1}{$\frac{#2}{#3}$}}
\newcommand{\phn}[1]{\phantom{#1}}
\newcommand{\scb}[2][0.6]{\scalebox{#1}{#2}}
\newcommand{\dsum}{\displaystyle\sum}
\def\pow{$\hspace{-1.5mm}^\hspace{-1mm}$}
\def\dlim{\displaystyle\lim}
\def\dint{\displaystyle\int}

\setmargin{25}{145}{15}{100}

\ketslideinit

\pagestyle{empty}

\begin{document}

\begin{layer}{120}{0}
\putnotese{0}{0}{{\Large\bf
\color[cmyk]{1,1,0,0}

\begin{layer}{120}{0}
{\Huge \putnotes{60}{20}{2次関数と2次方程式}}
\putnotes{60}{70}{2023.04.24}
\end{layer}

}
}
\end{layer}

\def\mainslidetitley{22}
\def\ketcletter{slidecolora}
\def\ketcbox{slidecolorb}
\def\ketdbox{slidecolorc}
\def\ketcframe{slidecolord}
\def\ketcshadow{slidecolore}
\def\ketdshadow{slidecolorf}
\def\slidetitlex{6}
\def\slidetitlesize{1.3}
\def\mketcletter{slidecolori}
\def\mketcbox{yellow}
\def\mketdbox{yellow}
\def\mketcframe{yellow}
\def\mslidetitlex{62}
\def\mslidetitlesize{2}

\color{black}
\Large\bf\boldmath
\addtocounter{page}{-1}

\def\MARU{}
\renewcommand{\MARU}[1]{{\ooalign{\hfil$#1$\/\hfil\crcr\raise.167ex\hbox{\mathhexbox20D}}}}
\renewcommand{\slidepage}[1][s]{%
\setcounter{ketpicctra}{18}%
\if#1m \setcounter{ketpicctra}{1}\fi
\hypersetup{linkcolor=black}%
\begin{layer}{118}{0}
\putnotee{115}{-\theketpicctra.05}{\small\hiduke-\thepage/\pageref{pageend}}
\end{layer}\hypersetup{linkcolor=blue}
}
\newcounter{ban}
\setcounter{ban}{1}
\newcommand{\monban}[1][\hiduke]{%
#1-\theban\ %
\addtocounter{ban}{1}%
}
\newcommand{\monbannoadd}[1][\hiduke]{%
#1-\theban\ %
}
\newcommand{\addban}{%
\addtocounter{ban}{1}%
}
\newcounter{edawidth}
\newcounter{edactr}
\newcommand{\seteda}[1]{% 20220708 modified
\setcounter{edawidth}{#1}
\setcounter{edactr}{1}
}
\newcommand{\eda}[2][\theedawidth]{%
\Ltab{#1 mm}{[\theedactr]\ #2}%
\addtocounter{edactr}{1}%
}
%%%%%%%%%%%%

%%%%%%%%%%%%%%%%%%%%

\mainslide{三角比から三角関数へ}


\slidepage[m]
%%%%%%%%%%%%

%%%%%%%%%%%%%%%%%%%%

\newslide{三角比(復習)}

\vspace*{18mm}

\slidepage

\begin{layer}{120}{0}
\putnotese{60}{-5}{\scalebox{0.7}{\input{fig/fig22104_1.tex}}}
\end{layer}

\begin{itemize}
\item
[]$\cos A=\bunsuu{\mbox{AB}}{\mbox{AC}}=\bunsuu{\mbox{底辺}}{\mbox{斜辺}}$
\item
[]$\sin A=\bunsuu{\mbox{CB}}{\mbox{AC}}=\bunsuu{\mbox{高さ}}{\mbox{斜辺}}$
\item
[]$\tan A=\bunsuu{\mbox{BC}}{\mbox{AB}}=\bunsuu{\mbox{高さ}}{\mbox{底辺}}$
\item
辺の比だから,三角形の大きさによらない.
\end{itemize}
%%%%%%%%%%%%

%%%%%%%%%%%%%%%%%%%%


\newslide{角が$90\degree$より小さい場合(鋭角)}

\vspace*{18mm}

\slidepage

\begin{layer}{120}{0}
\putnotese{65}{8}{\scalebox{0.65}{\input{fig/fig2210431.tex}}}
\end{layer}

\begin{itemize}
\item
角を$\theta$とおく
\item
左の角が$\theta$の直角三角形がかける
\item
斜辺$r$,底辺$x$,高さ$y$とすると\\
 $\cos \theta=\bunsuu{x}{r}$\\
 $\sin \theta=\bunsuu{y}{r}$\\
 $\tan \theta=\bunsuu{y}{x}$\\
\end{itemize}

\newslide{角が$90\degree$より大きい場合}

\vspace*{18mm}

\slidepage

\begin{layer}{120}{0}
\putnotese{65}{8}{\scalebox{0.65}{\input{fig/fig2210434.tex}}}
\end{layer}

\begin{itemize}
\item
角$\theta$の直角三角形がかけない
\item
半径$r$の円上に$x$軸との角が$\theta$\\である点Pはとれる
\item
Pの$x$座標は底辺\\
  $y$座標は高さに対応\\
 $\cos \theta=\bunsuu{x}{r}$\\
 $\sin \theta=\bunsuu{y}{r}$\\
 $\tan \theta=\bunsuu{y}{x}$
\end{itemize}

\newslide{一般角の三角関数の値}

\vspace*{18mm}

\slidepage

\begin{layer}{120}{0}
\putnotese{80}{15}{\input{fig/ippansankaku.tex}}
\end{layer}

\begin{itemize}
\item
半径$r$の円上に一般角$\theta$の点Pをとる
\item
Pの座標を$(x,\ y)$とすると\\
 $\cos \theta=\bunsuu{x}{r}$\\
 $\sin \theta=\bunsuu{y}{r}$\\
 $\tan \theta=\bunsuu{y}{x}$
\item
[課題]\monban 図の$\theta$について求めよ\seteda{25}\\
\eda{$\cos\theta$}\eda{$\sin\theta$}\eda{$\tan\theta$}
\end{itemize}

\newslide{三角関数の値の符号}

\vspace*{18mm}

\slidepage

\begin{layer}{120}{0}
\putnotese{75}{8}{\scalebox{0.5}{\input{fig/fig2210434.tex}}}
\end{layer}

\begin{itemize}
\item
[]\Ltab{20mm}{}\Ctab{15mm}{$\cos\theta$}\Ctab{15mm}{$\sin\theta$}\Ctab{15mm}{$\tan\theta$}
\item
第1象限\Ctab{15mm}{$+$}\Ctab{15mm}{$+$}\Ctab{15mm}{$+$}
\item
第2象限\Ctab{15mm}{$-$}\Ctab{15mm}{$+$}\Ctab{15mm}{$-$}
\item
第3象限\Ctab{15mm}{$-$}\Ctab{15mm}{$-$}\Ctab{15mm}{$+$}
\item
第4象限
\item
[課題]\monban 第4象限での符号を答えよ\seteda{40}\\
\eda{$\cos\theta$の符号}\eda{$\sin\theta$の符号}\eda{$\tan\theta$の符号}
\end{itemize}

\newslide{三角関数の相互関係}

\vspace*{18mm}

\slidepage

\begin{layer}{120}{0}
\putnotee{60}{46}{\color{red}$\bigl(\cos(\theta)\bigr)^2$を$\cos^2 \theta$と書く}
\putnotese{90}{6}{\input{fig/presen10310305.tex}}
\end{layer}

\begin{itemize}
\item
[(1)\ ]$\tan\theta=\bunsuu{\sin\theta}{\cos\theta}$
\item
[]以下の証明では$\mathrm{OP}=r$とおく
\item
[\color{blue}証)]{\color{blue}$\tan\theta=\bunsuu{y}{x}=\bunsuu{\frac{y}{r}}{\frac{x}{r}}$}
{\color{blue}$=\bunsuu{\sin\theta}{\cos\theta}$}
\item
[(2)\ ]$\cos^2\theta+\sin^2\theta=1$
\item
[\color{blue}証)]{\color{blue}$\cos^2\theta+\sin^2\theta=\bunsuu{x^2}{r^2}+\bunsuu{y^2}{r^2}$}
{\color{blue}$=\bunsuu{x^2+y^2}{r^2}=1$}
\end{itemize}

\newslide{弧度法}

\vspace*{18mm}


\begin{layer}{120}{0}
\putnotew{96}{73}{\hyperlink{para0pg7}{\fbox{\Ctab{2.5mm}{\scalebox{1}{\scriptsize $\mathstrut||\!\lhd$}}}}}
\putnotew{101}{73}{\hyperlink{para1pg1}{\fbox{\Ctab{2.5mm}{\scalebox{1}{\scriptsize $\mathstrut|\!\lhd$}}}}}
\putnotew{108}{73}{\hyperlink{para1pg8}{\fbox{\Ctab{4.5mm}{\scalebox{1}{\scriptsize $\mathstrut\!\!\lhd\!\!$}}}}}
\putnotew{115}{73}{\hyperlink{para1pg9}{\fbox{\Ctab{4.5mm}{\scalebox{1}{\scriptsize $\mathstrut\!\rhd\!$}}}}}
\putnotew{120}{73}{\hyperlink{para1pg9}{\fbox{\Ctab{2.5mm}{\scalebox{1}{\scriptsize $\mathstrut \!\rhd\!\!|$}}}}}
\putnotew{125}{73}{\hyperlink{para2pg1}{\fbox{\Ctab{2.5mm}{\scalebox{1}{\scriptsize $\mathstrut \!\rhd\!\!||$}}}}}
\putnotee{126}{73}{\scriptsize\color{blue} 9/9}
\end{layer}

\slidepage

\begin{layer}{120}{0}
\putnotese{89}{10}{\input{fig/radian.tex}}
\end{layer}

\begin{itemize}
\item
弧の長さ$\ell$と半径$r$の比\ $\theta(ラジアン)=\bunsuu{\ell}{r}$\vspace{-2mm}
\item
半径$r$の円周は$2\pi r$だから\vspace{2mm}\\
 $\mbox{1周の角}(360^{\circ})=\bunsuu{2\pi r}{r}=2\pi$
\item
$\mbox{半周の角}(180^{\circ})=\pi$
\item
比なので単位はない($\sin$などと同じ)\\
 度と区別するときは,ラジアン(rad)を付ける
\end{itemize}

\newslide{度とラジアンの換算}

\vspace*{18mm}

\slidepage
\begin{itemize}
\item
[]1つの角について,$x\text{度}=y$(ラジアン)とする
\item
[]$1$度は$\bunsuu{\pi}{180}$
  $x$度は$\bunsuu{\pi}{180}\times x$
\item
[]$1$は$\bunsuu{180}{\pi}$度
  $y$は$\bunsuu{180}{\pi}\times y$度
\item
[課題]\monban 次の角を変換せよ(整数か$\pi$を含む分数で)\seteda{30}\\
\eda{$3.1416$}\eda{$10\degree$}\eda{$1$}\eda{$60\degree$}
\end{itemize}

\newslide{正弦関数と正弦曲線}

\vspace*{18mm}

\slidepage

\begin{layer}{120}{0}
\putnotes{65}{70}{\input{fig/fig221045.tex}}
\end{layer}

\begin{itemize}
\item
一般角を$x$とおく.
\item
任意の$x$に対して,$y=\sin x$の値が定まる.
\item
これを正弦関数という(三角関数の1つ).
\item
$y=\sin x$のグラフを正弦曲線という.
\item
{\color{red}$x$はラジアンとする.}\vspace{-2mm}
\item
[] 横軸を度とすると下図になってしまう
\end{itemize}

\newslide{$y=\sin x$のグラフ}

\vspace*{18mm}

\slidepage

\begin{layer}{120}{0}
\putnotese{80}{3}{\scalebox{0.8}{\input{fig/fig221046b.tex}}}
\end{layer}

{\color{red}

\begin{layer}{120}{0}
\qarrowline[8]{43}{26}{33}{150}{40}
\circleline{46}{25}{1}
\qarrowline[8]{35}{43}{38}{122}{40}
\circleline{37}{44}{1}
\end{layer}

}
\begin{itemize}
\item
{\color{red}半径$1$}の円に点$\mathrm{P}(X,Y)$をとる
\item
[]\hspace*{3zw}$\sin x=\bunsuu{Y}{r}=Y$
\item
また弧の長さを$\ell$とすると\\
\hspace*{3zw}$x=\bunsuu{\ell}{r}=\ell$
\item
[課題]\monban $x,\ \sin x$は\\(1)-(4)のどの長さで表されるか.\seteda{40}\\
\eda{$x$は}\eda{$\sin x$は}
\end{itemize}

\newslide{正弦曲線を描く}

\vspace*{18mm}

\slidepage
\begin{itemize}
\item
アプリ「$y=\sin x$のグラフ」を動かしてみよう\vspace{-2mm}
\item
使い方\vspace{-2mm}
\begin{enumerate}[(1)]
\item
学生番号を入れる\vspace{-2mm}
\item
赤い点を動かして$x$を決め,「点を打つ」\\
 長さが$x$の弧を表示して$(x,\sin x)$に点を打つ.\vspace{-2mm}
\item
いくつかの点を打って「点を結ぶ」\\
 正弦曲線との違いが表示される\\
 さらに「点を打つ」,「点を結ぶ」を繰り返す.\vspace{-2mm}
\end{enumerate}
\item
[課題]\monban 「REC」を押して表示されるデータを提出せよ.
\end{itemize}
%%%%%%%%%%%%

%%%%%%%%%%%%%%%%%%%%


\newslide{正弦曲線の特徴}

\vspace*{18mm}

\slidepage

\begin{layer}{120}{0}
\putnotes{60}{5}{\input{fig/graphsin.tex}}
\end{layer}

\vspace{30mm}
\begin{itemize}
\item
{\color{red}振幅}は$1$(値の範囲は$-1$から$1$)
\item
{\color{red}周期}は$2\pi$($2\pi$で元に戻る)
\item
原点対称
\end{itemize}

\newslide{正弦曲線(課題)}

\vspace*{18mm}

\slidepage
\down
「関数のグラフ」でグラフをかいてみよう.
\begin{itemize}
\item
[課題]\monban 次の関数の振幅と周期を答えよ\seteda{50}\vspace{2mm}\\
\eda{$y=2\sin x$}\eda{$y=\bunsuu{1}{3}\sin x$}\\
\eda{$y=\sin 2x$}\eda{$y=4\sin\bunsuu{x}{2}$}
\item
[課題]\monban 次の関数の振幅と周期を答えよ\seteda{50}\vspace{2mm}\\
\eda{$y=A\sin x$}\eda{$y=\sin bx$}
\end{itemize}
%%%%%%%%%%%%

%%%%%%%%%%%%%%%%%%%%


\newslide{振幅・周期}

\vspace*{18mm}

\slidepage
\begin{itemize}
\item
$y=\sin x$の振幅は$1$,周期は$2\pi$
\item
$y=A \sin x$の振幅は$A$,周期は$2\pi$
\item
$y=\sin(bx)$の振幅は$1$,周期は$\bunsuu{2\pi}{b}$
\end{itemize}

\newslide{位相}

\vspace*{18mm}

\slidepage
\begin{itemize}
\item
「関数のグラフ」でグラフをかいてみよう.
\item
[課題]\monbannoadd $y=\sin x$のグラフとの関係を答えよ.\seteda{60}\\
\eda{$y=\sin(x-1)$}\eda{$y=\sin(x-2)$}\\
\eda{$y=\sin(x+1)$}\eda{$y=\sin(x+\bunsuu{\pi}{2})$}
\item
$y=\sin(x-c)$は$y=\sin x$を\\
\hspace*{3zw}右方向に$c$だけ平行移動 {\color{red}位相が$c$だけ遅れる}
\item
$y=\sin(x+c)$は$y=\sin x$を\\
\hspace*{3zw}左方向に$c$だけ平行移動 {\color{red}位相が$-c$だけ進む}
\end{itemize}

\newslide{$y=\cos x$のグラフ(余弦曲線)}

\vspace*{18mm}

\slidepage

\begin{layer}{120}{0}
\putnotes{62}{6}{\input{fig/graphcos.tex}}
\putnoten{118}{6}{\small $y=\cos x$}
\putnotes{62}{6}{\input{fig/graphsinadd.tex}}
\putnoten{80}{6}{\color{red}\small $y=\sin x$}
\end{layer}

\vspace*{28mm}
\begin{itemize}
\item
{\color{red}振幅}は$1$(値の範囲は$-1$から$1$)\vspace{-1mm}
\item
{\color{red}周期}は$2\pi$($2\pi$で元に戻る)\vspace{-1mm}
\item
$\cos x$は$y$軸対称\vspace{-1mm}
\item
$\cos x$は$\sin x$を左に$\frac{\pi}{2}$平行移動({\color{red}位相}が$\frac{\pi}{2}$進む)
\end{itemize}

\newslide{角度の和の三角関数}

\vspace*{18mm}

\slidepage
\begin{itemize}
\item
2つの角を$A,\ B$とする(通常はギリシャ文字 $\alpha,\  \beta$)
\item
$\sin(A+B)=\sin A+\sin B$が成り立つかを考えよう
\item
$\sin\deg{30}+\sin\deg{60}=\sin(\deg{30}+\deg{60})$になるかを調べる
\item
{\large $\sin \deg{90}=1,\ \sin \deg{30}=\bunsuu{1}{2},\ \sin \deg{60}=\bunsuu{\sqrt{3}}{2}$}
\item
[課題]\monban $\sqrt{3}=1.732$を用いて答えよ.\seteda{90}\\
\eda{$\sin\deg{30}+\sin\deg{60}$を計算せよ}\\
\eda{$\sin(A+B)=\sin A+\sin B$は成り立つと言えるか}
\end{itemize}
%%%%%%%%%%%%

%%%%%%%%%%%%%%%%%%%%


\newslide{加法定理}

\vspace*{18mm}

\slidepage
\begin{itemize}
\item
[]$\sin(A+B)=\sin A \cos B+\cos A\sin B$
\item
[]$\sin( A- B)=\sin A\cos B-\cos A\sin B$
\item
[]$\cos( A+ B)=\cos A\cos B-\sin A\sin B$
\item
[]$\cos( A- B)=\cos A\cos B+\sin A\sin B$
\end{itemize}

\newslide{具体例(テキストP181)}

\vspace*{18mm}


\begin{layer}{120}{0}
\putnotew{96}{73}{\hyperlink{para1pg2}{\fbox{\Ctab{2.5mm}{\scalebox{1}{\scriptsize $\mathstrut||\!\lhd$}}}}}
\putnotew{101}{73}{\hyperlink{para2pg1}{\fbox{\Ctab{2.5mm}{\scalebox{1}{\scriptsize $\mathstrut|\!\lhd$}}}}}
\putnotew{108}{73}{\hyperlink{para2pg10}{\fbox{\Ctab{4.5mm}{\scalebox{1}{\scriptsize $\mathstrut\!\!\lhd\!\!$}}}}}
\putnotew{115}{73}{\hyperlink{para2pg11}{\fbox{\Ctab{4.5mm}{\scalebox{1}{\scriptsize $\mathstrut\!\rhd\!$}}}}}
\putnotew{120}{73}{\hyperlink{para2pg11}{\fbox{\Ctab{2.5mm}{\scalebox{1}{\scriptsize $\mathstrut \!\rhd\!\!|$}}}}}
\putnotew{125}{73}{\hyperlink{para3pg1}{\fbox{\Ctab{2.5mm}{\scalebox{1}{\scriptsize $\mathstrut \!\rhd\!\!||$}}}}}
\putnotee{126}{73}{\scriptsize\color{blue} 11/11}
\end{layer}

\slidepage
\begin{itemize}
\item
{\color{blue}\normalsize $\sin 30\degree=\hakoa{$\bunsuu{1}{2}$},\ \sin 45\degree=\hakoa{$\bunsuu{1}{\sqrt{2}}$},\ \sin 60\degree=\hakoa{$\bunsuu{\sqrt{3}}{2}$}$}\\
{\color{blue}\normalsize $\cos 30\degree=\hakoa{$\bunsuu{\sqrt{3}}{2}$},\ \cos 45\degree=\hakoa{$\bunsuu{1}{\sqrt{2}}$},\ \cos 60\degree=\hakoa{$\bunsuu{1}{2}$}$}
\item
$\sin 75\degree$\\
$=\sin(45\degree+30\degree)$
$=\sin 45\degree \cos 30\degree+\cos 45\degree \sin 30\degree$
$=\bunsuu{1}{\sqrt{2}}\bunsuu{\sqrt{3}}{2}+\bunsuu{1}{\sqrt{2}}\bunsuu{1}{2}=$
$\bunsuu{\sqrt{3}+1}{2\sqrt{2}}=$
$\bunsuu{\sqrt{6}+\sqrt{2}}{4}$
\item
[課題]\monban 次を求めよ\seteda{50}\\
\eda{$\sin 15\degree$}\eda{$\cos 75\degree$}
\end{itemize}

\newslide{グラフの対称性}

\vspace*{18mm}

\slidepage

\begin{layer}{120}{0}
\putnotes{62}{1}{\scalebox{0.9}{\input{fig/graphsin.tex}}}
\putnotes{62}{28}{\scalebox{0.9}{\input{fig/graphcos.tex}}}
\putnotese{10}{60}{[1] $\sin(-x)$を$\sin x$または$\cos x$で表せ}
\putnotese{10}{67}{[2] $\cos(-x)$を$\sin x$または$\cos x$で表せ}
\putnotese{70}{-2}{{\color{red}原点対称(奇関数)}}
\putnotese{70}{25}{{\color{red}$y$軸対称(偶関数)}}
\end{layer}

\vspace{46mm}

\begin{itemize}
\item
[課題]\monbannoadd 曲線上の点を動かしてみて答えよ
\end{itemize}

\newslide{加法定理による等式証明($-x$)}

\vspace*{18mm}

\slidepage
\begin{itemize}
\item
{\normalsize\color{blue} $\sin 0=0,\ \cos 0=1,\ \sin\pi=0,\ \cos\pi=-1$}
\item
$\sin(-x)$\\
$=\sin(0-x)=$
$\sin 0 \cos x-\cos 0\sin x=$
$-\sin x$
\item
$\cos(-x)$\\
$=\cos(0-x)=$
$\cos 0 \cos x+\sin 0\sin x=$
$\cos x$
\end{itemize}
\label{pageend}\mbox{}

\end{document}
